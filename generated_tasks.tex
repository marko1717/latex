\documentclass[14pt]{extarticle}
\usepackage{fontspec}
\usepackage{polyglossia}
\setdefaultlanguage{ukrainian}

\defaultfontfeatures{Ligatures=TeX}
\setmainfont{Liberation Serif}
\setsansfont{Liberation Sans}
\setmonofont{Liberation Mono}

\usepackage[a4paper,margin=1.5cm,bottom=2cm,top=2cm]{geometry}
\usepackage{amsmath,amssymb}
\usepackage{enumitem}
\usepackage{tikz}
\usepackage{pgfplots}
\pgfplotsset{compat=1.16}

\usetikzlibrary{calc,patterns,angles,quotes,intersections,babel}
\usetikzlibrary{3d}
\definecolor{woodinner}{RGB}{222, 184, 135}
\definecolor{woodouter}{RGB}{139, 69, 19}
\usepackage{xcolor}
\usepackage{array}
\usepackage{fancyhdr}
\usepackage{multirow}

\definecolor{headerblue}{RGB}{0, 102, 204}
\definecolor{yearcolor}{RGB}{128, 0, 128}

\pagestyle{fancy}
\fancyhf{}
\renewcommand{\headrulewidth}{0pt}
\fancyfoot[C]{\thepage}

\setlength{\headheight}{15pt}
\setlength{\headsep}{10pt}
\setlength{\footskip}{25pt}

\widowpenalty=10000
\clubpenalty=10000

\newcommand{\answerTable}[5]{
\begin{center}
\begin{tabular}{|*{5}{>{\centering\arraybackslash}m{2.8cm}|}}
\hline
\rule[-0.3cm]{0pt}{0.8cm}\textbf{А} & \textbf{Б} & \textbf{В} & \textbf{Г} & \textbf{Д} \\
\hline
\rule[-0.4cm]{0pt}{1.0cm}#1 & \rule[-0.4cm]{0pt}{1.0cm}#2 & \rule[-0.4cm]{0pt}{1.0cm}#3 & \rule[-0.4cm]{0pt}{1.0cm}#4 & \rule[-0.4cm]{0pt}{1.0cm}#5 \\
\hline
\end{tabular}
\end{center}
}

\newcommand{\shortAnswer}{
\vspace{0.3cm}
\noindent\hspace{1cm}Відповідь: \framebox(18,18){}\framebox(18,18){}\framebox(18,18){}\framebox(18,18){}{,}\framebox(18,18){}\framebox(18,18){}\framebox(18,18){}
\vspace{0.5cm}
}

\newcommand{\nmtyear}[1]{\hfill{\small\color{yearcolor}(AI Gen)}}

\begin{document}

\begin{center}
{\Large\textbf{\color{headerblue}ЗГЕНЕРОВАНІ ЗАВДАННЯ (AI)}}
\end{center}

\begin{center}
{\large Тема: \textbf{Арифметична прогресія}}
\end{center}

\vspace{0.5cm}
% === Arithmetic Progression: Find d ===
\noindent\makebox[1.5em][l]{\textbf{1.}}\parbox[t]{\dimexpr\textwidth-1.5em}{В арифметичній прогресії $(a_n)$: $a_1 = -1$, $a_3 = -15$. Визначте різницю $d$ прогресії. \nmtyear{2026}}

\answerTable{$d = -6$}{$d = -7$}{$d = 7$}{$d = -8$}{$d = -14$}

\vspace{0.5cm}

\noindent\makebox[1.5em][l]{\textbf{2.}}\parbox[t]{\dimexpr\textwidth-1.5em}{В арифметичній прогресії $(a_n)$: $a_1 = 13$, $a_3 = 21$. Визначте різницю $d$ прогресії. \nmtyear{2026}}

\answerTable{$d = 8$}{$d = 4$}{$d = -4$}{$d = 5$}{$d = 17$}

\vspace{0.5cm}

% === Arithmetic Progression: Member Difference ===
\noindent\makebox[1.5em][l]{\textbf{3.}}\parbox[t]{\dimexpr\textwidth-1.5em}{В арифметичній прогресії $(a_n)$ відомо, що $a_{8} - a_{3} = -15$. Знайдіть значення виразу $a_{10} - a_{8}$. \nmtyear{2026}}

\answerTable{6}{-6}{-3}{-15}{-9}

\vspace{0.5cm}

\noindent\makebox[1.5em][l]{\textbf{4.}}\parbox[t]{\dimexpr\textwidth-1.5em}{В арифметичній прогресії $(a_n)$ відомо, що $a_{4} - a_{2} = 4$. Знайдіть значення виразу $a_{4} - a_{6}$. \nmtyear{2026}}

\answerTable{-8}{-4}{-6}{-2}{4}

\vspace{0.5cm}

% === Arithmetic Progression: Sum ===
\noindent\makebox[1.5em][l]{\textbf{5.}}\parbox[t]{\dimexpr\textwidth-1.5em}{Обчисліть суму перших 8-ти членів арифметичної прогресії $(a_n)$, якщо $a_1 + a_{8} = -29$. \nmtyear{2026}}

\answerTable{-29}{-124}{-116}{116}{-232}

\vspace{0.5cm}

\noindent\makebox[1.5em][l]{\textbf{6.}}\parbox[t]{\dimexpr\textwidth-1.5em}{Обчисліть суму перших 16-ти членів арифметичної прогресії $(a_n)$, якщо $a_1 + a_{16} = 29$. \nmtyear{2026}}

\answerTable{216}{232}{464}{248}{-232}

\vspace{0.5cm}

% === Arithmetic Progression: Count Terms ===
\noindent\makebox[1.5em][l]{\textbf{7.}}\parbox[t]{\dimexpr\textwidth-1.5em}{В арифметичній прогресії $(a_n)$ перший член $a_1 = 24{,}5$, різниця $d = -3$. Скільки всього \textit{додатних} членів має ця прогресія? \nmtyear{2026}}

\answerTable{9}{10}{11}{8}{7}

\vspace{0.5cm}

\noindent\makebox[1.5em][l]{\textbf{8.}}\parbox[t]{\dimexpr\textwidth-1.5em}{В арифметичній прогресії $(a_n)$ перший член $a_1 = -11{,}2$, різниця $d = 1$. Скільки всього \textit{від'ємних} членів має ця прогресія? \nmtyear{2026}}

\answerTable{10}{13}{14}{11}{12}

\vspace{0.5cm}

% === Arithmetic Progression: Middle Term ===
\noindent\makebox[1.5em][l]{\textbf{9.}}\parbox[t]{\dimexpr\textwidth-1.5em}{Визначте 18-й член $a_{18}$ арифметичної прогресії $(a_n)$, у якої $a_{17} = -4$, $a_{19} = 2$. \nmtyear{2026}}

\answerTable{2}{-2}{3}{6}{-1}

\vspace{0.5cm}

\noindent\makebox[1.5em][l]{\textbf{10.}}\parbox[t]{\dimexpr\textwidth-1.5em}{Визначте 3-й член $a_{3}$ арифметичної прогресії $(a_n)$, у якої $a_{2} = 4$, $a_{4} = 8$. \nmtyear{2026}}

\answerTable{2}{6}{12}{4}{8}

\vspace{0.5cm}

% === Arithmetic Progression: Formula Search ===
\noindent\makebox[1.5em][l]{\textbf{11.}}\parbox[t]{\dimexpr\textwidth-1.5em}{Арифметичну прогресію $(a_n)$ задано формулою $n$-го члена $a_n = 15 + 1{,}5n$. Визначте номер члена, значення якого дорівнює $84$. \nmtyear{2026}}

\answerTable{36}{56}{45}{47}{46}

\vspace{0.5cm}

\noindent\makebox[1.5em][l]{\textbf{12.}}\parbox[t]{\dimexpr\textwidth-1.5em}{Арифметичну прогресію $(a_n)$ задано формулою $n$-го члена $a_n = 32  -2{,}5n$. Визначте номер члена, значення якого дорівнює $-30{,}5$. \nmtyear{2026}}

\answerTable{24}{12{,}2}{35}{25}{26}

\vspace{0.5cm}

% === Arithmetic Progression: Word Problem ===
\noindent\makebox[1.5em][l]{\textbf{13.}}\parbox[t]{\dimexpr\textwidth-1.5em}{У залі для глядачів цирку встановлено 14 рядів крісел: у першому ряду 48 крісла, а в кожному наступному ряду кількість крісел на те саме число більше, ніж у попередньому. Визначте кількість крісел у \textit{7-му} ряду, якщо в останньому ряду 61 крісла. \nmtyear{2026}}

\answerTable{56}{54}{52}{53}{55}

\vspace{0.5cm}

\noindent\makebox[1.5em][l]{\textbf{14.}}\parbox[t]{\dimexpr\textwidth-1.5em}{На рисунку зображено (уявно) поперечний переріз стосу дерев'яних колод. У нижньому ряду стосу 15 колод, а у верхньому — 1. Визначте загальну кількість колод у цьому стосі, якщо кожен наступний ряд містить на одну колоду менше, ніж попередній. \nmtyear{2026}}

\answerTable{130}{110}{120}{135}{105}

\vspace{0.5cm}


\end{document}
