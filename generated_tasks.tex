\documentclass[14pt]{extarticle}
\usepackage{fontspec}
\usepackage{polyglossia}
\setdefaultlanguage{ukrainian}

\defaultfontfeatures{Ligatures=TeX}
\setmainfont{Liberation Serif}
\setsansfont{Liberation Sans}
\setmonofont{Liberation Mono}

\usepackage[a4paper,margin=1.5cm,bottom=2cm,top=2cm]{geometry}
\usepackage{amsmath,amssymb}
\usepackage{enumitem}
\usepackage{tikz}
\usepackage{pgfplots}
\pgfplotsset{compat=1.16}

\usetikzlibrary{calc,patterns,angles,quotes,intersections,babel}
\usetikzlibrary{3d}
\definecolor{woodinner}{RGB}{222, 184, 135}
\definecolor{woodouter}{RGB}{139, 69, 19}
\usepackage{xcolor}
\usepackage{array}
\usepackage{fancyhdr}
\usepackage{multirow}

\definecolor{headerblue}{RGB}{0, 102, 204}
\definecolor{yearcolor}{RGB}{128, 0, 128}

\pagestyle{fancy}
\fancyhf{}
\renewcommand{\headrulewidth}{0pt}
\fancyfoot[C]{\thepage}

\setlength{\headheight}{15pt}
\setlength{\headsep}{10pt}
\setlength{\footskip}{25pt}

\widowpenalty=10000
\clubpenalty=10000

\newcommand{\answerTable}[5]{
\begin{center}
\begin{tabular}{|*{5}{>{\centering\arraybackslash}m{2.8cm}|}}
\hline
\rule[-0.3cm]{0pt}{0.8cm}\textbf{А} & \textbf{Б} & \textbf{В} & \textbf{Г} & \textbf{Д} \\
\hline
\rule[-0.4cm]{0pt}{1.0cm}#1 & \rule[-0.4cm]{0pt}{1.0cm}#2 & \rule[-0.4cm]{0pt}{1.0cm}#3 & \rule[-0.4cm]{0pt}{1.0cm}#4 & \rule[-0.4cm]{0pt}{1.0cm}#5 \\
\hline
\end{tabular}
\end{center}
}

\newcommand{\shortAnswer}{
\vspace{0.3cm}
\noindent\hspace{1cm}Відповідь: \framebox(18,18){}\framebox(18,18){}\framebox(18,18){}\framebox(18,18){}{,}\framebox(18,18){}\framebox(18,18){}\framebox(18,18){}
\vspace{0.5cm}
}

\newcommand{\nmtyear}[1]{\hfill{\small\color{yearcolor}(AI Gen)}}

\begin{document}

\begin{center}
{\Large\textbf{\color{headerblue}ЗГЕНЕРОВАНІ ЗАВДАННЯ (AI)}}
\end{center}

\begin{center}
{\large Тема: \textbf{Арифметична та Геометрична прогресії}}
\end{center}

\vspace{0.5cm}
% === Arithmetic Progression: Find d ===
\noindent\makebox[1.5em][l]{\textbf{1.}}\parbox[t]{\dimexpr\textwidth-1.5em}{В арифметичній прогресії $(a_n)$: $a_1 = 12$, $a_3 = 16$. Визначте різницю $d$ прогресії. \nmtyear{2026}}

\answerTable{$d = -2$}{$d = 4$}{$d = 2$}{$d = 3$}{$d = 14$}

\vspace{0.5cm}

\noindent\makebox[1.5em][l]{\textbf{2.}}\parbox[t]{\dimexpr\textwidth-1.5em}{В арифметичній прогресії $(a_n)$: $a_1 = -2$, $a_3 = 2$. Визначте різницю $d$ прогресії. \nmtyear{2026}}

\answerTable{$d = 4$}{$d = 3$}{$d = 0$}{$d = -2$}{$d = 2$}

\vspace{0.5cm}

% === Arithmetic Progression: Member Difference ===
\noindent\makebox[1.5em][l]{\textbf{3.}}\parbox[t]{\dimexpr\textwidth-1.5em}{В арифметичній прогресії $(a_n)$ відомо, що $a_{9} - a_{4} = -35$. Знайдіть значення виразу $a_{12} - a_{9}$. \nmtyear{2026}}

\answerTable{21}{-14}{-35}{-28}{-21}

\vspace{0.5cm}

\noindent\makebox[1.5em][l]{\textbf{4.}}\parbox[t]{\dimexpr\textwidth-1.5em}{В арифметичній прогресії $(a_n)$ відомо, що $a_{3} - a_{1} = -12$. Знайдіть значення виразу $a_{5} - a_{3}$. \nmtyear{2026}}

\answerTable{-6}{-12}{12}{-11}{-18}

\vspace{0.5cm}

% === Arithmetic Progression: Sum ===
\noindent\makebox[1.5em][l]{\textbf{5.}}\parbox[t]{\dimexpr\textwidth-1.5em}{Обчисліть суму перших 10-ти членів арифметичної прогресії $(a_n)$, якщо $a_1 + a_{10} = 41$. \nmtyear{2026}}

\answerTable{195}{205}{215}{-205}{41}

\vspace{0.5cm}

\noindent\makebox[1.5em][l]{\textbf{6.}}\parbox[t]{\dimexpr\textwidth-1.5em}{Обчисліть суму перших 12-ти членів арифметичної прогресії $(a_n)$, якщо $a_1 + a_{12} = 32$. \nmtyear{2026}}

\answerTable{-192}{32}{192}{204}{180}

\vspace{0.5cm}

% === Arithmetic Progression: Count Terms ===
\noindent\makebox[1.5em][l]{\textbf{7.}}\parbox[t]{\dimexpr\textwidth-1.5em}{В арифметичній прогресії $(a_n)$ перший член $a_1 = 7{,}1$, різниця $d = -0{,}5$. Скільки всього \textit{додатних} членів має ця прогресія? \nmtyear{2026}}

\answerTable{14}{16}{17}{15}{13}

\vspace{0.5cm}

\noindent\makebox[1.5em][l]{\textbf{8.}}\parbox[t]{\dimexpr\textwidth-1.5em}{В арифметичній прогресії $(a_n)$ перший член $a_1 = 4{,}5$, різниця $d = -0{,}5$. Скільки всього \textit{додатних} членів має ця прогресія? \nmtyear{2026}}

\answerTable{11}{8}{7}{10}{9}

\vspace{0.5cm}

% === Arithmetic Progression: Middle Term ===
\noindent\makebox[1.5em][l]{\textbf{9.}}\parbox[t]{\dimexpr\textwidth-1.5em}{Визначте 13-й член $a_{13}$ арифметичної прогресії $(a_n)$, у якої $a_{12} = 21$, $a_{14} = 24$. \nmtyear{2026}}

\answerTable{21}{22{,}5}{24}{1{,}5}{3}

\vspace{0.5cm}

\noindent\makebox[1.5em][l]{\textbf{10.}}\parbox[t]{\dimexpr\textwidth-1.5em}{Визначте 5-й член $a_{5}$ арифметичної прогресії $(a_n)$, у якої $a_{4} = 30$, $a_{6} = 38$. \nmtyear{2026}}

\answerTable{68}{34}{30}{4}{38}

\vspace{0.5cm}

% === Arithmetic Progression: Formula Search ===
\noindent\makebox[1.5em][l]{\textbf{11.}}\parbox[t]{\dimexpr\textwidth-1.5em}{Арифметичну прогресію $(a_n)$ задано формулою $n$-го члена $a_n = 25  -0{,}5n$. Визначте номер члена, значення якого дорівнює $4$. \nmtyear{2026}}

\answerTable{52}{42}{43}{32}{41}

\vspace{0.5cm}

\noindent\makebox[1.5em][l]{\textbf{12.}}\parbox[t]{\dimexpr\textwidth-1.5em}{Арифметичну прогресію $(a_n)$ задано формулою $n$-го члена $a_n = 31  -1{,}5n$. Визначте номер члена, значення якого дорівнює $8{,}5$. \nmtyear{2026}}

\answerTable{16}{15}{25}{5}{14}

\vspace{0.5cm}

% === Arithmetic Progression: Word Problem ===
\noindent\makebox[1.5em][l]{\textbf{13.}}\parbox[t]{\dimexpr\textwidth-1.5em}{Студент вивчав мову за методикою: у перший день він запам'ятав 6 слів, а кожного наступного дня — на 4 слів більше, ніж попереднього. Скільки всього слів запам'ятав студент за 28 днів? \nmtyear{2026}}

\answerTable{1700}{168}{3192}{1660}{1680}

\vspace{0.5cm}

\noindent\makebox[1.5em][l]{\textbf{14.}}\parbox[t]{\dimexpr\textwidth-1.5em}{Студент вивчав мову за методикою: у перший день він запам'ятав 15 слів, а кожного наступного дня — на 5 слів більше, ніж попереднього. Скільки всього слів запам'ятав студент за 10 днів? \nmtyear{2026}}

\answerTable{355}{600}{150}{395}{375}

\vspace{0.5cm}

% === Geometric Progression: Find Term (b_1, b_2 -> b_n) ===
\noindent\makebox[1.5em][l]{\textbf{15.}}\parbox[t]{\dimexpr\textwidth-1.5em}{У геометричній прогресії $(b_n)$ відомо, що $b_1 = 2$, $b_2 = 8$. Визначте $b_{4}$. \nmtyear{2026}}

\answerTable{512}{32}{128}{20}{-128}

\vspace{0.5cm}

\noindent\makebox[1.5em][l]{\textbf{16.}}\parbox[t]{\dimexpr\textwidth-1.5em}{У геометричній прогресії $(b_n)$ відомо, що $b_1 = 27$, $b_2 = 13{,}5$. Визначте $b_{7}$. \nmtyear{2026}}

\answerTable{0{,}4219}{0{,}8438}{0{,}2109}{1728}{-54}

\vspace{0.5cm}

% === Geometric Progression: Term Ratio ===
\noindent\makebox[1.5em][l]{\textbf{17.}}\parbox[t]{\dimexpr\textwidth-1.5em}{У геометричній прогресії $(b_n)$ відомо, що $b_1 = 32$, $b_2 = 6{,}4$. Обчисліть $\dfrac{b_{6}}{b_{7}}$. \nmtyear{2026}}

\answerTable{-2}{4}{5}{0{,}2}{0{,}0082}

\vspace{0.5cm}

\noindent\makebox[1.5em][l]{\textbf{18.}}\parbox[t]{\dimexpr\textwidth-1.5em}{У геометричній прогресії $(b_n)$ відомо, що $b_1 = 4$, $b_2 = 0{,}8$. Обчисліть $\dfrac{b_{7}}{b_{8}}$. \nmtyear{2026}}

\answerTable{15}{0{,}0002}{12}{0{,}2}{5}

\vspace{0.5cm}

% === Geometric Progression: Formula ===
\noindent\makebox[1.5em][l]{\textbf{19.}}\parbox[t]{\dimexpr\textwidth-1.5em}{Послідовність задано формулою $n$-го члена $b_n = 0{,}8 \cdot 3^n + 3n$. Визначте 5-й член цієї послідовності. \nmtyear{2026}}

\answerTable{209{,}4}{-209{,}4}{418{,}8}{210{,}4}{208{,}4}

\vspace{0.5cm}

\noindent\makebox[1.5em][l]{\textbf{20.}}\parbox[t]{\dimexpr\textwidth-1.5em}{Геометричну прогресію задано формулою $n$-го члена $b_n = 0{,}5 \cdot 3^{n-2}$. Визначте 4-й член цієї прогресії. \nmtyear{2026}}

\answerTable{-4{,}5}{9}{5{,}5}{4{,}5}{3{,}5}

\vspace{0.5cm}

% === Geometric Progression: Sum ===
\noindent\makebox[1.5em][l]{\textbf{21.}}\parbox[t]{\dimexpr\textwidth-1.5em}{Знайдіть суму 3 перших членів геометричної прогресії $(b_n)$, у якої $b_2 = 4$, а знаменник $q = 3$. \nmtyear{2026}}

\answerTable{-17{,}333}{5{,}778}{46{,}333}{17{,}333}{12}

\vspace{0.5cm}

\noindent\makebox[1.5em][l]{\textbf{22.}}\parbox[t]{\dimexpr\textwidth-1.5em}{Знайдіть суму 5 перших членів геометричної прогресії $(b_n)$, у якої $b_2 = 8$, а знаменник $q = 3$. \nmtyear{2026}}

\answerTable{345{,}667}{107{,}556}{216}{322{,}667}{-322{,}667}

\vspace{0.5cm}

% === Geometric Progression: Word Problem ===
\noindent\makebox[1.5em][l]{\textbf{23.}}\parbox[t]{\dimexpr\textwidth-1.5em}{Бактерія ділиться. Першого дня колонія налічувала 5 бактерій. Кожного наступного дня кількість збільшувалася вдвічі. За яку \textit{найменшу} кількість днів сумарна кількість бактерій перевищить 1000? \nmtyear{2026}}

\answerTable{6}{7}{9}{8}{10}

\vspace{0.5cm}

\noindent\makebox[1.5em][l]{\textbf{24.}}\parbox[t]{\dimexpr\textwidth-1.5em}{Інвестор вклав гроші. Першого дня прибуток склав 100 доларів. Кожного наступного дня кількість збільшувалася вдвічі. За яку \textit{найменшу} кількість днів сумарна кількість доларів перевищить 5000? \nmtyear{2026}}

\answerTable{5}{6}{8}{4}{7}

\vspace{0.5cm}


\end{document}
