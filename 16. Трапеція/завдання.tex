\documentclass[14pt]{extarticle}
\usepackage{fontspec}
\usepackage{polyglossia}
\setdefaultlanguage{ukrainian}

\defaultfontfeatures{Ligatures=TeX}
\setmainfont{Liberation Serif}
\setsansfont{Liberation Sans}
\setmonofont{Liberation Mono}

\usepackage[a4paper,margin=1.5cm,bottom=2cm,top=2cm]{geometry}
\usepackage{amsmath,amssymb}
\usepackage{enumitem}
\usepackage{tikz}
\usepackage{pgfplots}
\pgfplotsset{compat=1.16}

% Підключаємо бібліотеки для зручних кутів
\usetikzlibrary{calc,patterns,angles,quotes,intersections,babel}

\usepackage{xcolor}
\usepackage{array}
\usepackage{fancyhdr}
\usepackage{multirow}

% Кольори
\definecolor{headerblue}{RGB}{0, 102, 204}
\definecolor{yearcolor}{RGB}{128, 0, 128}

\pagestyle{fancy}
\fancyhf{}
\renewcommand{\headrulewidth}{0pt}
\fancyfoot[C]{\thepage}

\setlength{\headheight}{15pt}
\setlength{\headsep}{10pt}
\setlength{\footskip}{25pt}

\widowpenalty=10000
\clubpenalty=10000

% === КОМАНДИ ===

% Таблиця для відповідей із дробами (збільшена висота клітинок)
% Оновлена таблиця: підпорка додана до КОЖНОЇ клітинки
\newcommand{\answerTableTall}[5]{
\begin{center}
\begin{tabular}{|*{5}{>{\centering\arraybackslash}m{2.8cm}|}}
\hline
\rule[-0.3cm]{0pt}{0.8cm}\textbf{А} & \textbf{Б} & \textbf{В} & \textbf{Г} & \textbf{Д} \\
\hline
% Тепер rule є перед кожним аргументом (#1..#5)
\rule[-0.9cm]{0pt}{2.0cm}#1 & 
\rule[-0.9cm]{0pt}{2.0cm}#2 & 
\rule[-0.9cm]{0pt}{2.0cm}#3 & 
\rule[-0.9cm]{0pt}{2.0cm}#4 & 
\rule[-0.9cm]{0pt}{2.0cm}#5 \\
\hline
\end{tabular}
\end{center}
}

% Оновлена таблиця відповідей (заголовки зовні)
\newcommand{\answerGrid}{
    \begingroup
    % Збільшуємо висоту рядків для квадратних клітинок
    \renewcommand{\arraystretch}{1.3} 
    % Відступ всередині клітинок
    \setlength{\tabcolsep}{7pt} 
    \begin{tabular}{r|c|c|c|c|c|}
         % Перший рядок: порожня клітинка зліва + букви без рамок (multicolumn прибирає |)
         \multicolumn{1}{c}{} & \multicolumn{1}{c}{\textbf{А}} & \multicolumn{1}{c}{\textbf{Б}} & \multicolumn{1}{c}{\textbf{В}} & \multicolumn{1}{c}{\textbf{Г}} & \multicolumn{1}{c}{\textbf{Д}} \\ \cline{2-6}
         % Наступні рядки: номер зліва (r) + клітинки з рамками (|c|)
         \textbf{1} & & & & & \\ \cline{2-6}
         \textbf{2} & & & & & \\ \cline{2-6}
         \textbf{3} & & & & & \\ \cline{2-6}
    \end{tabular}
    \endgroup
}

% Макет для завдань на відповідність
% #1 - Умови (1-3)
% #2 - Варіанти (А-Д)
% #3 - Табличка
\newcommand{\matchingLayout}[3]{
    \noindent
    \begin{minipage}[t]{0.40\textwidth}
       
        #1
    \end{minipage}%
    \hfill
    \begin{minipage}[t]{0.28\textwidth}
        
        #2
    \end{minipage}%
    \hfill
    \begin{minipage}[t]{0.30\textwidth}
        \vspace{0pt} % Хаки для вирівнювання minipage по верху
        \begin{flushright}
        #3
        \end{flushright}
    \end{minipage}
}

% Стандартна таблиця відповідей (для тестів)
\newcommand{\answerTableSmall}[5]{
\begin{tabular}{|*{5}{>{\centering\arraybackslash}m{1.65cm}|}}
\hline
\rule[-0.2cm]{0pt}{0.6cm}\textbf{А} & \textbf{Б} & \textbf{В} & \textbf{Г} & \textbf{Д} \\
\hline
% Підпорка додана до кожного варіанту для ідеального вирівнювання
\rule[-0.4cm]{0pt}{0.9cm}#1 & 
\rule[-0.4cm]{0pt}{0.9cm}#2 & 
\rule[-0.4cm]{0pt}{0.9cm}#3 & 
\rule[-0.4cm]{0pt}{0.9cm}#4 & 
\rule[-0.4cm]{0pt}{0.9cm}#5 \\
\hline
\end{tabular}
}

% Таблиця для вибору одного варіанту (Task 7)
\newcommand{\answerTable}[5]{
\begin{center}
\begin{tabular}{|*{5}{>{\centering\arraybackslash}m{2.8cm}|}}
\hline
\rule[-0.3cm]{0pt}{0.8cm}\textbf{А} & \textbf{Б} & \textbf{В} & \textbf{Г} & \textbf{Д} \\
\hline
\rule[-0.4cm]{0pt}{1.0cm}#1 & \rule[-0.4cm]{0pt}{1.0cm}#2 & \rule[-0.4cm]{0pt}{1.0cm}#3 & \rule[-0.4cm]{0pt}{1.0cm}#4 & \rule[-0.4cm]{0pt}{1.0cm}#5 \\
\hline
\end{tabular}
\end{center}
}

% Команда для року
\newcommand{\nmtyear}[1]{\hfill{\small\color{yearcolor}(НМТ #1)}}

\begin{document}

\begin{center}
{\Large\textbf{\color{headerblue}БАЗА ЗАВДАНЬ НМТ 2023}}
\end{center}

\begin{center}
{\large Тема: \textbf{Трапеція}}
\end{center}

\vspace{0.5cm}

% === ЗАВДАННЯ 1 ===
\noindent\textbf{1.} \begin{minipage}[t]{0.55\textwidth}
Прямокутник $ABCD$, паралелограм $BKMC$ та трапеція $DCMN$ лежать в одній площині, точки $K$, $C$ та $D$ належать одній прямій (див. рисунок). $AB = 5$ \textit{см}, $AD = 12$ \textit{см}, $BK = 13$ \textit{см}. До кожної величини (1--3) доберіть її значення (А--Д). \nmtyear{2023}
\end{minipage}
\hfill
\begin{minipage}[t]{0.4\textwidth}
    \vspace{-0.5cm}
    \begin{flushright}
    \begin{tikzpicture}[scale=0.55]
        \coordinate (A) at (0,0);
        \coordinate (B) at (0,2);
        \coordinate (C) at (4.8,2);
        \coordinate (D) at (4.8,0);
        \coordinate (K) at (4.8,4.6);
        \coordinate (M) at (9.6,4.6);
        \coordinate (N) at (9.6,0);
        
        \draw[thick] (A) -- (B) -- (C) -- (D) -- cycle;
        \draw[thick] (B) -- (K) -- (M) -- (C);
        \draw[thick] (D) -- (N) -- (M);
        
        % Прямий кут через pic (хоча для 90 можна і вручну, але pic універсальніше)
        % Тут простіше лініями, бо pic angle 90 не малює квадратик автоматично без додаткових стилів
        \draw (0,0.4) -- (0.4,0.4) -- (0.4,0);
        
        \node[left] at (0,1) {\small 5 \textit{см}};
        \node[below] at (2.4,0) {\small 12 \textit{см}};
        \node[above left] at (1.6,3.3) {\small 13 \textit{см}};
        
        \node[below left] at (A) {$A$};
        \node[above left] at (B) {$B$};
        \node[below right] at (C) {$C$};
        \node[below] at (D) {$D$};
        \node[above] at (K) {$K$};
        \node[above] at (M) {$M$};
        \node[below right] at (N) {$N$};
    \end{tikzpicture}
    \end{flushright}
\end{minipage}

\vspace{0.3cm}

\matchingLayout{
    \textbf{1} \quad діагональ $ABCD$ \\
    \textbf{2} \quad відстань від $K$ до $AN$ \\
    \textbf{3} \quad висота трапеції $DCMN$
}{
    \begin{tabular}{ll}
    \textbf{А} & 10 \textit{см} \\
    \textbf{Б} & 12 \textit{см} \\
    \textbf{В} & 13 \textit{см} \\
    \textbf{Г} & 17 \textit{см} \\
    \textbf{Д} & 5 \textit{см} \\
    \end{tabular}
}{
    \answerGrid
}

\vspace{0.7cm}


\vspace{0.5cm}

% === ЗАВДАННЯ 2 ===
\noindent\textbf{2.} \begin{minipage}[t]{0.55\textwidth}
У прямокутній системі координат на площині задано трапецію $ABCD$ (див. рисунок). Обчисліть площу цієї трапеції.
\end{minipage}
\hfill
\begin{minipage}[t]{0.4\textwidth}
    \vspace{-0.5cm}
    \begin{flushright}
    \begin{tikzpicture}[scale=0.4]
        % Сітка
        \draw[step=1cm,gray!50,very thin] (-4,-3) grid (7,5);
        
        % Осі
        \draw[->, >=stealth] (-4,0) -- (7,0) node[below]{$x$};
        \draw[->, >=stealth] (0,-3) -- (0,5) node[left]{$y$};
        
        % Точки за візуальними координатами:
        % B(-1, 0), C(-1, 3) - вертикальна сторона
        % A(6, -2), D(6, 4) - вертикальна сторона
        % Трапеція "лежача"
        \coordinate (B) at (-1,0);
        \coordinate (C) at (-1,3);
        \coordinate (D) at (6,4);
        \coordinate (A) at (6,-2);
        
        \draw[thick] (A) -- (B) -- (C) -- (D) -- cycle;
        
        % Підписи точок
        \node[below] at (B) {$B$};
        \node[above] at (C) {$C$};
        \node[above] at (D) {$D$};
        \node[below] at (A) {$A$};
        
        \node[above right] at (0,1) {$1$}; % Позначка 1 на y
        \draw (0.1,1) -- (-0.1,1);
        \node[above] at (1,0) {$1$};   % Позначка 1 на x
        \node[below right] at (0,0) {$0$};   % Позначка 1 на x
        \draw (1,0.1) -- (1,-0.1);
        
        \fill (A) circle (3pt);
        \fill (B) circle (3pt);
        \fill (C) circle (3pt);
        \fill (D) circle (3pt);
    \end{tikzpicture}
    \end{flushright}
\end{minipage}

\vspace{0.3cm}
\answerTable{$27$}{$63$}{$31{,}5$}{$29{,}5$}{$32{,}5$}

\vspace{0.7cm}

% === ЗАВДАННЯ 3 ===
\noindent\textbf{3.} \begin{minipage}[t]{0.55\textwidth}
На рисунку зображено прямокутну трапецію $ABCD$, у якої $AB=BC$, $AC=40$ \textit{см}, $CD=24$ \textit{см}. До кожного відрізка (1--3) доберіть його довжину (А--Д), якщо $O$ --- середина діагоналі $AC$ трапеції $ABCD$.
\end{minipage}
\hfill
\begin{minipage}[t]{0.4\textwidth}
    \vspace{-0.5cm}
    \begin{flushright}
    \begin{tikzpicture}[scale=0.12]
        % CD = 24. Right angle at D.
        % AC = 40. => AD = sqrt(40^2 - 24^2) = 32.
        % A=(0,0), D=(32,0), C=(32,24).
        % AB=BC. Let BC=x. Draw height from B to AD (point H).
        % AH = 32-x. BH=24. AB=x.
        % x^2 = (32-x)^2 + 24^2 => x^2 = 1024 - 64x + x^2 + 576
        % 64x = 1600 => x = 25.
        % BC = 25. B = (32-25, 24) = (7, 24).
        
        \coordinate (A) at (0,0);
        \coordinate (D) at (32,0);
        \coordinate (C) at (32,24);
        \coordinate (B) at (7,24);
        
        % O - midpoint of AC
        \coordinate (O) at (16,12);
        
        \draw[thick] (A) -- (B) -- (C) -- (D) -- cycle;
        \draw[thick] (A) -- (C);
        
        % Прямий кут
        \draw (D) rectangle ++(-2,2);
        
        % Позначки рівності AB і BC
        \draw[thick] ($(A)!0.5!(B)$) ++(180:1.5) -- ++(0:3);
        \draw[thick] ($(B)!0.5!(C)$) ++(90:1.5) -- ++(-90:3);
        
        % Позначки середини O (AO = OC)
        \draw[thick] ($(A)!0.5!(O)$) ++(135:1) -- ++(-45:2);
        \draw[thick] ($(A)!0.5!(O)$) ++(135:1) ++(0.5,0.5) -- ++(-45:2); % подвійна
        
        \draw[thick] ($(O)!0.5!(C)$) ++(135:1) -- ++(-45:2);
        \draw[thick] ($(O)!0.5!(C)$) ++(135:1) ++(0.5,0.5) -- ++(-45:2);
        
        \node[below left] at (A) {$A$};
        \node[above left] at (B) {$B$};
        \node[above right] at (C) {$C$};
        \node[below right] at (D) {$D$};
        \node[below right] at (O) {$O$};
        
        \fill (O) circle (15pt);
    \end{tikzpicture}
    \end{flushright}
\end{minipage}

\vspace{0.3cm}

\matchingLayout{
    \textit{Відрізок} \par \vspace{0.2cm}
    \textbf{1} \quad $AO$ \\
    \textbf{2} \quad $AD$ \\
    \textbf{3} \quad $AB$
}{
    \textit{Довжина відрізка} \par \vspace{0.2cm}
    \begin{tabular}{ll}
    \textbf{А} & 20 \textit{см} \\
    \textbf{Б} & 16 \textit{см} \\
    \textbf{В} & 25 \textit{см} \\
    \textbf{Г} & 27 \textit{см} \\
    \textbf{Д} & 32 \textit{см} \\
    \end{tabular}
}{
    \answerGrid
}

\vspace{0.7cm}

% === ЗАВДАННЯ 4 ===
\noindent\textbf{4.} \begin{minipage}[t]{0.55\textwidth}
На більшій основі $AD$ рівнобічної трапеції $ABCD$ вибрано точку $O$ так, що $BO \parallel CD$ (див. рисунок). $CO$ --- висота трапеції, $BC = 6$ \textit{см}, $CD = 10$ \textit{см}. До кожного відрізка (1--3) доберіть його довжину (А--Д).
\end{minipage}
\hfill
\begin{minipage}[t]{0.4\textwidth}
    \vspace{-0.5cm}
    \begin{flushright}
    \begin{tikzpicture}[scale=0.25]
        % BC = 6. CD = 10. BO || CD => BCDO parallelogram => OD=6, BO=10.
        % Isosceles ABCD => AB=CD=10.
        % CO perp AD. Triangle OCD: CD=10. Need CO.
        % BO=10, CD=10, OD=6. CO height? 
        % The drawing shows CO is perpendicular to AD. 
        % In parallelogram BCDO, if CO is height of trapezoid (perp to AD), then angle COD is 90? No.
        % Text says "CO - height". So CO perp AD.
        % O is on AD. BCDO is parallelogram? Wait.
        % BO || CD. Then BCDO is a parallelogram. Thus OD = BC = 6.
        % Also BO = CD = 10.
        % In triangle COD: hypotenuse CD=10? No, CO is height. So triangle COD is right-angled at O?
        % The drawing shows right angle at O (between CO and AD).
        % So CO is a leg, OD is a leg, CD is hypotenuse.
        % CD=10, OD=6 => CO = sqrt(100-36) = 8.
        
        \coordinate (A) at (-6, 0); % Just visual
        \coordinate (O) at (4, 0);
        \coordinate (D) at (10, 0); % OD = 6
        \coordinate (C) at (4, 8);  % CO = 8
        \coordinate (B) at (-2, 8); % BC = 6
        
        \draw[thick] (A) -- (B) -- (C) -- (D) -- cycle;
        \draw[thick] (B) -- (O);
        \draw[thick] (C) -- (O);
        
        % Right angle
        \draw (O) rectangle ++(0.8,0.8);
        
        % Labels
        \node[below left] at (A) {$A$};
        \node[above left] at (B) {$B$};
        \node[above right] at (C) {$C$};
        \node[below right] at (D) {$D$};
        \node[below] at (O) {$O$};
        
        \node[above] at ($(B)!0.5!(C)$) {$6$ \textit{см}};
        \node[right] at ($(C)!0.5!(D)$) {$10$ \textit{см}};
        
        % Marks on AB and CD (isosceles)
        \draw[thick] ($(A)!0.5!(B)$) ++(150:0.5) -- ++(-30:1);
        \draw[thick] ($(C)!0.5!(D)$) ++(60:0.5) -- ++(-120:1);
        
    \end{tikzpicture}
    \end{flushright}
\end{minipage}

\vspace{0.3cm}

\matchingLayout{
    \textit{Відрізок} \par \vspace{0.2cm}
    \textbf{1} \quad $OD$ \\
    \textbf{2} \quad $CO$ \\
    \textbf{3} \quad середня лінія \par \quad трапеції $ABCD$
}{
    \textit{Довжина відрізка} \par \vspace{0.2cm}
    \begin{tabular}{ll}
    \textbf{А} & 6 \textit{см} \\
    \textbf{Б} & 8 \textit{см} \\
    \textbf{В} & 9 \textit{см} \\
    \textbf{Г} & 12 \textit{см} \\
    \textbf{Д} & 18 \textit{см} \\
    \end{tabular}
}{
    \answerGrid
}

\vspace{0.7cm}

% === ЗАВДАННЯ 5 ===
\noindent\makebox[1.5em][l]{\textbf{5.}}\parbox[t]{\dimexpr\textwidth-1.5em}{Обчисліть довжину середньої лінії трапеції, основи якої дорівнюють $7$ \textit{м} і $12$ \textit{м}.}

\vspace{0.3cm}
\answerTable{$5$ \textit{м}}{$6$ \textit{м}}{$8{,}5$ \textit{м}}{$19$ \textit{м}}{$9{,}5$ \textit{м}}

\vspace{0.7cm}

% === ЗАВДАННЯ 6 ===
\noindent\textbf{6.} \begin{minipage}[t]{0.55\textwidth}
У прямокутній системі координат на площині задано трапецію $ABCD$ (див. рисунок). Обчисліть площу цієї трапеції.
\end{minipage}
\hfill
\begin{minipage}[t]{0.4\textwidth}
    \vspace{-0.5cm}
    \begin{flushright}
    \begin{tikzpicture}[scale=0.5]
        % Сітка
        \draw[step=1cm,gray!50,very thin] (-3,-3) grid (5,6);
        
        % Осі
        \draw[->, >=stealth] (-3,0) -- (5,0) node[below]{$x$};
        \draw[->, >=stealth] (0,-3) -- (0,6) node[left]{$y$};
        
        % Точки
        % A(-2, -2)
        % D(4, -2)
        % B(-1, 4)
        % C(2, 4)
        \coordinate (A) at (-2,-2);
        \coordinate (D) at (4,-2);
        \coordinate (B) at (-1,4);
        \coordinate (C) at (2,4);
        
        \draw[thick] (A) -- (B) -- (C) -- (D) -- cycle;
        
        \node[left] at (A) {$A$};
        \node[right] at (D) {$D$};
        \node[above left] at (B) {$B$};
        \node[above right] at (C) {$C$};
        
        \node[below left] at (0,0) {$0$};
        \node[left] at (0,1) {$1$};
        \draw (-0.1,1) -- (0.1,1);
        \node[below] at (1,0) {$1$};
        \draw (1,-0.1) -- (1,0.1);
        
        \fill (A) circle (3pt);
        \fill (B) circle (3pt);
        \fill (C) circle (3pt);
        \fill (D) circle (3pt);
    \end{tikzpicture}
    \end{flushright}
\end{minipage}

\vspace{0.3cm}
\answerTable{$54$}{$24$}{$27$}{$25$}{$45$}

% === ЗАВДАННЯ 7 ===
\noindent\textbf{7.} \begin{minipage}[t]{0.55\textwidth}
У рівнобічній трапеції $ABCD$ $AB=BC=CD=6$ \textit{см}, діагональ $AC$ утворює з основою $AD$ кут $30^\circ$ (див. рисунок). До кожного відрізка (1--3) доберіть його довжину (А--Д).
\end{minipage}
\hfill
\begin{minipage}[t]{0.4\textwidth}
    \vspace{-0.5cm}
    \begin{flushright}
    \begin{tikzpicture}[scale=0.3]
        % Рівнобічна трапеція. AB=BC=CD=6.
        % Кут CAD = 30. Оскільки BC || AD, кут BCA = 30.
        % Трикутник ABC рівнобедрений (AB=BC), отже кут BAC = 30.
        % Тоді кут A = 60.
        % Висота = 6 * sin(60) = 3*sqrt(3) approx 5.2
        % Проєкція AB на AD = 3.
        % A=(0,0), B=(3, 5.2), C=(9, 5.2), D=(12, 0).
        
        \coordinate (A) at (0,0);
        \coordinate (B) at (3,5.2);
        \coordinate (C) at (9,5.2);
        \coordinate (D) at (12,0);
        
        \draw[thick] (A) -- (B) -- (C) -- (D) -- cycle;
        \draw[thick] (A) -- (C);
        
        % Кут 30 градусів
        \pic [draw, pic text={\scriptsize $30^\circ$}, angle radius=0.5cm, angle eccentricity=1.8] {angle = D--A--C};
        
        % Позначки рівності AB=BC=CD
        \draw[thick] ($(A)!0.5!(B)$) ++(150:0.4) -- ++(-30:0.8);
        \draw[thick] ($(B)!0.5!(C)$) ++(90:0.4) -- ++(-90:0.8);
        \draw[thick] ($(C)!0.5!(D)$) ++(60:0.4) -- ++(-120:0.8);
        
        \node[below left] at (A) {$A$};
        \node[above left] at (B) {$B$};
        \node[above right] at (C) {$C$};
        \node[below right] at (D) {$D$};
    \end{tikzpicture}
    \end{flushright}
\end{minipage}

\vspace{0.3cm}

\matchingLayout{
    \textit{Відрізок} \par \vspace{0.2cm}
    \textbf{1} \quad основа $AD$ \\
    \textbf{2} \quad висота трапеції \par \quad $ABCD$ \\
    \textbf{3} \quad діагональ $AC$
}{
    \textit{Довжина відрізка} \par \vspace{0.2cm}
    \begin{tabular}{ll}
    \textbf{А} & $6\sqrt{3}$ \textit{см} \\
    \textbf{Б} & 6 \textit{см} \\
    \textbf{В} & $3\sqrt{3}$ \textit{см} \\
    \textbf{Г} & 12 \textit{см} \\
    \textbf{Д} & 3 \textit{см} \\
    \end{tabular}
}{
    \answerGrid
}

\vspace{0.7cm}

% === ЗАВДАННЯ 8 ===
\noindent\textbf{8.} \begin{minipage}[t]{0.55\textwidth}
У рівнобічній трапеції $ABCD$ діагоналі $AC$ і $BD$ взаємно перпендикулярні і перетинаються в точці $O$, $KM$ --- середня лінія трикутника $AOD$, $BO = 6\sqrt{2}$ \textit{см}, $KM = 12$ \textit{см} (див. рисунок). До кожного відрізка (1--3) доберіть його довжину (А--Д).
\end{minipage}
\hfill
\begin{minipage}[t]{0.4\textwidth}
    \vspace{-0.5cm}
    \begin{flushright}
    \begin{tikzpicture}[scale=0.18]
        % Трапеція з перп діагоналями.
        % AOD рівнобедрений прямокутний.
        % KM середня лінія AOD = 12 => AD = 24.
        % Висота з O на AD = 12.
        % A = (-12, -12), D = (12, -12). O = (0,0).
        % BO = 6*sqrt(2) approx 8.5. 
        % BOC рівнобедрений прямокутний. Висота = BO/sqrt(2) = 6.
        % BC = 12. B = (-6, 6), C = (6, 6).
        
        \coordinate (O) at (0,0);
        \coordinate (A) at (-12,-12);
        \coordinate (D) at (12,-12);
        \coordinate (B) at (-6,6);
        \coordinate (C) at (6,6);
        
        % Середня лінія KM трикутника AOD
        \coordinate (K) at ($(A)!0.5!(O)$);
        \coordinate (M) at ($(D)!0.5!(O)$);
        
        \draw[thick] (A) -- (B) -- (C) -- (D) -- cycle;
        \draw[thick] (A) -- (C);
        \draw[thick] (B) -- (D);
        \draw[thick] (K) -- (M);
        
        % Прямий кут
        \draw (O) ++(-135:1.5) -- ++(-45:1.5) -- ++(45:1.5); % Схематично
        
        % Позначки рівнобічності
        \draw[thick] ($(A)!0.5!(B)$) ++(150:1) -- ++(-30:2);
        \draw[thick] ($(C)!0.5!(D)$) ++(60:1) -- ++(-120:2);
        
        \node[below left] at (A) {$A$};
        \node[above left] at (B) {$B$};
        \node[above right] at (C) {$C$};
        \node[below right] at (D) {$D$};
        \node[above] at (O) {$O$};
        \node[below] at (K) {$K$};
        \node[below] at (M) {$M$};
        
        \fill (K) circle (10pt);
        \fill (M) circle (10pt);
        
    \end{tikzpicture}
    \end{flushright}
\end{minipage}

\vspace{0.3cm}

\matchingLayout{
    \textit{Відрізок} \par \vspace{0.2cm}
    \textbf{1} \quad $BC$ \\
    \textbf{2} \quad $AD$ \\
    \textbf{3} \quad висота трапеції \par \quad $ABCD$
}{
    \textit{Довжина відрізка} \par \vspace{0.2cm}
    \begin{tabular}{ll}
    \textbf{А} & 12 \textit{см} \\
    \textbf{Б} & 15 \textit{см} \\
    \textbf{В} & 18 \textit{см} \\
    \textbf{Г} & 21 \textit{см} \\
    \textbf{Д} & 24 \textit{см} \\
    \end{tabular}
}{
    \answerGrid
}

\vspace{0.7cm}

% === ЗАВДАННЯ 9 ===
\noindent\makebox[1.5em][l]{\textbf{9.}}\parbox[t]{\dimexpr\textwidth-1.5em}{Довжини сторін трапеції $ABCD$ ($AD \parallel BC$) задовольняють співвідношення $AB : BC : CD : AD = 2 : 3 : 4 : 7$. Точки $K$ і $M$ --- середини сторін $AB$ і $CD$ відповідно. Обчисліть периметр чотирикутника $AKMD$, якщо периметр трапеції $ABCD$ дорівнює $80$ \textit{см}.}

\vspace{0.3cm}
\answerTable{$60$ \textit{см}}{$55$ \textit{см}$}{$70$ \textit{см}}{$40$ \textit{см}}{$75$ \textit{см}}

\vspace{0.7cm}

% === ЗАВДАННЯ 10 ===
\noindent\makebox[1.5em][l]{\textbf{10.}}\parbox[t]{\dimexpr\textwidth-1.5em}{На сторонах $AB$ та $BC$ довільного трикутника $ABC$ вибрано точки $M$ та $N$ відповідно так, що $AM = MB$, $CN = NB$. Які з наведених тверджень є правильними?
\begin{enumerate}[label=\Roman*., nosep, leftmargin=*]
    \item Чотирикутник $AMNC$ є трапецією.
    \item Трикутник $MBN$ є рівнобедреним.
    \item Периметр трикутника $MBN$ дорівнює половині периметра трикутника $ABC$.
\end{enumerate}}

\vspace{0.3cm}
\answerTable{лише II}{лише I та III}{лише I та II}{I, II та III}{лише II та III}

\vspace{0.7cm}

% === ЗАВДАННЯ 11 ===
\noindent\textbf{11.} \begin{minipage}[t]{0.55\textwidth}
На рисунку зображено прямокутну трапецію $ABCD$, у якої $AC = DC = 15$ \textit{см}, $BC = 12$ \textit{см}. $O$ --- середина діагоналі $AC$ трапеції $ABCD$. До кожного відрізка (1--3) доберіть його довжину (А--Д).
\end{minipage}
\hfill
\begin{minipage}[t]{0.4\textwidth}
    \vspace{-0.5cm}
    \begin{flushright}
    \begin{tikzpicture}[scale=0.18]
        % Прямокутна трапеція. A=90, B=90.
        % BC=12. AC=15. => AB = sqrt(225-144) = 9.
        % DC=15. Проведемо висоту CH. CH=9. HD = sqrt(225-81)=12.
        % AD = AH + HD = 12 + 12 = 24.
        
        \coordinate (A) at (0,0);
        \coordinate (B) at (0,9);
        \coordinate (C) at (12,9);
        \coordinate (D) at (24,0); % AD візуально довга
        
        \coordinate (O) at ($(A)!0.5!(C)$);
        
        \draw[thick] (A) -- (B) -- (C) -- (D) -- cycle;
        \draw[thick] (A) -- (C);
        
        % Прямий кут
        \draw (A) rectangle ++(1.5,1.5);
        
        
        
        % Позначки середини O (AO=OC вже є частиною AC, але на малюнку O окремо)
        % На малюнку штрих на AO і штрих на OC, але AC=DC, тому там по одному штриху скрізь?
        % Ні, на малюнку AC перекреслено рискою і DC рискою.
        % А O --- середина, тому там маленькі точки або інші позначки?
        % На скріншоті: AC поділено точкою O. На AO риска, на OC риска. На CD риска.
        % Тобто AO = OC = CD? Ні, це неможливо (7.5 != 15).
        % Скоріше: AC і CD мають однакові позначки (одна риска). 
        % А O просто точка.
        % Але в тексті O - середина.
        % Дивимось уважно на скріншот 21.34.27.
        % На AO одна риска, на OC одна риска. 
        % На CD ТЕЖ одна риска? Якщо так, то AO=OC=CD=7.5. Тоді AC=15, CD=7.5.
        % АЛЕ в умові AC=DC=15.
        % Значить на CD має бути інша позначка або це помилка рисунка/сприйняття.
        % На малюнку риска на AO, риска на OC. І риска на CD.
        % Це візуально, відтворимо як є (риски).
        
        \draw[thick] ($(A)!0.5!(O)$) ++(120:0.8) -- ++(-60:1.6);
        \draw[thick] ($(O)!0.5!(C)$) ++(120:0.8) -- ++(-60:1.6);
        % Риска на CD
        %\draw[thick] ($(C)!0.5!(D)$) ++(30:0.8) -- ++(-150:1.6); 
        % Щоб не плутати, не буду ставити риску на CD таку ж саму, бо це математично неправильно, 
        % але якщо треба "суто рисунок", то ось:
        
        \node[below left] at (A) {$A$};
        \node[above left] at (B) {$B$};
        \node[above] at (C) {$C$};
        \node[below] at (D) {$D$};
        \node[below right] at (O) {$O$};
        
        \fill (O) circle (12pt);
    \end{tikzpicture}
    \end{flushright}
\end{minipage}

\vspace{0.3cm}

\matchingLayout{
    \textit{Відрізок} \par \vspace{0.2cm}
    \textbf{1} \quad $OB$ \\
    \textbf{2} \quad $AB$ \\
    \textbf{3} \quad середня лінія \par \quad трапеції $ABCD$
}{
    \textit{Довжина відрізка} \par \vspace{0.2cm}
    \begin{tabular}{ll}
    \textbf{А} & 7{,}5 \textit{см} \\
    \textbf{Б} & 8 \textit{см} \\
    \textbf{В} & 9 \textit{см} \\
    \textbf{Г} & 15 \textit{см} \\
    \textbf{Д} & 18 \textit{см} \\
    \end{tabular}
}{
    \answerGrid
}

% === ЗАВДАННЯ 12 ===
\noindent\makebox[1.5em][l]{\textbf{12.}}\parbox[t]{\dimexpr\textwidth-1.5em}{Які з наведених тверджень є правильними?
\begin{enumerate}[label=\Roman*., nosep, leftmargin=*]
    \item Середня лінія трапеції проходить через точку перетину її діагоналей.
    \item Діагональ трапеції ділить її на два рівних трикутники.
    \item Діагоналі рівнобічної трапеції рівні.
\end{enumerate}}

\vspace{0.3cm}
\answerTable{лише I та III}{лише II та III}{лише III}{лише I та II}{I, II та III}

\vspace{0.7cm}

% === ЗАВДАННЯ 13 ===
\noindent\makebox[1.5em][l]{\textbf{13.}}\parbox[t]{\dimexpr\textwidth-1.5em}{Знайдіть периметр рівнобічної трапеції, основи якої дорівнюють $18$ \textit{см} і $7$ \textit{см}, а бічна сторона --- $11$ \textit{см}.}

\vspace{0.3cm}
\answerTable{$57$ \textit{см}$}{$42$ \textit{см}}{$47$ \textit{см}}{$37$ \textit{см}}{$52$ \textit{см}}

\vspace{0.7cm}

% === ЗАВДАННЯ 14 ===
\noindent\textbf{14.} \begin{minipage}[t]{0.55\textwidth}
У рівнобічній трапеції $ABCD$ проведено висоту $CK$ так, що $AK = 16$ \textit{см}, $KD = 9$ \textit{см} (див. рисунок). Діагональ $AC$ перпендикулярна до бічної сторони $CD$. Визначте площу цієї трапеції.
\end{minipage}
\hfill
\begin{minipage}[t]{0.4\textwidth}
    \vspace{-0.5cm}
    \begin{flushright}
    \begin{tikzpicture}[scale=0.18]
        % AK=16, KD=9 => AD=25.
        % Висота CK^2 = AK * KD (метричні співвідношення у прямокутному трикутнику ACD)
        % CK^2 = 16 * 9 = 144 => CK = 12.
        % Рівнобічна трапеція. AD=25. 
        % Проєкція бічної сторони CD на AD це KD=9.
        % Тоді проєкція AB на AD теж 9.
        % BC = AD - 9 - 9 = 25 - 18 = 7.
        % A=(0,0), D=(25,0), K=(16,0).
        % C=(16, 12). B=(9, 12).
        
        \coordinate (A) at (0,0);
        \coordinate (D) at (25,0);
        \coordinate (K) at (16,0);
        \coordinate (C) at (16,12);
        \coordinate (B) at (9,12);
        
        \draw[thick] (A) -- (B) -- (C) -- (D) -- cycle;
        \draw[thick] (C) -- (K);
        \draw[thick] (A) -- (C);
        
        % Прямі кути
        \draw (K) rectangle ++(1.5,1.5); % CKD
        
        % Точніше, прямий кут при вершині C у трикутнику ACD
        \pic [draw, angle radius=0.4cm] {right angle = A--C--D};
        
        % Позначки рівнобічності
        \draw[thick] ($(A)!0.5!(B)$) ++(150:0.8) -- ++(-30:1.6);
        \draw[thick] ($(C)!0.5!(D)$) ++(60:0.8) -- ++(-120:1.6);
        
        \node[below] at (A) {$A$};
        \node[above] at (B) {$B$};
        \node[above] at (C) {$C$};
        \node[below] at (D) {$D$};
        \node[below] at (K) {$K$};
        
    \end{tikzpicture}
    \end{flushright}
\end{minipage}

\vspace{0.3cm}
\answerTable{$384$ \textit{см}$^2$}{$240$ \textit{см}$^2$}{$288$ \textit{см}$^2$}{$144$ \textit{см}$^2$}{$192$ \textit{см}$^2$}

\vspace{0.7cm}

% === ЗАВДАННЯ 15 ===
\noindent\textbf{15.} \begin{minipage}[t]{0.55\textwidth}
У прямокутній трапеції $ABCD$ бічні сторони $AB$ і $CD$ дорівнюють $12$ \textit{см} і $20$ \textit{см} відповідно, а $BC : AD = 3 : 5$ (див. рисунок). Знайдіть площу цієї трапеції.
\end{minipage}
\hfill
\begin{minipage}[t]{0.4\textwidth}
    \vspace{-0.5cm}
    \begin{flushright}
    \begin{tikzpicture}[scale=0.15]
        % AB=12 (висота). CD=20.
        % Проведемо висоту з C на AD (CH). CH=12.
        % Трикутник CHD: CD=20, CH=12 => HD = sqrt(400-144) = 16.
        % BC = 3x, AD = 5x.
        % AD - BC = HD => 5x - 3x = 16 => 2x = 16 => x = 8.
        % BC = 24, AD = 40.
        
        \coordinate (A) at (0,0);
        \coordinate (B) at (0,12);
        \coordinate (C) at (24,12); % BC = 24
        \coordinate (D) at (40,0);  % AD = 40
        
        \draw[thick] (A) -- (B) -- (C) -- (D) -- cycle;
        
        % Прямий кут
        \draw (A) rectangle ++(2,2);
        
        \node[below left] at (A) {$A$};
        \node[above left] at (B) {$B$};
        \node[above right] at (C) {$C$};
        \node[below right] at (D) {$D$};
        
    \end{tikzpicture}
    \end{flushright}
\end{minipage}

\vspace{0.3cm}
\answerTable{$192$ \textit{см}$^2$}{$96$ \textit{см}$^2$}{$364$ \textit{см}$^2$}{$768$ \textit{см}$^2$}{$384$ \textit{см}$^2$}

\begin{center}
{\Large\textbf{\color{headerblue}БАЗА ЗАВДАНЬ НМТ 2024}}
\end{center}

% === ЗАВДАННЯ 16 ===
\noindent\makebox[1.5em][l]{\textbf{16.}}\parbox[t]{\dimexpr\textwidth-1.5em}{Які з наведених тверджень є правильними?

\vspace{0.1cm}
I. \hspace{0.2em} Існує трапеція, у якої одна з бічних сторін перпендикулярна до її основ. \\
II. \hspace{0.1em} Існує трапеція, у якої суми протилежних сторін рівні. \\
III. Існує трапеція, у якої суми протилежних кутів рівні.}

\vspace{0.3cm}
\answerTable{лише II та III}{I, II та III}{лише I}{лише II}{лише III}

\vspace{0.7cm}

% === ЗАВДАННЯ 17 ===
\noindent\textbf{17.} \begin{minipage}[t]{0.55\textwidth}
На рисунку зображено рівнобічну трапецію $ABCD$ ($BC \parallel AD$), $\angle DAB = 28^\circ$ (див. рисунок). Знайдіть градусну міру кута $BCD$. \nmtyear{2024}
\end{minipage}
\hfill
\begin{minipage}[t]{0.4\textwidth}
    \vspace{-0.5cm}
    \begin{flushright}
    \begin{tikzpicture}[scale=0.25]
        \coordinate (A) at (-8,0);
        \coordinate (B) at (-4,5);
        \coordinate (C) at (4,5);
        \coordinate (D) at (8,0);
        
        \draw[thick] (A) -- (B) -- (C) -- (D) -- cycle;
        
        % Кут 28
        \pic [draw, pic text={\small $28^\circ$}, angle radius=0.5cm, angle eccentricity=1.7] {angle = D--A--B};
        
        % Позначки рівнобічності
        \draw[thick] ($(A)!0.5!(B)$) ++(150:0.5) -- ++(-30:1);
        \draw[thick] ($(C)!0.5!(D)$) ++(60:0.5) -- ++(-120:1);
        
        \node[below left] at (A) {$A$};
        \node[above left] at (B) {$B$};
        \node[above right] at (C) {$C$};
        \node[below right] at (D) {$D$};
    \end{tikzpicture}
    \end{flushright}
\end{minipage}

\vspace{0.3cm}
\answerTable{$152^\circ$}{$162^\circ$}{$142^\circ$}{$62^\circ$}{$118^\circ$}

\vspace{0.7cm}

% === ЗАВДАННЯ 18 ===
\noindent\textbf{18.} \begin{minipage}[t]{0.55\textwidth}
На більшій основі $AD$ рівнобічної трапеції $ABCD$ вибрано точку $O$ так, що $BO \parallel CD$, $AO = OD$ (див. рисунок). $AD = 12$, $\angle BAD = \alpha$. Знайдіть площу цієї трапеції. \nmtyear{2024}
\end{minipage}
\hfill
\begin{minipage}[t]{0.4\textwidth}
    \vspace{-0.5cm}
    \begin{flushright}
    \begin{tikzpicture}[scale=0.35]
        \coordinate (A) at (-6,0);
        \coordinate (D) at (6,0);
        \coordinate (O) at (0,0);
        
        % AO = OD = 6.
        % BO || CD. Trapezia ABCD is isosceles -> AB = CD.
        % Quadrilateral OBCD: BC || OD and BO || CD -> Parallelogram.
        % BC = OD = 6. BO = CD = AB.
        % Triangle ABO: AB = BO = 6. Isosceles triangle.
        % Angle A = alpha.
        % Let's draw it symmetric visually.
        
        \coordinate (B) at (-3, 5.2); % Approx
        \coordinate (C) at (3, 5.2);
        
        \draw[thick] (A) -- (B) -- (C) -- (D) -- cycle;
        \draw[thick] (B) -- (O);
        
        % Marks
        \draw[thick] ($(A)!0.5!(B)$) ++(150:0.4) -- ++(-30:0.8); % AB
        \draw[thick] ($(C)!0.5!(D)$) ++(60:0.4) -- ++(-120:0.8); % CD
        
        % AO = OD marks
        \draw[thick] ($(A)!0.5!(O)$) ++(0, -0.3) -- ++(0, 0.6); 
        \draw[thick] ($(A)!0.5!(O)$) ++(0.2, -0.3) -- ++(0.2, 0.6); % double
        
        \draw[thick] ($(O)!0.5!(D)$) ++(0, -0.3) -- ++(0, 0.6);
        \draw[thick] ($(O)!0.5!(D)$) ++(0.2, -0.3) -- ++(0.2, 0.6); % double
        
        % Angle alpha
        \pic [draw, pic text={\small $\alpha$}, angle radius=.5cm, angle eccentricity=1.7] {angle = D--A--B};

        \node[below left] at (A) {$A$};
        \node[above left] at (B) {$B$};
        \node[above right] at (C) {$C$};
        \node[below right] at (D) {$D$};
        \node[below] at (O) {$O$};
        
        \fill (O) circle (4pt);
    \end{tikzpicture}
    \end{flushright}
\end{minipage}

\vspace{0.3cm}
\answerTableTall{$27\,\mathrm{tg}\,\alpha$}{$27\sin\alpha$}{$54\,\mathrm{tg}\,\alpha$}{$\dfrac{27}{\sin\alpha}$}{$\dfrac{27}{\mathrm{tg}\,\alpha}$}

\vspace{0.7cm}

% === ЗАВДАННЯ 19 ===
\noindent\textbf{19.} \begin{minipage}[t]{0.55\textwidth}
На більшій основі $AD$ рівнобічної трапеції $ABCD$ вибрано точку $K$ так, що $BK \parallel CD$, $CK \parallel AB$ (див. рисунок). $KD=8$, $\angle BAD = \beta$. Знайдіть площу цієї трапеції. \nmtyear{2024}
\end{minipage}
\hfill
\begin{minipage}[t]{0.4\textwidth}
    \vspace{-0.5cm}
    \begin{flushright}
    \begin{tikzpicture}[scale=0.3]
        % AB || CK, BK || CD.
        % Triangles ABK, KBC, KCD looks congruent equilateral-ish.
        % Isosceles trapezoid. AB=CD.
        % AB || CK => ABCK parallelogram => BC = AK, AB = CK.
        % BK || CD => KBCD parallelogram => BC = KD, BK = CD.
        % So AK = BC = KD = 8. AD = 16.
        % Triangles are congruent.
        
        \coordinate (A) at (-8,0);
        \coordinate (K) at (0,0);
        \coordinate (D) at (8,0);
        
        \coordinate (B) at (-4, 6.9); % Height for equilateral roughly
        \coordinate (C) at (4, 6.9);
        
        \draw[thick] (A) -- (B) -- (C) -- (D) -- cycle;
        \draw[thick] (B) -- (K);
        \draw[thick] (C) -- (K);
        
        % Angle beta
        \pic [draw, pic text={\small $\beta$}, angle radius=0.5cm, angle eccentricity=1.7] {angle = D--A--B};
        
        % Marks on sides
        \draw[thick] ($(A)!0.5!(B)$) ++(150:0.4) -- ++(-30:0.8);
        \draw[thick] ($(C)!0.5!(D)$) ++(60:0.4) -- ++(-120:0.8);

        \node[below left] at (A) {$A$};
        \node[above left] at (B) {$B$};
        \node[above right] at (C) {$C$};
        \node[below right] at (D) {$D$};
        \node[below] at (K) {$K$};
        
        \fill (K) circle (5pt);
    \end{tikzpicture}
    \end{flushright}
\end{minipage}

\vspace{0.3cm}
\answerTableTall{$12\sin\beta$}{$12\,\mathrm{tg}\,\beta$}{$\dfrac{48}{\mathrm{tg}\,\beta}$}{$48\,\mathrm{tg}\,\beta$}{$\dfrac{12}{\mathrm{tg}\,\beta}$}

\vspace{0.7cm}

% === ЗАВДАННЯ 20 ===
\noindent\textbf{20.} \begin{minipage}[t]{0.55\textwidth}
У рівнобічній трапеції $ABCD$ ($BC \parallel AD$) проведено висоту $CO$ (див. рисунок). Середня лінія трапеції дорівнює $23$ \textit{см}, $AD = 38$ \textit{см}, $\angle AOB = 45^\circ$. До кожного відрізка (1--3) доберіть його довжину (А--Д). \nmtyear{2024}
\end{minipage}
\hfill
\begin{minipage}[t]{0.4\textwidth}
    \vspace{-0.5cm}
    \begin{flushright}
    \begin{tikzpicture}[scale=0.18]
        \coordinate (A) at (-20,0);
        \coordinate (D) at (18,0); % AD = 38 total width
        \coordinate (O) at (0,0);  % O is projection of C? No, CO is height.
        % Text says CO is height. So C is directly above O.
        % Let's set O as (0,0). C = (0, h).
        % ABCD isosceles.
        % Middle line = 23. (BC+AD)/2 = 23 => BC + 38 = 46 => BC = 8.
        % If BC=8, and O is projection of C, then OD = (AD-BC)/2 is wrong.
        % In isosceles trapezoid, OD = (AD+BC)/2 = (38+8)/2 = 23? No.
        % Usually projection of C is H. HD = (AD-BC)/2 = (38-8)/2 = 15.
        % AH = (AD+BC)/2 = 23.
        % So if O is the foot of height C, then AO = 23, OD = 15 (if D is right).
        % Wait, drawing shows O is between A and D.
        % Angle AOB = 45. B is top left vertex.
        % Let's construct: O=(0,0). C=(0,h). D=(15,0). A=(-23,0).
        % B must be at (-15, h).
        % Connect B to O. Angle AOB is angle between AO (x-axis left) and BO.
        % If B=(-15, h), and angle(BO, AO) = 45, then B must be on y=x line (relative to O, but A is left).
        % So angle between (-1,0) and (-15, h) is 45.
        % This implies height h = 15.
        
        \coordinate (O) at (0,0);
        \coordinate (C) at (0,15);
        \coordinate (D) at (15,0);
        \coordinate (A) at (-23,0);
        \coordinate (B) at (-15,15); % BC = 15 - (-15)? No, B is symmetric to C relative to axis? No.
        % Isosceles trapezoid properties:
        % Projection of B (let's say B') and C (O) on AD.
        % B'O = BC. B'A = OD.
        % O is projection of C. OD = (AD-BC)/2 = 15. Correct.
        % AO = AD - OD = 38 - 15 = 23. Correct.
        % B is at x = -8 (relative to O). No, B' is at -8. B is at (-8, 15).
        % BC = 8.
        % Check angle AOB = 45?
        % Vector OA = (-23, 0). Vector OB = (-8, 15).
        % This does not look like 45 degrees.
        % Let's look at the drawing "21.52.02".
        % Visual: O is distinct point on base. C is above O. CO perp AD.
        % B connects to O. Angle AOB is marked.
        % Angle is clearly acute. 45 degrees.
        
        \draw[thick] (A) -- (B) -- (C) -- (D) -- cycle;
        \draw[thick] (C) -- (O);
        \draw[thick] (B) -- (O);
        
        % Right angle at O
        \draw (O) rectangle ++(1.5,1.5);
        
        % Angle 45 at O (AOB)
        \pic [draw, pic text={\small $45^\circ$}, angle radius=.6cm, angle eccentricity=1.5] {angle = B--O--A};
        
        % Marks
        \draw[thick] ($(A)!0.5!(B)$) ++(150:0.8) -- ++(-30:1.6);
        \draw[thick] ($(C)!0.5!(D)$) ++(60:0.8) -- ++(-120:1.6);
        
        \node[below left] at (A) {$A$};
        \node[above left] at (B) {$B$};
        \node[above right] at (C) {$C$};
        \node[below right] at (D) {$D$};
        \node[below] at (O) {$O$};
        
    \end{tikzpicture}
    \end{flushright}
\end{minipage}

\vspace{0.3cm}

\matchingLayout{
    \textit{Відрізок} \par \vspace{0.2cm}
    \textbf{1} \quad $BC$ \\
    \textbf{2} \quad $OD$ \\
    \textbf{3} \quad $AB$
}{
    \textit{Довжина відрізка} \par \vspace{0.2cm}
    \begin{tabular}{ll}
    \textbf{А} & 8 \textit{см} \\
    \textbf{Б} & 12 \textit{см} \\
    \textbf{В} & 15 \textit{см} \\
    \textbf{Г} & 17 \textit{см} \\
    \textbf{Д} & 18 \textit{см} \\
    \end{tabular}
}{
    \answerGrid
}

\vspace{0.7cm}

% === ЗАВДАННЯ 21 ===
\noindent\textbf{21.} \begin{minipage}[t]{0.55\textwidth}
Навколо кола описано рівнобічну трапецію (див. рисунок), периметр якої дорівнює $100$ \textit{см}. Різниця основ трапеції дорівнює $14$ \textit{см}. До кожного початку речення (1--3) доберіть його закінчення (А--Д) так, щоб утворилося правильне твердження. \nmtyear{2024}
\end{minipage}
\hfill
\begin{minipage}[t]{0.4\textwidth}
    \vspace{-0.5cm}
    \begin{flushright}
    \begin{tikzpicture}[scale=0.15]
        % Circumscribed isosceles trapezoid.
        % Sum of bases = Sum of legs = Perimeter / 2 = 50.
        % a + b = 50. a - b = 14.
        % 2a = 64 -> a = 32 (bottom). b = 18 (top).
        % c = 25.
        % Height h = sqrt(c^2 - ((a-b)/2)^2) = sqrt(25^2 - 7^2) = sqrt(625-49) = sqrt(576) = 24.
        % Radius = 12.
        
        \coordinate (O) at (0,12);
        \coordinate (A) at (-16,0);
        \coordinate (D) at (16,0);
        \coordinate (B) at (-9,24);
        \coordinate (C) at (9,24);
        
        \draw[thick] (A) -- (B) -- (C) -- (D) -- cycle;
        \draw[thick] (O) circle (12cm);
        
    \end{tikzpicture}
    \end{flushright}
\end{minipage}

\vspace{0.3cm}

\matchingLayout{
    \textit{Початок речення} \par \vspace{0.2cm}
    \textbf{1} \quad Довжина середньої лінії \par \quad трапеції \\
    \textbf{2} \quad Довжина більшої основи \par \quad трапеції \\
    \textbf{3} \quad Довжина висоти трапеції
}{
    \textit{Закінчення речення} \par \vspace{0.2cm}
    \begin{tabular}{ll}
    \textbf{А} & дорівнює 18 \textit{см}. \\
    \textbf{Б} & дорівнює 24 \textit{см}. \\
    \textbf{В} & дорівнює 25 \textit{см}. \\
    \textbf{Г} & дорівнює 32 \textit{см}. \\
    \textbf{Д} & дорівнює 36 \textit{см}. \\
    \end{tabular}
}{
    \answerGrid
}

% === ЗАВДАННЯ 14 ===
\noindent\textbf{22.} \begin{minipage}[t]{0.95\textwidth}
На паралельних прямих $m$ та $n$ побудовано прямокутник $ABCD$, прямокутну трапецію $DKLM$ і прямокутний трикутник $MQP$ (див. рисунок). Користуючись даними на рисунку, узгодьте фігуру (1--3) з її площею (А--Д). \nmtyear{2024}
\end{minipage}

\vspace{0.3cm}
\begin{center}
\begin{tikzpicture}[scale=0.6]
    % Coordinates logic based on numbers:
    % ABCD: width 4. height h.
    % DKLM: base KL=2, DM=5.
    % MQP: leg MQ=6. Angle 45 -> PQ=6. So h=6.
    
    \def\h{6}
    
    \coordinate (A) at (0,0);
    \coordinate (D) at (4,0);
    \coordinate (B) at (0,\h);
    \coordinate (C) at (4,\h);
    
    \coordinate (M) at (9,0); % D(4) + 5 = 9
    \coordinate (K) at (7,\h); % M(9) is vertical side? No, Trap DKLM usually means D->K->L->M.
    % Image: D to K is slant. L to M is vertical? 
    % Let's look at numbers. "2" is between K and L. "5" is between D and M.
    % Angle at M is not marked 90. But triangle MQP has right angle at Q.
    % The text says "Rectangular Trapezoid DKLM". In drawing, side LM is vertical.
    % So L is above M. M is at 9. L is at (9, h).
    % K is left of L. KL=2. K is at (7, h).
    % D is at (4,0).
    \coordinate (L) at (9,\h);
    
    \coordinate (Q) at (15,0); % M(9) + 6 = 15.
    \coordinate (P) at (15,\h); % P is above Q. Triangle MQP, right angle at Q.
    
    % Colors
    \fill[cyan!20] (A) -- (B) -- (C) -- (D) -- cycle;
    \fill[violet!20] (D) -- (K) -- (L) -- (M) -- cycle;
    \fill[yellow!20] (M) -- (Q) -- (P) -- cycle;
    
    % Lines
    \draw[thick] (-1,0) -- (16,0) node[above] {$n$};
    \draw[thick] (-1,\h) -- (16,\h) node[above] {$m$};
    
    % Shapes
    \draw[thick] (A) -- (B) -- (C) -- (D) -- cycle;
    \draw[thick] (D) -- (K) -- (L) -- (M); % Base is on line n
    \draw[thick] (M) -- (P) -- (Q); % Triangle MQP (Q is right angle)
    
    % Labels and Values
    \node[below] at (A) {$A$};
    \node[below] at (2,0) {4};
    \node[below] at (D) {$D$};
    \node[above] at (B) {$B$};
    \node[above] at (C) {$C$};
    
    \node[below] at (6.5,0) {5};
    \node[below] at (M) {$M$};
    \node[above] at (K) {$K$};
    \node[above] at (8, \h) {2};
    \node[above] at (L) {$L$};
    
    \node[below] at (12,0) {6};
    \node[below] at (Q) {$Q$};
    \node[above] at (P) {$P$};
    
    % Angles
    \draw (Q) ++(-0.4,0) -- ++(0,0.4) -- ++(0.4,0); % Right angle at Q
    \draw (M) ++(0,0.4) -- ++(-0.4,0) -- ++(0,-0.4); % Right angle at M (trap)? No, M is just point. 
    % Wait, text says DKLM is rectangular trap. Usually one leg is perp. Side LM corresponds to vertical line.
    
    % 45 degrees at M in triangle MQP?
    % Image shows angle 45 at M inside triangle MQP.
    \pic [draw, pic text={\small $45^\circ$}, angle radius=0.9cm, angle eccentricity=1.6] {angle = Q--M--P};

\end{tikzpicture}
\end{center}

\matchingLayout{
    \textit{Величина} \par \vspace{0.2cm}
    \textbf{1} \quad Прямокутник $ABCD$ \\
    \textbf{2} \quad Трапеція $DKLM$ \\
    \textbf{3} \quad Трикутник $MQP$
}{

    \textit{Значення величини} \par \vspace{0.2cm}
    \begin{tabular}{ll}
    \textbf{А} & 12 \\
    \textbf{Б} & 18 \\
    \textbf{В} & 21 \\
    \textbf{Г} & 24 \\
    \textbf{Д} & 36 \\
    \end{tabular}
}{
    \answerGrid
}

% === ЗАВДАННЯ 14 (Відповідність з кольоровим рисунком) ===
\noindent\textbf{23.} \begin{minipage}[t]{0.95\textwidth}
На паралельних прямих $m$ та $n$ розміщено основи трапеції $ABCD$, сторони квадрата $DKLM$ та сторони паралелограма $MNPQ$ (див. рисунок). Периметр квадрата дорівнює $24$, $BC=KL$, $BC:AD = 2:3$, $AD=MQ$. Узгодьте фігуру (1--3) з її площею (А--Д).\nmtyear{2024}
\end{minipage}

\vspace{0.3cm}
\begin{center}
\begin{tikzpicture}[scale=0.6]
    % Лінії m та n
    
    
    % Координати
    % Висота квадрата = 24/4 = 6. У нас масштаб 0.6, тому y=4 відповідає 6 одиницям
    % BC = KL = 6. AD = 9 (бо 6:AD=2:3). MQ = 9.
    % Приймемо в TikZ одиницях: Висота = 4. Тоді реальна = 6. k = 1.5
    % Тоді ширина квадрата в TikZ = 4.
    
    % Квадрат DKLM
    \coordinate (D) at (4,0);
    \coordinate (M) at (8,0);
    \coordinate (L) at (8,4);
    \coordinate (K) at (4,4);
    
  
    
    
    % Трапеція ABCD
    % BC = 6 (реальних) -> 4 (TikZ). AD = 9 (реальних) -> 6 (TikZ)
    % A має бути лівіше D на 6. 4-6 = -2.
    % B має бути над A? На рисунку кут A прямий.
    \coordinate (A) at (0,0); % Зсунемо все, щоб D було на 4
    % Тоді A = (4-4, 0)? Ні, AD=6 (TikZ). D=4, тоді A=-2.
    % Перерахуємо координати для гарного вигляду (зсув вправо)
    
    % Зсув +2 по X
    \coordinate (D) at (6,0);
    \coordinate (M) at (10,0);
    \coordinate (L) at (10,4);
    \coordinate (K) at (6,4);
    
    % Трапеція (прямокутна за рисунком)
    \coordinate (A) at (2,0); % AD = 4 (TikZ) -> 6 реальних? Ні.
    % Давайте простіше: Висота h. Квадрат h x h.
    % BC = h. AD = 1.5h. 
    % A=(0,0), B=(0,h), C=(h,h), D=(1.5h, 0).
    % Квадрат DKLM: D=(1.5h, 0), M=(2.5h, 0), L=(2.5h, h), K=(1.5h, h).
    % Паралелограм: M=(2.5h, 0), Q=(4h, 0) (бо MQ=AD=1.5h).
    
    % Масштаб h=3
    \coordinate (A) at (0,0);
    \coordinate (B) at (0,3);
    \coordinate (C) at (3,3); % BC=3
    \coordinate (D) at (4.5,0); % AD=4.5 (3*1.5)
    
    \coordinate (K_sq) at (4.5,3);
    \coordinate (L_sq) at (7.5,3); % KL=3
    \coordinate (M_sq) at (7.5,0);
    
    \coordinate (N) at (9.5,3); % На око зсув
    \coordinate (P) at (14,3); % NP = MQ = 4.5
    \coordinate (Q) at (12,0); % MQ = 4.5. M=7.5 -> Q=12.
    \coordinate (N_par) at (9.5,3); % P - N = 4.5. 14-9.5=4.5. OK.
    
    % Заливаємо
    \fill[cyan!20] (A) -- (B) -- (C) -- (D) -- cycle;
    \fill[violet!20] (D) -- (K_sq) -- (L_sq) -- (M_sq) -- cycle;
    \fill[yellow!20] (M_sq) -- (N_par) -- (P) -- (Q) -- cycle;
    
    % Контури
    \draw[thick] (A) -- (B) -- (C) -- (D) -- cycle; % Трапеція
    \draw[thick] (D) -- (K_sq) -- (L_sq) -- (M_sq) -- cycle; % Квадрат
    \draw[thick] (M_sq) -- (N_par) -- (P) -- (Q) -- cycle; % Паралелограм
    
    % Лінії m та n (довгі)
    \draw[thick] (-1,0) -- (15,0) node[above] {$n$};
    \draw[thick] (-1,3) -- (15,3) node[above] {$m$};
    
    % Прямий кут
    \draw (A) rectangle ++(0.3,0.3);
    
    % Підписи
    \node[below] at (A) {$A$};
    \node[above] at (B) {$B$};
    \node[above] at (C) {$C$};
    \node[below] at (D) {$D$};
    \node[above] at (K_sq) {$K$};
    \node[above] at (L_sq) {$L$};
    \node[below] at (M_sq) {$M$};
    \node[above] at (N_par) {$N$};
    \node[above] at (P) {$P$};
    \node[below] at (Q) {$Q$};
    
\end{tikzpicture}
\end{center}

\matchingLayout{
    \textbf{1} \quad квадрат $DKLM$ \\
    \textbf{2} \quad паралелограм $MNPQ$ \\
    \textbf{3} \quad трапеція $ABCD$
}{
    \begin{tabular}{ll}
    \textbf{А} & 48 \\
    \textbf{Б} & 90 \\
    \textbf{В} & 54 \\
    \textbf{Г} & 36 \\
    \textbf{Д} & 45 \\
    \end{tabular}
}{
    \answerGrid
}

\vspace{0.7cm}

\noindent\textbf{24.} \begin{minipage}[t]{0.55\textwidth}
На рисунку зображено квадрат $ABCD$ і прямокутний трикутник $KBC$ ($\angle B = 90^\circ$), що лежать в одній площині. Периметр квадрата $ABCD$ дорівнює $24$ \textit{см}, середня лінія трапеції $AKCD$ дорівнює $10$ \textit{см}. До кожного відрізка (1--3) доберіть його довжину (А--Д). \nmtyear{2024}
\end{minipage}
\hfill
\begin{minipage}[t]{0.4\textwidth}
    \vspace{-0.5cm}
    \begin{flushright}
    \begin{tikzpicture}[scale=0.3]
        % P=24 => сторона 6.
        % Трапеція AKCD. Основи AK і CD. CD=6. Середня лінія 10 -> AK+6 = 20 -> AK=14.
        % BK = 14-6 = 8.
        \coordinate (A) at (0,0);
        \coordinate (D) at (6,0);
        \coordinate (C) at (6,6);
        \coordinate (B) at (0,6);
        \coordinate (K) at (0,14);
        
        \draw[thick] (A) -- (D) -- (C) -- (K) -- cycle; % Контур трапеції
        \draw[thick] (B) -- (C); % Верхня сторона квадрата
        
        % Прямий кут
        \draw (0,6.5) -- (0.5,6.5) -- (0.5,6);
        
        \node[left] at (A) {$A$};
        \node[below right] at (D) {$D$};
        \node[right] at (C) {$C$};
        \node[left] at (B) {$B$};
        \node[left] at (K) {$K$};
    \end{tikzpicture}
    \end{flushright}
\end{minipage}

\vspace{0.3cm}

\matchingLayout{
    \textit{Відрізок} \par \vspace{0.2cm}
    \textbf{1} \quad $BK$ \\
    \textbf{2} \quad $KC$ \\
    \textbf{3} \quad відстань між центрами кіл, описаних навколо квадрата $ABCD$ та трикутника $KBC$
}{
    \textit{Довжина відрізка} \par \vspace{0.2cm}
    \begin{tabular}{ll}
    \textbf{А} & 6 \textit{см} \\
    \textbf{Б} & 7 \textit{см} \\
    \textbf{В} & 8 \textit{см} \\
    \textbf{Г} & 9 \textit{см} \\
    \textbf{Д} & 10 \textit{см} \\
    \end{tabular}
}{
    \answerGrid
}


\vspace{0.5cm}



% === ЗАВДАННЯ 25 ===
\noindent\makebox[1.5em][l]{\textbf{25.}}\parbox[t]{\dimexpr\textwidth-1.5em}{Які з наведених тверджень є правильними?\nmtyear{2024}

\vspace{0.1cm}
I. \hspace{0.2em} Середня лінія трапеції дорівнює півсумі її бічних сторін. \\
II. \hspace{0.1em} Середня лінія трапеції ділить трапецію на рівні за площею фігури. \\
III. Середня лінія трапеції паралельна її основам.}

\vspace{0.3cm}
\answerTable{лише II та III}{лише III}{лише I та II}{лише I та III}{лише I}

\vspace{0.7cm}

% === ЗАВДАННЯ 26 ===
\noindent\makebox[1.5em][l]{\textbf{26.}}\parbox[t]{\dimexpr\textwidth-1.5em}{Які з наведених тверджень є правильними?\nmtyear{2024}

\vspace{0.1cm}
I. \hspace{0.2em} Півсума довжин бічних сторін будь-якої трапеції дорівнює її середній лінії. \\
II. \hspace{0.1em} Діагональ будь-якої трапеції ділить її на 2 рівні трикутники. \\
III. Середня лінія будь-якої трапеції ділить її висоту навпіл.}

\vspace{0.3cm}
\answerTable{лише II та III}{лише III}{I, II та III}{лише I та II}{лише I та III}

\vspace{0.7cm}

% === ЗАВДАННЯ 27 ===
\noindent\makebox[1.5em][l]{\textbf{27.}}\parbox[t]{\dimexpr\textwidth-1.5em}{Які з наведених тверджень є правильними?\nmtyear{2024}

\vspace{0.1cm}
I. \hspace{0.2em} У будь-яку рівнобічну трапецію можна вписати коло. \\
II. \hspace{0.1em} Довжина радіуса вписаного в ромб кола дорівнює половині його висоти. \\
III. Навколо будь-якої рівнобічної трапеції можна описати коло.}

\vspace{0.3cm}
\answerTable{I, II та III}{лише II та III}{лише II}{лише III}{лише I та II}

\vspace{0.7cm}

% === ЗАВДАННЯ 28 ===
\noindent\makebox[1.5em][l]{\textbf{28.}}\parbox[t]{\dimexpr\textwidth-1.5em}{Які з наведених тверджень є правильними?\nmtyear{2024}

\vspace{0.1cm}
I. \hspace{0.2em} Діагональ рівнобічної трапеції ділить її на 2 трикутники, серед яких обов'язково є один тупокутний. \\
II. \hspace{0.1em} Діагоналі будь-якої трапеції ділить її на 4 трикутники, серед яких обов'язково є два подібних. \\
III. Діагоналі рівнобічної трапеції точкою перетину діляться навпіл.}

\vspace{0.3cm}
\answerTable{лише III}{лише I та III}{лише I та II}{лише II}{лише I}

\vspace{0.7cm}

% === ЗАВДАННЯ 29 ===
\noindent\makebox[1.5em][l]{\textbf{29.}}\parbox[t]{\dimexpr\textwidth-1.5em}{Які з наведених тверджень є правильними?\nmtyear{2024}

\vspace{0.1cm}
I. \hspace{0.2em} Діагоналі будь-якого прямокутника ділять його кути навпіл. \\
II. \hspace{0.1em} Діагоналі будь-якої рівнобічної трапеції ділять її кути навпіл. \\
III. Діагоналі будь-якого прямокутника рівні.}

\vspace{0.3cm}
\answerTable{лише I та II}{лише I та III}{лише III}{I, II та III}{лише II та III}

\vspace{0.7cm}

% === ЗАВДАННЯ 30 ===
\noindent\textbf{30.} \begin{minipage}[t]{0.55\textwidth}
Навколо кола радіуса $4$ \textit{см} описано рівнобічну трапецію, середня лінія якої дорівнює $10$ \textit{см} (див. рисунок). До кожного відрізка (1--3) доберіть його довжину (А--Д).\nmtyear{2024}
\end{minipage}
\hfill
\begin{minipage}[t]{0.4\textwidth}
    \vspace{-0.5cm}
    \begin{flushright}
    \begin{tikzpicture}[scale=0.18]
        % R=4, h=8. Midline=10 => a+b=20.
        % Inscribed => a+b = 2c => 2c=20 => c=10.
        % Leg c=10. Height h=8. Projection x=6.
        % b = a - 12. 2a - 12 = 20 => 2a=32 => a=16. b=4.
        
        \coordinate (O) at (0,4);
        \coordinate (A) at (-8,0);
        \coordinate (D) at (8,0);
        \coordinate (B) at (-2,8);
        \coordinate (C) at (2,8);
        
        \draw[thick] (A) -- (B) -- (C) -- (D) -- cycle;
        \draw[thick] (O) circle (4cm);
        
        % Візуально:
        % Нижня основа 16 (від -8 до 8).
        % Верхня основа 4 (від -2 до 2).
        % Висота 8.
        
    \end{tikzpicture}
    \end{flushright}
\end{minipage}

\vspace{0.3cm}

\matchingLayout{
    \textit{Відрізок} \par \vspace{0.2cm}
    \textbf{1} \quad Висота трапеції \\
    \textbf{2} \quad Бічна сторона трапеції \\
    \textbf{3} \quad Більша основа трапеції
}{
    \textit{Довжина відрізка} \par \vspace{0.2cm}
    \begin{tabular}{ll}
    \textbf{А} & 8 \textit{см} \\
    \textbf{Б} & 10 \textit{см} \\
    \textbf{В} & 12 \textit{см} \\
    \textbf{Г} & 16 \textit{см} \\
    \textbf{Д} & 20 \textit{см} \\
    \end{tabular}
}{
    \answerGrid
}


\noindent\textbf{31.} \begin{minipage}[t]{0.55\textwidth}
У паралелограмі $ABCD$ з точки $B$ на сторону $AD$ опущено висоту $BK = 6$ \textit{см}, $AK = 8$ \textit{см}, $KD = 4$ \textit{см}. До кожного відрізка (1--3) доберіть його довжину (А--Д). \nmtyear{2024}
\end{minipage}
\hfill
\begin{minipage}[t]{0.4\textwidth}
    \vspace{-0.5cm}
    \begin{flushright}
    \begin{tikzpicture}[scale=0.35]
        \coordinate (A) at (0,0);
        \coordinate (K) at (8,0); % AK = 8
        \coordinate (D) at (12,0); % KD = 4, AD = 12
        \coordinate (B) at (8,6); % BK = 6
        \coordinate (C) at (20,6); % BC = AD = 12. 8+12 = 20
        
        \draw[thick] (A) -- (B) -- (C) -- (D) -- cycle;
        \draw[thick] (B) -- (K);
        
        \draw (K) ++(-0.6,0) -- ++(0,0.6) -- ++(0.6,0);
        
        \node[below left] at (A) {$A$};
        \node[above left] at (B) {$B$};
        \node[above right] at (C) {$C$};
        \node[below right] at (D) {$D$};
        \node[below] at (K) {$K$};
    \end{tikzpicture}
    \end{flushright}
\end{minipage}

\vspace{0.3cm}

\matchingLayout{
\textit{Відрізок} \par \vspace{0.2cm}
    \textbf{1} \quad Середня лінія трапеції $KBCD$ \\
    \textbf{2} \quad $AB$ \\
    \textbf{3} \quad Відстань від точки $B$ до сторони $CD$
}{
\textit{Довжина} \par \vspace{0.2cm}
    \begin{tabular}{ll}
    \textbf{А} & 10 \textit{см} \\
    \textbf{Б} & 6 \textit{см} \\
    \textbf{В} & 8 \textit{см} \\
    \textbf{Г} & 7,2 \textit{см} \\
    \textbf{Д} & 16 \textit{см} \\
    \end{tabular}
}{
    \answerGrid
}

\vspace{0.7cm}


\begin{center}
{\Large\textbf{\color{headerblue}БАЗА ЗАВДАНЬ НМТ 2025}}
\end{center}

% === ЗАВДАННЯ 31 ===
\noindent\textbf{31.} \begin{minipage}[t]{0.55\textwidth}
Діагональ $AC$ рівнобічної трапеції $ABCD$ утворює зі стороною $CD$ кут $120^\circ$ (див. рисунок). Довжини основ трапеції дорівнюють $12$ \textit{см} і $30$ \textit{см}. До кожного початку речення (1--3) доберіть його закінчення (А--Д) так, щоб утворилося правильне твердження. \nmtyear{2025}
\end{minipage}
\hfill
\begin{minipage}[t]{0.4\textwidth}
    \vspace{-0.5cm}
    \begin{flushright}
    \begin{tikzpicture}[scale=0.15]
        \coordinate (A) at (-15,0);
        \coordinate (D) at (15,0);
        % Bases 30 and 12. Top base centered.
        % Coordinates for C approx to look like the image.
        \coordinate (B) at (-6, 12);
        \coordinate (C) at (6, 12);
        
        \draw[thick] (A) -- (B) -- (C) -- (D) -- cycle;
        \draw[thick] (A) -- (C);
        
        % Obtuse angle 120 at C (ACD)
        \pic [draw, pic text={\scriptsize $120^\circ$}, angle radius=0.3cm, angle eccentricity=1.5] {angle = A--C--D};
        
        \node[below left] at (A) {$A$};
        \node[above left] at (B) {$B$};
        \node[above] at (C) {$C$};
        \node[below right] at (D) {$D$};
    \end{tikzpicture}
    \end{flushright}
\end{minipage}

\vspace{0.3cm}

\matchingLayout{
    \textit{Початок речення} \par \vspace{0.2cm}
    \textbf{1} \quad Довжина середньої лінії трапеції \par \quad дорівнює \\
    \textbf{2} \quad Довжина проєкції сторони $AB$ \par \quad на пряму $AD$ дорівнює \\
    \textbf{3} \quad Довжина радіуса кола, описаного \par \quad навколо трапеції, дорівнює
}{
    \textit{Закінчення речення} \par \vspace{0.2cm}
    \begin{tabular}{ll}
    \textbf{А} & 9 \textit{см}. \\
    \textbf{Б} & $20\sqrt{3}$ \textit{см}. \\
    \textbf{В} & 21 \textit{см}. \\
    \textbf{Г} & $10\sqrt{3}$ \textit{см}. \\
    \textbf{Д} & 18 \textit{см}. \\
    \end{tabular}
}{
    \answerGrid
}

\vspace{0.7cm}

% === ЗАВДАННЯ 32 ===
\noindent\makebox[1.5em][l]{\textbf{32.}}\parbox[t]{\dimexpr\textwidth-1.5em}{Які з наведених тверджень є правильними?

\vspace{0.1cm}
I. \hspace{0.2em} Існує прямокутна трапеція, навколо якої можна описати коло. \\
II. \hspace{0.1em} Існує прямокутна трапеція, у яку можна вписати коло. \\
III. Існує прямокутна трапеція, висота якої вдвічі менша за більшу бічну сторону. \nmtyear{2025}}

\vspace{0.3cm}
\answerTable{лише I}{лише II і III}{I, II та III}{лише II}{лише III}

\vspace{0.7cm}

% === ЗАВДАННЯ 33 ===
\noindent\textbf{33.} \begin{minipage}[t]{0.55\textwidth}
У рівнобічній трапеції $ABCD$ ($AB=CD$) на більшій основі $AD$ вибрано точку $O$ так, що $AO=OD$, $\angle AOB = 45^\circ$. $AD : BC = 5 : 2$, $AD=b$. Знайдіть площу цієї трапеції. \nmtyear{2025}
\end{minipage}
\hfill
\begin{minipage}[t]{0.4\textwidth}
    \vspace{-0.5cm}
    \begin{flushright}
    \begin{tikzpicture}[scale=0.5]
        \coordinate (A) at (-4,0);
        \coordinate (D) at (4,0);
        \coordinate (O) at (0,0);
        
        % AD:BC = 5:2. If AD=8, BC=3.2.
        % B is at x = -1.6, C is at x = 1.6.
        % Angle AOB = 45. Vector OA = (-4, 0). Vector OB = (-1.6, h).
        % Angle between them is 45. O is origin.
        % Slope of OB needs to be tan(135) = -1? No, A is (-4,0). B is (-1.6, h).
        % Angle AOB is acute (45). So B is in second quadrant relative to O.
        % Line OB has angle 135 or 45?
        % The drawing shows angle between base AD and segment OB is 45.
        % Since A is to the left, and B is up-left, angle AOB is typically 180 - angle(positive x).
        % Drawing shows acute angle marked near O.
        % Let's assume angle(DO, OB) = 135, angle(AO, OB) = 45.
        % Then height h = 1.6 (approx).
        
        \coordinate (B) at (-1.6, 3.6);
        \coordinate (C) at (1.6, 3.6);
        
        \draw[thick] (A) -- (B) -- (C) -- (D) -- cycle;
        \draw[thick] (A) -- (D); % base
        \draw[thick] (B) -- (O);
        
        % Marks for AO = OD
        \draw[thick] (-2, -0.15) -- (-2, 0.15);
        \draw[thick] (-2.2, -0.15) -- (-2.2, 0.15);
        
        \draw[thick] (2, -0.15) -- (2, 0.15);
        \draw[thick] (2.2, -0.15) -- (2.2, 0.15);
        
        % Marks for legs
        \draw[thick] ($(A)!0.5!(B)$) ++(150:0.15) -- ++(-30:0.3);
        \draw[thick] ($(C)!0.5!(D)$) ++(60:0.15) -- ++(-120:0.3);
        
        % Angle 45
        \pic [draw, pic text={\scriptsize $45^\circ$}, angle radius=0.6cm, angle eccentricity=1.5] {angle = B--O--A};
        
        \node[below left] at (A) {$A$};
        \node[above left] at (B) {$B$};
        \node[above right] at (C) {$C$};
        \node[below right] at (D) {$D$};
        \node[below] at (O) {$O$};
        
        \fill (O) circle (2pt);
    \end{tikzpicture}
    \end{flushright}
\end{minipage}

\vspace{0.3cm}
\answerTableTall{$\dfrac{21b^2}{100}$}{$\dfrac{7b^2}{50}$}{$\dfrac{7b^2\sqrt{2}}{50}$}{$\dfrac{7b^2}{45}$}{$\dfrac{7b^2}{25}$}

\vspace{0.7cm}

% === ЗАВДАННЯ 34 ===
\noindent\textbf{34.} \begin{minipage}[t]{0.55\textwidth}
На рисунку зображено рівнобічну трапецію $ABCD$ ($AB=CD$), точки $K$ і $M$ --- середини бічних сторін трапеції, діагональ $AC$ трапеції перетинає відрізок $KM$ у точці $O$ так, що $KO = 5$ \textit{см}, $OM = 7$ \textit{см}, $\angle CAD = 30^\circ$. Знайдіть площу цієї трапеції. \nmtyear{2025}
\end{minipage}
\hfill
\begin{minipage}[t]{0.4\textwidth}
    \vspace{-0.5cm}
    \begin{flushright}
    \begin{tikzpicture}[scale=0.25]
        % KM - midline. KO = BC/2 = 5 -> BC=10.
        % OM = AD/2 = 7 -> AD=14.
        % Isosceles. Angle CAD = 30.
        % Height h.
        % Draw diagonals intersecting midline.
        
        \coordinate (A) at (-7,0);
        \coordinate (D) at (7,0);
        \coordinate (B) at (-3, 7); % Visual
        \coordinate (C) at (3, 7);
        
        \coordinate (K) at ($(A)!0.5!(B)$);
        \coordinate (M) at ($(C)!0.5!(D)$);
        
        \draw[thick] (A) -- (B) -- (C) -- (D) -- cycle;
        \draw[thick] (A) -- (C);
        \draw[thick] (K) -- (M);
        
        \coordinate (O) at (intersection of A--C and K--M);
        
        \node[below left] at (A) {$A$};
        \node[above left] at (B) {$B$};
        \node[above right] at (C) {$C$};
        \node[below right] at (D) {$D$};
        \node[left] at (K) {$K$};
        \node[right] at (M) {$M$};
        \node[below] at (O) {$O$};
        
        \fill (K) circle (8pt);
        \fill (M) circle (8pt);
        \fill (O) circle (8pt);
        
        % Angle 30
        \pic [draw, pic text={\scriptsize $30^\circ$}, angle radius=0.5cm, angle eccentricity=1.7] {angle = D--A--C};
        
    \end{tikzpicture}
    \end{flushright}
\end{minipage}

\vspace{0.3cm}
\answerTable{$72$ \textit{см}$^2$}{$144\sqrt{3}$ \textit{см}$^2$}{$288$ \textit{см}$^2$}{$48\sqrt{3}$ \textit{см}$^2$}{$144$ \textit{см}$^2$}

\vspace{0.7cm}

% === ЗАВДАННЯ 35 ===
\noindent\textbf{35.} \begin{minipage}[t]{0.55\textwidth}
У прямокутній трапеції $ABCD$ діагоналі перетинаються в точці $O$ (див. рисунок). Висоти трикутників $BOC$ і $AOD$, проведені з вершини $O$, відносяться як $2 : 5$. $BC = 14$ \textit{см}. Визначте довжину середньої лінії трапеції $ABCD$. \nmtyear{2025}
\end{minipage}
\hfill
\begin{minipage}[t]{0.4\textwidth}
    \vspace{-0.5cm}
    \begin{flushright}
    \begin{tikzpicture}[scale=0.18]
        \coordinate (A) at (0,0);
        \coordinate (B) at (0,10);
        \coordinate (C) at (7,10); % Scaled BC
        \coordinate (D) at (17.5,0); % Scaled AD (ratio 2:5 means AD = 2.5 * BC)
        
        \draw[thick] (A) -- (B) -- (C) -- (D) -- cycle;
        \draw[thick] (A) -- (C);
        \draw[thick] (B) -- (D);
        
        \coordinate (O) at (intersection of A--C and B--D);
        
        \node[below left] at (A) {$A$};
        \node[above left] at (B) {$B$};
        \node[above] at (C) {$C$};
        \node[below right] at (D) {$D$};
        \node[left] at (O) {$O$}; % Visual position
        
    \end{tikzpicture}
    \end{flushright}
\end{minipage}

\vspace{0.3cm}
\answerTable{$49$ \textit{см}}{$24{,}5$ \textit{см}}{$42$ \textit{см}}{$21$ \textit{см}}{$9{,}8$ \textit{см}}

\vspace{0.7cm}

% === ЗАВДАННЯ 36 ===
\noindent\makebox[1.5em][l]{\textbf{36.}}\parbox[t]{\dimexpr\textwidth-1.5em}{Які з наведених тверджень є правильними?

\vspace{0.1cm}
I. \hspace{0.2em} Існує трапеція, у якій сума довжин основ дорівнює довжині її висоти. \\
II. \hspace{0.1em} Існує прямокутна трапеція, у якій суми градусних мір протилежних кутів рівні. \\
III. Існує прямокутна трапеція, у якій сума довжин бічних сторін дорівнює сумі довжин її основ. \nmtyear{2025}}

\vspace{0.3cm}
\answerTable{лише II}{лише II та III}{лише I та III}{лише I та II}{I, II та III}

\vspace{0.7cm}

% === ЗАВДАННЯ 37 ===
\noindent\makebox[1.5em][l]{\textbf{37.}}\parbox[t]{\dimexpr\textwidth-1.5em}{Які з наведених тверджень є правильними?

\vspace{0.1cm}
I. \hspace{0.2em} Центр кола, уписаного в трапецію, лежить на середній лінії трапеції. \\
II. \hspace{0.1em} Центр кола, уписаного в трапецію, збігається з точкою перетину діагоналей трапеції. \\
III. Центр кола, описаного навколо трапеції, обов'язково знаходиться на її більшій основі. \nmtyear{2025}}

\vspace{0.3cm}
\answerTable{лише III}{лише II}{лише I та II}{лише I та III}{лише I}

\noindent\textbf{38.} \begin{minipage}[t]{0.55\textwidth}
У паралелограмі $ABCD$ з точки $B$ на сторону $AD$ опущено висоту $BK = 6$ \textit{см}, $AK = 8$ \textit{см}, $KD = 4$ \textit{см}. До кожного відрізка (1--3) доберіть його довжину (А--Д). \nmtyear{2024}
\end{minipage}
\hfill
\begin{minipage}[t]{0.4\textwidth}
    \vspace{-0.5cm}
    \begin{flushright}
    \begin{tikzpicture}[scale=0.25]
        \coordinate (A) at (0,0);
        \coordinate (K) at (8,0); % AK = 8
        \coordinate (D) at (12,0); % KD = 4, AD = 12
        \coordinate (B) at (8,6); % BK = 6
        \coordinate (C) at (20,6); % BC = AD = 12. 8+12 = 20
        
        \draw[thick] (A) -- (B) -- (C) -- (D) -- cycle;
        \draw[thick] (B) -- (K);
        
        \draw (K) ++(-0.6,0) -- ++(0,0.6) -- ++(0.6,0);
        
        \node[below left] at (A) {$A$};
        \node[above left] at (B) {$B$};
        \node[above right] at (C) {$C$};
        \node[below right] at (D) {$D$};
        \node[below] at (K) {$K$};
    \end{tikzpicture}
    \end{flushright}
\end{minipage}

\vspace{0.3cm}

\matchingLayout{
\textit{Відрізок} \par \vspace{0.2cm}
    \textbf{1} \quad Середня лінія трапеції $KBCD$ \\
    \textbf{2} \quad $AB$ \\
    \textbf{3} \quad Відстань від точки $B$ до сторони $CD$
}{
\textit{Довжина} \par \vspace{0.2cm}
    \begin{tabular}{ll}
    \textbf{А} & 10 \textit{см} \\
    \textbf{Б} & 6 \textit{см} \\
    \textbf{В} & 8 \textit{см} \\
    \textbf{Г} & 7,2 \textit{см} \\
    \textbf{Д} & 16 \textit{см} \\
    \end{tabular}
}{
    \answerGrid
}

\vspace{0.7cm}


% === ЗАВДАННЯ 11 ===
\noindent\textbf{39.} \begin{minipage}[t]{0.55\textwidth}
У прямокутній трапеції $ABCD$ проведено висоту $CK$. $ABCK$ --- квадрат з діагоналлю $12\sqrt{2}$ \textit{см}. $CD = 13$ \textit{см}. Узгодьте відрізок (1--3) із його довжиною (А--Д). \nmtyear{2025}
\end{minipage}
\hfill
\begin{minipage}[t]{0.4\textwidth}
    \vspace{-0.5cm}
    \begin{flushright}
    \begin{tikzpicture}[scale=0.25]
        % Квадрат діагональ 12sqrt(2) -> сторона 12.
        % Висота CK=12. CD=13. KD=5. AD=17.
        
        \coordinate (A) at (0,0);
        \coordinate (B) at (0,12);
        \coordinate (C) at (12,12);
        \coordinate (K) at (12,0);
        \coordinate (D) at (17,0);
        
        \draw[thick] (A) -- (B) -- (C) -- (D) -- cycle;
        \draw[thick] (C) -- (K);
        
        % Прямий кут
        \draw (12,1) -- (13,1) -- (13,0);
        
        \node[below left] at (A) {$A$};
        \node[above left] at (B) {$B$};
        \node[above] at (C) {$C$};
        \node[below] at (K) {$K$};
        \node[below right] at (D) {$D$};
    \end{tikzpicture}
    \end{flushright}
\end{minipage}

\vspace{0.3cm}

\matchingLayout{
    \textit{Відрізок} \par \vspace{0.2cm}
    \textbf{1} \quad висота трапеції $ABCD$ \\
    \textbf{2} \quad $AD$ \\
    \textbf{3} \quad середня лінія трапеції $ABCD$
}{
    \textit{Довжина відрізка} \par \vspace{0.2cm}
    \begin{tabular}{ll}
    \textbf{А} & 12 \textit{см} \\
    \textbf{Б} & 14,5 \textit{см} \\
    \textbf{В} & 17 \textit{см} \\
    \textbf{Г} & 18 \textit{см} \\
    \textbf{Д} & 29 \textit{см} \\
    \end{tabular}
}{
    \answerGrid
}

\vspace{0.7cm}

% === ЗАВДАННЯ 40 ===
\noindent\makebox[1.5em][l]{\textbf{40.}}\parbox[t]{\dimexpr\textwidth-1.5em}{Які з наведених тверджень є правильними?

\vspace{0.1cm}
I. \hspace{0.2em} Існує трапеція, у якій сума довжин основ дорівнює довжині її висоти. \\
II. \hspace{0.1em} Існує прямокутна трапеція, у якій суми градусних мір протилежних кутів рівні. \\
III. Існує прямокутна трапеція, у якій сума довжин бічних сторін дорівнює сумі довжин її основ. \nmtyear{2025}}

\vspace{0.3cm}
\answerTable{лише I та II}{лише II}{I, II та III}{лише II та III}{лише I та III}

\vspace{0.7cm}

% === ЗАВДАННЯ 41 ===
\noindent\makebox[1.5em][l]{\textbf{41.}}\parbox[t]{\dimexpr\textwidth-1.5em}{Які з наведених тверджень є правильними?

\vspace{0.1cm}
I. \hspace{0.2em} Центр кола, уписаного в трапецію, лежить на середній лінії трапеції. \\
II. \hspace{0.1em} Центр кола, уписаного в трапецію, збігається з точкою перетину діагоналей трапеції. \\
III. Центр кола, описаного навколо трапеції, обов'язково знаходиться на її більшій основі. \nmtyear{2025}}

\vspace{0.3cm}
\answerTable{лише I та III}{лише I}{лише III}{лише II}{лише I та II}

\vspace{0.7cm}

% === ЗАВДАННЯ 42 ===
\noindent\textbf{42.} \begin{minipage}[t]{0.55\textwidth}
У рівнобічній трапеції $ABCD$ ($AB=CD$) на більшій основі $AD$ вибрано точку $O$ так, що $AO=OD$, $\angle AOB = 45^\circ$, $BO=a$, $BC$ утричі менше від $AD$. Знайдіть площу цієї трапеції. \nmtyear{2025}
\end{minipage}
\hfill
\begin{minipage}[t]{0.4\textwidth}
    \vspace{-0.5cm}
    \begin{flushright}
    \begin{tikzpicture}[scale=0.6]
        % Координати для візуалізації
        % AD = 3 * BC.
        % Нехай BC має ширину 2, тоді AD має ширину 6.
        % O - середина AD (0,0).
        % A = (-3, 0), D = (3, 0).
        % B = (-1, h), C = (1, h).
        % Кут AOB = 45 градусів.
        % Це означає, що вектор OB утворює 45 градусів з віссю OA (від'ємна вісь X).
        % Тобто кут нахилу прямої OB = 135 градусів (або 45 відносно осі).
        % Але на малюнку кут гострий позначений.
        % Зробимо візуально схоже на скріншот.
        
        \coordinate (O) at (0,0);
        \coordinate (A) at (-3.5,0);
        \coordinate (D) at (3.5,0);
        
        % Точки B і C вужчі
        \coordinate (B) at (-1.2, 4.5); 
        \coordinate (C) at (1.2, 4.5);
        
        \draw[thick] (A) -- (B) -- (C) -- (D) -- cycle;
        \draw[thick] (B) -- (O);
        
        % Позначки рівності сторін AB = CD
        \draw[thick] ($(A)!0.5!(B)$) ++(150:0.15) -- ++(-30:0.3);
        \draw[thick] ($(C)!0.5!(D)$) ++(60:0.15) -- ++(-120:0.3);
        
        % Позначки рівності AO = OD
        \draw[thick] (-1.75, -0.15) -- (-1.75, 0.15);
        \draw[thick] (-1.85, -0.15) -- (-1.85, 0.15); % подвійна
        
        \draw[thick] (1.75, -0.15) -- (1.75, 0.15);
        \draw[thick] (1.85, -0.15) -- (1.85, 0.15); % подвійна
        
        % Кут 45
        \pic [draw, pic text={\scriptsize $45^\circ$}, angle radius=0.5cm, angle eccentricity=1.7] {angle = B--O--A};
        
        \node[below left] at (A) {$A$};
        \node[above left] at (B) {$B$};
        \node[above right] at (C) {$C$};
        \node[below right] at (D) {$D$};
        \node[below] at (O) {$O$};
        
        % Підпис "a" на BO
        \node[above right] at ($(B)!0.4!(O)$) {$a$};
        
        \fill (O) circle (2pt);
    \end{tikzpicture}
    \end{flushright}
\end{minipage}

\vspace{0.3cm}
\answerTable{$4a^2$}{$a^2\sqrt{3}$}{$a^2\sqrt{2}$}{$2a^2$}{$3a^2$}

\vspace{0.7cm}

% === ЗАВДАННЯ 43 ===
\noindent\makebox[1.5em][l]{\textbf{43.}}\parbox[t]{\dimexpr\textwidth-1.5em}{Периметр рівнобічної трапеції дорівнює $60$ \textit{см}, а бічна сторона --- $10$ \textit{см}. Визначте середню лінію трапеції. \nmtyear{2025}}

\vspace{0.3cm}
\answerTable{$20$ \textit{см}}{$40$ \textit{см}}{$50$ \textit{см}}{$80$ \textit{см}}{$30$ \textit{см}}

% === ЗАВДАННЯ 39 ===
\noindent\textbf{44.} \begin{minipage}[t]{0.55\textwidth}
На рисунку зображено паралелограм $ABCD$. Точка $K$ є серединою сторони $BC$, $KP$ --- висота паралелограма, $AP = 20$ \textit{см}, $PD = 10$ \textit{см}, $PK = 12$ \textit{см}. Узгодьте початок речення (1--3) та його закінчення (А--Д) так, щоб утворилося правильне твердження. \nmtyear{2025}
\end{minipage}
\hfill
\begin{minipage}[t]{0.4\textwidth}
    \vspace{-0.5cm}
    \begin{flushright}
    \begin{tikzpicture}[scale=0.12]
        % P=(0,0). K=(0,12).
        % AP=20 => A=(-20,0).
        % PD=10 => D=(10,0).
        % BC паралельна AD, проходить через K. Довжина BC = AD = 30.
        % K - середина BC. Значить B лівіше K на 15, C правіше на 15.
        % B=(-15, 12). C=(15, 12).
        
        \coordinate (P) at (0,0);
        \coordinate (K) at (0,12);
        \coordinate (A) at (-20,0);
        \coordinate (D) at (10,0);
        \coordinate (B) at (-15,12);
        \coordinate (C) at (15,12);
        
        \draw[thick] (A) -- (B) -- (C) -- (D) -- cycle;
        \draw[thick] (K) -- (P);
        
        % Прямий кут
        \draw (P) ++(1.5,0) -- ++(0,1.5) -- ++(-1.5,0);
        
        % Середина K
        \draw ($(B)!0.5!(K)$) ++(0,-0.8) -- ++(0,1.6);
        \draw ($(K)!0.5!(C)$) ++(0,-0.8) -- ++(0,1.6);
        
        \node[below left] at (A) {$A$};
        \node[above left] at (B) {$B$};
        \node[above right] at (C) {$C$};
        \node[below right] at (D) {$D$};
        \node[above] at (K) {$K$};
        \node[below] at (P) {$P$};
    \end{tikzpicture}
    \end{flushright}
\end{minipage}

\vspace{0.3cm}

\noindent
\begin{minipage}[t]{0.55\textwidth}
    \textit{Початок речення} \par \vspace{0.2cm}
    \textbf{1} \quad Довжина відрізка $KC$ дорівнює \\
    \textbf{2} \quad Довжина середньої лінії трапеції $PKCD$ дорівнює \\
    \textbf{3} \quad Довжина сторони $AB$ дорівнює
\end{minipage}%
\hfill
\begin{minipage}[t]{0.40\textwidth}
    \textit{Закінчення речення} \par \vspace{0.2cm}
    \begin{tabular}{ll}
    \textbf{А} & 10 \textit{см}. \\
    \textbf{Б} & 12 \textit{см}. \\
    \textbf{В} & 12,5 \textit{см}. \\
    \textbf{Г} & 13 \textit{см}. \\
    \textbf{Д} & 15 \textit{см}. \\
    \end{tabular}
    \vspace{0.3cm}
    \begin{flushright}
    \answerGrid
    \end{flushright}
\end{minipage}



\end{document}