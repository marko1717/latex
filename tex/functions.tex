\documentclass[14pt]{extarticle}
\usepackage{fontspec}
\usepackage{polyglossia}
\setdefaultlanguage{ukrainian}

\defaultfontfeatures{Ligatures=TeX}
\setmainfont{Liberation Serif}
\setsansfont{Liberation Sans}
\setmonofont{Liberation Mono}

\usepackage[a4paper,margin=1.5cm,bottom=2cm,top=2cm]{geometry}
\usepackage{amsmath,amssymb}
\usepackage{enumitem}
\usepackage{tikz}
\usepackage{pgfplots}
\pgfplotsset{compat=1.18}

\usetikzlibrary{calc,patterns,angles,quotes,intersections,babel}
\usetikzlibrary{3d}

\usepackage{xcolor}
\usepackage{array}
\usepackage{fancyhdr}
\usepackage{multirow}

% Кольори
\definecolor{headerblue}{RGB}{0, 102, 204}
\definecolor{yearcolor}{RGB}{128, 0, 128}

\pagestyle{fancy}
\fancyhf{}
\renewcommand{\headrulewidth}{0pt}
\fancyfoot[C]{\thepage}

\setlength{\headheight}{15pt}
\setlength{\headsep}{10pt}
\setlength{\footskip}{25pt}

\widowpenalty=10000
\clubpenalty=10000

% === КОМАНДИ ===

% Таблиця відповідей для відповідностей
\newcommand{\answerGrid}{
    \begingroup
    \renewcommand{\arraystretch}{1.3} 
    \setlength{\tabcolsep}{7pt} 
    \begin{tabular}{r|c|c|c|c|c|}
         \multicolumn{1}{c}{} & \multicolumn{1}{c}{\textbf{А}} & \multicolumn{1}{c}{\textbf{Б}} & \multicolumn{1}{c}{\textbf{В}} & \multicolumn{1}{c}{\textbf{Г}} & \multicolumn{1}{c}{\textbf{Д}} \\ \cline{2-6}
         \textbf{1} & & & & & \\ \cline{2-6}
         \textbf{2} & & & & & \\ \cline{2-6}
         \textbf{3} & & & & & \\ \cline{2-6}
    \end{tabular}
    \endgroup
}

% Макет для завдань на відповідність
\newcommand{\matchingLayout}[3]{
    \noindent
    \begin{minipage}[t]{0.40\textwidth}
        #1
    \end{minipage}%
    \hfill
    \begin{minipage}[t]{0.28\textwidth}
        #2
    \end{minipage}%
    \hfill
    \begin{minipage}[t]{0.30\textwidth}
        \vspace{0pt}
        \begin{flushright}
        #3
        \end{flushright}
    \end{minipage}
}

% Стандартна таблиця відповідей
\newcommand{\answerTable}[5]{
\begin{center}
\begin{tabular}{|*{5}{>{\centering\arraybackslash}m{2.8cm}|}}
\hline
\rule[-0.3cm]{0pt}{0.8cm}\textbf{А} & \textbf{Б} & \textbf{В} & \textbf{Г} & \textbf{Д} \\
\hline
\rule[-0.4cm]{0pt}{1.0cm}#1 & \rule[-0.4cm]{0pt}{1.0cm}#2 & \rule[-0.4cm]{0pt}{1.0cm}#3 & \rule[-0.4cm]{0pt}{1.0cm}#4 & \rule[-0.4cm]{0pt}{1.0cm}#5 \\
\hline
\end{tabular}
\end{center}
}

% Таблиця для відповідей із дробами
\newcommand{\answerTableTall}[5]{
\begin{center}
\begin{tabular}{|*{5}{>{\centering\arraybackslash}m{2.8cm}|}}
\hline
\rule[-0.3cm]{0pt}{0.8cm}\textbf{А} & \textbf{Б} & \textbf{В} & \textbf{Г} & \textbf{Д} \\
\hline
\rule[-0.9cm]{0pt}{2.0cm}#1 & 
\rule[-0.9cm]{0pt}{2.0cm}#2 & 
\rule[-0.9cm]{0pt}{2.0cm}#3 & 
\rule[-0.9cm]{0pt}{2.0cm}#4 & 
\rule[-0.9cm]{0pt}{2.0cm}#5 \\
\hline
\end{tabular}
\end{center}
}

\newcommand{\nmtyear}[1]{\hfill{\small\color{yearcolor}(AI Gen)}}

\begin{document}

\vspace{1cm}

\begin{center}
{\Large\textbf{\color{headerblue}ЗГЕНЕРОВАНІ ЗАВДАННЯ (AI)}}
\end{center}

\begin{center}
{\large Тема: \textbf{Функції та їх властивості}}
\end{center}

\vspace{0.5cm}
% === Functions: Graph Shifts ===
\noindent\makebox[1.5em][l]{\textbf{1.}}\parbox[t]{\dimexpr\textwidth-1.5em}{Графік функції $y = f(x)$ паралельно перенесли вздовж осі $Ox$ на 4 одиниць ліворуч. Укажіть формулу для отриманої функції $y = g(x)$. \nmtyear{2026}}
\vspace{0.3cm}

\answerTable{$y = f(x + 4)$}{$y = f(x) + 4$}{$y = f(x - 4)$}{$y = 4f(x)$}{$y = f(x) - 4$}

\vspace{0.5cm}

\noindent\makebox[1.5em][l]{\textbf{2.}}\parbox[t]{\dimexpr\textwidth-1.5em}{Графік функції $y = f(x)$ паралельно перенесли вздовж осі $Oy$ на 4 одиниць вгору. Укажіть формулу для отриманої функції $y = g(x)$. \nmtyear{2026}}
\vspace{0.3cm}

\answerTable{$y = 4f(x)$}{$y = f(x) - 4$}{$y = f(x - 4)$}{$y = f(x) + 4$}{$y = f(x + 4)$}

\vspace{0.5cm}

\noindent\makebox[1.5em][l]{\textbf{3.}}\parbox[t]{\dimexpr\textwidth-1.5em}{Графік функції $y = f(x)$ паралельно перенесли вздовж осі $Oy$ на 1 одиниць вниз. Укажіть формулу для отриманої функції $y = g(x)$. \nmtyear{2026}}
\vspace{0.3cm}

\answerTable{$y = f(x) + 1$}{$y = f(x + 1)$}{$y = f(x - 1)$}{$y = 1f(x)$}{$y = f(x) - 1$}

\vspace{0.5cm}

\noindent\makebox[1.5em][l]{\textbf{4.}}\parbox[t]{\dimexpr\textwidth-1.5em}{Графік функції $y = f(x)$ паралельно перенесли вздовж осі $Oy$ на 1 одиниць вгору. Укажіть формулу для отриманої функції $y = g(x)$. \nmtyear{2026}}
\vspace{0.3cm}

\answerTable{$y = f(x - 1)$}{$y = f(x + 1)$}{$y = f(x) + 1$}{$y = f(x) - 1$}{$y = 1f(x)$}

\vspace{0.5cm}

\noindent\makebox[1.5em][l]{\textbf{5.}}\parbox[t]{\dimexpr\textwidth-1.5em}{Графік функції $y = f(x)$ паралельно перенесли вздовж осі $Oy$ на 3 одиниць вниз. Укажіть формулу для отриманої функції $y = g(x)$. \nmtyear{2026}}
\vspace{0.3cm}

\answerTable{$y = f(x) - 3$}{$y = 3f(x)$}{$y = f(x + 3)$}{$y = f(x) + 3$}{$y = f(x - 3)$}

\vspace{0.5cm}

\noindent\makebox[1.5em][l]{\textbf{6.}}\parbox[t]{\dimexpr\textwidth-1.5em}{Графік функції $y = f(x)$ паралельно перенесли вздовж осі $Ox$ на 3 одиниць праворуч. Укажіть формулу для отриманої функції $y = g(x)$. \nmtyear{2026}}
\vspace{0.3cm}

\answerTable{$y = f(x + 3)$}{$y = f(x - 3)$}{$y = f(x) - 3$}{$y = 3f(x)$}{$y = f(x) + 3$}

\vspace{0.5cm}

\noindent\makebox[1.5em][l]{\textbf{7.}}\parbox[t]{\dimexpr\textwidth-1.5em}{Графік функції $y = f(x)$ паралельно перенесли вздовж осі $Ox$ на 5 одиниць праворуч. Укажіть формулу для отриманої функції $y = g(x)$. \nmtyear{2026}}
\vspace{0.3cm}

\answerTable{$y = 5f(x)$}{$y = f(x) - 5$}{$y = f(x + 5)$}{$y = f(x) + 5$}{$y = f(x - 5)$}

\vspace{0.5cm}

\noindent\makebox[1.5em][l]{\textbf{8.}}\parbox[t]{\dimexpr\textwidth-1.5em}{Графік функції $y = f(x)$ паралельно перенесли вздовж осі $Oy$ на 1 одиниць вгору. Укажіть формулу для отриманої функції $y = g(x)$. \nmtyear{2026}}
\vspace{0.3cm}

\answerTable{$y = f(x + 1)$}{$y = 1f(x)$}{$y = f(x) - 1$}{$y = f(x - 1)$}{$y = f(x) + 1$}

\vspace{0.5cm}

\noindent\makebox[1.5em][l]{\textbf{9.}}\parbox[t]{\dimexpr\textwidth-1.5em}{Графік функції $y = f(x)$ паралельно перенесли вздовж осі $Ox$ на 2 одиниць праворуч. Укажіть формулу для отриманої функції $y = g(x)$. \nmtyear{2026}}
\vspace{0.3cm}

\answerTable{$y = f(x + 2)$}{$y = f(x - 2)$}{$y = f(x) + 2$}{$y = f(x) - 2$}{$y = 2f(x)$}

\vspace{0.5cm}

\noindent\makebox[1.5em][l]{\textbf{10.}}\parbox[t]{\dimexpr\textwidth-1.5em}{Графік функції $y = f(x)$ паралельно перенесли вздовж осі $Oy$ на 5 одиниць вниз. Укажіть формулу для отриманої функції $y = g(x)$. \nmtyear{2026}}
\vspace{0.3cm}

\answerTable{$y = f(x + 5)$}{$y = 5f(x)$}{$y = f(x) - 5$}{$y = f(x - 5)$}{$y = f(x) + 5$}

\vspace{0.5cm}

\noindent\makebox[1.5em][l]{\textbf{11.}}\parbox[t]{\dimexpr\textwidth-1.5em}{Графік функції $y = f(x)$ паралельно перенесли вздовж осі $Oy$ на 4 одиниць вниз. Укажіть формулу для отриманої функції $y = g(x)$. \nmtyear{2026}}
\vspace{0.3cm}

\answerTable{$y = f(x) + 4$}{$y = f(x - 4)$}{$y = f(x + 4)$}{$y = 4f(x)$}{$y = f(x) - 4$}

\vspace{0.5cm}

\noindent\makebox[1.5em][l]{\textbf{12.}}\parbox[t]{\dimexpr\textwidth-1.5em}{Графік функції $y = f(x)$ паралельно перенесли вздовж осі $Ox$ на 2 одиниць ліворуч. Укажіть формулу для отриманої функції $y = g(x)$. \nmtyear{2026}}
\vspace{0.3cm}

\answerTable{$y = f(x + 2)$}{$y = 2f(x)$}{$y = f(x - 2)$}{$y = f(x) - 2$}{$y = f(x) + 2$}

\vspace{0.5cm}

\noindent\makebox[1.5em][l]{\textbf{13.}}\parbox[t]{\dimexpr\textwidth-1.5em}{Графік функції $y = f(x)$ паралельно перенесли вздовж осі $Ox$ на 5 одиниць праворуч. Укажіть формулу для отриманої функції $y = g(x)$. \nmtyear{2026}}
\vspace{0.3cm}

\answerTable{$y = f(x) - 5$}{$y = 5f(x)$}{$y = f(x - 5)$}{$y = f(x + 5)$}{$y = f(x) + 5$}

\vspace{0.5cm}

\noindent\makebox[1.5em][l]{\textbf{14.}}\parbox[t]{\dimexpr\textwidth-1.5em}{Графік функції $y = f(x)$ паралельно перенесли вздовж осі $Ox$ на 3 одиниць ліворуч. Укажіть формулу для отриманої функції $y = g(x)$. \nmtyear{2026}}
\vspace{0.3cm}

\answerTable{$y = f(x) + 3$}{$y = f(x + 3)$}{$y = 3f(x)$}{$y = f(x) - 3$}{$y = f(x - 3)$}

\vspace{0.5cm}

\noindent\makebox[1.5em][l]{\textbf{15.}}\parbox[t]{\dimexpr\textwidth-1.5em}{Графік функції $y = f(x)$ паралельно перенесли вздовж осі $Ox$ на 2 одиниць праворуч. Укажіть формулу для отриманої функції $y = g(x)$. \nmtyear{2026}}
\vspace{0.3cm}

\answerTable{$y = f(x + 2)$}{$y = 2f(x)$}{$y = f(x - 2)$}{$y = f(x) - 2$}{$y = f(x) + 2$}

\vspace{0.5cm}

\noindent\makebox[1.5em][l]{\textbf{16.}}\parbox[t]{\dimexpr\textwidth-1.5em}{Графік функції $y = f(x)$ паралельно перенесли вздовж осі $Ox$ на 4 одиниць праворуч. Укажіть формулу для отриманої функції $y = g(x)$. \nmtyear{2026}}
\vspace{0.3cm}

\answerTable{$y = f(x - 4)$}{$y = 4f(x)$}{$y = f(x + 4)$}{$y = f(x) + 4$}{$y = f(x) - 4$}

\vspace{0.5cm}

\noindent\makebox[1.5em][l]{\textbf{17.}}\parbox[t]{\dimexpr\textwidth-1.5em}{Графік функції $y = f(x)$ паралельно перенесли вздовж осі $Ox$ на 5 одиниць праворуч. Укажіть формулу для отриманої функції $y = g(x)$. \nmtyear{2026}}
\vspace{0.3cm}

\answerTable{$y = f(x) + 5$}{$y = 5f(x)$}{$y = f(x + 5)$}{$y = f(x - 5)$}{$y = f(x) - 5$}

\vspace{0.5cm}

\noindent\makebox[1.5em][l]{\textbf{18.}}\parbox[t]{\dimexpr\textwidth-1.5em}{Графік функції $y = f(x)$ паралельно перенесли вздовж осі $Oy$ на 2 одиниць вниз. Укажіть формулу для отриманої функції $y = g(x)$. \nmtyear{2026}}
\vspace{0.3cm}

\answerTable{$y = f(x) - 2$}{$y = f(x - 2)$}{$y = f(x) + 2$}{$y = 2f(x)$}{$y = f(x + 2)$}

\vspace{0.5cm}

\noindent\makebox[1.5em][l]{\textbf{19.}}\parbox[t]{\dimexpr\textwidth-1.5em}{Графік функції $y = f(x)$ паралельно перенесли вздовж осі $Ox$ на 5 одиниць праворуч. Укажіть формулу для отриманої функції $y = g(x)$. \nmtyear{2026}}
\vspace{0.3cm}

\answerTable{$y = f(x) + 5$}{$y = f(x - 5)$}{$y = f(x + 5)$}{$y = 5f(x)$}{$y = f(x) - 5$}

\vspace{0.5cm}

\noindent\makebox[1.5em][l]{\textbf{20.}}\parbox[t]{\dimexpr\textwidth-1.5em}{Графік функції $y = f(x)$ паралельно перенесли вздовж осі $Ox$ на 4 одиниць ліворуч. Укажіть формулу для отриманої функції $y = g(x)$. \nmtyear{2026}}
\vspace{0.3cm}

\answerTable{$y = 4f(x)$}{$y = f(x) - 4$}{$y = f(x + 4)$}{$y = f(x) + 4$}{$y = f(x - 4)$}

\vspace{0.5cm}

% === Matching: Functions: Graph Shifts ===
\noindent\textbf{21.} Установіть відповідність між виразом (1--3) та його значенням (А--Д). \nmtyear{2026}
\vspace{0.3cm}

\matchingLayout{
\textit{Вираз}

\textbf{1} \quad Графік функції $y = f(x)$ паралельно перенесли вздовж осі $Ox$ на 4 одиниць ліворуч. Укажіть формулу для отриманої функції $y = g(x)$.

\vspace{0.4cm}

\textbf{2} \quad Графік функції $y = f(x)$ паралельно перенесли вздовж осі $Oy$ на 2 одиниць вниз. Укажіть формулу для отриманої функції $y = g(x)$.

\vspace{0.4cm}

\textbf{3} \quad Графік функції $y = f(x)$ паралельно перенесли вздовж осі $Ox$ на 5 одиниць праворуч. Укажіть формулу для отриманої функції $y = g(x)$.

\vspace{0.4cm}

}{
\textit{Значення}

\textbf{А} \quad $y = f(x + 4)$

\vspace{0.4cm}

\textbf{Б} \quad $y = f(x) + 2$

\vspace{0.4cm}

\textbf{В} \quad $y = f(x + 2)$

\vspace{0.4cm}

\textbf{Г} \quad $y = f(x) - 2$

\vspace{0.4cm}

\textbf{Д} \quad $y = f(x - 5)$

\vspace{0.4cm}

}{\answerGrid}

\vspace{0.8cm}

\noindent\textbf{22.} Установіть відповідність між виразом (1--3) та його значенням (А--Д). \nmtyear{2026}
\vspace{0.3cm}

\matchingLayout{
\textit{Вираз}

\textbf{1} \quad Графік функції $y = f(x)$ паралельно перенесли вздовж осі $Oy$ на 5 одиниць вгору. Укажіть формулу для отриманої функції $y = g(x)$.

\vspace{0.4cm}

\textbf{2} \quad Графік функції $y = f(x)$ паралельно перенесли вздовж осі $Ox$ на 5 одиниць праворуч. Укажіть формулу для отриманої функції $y = g(x)$.

\vspace{0.4cm}

\textbf{3} \quad Графік функції $y = f(x)$ паралельно перенесли вздовж осі $Ox$ на 1 одиниць праворуч. Укажіть формулу для отриманої функції $y = g(x)$.

\vspace{0.4cm}

}{
\textit{Значення}

\textbf{А} \quad $y = f(x - 1)$

\vspace{0.4cm}

\textbf{Б} \quad $y = 1f(x)$

\vspace{0.4cm}

\textbf{В} \quad $y = f(x - 5)$

\vspace{0.4cm}

\textbf{Г} \quad $y = f(x) + 5$

\vspace{0.4cm}

\textbf{Д} \quad $y = f(x + 5)$

\vspace{0.4cm}

}{\answerGrid}

\vspace{0.8cm}

\noindent\textbf{23.} Установіть відповідність між виразом (1--3) та його значенням (А--Д). \nmtyear{2026}
\vspace{0.3cm}

\matchingLayout{
\textit{Вираз}

\textbf{1} \quad Графік функції $y = f(x)$ паралельно перенесли вздовж осі $Oy$ на 1 одиниць вгору. Укажіть формулу для отриманої функції $y = g(x)$.

\vspace{0.4cm}

\textbf{2} \quad Графік функції $y = f(x)$ паралельно перенесли вздовж осі $Ox$ на 2 одиниць праворуч. Укажіть формулу для отриманої функції $y = g(x)$.

\vspace{0.4cm}

\textbf{3} \quad Графік функції $y = f(x)$ паралельно перенесли вздовж осі $Oy$ на 4 одиниць вгору. Укажіть формулу для отриманої функції $y = g(x)$.

\vspace{0.4cm}

}{
\textit{Значення}

\textbf{А} \quad $y = f(x) - 4$

\vspace{0.4cm}

\textbf{Б} \quad $y = f(x - 2)$

\vspace{0.4cm}

\textbf{В} \quad $y = f(x) + 1$

\vspace{0.4cm}

\textbf{Г} \quad $y = f(x) + 4$

\vspace{0.4cm}

\textbf{Д} \quad $y = f(x - 4)$

\vspace{0.4cm}

}{\answerGrid}

\vspace{0.8cm}

\noindent\textbf{24.} Установіть відповідність між виразом (1--3) та його значенням (А--Д). \nmtyear{2026}
\vspace{0.3cm}

\matchingLayout{
\textit{Вираз}

\textbf{1} \quad Графік функції $y = f(x)$ паралельно перенесли вздовж осі $Oy$ на 4 одиниць вгору. Укажіть формулу для отриманої функції $y = g(x)$.

\vspace{0.4cm}

\textbf{2} \quad Графік функції $y = f(x)$ паралельно перенесли вздовж осі $Ox$ на 3 одиниць праворуч. Укажіть формулу для отриманої функції $y = g(x)$.

\vspace{0.4cm}

\textbf{3} \quad Графік функції $y = f(x)$ паралельно перенесли вздовж осі $Oy$ на 4 одиниць вгору. Укажіть формулу для отриманої функції $y = g(x)$.

\vspace{0.4cm}

}{
\textit{Значення}

\textbf{А} \quad $y = f(x) + 4$

\vspace{0.4cm}

\textbf{Б} \quad $y = f(x) + 4$

\vspace{0.4cm}

\textbf{В} \quad $y = f(x - 4)$

\vspace{0.4cm}

\textbf{Г} \quad $y = f(x) + 3$

\vspace{0.4cm}

\textbf{Д} \quad $y = f(x - 3)$

\vspace{0.4cm}

}{\answerGrid}

\vspace{0.8cm}

\noindent\textbf{25.} Установіть відповідність між виразом (1--3) та його значенням (А--Д). \nmtyear{2026}
\vspace{0.3cm}

\matchingLayout{
\textit{Вираз}

\textbf{1} \quad Графік функції $y = f(x)$ паралельно перенесли вздовж осі $Ox$ на 5 одиниць праворуч. Укажіть формулу для отриманої функції $y = g(x)$.

\vspace{0.4cm}

\textbf{2} \quad Графік функції $y = f(x)$ паралельно перенесли вздовж осі $Oy$ на 1 одиниць вниз. Укажіть формулу для отриманої функції $y = g(x)$.

\vspace{0.4cm}

\textbf{3} \quad Графік функції $y = f(x)$ паралельно перенесли вздовж осі $Ox$ на 5 одиниць праворуч. Укажіть формулу для отриманої функції $y = g(x)$.

\vspace{0.4cm}

}{
\textit{Значення}

\textbf{А} \quad $y = f(x) - 1$

\vspace{0.4cm}

\textbf{Б} \quad $y = f(x) + 5$

\vspace{0.4cm}

\textbf{В} \quad $y = f(x - 1)$

\vspace{0.4cm}

\textbf{Г} \quad $y = f(x - 5)$

\vspace{0.4cm}

\textbf{Д} \quad $y = f(x - 5)$

\vspace{0.4cm}

}{\answerGrid}

\vspace{0.8cm}

\noindent\textbf{26.} Установіть відповідність між виразом (1--3) та його значенням (А--Д). \nmtyear{2026}
\vspace{0.3cm}

\matchingLayout{
\textit{Вираз}

\textbf{1} \quad Графік функції $y = f(x)$ паралельно перенесли вздовж осі $Ox$ на 3 одиниць праворуч. Укажіть формулу для отриманої функції $y = g(x)$.

\vspace{0.4cm}

\textbf{2} \quad Графік функції $y = f(x)$ паралельно перенесли вздовж осі $Oy$ на 1 одиниць вгору. Укажіть формулу для отриманої функції $y = g(x)$.

\vspace{0.4cm}

\textbf{3} \quad Графік функції $y = f(x)$ паралельно перенесли вздовж осі $Ox$ на 1 одиниць ліворуч. Укажіть формулу для отриманої функції $y = g(x)$.

\vspace{0.4cm}

}{
\textit{Значення}

\textbf{А} \quad $y = f(x + 1)$

\vspace{0.4cm}

\textbf{Б} \quad $y = f(x - 3)$

\vspace{0.4cm}

\textbf{В} \quad $y = f(x) + 1$

\vspace{0.4cm}

\textbf{Г} \quad $y = f(x - 1)$

\vspace{0.4cm}

\textbf{Д} \quad $y = f(x) + 3$

\vspace{0.4cm}

}{\answerGrid}

\vspace{0.8cm}

\noindent\textbf{27.} Установіть відповідність між виразом (1--3) та його значенням (А--Д). \nmtyear{2026}
\vspace{0.3cm}

\matchingLayout{
\textit{Вираз}

\textbf{1} \quad Графік функції $y = f(x)$ паралельно перенесли вздовж осі $Ox$ на 5 одиниць праворуч. Укажіть формулу для отриманої функції $y = g(x)$.

\vspace{0.4cm}

\textbf{2} \quad Графік функції $y = f(x)$ паралельно перенесли вздовж осі $Ox$ на 3 одиниць ліворуч. Укажіть формулу для отриманої функції $y = g(x)$.

\vspace{0.4cm}

\textbf{3} \quad Графік функції $y = f(x)$ паралельно перенесли вздовж осі $Ox$ на 5 одиниць праворуч. Укажіть формулу для отриманої функції $y = g(x)$.

\vspace{0.4cm}

}{
\textit{Значення}

\textbf{А} \quad $y = f(x) + 5$

\vspace{0.4cm}

\textbf{Б} \quad $y = f(x - 5)$

\vspace{0.4cm}

\textbf{В} \quad $y = f(x - 5)$

\vspace{0.4cm}

\textbf{Г} \quad $y = f(x + 3)$

\vspace{0.4cm}

\textbf{Д} \quad $y = f(x) - 3$

\vspace{0.4cm}

}{\answerGrid}

\vspace{0.8cm}

\noindent\textbf{28.} Установіть відповідність між виразом (1--3) та його значенням (А--Д). \nmtyear{2026}
\vspace{0.3cm}

\matchingLayout{
\textit{Вираз}

\textbf{1} \quad Графік функції $y = f(x)$ паралельно перенесли вздовж осі $Ox$ на 1 одиниць ліворуч. Укажіть формулу для отриманої функції $y = g(x)$.

\vspace{0.4cm}

\textbf{2} \quad Графік функції $y = f(x)$ паралельно перенесли вздовж осі $Oy$ на 1 одиниць вниз. Укажіть формулу для отриманої функції $y = g(x)$.

\vspace{0.4cm}

\textbf{3} \quad Графік функції $y = f(x)$ паралельно перенесли вздовж осі $Ox$ на 3 одиниць праворуч. Укажіть формулу для отриманої функції $y = g(x)$.

\vspace{0.4cm}

}{
\textit{Значення}

\textbf{А} \quad $y = f(x) - 1$

\vspace{0.4cm}

\textbf{Б} \quad $y = f(x) + 1$

\vspace{0.4cm}

\textbf{В} \quad $y = 1f(x)$

\vspace{0.4cm}

\textbf{Г} \quad $y = f(x - 3)$

\vspace{0.4cm}

\textbf{Д} \quad $y = f(x + 1)$

\vspace{0.4cm}

}{\answerGrid}

\vspace{0.8cm}

\noindent\textbf{29.} Установіть відповідність між виразом (1--3) та його значенням (А--Д). \nmtyear{2026}
\vspace{0.3cm}

\matchingLayout{
\textit{Вираз}

\textbf{1} \quad Графік функції $y = f(x)$ паралельно перенесли вздовж осі $Ox$ на 3 одиниць ліворуч. Укажіть формулу для отриманої функції $y = g(x)$.

\vspace{0.4cm}

\textbf{2} \quad Графік функції $y = f(x)$ паралельно перенесли вздовж осі $Ox$ на 1 одиниць праворуч. Укажіть формулу для отриманої функції $y = g(x)$.

\vspace{0.4cm}

\textbf{3} \quad Графік функції $y = f(x)$ паралельно перенесли вздовж осі $Ox$ на 2 одиниць ліворуч. Укажіть формулу для отриманої функції $y = g(x)$.

\vspace{0.4cm}

}{
\textit{Значення}

\textbf{А} \quad $y = f(x) + 2$

\vspace{0.4cm}

\textbf{Б} \quad $y = f(x + 3)$

\vspace{0.4cm}

\textbf{В} \quad $y = f(x + 2)$

\vspace{0.4cm}

\textbf{Г} \quad $y = 2f(x)$

\vspace{0.4cm}

\textbf{Д} \quad $y = f(x - 1)$

\vspace{0.4cm}

}{\answerGrid}

\vspace{0.8cm}

\noindent\textbf{30.} Установіть відповідність між виразом (1--3) та його значенням (А--Д). \nmtyear{2026}
\vspace{0.3cm}

\matchingLayout{
\textit{Вираз}

\textbf{1} \quad Графік функції $y = f(x)$ паралельно перенесли вздовж осі $Oy$ на 2 одиниць вгору. Укажіть формулу для отриманої функції $y = g(x)$.

\vspace{0.4cm}

\textbf{2} \quad Графік функції $y = f(x)$ паралельно перенесли вздовж осі $Ox$ на 1 одиниць праворуч. Укажіть формулу для отриманої функції $y = g(x)$.

\vspace{0.4cm}

\textbf{3} \quad Графік функції $y = f(x)$ паралельно перенесли вздовж осі $Oy$ на 1 одиниць вниз. Укажіть формулу для отриманої функції $y = g(x)$.

\vspace{0.4cm}

}{
\textit{Значення}

\textbf{А} \quad $y = f(x - 2)$

\vspace{0.4cm}

\textbf{Б} \quad $y = f(x - 1)$

\vspace{0.4cm}

\textbf{В} \quad $y = f(x) + 2$

\vspace{0.4cm}

\textbf{Г} \quad $y = 2f(x)$

\vspace{0.4cm}

\textbf{Д} \quad $y = f(x) - 1$

\vspace{0.4cm}

}{\answerGrid}

\vspace{0.8cm}

\noindent\textbf{31.} Установіть відповідність між виразом (1--3) та його значенням (А--Д). \nmtyear{2026}
\vspace{0.3cm}

\matchingLayout{
\textit{Вираз}

\textbf{1} \quad Графік функції $y = f(x)$ паралельно перенесли вздовж осі $Oy$ на 2 одиниць вгору. Укажіть формулу для отриманої функції $y = g(x)$.

\vspace{0.4cm}

\textbf{2} \quad Графік функції $y = f(x)$ паралельно перенесли вздовж осі $Oy$ на 5 одиниць вгору. Укажіть формулу для отриманої функції $y = g(x)$.

\vspace{0.4cm}

\textbf{3} \quad Графік функції $y = f(x)$ паралельно перенесли вздовж осі $Ox$ на 3 одиниць праворуч. Укажіть формулу для отриманої функції $y = g(x)$.

\vspace{0.4cm}

}{
\textit{Значення}

\textbf{А} \quad $y = f(x) + 5$

\vspace{0.4cm}

\textbf{Б} \quad $y = f(x) - 2$

\vspace{0.4cm}

\textbf{В} \quad $y = f(x) + 2$

\vspace{0.4cm}

\textbf{Г} \quad $y = f(x) - 3$

\vspace{0.4cm}

\textbf{Д} \quad $y = f(x - 3)$

\vspace{0.4cm}

}{\answerGrid}

\vspace{0.8cm}

\noindent\textbf{32.} Установіть відповідність між виразом (1--3) та його значенням (А--Д). \nmtyear{2026}
\vspace{0.3cm}

\matchingLayout{
\textit{Вираз}

\textbf{1} \quad Графік функції $y = f(x)$ паралельно перенесли вздовж осі $Ox$ на 1 одиниць праворуч. Укажіть формулу для отриманої функції $y = g(x)$.

\vspace{0.4cm}

\textbf{2} \quad Графік функції $y = f(x)$ паралельно перенесли вздовж осі $Oy$ на 2 одиниць вниз. Укажіть формулу для отриманої функції $y = g(x)$.

\vspace{0.4cm}

\textbf{3} \quad Графік функції $y = f(x)$ паралельно перенесли вздовж осі $Ox$ на 5 одиниць ліворуч. Укажіть формулу для отриманої функції $y = g(x)$.

\vspace{0.4cm}

}{
\textit{Значення}

\textbf{А} \quad $y = f(x + 5)$

\vspace{0.4cm}

\textbf{Б} \quad $y = f(x) + 1$

\vspace{0.4cm}

\textbf{В} \quad $y = f(x - 1)$

\vspace{0.4cm}

\textbf{Г} \quad $y = f(x) - 2$

\vspace{0.4cm}

\textbf{Д} \quad $y = f(x) - 1$

\vspace{0.4cm}

}{\answerGrid}

\vspace{0.8cm}

\noindent\textbf{33.} Установіть відповідність між виразом (1--3) та його значенням (А--Д). \nmtyear{2026}
\vspace{0.3cm}

\matchingLayout{
\textit{Вираз}

\textbf{1} \quad Графік функції $y = f(x)$ паралельно перенесли вздовж осі $Ox$ на 5 одиниць праворуч. Укажіть формулу для отриманої функції $y = g(x)$.

\vspace{0.4cm}

\textbf{2} \quad Графік функції $y = f(x)$ паралельно перенесли вздовж осі $Oy$ на 2 одиниць вгору. Укажіть формулу для отриманої функції $y = g(x)$.

\vspace{0.4cm}

\textbf{3} \quad Графік функції $y = f(x)$ паралельно перенесли вздовж осі $Oy$ на 2 одиниць вниз. Укажіть формулу для отриманої функції $y = g(x)$.

\vspace{0.4cm}

}{
\textit{Значення}

\textbf{А} \quad $y = f(x) + 2$

\vspace{0.4cm}

\textbf{Б} \quad $y = f(x - 5)$

\vspace{0.4cm}

\textbf{В} \quad $y = f(x) - 2$

\vspace{0.4cm}

\textbf{Г} \quad $y = 2f(x)$

\vspace{0.4cm}

\textbf{Д} \quad $y = f(x - 2)$

\vspace{0.4cm}

}{\answerGrid}

\vspace{0.8cm}

\noindent\textbf{34.} Установіть відповідність між виразом (1--3) та його значенням (А--Д). \nmtyear{2026}
\vspace{0.3cm}

\matchingLayout{
\textit{Вираз}

\textbf{1} \quad Графік функції $y = f(x)$ паралельно перенесли вздовж осі $Ox$ на 1 одиниць ліворуч. Укажіть формулу для отриманої функції $y = g(x)$.

\vspace{0.4cm}

\textbf{2} \quad Графік функції $y = f(x)$ паралельно перенесли вздовж осі $Ox$ на 2 одиниць праворуч. Укажіть формулу для отриманої функції $y = g(x)$.

\vspace{0.4cm}

\textbf{3} \quad Графік функції $y = f(x)$ паралельно перенесли вздовж осі $Ox$ на 5 одиниць праворуч. Укажіть формулу для отриманої функції $y = g(x)$.

\vspace{0.4cm}

}{
\textit{Значення}

\textbf{А} \quad $y = f(x - 5)$

\vspace{0.4cm}

\textbf{Б} \quad $y = f(x + 1)$

\vspace{0.4cm}

\textbf{В} \quad $y = f(x - 1)$

\vspace{0.4cm}

\textbf{Г} \quad $y = f(x - 2)$

\vspace{0.4cm}

\textbf{Д} \quad $y = f(x) - 1$

\vspace{0.4cm}

}{\answerGrid}

\vspace{0.8cm}

\noindent\textbf{35.} Установіть відповідність між виразом (1--3) та його значенням (А--Д). \nmtyear{2026}
\vspace{0.3cm}

\matchingLayout{
\textit{Вираз}

\textbf{1} \quad Графік функції $y = f(x)$ паралельно перенесли вздовж осі $Oy$ на 3 одиниць вгору. Укажіть формулу для отриманої функції $y = g(x)$.

\vspace{0.4cm}

\textbf{2} \quad Графік функції $y = f(x)$ паралельно перенесли вздовж осі $Oy$ на 3 одиниць вгору. Укажіть формулу для отриманої функції $y = g(x)$.

\vspace{0.4cm}

\textbf{3} \quad Графік функції $y = f(x)$ паралельно перенесли вздовж осі $Ox$ на 5 одиниць праворуч. Укажіть формулу для отриманої функції $y = g(x)$.

\vspace{0.4cm}

}{
\textit{Значення}

\textbf{А} \quad $y = f(x - 5)$

\vspace{0.4cm}

\textbf{Б} \quad $y = f(x + 3)$

\vspace{0.4cm}

\textbf{В} \quad $y = f(x) + 3$

\vspace{0.4cm}

\textbf{Г} \quad $y = f(x) - 5$

\vspace{0.4cm}

\textbf{Д} \quad $y = f(x) + 3$

\vspace{0.4cm}

}{\answerGrid}

\vspace{0.8cm}

\noindent\textbf{36.} Установіть відповідність між виразом (1--3) та його значенням (А--Д). \nmtyear{2026}
\vspace{0.3cm}

\matchingLayout{
\textit{Вираз}

\textbf{1} \quad Графік функції $y = f(x)$ паралельно перенесли вздовж осі $Oy$ на 2 одиниць вниз. Укажіть формулу для отриманої функції $y = g(x)$.

\vspace{0.4cm}

\textbf{2} \quad Графік функції $y = f(x)$ паралельно перенесли вздовж осі $Oy$ на 4 одиниць вгору. Укажіть формулу для отриманої функції $y = g(x)$.

\vspace{0.4cm}

\textbf{3} \quad Графік функції $y = f(x)$ паралельно перенесли вздовж осі $Oy$ на 2 одиниць вгору. Укажіть формулу для отриманої функції $y = g(x)$.

\vspace{0.4cm}

}{
\textit{Значення}

\textbf{А} \quad $y = f(x) - 2$

\vspace{0.4cm}

\textbf{Б} \quad $y = f(x) + 2$

\vspace{0.4cm}

\textbf{В} \quad $y = f(x - 2)$

\vspace{0.4cm}

\textbf{Г} \quad $y = f(x) + 4$

\vspace{0.4cm}

\textbf{Д} \quad $y = f(x + 4)$

\vspace{0.4cm}

}{\answerGrid}

\vspace{0.8cm}

\noindent\textbf{37.} Установіть відповідність між виразом (1--3) та його значенням (А--Д). \nmtyear{2026}
\vspace{0.3cm}

\matchingLayout{
\textit{Вираз}

\textbf{1} \quad Графік функції $y = f(x)$ паралельно перенесли вздовж осі $Ox$ на 1 одиниць праворуч. Укажіть формулу для отриманої функції $y = g(x)$.

\vspace{0.4cm}

\textbf{2} \quad Графік функції $y = f(x)$ паралельно перенесли вздовж осі $Ox$ на 4 одиниць ліворуч. Укажіть формулу для отриманої функції $y = g(x)$.

\vspace{0.4cm}

\textbf{3} \quad Графік функції $y = f(x)$ паралельно перенесли вздовж осі $Oy$ на 3 одиниць вниз. Укажіть формулу для отриманої функції $y = g(x)$.

\vspace{0.4cm}

}{
\textit{Значення}

\textbf{А} \quad $y = f(x + 4)$

\vspace{0.4cm}

\textbf{Б} \quad $y = f(x - 1)$

\vspace{0.4cm}

\textbf{В} \quad $y = 3f(x)$

\vspace{0.4cm}

\textbf{Г} \quad $y = f(x) - 3$

\vspace{0.4cm}

\textbf{Д} \quad $y = f(x) + 3$

\vspace{0.4cm}

}{\answerGrid}

\vspace{0.8cm}

\noindent\textbf{38.} Установіть відповідність між виразом (1--3) та його значенням (А--Д). \nmtyear{2026}
\vspace{0.3cm}

\matchingLayout{
\textit{Вираз}

\textbf{1} \quad Графік функції $y = f(x)$ паралельно перенесли вздовж осі $Oy$ на 4 одиниць вниз. Укажіть формулу для отриманої функції $y = g(x)$.

\vspace{0.4cm}

\textbf{2} \quad Графік функції $y = f(x)$ паралельно перенесли вздовж осі $Oy$ на 3 одиниць вгору. Укажіть формулу для отриманої функції $y = g(x)$.

\vspace{0.4cm}

\textbf{3} \quad Графік функції $y = f(x)$ паралельно перенесли вздовж осі $Oy$ на 1 одиниць вгору. Укажіть формулу для отриманої функції $y = g(x)$.

\vspace{0.4cm}

}{
\textit{Значення}

\textbf{А} \quad $y = f(x - 3)$

\vspace{0.4cm}

\textbf{Б} \quad $y = f(x) - 1$

\vspace{0.4cm}

\textbf{В} \quad $y = f(x) + 1$

\vspace{0.4cm}

\textbf{Г} \quad $y = f(x) + 3$

\vspace{0.4cm}

\textbf{Д} \quad $y = f(x) - 4$

\vspace{0.4cm}

}{\answerGrid}

\vspace{0.8cm}

\noindent\textbf{39.} Установіть відповідність між виразом (1--3) та його значенням (А--Д). \nmtyear{2026}
\vspace{0.3cm}

\matchingLayout{
\textit{Вираз}

\textbf{1} \quad Графік функції $y = f(x)$ паралельно перенесли вздовж осі $Oy$ на 1 одиниць вгору. Укажіть формулу для отриманої функції $y = g(x)$.

\vspace{0.4cm}

\textbf{2} \quad Графік функції $y = f(x)$ паралельно перенесли вздовж осі $Oy$ на 3 одиниць вниз. Укажіть формулу для отриманої функції $y = g(x)$.

\vspace{0.4cm}

\textbf{3} \quad Графік функції $y = f(x)$ паралельно перенесли вздовж осі $Oy$ на 4 одиниць вниз. Укажіть формулу для отриманої функції $y = g(x)$.

\vspace{0.4cm}

}{
\textit{Значення}

\textbf{А} \quad $y = f(x) - 3$

\vspace{0.4cm}

\textbf{Б} \quad $y = f(x) - 1$

\vspace{0.4cm}

\textbf{В} \quad $y = f(x) - 4$

\vspace{0.4cm}

\textbf{Г} \quad $y = f(x) + 1$

\vspace{0.4cm}

\textbf{Д} \quad $y = 4f(x)$

\vspace{0.4cm}

}{\answerGrid}

\vspace{0.8cm}

\noindent\textbf{40.} Установіть відповідність між виразом (1--3) та його значенням (А--Д). \nmtyear{2026}
\vspace{0.3cm}

\matchingLayout{
\textit{Вираз}

\textbf{1} \quad Графік функції $y = f(x)$ паралельно перенесли вздовж осі $Oy$ на 2 одиниць вгору. Укажіть формулу для отриманої функції $y = g(x)$.

\vspace{0.4cm}

\textbf{2} \quad Графік функції $y = f(x)$ паралельно перенесли вздовж осі $Oy$ на 4 одиниць вгору. Укажіть формулу для отриманої функції $y = g(x)$.

\vspace{0.4cm}

\textbf{3} \quad Графік функції $y = f(x)$ паралельно перенесли вздовж осі $Ox$ на 1 одиниць ліворуч. Укажіть формулу для отриманої функції $y = g(x)$.

\vspace{0.4cm}

}{
\textit{Значення}

\textbf{А} \quad $y = f(x - 2)$

\vspace{0.4cm}

\textbf{Б} \quad $y = f(x + 2)$

\vspace{0.4cm}

\textbf{В} \quad $y = f(x) + 2$

\vspace{0.4cm}

\textbf{Г} \quad $y = f(x) + 4$

\vspace{0.4cm}

\textbf{Д} \quad $y = f(x + 1)$

\vspace{0.4cm}

}{\answerGrid}

\vspace{0.8cm}

% === Functions: Domain ===
\noindent\makebox[1.5em][l]{\textbf{41.}}\parbox[t]{\dimexpr\textwidth-1.5em}{Укажіть область визначення функції $y = \log_{2} (x + 4)$. \nmtyear{2026}}
\vspace{0.3cm}

\answerTable{[-4; +\infty)}{(-4; +\infty)}{22}{(-\infty; -4)}{(4; +\infty)}

\vspace{0.5cm}

\noindent\makebox[1.5em][l]{\textbf{42.}}\parbox[t]{\dimexpr\textwidth-1.5em}{Укажіть область визначення функції $y = \frac{3}{9 - x}$. \nmtyear{2026}}
\vspace{0.3cm}

\answerTable{(9; +\infty)}{(-\infty; 9) \cup (9; +\infty)}{(-\infty; -9) \cup (-9; +\infty)}{10}{(-\infty; 9)}

\vspace{0.5cm}

\noindent\makebox[1.5em][l]{\textbf{43.}}\parbox[t]{\dimexpr\textwidth-1.5em}{Укажіть область визначення функції $y = \frac{5}{8 - x}$. \nmtyear{2026}}
\vspace{0.3cm}

\answerTable{(-\infty; 8) \cup (8; +\infty)}{(-\infty; -8) \cup (-8; +\infty)}{49}{(8; +\infty)}{(-\infty; 8)}

\vspace{0.5cm}

\noindent\makebox[1.5em][l]{\textbf{44.}}\parbox[t]{\dimexpr\textwidth-1.5em}{Укажіть область визначення функції $y = \sqrt{x + 4}$. \nmtyear{2026}}
\vspace{0.3cm}

\answerTable{(-4; +\infty)}{[-4; +\infty)}{(-\infty; -4]}{[4; +\infty)}{97}

\vspace{0.5cm}

\noindent\makebox[1.5em][l]{\textbf{45.}}\parbox[t]{\dimexpr\textwidth-1.5em}{Укажіть область визначення функції $y = \frac{3}{6 - x}$. \nmtyear{2026}}
\vspace{0.3cm}

\answerTable{20}{(6; +\infty)}{(-\infty; 6) \cup (6; +\infty)}{(-\infty; 6)}{(-\infty; -6) \cup (-6; +\infty)}

\vspace{0.5cm}

\noindent\makebox[1.5em][l]{\textbf{46.}}\parbox[t]{\dimexpr\textwidth-1.5em}{Укажіть область визначення функції $y = \sqrt{x + 7}$. \nmtyear{2026}}
\vspace{0.3cm}

\answerTable{16}{(-7; +\infty)}{(-\infty; -7]}{[7; +\infty)}{[-7; +\infty)}

\vspace{0.5cm}

\noindent\makebox[1.5em][l]{\textbf{47.}}\parbox[t]{\dimexpr\textwidth-1.5em}{Укажіть область визначення функції $y = \sqrt{x + 8}$. \nmtyear{2026}}
\vspace{0.3cm}

\answerTable{(-8; +\infty)}{81}{(-\infty; -8]}{[8; +\infty)}{[-8; +\infty)}

\vspace{0.5cm}

\noindent\makebox[1.5em][l]{\textbf{48.}}\parbox[t]{\dimexpr\textwidth-1.5em}{Укажіть область визначення функції $y = \log_{5} (x + 8)$. \nmtyear{2026}}
\vspace{0.3cm}

\answerTable{(-\infty; -8)}{(-8; +\infty)}{(8; +\infty)}{[-8; +\infty)}{79}

\vspace{0.5cm}

\noindent\makebox[1.5em][l]{\textbf{49.}}\parbox[t]{\dimexpr\textwidth-1.5em}{Укажіть область визначення функції $y = \frac{3}{3 - x}$. \nmtyear{2026}}
\vspace{0.3cm}

\answerTable{(-\infty; 3)}{(-\infty; -3) \cup (-3; +\infty)}{17}{(3; +\infty)}{(-\infty; 3) \cup (3; +\infty)}

\vspace{0.5cm}

\noindent\makebox[1.5em][l]{\textbf{50.}}\parbox[t]{\dimexpr\textwidth-1.5em}{Укажіть область визначення функції $y = \sqrt{9 - x}$. \nmtyear{2026}}
\vspace{0.3cm}

\answerTable{84}{93}{(-\infty; 9]}{[9; +\infty)}{(-\infty; -9]}

\vspace{0.5cm}

\noindent\makebox[1.5em][l]{\textbf{51.}}\parbox[t]{\dimexpr\textwidth-1.5em}{Укажіть область визначення функції $y = \log_{0.5} (x + 8)$. \nmtyear{2026}}
\vspace{0.3cm}

\answerTable{4}{(8; +\infty)}{(-8; +\infty)}{[-8; +\infty)}{(-\infty; -8)}

\vspace{0.5cm}

\noindent\makebox[1.5em][l]{\textbf{52.}}\parbox[t]{\dimexpr\textwidth-1.5em}{Укажіть область визначення функції $y = \sqrt{1 - x}$. \nmtyear{2026}}
\vspace{0.3cm}

\answerTable{82}{[1; +\infty)}{(-\infty; -1]}{(-\infty; 1]}{81}

\vspace{0.5cm}

\noindent\makebox[1.5em][l]{\textbf{53.}}\parbox[t]{\dimexpr\textwidth-1.5em}{Укажіть область визначення функції $y = \log_{5} (6 - x)$. \nmtyear{2026}}
\vspace{0.3cm}

\answerTable{97}{(-\infty; -6)}{(-\infty; 6]}{(-\infty; 6)}{(6; +\infty)}

\vspace{0.5cm}

\noindent\makebox[1.5em][l]{\textbf{54.}}\parbox[t]{\dimexpr\textwidth-1.5em}{Укажіть область визначення функції $y = \frac{1}{x + 9}$. \nmtyear{2026}}
\vspace{0.3cm}

\answerTable{(-\infty; 9) \cup (9; +\infty)}{61}{(-9; +\infty)}{(-\infty; -9)}{(-\infty; -9) \cup (-9; +\infty)}

\vspace{0.5cm}

\noindent\makebox[1.5em][l]{\textbf{55.}}\parbox[t]{\dimexpr\textwidth-1.5em}{Укажіть область визначення функції $y = \frac{1}{5 - x}$. \nmtyear{2026}}
\vspace{0.3cm}

\answerTable{(-\infty; 5) \cup (5; +\infty)}{(-\infty; -5) \cup (-5; +\infty)}{44}{(5; +\infty)}{(-\infty; 5)}

\vspace{0.5cm}

\noindent\makebox[1.5em][l]{\textbf{56.}}\parbox[t]{\dimexpr\textwidth-1.5em}{Укажіть область визначення функції $y = \log_{2} (x + 6)$. \nmtyear{2026}}
\vspace{0.3cm}

\answerTable{(-\infty; -6)}{(6; +\infty)}{81}{(-6; +\infty)}{[-6; +\infty)}

\vspace{0.5cm}

\noindent\makebox[1.5em][l]{\textbf{57.}}\parbox[t]{\dimexpr\textwidth-1.5em}{Укажіть область визначення функції $y = \frac{4}{x + 4}$. \nmtyear{2026}}
\vspace{0.3cm}

\answerTable{39}{(-4; +\infty)}{(-\infty; 4) \cup (4; +\infty)}{(-\infty; -4)}{(-\infty; -4) \cup (-4; +\infty)}

\vspace{0.5cm}

\noindent\makebox[1.5em][l]{\textbf{58.}}\parbox[t]{\dimexpr\textwidth-1.5em}{Укажіть область визначення функції $y = \sqrt{7 - x}$. \nmtyear{2026}}
\vspace{0.3cm}

\answerTable{(-\infty; 7]}{70}{[7; +\infty)}{91}{(-\infty; -7]}

\vspace{0.5cm}

\noindent\makebox[1.5em][l]{\textbf{59.}}\parbox[t]{\dimexpr\textwidth-1.5em}{Укажіть область визначення функції $y = \sqrt{4 - x}$. \nmtyear{2026}}
\vspace{0.3cm}

\answerTable{(-\infty; -4]}{[4; +\infty)}{(-\infty; 4]}{90}{37}

\vspace{0.5cm}

\noindent\makebox[1.5em][l]{\textbf{60.}}\parbox[t]{\dimexpr\textwidth-1.5em}{Укажіть область визначення функції $y = \frac{5}{4 - x}$. \nmtyear{2026}}
\vspace{0.3cm}

\answerTable{(4; +\infty)}{62}{(-\infty; 4)}{(-\infty; -4) \cup (-4; +\infty)}{(-\infty; 4) \cup (4; +\infty)}

\vspace{0.5cm}

% === Matching: Functions: Domain ===
\noindent\textbf{61.} Установіть відповідність між виразом (1--3) та його значенням (А--Д). \nmtyear{2026}
\vspace{0.3cm}

\matchingLayout{
\textit{Вираз}

\textbf{1} \quad Укажіть область визначення функції $y = \sqrt{5 - x}$.

\vspace{0.4cm}

\textbf{2} \quad Укажіть область визначення функції $y = \sqrt{x + 3}$.

\vspace{0.4cm}

\textbf{3} \quad Укажіть область визначення функції $y = \sqrt{8 - x}$.

\vspace{0.4cm}

}{
\textit{Значення}

\textbf{А} \quad (-\infty; 8]

\vspace{0.4cm}

\textbf{Б} \quad [-3; +\infty)

\vspace{0.4cm}

\textbf{В} \quad 33

\vspace{0.4cm}

\textbf{Г} \quad [8; +\infty)

\vspace{0.4cm}

\textbf{Д} \quad (-\infty; 5]

\vspace{0.4cm}

}{\answerGrid}

\vspace{0.8cm}

\noindent\textbf{62.} Установіть відповідність між виразом (1--3) та його значенням (А--Д). \nmtyear{2026}
\vspace{0.3cm}

\matchingLayout{
\textit{Вираз}

\textbf{1} \quad Укажіть область визначення функції $y = \frac{4}{1 - x}$.

\vspace{0.4cm}

\textbf{2} \quad Укажіть область визначення функції $y = \sqrt{3 - x}$.

\vspace{0.4cm}

\textbf{3} \quad Укажіть область визначення функції $y = \log_{5} (x + 2)$.

\vspace{0.4cm}

}{
\textit{Значення}

\textbf{А} \quad (-2; +\infty)

\vspace{0.4cm}

\textbf{Б} \quad (-\infty; 1) \cup (1; +\infty)

\vspace{0.4cm}

\textbf{В} \quad 83

\vspace{0.4cm}

\textbf{Г} \quad (-\infty; -1) \cup (-1; +\infty)

\vspace{0.4cm}

\textbf{Д} \quad (-\infty; 3]

\vspace{0.4cm}

}{\answerGrid}

\vspace{0.8cm}

\noindent\textbf{63.} Установіть відповідність між виразом (1--3) та його значенням (А--Д). \nmtyear{2026}
\vspace{0.3cm}

\matchingLayout{
\textit{Вираз}

\textbf{1} \quad Укажіть область визначення функції $y = \sqrt{x + 9}$.

\vspace{0.4cm}

\textbf{2} \quad Укажіть область визначення функції $y = \frac{1}{2 - x}$.

\vspace{0.4cm}

\textbf{3} \quad Укажіть область визначення функції $y = \sqrt{x + 4}$.

\vspace{0.4cm}

}{
\textit{Значення}

\textbf{А} \quad [-9; +\infty)

\vspace{0.4cm}

\textbf{Б} \quad [-4; +\infty)

\vspace{0.4cm}

\textbf{В} \quad (-\infty; 2) \cup (2; +\infty)

\vspace{0.4cm}

\textbf{Г} \quad 78

\vspace{0.4cm}

\textbf{Д} \quad [4; +\infty)

\vspace{0.4cm}

}{\answerGrid}

\vspace{0.8cm}

\noindent\textbf{64.} Установіть відповідність між виразом (1--3) та його значенням (А--Д). \nmtyear{2026}
\vspace{0.3cm}

\matchingLayout{
\textit{Вираз}

\textbf{1} \quad Укажіть область визначення функції $y = \log_{5} (7 - x)$.

\vspace{0.4cm}

\textbf{2} \quad Укажіть область визначення функції $y = \log_{0.5} (5 - x)$.

\vspace{0.4cm}

\textbf{3} \quad Укажіть область визначення функції $y = \frac{4}{x + 8}$.

\vspace{0.4cm}

}{
\textit{Значення}

\textbf{А} \quad (-\infty; 7)

\vspace{0.4cm}

\textbf{Б} \quad (-\infty; -8) \cup (-8; +\infty)

\vspace{0.4cm}

\textbf{В} \quad (7; +\infty)

\vspace{0.4cm}

\textbf{Г} \quad (-\infty; 5)

\vspace{0.4cm}

\textbf{Д} \quad (-\infty; 7]

\vspace{0.4cm}

}{\answerGrid}

\vspace{0.8cm}

\noindent\textbf{65.} Установіть відповідність між виразом (1--3) та його значенням (А--Д). \nmtyear{2026}
\vspace{0.3cm}

\matchingLayout{
\textit{Вираз}

\textbf{1} \quad Укажіть область визначення функції $y = \sqrt{x + 5}$.

\vspace{0.4cm}

\textbf{2} \quad Укажіть область визначення функції $y = \sqrt{x + 8}$.

\vspace{0.4cm}

\textbf{3} \quad Укажіть область визначення функції $y = \log_{5} (9 - x)$.

\vspace{0.4cm}

}{
\textit{Значення}

\textbf{А} \quad [-5; +\infty)

\vspace{0.4cm}

\textbf{Б} \quad (9; +\infty)

\vspace{0.4cm}

\textbf{В} \quad [-8; +\infty)

\vspace{0.4cm}

\textbf{Г} \quad 33

\vspace{0.4cm}

\textbf{Д} \quad (-\infty; 9)

\vspace{0.4cm}

}{\answerGrid}

\vspace{0.8cm}

\noindent\textbf{66.} Установіть відповідність між виразом (1--3) та його значенням (А--Д). \nmtyear{2026}
\vspace{0.3cm}

\matchingLayout{
\textit{Вираз}

\textbf{1} \quad Укажіть область визначення функції $y = \log_{2} (x + 5)$.

\vspace{0.4cm}

\textbf{2} \quad Укажіть область визначення функції $y = \log_{5} (3 - x)$.

\vspace{0.4cm}

\textbf{3} \quad Укажіть область визначення функції $y = \frac{4}{6 - x}$.

\vspace{0.4cm}

}{
\textit{Значення}

\textbf{А} \quad (-\infty; 6) \cup (6; +\infty)

\vspace{0.4cm}

\textbf{Б} \quad (-\infty; 3]

\vspace{0.4cm}

\textbf{В} \quad (5; +\infty)

\vspace{0.4cm}

\textbf{Г} \quad (-5; +\infty)

\vspace{0.4cm}

\textbf{Д} \quad (-\infty; 3)

\vspace{0.4cm}

}{\answerGrid}

\vspace{0.8cm}

\noindent\textbf{67.} Установіть відповідність між виразом (1--3) та його значенням (А--Д). \nmtyear{2026}
\vspace{0.3cm}

\matchingLayout{
\textit{Вираз}

\textbf{1} \quad Укажіть область визначення функції $y = \sqrt{9 - x}$.

\vspace{0.4cm}

\textbf{2} \quad Укажіть область визначення функції $y = \log_{5} (x + 5)$.

\vspace{0.4cm}

\textbf{3} \quad Укажіть область визначення функції $y = \frac{5}{9 - x}$.

\vspace{0.4cm}

}{
\textit{Значення}

\textbf{А} \quad (-5; +\infty)

\vspace{0.4cm}

\textbf{Б} \quad (-\infty; 9]

\vspace{0.4cm}

\textbf{В} \quad (-\infty; -9]

\vspace{0.4cm}

\textbf{Г} \quad (-\infty; 9) \cup (9; +\infty)

\vspace{0.4cm}

\textbf{Д} \quad [-5; +\infty)

\vspace{0.4cm}

}{\answerGrid}

\vspace{0.8cm}

\noindent\textbf{68.} Установіть відповідність між виразом (1--3) та його значенням (А--Д). \nmtyear{2026}
\vspace{0.3cm}

\matchingLayout{
\textit{Вираз}

\textbf{1} \quad Укажіть область визначення функції $y = \log_{5} (x + 1)$.

\vspace{0.4cm}

\textbf{2} \quad Укажіть область визначення функції $y = \sqrt{x + 9}$.

\vspace{0.4cm}

\textbf{3} \quad Укажіть область визначення функції $y = \sqrt{3 - x}$.

\vspace{0.4cm}

}{
\textit{Значення}

\textbf{А} \quad (1; +\infty)

\vspace{0.4cm}

\textbf{Б} \quad (-9; +\infty)

\vspace{0.4cm}

\textbf{В} \quad (-\infty; 3]

\vspace{0.4cm}

\textbf{Г} \quad (-1; +\infty)

\vspace{0.4cm}

\textbf{Д} \quad [-9; +\infty)

\vspace{0.4cm}

}{\answerGrid}

\vspace{0.8cm}

\noindent\textbf{69.} Установіть відповідність між виразом (1--3) та його значенням (А--Д). \nmtyear{2026}
\vspace{0.3cm}

\matchingLayout{
\textit{Вираз}

\textbf{1} \quad Укажіть область визначення функції $y = \frac{2}{x + 6}$.

\vspace{0.4cm}

\textbf{2} \quad Укажіть область визначення функції $y = \frac{5}{x + 6}$.

\vspace{0.4cm}

\textbf{3} \quad Укажіть область визначення функції $y = \sqrt{x + 7}$.

\vspace{0.4cm}

}{
\textit{Значення}

\textbf{А} \quad (-\infty; -7]

\vspace{0.4cm}

\textbf{Б} \quad (-\infty; -6) \cup (-6; +\infty)

\vspace{0.4cm}

\textbf{В} \quad (-\infty; -6) \cup (-6; +\infty)

\vspace{0.4cm}

\textbf{Г} \quad 17

\vspace{0.4cm}

\textbf{Д} \quad [-7; +\infty)

\vspace{0.4cm}

}{\answerGrid}

\vspace{0.8cm}

\noindent\textbf{70.} Установіть відповідність між виразом (1--3) та його значенням (А--Д). \nmtyear{2026}
\vspace{0.3cm}

\matchingLayout{
\textit{Вираз}

\textbf{1} \quad Укажіть область визначення функції $y = \sqrt{8 - x}$.

\vspace{0.4cm}

\textbf{2} \quad Укажіть область визначення функції $y = \log_{3} (x + 4)$.

\vspace{0.4cm}

\textbf{3} \quad Укажіть область визначення функції $y = \sqrt{x + 8}$.

\vspace{0.4cm}

}{
\textit{Значення}

\textbf{А} \quad (-4; +\infty)

\vspace{0.4cm}

\textbf{Б} \quad (-\infty; 8]

\vspace{0.4cm}

\textbf{В} \quad [-8; +\infty)

\vspace{0.4cm}

\textbf{Г} \quad 100

\vspace{0.4cm}

\textbf{Д} \quad [-4; +\infty)

\vspace{0.4cm}

}{\answerGrid}

\vspace{0.8cm}

\noindent\textbf{71.} Установіть відповідність між виразом (1--3) та його значенням (А--Д). \nmtyear{2026}
\vspace{0.3cm}

\matchingLayout{
\textit{Вираз}

\textbf{1} \quad Укажіть область визначення функції $y = \frac{5}{x + 6}$.

\vspace{0.4cm}

\textbf{2} \quad Укажіть область визначення функції $y = \frac{4}{x + 9}$.

\vspace{0.4cm}

\textbf{3} \quad Укажіть область визначення функції $y = \frac{5}{7 - x}$.

\vspace{0.4cm}

}{
\textit{Значення}

\textbf{А} \quad (7; +\infty)

\vspace{0.4cm}

\textbf{Б} \quad 94

\vspace{0.4cm}

\textbf{В} \quad (-\infty; -6) \cup (-6; +\infty)

\vspace{0.4cm}

\textbf{Г} \quad (-\infty; 7) \cup (7; +\infty)

\vspace{0.4cm}

\textbf{Д} \quad (-\infty; -9) \cup (-9; +\infty)

\vspace{0.4cm}

}{\answerGrid}

\vspace{0.8cm}

\noindent\textbf{72.} Установіть відповідність між виразом (1--3) та його значенням (А--Д). \nmtyear{2026}
\vspace{0.3cm}

\matchingLayout{
\textit{Вираз}

\textbf{1} \quad Укажіть область визначення функції $y = \log_{3} (4 - x)$.

\vspace{0.4cm}

\textbf{2} \quad Укажіть область визначення функції $y = \sqrt{x + 8}$.

\vspace{0.4cm}

\textbf{3} \quad Укажіть область визначення функції $y = \log_{2} (x + 8)$.

\vspace{0.4cm}

}{
\textit{Значення}

\textbf{А} \quad [-8; +\infty)

\vspace{0.4cm}

\textbf{Б} \quad (-\infty; -4)

\vspace{0.4cm}

\textbf{В} \quad (-8; +\infty)

\vspace{0.4cm}

\textbf{Г} \quad (-\infty; -8]

\vspace{0.4cm}

\textbf{Д} \quad (-\infty; 4)

\vspace{0.4cm}

}{\answerGrid}

\vspace{0.8cm}

\noindent\textbf{73.} Установіть відповідність між виразом (1--3) та його значенням (А--Д). \nmtyear{2026}
\vspace{0.3cm}

\matchingLayout{
\textit{Вираз}

\textbf{1} \quad Укажіть область визначення функції $y = \frac{2}{7 - x}$.

\vspace{0.4cm}

\textbf{2} \quad Укажіть область визначення функції $y = \frac{5}{8 - x}$.

\vspace{0.4cm}

\textbf{3} \quad Укажіть область визначення функції $y = \log_{0.5} (6 - x)$.

\vspace{0.4cm}

}{
\textit{Значення}

\textbf{А} \quad (-\infty; 6]

\vspace{0.4cm}

\textbf{Б} \quad (6; +\infty)

\vspace{0.4cm}

\textbf{В} \quad (-\infty; 8) \cup (8; +\infty)

\vspace{0.4cm}

\textbf{Г} \quad (-\infty; 7) \cup (7; +\infty)

\vspace{0.4cm}

\textbf{Д} \quad (-\infty; 6)

\vspace{0.4cm}

}{\answerGrid}

\vspace{0.8cm}

\noindent\textbf{74.} Установіть відповідність між виразом (1--3) та його значенням (А--Д). \nmtyear{2026}
\vspace{0.3cm}

\matchingLayout{
\textit{Вираз}

\textbf{1} \quad Укажіть область визначення функції $y = \sqrt{3 - x}$.

\vspace{0.4cm}

\textbf{2} \quad Укажіть область визначення функції $y = \frac{3}{x + 2}$.

\vspace{0.4cm}

\textbf{3} \quad Укажіть область визначення функції $y = \log_{3} (9 - x)$.

\vspace{0.4cm}

}{
\textit{Значення}

\textbf{А} \quad (-\infty; -3]

\vspace{0.4cm}

\textbf{Б} \quad (-\infty; 3]

\vspace{0.4cm}

\textbf{В} \quad (-\infty; -2) \cup (-2; +\infty)

\vspace{0.4cm}

\textbf{Г} \quad (-\infty; 2) \cup (2; +\infty)

\vspace{0.4cm}

\textbf{Д} \quad (-\infty; 9)

\vspace{0.4cm}

}{\answerGrid}

\vspace{0.8cm}

\noindent\textbf{75.} Установіть відповідність між виразом (1--3) та його значенням (А--Д). \nmtyear{2026}
\vspace{0.3cm}

\matchingLayout{
\textit{Вираз}

\textbf{1} \quad Укажіть область визначення функції $y = \sqrt{4 - x}$.

\vspace{0.4cm}

\textbf{2} \quad Укажіть область визначення функції $y = \sqrt{7 - x}$.

\vspace{0.4cm}

\textbf{3} \quad Укажіть область визначення функції $y = \frac{5}{7 - x}$.

\vspace{0.4cm}

}{
\textit{Значення}

\textbf{А} \quad 23

\vspace{0.4cm}

\textbf{Б} \quad (-\infty; 7) \cup (7; +\infty)

\vspace{0.4cm}

\textbf{В} \quad (-\infty; 4]

\vspace{0.4cm}

\textbf{Г} \quad (-\infty; 7]

\vspace{0.4cm}

\textbf{Д} \quad (-\infty; -7]

\vspace{0.4cm}

}{\answerGrid}

\vspace{0.8cm}

\noindent\textbf{76.} Установіть відповідність між виразом (1--3) та його значенням (А--Д). \nmtyear{2026}
\vspace{0.3cm}

\matchingLayout{
\textit{Вираз}

\textbf{1} \quad Укажіть область визначення функції $y = \log_{2} (6 - x)$.

\vspace{0.4cm}

\textbf{2} \quad Укажіть область визначення функції $y = \frac{1}{6 - x}$.

\vspace{0.4cm}

\textbf{3} \quad Укажіть область визначення функції $y = \frac{5}{x + 7}$.

\vspace{0.4cm}

}{
\textit{Значення}

\textbf{А} \quad (-\infty; -7) \cup (-7; +\infty)

\vspace{0.4cm}

\textbf{Б} \quad (-\infty; 6)

\vspace{0.4cm}

\textbf{В} \quad (-\infty; 6]

\vspace{0.4cm}

\textbf{Г} \quad (6; +\infty)

\vspace{0.4cm}

\textbf{Д} \quad (-\infty; 6) \cup (6; +\infty)

\vspace{0.4cm}

}{\answerGrid}

\vspace{0.8cm}

\noindent\textbf{77.} Установіть відповідність між виразом (1--3) та його значенням (А--Д). \nmtyear{2026}
\vspace{0.3cm}

\matchingLayout{
\textit{Вираз}

\textbf{1} \quad Укажіть область визначення функції $y = \frac{4}{x + 1}$.

\vspace{0.4cm}

\textbf{2} \quad Укажіть область визначення функції $y = \frac{2}{x + 3}$.

\vspace{0.4cm}

\textbf{3} \quad Укажіть область визначення функції $y = \log_{0.5} (x + 7)$.

\vspace{0.4cm}

}{
\textit{Значення}

\textbf{А} \quad (-7; +\infty)

\vspace{0.4cm}

\textbf{Б} \quad (-\infty; -1) \cup (-1; +\infty)

\vspace{0.4cm}

\textbf{В} \quad [-7; +\infty)

\vspace{0.4cm}

\textbf{Г} \quad (-\infty; -3) \cup (-3; +\infty)

\vspace{0.4cm}

\textbf{Д} \quad (7; +\infty)

\vspace{0.4cm}

}{\answerGrid}

\vspace{0.8cm}

\noindent\textbf{78.} Установіть відповідність між виразом (1--3) та його значенням (А--Д). \nmtyear{2026}
\vspace{0.3cm}

\matchingLayout{
\textit{Вираз}

\textbf{1} \quad Укажіть область визначення функції $y = \frac{4}{4 - x}$.

\vspace{0.4cm}

\textbf{2} \quad Укажіть область визначення функції $y = \sqrt{x + 8}$.

\vspace{0.4cm}

\textbf{3} \quad Укажіть область визначення функції $y = \frac{3}{6 - x}$.

\vspace{0.4cm}

}{
\textit{Значення}

\textbf{А} \quad (-\infty; 4) \cup (4; +\infty)

\vspace{0.4cm}

\textbf{Б} \quad (-\infty; 6) \cup (6; +\infty)

\vspace{0.4cm}

\textbf{В} \quad (4; +\infty)

\vspace{0.4cm}

\textbf{Г} \quad [-8; +\infty)

\vspace{0.4cm}

\textbf{Д} \quad (-\infty; 4)

\vspace{0.4cm}

}{\answerGrid}

\vspace{0.8cm}

\noindent\textbf{79.} Установіть відповідність між виразом (1--3) та його значенням (А--Д). \nmtyear{2026}
\vspace{0.3cm}

\matchingLayout{
\textit{Вираз}

\textbf{1} \quad Укажіть область визначення функції $y = \log_{3} (x + 4)$.

\vspace{0.4cm}

\textbf{2} \quad Укажіть область визначення функції $y = \log_{3} (9 - x)$.

\vspace{0.4cm}

\textbf{3} \quad Укажіть область визначення функції $y = \log_{0.5} (x + 5)$.

\vspace{0.4cm}

}{
\textit{Значення}

\textbf{А} \quad (-4; +\infty)

\vspace{0.4cm}

\textbf{Б} \quad (-\infty; -4)

\vspace{0.4cm}

\textbf{В} \quad (-\infty; 9)

\vspace{0.4cm}

\textbf{Г} \quad (5; +\infty)

\vspace{0.4cm}

\textbf{Д} \quad (-5; +\infty)

\vspace{0.4cm}

}{\answerGrid}

\vspace{0.8cm}

\noindent\textbf{80.} Установіть відповідність між виразом (1--3) та його значенням (А--Д). \nmtyear{2026}
\vspace{0.3cm}

\matchingLayout{
\textit{Вираз}

\textbf{1} \quad Укажіть область визначення функції $y = \frac{3}{x + 4}$.

\vspace{0.4cm}

\textbf{2} \quad Укажіть область визначення функції $y = \log_{3} (1 - x)$.

\vspace{0.4cm}

\textbf{3} \quad Укажіть область визначення функції $y = \sqrt{5 - x}$.

\vspace{0.4cm}

}{
\textit{Значення}

\textbf{А} \quad (-\infty; -4) \cup (-4; +\infty)

\vspace{0.4cm}

\textbf{Б} \quad (-\infty; 5]

\vspace{0.4cm}

\textbf{В} \quad (-\infty; 1)

\vspace{0.4cm}

\textbf{Г} \quad (-\infty; -5]

\vspace{0.4cm}

\textbf{Д} \quad [5; +\infty)

\vspace{0.4cm}

}{\answerGrid}

\vspace{0.8cm}


\end{document}
