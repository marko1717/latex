\documentclass[14pt]{extarticle}
\usepackage{fontspec}
\usepackage{polyglossia}
\setdefaultlanguage{ukrainian}

\defaultfontfeatures{Ligatures=TeX}
\setmainfont{Liberation Serif}
\setsansfont{Liberation Sans}
\setmonofont{Liberation Mono}

\usepackage[a4paper,margin=1.5cm,bottom=2cm,top=2cm]{geometry}
\usepackage{amsmath,amssymb}
\usepackage{enumitem}
\usepackage{tikz}
\usepackage{pgfplots}
\pgfplotsset{compat=1.16}

\usetikzlibrary{calc,patterns,angles,quotes,intersections,babel}
\usetikzlibrary{3d}
\definecolor{woodinner}{RGB}{222, 184, 135}
\definecolor{woodouter}{RGB}{139, 69, 19}
\usepackage{xcolor}
\usepackage{array}
\usepackage{fancyhdr}
\usepackage{multirow}

\definecolor{headerblue}{RGB}{0, 102, 204}
\definecolor{yearcolor}{RGB}{128, 0, 128}

\pagestyle{fancy}
\fancyhf{}
\renewcommand{\headrulewidth}{0pt}
\fancyfoot[C]{\thepage}

\setlength{\headheight}{15pt}
\setlength{\headsep}{10pt}
\setlength{\footskip}{25pt}

\widowpenalty=10000
\clubpenalty=10000

\newcommand{\answerTable}[5]{
\begin{center}
\begin{tabular}{|*{5}{>{\centering\arraybackslash}m{2.8cm}|}}
\hline
\rule[-0.3cm]{0pt}{0.8cm}\textbf{А} & \textbf{Б} & \textbf{В} & \textbf{Г} & \textbf{Д} \\
\hline
\rule[-0.4cm]{0pt}{1.0cm}#1 & \rule[-0.4cm]{0pt}{1.0cm}#2 & \rule[-0.4cm]{0pt}{1.0cm}#3 & \rule[-0.4cm]{0pt}{1.0cm}#4 & \rule[-0.4cm]{0pt}{1.0cm}#5 \\
\hline
\end{tabular}
\end{center}
}

\newcommand{\shortAnswer}{
\vspace{0.3cm}
\noindent\hspace{1cm}Відповідь: \framebox(18,18){}\framebox(18,18){}\framebox(18,18){}\framebox(18,18){}{,}\framebox(18,18){}\framebox(18,18){}\framebox(18,18){}
\vspace{0.5cm}
}

\newcommand{\nmtyear}[1]{\hfill{\small\color{yearcolor}(AI Gen)}}

\begin{document}

\begin{center}
{\Large\textbf{\color{headerblue}ЗГЕНЕРОВАНІ ЗАВДАННЯ (AI)}}
\end{center}

\begin{center}
{\large Тема: \textbf{Геометрична прогресія}}
\end{center}

\vspace{0.5cm}
% === Geometric Progression: Find Term (b_1, b_2 -> b_n) ===
\noindent\makebox[1.5em][l]{\textbf{1.}}\parbox[t]{\dimexpr\textwidth-1.5em}{У геометричній прогресії $(b_n)$ відомо, що $b_1 = 2$, $b_2 = 1$. Визначте $b_{7}$. \nmtyear{2026}}

\answerTable{128}{0{,}0156}{-0{,}0312}{0{,}0312}{0{,}0625}

\vspace{0.5cm}

\noindent\makebox[1.5em][l]{\textbf{2.}}\parbox[t]{\dimexpr\textwidth-1.5em}{У геометричній прогресії $(b_n)$ відомо, що $b_1 = 2$, $b_2 = 8$. Визначте $b_{3}$. \nmtyear{2026}}

\answerTable{32}{128}{14}{0{,}125}{8}

\vspace{0.5cm}

\noindent\makebox[1.5em][l]{\textbf{3.}}\parbox[t]{\dimexpr\textwidth-1.5em}{У геометричній прогресії $(b_n)$ відомо, що $b_1 = 4$, $b_2 = -2$. Визначте $b_{6}$. \nmtyear{2026}}

\answerTable{-0{,}125}{0{,}0625}{-128}{-26}{0{,}125}

\vspace{0.5cm}

\noindent\makebox[1.5em][l]{\textbf{4.}}\parbox[t]{\dimexpr\textwidth-1.5em}{У геометричній прогресії $(b_n)$ відомо, що $b_1 = 10$, $b_2 = -20$. Визначте $b_{6}$. \nmtyear{2026}}

\answerTable{640}{320}{-0{,}3125}{160}{-320}

\vspace{0.5cm}

\noindent\makebox[1.5em][l]{\textbf{5.}}\parbox[t]{\dimexpr\textwidth-1.5em}{У геометричній прогресії $(b_n)$ відомо, що $b_1 = 4$, $b_2 = 16$. Визначте $b_{4}$. \nmtyear{2026}}

\answerTable{-256}{256}{64}{1024}{0{,}0625}

\vspace{0.5cm}

% === Geometric Progression: Term Ratio ===
\noindent\makebox[1.5em][l]{\textbf{6.}}\parbox[t]{\dimexpr\textwidth-1.5em}{У геометричній прогресії $(b_n)$ відомо, що $b_1 = 32$, $b_2 = 64$. Обчисліть $\dfrac{b_{6}}{b_{7}}$. \nmtyear{2026}}

\answerTable{2}{-1024}{61}{0{,}5}{86}

\vspace{0.5cm}

\noindent\makebox[1.5em][l]{\textbf{7.}}\parbox[t]{\dimexpr\textwidth-1.5em}{У геометричній прогресії $(b_n)$ відомо, що $b_1 = 10$, $b_2 = 50$. Обчисліть $\dfrac{b_{4}}{b_{5}}$. \nmtyear{2026}}

\answerTable{0{,}2}{16}{70}{5}{-5000}

\vspace{0.5cm}

\noindent\makebox[1.5em][l]{\textbf{8.}}\parbox[t]{\dimexpr\textwidth-1.5em}{У геометричній прогресії $(b_n)$ відомо, що $b_1 = 10$, $b_2 = 20$. Обчисліть $\dfrac{b_{4}}{b_{5}}$. \nmtyear{2026}}

\answerTable{98}{2}{0{,}5}{-80}{87}

\vspace{0.5cm}

\noindent\makebox[1.5em][l]{\textbf{9.}}\parbox[t]{\dimexpr\textwidth-1.5em}{У геометричній прогресії $(b_n)$ відомо, що $b_1 = 32$, $b_2 = 64$. Обчисліть $\dfrac{b_{4}}{b_{5}}$. \nmtyear{2026}}

\answerTable{-256}{0{,}5}{69}{2}{98}

\vspace{0.5cm}

\noindent\makebox[1.5em][l]{\textbf{10.}}\parbox[t]{\dimexpr\textwidth-1.5em}{У геометричній прогресії $(b_n)$ відомо, що $b_1 = 4$, $b_2 = 0{,}8$. Обчисліть $\dfrac{b_{5}}{b_{8}}$. \nmtyear{2026}}

\answerTable{5}{0{,}2}{125{,}0}{0{,}008}{0{,}0063}

\vspace{0.5cm}

% === Geometric Progression: Formula ===
\noindent\makebox[1.5em][l]{\textbf{11.}}\parbox[t]{\dimexpr\textwidth-1.5em}{Послідовність задано формулою $n$-го члена $b_n = 2 \cdot 4^n$. Визначте 3-й член цієї послідовності. \nmtyear{2026}}

\answerTable{127}{256}{129}{128}{-128}

\vspace{0.5cm}

\noindent\makebox[1.5em][l]{\textbf{12.}}\parbox[t]{\dimexpr\textwidth-1.5em}{Послідовність задано формулою $n$-го члена $b_n = (-1)^n \cdot n$. Визначте 6-й член цієї послідовності. \nmtyear{2026}}

\answerTable{6}{-6}{12}{7}{5}

\vspace{0.5cm}

\noindent\makebox[1.5em][l]{\textbf{13.}}\parbox[t]{\dimexpr\textwidth-1.5em}{Послідовність задано формулою $n$-го члена $b_n = \dfrac{(-1)^n}{n}$. Визначте 4-й член цієї послідовності. \nmtyear{2026}}

\answerTable{1{,}25}{0{,}5}{0{,}25}{-0{,}75}{-0{,}25}

\vspace{0.5cm}

\noindent\makebox[1.5em][l]{\textbf{14.}}\parbox[t]{\dimexpr\textwidth-1.5em}{Геометричну прогресію задано формулою $n$-го члена $b_n = 2 \cdot 3^{n-4}$. Визначте 6-й член цієї прогресії. \nmtyear{2026}}

\answerTable{17}{36}{-18}{19}{18}

\vspace{0.5cm}

\noindent\makebox[1.5em][l]{\textbf{15.}}\parbox[t]{\dimexpr\textwidth-1.5em}{Геометричну прогресію задано формулою $n$-го члена $b_n = 5 \cdot 3^{n-3}$. Визначте 6-й член цієї прогресії. \nmtyear{2026}}

\answerTable{134}{136}{270}{-135}{135}

\vspace{0.5cm}

% === Geometric Progression: Sum ===
\noindent\makebox[1.5em][l]{\textbf{16.}}\parbox[t]{\dimexpr\textwidth-1.5em}{Знайдіть суму 3 перших членів геометричної прогресії $(b_n)$, у якої $b_2 = -4$, а знаменник $q = -2$. \nmtyear{2026}}

\answerTable{6}{-26}{-6}{8}{-3}

\vspace{0.5cm}

\noindent\makebox[1.5em][l]{\textbf{17.}}\parbox[t]{\dimexpr\textwidth-1.5em}{Знайдіть суму 4 перших членів геометричної прогресії $(b_n)$, у якої $b_2 = 4$, а знаменник $q = -2$. \nmtyear{2026}}

\answerTable{10}{-10}{22}{16}{-5}

\vspace{0.5cm}

\noindent\makebox[1.5em][l]{\textbf{18.}}\parbox[t]{\dimexpr\textwidth-1.5em}{Знайдіть суму 4 перших членів геометричної прогресії $(b_n)$, у якої $b_2 = 6$, а знаменник $q = 0{,}5$. \nmtyear{2026}}

\answerTable{45}{1{,}5}{-22{,}5}{22{,}5}{11{,}5}

\vspace{0.5cm}

\noindent\makebox[1.5em][l]{\textbf{19.}}\parbox[t]{\dimexpr\textwidth-1.5em}{Знайдіть суму 5 перших членів геометричної прогресії $(b_n)$, у якої $b_2 = 8$, а знаменник $q = 2$. \nmtyear{2026}}

\answerTable{124}{-124}{62}{93}{64}

\vspace{0.5cm}

\noindent\makebox[1.5em][l]{\textbf{20.}}\parbox[t]{\dimexpr\textwidth-1.5em}{Знайдіть суму 5 перших членів геометричної прогресії $(b_n)$, у якої $b_2 = 6$, а знаменник $q = 0{,}5$. \nmtyear{2026}}

\answerTable{-23{,}25}{23{,}25}{0{,}75}{37{,}25}{46{,}5}

\vspace{0.5cm}

% === Geometric Progression: Word Problem ===
\noindent\makebox[1.5em][l]{\textbf{21.}}\parbox[t]{\dimexpr\textwidth-1.5em}{Бактерія ділиться. Першого дня колонія налічувала 10 бактерій. Кожного наступного дня кількість збільшувалася вдвічі. За яку \textit{найменшу} кількість днів сумарна кількість бактерій перевищить 2000? \nmtyear{2026}}

\answerTable{9}{10}{8}{7}{6}

\vspace{0.5cm}

\noindent\makebox[1.5em][l]{\textbf{22.}}\parbox[t]{\dimexpr\textwidth-1.5em}{Бактерія ділиться. Першого дня колонія налічувала 50 бактерій. Кожного наступного дня кількість збільшувалася вдвічі. За яку \textit{найменшу} кількість днів сумарна кількість бактерій перевищить 500? \nmtyear{2026}}

\answerTable{4}{5}{6}{2}{3}

\vspace{0.5cm}

\noindent\makebox[1.5em][l]{\textbf{23.}}\parbox[t]{\dimexpr\textwidth-1.5em}{Інвестор вклав гроші. Першого дня прибуток склав 5 доларів. Кожного наступного дня кількість збільшувалася вдвічі. За яку \textit{найменшу} кількість днів сумарна кількість доларів перевищить 500? \nmtyear{2026}}

\answerTable{8}{6}{7}{9}{5}

\vspace{0.5cm}

\noindent\makebox[1.5em][l]{\textbf{24.}}\parbox[t]{\dimexpr\textwidth-1.5em}{Бактерія ділиться. Першого дня колонія налічувала 5 бактерій. Кожного наступного дня кількість збільшувалася вдвічі. За яку \textit{найменшу} кількість днів сумарна кількість бактерій перевищить 1000? \nmtyear{2026}}

\answerTable{6}{10}{9}{7}{8}

\vspace{0.5cm}

\noindent\makebox[1.5em][l]{\textbf{25.}}\parbox[t]{\dimexpr\textwidth-1.5em}{Інвестор вклав гроші. Першого дня прибуток склав 100 доларів. Кожного наступного дня кількість збільшувалася вдвічі. За яку \textit{найменшу} кількість днів сумарна кількість доларів перевищить 2000? \nmtyear{2026}}

\answerTable{4}{6}{5}{3}{7}

\vspace{0.5cm}


\end{document}
