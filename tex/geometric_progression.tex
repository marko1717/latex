\documentclass[14pt]{extarticle}
\usepackage{fontspec}
\usepackage{polyglossia}
\setdefaultlanguage{ukrainian}

\defaultfontfeatures{Ligatures=TeX}
\setmainfont{Liberation Serif}
\setsansfont{Liberation Sans}
\setmonofont{Liberation Mono}

\usepackage[a4paper,margin=1.5cm,bottom=2cm,top=2cm]{geometry}
\usepackage{amsmath,amssymb}
\usepackage{enumitem}
\usepackage{tikz}
\usepackage{pgfplots}
\pgfplotsset{compat=1.18}

\usetikzlibrary{calc,patterns,angles,quotes,intersections,babel}
\usetikzlibrary{3d}

\usepackage{xcolor}
\usepackage{array}
\usepackage{fancyhdr}
\usepackage{multirow}

% Кольори
\definecolor{headerblue}{RGB}{0, 102, 204}
\definecolor{yearcolor}{RGB}{128, 0, 128}

\pagestyle{fancy}
\fancyhf{}
\renewcommand{\headrulewidth}{0pt}
\fancyfoot[C]{\thepage}

\setlength{\headheight}{15pt}
\setlength{\headsep}{10pt}
\setlength{\footskip}{25pt}

\widowpenalty=10000
\clubpenalty=10000

% === КОМАНДИ ===

% Таблиця відповідей для відповідностей
\newcommand{\answerGrid}{
    \begingroup
    \renewcommand{\arraystretch}{1.3} 
    \setlength{\tabcolsep}{7pt} 
    \begin{tabular}{r|c|c|c|c|c|}
         \multicolumn{1}{c}{} & \multicolumn{1}{c}{\textbf{А}} & \multicolumn{1}{c}{\textbf{Б}} & \multicolumn{1}{c}{\textbf{В}} & \multicolumn{1}{c}{\textbf{Г}} & \multicolumn{1}{c}{\textbf{Д}} \\ \cline{2-6}
         \textbf{1} & & & & & \\ \cline{2-6}
         \textbf{2} & & & & & \\ \cline{2-6}
         \textbf{3} & & & & & \\ \cline{2-6}
    \end{tabular}
    \endgroup
}

% Макет для завдань на відповідність
\newcommand{\matchingLayout}[3]{
    \noindent
    \begin{minipage}[t]{0.40\textwidth}
        #1
    \end{minipage}%
    \hfill
    \begin{minipage}[t]{0.28\textwidth}
        #2
    \end{minipage}%
    \hfill
    \begin{minipage}[t]{0.30\textwidth}
        \vspace{0pt}
        \begin{flushright}
        #3
        \end{flushright}
    \end{minipage}
}

% Стандартна таблиця відповідей
\newcommand{\answerTable}[5]{
\begin{center}
\begin{tabular}{|*{5}{>{\centering\arraybackslash}m{2.8cm}|}}
\hline
\rule[-0.3cm]{0pt}{0.8cm}\textbf{А} & \textbf{Б} & \textbf{В} & \textbf{Г} & \textbf{Д} \\
\hline
\rule[-0.4cm]{0pt}{1.0cm}#1 & \rule[-0.4cm]{0pt}{1.0cm}#2 & \rule[-0.4cm]{0pt}{1.0cm}#3 & \rule[-0.4cm]{0pt}{1.0cm}#4 & \rule[-0.4cm]{0pt}{1.0cm}#5 \\
\hline
\end{tabular}
\end{center}
}

% Таблиця для відповідей із дробами
\newcommand{\answerTableTall}[5]{
\begin{center}
\begin{tabular}{|*{5}{>{\centering\arraybackslash}m{2.8cm}|}}
\hline
\rule[-0.3cm]{0pt}{0.8cm}\textbf{А} & \textbf{Б} & \textbf{В} & \textbf{Г} & \textbf{Д} \\
\hline
\rule[-0.9cm]{0pt}{2.0cm}#1 & 
\rule[-0.9cm]{0pt}{2.0cm}#2 & 
\rule[-0.9cm]{0pt}{2.0cm}#3 & 
\rule[-0.9cm]{0pt}{2.0cm}#4 & 
\rule[-0.9cm]{0pt}{2.0cm}#5 \\
\hline
\end{tabular}
\end{center}
}

\newcommand{\nmtyear}[1]{\hfill{\small\color{yearcolor}(AI Gen)}}

\begin{document}

\vspace{1cm}

\begin{center}
{\Large\textbf{\color{headerblue}ЗГЕНЕРОВАНІ ЗАВДАННЯ (AI)}}
\end{center}

\begin{center}
{\large Тема: \textbf{Геометрична прогресія}}
\end{center}

\vspace{0.5cm}
% === Geometric Progression: Find Term (b_1, b_2 -> b_n) ===
\noindent\makebox[1.5em][l]{\textbf{1.}}\parbox[t]{\dimexpr\textwidth-1.5em}{У геометричній прогресії $(b_n)$ відомо, що $b_1 = 3$, $b_2 = 0{,}75$. Визначте $b_{7}$. \nmtyear{2026}}
\vspace{0.3cm}

\answerTable{0{,}0029}{0{,}0007}{12288}{-10{,}5}{0{,}0002}

\vspace{0.5cm}

\noindent\makebox[1.5em][l]{\textbf{2.}}\parbox[t]{\dimexpr\textwidth-1.5em}{У геометричній прогресії $(b_n)$ відомо, що $b_1 = 32$, $b_2 = 96$. Визначте $b_{4}$. \nmtyear{2026}}
\vspace{0.3cm}

\answerTable{2592}{288}{864}{-864}{1{,}1852}

\vspace{0.5cm}

\noindent\makebox[1.5em][l]{\textbf{3.}}\parbox[t]{\dimexpr\textwidth-1.5em}{У геометричній прогресії $(b_n)$ відомо, що $b_1 = 10$, $b_2 = -20$. Визначте $b_{7}$. \nmtyear{2026}}
\vspace{0.3cm}

\answerTable{-1280}{-170}{640}{0{,}1562}{-640}

\vspace{0.5cm}

\noindent\makebox[1.5em][l]{\textbf{4.}}\parbox[t]{\dimexpr\textwidth-1.5em}{У геометричній прогресії $(b_n)$ відомо, що $b_1 = 81$, $b_2 = 324$. Визначте $b_{5}$. \nmtyear{2026}}
\vspace{0.3cm}

\answerTable{5184}{0{,}3164}{-20736}{20736}{1053}

\vspace{0.5cm}

\noindent\makebox[1.5em][l]{\textbf{5.}}\parbox[t]{\dimexpr\textwidth-1.5em}{У геометричній прогресії $(b_n)$ відомо, що $b_1 = 10$, $b_2 = -5$. Визначте $b_{5}$. \nmtyear{2026}}
\vspace{0.3cm}

\answerTable{-0{,}3125}{-1{,}25}{160}{0{,}625}{-50}

\vspace{0.5cm}

\noindent\makebox[1.5em][l]{\textbf{6.}}\parbox[t]{\dimexpr\textwidth-1.5em}{У геометричній прогресії $(b_n)$ відомо, що $b_1 = 4$, $b_2 = -8$. Визначте $b_{6}$. \nmtyear{2026}}
\vspace{0.3cm}

\answerTable{-128}{-56}{128}{-0{,}125}{64}

\vspace{0.5cm}

\noindent\makebox[1.5em][l]{\textbf{7.}}\parbox[t]{\dimexpr\textwidth-1.5em}{У геометричній прогресії $(b_n)$ відомо, що $b_1 = 3$, $b_2 = 12$. Визначте $b_{3}$. \nmtyear{2026}}
\vspace{0.3cm}

\answerTable{21}{12}{48}{192}{-48}

\vspace{0.5cm}

\noindent\makebox[1.5em][l]{\textbf{8.}}\parbox[t]{\dimexpr\textwidth-1.5em}{У геометричній прогресії $(b_n)$ відомо, що $b_1 = 5$, $b_2 = 15$. Визначте $b_{4}$. \nmtyear{2026}}
\vspace{0.3cm}

\answerTable{45}{135}{0{,}1852}{405}{-135}

\vspace{0.5cm}

\noindent\makebox[1.5em][l]{\textbf{9.}}\parbox[t]{\dimexpr\textwidth-1.5em}{У геометричній прогресії $(b_n)$ відомо, що $b_1 = 5$, $b_2 = 1{,}25$. Визначте $b_{4}$. \nmtyear{2026}}
\vspace{0.3cm}

\answerTable{0{,}0195}{0{,}0781}{320}{-0{,}0781}{-6{,}25}

\vspace{0.5cm}

\noindent\makebox[1.5em][l]{\textbf{10.}}\parbox[t]{\dimexpr\textwidth-1.5em}{У геометричній прогресії $(b_n)$ відомо, що $b_1 = 4$, $b_2 = 16$. Визначте $b_{4}$. \nmtyear{2026}}
\vspace{0.3cm}

\answerTable{-256}{1024}{64}{256}{0{,}0625}

\vspace{0.5cm}

\noindent\makebox[1.5em][l]{\textbf{11.}}\parbox[t]{\dimexpr\textwidth-1.5em}{У геометричній прогресії $(b_n)$ відомо, що $b_1 = 10$, $b_2 = 5$. Визначте $b_{4}$. \nmtyear{2026}}
\vspace{0.3cm}

\answerTable{1{,}25}{-5}{0{,}625}{80}{2{,}5}

\vspace{0.5cm}

\noindent\makebox[1.5em][l]{\textbf{12.}}\parbox[t]{\dimexpr\textwidth-1.5em}{У геометричній прогресії $(b_n)$ відомо, що $b_1 = 3$, $b_2 = -1{,}5$. Визначте $b_{5}$. \nmtyear{2026}}
\vspace{0.3cm}

\answerTable{-0{,}0938}{-0{,}1875}{0{,}1875}{48}{-15}

\vspace{0.5cm}

\noindent\makebox[1.5em][l]{\textbf{13.}}\parbox[t]{\dimexpr\textwidth-1.5em}{У геометричній прогресії $(b_n)$ відомо, що $b_1 = 10$, $b_2 = 5$. Визначте $b_{6}$. \nmtyear{2026}}
\vspace{0.3cm}

\answerTable{-0{,}3125}{0{,}625}{0{,}3125}{-15}{0{,}1562}

\vspace{0.5cm}

\noindent\makebox[1.5em][l]{\textbf{14.}}\parbox[t]{\dimexpr\textwidth-1.5em}{У геометричній прогресії $(b_n)$ відомо, що $b_1 = 2$, $b_2 = 8$. Визначте $b_{4}$. \nmtyear{2026}}
\vspace{0.3cm}

\answerTable{128}{0{,}0312}{-128}{512}{32}

\vspace{0.5cm}

\noindent\makebox[1.5em][l]{\textbf{15.}}\parbox[t]{\dimexpr\textwidth-1.5em}{У геометричній прогресії $(b_n)$ відомо, що $b_1 = 27$, $b_2 = -54$. Визначте $b_{4}$. \nmtyear{2026}}
\vspace{0.3cm}

\answerTable{432}{108}{-3{,}375}{216}{-216}

\vspace{0.5cm}

\noindent\makebox[1.5em][l]{\textbf{16.}}\parbox[t]{\dimexpr\textwidth-1.5em}{У геометричній прогресії $(b_n)$ відомо, що $b_1 = 27$, $b_2 = 81$. Визначте $b_{4}$. \nmtyear{2026}}
\vspace{0.3cm}

\answerTable{243}{2187}{-729}{729}{189}

\vspace{0.5cm}

\noindent\makebox[1.5em][l]{\textbf{17.}}\parbox[t]{\dimexpr\textwidth-1.5em}{У геометричній прогресії $(b_n)$ відомо, що $b_1 = 10$, $b_2 = -5$. Визначте $b_{4}$. \nmtyear{2026}}
\vspace{0.3cm}

\answerTable{0{,}625}{-80}{1{,}25}{-35}{-1{,}25}

\vspace{0.5cm}

\noindent\makebox[1.5em][l]{\textbf{18.}}\parbox[t]{\dimexpr\textwidth-1.5em}{У геометричній прогресії $(b_n)$ відомо, що $b_1 = 4$, $b_2 = -8$. Визначте $b_{4}$. \nmtyear{2026}}
\vspace{0.3cm}

\answerTable{16}{-32}{-0{,}5}{32}{64}

\vspace{0.5cm}

\noindent\makebox[1.5em][l]{\textbf{19.}}\parbox[t]{\dimexpr\textwidth-1.5em}{У геометричній прогресії $(b_n)$ відомо, що $b_1 = 16$, $b_2 = -32$. Визначте $b_{6}$. \nmtyear{2026}}
\vspace{0.3cm}

\answerTable{-0{,}5}{-224}{-512}{512}{1024}

\vspace{0.5cm}

\noindent\makebox[1.5em][l]{\textbf{20.}}\parbox[t]{\dimexpr\textwidth-1.5em}{У геометричній прогресії $(b_n)$ відомо, що $b_1 = 64$, $b_2 = 16$. Визначте $b_{4}$. \nmtyear{2026}}
\vspace{0.3cm}

\answerTable{-1}{1}{4}{-80}{0{,}25}

\vspace{0.5cm}

% === Geometric Progression: Term Ratio ===
\noindent\makebox[1.5em][l]{\textbf{21.}}\parbox[t]{\dimexpr\textwidth-1.5em}{У геометричній прогресії $(b_n)$ відомо, що $b_1 = 2$, $b_2 = 1$. Обчисліть $\dfrac{b_{5}}{b_{6}}$. \nmtyear{2026}}
\vspace{0.3cm}

\answerTable{11}{7}{0{,}5}{0{,}0625}{2}

\vspace{0.5cm}

\noindent\makebox[1.5em][l]{\textbf{22.}}\parbox[t]{\dimexpr\textwidth-1.5em}{У геометричній прогресії $(b_n)$ відомо, що $b_1 = 5$, $b_2 = 10$. Обчисліть $\dfrac{b_{8}}{b_{10}}$. \nmtyear{2026}}
\vspace{0.3cm}

\answerTable{0{,}25}{-1920}{4}{0{,}5}{2}

\vspace{0.5cm}

\noindent\makebox[1.5em][l]{\textbf{23.}}\parbox[t]{\dimexpr\textwidth-1.5em}{У геометричній прогресії $(b_n)$ відомо, що $b_1 = 32$, $b_2 = 128$. Обчисліть $\dfrac{b_{8}}{b_{10}}$. \nmtyear{2026}}
\vspace{0.3cm}

\answerTable{16}{4}{0{,}0625}{0{,}25}{-7864320}

\vspace{0.5cm}

\noindent\makebox[1.5em][l]{\textbf{24.}}\parbox[t]{\dimexpr\textwidth-1.5em}{У геометричній прогресії $(b_n)$ відомо, що $b_1 = 2$, $b_2 = 20$. Обчисліть $\dfrac{b_{3}}{b_{4}}$. \nmtyear{2026}}
\vspace{0.3cm}

\answerTable{0{,}1}{-1800}{32}{10}{52}

\vspace{0.5cm}

\noindent\makebox[1.5em][l]{\textbf{25.}}\parbox[t]{\dimexpr\textwidth-1.5em}{У геометричній прогресії $(b_n)$ відомо, що $b_1 = 10$, $b_2 = 5$. Обчисліть $\dfrac{b_{3}}{b_{5}}$. \nmtyear{2026}}
\vspace{0.3cm}

\answerTable{2}{1{,}875}{4}{0{,}25}{0{,}5}

\vspace{0.5cm}

\noindent\makebox[1.5em][l]{\textbf{26.}}\parbox[t]{\dimexpr\textwidth-1.5em}{У геометричній прогресії $(b_n)$ відомо, що $b_1 = 2$, $b_2 = 0{,}4$. Обчисліть $\dfrac{b_{3}}{b_{6}}$. \nmtyear{2026}}
\vspace{0.3cm}

\answerTable{0{,}0794}{5}{0{,}2}{125{,}0}{0{,}008}

\vspace{0.5cm}

\noindent\makebox[1.5em][l]{\textbf{27.}}\parbox[t]{\dimexpr\textwidth-1.5em}{У геометричній прогресії $(b_n)$ відомо, що $b_1 = 5$, $b_2 = 15$. Обчисліть $\dfrac{b_{8}}{b_{10}}$. \nmtyear{2026}}
\vspace{0.3cm}

\answerTable{0{,}3333}{-87480}{0{,}1111}{9}{3}

\vspace{0.5cm}

\noindent\makebox[1.5em][l]{\textbf{28.}}\parbox[t]{\dimexpr\textwidth-1.5em}{У геометричній прогресії $(b_n)$ відомо, що $b_1 = 10$, $b_2 = 20$. Обчисліть $\dfrac{b_{8}}{b_{10}}$. \nmtyear{2026}}
\vspace{0.3cm}

\answerTable{0{,}5}{2}{4}{-3840}{0{,}25}

\vspace{0.5cm}

\noindent\makebox[1.5em][l]{\textbf{29.}}\parbox[t]{\dimexpr\textwidth-1.5em}{У геометричній прогресії $(b_n)$ відомо, що $b_1 = 32$, $b_2 = 64$. Обчисліть $\dfrac{b_{7}}{b_{9}}$. \nmtyear{2026}}
\vspace{0.3cm}

\answerTable{-6144}{2}{0{,}25}{0{,}5}{4}

\vspace{0.5cm}

\noindent\makebox[1.5em][l]{\textbf{30.}}\parbox[t]{\dimexpr\textwidth-1.5em}{У геометричній прогресії $(b_n)$ відомо, що $b_1 = 10$, $b_2 = 5$. Обчисліть $\dfrac{b_{4}}{b_{7}}$. \nmtyear{2026}}
\vspace{0.3cm}

\answerTable{1{,}0938}{0{,}125}{2}{0{,}5}{8}

\vspace{0.5cm}

\noindent\makebox[1.5em][l]{\textbf{31.}}\parbox[t]{\dimexpr\textwidth-1.5em}{У геометричній прогресії $(b_n)$ відомо, що $b_1 = 10$, $b_2 = 20$. Обчисліть $\dfrac{b_{5}}{b_{7}}$. \nmtyear{2026}}
\vspace{0.3cm}

\answerTable{-480}{0{,}5}{0{,}25}{2}{4}

\vspace{0.5cm}

\noindent\makebox[1.5em][l]{\textbf{32.}}\parbox[t]{\dimexpr\textwidth-1.5em}{У геометричній прогресії $(b_n)$ відомо, що $b_1 = 10$, $b_2 = 40$. Обчисліть $\dfrac{b_{4}}{b_{7}}$. \nmtyear{2026}}
\vspace{0.3cm}

\answerTable{4}{-40320}{64}{0{,}0156}{0{,}25}

\vspace{0.5cm}

\noindent\makebox[1.5em][l]{\textbf{33.}}\parbox[t]{\dimexpr\textwidth-1.5em}{У геометричній прогресії $(b_n)$ відомо, що $b_1 = 10$, $b_2 = 30$. Обчисліть $\dfrac{b_{5}}{b_{8}}$. \nmtyear{2026}}
\vspace{0.3cm}

\answerTable{0{,}037}{-21060}{0{,}3333}{3}{27}

\vspace{0.5cm}

\noindent\makebox[1.5em][l]{\textbf{34.}}\parbox[t]{\dimexpr\textwidth-1.5em}{У геометричній прогресії $(b_n)$ відомо, що $b_1 = 32$, $b_2 = 160$. Обчисліть $\dfrac{b_{6}}{b_{9}}$. \nmtyear{2026}}
\vspace{0.3cm}

\answerTable{5}{0{,}008}{-12400000}{0{,}2}{125}

\vspace{0.5cm}

\noindent\makebox[1.5em][l]{\textbf{35.}}\parbox[t]{\dimexpr\textwidth-1.5em}{У геометричній прогресії $(b_n)$ відомо, що $b_1 = 10$, $b_2 = 100$. Обчисліть $\dfrac{b_{3}}{b_{4}}$. \nmtyear{2026}}
\vspace{0.3cm}

\answerTable{-9000}{10}{0{,}1}{89}{74}

\vspace{0.5cm}

\noindent\makebox[1.5em][l]{\textbf{36.}}\parbox[t]{\dimexpr\textwidth-1.5em}{У геометричній прогресії $(b_n)$ відомо, що $b_1 = 10$, $b_2 = 5$. Обчисліть $\dfrac{b_{4}}{b_{7}}$. \nmtyear{2026}}
\vspace{0.3cm}

\answerTable{0{,}125}{1{,}0938}{2}{8}{0{,}5}

\vspace{0.5cm}

\noindent\makebox[1.5em][l]{\textbf{37.}}\parbox[t]{\dimexpr\textwidth-1.5em}{У геометричній прогресії $(b_n)$ відомо, що $b_1 = 32$, $b_2 = 160$. Обчисліть $\dfrac{b_{5}}{b_{8}}$. \nmtyear{2026}}
\vspace{0.3cm}

\answerTable{5}{0{,}008}{0{,}2}{-2480000}{125}

\vspace{0.5cm}

\noindent\makebox[1.5em][l]{\textbf{38.}}\parbox[t]{\dimexpr\textwidth-1.5em}{У геометричній прогресії $(b_n)$ відомо, що $b_1 = 4$, $b_2 = 20$. Обчисліть $\dfrac{b_{6}}{b_{7}}$. \nmtyear{2026}}
\vspace{0.3cm}

\answerTable{47}{46}{-50000}{0{,}2}{5}

\vspace{0.5cm}

\noindent\makebox[1.5em][l]{\textbf{39.}}\parbox[t]{\dimexpr\textwidth-1.5em}{У геометричній прогресії $(b_n)$ відомо, що $b_1 = 2$, $b_2 = 8$. Обчисліть $\dfrac{b_{8}}{b_{10}}$. \nmtyear{2026}}
\vspace{0.3cm}

\answerTable{-491520}{4}{0{,}0625}{0{,}25}{16}

\vspace{0.5cm}

\noindent\makebox[1.5em][l]{\textbf{40.}}\parbox[t]{\dimexpr\textwidth-1.5em}{У геометричній прогресії $(b_n)$ відомо, що $b_1 = 10$, $b_2 = 100$. Обчисліть $\dfrac{b_{5}}{b_{6}}$. \nmtyear{2026}}
\vspace{0.3cm}

\answerTable{21}{42}{10}{-900000}{0{,}1}

\vspace{0.5cm}

% === Geometric Progression: Formula ===
\noindent\makebox[1.5em][l]{\textbf{41.}}\parbox[t]{\dimexpr\textwidth-1.5em}{Послідовність задано формулою $n$-го члена $b_n = 0{,}8 \cdot 2^n + 3n$. Визначте 6-й член цієї послідовності. \nmtyear{2026}}
\vspace{0.3cm}

\answerTable{69{,}2}{70{,}2}{138{,}4}{-69{,}2}{68{,}2}

\vspace{0.5cm}

\noindent\makebox[1.5em][l]{\textbf{42.}}\parbox[t]{\dimexpr\textwidth-1.5em}{Послідовність задано формулою $n$-го члена $b_n = 2 \cdot 2^n$. Визначте 3-й член цієї послідовності. \nmtyear{2026}}
\vspace{0.3cm}

\answerTable{32}{-16}{15}{17}{16}

\vspace{0.5cm}

\noindent\makebox[1.5em][l]{\textbf{43.}}\parbox[t]{\dimexpr\textwidth-1.5em}{Геометричну прогресію задано формулою $n$-го члена $b_n = 2 \cdot 4^{n-1}$. Визначте 3-й член цієї прогресії. \nmtyear{2026}}
\vspace{0.3cm}

\answerTable{-32}{31}{64}{32}{33}

\vspace{0.5cm}

\noindent\makebox[1.5em][l]{\textbf{44.}}\parbox[t]{\dimexpr\textwidth-1.5em}{Послідовність задано формулою $n$-го члена $b_n = 0{,}5 \cdot 2^n + 3n$. Визначте 4-й член цієї послідовності. \nmtyear{2026}}
\vspace{0.3cm}

\answerTable{21}{-20}{20}{40}{19}

\vspace{0.5cm}

\noindent\makebox[1.5em][l]{\textbf{45.}}\parbox[t]{\dimexpr\textwidth-1.5em}{Послідовність задано формулою $n$-го члена $b_n = (-1)^n \cdot n$. Визначте 4-й член цієї послідовності. \nmtyear{2026}}
\vspace{0.3cm}

\answerTable{3}{8}{-4}{4}{5}

\vspace{0.5cm}

\noindent\makebox[1.5em][l]{\textbf{46.}}\parbox[t]{\dimexpr\textwidth-1.5em}{Послідовність задано формулою $n$-го члена $b_n = 5 \cdot 2^n$. Визначте 6-й член цієї послідовності. \nmtyear{2026}}
\vspace{0.3cm}

\answerTable{640}{319}{320}{321}{-320}

\vspace{0.5cm}

\noindent\makebox[1.5em][l]{\textbf{47.}}\parbox[t]{\dimexpr\textwidth-1.5em}{Геометричну прогресію задано формулою $n$-го члена $b_n = 5 \cdot 3^{n-4}$. Визначте 3-й член цієї прогресії. \nmtyear{2026}}
\vspace{0.3cm}

\answerTable{1{,}667}{-1{,}667}{3{,}333}{0{,}667}{2{,}667}

\vspace{0.5cm}

\noindent\makebox[1.5em][l]{\textbf{48.}}\parbox[t]{\dimexpr\textwidth-1.5em}{Геометричну прогресію задано формулою $n$-го члена $b_n = 1 \cdot 3^{n-4}$. Визначте 3-й член цієї прогресії. \nmtyear{2026}}
\vspace{0.3cm}

\answerTable{-0{,}667}{0{,}333}{-0{,}333}{0{,}667}{1{,}333}

\vspace{0.5cm}

\noindent\makebox[1.5em][l]{\textbf{49.}}\parbox[t]{\dimexpr\textwidth-1.5em}{Геометричну прогресію задано формулою $n$-го члена $b_n = 0{,}8 \cdot 4^{n-1}$. Визначте 4-й член цієї прогресії. \nmtyear{2026}}
\vspace{0.3cm}

\answerTable{52{,}2}{51{,}2}{-51{,}2}{102{,}4}{50{,}2}

\vspace{0.5cm}

\noindent\makebox[1.5em][l]{\textbf{50.}}\parbox[t]{\dimexpr\textwidth-1.5em}{Послідовність задано формулою $n$-го члена $b_n = 0{,}5 \cdot 2^n + 3n$. Визначте 3-й член цієї послідовності. \nmtyear{2026}}
\vspace{0.3cm}

\answerTable{-13}{12}{26}{13}{14}

\vspace{0.5cm}

\noindent\makebox[1.5em][l]{\textbf{51.}}\parbox[t]{\dimexpr\textwidth-1.5em}{Послідовність задано формулою $n$-го члена $b_n = 3 \cdot 4^n + 2n$. Визначте 5-й член цієї послідовності. \nmtyear{2026}}
\vspace{0.3cm}

\answerTable{3083}{6164}{-3082}{3081}{3082}

\vspace{0.5cm}

\noindent\makebox[1.5em][l]{\textbf{52.}}\parbox[t]{\dimexpr\textwidth-1.5em}{Послідовність задано формулою $n$-го члена $b_n = (-1)^n \cdot n$. Визначте 5-й член цієї послідовності. \nmtyear{2026}}
\vspace{0.3cm}

\answerTable{5}{-4}{-5}{-10}{-6}

\vspace{0.5cm}

\noindent\makebox[1.5em][l]{\textbf{53.}}\parbox[t]{\dimexpr\textwidth-1.5em}{Послідовність задано формулою $n$-го члена $b_n = 1 \cdot 4^n + 3n$. Визначте 3-й член цієї послідовності. \nmtyear{2026}}
\vspace{0.3cm}

\answerTable{-73}{72}{146}{73}{74}

\vspace{0.5cm}

\noindent\makebox[1.5em][l]{\textbf{54.}}\parbox[t]{\dimexpr\textwidth-1.5em}{Послідовність задано формулою $n$-го члена $b_n = \dfrac{(-1)^n}{n}$. Визначте 6-й член цієї послідовності. \nmtyear{2026}}
\vspace{0.3cm}

\answerTable{0{,}167}{1{,}167}{0{,}333}{-0{,}833}{-0{,}167}

\vspace{0.5cm}

\noindent\makebox[1.5em][l]{\textbf{55.}}\parbox[t]{\dimexpr\textwidth-1.5em}{Послідовність задано формулою $n$-го члена $b_n = (-1)^n \cdot n$. Визначте 6-й член цієї послідовності. \nmtyear{2026}}
\vspace{0.3cm}

\answerTable{-6}{6}{5}{7}{12}

\vspace{0.5cm}

\noindent\makebox[1.5em][l]{\textbf{56.}}\parbox[t]{\dimexpr\textwidth-1.5em}{Послідовність задано формулою $n$-го члена $b_n = \dfrac{(-1)^n}{n}$. Визначте 3-й член цієї послідовності. \nmtyear{2026}}
\vspace{0.3cm}

\answerTable{-0{,}667}{0{,}333}{-1{,}333}{-0{,}333}{0{,}667}

\vspace{0.5cm}

\noindent\makebox[1.5em][l]{\textbf{57.}}\parbox[t]{\dimexpr\textwidth-1.5em}{Геометричну прогресію задано формулою $n$-го члена $b_n = 1 \cdot 2^{n-1}$. Визначте 5-й член цієї прогресії. \nmtyear{2026}}
\vspace{0.3cm}

\answerTable{17}{-16}{32}{16}{15}

\vspace{0.5cm}

\noindent\makebox[1.5em][l]{\textbf{58.}}\parbox[t]{\dimexpr\textwidth-1.5em}{Геометричну прогресію задано формулою $n$-го члена $b_n = 1 \cdot 3^{n-2}$. Визначте 6-й член цієї прогресії. \nmtyear{2026}}
\vspace{0.3cm}

\answerTable{81}{-81}{82}{162}{80}

\vspace{0.5cm}

\noindent\makebox[1.5em][l]{\textbf{59.}}\parbox[t]{\dimexpr\textwidth-1.5em}{Геометричну прогресію задано формулою $n$-го члена $b_n = 0{,}5 \cdot 2^{n-2}$. Визначте 5-й член цієї прогресії. \nmtyear{2026}}
\vspace{0.3cm}

\answerTable{3}{-4}{4}{5}{8}

\vspace{0.5cm}

\noindent\makebox[1.5em][l]{\textbf{60.}}\parbox[t]{\dimexpr\textwidth-1.5em}{Послідовність задано формулою $n$-го члена $b_n = (-1)^n \cdot n$. Визначте 4-й член цієї послідовності. \nmtyear{2026}}
\vspace{0.3cm}

\answerTable{3}{5}{-4}{8}{4}

\vspace{0.5cm}

% === Geometric Progression: Sum ===
\noindent\makebox[1.5em][l]{\textbf{61.}}\parbox[t]{\dimexpr\textwidth-1.5em}{Знайдіть суму 3 перших членів геометричної прогресії $(b_n)$, у якої $b_2 = 4$, а знаменник $q = 3$. \nmtyear{2026}}
\vspace{0.3cm}

\answerTable{59{,}333}{5{,}778}{12}{-17{,}333}{17{,}333}

\vspace{0.5cm}

\noindent\makebox[1.5em][l]{\textbf{62.}}\parbox[t]{\dimexpr\textwidth-1.5em}{Знайдіть суму 3 перших членів геометричної прогресії $(b_n)$, у якої $b_2 = 12$, а знаменник $q = -2$. \nmtyear{2026}}
\vspace{0.3cm}

\answerTable{9}{18}{-24}{-18}{-47}

\vspace{0.5cm}

\noindent\makebox[1.5em][l]{\textbf{63.}}\parbox[t]{\dimexpr\textwidth-1.5em}{Знайдіть суму 4 перших членів геометричної прогресії $(b_n)$, у якої $b_2 = 8$, а знаменник $q = 3$. \nmtyear{2026}}
\vspace{0.3cm}

\answerTable{72}{152{,}667}{35{,}556}{106{,}667}{-106{,}667}

\vspace{0.5cm}

\noindent\makebox[1.5em][l]{\textbf{64.}}\parbox[t]{\dimexpr\textwidth-1.5em}{Знайдіть суму 4 перших членів геометричної прогресії $(b_n)$, у якої $b_2 = -4$, а знаменник $q = 2$. \nmtyear{2026}}
\vspace{0.3cm}

\answerTable{-30}{-15}{30}{-16}{-12}

\vspace{0.5cm}

\noindent\makebox[1.5em][l]{\textbf{65.}}\parbox[t]{\dimexpr\textwidth-1.5em}{Знайдіть суму 5 перших членів геометричної прогресії $(b_n)$, у якої $b_2 = -6$, а знаменник $q = -2$. \nmtyear{2026}}
\vspace{0.3cm}

\answerTable{48}{-16{,}5}{-33}{16}{33}

\vspace{0.5cm}

\noindent\makebox[1.5em][l]{\textbf{66.}}\parbox[t]{\dimexpr\textwidth-1.5em}{Знайдіть суму 4 перших членів геометричної прогресії $(b_n)$, у якої $b_2 = 8$, а знаменник $q = 3$. \nmtyear{2026}}
\vspace{0.3cm}

\answerTable{72}{64{,}667}{106{,}667}{35{,}556}{-106{,}667}

\vspace{0.5cm}

\noindent\makebox[1.5em][l]{\textbf{67.}}\parbox[t]{\dimexpr\textwidth-1.5em}{Знайдіть суму 3 перших членів геометричної прогресії $(b_n)$, у якої $b_2 = -4$, а знаменник $q = -2$. \nmtyear{2026}}
\vspace{0.3cm}

\answerTable{6}{-4}{-6}{-3}{8}

\vspace{0.5cm}

\noindent\makebox[1.5em][l]{\textbf{68.}}\parbox[t]{\dimexpr\textwidth-1.5em}{Знайдіть суму 4 перших членів геометричної прогресії $(b_n)$, у якої $b_2 = 6$, а знаменник $q = 0{,}5$. \nmtyear{2026}}
\vspace{0.3cm}

\answerTable{22{,}5}{1{,}5}{45}{-22{,}5}{5{,}5}

\vspace{0.5cm}

\noindent\makebox[1.5em][l]{\textbf{69.}}\parbox[t]{\dimexpr\textwidth-1.5em}{Знайдіть суму 5 перших членів геометричної прогресії $(b_n)$, у якої $b_2 = 4$, а знаменник $q = -2$. \nmtyear{2026}}
\vspace{0.3cm}

\answerTable{11}{-32}{-22}{22}{-64}

\vspace{0.5cm}

\noindent\makebox[1.5em][l]{\textbf{70.}}\parbox[t]{\dimexpr\textwidth-1.5em}{Знайдіть суму 4 перших членів геометричної прогресії $(b_n)$, у якої $b_2 = -6$, а знаменник $q = 2$. \nmtyear{2026}}
\vspace{0.3cm}

\answerTable{45}{-24}{-22{,}5}{-45}{-88}

\vspace{0.5cm}

\noindent\makebox[1.5em][l]{\textbf{71.}}\parbox[t]{\dimexpr\textwidth-1.5em}{Знайдіть суму 3 перших членів геометричної прогресії $(b_n)$, у якої $b_2 = -6$, а знаменник $q = 3$. \nmtyear{2026}}
\vspace{0.3cm}

\answerTable{-18}{-26}{-17}{-8{,}667}{26}

\vspace{0.5cm}

\noindent\makebox[1.5em][l]{\textbf{72.}}\parbox[t]{\dimexpr\textwidth-1.5em}{Знайдіть суму 3 перших членів геометричної прогресії $(b_n)$, у якої $b_2 = -4$, а знаменник $q = 3$. \nmtyear{2026}}
\vspace{0.3cm}

\answerTable{-12}{17{,}333}{-17{,}333}{-5{,}778}{-12{,}333}

\vspace{0.5cm}

\noindent\makebox[1.5em][l]{\textbf{73.}}\parbox[t]{\dimexpr\textwidth-1.5em}{Знайдіть суму 4 перших членів геометричної прогресії $(b_n)$, у якої $b_2 = -4$, а знаменник $q = -3$. \nmtyear{2026}}
\vspace{0.3cm}

\answerTable{8{,}889}{22{,}333}{-26{,}667}{-36}{26{,}667}

\vspace{0.5cm}

\noindent\makebox[1.5em][l]{\textbf{74.}}\parbox[t]{\dimexpr\textwidth-1.5em}{Знайдіть суму 3 перших членів геометричної прогресії $(b_n)$, у якої $b_2 = 18$, а знаменник $q = 0{,}5$. \nmtyear{2026}}
\vspace{0.3cm}

\answerTable{126}{63}{57}{9}{-63}

\vspace{0.5cm}

\noindent\makebox[1.5em][l]{\textbf{75.}}\parbox[t]{\dimexpr\textwidth-1.5em}{Знайдіть суму 4 перших членів геометричної прогресії $(b_n)$, у якої $b_2 = 18$, а знаменник $q = -2$. \nmtyear{2026}}
\vspace{0.3cm}

\answerTable{-22{,}5}{79}{72}{45}{-45}

\vspace{0.5cm}

\noindent\makebox[1.5em][l]{\textbf{76.}}\parbox[t]{\dimexpr\textwidth-1.5em}{Знайдіть суму 5 перших членів геометричної прогресії $(b_n)$, у якої $b_2 = 18$, а знаменник $q = 2$. \nmtyear{2026}}
\vspace{0.3cm}

\answerTable{279}{144}{273}{-279}{139{,}5}

\vspace{0.5cm}

\noindent\makebox[1.5em][l]{\textbf{77.}}\parbox[t]{\dimexpr\textwidth-1.5em}{Знайдіть суму 5 перших членів геометричної прогресії $(b_n)$, у якої $b_2 = -4$, а знаменник $q = -3$. \nmtyear{2026}}
\vspace{0.3cm}

\answerTable{81{,}333}{-27{,}111}{-81{,}333}{108}{35{,}333}

\vspace{0.5cm}

\noindent\makebox[1.5em][l]{\textbf{78.}}\parbox[t]{\dimexpr\textwidth-1.5em}{Знайдіть суму 4 перших членів геометричної прогресії $(b_n)$, у якої $b_2 = 8$, а знаменник $q = 0{,}5$. \nmtyear{2026}}
\vspace{0.3cm}

\answerTable{2}{-30}{60}{30}{45}

\vspace{0.5cm}

\noindent\makebox[1.5em][l]{\textbf{79.}}\parbox[t]{\dimexpr\textwidth-1.5em}{Знайдіть суму 4 перших членів геометричної прогресії $(b_n)$, у якої $b_2 = 4$, а знаменник $q = -2$. \nmtyear{2026}}
\vspace{0.3cm}

\answerTable{16}{-5}{-10}{10}{20}

\vspace{0.5cm}

\noindent\makebox[1.5em][l]{\textbf{80.}}\parbox[t]{\dimexpr\textwidth-1.5em}{Знайдіть суму 4 перших членів геометричної прогресії $(b_n)$, у якої $b_2 = -6$, а знаменник $q = 0{,}5$. \nmtyear{2026}}
\vspace{0.3cm}

\answerTable{-65{,}5}{-1{,}5}{-22{,}5}{-45}{22{,}5}

\vspace{0.5cm}

% === Geometric Progression: Word Problem ===
\noindent\makebox[1.5em][l]{\textbf{81.}}\parbox[t]{\dimexpr\textwidth-1.5em}{Марійка викладала відео. Першого дня відео набрало 100 переглядів. Кожного наступного дня кількість збільшувалася вдвічі. За яку \textit{найменшу} кількість днів сумарна кількість переглядів перевищить 2000? \nmtyear{2026}}
\vspace{0.3cm}

\answerTable{6}{7}{3}{4}{5}

\vspace{0.5cm}

\noindent\makebox[1.5em][l]{\textbf{82.}}\parbox[t]{\dimexpr\textwidth-1.5em}{Марійка викладала відео. Першого дня відео набрало 100 переглядів. Кожного наступного дня кількість збільшувалася вдвічі. За яку \textit{найменшу} кількість днів сумарна кількість переглядів перевищить 2000? \nmtyear{2026}}
\vspace{0.3cm}

\answerTable{7}{6}{3}{4}{5}

\vspace{0.5cm}

\noindent\makebox[1.5em][l]{\textbf{83.}}\parbox[t]{\dimexpr\textwidth-1.5em}{Бактерія ділиться. Першого дня колонія налічувала 100 бактерій. Кожного наступного дня кількість збільшувалася вдвічі. За яку \textit{найменшу} кількість днів сумарна кількість бактерій перевищить 2000? \nmtyear{2026}}
\vspace{0.3cm}

\answerTable{5}{4}{6}{7}{3}

\vspace{0.5cm}

\noindent\makebox[1.5em][l]{\textbf{84.}}\parbox[t]{\dimexpr\textwidth-1.5em}{Бактерія ділиться. Першого дня колонія налічувала 50 бактерій. Кожного наступного дня кількість збільшувалася вдвічі. За яку \textit{найменшу} кількість днів сумарна кількість бактерій перевищить 500? \nmtyear{2026}}
\vspace{0.3cm}

\answerTable{6}{5}{4}{2}{3}

\vspace{0.5cm}

\noindent\makebox[1.5em][l]{\textbf{85.}}\parbox[t]{\dimexpr\textwidth-1.5em}{Бактерія ділиться. Першого дня колонія налічувала 10 бактерій. Кожного наступного дня кількість збільшувалася вдвічі. За яку \textit{найменшу} кількість днів сумарна кількість бактерій перевищить 2000? \nmtyear{2026}}
\vspace{0.3cm}

\answerTable{8}{6}{10}{9}{7}

\vspace{0.5cm}

\noindent\makebox[1.5em][l]{\textbf{86.}}\parbox[t]{\dimexpr\textwidth-1.5em}{Інвестор вклав гроші. Першого дня прибуток склав 5 доларів. Кожного наступного дня кількість збільшувалася вдвічі. За яку \textit{найменшу} кількість днів сумарна кількість доларів перевищить 500? \nmtyear{2026}}
\vspace{0.3cm}

\answerTable{9}{5}{7}{6}{8}

\vspace{0.5cm}

\noindent\makebox[1.5em][l]{\textbf{87.}}\parbox[t]{\dimexpr\textwidth-1.5em}{Інвестор вклав гроші. Першого дня прибуток склав 5 доларів. Кожного наступного дня кількість збільшувалася вдвічі. За яку \textit{найменшу} кількість днів сумарна кількість доларів перевищить 2000? \nmtyear{2026}}
\vspace{0.3cm}

\answerTable{7}{11}{8}{9}{10}

\vspace{0.5cm}

\noindent\makebox[1.5em][l]{\textbf{88.}}\parbox[t]{\dimexpr\textwidth-1.5em}{Інвестор вклав гроші. Першого дня прибуток склав 5 доларів. Кожного наступного дня кількість збільшувалася вдвічі. За яку \textit{найменшу} кількість днів сумарна кількість доларів перевищить 500? \nmtyear{2026}}
\vspace{0.3cm}

\answerTable{7}{6}{8}{5}{9}

\vspace{0.5cm}

\noindent\makebox[1.5em][l]{\textbf{89.}}\parbox[t]{\dimexpr\textwidth-1.5em}{Марійка викладала відео. Першого дня відео набрало 10 переглядів. Кожного наступного дня кількість збільшувалася вдвічі. За яку \textit{найменшу} кількість днів сумарна кількість переглядів перевищить 5000? \nmtyear{2026}}
\vspace{0.3cm}

\answerTable{7}{8}{9}{10}{11}

\vspace{0.5cm}

\noindent\makebox[1.5em][l]{\textbf{90.}}\parbox[t]{\dimexpr\textwidth-1.5em}{Марійка викладала відео. Першого дня відео набрало 50 переглядів. Кожного наступного дня кількість збільшувалася вдвічі. За яку \textit{найменшу} кількість днів сумарна кількість переглядів перевищить 1000? \nmtyear{2026}}
\vspace{0.3cm}

\answerTable{5}{6}{7}{3}{4}

\vspace{0.5cm}

\noindent\makebox[1.5em][l]{\textbf{91.}}\parbox[t]{\dimexpr\textwidth-1.5em}{Бактерія ділиться. Першого дня колонія налічувала 5 бактерій. Кожного наступного дня кількість збільшувалася вдвічі. За яку \textit{найменшу} кількість днів сумарна кількість бактерій перевищить 500? \nmtyear{2026}}
\vspace{0.3cm}

\answerTable{8}{6}{9}{7}{5}

\vspace{0.5cm}

\noindent\makebox[1.5em][l]{\textbf{92.}}\parbox[t]{\dimexpr\textwidth-1.5em}{Інвестор вклав гроші. Першого дня прибуток склав 100 доларів. Кожного наступного дня кількість збільшувалася вдвічі. За яку \textit{найменшу} кількість днів сумарна кількість доларів перевищить 5000? \nmtyear{2026}}
\vspace{0.3cm}

\answerTable{7}{5}{6}{8}{4}

\vspace{0.5cm}

\noindent\makebox[1.5em][l]{\textbf{93.}}\parbox[t]{\dimexpr\textwidth-1.5em}{Інвестор вклав гроші. Першого дня прибуток склав 100 доларів. Кожного наступного дня кількість збільшувалася вдвічі. За яку \textit{найменшу} кількість днів сумарна кількість доларів перевищить 2000? \nmtyear{2026}}
\vspace{0.3cm}

\answerTable{6}{3}{5}{7}{4}

\vspace{0.5cm}

\noindent\makebox[1.5em][l]{\textbf{94.}}\parbox[t]{\dimexpr\textwidth-1.5em}{Марійка викладала відео. Першого дня відео набрало 50 переглядів. Кожного наступного дня кількість збільшувалася вдвічі. За яку \textit{найменшу} кількість днів сумарна кількість переглядів перевищить 500? \nmtyear{2026}}
\vspace{0.3cm}

\answerTable{3}{6}{2}{4}{5}

\vspace{0.5cm}

\noindent\makebox[1.5em][l]{\textbf{95.}}\parbox[t]{\dimexpr\textwidth-1.5em}{Бактерія ділиться. Першого дня колонія налічувала 50 бактерій. Кожного наступного дня кількість збільшувалася вдвічі. За яку \textit{найменшу} кількість днів сумарна кількість бактерій перевищить 1000? \nmtyear{2026}}
\vspace{0.3cm}

\answerTable{3}{6}{5}{7}{4}

\vspace{0.5cm}

\noindent\makebox[1.5em][l]{\textbf{96.}}\parbox[t]{\dimexpr\textwidth-1.5em}{Інвестор вклав гроші. Першого дня прибуток склав 100 доларів. Кожного наступного дня кількість збільшувалася вдвічі. За яку \textit{найменшу} кількість днів сумарна кількість доларів перевищить 5000? \nmtyear{2026}}
\vspace{0.3cm}

\answerTable{5}{7}{8}{6}{4}

\vspace{0.5cm}

\noindent\makebox[1.5em][l]{\textbf{97.}}\parbox[t]{\dimexpr\textwidth-1.5em}{Марійка викладала відео. Першого дня відео набрало 50 переглядів. Кожного наступного дня кількість збільшувалася вдвічі. За яку \textit{найменшу} кількість днів сумарна кількість переглядів перевищить 1000? \nmtyear{2026}}
\vspace{0.3cm}

\answerTable{7}{6}{3}{5}{4}

\vspace{0.5cm}

\noindent\makebox[1.5em][l]{\textbf{98.}}\parbox[t]{\dimexpr\textwidth-1.5em}{Марійка викладала відео. Першого дня відео набрало 100 переглядів. Кожного наступного дня кількість збільшувалася вдвічі. За яку \textit{найменшу} кількість днів сумарна кількість переглядів перевищить 2000? \nmtyear{2026}}
\vspace{0.3cm}

\answerTable{7}{4}{5}{3}{6}

\vspace{0.5cm}

\noindent\makebox[1.5em][l]{\textbf{99.}}\parbox[t]{\dimexpr\textwidth-1.5em}{Інвестор вклав гроші. Першого дня прибуток склав 10 доларів. Кожного наступного дня кількість збільшувалася вдвічі. За яку \textit{найменшу} кількість днів сумарна кількість доларів перевищить 1000? \nmtyear{2026}}
\vspace{0.3cm}

\answerTable{5}{6}{9}{8}{7}

\vspace{0.5cm}

\noindent\makebox[1.5em][l]{\textbf{100.}}\parbox[t]{\dimexpr\textwidth-1.5em}{Інвестор вклав гроші. Першого дня прибуток склав 100 доларів. Кожного наступного дня кількість збільшувалася вдвічі. За яку \textit{найменшу} кількість днів сумарна кількість доларів перевищить 1000? \nmtyear{2026}}
\vspace{0.3cm}

\answerTable{2}{6}{4}{5}{3}

\vspace{0.5cm}


\end{document}
