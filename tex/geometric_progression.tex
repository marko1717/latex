\documentclass[14pt]{extarticle}
\usepackage{fontspec}
\usepackage{polyglossia}
\setdefaultlanguage{ukrainian}

\defaultfontfeatures{Ligatures=TeX}
\setmainfont{Liberation Serif}
\setsansfont{Liberation Sans}
\setmonofont{Liberation Mono}

\usepackage[a4paper,margin=1.5cm,bottom=2cm,top=2cm]{geometry}
\usepackage{amsmath,amssymb}
\usepackage{enumitem}
\usepackage{tikz}
\usepackage{pgfplots}
\pgfplotsset{compat=1.18}

\usetikzlibrary{calc,patterns,angles,quotes,intersections,babel}
\usetikzlibrary{3d}

\usepackage{xcolor}
\usepackage{array}
\usepackage{fancyhdr}
\usepackage{multirow}

% Кольори
\definecolor{headerblue}{RGB}{0, 102, 204}
\definecolor{yearcolor}{RGB}{128, 0, 128}

\pagestyle{fancy}
\fancyhf{}
\renewcommand{\headrulewidth}{0pt}
\fancyfoot[C]{\thepage}

\setlength{\headheight}{15pt}
\setlength{\headsep}{10pt}
\setlength{\footskip}{25pt}

\widowpenalty=10000
\clubpenalty=10000

% === КОМАНДИ ===

% Таблиця відповідей для відповідностей
\newcommand{\answerGrid}{
    \begingroup
    \renewcommand{\arraystretch}{1.3} 
    \setlength{\tabcolsep}{7pt} 
    \begin{tabular}{r|c|c|c|c|c|}
         \multicolumn{1}{c}{} & \multicolumn{1}{c}{\textbf{А}} & \multicolumn{1}{c}{\textbf{Б}} & \multicolumn{1}{c}{\textbf{В}} & \multicolumn{1}{c}{\textbf{Г}} & \multicolumn{1}{c}{\textbf{Д}} \\ \cline{2-6}
         \textbf{1} & & & & & \\ \cline{2-6}
         \textbf{2} & & & & & \\ \cline{2-6}
         \textbf{3} & & & & & \\ \cline{2-6}
    \end{tabular}
    \endgroup
}

% Макет для завдань на відповідність
\newcommand{\matchingLayout}[3]{
    \noindent
    \begin{minipage}[t]{0.40\textwidth}
        #1
    \end{minipage}%
    \hfill
    \begin{minipage}[t]{0.28\textwidth}
        #2
    \end{minipage}%
    \hfill
    \begin{minipage}[t]{0.30\textwidth}
        \vspace{0pt}
        \begin{flushright}
        #3
        \end{flushright}
    \end{minipage}
}

% Стандартна таблиця відповідей
\newcommand{\answerTable}[5]{
\begin{center}
\begin{tabular}{|*{5}{>{\centering\arraybackslash}m{2.8cm}|}}
\hline
\rule[-0.3cm]{0pt}{0.8cm}\textbf{А} & \textbf{Б} & \textbf{В} & \textbf{Г} & \textbf{Д} \\
\hline
\rule[-0.4cm]{0pt}{1.0cm}#1 & \rule[-0.4cm]{0pt}{1.0cm}#2 & \rule[-0.4cm]{0pt}{1.0cm}#3 & \rule[-0.4cm]{0pt}{1.0cm}#4 & \rule[-0.4cm]{0pt}{1.0cm}#5 \\
\hline
\end{tabular}
\end{center}
}

% Таблиця для відповідей із дробами
\newcommand{\answerTableTall}[5]{
\begin{center}
\begin{tabular}{|*{5}{>{\centering\arraybackslash}m{2.8cm}|}}
\hline
\rule[-0.3cm]{0pt}{0.8cm}\textbf{А} & \textbf{Б} & \textbf{В} & \textbf{Г} & \textbf{Д} \\
\hline
\rule[-0.9cm]{0pt}{2.0cm}#1 & 
\rule[-0.9cm]{0pt}{2.0cm}#2 & 
\rule[-0.9cm]{0pt}{2.0cm}#3 & 
\rule[-0.9cm]{0pt}{2.0cm}#4 & 
\rule[-0.9cm]{0pt}{2.0cm}#5 \\
\hline
\end{tabular}
\end{center}
}

\newcommand{\nmtyear}[1]{\hfill{\small\color{yearcolor}(AI Gen)}}

\begin{document}

\vspace{1cm}

\begin{center}
{\Large\textbf{\color{headerblue}ЗГЕНЕРОВАНІ ЗАВДАННЯ (AI)}}
\end{center}

\begin{center}
{\large Тема: \textbf{Геометрична прогресія}}
\end{center}

\vspace{0.5cm}
% === Geometric Progression: Find Term (b_1, b_2 -> b_n) ===
\noindent\makebox[1.5em][l]{\textbf{1.}}\parbox[t]{\dimexpr\textwidth-1.5em}{У геометричній прогресії $(b_n)$ відомо, що $b_1 = 81$, $b_2 = 20{,}25$. Визначте $b_{7}$. \nmtyear{2026}}
\vspace{0.3cm}

\answerTable{-0{,}0198}{0{,}0791}{331776}{0{,}0198}{0{,}0049}

\vspace{0.5cm}

\noindent\makebox[1.5em][l]{\textbf{2.}}\parbox[t]{\dimexpr\textwidth-1.5em}{У геометричній прогресії $(b_n)$ відомо, що $b_1 = 81$, $b_2 = 324$. Визначте $b_{4}$. \nmtyear{2026}}
\vspace{0.3cm}

\answerTable{1{,}2656}{5184}{1296}{20736}{-5184}

\vspace{0.5cm}

\noindent\makebox[1.5em][l]{\textbf{3.}}\parbox[t]{\dimexpr\textwidth-1.5em}{У геометричній прогресії $(b_n)$ відомо, що $b_1 = 10$, $b_2 = -5$. Визначте $b_{6}$. \nmtyear{2026}}
\vspace{0.3cm}

\answerTable{-0{,}3125}{-320}{0{,}625}{0{,}1562}{0{,}3125}

\vspace{0.5cm}

\noindent\makebox[1.5em][l]{\textbf{4.}}\parbox[t]{\dimexpr\textwidth-1.5em}{У геометричній прогресії $(b_n)$ відомо, що $b_1 = 81$, $b_2 = 162$. Визначте $b_{4}$. \nmtyear{2026}}
\vspace{0.3cm}

\answerTable{10{,}125}{648}{324}{-648}{1296}

\vspace{0.5cm}

\noindent\makebox[1.5em][l]{\textbf{5.}}\parbox[t]{\dimexpr\textwidth-1.5em}{У геометричній прогресії $(b_n)$ відомо, що $b_1 = 27$, $b_2 = 108$. Визначте $b_{3}$. \nmtyear{2026}}
\vspace{0.3cm}

\answerTable{432}{1{,}6875}{189}{-432}{1728}

\vspace{0.5cm}

\noindent\makebox[1.5em][l]{\textbf{6.}}\parbox[t]{\dimexpr\textwidth-1.5em}{У геометричній прогресії $(b_n)$ відомо, що $b_1 = 32$, $b_2 = 16$. Визначте $b_{4}$. \nmtyear{2026}}
\vspace{0.3cm}

\answerTable{8}{-4}{4}{2}{-16}

\vspace{0.5cm}

\noindent\makebox[1.5em][l]{\textbf{7.}}\parbox[t]{\dimexpr\textwidth-1.5em}{У геометричній прогресії $(b_n)$ відомо, що $b_1 = 32$, $b_2 = -16$. Визначте $b_{7}$. \nmtyear{2026}}
\vspace{0.3cm}

\answerTable{-0{,}5}{0{,}5}{-256}{-1}{2048}

\vspace{0.5cm}

\noindent\makebox[1.5em][l]{\textbf{8.}}\parbox[t]{\dimexpr\textwidth-1.5em}{У геометричній прогресії $(b_n)$ відомо, що $b_1 = 10$, $b_2 = 40$. Визначте $b_{3}$. \nmtyear{2026}}
\vspace{0.3cm}

\answerTable{70}{160}{40}{640}{0{,}625}

\vspace{0.5cm}

\noindent\makebox[1.5em][l]{\textbf{9.}}\parbox[t]{\dimexpr\textwidth-1.5em}{У геометричній прогресії $(b_n)$ відомо, що $b_1 = 81$, $b_2 = -40{,}5$. Визначте $b_{5}$. \nmtyear{2026}}
\vspace{0.3cm}

\answerTable{-405}{-10{,}125}{1296}{5{,}0625}{-5{,}0625}

\vspace{0.5cm}

\noindent\makebox[1.5em][l]{\textbf{10.}}\parbox[t]{\dimexpr\textwidth-1.5em}{У геометричній прогресії $(b_n)$ відомо, що $b_1 = 10$, $b_2 = 40$. Визначте $b_{4}$. \nmtyear{2026}}
\vspace{0.3cm}

\answerTable{0{,}1562}{2560}{640}{100}{160}

\vspace{0.5cm}

\noindent\makebox[1.5em][l]{\textbf{11.}}\parbox[t]{\dimexpr\textwidth-1.5em}{У геометричній прогресії $(b_n)$ відомо, що $b_1 = 16$, $b_2 = 48$. Визначте $b_{4}$. \nmtyear{2026}}
\vspace{0.3cm}

\answerTable{0{,}5926}{-432}{432}{1296}{112}

\vspace{0.5cm}

\noindent\makebox[1.5em][l]{\textbf{12.}}\parbox[t]{\dimexpr\textwidth-1.5em}{У геометричній прогресії $(b_n)$ відомо, що $b_1 = 32$, $b_2 = -16$. Визначте $b_{7}$. \nmtyear{2026}}
\vspace{0.3cm}

\answerTable{0{,}5}{-256}{2048}{-0{,}5}{-1}

\vspace{0.5cm}

\noindent\makebox[1.5em][l]{\textbf{13.}}\parbox[t]{\dimexpr\textwidth-1.5em}{У геометричній прогресії $(b_n)$ відомо, що $b_1 = 5$, $b_2 = 1{,}25$. Визначте $b_{7}$. \nmtyear{2026}}
\vspace{0.3cm}

\answerTable{0{,}0012}{-0{,}0012}{-17{,}5}{0{,}0049}{20480}

\vspace{0.5cm}

\noindent\makebox[1.5em][l]{\textbf{14.}}\parbox[t]{\dimexpr\textwidth-1.5em}{У геометричній прогресії $(b_n)$ відомо, що $b_1 = 4$, $b_2 = 12$. Визначте $b_{4}$. \nmtyear{2026}}
\vspace{0.3cm}

\answerTable{-108}{28}{108}{324}{0{,}1481}

\vspace{0.5cm}

\noindent\makebox[1.5em][l]{\textbf{15.}}\parbox[t]{\dimexpr\textwidth-1.5em}{У геометричній прогресії $(b_n)$ відомо, що $b_1 = 5$, $b_2 = 20$. Визначте $b_{4}$. \nmtyear{2026}}
\vspace{0.3cm}

\answerTable{0{,}0781}{-320}{80}{1280}{320}

\vspace{0.5cm}

\noindent\makebox[1.5em][l]{\textbf{16.}}\parbox[t]{\dimexpr\textwidth-1.5em}{У геометричній прогресії $(b_n)$ відомо, що $b_1 = 27$, $b_2 = 108$. Визначте $b_{3}$. \nmtyear{2026}}
\vspace{0.3cm}

\answerTable{189}{432}{-432}{1728}{1{,}6875}

\vspace{0.5cm}

\noindent\makebox[1.5em][l]{\textbf{17.}}\parbox[t]{\dimexpr\textwidth-1.5em}{У геометричній прогресії $(b_n)$ відомо, що $b_1 = 3$, $b_2 = 0{,}75$. Визначте $b_{5}$. \nmtyear{2026}}
\vspace{0.3cm}

\answerTable{0{,}0469}{0{,}0117}{-6}{768}{0{,}0029}

\vspace{0.5cm}

\noindent\makebox[1.5em][l]{\textbf{18.}}\parbox[t]{\dimexpr\textwidth-1.5em}{У геометричній прогресії $(b_n)$ відомо, що $b_1 = 10$, $b_2 = -5$. Визначте $b_{4}$. \nmtyear{2026}}
\vspace{0.3cm}

\answerTable{2{,}5}{-1{,}25}{0{,}625}{-80}{-35}

\vspace{0.5cm}

\noindent\makebox[1.5em][l]{\textbf{19.}}\parbox[t]{\dimexpr\textwidth-1.5em}{У геометричній прогресії $(b_n)$ відомо, що $b_1 = 32$, $b_2 = 16$. Визначте $b_{4}$. \nmtyear{2026}}
\vspace{0.3cm}

\answerTable{8}{-4}{2}{4}{-16}

\vspace{0.5cm}

\noindent\makebox[1.5em][l]{\textbf{20.}}\parbox[t]{\dimexpr\textwidth-1.5em}{У геометричній прогресії $(b_n)$ відомо, що $b_1 = 2$, $b_2 = -1$. Визначте $b_{4}$. \nmtyear{2026}}
\vspace{0.3cm}

\answerTable{0{,}5}{0{,}125}{-16}{-7}{-0{,}25}

\vspace{0.5cm}

% === Geometric Progression: Term Ratio ===
\noindent\makebox[1.5em][l]{\textbf{21.}}\parbox[t]{\dimexpr\textwidth-1.5em}{У геометричній прогресії $(b_n)$ відомо, що $b_1 = 4$, $b_2 = 20$. Обчисліть $\dfrac{b_{5}}{b_{7}}$. \nmtyear{2026}}
\vspace{0.3cm}

\answerTable{5}{-60000}{0{,}04}{25}{0{,}2}

\vspace{0.5cm}

\noindent\makebox[1.5em][l]{\textbf{22.}}\parbox[t]{\dimexpr\textwidth-1.5em}{У геометричній прогресії $(b_n)$ відомо, що $b_1 = 32$, $b_2 = 128$. Обчисліть $\dfrac{b_{4}}{b_{5}}$. \nmtyear{2026}}
\vspace{0.3cm}

\answerTable{-6144}{4}{78}{67}{0{,}25}

\vspace{0.5cm}

\noindent\makebox[1.5em][l]{\textbf{23.}}\parbox[t]{\dimexpr\textwidth-1.5em}{У геометричній прогресії $(b_n)$ відомо, що $b_1 = 10$, $b_2 = 5$. Обчисліть $\dfrac{b_{6}}{b_{9}}$. \nmtyear{2026}}
\vspace{0.3cm}

\answerTable{8}{0{,}2734}{0{,}5}{2}{0{,}125}

\vspace{0.5cm}

\noindent\makebox[1.5em][l]{\textbf{24.}}\parbox[t]{\dimexpr\textwidth-1.5em}{У геометричній прогресії $(b_n)$ відомо, що $b_1 = 2$, $b_2 = 8$. Обчисліть $\dfrac{b_{5}}{b_{6}}$. \nmtyear{2026}}
\vspace{0.3cm}

\answerTable{36}{-1536}{0{,}25}{4}{7}

\vspace{0.5cm}

\noindent\makebox[1.5em][l]{\textbf{25.}}\parbox[t]{\dimexpr\textwidth-1.5em}{У геометричній прогресії $(b_n)$ відомо, що $b_1 = 5$, $b_2 = 25$. Обчисліть $\dfrac{b_{3}}{b_{4}}$. \nmtyear{2026}}
\vspace{0.3cm}

\answerTable{36}{81}{-500}{5}{0{,}2}

\vspace{0.5cm}

\noindent\makebox[1.5em][l]{\textbf{26.}}\parbox[t]{\dimexpr\textwidth-1.5em}{У геометричній прогресії $(b_n)$ відомо, що $b_1 = 4$, $b_2 = 12$. Обчисліть $\dfrac{b_{6}}{b_{9}}$. \nmtyear{2026}}
\vspace{0.3cm}

\answerTable{0{,}037}{-25272}{3}{27}{0{,}3333}

\vspace{0.5cm}

\noindent\makebox[1.5em][l]{\textbf{27.}}\parbox[t]{\dimexpr\textwidth-1.5em}{У геометричній прогресії $(b_n)$ відомо, що $b_1 = 32$, $b_2 = 128$. Обчисліть $\dfrac{b_{6}}{b_{8}}$. \nmtyear{2026}}
\vspace{0.3cm}

\answerTable{4}{0{,}25}{16}{0{,}0625}{-491520}

\vspace{0.5cm}

\noindent\makebox[1.5em][l]{\textbf{28.}}\parbox[t]{\dimexpr\textwidth-1.5em}{У геометричній прогресії $(b_n)$ відомо, що $b_1 = 2$, $b_2 = 10$. Обчисліть $\dfrac{b_{7}}{b_{10}}$. \nmtyear{2026}}
\vspace{0.3cm}

\answerTable{-3875000}{0{,}008}{125}{5}{0{,}2}

\vspace{0.5cm}

\noindent\makebox[1.5em][l]{\textbf{29.}}\parbox[t]{\dimexpr\textwidth-1.5em}{У геометричній прогресії $(b_n)$ відомо, що $b_1 = 5$, $b_2 = 25$. Обчисліть $\dfrac{b_{7}}{b_{8}}$. \nmtyear{2026}}
\vspace{0.3cm}

\answerTable{11}{0{,}2}{5}{7}{-312500}

\vspace{0.5cm}

\noindent\makebox[1.5em][l]{\textbf{30.}}\parbox[t]{\dimexpr\textwidth-1.5em}{У геометричній прогресії $(b_n)$ відомо, що $b_1 = 32$, $b_2 = 320$. Обчисліть $\dfrac{b_{6}}{b_{7}}$. \nmtyear{2026}}
\vspace{0.3cm}

\answerTable{10}{30}{96}{0{,}1}{-28800000}

\vspace{0.5cm}

\noindent\makebox[1.5em][l]{\textbf{31.}}\parbox[t]{\dimexpr\textwidth-1.5em}{У геометричній прогресії $(b_n)$ відомо, що $b_1 = 4$, $b_2 = 16$. Обчисліть $\dfrac{b_{7}}{b_{10}}$. \nmtyear{2026}}
\vspace{0.3cm}

\answerTable{64}{0{,}25}{4}{-1032192}{0{,}0156}

\vspace{0.5cm}

\noindent\makebox[1.5em][l]{\textbf{32.}}\parbox[t]{\dimexpr\textwidth-1.5em}{У геометричній прогресії $(b_n)$ відомо, що $b_1 = 5$, $b_2 = 50$. Обчисліть $\dfrac{b_{6}}{b_{7}}$. \nmtyear{2026}}
\vspace{0.3cm}

\answerTable{4}{10}{0{,}1}{-4500000}{82}

\vspace{0.5cm}

\noindent\makebox[1.5em][l]{\textbf{33.}}\parbox[t]{\dimexpr\textwidth-1.5em}{У геометричній прогресії $(b_n)$ відомо, що $b_1 = 4$, $b_2 = 16$. Обчисліть $\dfrac{b_{8}}{b_{10}}$. \nmtyear{2026}}
\vspace{0.3cm}

\answerTable{-983040}{0{,}25}{0{,}0625}{4}{16}

\vspace{0.5cm}

\noindent\makebox[1.5em][l]{\textbf{34.}}\parbox[t]{\dimexpr\textwidth-1.5em}{У геометричній прогресії $(b_n)$ відомо, що $b_1 = 10$, $b_2 = 50$. Обчисліть $\dfrac{b_{5}}{b_{7}}$. \nmtyear{2026}}
\vspace{0.3cm}

\answerTable{25}{5}{-150000}{0{,}2}{0{,}04}

\vspace{0.5cm}

\noindent\makebox[1.5em][l]{\textbf{35.}}\parbox[t]{\dimexpr\textwidth-1.5em}{У геометричній прогресії $(b_n)$ відомо, що $b_1 = 4$, $b_2 = 12$. Обчисліть $\dfrac{b_{3}}{b_{4}}$. \nmtyear{2026}}
\vspace{0.3cm}

\answerTable{-72}{0{,}3333}{34}{44}{3}

\vspace{0.5cm}

\noindent\makebox[1.5em][l]{\textbf{36.}}\parbox[t]{\dimexpr\textwidth-1.5em}{У геометричній прогресії $(b_n)$ відомо, що $b_1 = 5$, $b_2 = 20$. Обчисліть $\dfrac{b_{4}}{b_{7}}$. \nmtyear{2026}}
\vspace{0.3cm}

\answerTable{64}{0{,}25}{-20160}{0{,}0156}{4}

\vspace{0.5cm}

\noindent\makebox[1.5em][l]{\textbf{37.}}\parbox[t]{\dimexpr\textwidth-1.5em}{У геометричній прогресії $(b_n)$ відомо, що $b_1 = 10$, $b_2 = 30$. Обчисліть $\dfrac{b_{4}}{b_{7}}$. \nmtyear{2026}}
\vspace{0.3cm}

\answerTable{3}{-7020}{27}{0{,}3333}{0{,}037}

\vspace{0.5cm}

\noindent\makebox[1.5em][l]{\textbf{38.}}\parbox[t]{\dimexpr\textwidth-1.5em}{У геометричній прогресії $(b_n)$ відомо, що $b_1 = 10$, $b_2 = 100$. Обчисліть $\dfrac{b_{8}}{b_{10}}$. \nmtyear{2026}}
\vspace{0.3cm}

\answerTable{0{,}01}{-9900000000}{100}{10}{0{,}1}

\vspace{0.5cm}

\noindent\makebox[1.5em][l]{\textbf{39.}}\parbox[t]{\dimexpr\textwidth-1.5em}{У геометричній прогресії $(b_n)$ відомо, що $b_1 = 2$, $b_2 = 6$. Обчисліть $\dfrac{b_{3}}{b_{6}}$. \nmtyear{2026}}
\vspace{0.3cm}

\answerTable{3}{-468}{0{,}037}{27}{0{,}3333}

\vspace{0.5cm}

\noindent\makebox[1.5em][l]{\textbf{40.}}\parbox[t]{\dimexpr\textwidth-1.5em}{У геометричній прогресії $(b_n)$ відомо, що $b_1 = 10$, $b_2 = 100$. Обчисліть $\dfrac{b_{6}}{b_{7}}$. \nmtyear{2026}}
\vspace{0.3cm}

\answerTable{46}{10}{-9000000}{54}{0{,}1}

\vspace{0.5cm}

% === Geometric Progression: Formula ===
\noindent\makebox[1.5em][l]{\textbf{41.}}\parbox[t]{\dimexpr\textwidth-1.5em}{Геометричну прогресію задано формулою $n$-го члена $b_n = 0{,}5 \cdot 2^{n-2}$. Визначте 4-й член цієї прогресії. \nmtyear{2026}}
\vspace{0.3cm}

\answerTable{-2}{1}{2}{4}{3}

\vspace{0.5cm}

\noindent\makebox[1.5em][l]{\textbf{42.}}\parbox[t]{\dimexpr\textwidth-1.5em}{Геометричну прогресію задано формулою $n$-го члена $b_n = 4 \cdot 4^{n-1}$. Визначте 3-й член цієї прогресії. \nmtyear{2026}}
\vspace{0.3cm}

\answerTable{64}{63}{-64}{65}{128}

\vspace{0.5cm}

\noindent\makebox[1.5em][l]{\textbf{43.}}\parbox[t]{\dimexpr\textwidth-1.5em}{Послідовність задано формулою $n$-го члена $b_n = 0{,}8 \cdot 3^n + 3n$. Визначте 5-й член цієї послідовності. \nmtyear{2026}}
\vspace{0.3cm}

\answerTable{-209{,}4}{210{,}4}{209{,}4}{418{,}8}{208{,}4}

\vspace{0.5cm}

\noindent\makebox[1.5em][l]{\textbf{44.}}\parbox[t]{\dimexpr\textwidth-1.5em}{Послідовність задано формулою $n$-го члена $b_n = 0{,}5 \cdot 2^n + 1n$. Визначте 6-й член цієї послідовності. \nmtyear{2026}}
\vspace{0.3cm}

\answerTable{39}{-38}{37}{38}{76}

\vspace{0.5cm}

\noindent\makebox[1.5em][l]{\textbf{45.}}\parbox[t]{\dimexpr\textwidth-1.5em}{Послідовність задано формулою $n$-го члена $b_n = 3 \cdot 2^n$. Визначте 5-й член цієї послідовності. \nmtyear{2026}}
\vspace{0.3cm}

\answerTable{97}{96}{192}{95}{-96}

\vspace{0.5cm}

\noindent\makebox[1.5em][l]{\textbf{46.}}\parbox[t]{\dimexpr\textwidth-1.5em}{Геометричну прогресію задано формулою $n$-го члена $b_n = 2 \cdot 4^{n-4}$. Визначте 5-й член цієї прогресії. \nmtyear{2026}}
\vspace{0.3cm}

\answerTable{7}{9}{16}{-8}{8}

\vspace{0.5cm}

\noindent\makebox[1.5em][l]{\textbf{47.}}\parbox[t]{\dimexpr\textwidth-1.5em}{Послідовність задано формулою $n$-го члена $b_n = \dfrac{(-1)^n}{n}$. Визначте 6-й член цієї послідовності. \nmtyear{2026}}
\vspace{0.3cm}

\answerTable{1{,}167}{-0{,}167}{0{,}333}{0{,}167}{-0{,}833}

\vspace{0.5cm}

\noindent\makebox[1.5em][l]{\textbf{48.}}\parbox[t]{\dimexpr\textwidth-1.5em}{Послідовність задано формулою $n$-го члена $b_n = 4 \cdot 4^n + 2n$. Визначте 3-й член цієї послідовності. \nmtyear{2026}}
\vspace{0.3cm}

\answerTable{261}{262}{-262}{524}{263}

\vspace{0.5cm}

\noindent\makebox[1.5em][l]{\textbf{49.}}\parbox[t]{\dimexpr\textwidth-1.5em}{Послідовність задано формулою $n$-го члена $b_n = \dfrac{(-1)^n}{n}$. Визначте 3-й член цієї послідовності. \nmtyear{2026}}
\vspace{0.3cm}

\answerTable{-1{,}333}{-0{,}333}{-0{,}667}{0{,}667}{0{,}333}

\vspace{0.5cm}

\noindent\makebox[1.5em][l]{\textbf{50.}}\parbox[t]{\dimexpr\textwidth-1.5em}{Послідовність задано формулою $n$-го члена $b_n = 0{,}8 \cdot 2^n + 2n$. Визначте 4-й член цієї послідовності. \nmtyear{2026}}
\vspace{0.3cm}

\answerTable{41{,}6}{-20{,}8}{21{,}8}{19{,}8}{20{,}8}

\vspace{0.5cm}

\noindent\makebox[1.5em][l]{\textbf{51.}}\parbox[t]{\dimexpr\textwidth-1.5em}{Послідовність задано формулою $n$-го члена $b_n = 0{,}5 \cdot 4^n + 3n$. Визначте 6-й член цієї послідовності. \nmtyear{2026}}
\vspace{0.3cm}

\answerTable{2065}{-2066}{4132}{2066}{2067}

\vspace{0.5cm}

\noindent\makebox[1.5em][l]{\textbf{52.}}\parbox[t]{\dimexpr\textwidth-1.5em}{Геометричну прогресію задано формулою $n$-го члена $b_n = 0{,}5 \cdot 2^{n-3}$. Визначте 4-й член цієї прогресії. \nmtyear{2026}}
\vspace{0.3cm}

\answerTable{0}{2}{1}{9}{-1}

\vspace{0.5cm}

\noindent\makebox[1.5em][l]{\textbf{53.}}\parbox[t]{\dimexpr\textwidth-1.5em}{Геометричну прогресію задано формулою $n$-го члена $b_n = 5 \cdot 4^{n-3}$. Визначте 5-й член цієї прогресії. \nmtyear{2026}}
\vspace{0.3cm}

\answerTable{-80}{79}{160}{80}{81}

\vspace{0.5cm}

\noindent\makebox[1.5em][l]{\textbf{54.}}\parbox[t]{\dimexpr\textwidth-1.5em}{Послідовність задано формулою $n$-го члена $b_n = 5 \cdot 4^n + 3n$. Визначте 3-й член цієї послідовності. \nmtyear{2026}}
\vspace{0.3cm}

\answerTable{330}{658}{-329}{329}{328}

\vspace{0.5cm}

\noindent\makebox[1.5em][l]{\textbf{55.}}\parbox[t]{\dimexpr\textwidth-1.5em}{Геометричну прогресію задано формулою $n$-го члена $b_n = 0{,}8 \cdot 2^{n-4}$. Визначте 6-й член цієї прогресії. \nmtyear{2026}}
\vspace{0.3cm}

\answerTable{2{,}2}{4{,}2}{6{,}4}{-3{,}2}{3{,}2}

\vspace{0.5cm}

\noindent\makebox[1.5em][l]{\textbf{56.}}\parbox[t]{\dimexpr\textwidth-1.5em}{Послідовність задано формулою $n$-го члена $b_n = 5 \cdot 3^n$. Визначте 4-й член цієї послідовності. \nmtyear{2026}}
\vspace{0.3cm}

\answerTable{404}{-405}{405}{810}{406}

\vspace{0.5cm}

\noindent\makebox[1.5em][l]{\textbf{57.}}\parbox[t]{\dimexpr\textwidth-1.5em}{Послідовність задано формулою $n$-го члена $b_n = 1 \cdot 4^n$. Визначте 3-й член цієї послідовності. \nmtyear{2026}}
\vspace{0.3cm}

\answerTable{128}{65}{63}{-64}{64}

\vspace{0.5cm}

\noindent\makebox[1.5em][l]{\textbf{58.}}\parbox[t]{\dimexpr\textwidth-1.5em}{Послідовність задано формулою $n$-го члена $b_n = \dfrac{(-1)^n}{n}$. Визначте 5-й член цієї послідовності. \nmtyear{2026}}
\vspace{0.3cm}

\answerTable{0{,}8}{-1{,}2}{-0{,}4}{-0{,}2}{0{,}2}

\vspace{0.5cm}

\noindent\makebox[1.5em][l]{\textbf{59.}}\parbox[t]{\dimexpr\textwidth-1.5em}{Послідовність задано формулою $n$-го члена $b_n = 5 \cdot 3^n + 3n$. Визначте 6-й член цієї послідовності. \nmtyear{2026}}
\vspace{0.3cm}

\answerTable{3662}{-3663}{3663}{7326}{3664}

\vspace{0.5cm}

\noindent\makebox[1.5em][l]{\textbf{60.}}\parbox[t]{\dimexpr\textwidth-1.5em}{Послідовність задано формулою $n$-го члена $b_n = 5 \cdot 3^n + 3n$. Визначте 4-й член цієї послідовності. \nmtyear{2026}}
\vspace{0.3cm}

\answerTable{417}{418}{416}{-417}{834}

\vspace{0.5cm}

% === Geometric Progression: Sum ===
\noindent\makebox[1.5em][l]{\textbf{61.}}\parbox[t]{\dimexpr\textwidth-1.5em}{Знайдіть суму 5 перших членів геометричної прогресії $(b_n)$, у якої $b_2 = -4$, а знаменник $q = -3$. \nmtyear{2026}}
\vspace{0.3cm}

\answerTable{-27{,}111}{81{,}333}{60{,}333}{-81{,}333}{108}

\vspace{0.5cm}

\noindent\makebox[1.5em][l]{\textbf{62.}}\parbox[t]{\dimexpr\textwidth-1.5em}{Знайдіть суму 5 перших членів геометричної прогресії $(b_n)$, у якої $b_2 = -6$, а знаменник $q = 3$. \nmtyear{2026}}
\vspace{0.3cm}

\answerTable{-202}{-162}{242}{-80{,}667}{-242}

\vspace{0.5cm}

\noindent\makebox[1.5em][l]{\textbf{63.}}\parbox[t]{\dimexpr\textwidth-1.5em}{Знайдіть суму 4 перших членів геометричної прогресії $(b_n)$, у якої $b_2 = -4$, а знаменник $q = 2$. \nmtyear{2026}}
\vspace{0.3cm}

\answerTable{-16}{30}{-42}{-30}{-15}

\vspace{0.5cm}

\noindent\makebox[1.5em][l]{\textbf{64.}}\parbox[t]{\dimexpr\textwidth-1.5em}{Знайдіть суму 5 перших членів геометричної прогресії $(b_n)$, у якої $b_2 = -4$, а знаменник $q = 3$. \nmtyear{2026}}
\vspace{0.3cm}

\answerTable{-108}{-152{,}333}{161{,}333}{-53{,}778}{-161{,}333}

\vspace{0.5cm}

\noindent\makebox[1.5em][l]{\textbf{65.}}\parbox[t]{\dimexpr\textwidth-1.5em}{Знайдіть суму 3 перших членів геометричної прогресії $(b_n)$, у якої $b_2 = 18$, а знаменник $q = -3$. \nmtyear{2026}}
\vspace{0.3cm}

\answerTable{14}{-54}{42}{-24}{-42}

\vspace{0.5cm}

\noindent\makebox[1.5em][l]{\textbf{66.}}\parbox[t]{\dimexpr\textwidth-1.5em}{Знайдіть суму 4 перших членів геометричної прогресії $(b_n)$, у якої $b_2 = -6$, а знаменник $q = -2$. \nmtyear{2026}}
\vspace{0.3cm}

\answerTable{2}{-24}{15}{-15}{7{,}5}

\vspace{0.5cm}

\noindent\makebox[1.5em][l]{\textbf{67.}}\parbox[t]{\dimexpr\textwidth-1.5em}{Знайдіть суму 4 перших членів геометричної прогресії $(b_n)$, у якої $b_2 = 18$, а знаменник $q = 2$. \nmtyear{2026}}
\vspace{0.3cm}

\answerTable{67{,}5}{135}{101}{-135}{72}

\vspace{0.5cm}

\noindent\makebox[1.5em][l]{\textbf{68.}}\parbox[t]{\dimexpr\textwidth-1.5em}{Знайдіть суму 3 перших членів геометричної прогресії $(b_n)$, у якої $b_2 = -4$, а знаменник $q = -3$. \nmtyear{2026}}
\vspace{0.3cm}

\answerTable{-9{,}333}{12}{9{,}333}{-23{,}667}{-3{,}111}

\vspace{0.5cm}

\noindent\makebox[1.5em][l]{\textbf{69.}}\parbox[t]{\dimexpr\textwidth-1.5em}{Знайдіть суму 3 перших членів геометричної прогресії $(b_n)$, у якої $b_2 = 12$, а знаменник $q = 3$. \nmtyear{2026}}
\vspace{0.3cm}

\answerTable{-52}{35}{52}{36}{17{,}333}

\vspace{0.5cm}

\noindent\makebox[1.5em][l]{\textbf{70.}}\parbox[t]{\dimexpr\textwidth-1.5em}{Знайдіть суму 3 перших членів геометричної прогресії $(b_n)$, у якої $b_2 = -4$, а знаменник $q = -2$. \nmtyear{2026}}
\vspace{0.3cm}

\answerTable{-6}{6}{8}{-3}{-23}

\vspace{0.5cm}

\noindent\makebox[1.5em][l]{\textbf{71.}}\parbox[t]{\dimexpr\textwidth-1.5em}{Знайдіть суму 4 перших членів геометричної прогресії $(b_n)$, у якої $b_2 = 12$, а знаменник $q = -3$. \nmtyear{2026}}
\vspace{0.3cm}

\answerTable{85}{108}{-26{,}667}{-80}{80}

\vspace{0.5cm}

\noindent\makebox[1.5em][l]{\textbf{72.}}\parbox[t]{\dimexpr\textwidth-1.5em}{Знайдіть суму 4 перших членів геометричної прогресії $(b_n)$, у якої $b_2 = 4$, а знаменник $q = -2$. \nmtyear{2026}}
\vspace{0.3cm}

\answerTable{-10}{-14}{-5}{10}{16}

\vspace{0.5cm}

\noindent\makebox[1.5em][l]{\textbf{73.}}\parbox[t]{\dimexpr\textwidth-1.5em}{Знайдіть суму 4 перших членів геометричної прогресії $(b_n)$, у якої $b_2 = 4$, а знаменник $q = -2$. \nmtyear{2026}}
\vspace{0.3cm}

\answerTable{-5}{10}{27}{-10}{16}

\vspace{0.5cm}

\noindent\makebox[1.5em][l]{\textbf{74.}}\parbox[t]{\dimexpr\textwidth-1.5em}{Знайдіть суму 5 перших членів геометричної прогресії $(b_n)$, у якої $b_2 = 12$, а знаменник $q = 0{,}5$. \nmtyear{2026}}
\vspace{0.3cm}

\answerTable{-46{,}5}{94{,}5}{93}{46{,}5}{1{,}5}

\vspace{0.5cm}

\noindent\makebox[1.5em][l]{\textbf{75.}}\parbox[t]{\dimexpr\textwidth-1.5em}{Знайдіть суму 4 перших членів геометричної прогресії $(b_n)$, у якої $b_2 = -4$, а знаменник $q = -3$. \nmtyear{2026}}
\vspace{0.3cm}

\answerTable{8{,}889}{26{,}667}{-36}{-26{,}667}{-22{,}667}

\vspace{0.5cm}

\noindent\makebox[1.5em][l]{\textbf{76.}}\parbox[t]{\dimexpr\textwidth-1.5em}{Знайдіть суму 3 перших членів геометричної прогресії $(b_n)$, у якої $b_2 = 4$, а знаменник $q = 2$. \nmtyear{2026}}
\vspace{0.3cm}

\answerTable{-14}{14}{7}{8}{-3}

\vspace{0.5cm}

\noindent\makebox[1.5em][l]{\textbf{77.}}\parbox[t]{\dimexpr\textwidth-1.5em}{Знайдіть суму 5 перших членів геометричної прогресії $(b_n)$, у якої $b_2 = 18$, а знаменник $q = 0{,}5$. \nmtyear{2026}}
\vspace{0.3cm}

\answerTable{2{,}25}{-69{,}75}{139{,}5}{69{,}75}{100{,}75}

\vspace{0.5cm}

\noindent\makebox[1.5em][l]{\textbf{78.}}\parbox[t]{\dimexpr\textwidth-1.5em}{Знайдіть суму 3 перших членів геометричної прогресії $(b_n)$, у якої $b_2 = -6$, а знаменник $q = 2$. \nmtyear{2026}}
\vspace{0.3cm}

\answerTable{-21}{-10{,}5}{21}{-23}{-12}

\vspace{0.5cm}

\noindent\makebox[1.5em][l]{\textbf{79.}}\parbox[t]{\dimexpr\textwidth-1.5em}{Знайдіть суму 5 перших членів геометричної прогресії $(b_n)$, у якої $b_2 = 12$, а знаменник $q = 3$. \nmtyear{2026}}
\vspace{0.3cm}

\answerTable{478}{-484}{324}{484}{161{,}333}

\vspace{0.5cm}

\noindent\makebox[1.5em][l]{\textbf{80.}}\parbox[t]{\dimexpr\textwidth-1.5em}{Знайдіть суму 5 перших членів геометричної прогресії $(b_n)$, у якої $b_2 = 4$, а знаменник $q = 0{,}5$. \nmtyear{2026}}
\vspace{0.3cm}

\answerTable{-20{,}5}{0{,}5}{31}{15{,}5}{-15{,}5}

\vspace{0.5cm}

% === Geometric Progression: Word Problem ===
\noindent\makebox[1.5em][l]{\textbf{81.}}\parbox[t]{\dimexpr\textwidth-1.5em}{Інвестор вклав гроші. Першого дня прибуток склав 50 доларів. Кожного наступного дня кількість збільшувалася вдвічі. За яку \textit{найменшу} кількість днів сумарна кількість доларів перевищить 500? \nmtyear{2026}}
\vspace{0.3cm}

\answerTable{4}{3}{5}{2}{6}

\vspace{0.5cm}

\noindent\makebox[1.5em][l]{\textbf{82.}}\parbox[t]{\dimexpr\textwidth-1.5em}{Марійка викладала відео. Першого дня відео набрало 10 переглядів. Кожного наступного дня кількість збільшувалася вдвічі. За яку \textit{найменшу} кількість днів сумарна кількість переглядів перевищить 2000? \nmtyear{2026}}
\vspace{0.3cm}

\answerTable{7}{8}{6}{10}{9}

\vspace{0.5cm}

\noindent\makebox[1.5em][l]{\textbf{83.}}\parbox[t]{\dimexpr\textwidth-1.5em}{Інвестор вклав гроші. Першого дня прибуток склав 5 доларів. Кожного наступного дня кількість збільшувалася вдвічі. За яку \textit{найменшу} кількість днів сумарна кількість доларів перевищить 500? \nmtyear{2026}}
\vspace{0.3cm}

\answerTable{5}{7}{9}{6}{8}

\vspace{0.5cm}

\noindent\makebox[1.5em][l]{\textbf{84.}}\parbox[t]{\dimexpr\textwidth-1.5em}{Інвестор вклав гроші. Першого дня прибуток склав 100 доларів. Кожного наступного дня кількість збільшувалася вдвічі. За яку \textit{найменшу} кількість днів сумарна кількість доларів перевищить 5000? \nmtyear{2026}}
\vspace{0.3cm}

\answerTable{6}{5}{4}{8}{7}

\vspace{0.5cm}

\noindent\makebox[1.5em][l]{\textbf{85.}}\parbox[t]{\dimexpr\textwidth-1.5em}{Марійка викладала відео. Першого дня відео набрало 50 переглядів. Кожного наступного дня кількість збільшувалася вдвічі. За яку \textit{найменшу} кількість днів сумарна кількість переглядів перевищить 5000? \nmtyear{2026}}
\vspace{0.3cm}

\answerTable{5}{6}{8}{7}{9}

\vspace{0.5cm}

\noindent\makebox[1.5em][l]{\textbf{86.}}\parbox[t]{\dimexpr\textwidth-1.5em}{Бактерія ділиться. Першого дня колонія налічувала 50 бактерій. Кожного наступного дня кількість збільшувалася вдвічі. За яку \textit{найменшу} кількість днів сумарна кількість бактерій перевищить 5000? \nmtyear{2026}}
\vspace{0.3cm}

\answerTable{8}{5}{6}{9}{7}

\vspace{0.5cm}

\noindent\makebox[1.5em][l]{\textbf{87.}}\parbox[t]{\dimexpr\textwidth-1.5em}{Інвестор вклав гроші. Першого дня прибуток склав 100 доларів. Кожного наступного дня кількість збільшувалася вдвічі. За яку \textit{найменшу} кількість днів сумарна кількість доларів перевищить 500? \nmtyear{2026}}
\vspace{0.3cm}

\answerTable{2}{5}{1}{4}{3}

\vspace{0.5cm}

\noindent\makebox[1.5em][l]{\textbf{88.}}\parbox[t]{\dimexpr\textwidth-1.5em}{Інвестор вклав гроші. Першого дня прибуток склав 50 доларів. Кожного наступного дня кількість збільшувалася вдвічі. За яку \textit{найменшу} кількість днів сумарна кількість доларів перевищить 5000? \nmtyear{2026}}
\vspace{0.3cm}

\answerTable{6}{9}{8}{5}{7}

\vspace{0.5cm}

\noindent\makebox[1.5em][l]{\textbf{89.}}\parbox[t]{\dimexpr\textwidth-1.5em}{Марійка викладала відео. Першого дня відео набрало 10 переглядів. Кожного наступного дня кількість збільшувалася вдвічі. За яку \textit{найменшу} кількість днів сумарна кількість переглядів перевищить 2000? \nmtyear{2026}}
\vspace{0.3cm}

\answerTable{9}{6}{7}{10}{8}

\vspace{0.5cm}

\noindent\makebox[1.5em][l]{\textbf{90.}}\parbox[t]{\dimexpr\textwidth-1.5em}{Інвестор вклав гроші. Першого дня прибуток склав 5 доларів. Кожного наступного дня кількість збільшувалася вдвічі. За яку \textit{найменшу} кількість днів сумарна кількість доларів перевищить 2000? \nmtyear{2026}}
\vspace{0.3cm}

\answerTable{11}{9}{10}{8}{7}

\vspace{0.5cm}

\noindent\makebox[1.5em][l]{\textbf{91.}}\parbox[t]{\dimexpr\textwidth-1.5em}{Марійка викладала відео. Першого дня відео набрало 50 переглядів. Кожного наступного дня кількість збільшувалася вдвічі. За яку \textit{найменшу} кількість днів сумарна кількість переглядів перевищить 5000? \nmtyear{2026}}
\vspace{0.3cm}

\answerTable{5}{8}{7}{6}{9}

\vspace{0.5cm}

\noindent\makebox[1.5em][l]{\textbf{92.}}\parbox[t]{\dimexpr\textwidth-1.5em}{Інвестор вклав гроші. Першого дня прибуток склав 100 доларів. Кожного наступного дня кількість збільшувалася вдвічі. За яку \textit{найменшу} кількість днів сумарна кількість доларів перевищить 5000? \nmtyear{2026}}
\vspace{0.3cm}

\answerTable{6}{7}{8}{5}{4}

\vspace{0.5cm}

\noindent\makebox[1.5em][l]{\textbf{93.}}\parbox[t]{\dimexpr\textwidth-1.5em}{Бактерія ділиться. Першого дня колонія налічувала 50 бактерій. Кожного наступного дня кількість збільшувалася вдвічі. За яку \textit{найменшу} кількість днів сумарна кількість бактерій перевищить 1000? \nmtyear{2026}}
\vspace{0.3cm}

\answerTable{4}{6}{3}{5}{7}

\vspace{0.5cm}

\noindent\makebox[1.5em][l]{\textbf{94.}}\parbox[t]{\dimexpr\textwidth-1.5em}{Бактерія ділиться. Першого дня колонія налічувала 100 бактерій. Кожного наступного дня кількість збільшувалася вдвічі. За яку \textit{найменшу} кількість днів сумарна кількість бактерій перевищить 500? \nmtyear{2026}}
\vspace{0.3cm}

\answerTable{2}{4}{3}{5}{1}

\vspace{0.5cm}

\noindent\makebox[1.5em][l]{\textbf{95.}}\parbox[t]{\dimexpr\textwidth-1.5em}{Марійка викладала відео. Першого дня відео набрало 50 переглядів. Кожного наступного дня кількість збільшувалася вдвічі. За яку \textit{найменшу} кількість днів сумарна кількість переглядів перевищить 1000? \nmtyear{2026}}
\vspace{0.3cm}

\answerTable{3}{6}{4}{7}{5}

\vspace{0.5cm}

\noindent\makebox[1.5em][l]{\textbf{96.}}\parbox[t]{\dimexpr\textwidth-1.5em}{Бактерія ділиться. Першого дня колонія налічувала 50 бактерій. Кожного наступного дня кількість збільшувалася вдвічі. За яку \textit{найменшу} кількість днів сумарна кількість бактерій перевищить 2000? \nmtyear{2026}}
\vspace{0.3cm}

\answerTable{8}{5}{4}{7}{6}

\vspace{0.5cm}

\noindent\makebox[1.5em][l]{\textbf{97.}}\parbox[t]{\dimexpr\textwidth-1.5em}{Бактерія ділиться. Першого дня колонія налічувала 10 бактерій. Кожного наступного дня кількість збільшувалася вдвічі. За яку \textit{найменшу} кількість днів сумарна кількість бактерій перевищить 500? \nmtyear{2026}}
\vspace{0.3cm}

\answerTable{6}{4}{8}{7}{5}

\vspace{0.5cm}

\noindent\makebox[1.5em][l]{\textbf{98.}}\parbox[t]{\dimexpr\textwidth-1.5em}{Бактерія ділиться. Першого дня колонія налічувала 50 бактерій. Кожного наступного дня кількість збільшувалася вдвічі. За яку \textit{найменшу} кількість днів сумарна кількість бактерій перевищить 1000? \nmtyear{2026}}
\vspace{0.3cm}

\answerTable{5}{6}{4}{7}{3}

\vspace{0.5cm}

\noindent\makebox[1.5em][l]{\textbf{99.}}\parbox[t]{\dimexpr\textwidth-1.5em}{Марійка викладала відео. Першого дня відео набрало 50 переглядів. Кожного наступного дня кількість збільшувалася вдвічі. За яку \textit{найменшу} кількість днів сумарна кількість переглядів перевищить 2000? \nmtyear{2026}}
\vspace{0.3cm}

\answerTable{5}{4}{8}{7}{6}

\vspace{0.5cm}

\noindent\makebox[1.5em][l]{\textbf{100.}}\parbox[t]{\dimexpr\textwidth-1.5em}{Бактерія ділиться. Першого дня колонія налічувала 5 бактерій. Кожного наступного дня кількість збільшувалася вдвічі. За яку \textit{найменшу} кількість днів сумарна кількість бактерій перевищить 500? \nmtyear{2026}}
\vspace{0.3cm}

\answerTable{8}{9}{7}{5}{6}

\vspace{0.5cm}


\end{document}
