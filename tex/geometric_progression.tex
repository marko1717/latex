\documentclass[14pt]{extarticle}
\usepackage{fontspec}
\usepackage{polyglossia}
\setdefaultlanguage{ukrainian}

\defaultfontfeatures{Ligatures=TeX}
\setmainfont{Liberation Serif}
\setsansfont{Liberation Sans}
\setmonofont{Liberation Mono}

\usepackage[a4paper,margin=1.5cm,bottom=2cm,top=2cm]{geometry}
\usepackage{amsmath,amssymb}
\usepackage{enumitem}
\usepackage{tikz}
\usepackage{pgfplots}
\pgfplotsset{compat=1.18}

\usetikzlibrary{calc,patterns,angles,quotes,intersections,babel}
\usetikzlibrary{3d}

\usepackage{xcolor}
\usepackage{array}
\usepackage{fancyhdr}
\usepackage{multirow}

% Кольори
\definecolor{headerblue}{RGB}{0, 102, 204}
\definecolor{yearcolor}{RGB}{128, 0, 128}

\pagestyle{fancy}
\fancyhf{}
\renewcommand{\headrulewidth}{0pt}
\fancyfoot[C]{\thepage}

\setlength{\headheight}{15pt}
\setlength{\headsep}{10pt}
\setlength{\footskip}{25pt}

\widowpenalty=10000
\clubpenalty=10000

% === КОМАНДИ ===

% Таблиця відповідей для відповідностей
\newcommand{\answerGrid}{
    \begingroup
    \renewcommand{\arraystretch}{1.3} 
    \setlength{\tabcolsep}{7pt} 
    \begin{tabular}{r|c|c|c|c|c|}
         \multicolumn{1}{c}{} & \multicolumn{1}{c}{\textbf{А}} & \multicolumn{1}{c}{\textbf{Б}} & \multicolumn{1}{c}{\textbf{В}} & \multicolumn{1}{c}{\textbf{Г}} & \multicolumn{1}{c}{\textbf{Д}} \\ \cline{2-6}
         \textbf{1} & & & & & \\ \cline{2-6}
         \textbf{2} & & & & & \\ \cline{2-6}
         \textbf{3} & & & & & \\ \cline{2-6}
    \end{tabular}
    \endgroup
}

% Макет для завдань на відповідність
\newcommand{\matchingLayout}[3]{
    \noindent
    \begin{minipage}[t]{0.40\textwidth}
        #1
    \end{minipage}%
    \hfill
    \begin{minipage}[t]{0.28\textwidth}
        #2
    \end{minipage}%
    \hfill
    \begin{minipage}[t]{0.30\textwidth}
        \vspace{0pt}
        \begin{flushright}
        #3
        \end{flushright}
    \end{minipage}
}

% Стандартна таблиця відповідей
\newcommand{\answerTable}[5]{
\begin{center}
\begin{tabular}{|*{5}{>{\centering\arraybackslash}m{2.8cm}|}}
\hline
\rule[-0.3cm]{0pt}{0.8cm}\textbf{А} & \textbf{Б} & \textbf{В} & \textbf{Г} & \textbf{Д} \\
\hline
\rule[-0.4cm]{0pt}{1.0cm}#1 & \rule[-0.4cm]{0pt}{1.0cm}#2 & \rule[-0.4cm]{0pt}{1.0cm}#3 & \rule[-0.4cm]{0pt}{1.0cm}#4 & \rule[-0.4cm]{0pt}{1.0cm}#5 \\
\hline
\end{tabular}
\end{center}
}

% Таблиця для відповідей із дробами
\newcommand{\answerTableTall}[5]{
\begin{center}
\begin{tabular}{|*{5}{>{\centering\arraybackslash}m{2.8cm}|}}
\hline
\rule[-0.3cm]{0pt}{0.8cm}\textbf{А} & \textbf{Б} & \textbf{В} & \textbf{Г} & \textbf{Д} \\
\hline
\rule[-0.9cm]{0pt}{2.0cm}#1 & 
\rule[-0.9cm]{0pt}{2.0cm}#2 & 
\rule[-0.9cm]{0pt}{2.0cm}#3 & 
\rule[-0.9cm]{0pt}{2.0cm}#4 & 
\rule[-0.9cm]{0pt}{2.0cm}#5 \\
\hline
\end{tabular}
\end{center}
}

\newcommand{\nmtyear}[1]{\hfill{\small\color{yearcolor}(AI Gen)}}

\begin{document}

\vspace{1cm}

\begin{center}
{\Large\textbf{\color{headerblue}ЗГЕНЕРОВАНІ ЗАВДАННЯ (AI)}}
\end{center}

\begin{center}
{\large Тема: \textbf{Геометрична прогресія}}
\end{center}

\vspace{0.5cm}
% === Geometric Progression: Find Term (b_1, b_2 -> b_n) ===
\noindent\makebox[1.5em][l]{\textbf{1.}}\parbox[t]{\dimexpr\textwidth-1.5em}{У геометричній прогресії $(b_n)$ відомо, що $b_1 = 10$, $b_2 = -20$. Визначте $b_{5}$. \nmtyear{2026}}
\vspace{0.3cm}

\answerTable{-110}{160}{0{,}625}{-320}{-80}

\vspace{0.5cm}

\noindent\makebox[1.5em][l]{\textbf{2.}}\parbox[t]{\dimexpr\textwidth-1.5em}{У геометричній прогресії $(b_n)$ відомо, що $b_1 = 64$, $b_2 = 128$. Визначте $b_{6}$. \nmtyear{2026}}
\vspace{0.3cm}

\answerTable{-2048}{1024}{4096}{2048}{384}

\vspace{0.5cm}

\noindent\makebox[1.5em][l]{\textbf{3.}}\parbox[t]{\dimexpr\textwidth-1.5em}{У геометричній прогресії $(b_n)$ відомо, що $b_1 = 5$, $b_2 = 20$. Визначте $b_{3}$. \nmtyear{2026}}
\vspace{0.3cm}

\answerTable{0{,}3125}{320}{35}{-80}{80}

\vspace{0.5cm}

\noindent\makebox[1.5em][l]{\textbf{4.}}\parbox[t]{\dimexpr\textwidth-1.5em}{У геометричній прогресії $(b_n)$ відомо, що $b_1 = 3$, $b_2 = 1{,}5$. Визначте $b_{6}$. \nmtyear{2026}}
\vspace{0.3cm}

\answerTable{-4{,}5}{96}{0{,}1875}{0{,}0938}{-0{,}0938}

\vspace{0.5cm}

\noindent\makebox[1.5em][l]{\textbf{5.}}\parbox[t]{\dimexpr\textwidth-1.5em}{У геометричній прогресії $(b_n)$ відомо, що $b_1 = 64$, $b_2 = 128$. Визначте $b_{6}$. \nmtyear{2026}}
\vspace{0.3cm}

\answerTable{-2048}{4096}{384}{1024}{2048}

\vspace{0.5cm}

\noindent\makebox[1.5em][l]{\textbf{6.}}\parbox[t]{\dimexpr\textwidth-1.5em}{У геометричній прогресії $(b_n)$ відомо, що $b_1 = 4$, $b_2 = 1$. Визначте $b_{5}$. \nmtyear{2026}}
\vspace{0.3cm}

\answerTable{0{,}0156}{0{,}0039}{-0{,}0156}{1024}{0{,}0625}

\vspace{0.5cm}

\noindent\makebox[1.5em][l]{\textbf{7.}}\parbox[t]{\dimexpr\textwidth-1.5em}{У геометричній прогресії $(b_n)$ відомо, що $b_1 = 64$, $b_2 = 192$. Визначте $b_{3}$. \nmtyear{2026}}
\vspace{0.3cm}

\answerTable{-576}{320}{192}{576}{1728}

\vspace{0.5cm}

\noindent\makebox[1.5em][l]{\textbf{8.}}\parbox[t]{\dimexpr\textwidth-1.5em}{У геометричній прогресії $(b_n)$ відомо, що $b_1 = 2$, $b_2 = 1$. Визначте $b_{5}$. \nmtyear{2026}}
\vspace{0.3cm}

\answerTable{-2}{-0{,}125}{32}{0{,}0625}{0{,}125}

\vspace{0.5cm}

\noindent\makebox[1.5em][l]{\textbf{9.}}\parbox[t]{\dimexpr\textwidth-1.5em}{У геометричній прогресії $(b_n)$ відомо, що $b_1 = 5$, $b_2 = 20$. Визначте $b_{3}$. \nmtyear{2026}}
\vspace{0.3cm}

\answerTable{320}{0{,}3125}{35}{80}{-80}

\vspace{0.5cm}

\noindent\makebox[1.5em][l]{\textbf{10.}}\parbox[t]{\dimexpr\textwidth-1.5em}{У геометричній прогресії $(b_n)$ відомо, що $b_1 = 2$, $b_2 = 0{,}5$. Визначте $b_{7}$. \nmtyear{2026}}
\vspace{0.3cm}

\answerTable{-0{,}0005}{8192}{0{,}002}{0{,}0005}{-7}

\vspace{0.5cm}

\noindent\makebox[1.5em][l]{\textbf{11.}}\parbox[t]{\dimexpr\textwidth-1.5em}{У геометричній прогресії $(b_n)$ відомо, що $b_1 = 2$, $b_2 = 8$. Визначте $b_{5}$. \nmtyear{2026}}
\vspace{0.3cm}

\answerTable{2048}{128}{512}{26}{0{,}0078}

\vspace{0.5cm}

\noindent\makebox[1.5em][l]{\textbf{12.}}\parbox[t]{\dimexpr\textwidth-1.5em}{У геометричній прогресії $(b_n)$ відомо, що $b_1 = 81$, $b_2 = 40{,}5$. Визначте $b_{5}$. \nmtyear{2026}}
\vspace{0.3cm}

\answerTable{-5{,}0625}{2{,}5312}{-81}{5{,}0625}{10{,}125}

\vspace{0.5cm}

\noindent\makebox[1.5em][l]{\textbf{13.}}\parbox[t]{\dimexpr\textwidth-1.5em}{У геометричній прогресії $(b_n)$ відомо, що $b_1 = 64$, $b_2 = 32$. Визначте $b_{4}$. \nmtyear{2026}}
\vspace{0.3cm}

\answerTable{4}{16}{8}{-32}{512}

\vspace{0.5cm}

\noindent\makebox[1.5em][l]{\textbf{14.}}\parbox[t]{\dimexpr\textwidth-1.5em}{У геометричній прогресії $(b_n)$ відомо, що $b_1 = 3$, $b_2 = 9$. Визначте $b_{5}$. \nmtyear{2026}}
\vspace{0.3cm}

\answerTable{-243}{81}{243}{0{,}037}{729}

\vspace{0.5cm}

\noindent\makebox[1.5em][l]{\textbf{15.}}\parbox[t]{\dimexpr\textwidth-1.5em}{У геометричній прогресії $(b_n)$ відомо, що $b_1 = 64$, $b_2 = 256$. Визначте $b_{5}$. \nmtyear{2026}}
\vspace{0.3cm}

\answerTable{65536}{-16384}{832}{16384}{4096}

\vspace{0.5cm}

\noindent\makebox[1.5em][l]{\textbf{16.}}\parbox[t]{\dimexpr\textwidth-1.5em}{У геометричній прогресії $(b_n)$ відомо, що $b_1 = 16$, $b_2 = 64$. Визначте $b_{4}$. \nmtyear{2026}}
\vspace{0.3cm}

\answerTable{4096}{0{,}25}{-1024}{256}{1024}

\vspace{0.5cm}

\noindent\makebox[1.5em][l]{\textbf{17.}}\parbox[t]{\dimexpr\textwidth-1.5em}{У геометричній прогресії $(b_n)$ відомо, що $b_1 = 3$, $b_2 = 1{,}5$. Визначте $b_{6}$. \nmtyear{2026}}
\vspace{0.3cm}

\answerTable{0{,}1875}{0{,}0938}{96}{-0{,}0938}{-4{,}5}

\vspace{0.5cm}

\noindent\makebox[1.5em][l]{\textbf{18.}}\parbox[t]{\dimexpr\textwidth-1.5em}{У геометричній прогресії $(b_n)$ відомо, що $b_1 = 81$, $b_2 = 20{,}25$. Визначте $b_{6}$. \nmtyear{2026}}
\vspace{0.3cm}

\answerTable{0{,}3164}{-222{,}75}{82944}{0{,}0791}{0{,}0198}

\vspace{0.5cm}

\noindent\makebox[1.5em][l]{\textbf{19.}}\parbox[t]{\dimexpr\textwidth-1.5em}{У геометричній прогресії $(b_n)$ відомо, що $b_1 = 16$, $b_2 = 64$. Визначте $b_{3}$. \nmtyear{2026}}
\vspace{0.3cm}

\answerTable{64}{-256}{112}{1024}{256}

\vspace{0.5cm}

\noindent\makebox[1.5em][l]{\textbf{20.}}\parbox[t]{\dimexpr\textwidth-1.5em}{У геометричній прогресії $(b_n)$ відомо, що $b_1 = 4$, $b_2 = 2$. Визначте $b_{6}$. \nmtyear{2026}}
\vspace{0.3cm}

\answerTable{0{,}125}{0{,}0625}{128}{-6}{-0{,}125}

\vspace{0.5cm}

% === Geometric Progression: Term Ratio ===
\noindent\makebox[1.5em][l]{\textbf{21.}}\parbox[t]{\dimexpr\textwidth-1.5em}{У геометричній прогресії $(b_n)$ відомо, що $b_1 = 2$, $b_2 = 6$. Обчисліть $\dfrac{b_{7}}{b_{10}}$. \nmtyear{2026}}
\vspace{0.3cm}

\answerTable{27}{-37908}{0{,}3333}{0{,}037}{3}

\vspace{0.5cm}

\noindent\makebox[1.5em][l]{\textbf{22.}}\parbox[t]{\dimexpr\textwidth-1.5em}{У геометричній прогресії $(b_n)$ відомо, що $b_1 = 5$, $b_2 = 2{,}5$. Обчисліть $\dfrac{b_{6}}{b_{7}}$. \nmtyear{2026}}
\vspace{0.3cm}

\answerTable{2}{0{,}5}{0}{9}{0{,}0781}

\vspace{0.5cm}

\noindent\makebox[1.5em][l]{\textbf{23.}}\parbox[t]{\dimexpr\textwidth-1.5em}{У геометричній прогресії $(b_n)$ відомо, що $b_1 = 32$, $b_2 = 128$. Обчисліть $\dfrac{b_{3}}{b_{5}}$. \nmtyear{2026}}
\vspace{0.3cm}

\answerTable{0{,}25}{16}{0{,}0625}{4}{-7680}

\vspace{0.5cm}

\noindent\makebox[1.5em][l]{\textbf{24.}}\parbox[t]{\dimexpr\textwidth-1.5em}{У геометричній прогресії $(b_n)$ відомо, що $b_1 = 2$, $b_2 = 4$. Обчисліть $\dfrac{b_{8}}{b_{11}}$. \nmtyear{2026}}
\vspace{0.3cm}

\answerTable{0{,}125}{-1792}{8}{2}{0{,}5}

\vspace{0.5cm}

\noindent\makebox[1.5em][l]{\textbf{25.}}\parbox[t]{\dimexpr\textwidth-1.5em}{У геометричній прогресії $(b_n)$ відомо, що $b_1 = 32$, $b_2 = 64$. Обчисліть $\dfrac{b_{8}}{b_{11}}$. \nmtyear{2026}}
\vspace{0.3cm}

\answerTable{2}{0{,}5}{8}{-28672}{0{,}125}

\vspace{0.5cm}

\noindent\makebox[1.5em][l]{\textbf{26.}}\parbox[t]{\dimexpr\textwidth-1.5em}{У геометричній прогресії $(b_n)$ відомо, що $b_1 = 4$, $b_2 = 20$. Обчисліть $\dfrac{b_{5}}{b_{8}}$. \nmtyear{2026}}
\vspace{0.3cm}

\answerTable{125}{0{,}2}{5}{-310000}{0{,}008}

\vspace{0.5cm}

\noindent\makebox[1.5em][l]{\textbf{27.}}\parbox[t]{\dimexpr\textwidth-1.5em}{У геометричній прогресії $(b_n)$ відомо, що $b_1 = 5$, $b_2 = 50$. Обчисліть $\dfrac{b_{6}}{b_{7}}$. \nmtyear{2026}}
\vspace{0.3cm}

\answerTable{77}{-4500000}{0{,}1}{10}{54}

\vspace{0.5cm}

\noindent\makebox[1.5em][l]{\textbf{28.}}\parbox[t]{\dimexpr\textwidth-1.5em}{У геометричній прогресії $(b_n)$ відомо, що $b_1 = 10$, $b_2 = 5$. Обчисліть $\dfrac{b_{7}}{b_{9}}$. \nmtyear{2026}}
\vspace{0.3cm}

\answerTable{0{,}25}{0{,}1172}{2}{4}{0{,}5}

\vspace{0.5cm}

\noindent\makebox[1.5em][l]{\textbf{29.}}\parbox[t]{\dimexpr\textwidth-1.5em}{У геометричній прогресії $(b_n)$ відомо, що $b_1 = 5$, $b_2 = 20$. Обчисліть $\dfrac{b_{6}}{b_{8}}$. \nmtyear{2026}}
\vspace{0.3cm}

\answerTable{16}{0{,}25}{4}{-76800}{0{,}0625}

\vspace{0.5cm}

\noindent\makebox[1.5em][l]{\textbf{30.}}\parbox[t]{\dimexpr\textwidth-1.5em}{У геометричній прогресії $(b_n)$ відомо, що $b_1 = 5$, $b_2 = 50$. Обчисліть $\dfrac{b_{3}}{b_{6}}$. \nmtyear{2026}}
\vspace{0.3cm}

\answerTable{1000}{10}{0{,}1}{0{,}001}{-499500}

\vspace{0.5cm}

\noindent\makebox[1.5em][l]{\textbf{31.}}\parbox[t]{\dimexpr\textwidth-1.5em}{У геометричній прогресії $(b_n)$ відомо, що $b_1 = 2$, $b_2 = 20$. Обчисліть $\dfrac{b_{7}}{b_{8}}$. \nmtyear{2026}}
\vspace{0.3cm}

\answerTable{0{,}1}{35}{74}{10}{-18000000}

\vspace{0.5cm}

\noindent\makebox[1.5em][l]{\textbf{32.}}\parbox[t]{\dimexpr\textwidth-1.5em}{У геометричній прогресії $(b_n)$ відомо, що $b_1 = 32$, $b_2 = 6{,}4$. Обчисліть $\dfrac{b_{5}}{b_{7}}$. \nmtyear{2026}}
\vspace{0.3cm}

\answerTable{0{,}0492}{5}{0{,}2}{0{,}04}{25{,}0}

\vspace{0.5cm}

\noindent\makebox[1.5em][l]{\textbf{33.}}\parbox[t]{\dimexpr\textwidth-1.5em}{У геометричній прогресії $(b_n)$ відомо, що $b_1 = 4$, $b_2 = 0{,}8$. Обчисліть $\dfrac{b_{8}}{b_{11}}$. \nmtyear{2026}}
\vspace{0.3cm}

\answerTable{0{,}0001}{0{,}2}{125{,}0}{0{,}008}{5}

\vspace{0.5cm}

\noindent\makebox[1.5em][l]{\textbf{34.}}\parbox[t]{\dimexpr\textwidth-1.5em}{У геометричній прогресії $(b_n)$ відомо, що $b_1 = 2$, $b_2 = 1$. Обчисліть $\dfrac{b_{6}}{b_{9}}$. \nmtyear{2026}}
\vspace{0.3cm}

\answerTable{8}{0{,}125}{2}{0{,}0547}{0{,}5}

\vspace{0.5cm}

\noindent\makebox[1.5em][l]{\textbf{35.}}\parbox[t]{\dimexpr\textwidth-1.5em}{У геометричній прогресії $(b_n)$ відомо, що $b_1 = 4$, $b_2 = 8$. Обчисліть $\dfrac{b_{7}}{b_{9}}$. \nmtyear{2026}}
\vspace{0.3cm}

\answerTable{0{,}25}{2}{0{,}5}{4}{-768}

\vspace{0.5cm}

\noindent\makebox[1.5em][l]{\textbf{36.}}\parbox[t]{\dimexpr\textwidth-1.5em}{У геометричній прогресії $(b_n)$ відомо, що $b_1 = 10$, $b_2 = 50$. Обчисліть $\dfrac{b_{7}}{b_{8}}$. \nmtyear{2026}}
\vspace{0.3cm}

\answerTable{29}{90}{0{,}2}{-625000}{5}

\vspace{0.5cm}

\noindent\makebox[1.5em][l]{\textbf{37.}}\parbox[t]{\dimexpr\textwidth-1.5em}{У геометричній прогресії $(b_n)$ відомо, що $b_1 = 2$, $b_2 = 4$. Обчисліть $\dfrac{b_{8}}{b_{11}}$. \nmtyear{2026}}
\vspace{0.3cm}

\answerTable{0{,}125}{0{,}5}{8}{2}{-1792}

\vspace{0.5cm}

\noindent\makebox[1.5em][l]{\textbf{38.}}\parbox[t]{\dimexpr\textwidth-1.5em}{У геометричній прогресії $(b_n)$ відомо, що $b_1 = 10$, $b_2 = 5$. Обчисліть $\dfrac{b_{7}}{b_{8}}$. \nmtyear{2026}}
\vspace{0.3cm}

\answerTable{0{,}0781}{10}{0{,}5}{2}{3}

\vspace{0.5cm}

\noindent\makebox[1.5em][l]{\textbf{39.}}\parbox[t]{\dimexpr\textwidth-1.5em}{У геометричній прогресії $(b_n)$ відомо, що $b_1 = 10$, $b_2 = 40$. Обчисліть $\dfrac{b_{5}}{b_{7}}$. \nmtyear{2026}}
\vspace{0.3cm}

\answerTable{4}{0{,}25}{16}{0{,}0625}{-38400}

\vspace{0.5cm}

\noindent\makebox[1.5em][l]{\textbf{40.}}\parbox[t]{\dimexpr\textwidth-1.5em}{У геометричній прогресії $(b_n)$ відомо, що $b_1 = 10$, $b_2 = 2$. Обчисліть $\dfrac{b_{6}}{b_{8}}$. \nmtyear{2026}}
\vspace{0.3cm}

\answerTable{0{,}0031}{0{,}04}{5}{0{,}2}{25{,}0}

\vspace{0.5cm}

% === Geometric Progression: Formula ===
\noindent\makebox[1.5em][l]{\textbf{41.}}\parbox[t]{\dimexpr\textwidth-1.5em}{Послідовність задано формулою $n$-го члена $b_n = 4 \cdot 2^n$. Визначте 5-й член цієї послідовності. \nmtyear{2026}}
\vspace{0.3cm}

\answerTable{129}{128}{256}{127}{-128}

\vspace{0.5cm}

\noindent\makebox[1.5em][l]{\textbf{42.}}\parbox[t]{\dimexpr\textwidth-1.5em}{Послідовність задано формулою $n$-го члена $b_n = 5 \cdot 4^n$. Визначте 3-й член цієї послідовності. \nmtyear{2026}}
\vspace{0.3cm}

\answerTable{-320}{321}{320}{640}{319}

\vspace{0.5cm}

\noindent\makebox[1.5em][l]{\textbf{43.}}\parbox[t]{\dimexpr\textwidth-1.5em}{Послідовність задано формулою $n$-го члена $b_n = 3 \cdot 2^n + 3n$. Визначте 6-й член цієї послідовності. \nmtyear{2026}}
\vspace{0.3cm}

\answerTable{209}{-210}{420}{211}{210}

\vspace{0.5cm}

\noindent\makebox[1.5em][l]{\textbf{44.}}\parbox[t]{\dimexpr\textwidth-1.5em}{Послідовність задано формулою $n$-го члена $b_n = 0{,}8 \cdot 2^n + 1n$. Визначте 6-й член цієї послідовності. \nmtyear{2026}}
\vspace{0.3cm}

\answerTable{114{,}4}{57{,}2}{56{,}2}{-57{,}2}{58{,}2}

\vspace{0.5cm}

\noindent\makebox[1.5em][l]{\textbf{45.}}\parbox[t]{\dimexpr\textwidth-1.5em}{Послідовність задано формулою $n$-го члена $b_n = 0{,}8 \cdot 4^n + 3n$. Визначте 3-й член цієї послідовності. \nmtyear{2026}}
\vspace{0.3cm}

\answerTable{59{,}2}{60{,}2}{120{,}4}{-60{,}2}{61{,}2}

\vspace{0.5cm}

\noindent\makebox[1.5em][l]{\textbf{46.}}\parbox[t]{\dimexpr\textwidth-1.5em}{Послідовність задано формулою $n$-го члена $b_n = \dfrac{(-1)^n}{n}$. Визначте 6-й член цієї послідовності. \nmtyear{2026}}
\vspace{0.3cm}

\answerTable{1{,}167}{-0{,}167}{0{,}333}{0{,}167}{-0{,}833}

\vspace{0.5cm}

\noindent\makebox[1.5em][l]{\textbf{47.}}\parbox[t]{\dimexpr\textwidth-1.5em}{Послідовність задано формулою $n$-го члена $b_n = \dfrac{(-1)^n}{n}$. Визначте 4-й член цієї послідовності. \nmtyear{2026}}
\vspace{0.3cm}

\answerTable{1{,}25}{0{,}25}{-0{,}25}{0{,}5}{-0{,}75}

\vspace{0.5cm}

\noindent\makebox[1.5em][l]{\textbf{48.}}\parbox[t]{\dimexpr\textwidth-1.5em}{Геометричну прогресію задано формулою $n$-го члена $b_n = 5 \cdot 2^{n-4}$. Визначте 5-й член цієї прогресії. \nmtyear{2026}}
\vspace{0.3cm}

\answerTable{9}{20}{11}{-10}{10}

\vspace{0.5cm}

\noindent\makebox[1.5em][l]{\textbf{49.}}\parbox[t]{\dimexpr\textwidth-1.5em}{Послідовність задано формулою $n$-го члена $b_n = \dfrac{(-1)^n}{n}$. Визначте 5-й член цієї послідовності. \nmtyear{2026}}
\vspace{0.3cm}

\answerTable{0{,}2}{-0{,}4}{-0{,}2}{0{,}8}{-1{,}2}

\vspace{0.5cm}

\noindent\makebox[1.5em][l]{\textbf{50.}}\parbox[t]{\dimexpr\textwidth-1.5em}{Послідовність задано формулою $n$-го члена $b_n = 0{,}8 \cdot 4^n + 1n$. Визначте 4-й член цієї послідовності. \nmtyear{2026}}
\vspace{0.3cm}

\answerTable{208{,}8}{209{,}8}{207{,}8}{-208{,}8}{417{,}6}

\vspace{0.5cm}

\noindent\makebox[1.5em][l]{\textbf{51.}}\parbox[t]{\dimexpr\textwidth-1.5em}{Послідовність задано формулою $n$-го члена $b_n = 4 \cdot 4^n + 1n$. Визначте 6-й член цієї послідовності. \nmtyear{2026}}
\vspace{0.3cm}

\answerTable{16389}{-16390}{32780}{16391}{16390}

\vspace{0.5cm}

\noindent\makebox[1.5em][l]{\textbf{52.}}\parbox[t]{\dimexpr\textwidth-1.5em}{Послідовність задано формулою $n$-го члена $b_n = (-1)^n \cdot n$. Визначте 5-й член цієї послідовності. \nmtyear{2026}}
\vspace{0.3cm}

\answerTable{-5}{5}{-4}{-6}{-10}

\vspace{0.5cm}

\noindent\makebox[1.5em][l]{\textbf{53.}}\parbox[t]{\dimexpr\textwidth-1.5em}{Послідовність задано формулою $n$-го члена $b_n = (-1)^n \cdot n$. Визначте 3-й член цієї послідовності. \nmtyear{2026}}
\vspace{0.3cm}

\answerTable{-6}{3}{-3}{-4}{-2}

\vspace{0.5cm}

\noindent\makebox[1.5em][l]{\textbf{54.}}\parbox[t]{\dimexpr\textwidth-1.5em}{Послідовність задано формулою $n$-го члена $b_n = \dfrac{(-1)^n}{n}$. Визначте 4-й член цієї послідовності. \nmtyear{2026}}
\vspace{0.3cm}

\answerTable{1{,}25}{0{,}5}{0{,}25}{-0{,}75}{-0{,}25}

\vspace{0.5cm}

\noindent\makebox[1.5em][l]{\textbf{55.}}\parbox[t]{\dimexpr\textwidth-1.5em}{Послідовність задано формулою $n$-го члена $b_n = 0{,}5 \cdot 3^n$. Визначте 5-й член цієї послідовності. \nmtyear{2026}}
\vspace{0.3cm}

\answerTable{-121{,}5}{122{,}5}{243}{121{,}5}{120{,}5}

\vspace{0.5cm}

\noindent\makebox[1.5em][l]{\textbf{56.}}\parbox[t]{\dimexpr\textwidth-1.5em}{Послідовність задано формулою $n$-го члена $b_n = (-1)^n \cdot n$. Визначте 5-й член цієї послідовності. \nmtyear{2026}}
\vspace{0.3cm}

\answerTable{-4}{-6}{-5}{5}{-10}

\vspace{0.5cm}

\noindent\makebox[1.5em][l]{\textbf{57.}}\parbox[t]{\dimexpr\textwidth-1.5em}{Геометричну прогресію задано формулою $n$-го члена $b_n = 5 \cdot 3^{n-1}$. Визначте 5-й член цієї прогресії. \nmtyear{2026}}
\vspace{0.3cm}

\answerTable{404}{406}{810}{405}{-405}

\vspace{0.5cm}

\noindent\makebox[1.5em][l]{\textbf{58.}}\parbox[t]{\dimexpr\textwidth-1.5em}{Геометричну прогресію задано формулою $n$-го члена $b_n = 4 \cdot 3^{n-4}$. Визначте 5-й член цієї прогресії. \nmtyear{2026}}
\vspace{0.3cm}

\answerTable{-12}{24}{11}{13}{12}

\vspace{0.5cm}

\noindent\makebox[1.5em][l]{\textbf{59.}}\parbox[t]{\dimexpr\textwidth-1.5em}{Послідовність задано формулою $n$-го члена $b_n = (-1)^n \cdot n$. Визначте 6-й член цієї послідовності. \nmtyear{2026}}
\vspace{0.3cm}

\answerTable{6}{12}{7}{5}{-6}

\vspace{0.5cm}

\noindent\makebox[1.5em][l]{\textbf{60.}}\parbox[t]{\dimexpr\textwidth-1.5em}{Геометричну прогресію задано формулою $n$-го члена $b_n = 0{,}8 \cdot 3^{n-4}$. Визначте 5-й член цієї прогресії. \nmtyear{2026}}
\vspace{0.3cm}

\answerTable{3{,}4}{-2{,}4}{2{,}4}{4{,}8}{1{,}4}

\vspace{0.5cm}

% === Geometric Progression: Sum ===
\noindent\makebox[1.5em][l]{\textbf{61.}}\parbox[t]{\dimexpr\textwidth-1.5em}{Знайдіть суму 3 перших членів геометричної прогресії $(b_n)$, у якої $b_2 = -6$, а знаменник $q = 0{,}5$. \nmtyear{2026}}
\vspace{0.3cm}

\answerTable{-42}{-3}{-4}{21}{-21}

\vspace{0.5cm}

\noindent\makebox[1.5em][l]{\textbf{62.}}\parbox[t]{\dimexpr\textwidth-1.5em}{Знайдіть суму 3 перших членів геометричної прогресії $(b_n)$, у якої $b_2 = 12$, а знаменник $q = -3$. \nmtyear{2026}}
\vspace{0.3cm}

\answerTable{28}{9{,}333}{-28}{-76}{-36}

\vspace{0.5cm}

\noindent\makebox[1.5em][l]{\textbf{63.}}\parbox[t]{\dimexpr\textwidth-1.5em}{Знайдіть суму 5 перших членів геометричної прогресії $(b_n)$, у якої $b_2 = 18$, а знаменник $q = -2$. \nmtyear{2026}}
\vspace{0.3cm}

\answerTable{49{,}5}{-82}{-99}{-144}{99}

\vspace{0.5cm}

\noindent\makebox[1.5em][l]{\textbf{64.}}\parbox[t]{\dimexpr\textwidth-1.5em}{Знайдіть суму 3 перших членів геометричної прогресії $(b_n)$, у якої $b_2 = -6$, а знаменник $q = 2$. \nmtyear{2026}}
\vspace{0.3cm}

\answerTable{21}{-12}{-36}{-10{,}5}{-21}

\vspace{0.5cm}

\noindent\makebox[1.5em][l]{\textbf{65.}}\parbox[t]{\dimexpr\textwidth-1.5em}{Знайдіть суму 5 перших членів геометричної прогресії $(b_n)$, у якої $b_2 = 4$, а знаменник $q = -3$. \nmtyear{2026}}
\vspace{0.3cm}

\answerTable{-108}{-81{,}333}{27{,}111}{-51{,}333}{81{,}333}

\vspace{0.5cm}

\noindent\makebox[1.5em][l]{\textbf{66.}}\parbox[t]{\dimexpr\textwidth-1.5em}{Знайдіть суму 4 перших членів геометричної прогресії $(b_n)$, у якої $b_2 = -4$, а знаменник $q = -2$. \nmtyear{2026}}
\vspace{0.3cm}

\answerTable{-16}{5}{9}{10}{-10}

\vspace{0.5cm}

\noindent\makebox[1.5em][l]{\textbf{67.}}\parbox[t]{\dimexpr\textwidth-1.5em}{Знайдіть суму 5 перших членів геометричної прогресії $(b_n)$, у якої $b_2 = -4$, а знаменник $q = -2$. \nmtyear{2026}}
\vspace{0.3cm}

\answerTable{-22}{36}{32}{-11}{22}

\vspace{0.5cm}

\noindent\makebox[1.5em][l]{\textbf{68.}}\parbox[t]{\dimexpr\textwidth-1.5em}{Знайдіть суму 4 перших членів геометричної прогресії $(b_n)$, у якої $b_2 = 12$, а знаменник $q = 2$. \nmtyear{2026}}
\vspace{0.3cm}

\answerTable{48}{45}{90}{106}{-90}

\vspace{0.5cm}

\noindent\makebox[1.5em][l]{\textbf{69.}}\parbox[t]{\dimexpr\textwidth-1.5em}{Знайдіть суму 5 перших членів геометричної прогресії $(b_n)$, у якої $b_2 = -6$, а знаменник $q = -3$. \nmtyear{2026}}
\vspace{0.3cm}

\answerTable{162}{-40{,}667}{123}{122}{-122}

\vspace{0.5cm}

\noindent\makebox[1.5em][l]{\textbf{70.}}\parbox[t]{\dimexpr\textwidth-1.5em}{Знайдіть суму 4 перших членів геометричної прогресії $(b_n)$, у якої $b_2 = -4$, а знаменник $q = -3$. \nmtyear{2026}}
\vspace{0.3cm}

\answerTable{23{,}333}{8{,}889}{-26{,}667}{-36}{26{,}667}

\vspace{0.5cm}

\noindent\makebox[1.5em][l]{\textbf{71.}}\parbox[t]{\dimexpr\textwidth-1.5em}{Знайдіть суму 5 перших членів геометричної прогресії $(b_n)$, у якої $b_2 = 18$, а знаменник $q = -3$. \nmtyear{2026}}
\vspace{0.3cm}

\answerTable{-366}{122}{-395}{-486}{366}

\vspace{0.5cm}

\noindent\makebox[1.5em][l]{\textbf{72.}}\parbox[t]{\dimexpr\textwidth-1.5em}{Знайдіть суму 3 перших членів геометричної прогресії $(b_n)$, у якої $b_2 = 6$, а знаменник $q = 3$. \nmtyear{2026}}
\vspace{0.3cm}

\answerTable{8{,}667}{26}{1}{18}{-26}

\vspace{0.5cm}

\noindent\makebox[1.5em][l]{\textbf{73.}}\parbox[t]{\dimexpr\textwidth-1.5em}{Знайдіть суму 5 перших членів геометричної прогресії $(b_n)$, у якої $b_2 = 12$, а знаменник $q = 2$. \nmtyear{2026}}
\vspace{0.3cm}

\answerTable{-186}{186}{96}{93}{190}

\vspace{0.5cm}

\noindent\makebox[1.5em][l]{\textbf{74.}}\parbox[t]{\dimexpr\textwidth-1.5em}{Знайдіть суму 4 перших членів геометричної прогресії $(b_n)$, у якої $b_2 = 18$, а знаменник $q = 0{,}5$. \nmtyear{2026}}
\vspace{0.3cm}

\answerTable{135}{29{,}5}{4{,}5}{67{,}5}{-67{,}5}

\vspace{0.5cm}

\noindent\makebox[1.5em][l]{\textbf{75.}}\parbox[t]{\dimexpr\textwidth-1.5em}{Знайдіть суму 3 перших членів геометричної прогресії $(b_n)$, у якої $b_2 = -6$, а знаменник $q = 3$. \nmtyear{2026}}
\vspace{0.3cm}

\answerTable{-8{,}667}{26}{-26}{-18}{-33}

\vspace{0.5cm}

\noindent\makebox[1.5em][l]{\textbf{76.}}\parbox[t]{\dimexpr\textwidth-1.5em}{Знайдіть суму 4 перших членів геометричної прогресії $(b_n)$, у якої $b_2 = 12$, а знаменник $q = -3$. \nmtyear{2026}}
\vspace{0.3cm}

\answerTable{108}{-80}{106}{-26{,}667}{80}

\vspace{0.5cm}

\noindent\makebox[1.5em][l]{\textbf{77.}}\parbox[t]{\dimexpr\textwidth-1.5em}{Знайдіть суму 3 перших членів геометричної прогресії $(b_n)$, у якої $b_2 = 12$, а знаменник $q = 3$. \nmtyear{2026}}
\vspace{0.3cm}

\answerTable{17{,}333}{-52}{73}{52}{36}

\vspace{0.5cm}

\noindent\makebox[1.5em][l]{\textbf{78.}}\parbox[t]{\dimexpr\textwidth-1.5em}{Знайдіть суму 5 перших членів геометричної прогресії $(b_n)$, у якої $b_2 = -4$, а знаменник $q = -3$. \nmtyear{2026}}
\vspace{0.3cm}

\answerTable{-27{,}111}{-81{,}333}{108}{96{,}333}{81{,}333}

\vspace{0.5cm}

\noindent\makebox[1.5em][l]{\textbf{79.}}\parbox[t]{\dimexpr\textwidth-1.5em}{Знайдіть суму 3 перших членів геометричної прогресії $(b_n)$, у якої $b_2 = -6$, а знаменник $q = 3$. \nmtyear{2026}}
\vspace{0.3cm}

\answerTable{-18}{-8{,}667}{-16}{-26}{26}

\vspace{0.5cm}

\noindent\makebox[1.5em][l]{\textbf{80.}}\parbox[t]{\dimexpr\textwidth-1.5em}{Знайдіть суму 5 перших членів геометричної прогресії $(b_n)$, у якої $b_2 = 18$, а знаменник $q = 2$. \nmtyear{2026}}
\vspace{0.3cm}

\answerTable{139{,}5}{144}{296}{-279}{279}

\vspace{0.5cm}

% === Geometric Progression: Word Problem ===
\noindent\makebox[1.5em][l]{\textbf{81.}}\parbox[t]{\dimexpr\textwidth-1.5em}{Інвестор вклав гроші. Першого дня прибуток склав 5 доларів. Кожного наступного дня кількість збільшувалася вдвічі. За яку \textit{найменшу} кількість днів сумарна кількість доларів перевищить 500? \nmtyear{2026}}
\vspace{0.3cm}

\answerTable{8}{9}{5}{6}{7}

\vspace{0.5cm}

\noindent\makebox[1.5em][l]{\textbf{82.}}\parbox[t]{\dimexpr\textwidth-1.5em}{Бактерія ділиться. Першого дня колонія налічувала 50 бактерій. Кожного наступного дня кількість збільшувалася вдвічі. За яку \textit{найменшу} кількість днів сумарна кількість бактерій перевищить 2000? \nmtyear{2026}}
\vspace{0.3cm}

\answerTable{4}{7}{5}{6}{8}

\vspace{0.5cm}

\noindent\makebox[1.5em][l]{\textbf{83.}}\parbox[t]{\dimexpr\textwidth-1.5em}{Інвестор вклав гроші. Першого дня прибуток склав 100 доларів. Кожного наступного дня кількість збільшувалася вдвічі. За яку \textit{найменшу} кількість днів сумарна кількість доларів перевищить 500? \nmtyear{2026}}
\vspace{0.3cm}

\answerTable{3}{2}{4}{1}{5}

\vspace{0.5cm}

\noindent\makebox[1.5em][l]{\textbf{84.}}\parbox[t]{\dimexpr\textwidth-1.5em}{Бактерія ділиться. Першого дня колонія налічувала 100 бактерій. Кожного наступного дня кількість збільшувалася вдвічі. За яку \textit{найменшу} кількість днів сумарна кількість бактерій перевищить 2000? \nmtyear{2026}}
\vspace{0.3cm}

\answerTable{3}{6}{7}{5}{4}

\vspace{0.5cm}

\noindent\makebox[1.5em][l]{\textbf{85.}}\parbox[t]{\dimexpr\textwidth-1.5em}{Марійка викладала відео. Першого дня відео набрало 5 переглядів. Кожного наступного дня кількість збільшувалася вдвічі. За яку \textit{найменшу} кількість днів сумарна кількість переглядів перевищить 5000? \nmtyear{2026}}
\vspace{0.3cm}

\answerTable{10}{11}{12}{9}{8}

\vspace{0.5cm}

\noindent\makebox[1.5em][l]{\textbf{86.}}\parbox[t]{\dimexpr\textwidth-1.5em}{Інвестор вклав гроші. Першого дня прибуток склав 5 доларів. Кожного наступного дня кількість збільшувалася вдвічі. За яку \textit{найменшу} кількість днів сумарна кількість доларів перевищить 1000? \nmtyear{2026}}
\vspace{0.3cm}

\answerTable{7}{9}{10}{6}{8}

\vspace{0.5cm}

\noindent\makebox[1.5em][l]{\textbf{87.}}\parbox[t]{\dimexpr\textwidth-1.5em}{Марійка викладала відео. Першого дня відео набрало 5 переглядів. Кожного наступного дня кількість збільшувалася вдвічі. За яку \textit{найменшу} кількість днів сумарна кількість переглядів перевищить 2000? \nmtyear{2026}}
\vspace{0.3cm}

\answerTable{11}{10}{7}{9}{8}

\vspace{0.5cm}

\noindent\makebox[1.5em][l]{\textbf{88.}}\parbox[t]{\dimexpr\textwidth-1.5em}{Марійка викладала відео. Першого дня відео набрало 50 переглядів. Кожного наступного дня кількість збільшувалася вдвічі. За яку \textit{найменшу} кількість днів сумарна кількість переглядів перевищить 5000? \nmtyear{2026}}
\vspace{0.3cm}

\answerTable{8}{9}{6}{7}{5}

\vspace{0.5cm}

\noindent\makebox[1.5em][l]{\textbf{89.}}\parbox[t]{\dimexpr\textwidth-1.5em}{Інвестор вклав гроші. Першого дня прибуток склав 5 доларів. Кожного наступного дня кількість збільшувалася вдвічі. За яку \textit{найменшу} кількість днів сумарна кількість доларів перевищить 2000? \nmtyear{2026}}
\vspace{0.3cm}

\answerTable{9}{11}{8}{7}{10}

\vspace{0.5cm}

\noindent\makebox[1.5em][l]{\textbf{90.}}\parbox[t]{\dimexpr\textwidth-1.5em}{Бактерія ділиться. Першого дня колонія налічувала 5 бактерій. Кожного наступного дня кількість збільшувалася вдвічі. За яку \textit{найменшу} кількість днів сумарна кількість бактерій перевищить 500? \nmtyear{2026}}
\vspace{0.3cm}

\answerTable{8}{5}{9}{6}{7}

\vspace{0.5cm}

\noindent\makebox[1.5em][l]{\textbf{91.}}\parbox[t]{\dimexpr\textwidth-1.5em}{Бактерія ділиться. Першого дня колонія налічувала 100 бактерій. Кожного наступного дня кількість збільшувалася вдвічі. За яку \textit{найменшу} кількість днів сумарна кількість бактерій перевищить 5000? \nmtyear{2026}}
\vspace{0.3cm}

\answerTable{8}{4}{5}{7}{6}

\vspace{0.5cm}

\noindent\makebox[1.5em][l]{\textbf{92.}}\parbox[t]{\dimexpr\textwidth-1.5em}{Бактерія ділиться. Першого дня колонія налічувала 50 бактерій. Кожного наступного дня кількість збільшувалася вдвічі. За яку \textit{найменшу} кількість днів сумарна кількість бактерій перевищить 500? \nmtyear{2026}}
\vspace{0.3cm}

\answerTable{5}{4}{2}{6}{3}

\vspace{0.5cm}

\noindent\makebox[1.5em][l]{\textbf{93.}}\parbox[t]{\dimexpr\textwidth-1.5em}{Бактерія ділиться. Першого дня колонія налічувала 5 бактерій. Кожного наступного дня кількість збільшувалася вдвічі. За яку \textit{найменшу} кількість днів сумарна кількість бактерій перевищить 5000? \nmtyear{2026}}
\vspace{0.3cm}

\answerTable{8}{11}{12}{10}{9}

\vspace{0.5cm}

\noindent\makebox[1.5em][l]{\textbf{94.}}\parbox[t]{\dimexpr\textwidth-1.5em}{Бактерія ділиться. Першого дня колонія налічувала 100 бактерій. Кожного наступного дня кількість збільшувалася вдвічі. За яку \textit{найменшу} кількість днів сумарна кількість бактерій перевищить 1000? \nmtyear{2026}}
\vspace{0.3cm}

\answerTable{6}{3}{2}{4}{5}

\vspace{0.5cm}

\noindent\makebox[1.5em][l]{\textbf{95.}}\parbox[t]{\dimexpr\textwidth-1.5em}{Марійка викладала відео. Першого дня відео набрало 5 переглядів. Кожного наступного дня кількість збільшувалася вдвічі. За яку \textit{найменшу} кількість днів сумарна кількість переглядів перевищить 1000? \nmtyear{2026}}
\vspace{0.3cm}

\answerTable{7}{6}{10}{9}{8}

\vspace{0.5cm}

\noindent\makebox[1.5em][l]{\textbf{96.}}\parbox[t]{\dimexpr\textwidth-1.5em}{Марійка викладала відео. Першого дня відео набрало 10 переглядів. Кожного наступного дня кількість збільшувалася вдвічі. За яку \textit{найменшу} кількість днів сумарна кількість переглядів перевищить 2000? \nmtyear{2026}}
\vspace{0.3cm}

\answerTable{8}{9}{10}{6}{7}

\vspace{0.5cm}

\noindent\makebox[1.5em][l]{\textbf{97.}}\parbox[t]{\dimexpr\textwidth-1.5em}{Бактерія ділиться. Першого дня колонія налічувала 100 бактерій. Кожного наступного дня кількість збільшувалася вдвічі. За яку \textit{найменшу} кількість днів сумарна кількість бактерій перевищить 2000? \nmtyear{2026}}
\vspace{0.3cm}

\answerTable{4}{7}{5}{6}{3}

\vspace{0.5cm}

\noindent\makebox[1.5em][l]{\textbf{98.}}\parbox[t]{\dimexpr\textwidth-1.5em}{Бактерія ділиться. Першого дня колонія налічувала 5 бактерій. Кожного наступного дня кількість збільшувалася вдвічі. За яку \textit{найменшу} кількість днів сумарна кількість бактерій перевищить 2000? \nmtyear{2026}}
\vspace{0.3cm}

\answerTable{7}{9}{11}{10}{8}

\vspace{0.5cm}

\noindent\makebox[1.5em][l]{\textbf{99.}}\parbox[t]{\dimexpr\textwidth-1.5em}{Інвестор вклав гроші. Першого дня прибуток склав 5 доларів. Кожного наступного дня кількість збільшувалася вдвічі. За яку \textit{найменшу} кількість днів сумарна кількість доларів перевищить 1000? \nmtyear{2026}}
\vspace{0.3cm}

\answerTable{9}{10}{7}{8}{6}

\vspace{0.5cm}

\noindent\makebox[1.5em][l]{\textbf{100.}}\parbox[t]{\dimexpr\textwidth-1.5em}{Бактерія ділиться. Першого дня колонія налічувала 50 бактерій. Кожного наступного дня кількість збільшувалася вдвічі. За яку \textit{найменшу} кількість днів сумарна кількість бактерій перевищить 2000? \nmtyear{2026}}
\vspace{0.3cm}

\answerTable{7}{6}{4}{8}{5}

\vspace{0.5cm}


\end{document}
