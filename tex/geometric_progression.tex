\documentclass[14pt]{extarticle}
\usepackage{fontspec}
\usepackage{polyglossia}
\setdefaultlanguage{ukrainian}

\defaultfontfeatures{Ligatures=TeX}
\setmainfont{Liberation Serif}
\setsansfont{Liberation Sans}
\setmonofont{Liberation Mono}

\usepackage[a4paper,margin=1.5cm,bottom=2cm,top=2cm]{geometry}
\usepackage{amsmath,amssymb}
\usepackage{enumitem}
\usepackage{tikz}
\usepackage{pgfplots}
\pgfplotsset{compat=1.16}

\usetikzlibrary{calc,patterns,angles,quotes,intersections,babel}
\usetikzlibrary{3d}
\definecolor{woodinner}{RGB}{222, 184, 135}
\definecolor{woodouter}{RGB}{139, 69, 19}
\usepackage{xcolor}
\usepackage{array}
\usepackage{fancyhdr}
\usepackage{multirow}

\definecolor{headerblue}{RGB}{0, 102, 204}
\definecolor{yearcolor}{RGB}{128, 0, 128}

\pagestyle{fancy}
\fancyhf{}
\renewcommand{\headrulewidth}{0pt}
\fancyfoot[C]{\thepage}

\setlength{\headheight}{15pt}
\setlength{\headsep}{10pt}
\setlength{\footskip}{25pt}

\widowpenalty=10000
\clubpenalty=10000

\newcommand{\answerTable}[5]{
\begin{center}
\begin{tabular}{|*{5}{>{\centering\arraybackslash}m{2.8cm}|}}
\hline
\rule[-0.3cm]{0pt}{0.8cm}\textbf{А} & \textbf{Б} & \textbf{В} & \textbf{Г} & \textbf{Д} \\
\hline
\rule[-0.4cm]{0pt}{1.0cm}#1 & \rule[-0.4cm]{0pt}{1.0cm}#2 & \rule[-0.4cm]{0pt}{1.0cm}#3 & \rule[-0.4cm]{0pt}{1.0cm}#4 & \rule[-0.4cm]{0pt}{1.0cm}#5 \\
\hline
\end{tabular}
\end{center}
}

\newcommand{\shortAnswer}{
\vspace{0.3cm}
\noindent\hspace{1cm}Відповідь: \framebox(18,18){}\framebox(18,18){}\framebox(18,18){}\framebox(18,18){}{,}\framebox(18,18){}\framebox(18,18){}\framebox(18,18){}
\vspace{0.5cm}
}

\newcommand{\nmtyear}[1]{\hfill{\small\color{yearcolor}(AI Gen)}}

\begin{document}

\begin{center}
{\Large\textbf{\color{headerblue}ЗГЕНЕРОВАНІ ЗАВДАННЯ (AI)}}
\end{center}

\begin{center}
{\large Тема: \textbf{Геометрична прогресія}}
\end{center}

\vspace{0.5cm}
% === Geometric Progression: Find Term (b_1, b_2 -> b_n) ===
\noindent\makebox[1.5em][l]{\textbf{1.}}\parbox[t]{\dimexpr\textwidth-1.5em}{У геометричній прогресії $(b_n)$ відомо, що $b_1 = 5$, $b_2 = 20$. Визначте $b_{5}$. \nmtyear{2026}}

\answerTable{0{,}0195}{65}{-1280}{5120}{1280}

\vspace{0.5cm}

\noindent\makebox[1.5em][l]{\textbf{2.}}\parbox[t]{\dimexpr\textwidth-1.5em}{У геометричній прогресії $(b_n)$ відомо, що $b_1 = 32$, $b_2 = 96$. Визначте $b_{4}$. \nmtyear{2026}}

\answerTable{224}{288}{1{,}1852}{864}{-864}

\vspace{0.5cm}

\noindent\makebox[1.5em][l]{\textbf{3.}}\parbox[t]{\dimexpr\textwidth-1.5em}{У геометричній прогресії $(b_n)$ відомо, що $b_1 = 10$, $b_2 = 20$. Визначте $b_{7}$. \nmtyear{2026}}

\answerTable{70}{-640}{640}{0{,}1562}{320}

\vspace{0.5cm}

\noindent\makebox[1.5em][l]{\textbf{4.}}\parbox[t]{\dimexpr\textwidth-1.5em}{У геометричній прогресії $(b_n)$ відомо, що $b_1 = 4$, $b_2 = 16$. Визначте $b_{5}$. \nmtyear{2026}}

\answerTable{0{,}0156}{-1024}{256}{4096}{1024}

\vspace{0.5cm}

\noindent\makebox[1.5em][l]{\textbf{5.}}\parbox[t]{\dimexpr\textwidth-1.5em}{У геометричній прогресії $(b_n)$ відомо, що $b_1 = 5$, $b_2 = -10$. Визначте $b_{6}$. \nmtyear{2026}}

\answerTable{-160}{-0{,}1562}{160}{-70}{320}

\vspace{0.5cm}

% === Geometric Progression: Term Ratio ===
\noindent\makebox[1.5em][l]{\textbf{6.}}\parbox[t]{\dimexpr\textwidth-1.5em}{У геометричній прогресії $(b_n)$ відомо, що $b_1 = 32$, $b_2 = 320$. Обчисліть $\dfrac{b_{6}}{b_{8}}$. \nmtyear{2026}}

\answerTable{10}{0{,}1}{-316800000}{100}{0{,}01}

\vspace{0.5cm}

\noindent\makebox[1.5em][l]{\textbf{7.}}\parbox[t]{\dimexpr\textwidth-1.5em}{У геометричній прогресії $(b_n)$ відомо, що $b_1 = 5$, $b_2 = 10$. Обчисліть $\dfrac{b_{7}}{b_{9}}$. \nmtyear{2026}}

\answerTable{0{,}5}{-960}{2}{0{,}25}{4}

\vspace{0.5cm}

\noindent\makebox[1.5em][l]{\textbf{8.}}\parbox[t]{\dimexpr\textwidth-1.5em}{У геометричній прогресії $(b_n)$ відомо, що $b_1 = 2$, $b_2 = 0{,}4$. Обчисліть $\dfrac{b_{8}}{b_{11}}$. \nmtyear{2026}}

\answerTable{125{,}0}{5}{0{,}008}{0{,}2}{0{,}0}

\vspace{0.5cm}

\noindent\makebox[1.5em][l]{\textbf{9.}}\parbox[t]{\dimexpr\textwidth-1.5em}{У геометричній прогресії $(b_n)$ відомо, що $b_1 = 4$, $b_2 = 8$. Обчисліть $\dfrac{b_{3}}{b_{5}}$. \nmtyear{2026}}

\answerTable{-48}{0{,}5}{0{,}25}{2}{4}

\vspace{0.5cm}

\noindent\makebox[1.5em][l]{\textbf{10.}}\parbox[t]{\dimexpr\textwidth-1.5em}{У геометричній прогресії $(b_n)$ відомо, що $b_1 = 5$, $b_2 = 15$. Обчисліть $\dfrac{b_{8}}{b_{9}}$. \nmtyear{2026}}

\answerTable{-21870}{97}{66}{0{,}3333}{3}

\vspace{0.5cm}

% === Geometric Progression: Formula ===
\noindent\makebox[1.5em][l]{\textbf{11.}}\parbox[t]{\dimexpr\textwidth-1.5em}{Послідовність задано формулою $n$-го члена $b_n = 4 \cdot 3^n + 2n$. Визначте 5-й член цієї послідовності. \nmtyear{2026}}

\answerTable{1964}{983}{981}{-982}{982}

\vspace{0.5cm}

\noindent\makebox[1.5em][l]{\textbf{12.}}\parbox[t]{\dimexpr\textwidth-1.5em}{Геометричну прогресію задано формулою $n$-го члена $b_n = 0{,}8 \cdot 3^{n-3}$. Визначте 3-й член цієї прогресії. \nmtyear{2026}}

\answerTable{-0{,}2}{-0{,}8}{1{,}6}{1{,}8}{0{,}8}

\vspace{0.5cm}

\noindent\makebox[1.5em][l]{\textbf{13.}}\parbox[t]{\dimexpr\textwidth-1.5em}{Геометричну прогресію задано формулою $n$-го члена $b_n = 5 \cdot 3^{n-3}$. Визначте 6-й член цієї прогресії. \nmtyear{2026}}

\answerTable{135}{-135}{134}{136}{270}

\vspace{0.5cm}

\noindent\makebox[1.5em][l]{\textbf{14.}}\parbox[t]{\dimexpr\textwidth-1.5em}{Геометричну прогресію задано формулою $n$-го члена $b_n = 3 \cdot 3^{n-1}$. Визначте 3-й член цієї прогресії. \nmtyear{2026}}

\answerTable{54}{28}{27}{-27}{26}

\vspace{0.5cm}

\noindent\makebox[1.5em][l]{\textbf{15.}}\parbox[t]{\dimexpr\textwidth-1.5em}{Геометричну прогресію задано формулою $n$-го члена $b_n = 0{,}5 \cdot 4^{n-4}$. Визначте 3-й член цієї прогресії. \nmtyear{2026}}

\answerTable{0{,}25}{-0{,}875}{1{,}125}{-0{,}125}{0{,}125}

\vspace{0.5cm}

% === Geometric Progression: Sum ===
\noindent\makebox[1.5em][l]{\textbf{16.}}\parbox[t]{\dimexpr\textwidth-1.5em}{Знайдіть суму 4 перших членів геометричної прогресії $(b_n)$, у якої $b_2 = 12$, а знаменник $q = 0{,}5$. \nmtyear{2026}}

\answerTable{45}{3}{90}{94}{-45}

\vspace{0.5cm}

\noindent\makebox[1.5em][l]{\textbf{17.}}\parbox[t]{\dimexpr\textwidth-1.5em}{Знайдіть суму 5 перших членів геометричної прогресії $(b_n)$, у якої $b_2 = -4$, а знаменник $q = -3$. \nmtyear{2026}}

\answerTable{81{,}333}{-27{,}111}{108}{78{,}333}{-81{,}333}

\vspace{0.5cm}

\noindent\makebox[1.5em][l]{\textbf{18.}}\parbox[t]{\dimexpr\textwidth-1.5em}{Знайдіть суму 3 перших членів геометричної прогресії $(b_n)$, у якої $b_2 = 6$, а знаменник $q = -3$. \nmtyear{2026}}

\answerTable{-18}{4{,}667}{14}{0}{-14}

\vspace{0.5cm}

\noindent\makebox[1.5em][l]{\textbf{19.}}\parbox[t]{\dimexpr\textwidth-1.5em}{Знайдіть суму 5 перших членів геометричної прогресії $(b_n)$, у якої $b_2 = 18$, а знаменник $q = 3$. \nmtyear{2026}}

\answerTable{726}{768}{-726}{486}{242}

\vspace{0.5cm}

\noindent\makebox[1.5em][l]{\textbf{20.}}\parbox[t]{\dimexpr\textwidth-1.5em}{Знайдіть суму 4 перших членів геометричної прогресії $(b_n)$, у якої $b_2 = 18$, а знаменник $q = 0{,}5$. \nmtyear{2026}}

\answerTable{67{,}5}{135}{-67{,}5}{24{,}5}{4{,}5}

\vspace{0.5cm}

% === Geometric Progression: Word Problem ===
\noindent\makebox[1.5em][l]{\textbf{21.}}\parbox[t]{\dimexpr\textwidth-1.5em}{Інвестор вклав гроші. Першого дня прибуток склав 5 доларів. Кожного наступного дня кількість збільшувалася вдвічі. За яку \textit{найменшу} кількість днів сумарна кількість доларів перевищить 1000? \nmtyear{2026}}

\answerTable{9}{8}{10}{7}{6}

\vspace{0.5cm}

\noindent\makebox[1.5em][l]{\textbf{22.}}\parbox[t]{\dimexpr\textwidth-1.5em}{Бактерія ділиться. Першого дня колонія налічувала 10 бактерій. Кожного наступного дня кількість збільшувалася вдвічі. За яку \textit{найменшу} кількість днів сумарна кількість бактерій перевищить 2000? \nmtyear{2026}}

\answerTable{10}{7}{9}{8}{6}

\vspace{0.5cm}

\noindent\makebox[1.5em][l]{\textbf{23.}}\parbox[t]{\dimexpr\textwidth-1.5em}{Бактерія ділиться. Першого дня колонія налічувала 100 бактерій. Кожного наступного дня кількість збільшувалася вдвічі. За яку \textit{найменшу} кількість днів сумарна кількість бактерій перевищить 500? \nmtyear{2026}}

\answerTable{1}{3}{4}{2}{5}

\vspace{0.5cm}

\noindent\makebox[1.5em][l]{\textbf{24.}}\parbox[t]{\dimexpr\textwidth-1.5em}{Інвестор вклав гроші. Першого дня прибуток склав 50 доларів. Кожного наступного дня кількість збільшувалася вдвічі. За яку \textit{найменшу} кількість днів сумарна кількість доларів перевищить 500? \nmtyear{2026}}

\answerTable{6}{2}{3}{5}{4}

\vspace{0.5cm}

\noindent\makebox[1.5em][l]{\textbf{25.}}\parbox[t]{\dimexpr\textwidth-1.5em}{Бактерія ділиться. Першого дня колонія налічувала 50 бактерій. Кожного наступного дня кількість збільшувалася вдвічі. За яку \textit{найменшу} кількість днів сумарна кількість бактерій перевищить 5000? \nmtyear{2026}}

\answerTable{9}{5}{7}{8}{6}

\vspace{0.5cm}


\end{document}
