\documentclass[14pt]{extarticle}
\usepackage{fontspec}
\usepackage{polyglossia}
\setdefaultlanguage{ukrainian}

\defaultfontfeatures{Ligatures=TeX}
\setmainfont{Liberation Serif}
\setsansfont{Liberation Sans}
\setmonofont{Liberation Mono}

\usepackage[a4paper,margin=1.5cm,bottom=2cm,top=2cm]{geometry}
\usepackage{amsmath,amssymb}
\usepackage{enumitem}
\usepackage{tikz}
\usepackage{pgfplots}
\pgfplotsset{compat=1.18}

\usetikzlibrary{calc,patterns,angles,quotes,intersections,babel}
\usetikzlibrary{3d}

\usepackage{xcolor}
\usepackage{array}
\usepackage{fancyhdr}
\usepackage{multirow}

% Кольори
\definecolor{headerblue}{RGB}{0, 102, 204}
\definecolor{yearcolor}{RGB}{128, 0, 128}

\pagestyle{fancy}
\fancyhf{}
\renewcommand{\headrulewidth}{0pt}
\fancyfoot[C]{\thepage}

\setlength{\headheight}{15pt}
\setlength{\headsep}{10pt}
\setlength{\footskip}{25pt}

\widowpenalty=10000
\clubpenalty=10000

% === КОМАНДИ ===

% Таблиця відповідей для відповідностей
\newcommand{\answerGrid}{
    \begingroup
    \renewcommand{\arraystretch}{1.3} 
    \setlength{\tabcolsep}{7pt} 
    \begin{tabular}{r|c|c|c|c|c|}
         \multicolumn{1}{c}{} & \multicolumn{1}{c}{\textbf{А}} & \multicolumn{1}{c}{\textbf{Б}} & \multicolumn{1}{c}{\textbf{В}} & \multicolumn{1}{c}{\textbf{Г}} & \multicolumn{1}{c}{\textbf{Д}} \\ \cline{2-6}
         \textbf{1} & & & & & \\ \cline{2-6}
         \textbf{2} & & & & & \\ \cline{2-6}
         \textbf{3} & & & & & \\ \cline{2-6}
    \end{tabular}
    \endgroup
}

% Макет для завдань на відповідність
\newcommand{\matchingLayout}[3]{
    \noindent
    \begin{minipage}[t]{0.40\textwidth}
        #1
    \end{minipage}%
    \hfill
    \begin{minipage}[t]{0.28\textwidth}
        #2
    \end{minipage}%
    \hfill
    \begin{minipage}[t]{0.30\textwidth}
        \vspace{0pt}
        \begin{flushright}
        #3
        \end{flushright}
    \end{minipage}
}

% Стандартна таблиця відповідей
\newcommand{\answerTable}[5]{
\begin{center}
\begin{tabular}{|*{5}{>{\centering\arraybackslash}m{2.8cm}|}}
\hline
\rule[-0.3cm]{0pt}{0.8cm}\textbf{А} & \textbf{Б} & \textbf{В} & \textbf{Г} & \textbf{Д} \\
\hline
\rule[-0.4cm]{0pt}{1.0cm}#1 & \rule[-0.4cm]{0pt}{1.0cm}#2 & \rule[-0.4cm]{0pt}{1.0cm}#3 & \rule[-0.4cm]{0pt}{1.0cm}#4 & \rule[-0.4cm]{0pt}{1.0cm}#5 \\
\hline
\end{tabular}
\end{center}
}

% Таблиця для відповідей із дробами
\newcommand{\answerTableTall}[5]{
\begin{center}
\begin{tabular}{|*{5}{>{\centering\arraybackslash}m{2.8cm}|}}
\hline
\rule[-0.3cm]{0pt}{0.8cm}\textbf{А} & \textbf{Б} & \textbf{В} & \textbf{Г} & \textbf{Д} \\
\hline
\rule[-0.9cm]{0pt}{2.0cm}#1 & 
\rule[-0.9cm]{0pt}{2.0cm}#2 & 
\rule[-0.9cm]{0pt}{2.0cm}#3 & 
\rule[-0.9cm]{0pt}{2.0cm}#4 & 
\rule[-0.9cm]{0pt}{2.0cm}#5 \\
\hline
\end{tabular}
\end{center}
}

\newcommand{\nmtyear}[1]{\hfill{\small\color{yearcolor}(AI Gen)}}

\begin{document}

\vspace{1cm}

\begin{center}
{\Large\textbf{\color{headerblue}ЗГЕНЕРОВАНІ ЗАВДАННЯ (AI)}}
\end{center}

\begin{center}
{\large Тема: \textbf{Арифметична прогресія}}
\end{center}

\vspace{0.5cm}
% === Arithmetic Progression: Find d ===
\noindent\makebox[1.5em][l]{\textbf{1.}}\parbox[t]{\dimexpr\textwidth-1.5em}{В арифметичній прогресії $(a_n)$: $a_1 = -9$, $a_3 = -25$. Визначте різницю $d$ прогресії. \nmtyear{2026}}
\vspace{0.3cm}

\answerTable{$d = 8$}{$d = -7$}{$d = -16$}{$d = -8$}{$d = -17$}

\vspace{0.5cm}

\noindent\makebox[1.5em][l]{\textbf{2.}}\parbox[t]{\dimexpr\textwidth-1.5em}{В арифметичній прогресії $(a_n)$: $a_1 = 5$, $a_3 = 3$. Визначте різницю $d$ прогресії. \nmtyear{2026}}
\vspace{0.3cm}

\answerTable{$d = 0$}{$d = -1$}{$d = -2$}{$d = 1$}{$d = 4$}

\vspace{0.5cm}

\noindent\makebox[1.5em][l]{\textbf{3.}}\parbox[t]{\dimexpr\textwidth-1.5em}{В арифметичній прогресії $(a_n)$: $a_1 = 2$, $a_3 = 18$. Визначте різницю $d$ прогресії. \nmtyear{2026}}
\vspace{0.3cm}

\answerTable{$d = 10$}{$d = -8$}{$d = 9$}{$d = 8$}{$d = 16$}

\vspace{0.5cm}

\noindent\makebox[1.5em][l]{\textbf{4.}}\parbox[t]{\dimexpr\textwidth-1.5em}{В арифметичній прогресії $(a_n)$: $a_1 = -4$, $a_3 = -23$. Визначте різницю $d$ прогресії. \nmtyear{2026}}
\vspace{0.3cm}

\answerTable{$d = -19$}{$d = -9{,}5$}{$d = 9{,}5$}{$d = -13{,}5$}{$d = -8{,}5$}

\vspace{0.5cm}

\noindent\makebox[1.5em][l]{\textbf{5.}}\parbox[t]{\dimexpr\textwidth-1.5em}{В арифметичній прогресії $(a_n)$: $a_1 = 0$, $a_3 = 4$. Визначте різницю $d$ прогресії. \nmtyear{2026}}
\vspace{0.3cm}

\answerTable{$d = 4$}{93}{$d = -2$}{$d = 2$}{$d = 3$}

\vspace{0.5cm}

\noindent\makebox[1.5em][l]{\textbf{6.}}\parbox[t]{\dimexpr\textwidth-1.5em}{В арифметичній прогресії $(a_n)$: $a_1 = 7$, $a_3 = 19$. Визначте різницю $d$ прогресії. \nmtyear{2026}}
\vspace{0.3cm}

\answerTable{$d = 12$}{$d = -6$}{$d = 13$}{$d = 6$}{$d = 7$}

\vspace{0.5cm}

\noindent\makebox[1.5em][l]{\textbf{7.}}\parbox[t]{\dimexpr\textwidth-1.5em}{В арифметичній прогресії $(a_n)$: $a_1 = 4$, $a_3 = -15$. Визначте різницю $d$ прогресії. \nmtyear{2026}}
\vspace{0.3cm}

\answerTable{$d = -8{,}5$}{$d = -9{,}5$}{$d = 9{,}5$}{$d = -5{,}5$}{$d = -19$}

\vspace{0.5cm}

\noindent\makebox[1.5em][l]{\textbf{8.}}\parbox[t]{\dimexpr\textwidth-1.5em}{В арифметичній прогресії $(a_n)$: $a_1 = 18$, $a_3 = 20$. Визначте різницю $d$ прогресії. \nmtyear{2026}}
\vspace{0.3cm}

\answerTable{45}{$d = -1$}{$d = 1$}{$d = 2$}{$d = 19$}

\vspace{0.5cm}

\noindent\makebox[1.5em][l]{\textbf{9.}}\parbox[t]{\dimexpr\textwidth-1.5em}{В арифметичній прогресії $(a_n)$: $a_1 = 7$, $a_3 = 3$. Визначте різницю $d$ прогресії. \nmtyear{2026}}
\vspace{0.3cm}

\answerTable{$d = 2$}{$d = -1$}{$d = -4$}{$d = 5$}{$d = -2$}

\vspace{0.5cm}

\noindent\makebox[1.5em][l]{\textbf{10.}}\parbox[t]{\dimexpr\textwidth-1.5em}{В арифметичній прогресії $(a_n)$: $a_1 = -8$, $a_3 = -28$. Визначте різницю $d$ прогресії. \nmtyear{2026}}
\vspace{0.3cm}

\answerTable{$d = 10$}{$d = -18$}{$d = -9$}{$d = -10$}{$d = -20$}

\vspace{0.5cm}

\noindent\makebox[1.5em][l]{\textbf{11.}}\parbox[t]{\dimexpr\textwidth-1.5em}{В арифметичній прогресії $(a_n)$: $a_1 = -7$, $a_3 = -19$. Визначте різницю $d$ прогресії. \nmtyear{2026}}
\vspace{0.3cm}

\answerTable{$d = 6$}{$d = -6$}{$d = -12$}{$d = -5$}{$d = -13$}

\vspace{0.5cm}

\noindent\makebox[1.5em][l]{\textbf{12.}}\parbox[t]{\dimexpr\textwidth-1.5em}{В арифметичній прогресії $(a_n)$: $a_1 = -10$, $a_3 = -2$. Визначте різницю $d$ прогресії. \nmtyear{2026}}
\vspace{0.3cm}

\answerTable{$d = 8$}{$d = 4$}{$d = 5$}{$d = -4$}{$d = -6$}

\vspace{0.5cm}

\noindent\makebox[1.5em][l]{\textbf{13.}}\parbox[t]{\dimexpr\textwidth-1.5em}{В арифметичній прогресії $(a_n)$: $a_1 = -8$, $a_3 = 11$. Визначте різницю $d$ прогресії. \nmtyear{2026}}
\vspace{0.3cm}

\answerTable{$d = 1{,}5$}{$d = 10{,}5$}{$d = -9{,}5$}{$d = 19$}{$d = 9{,}5$}

\vspace{0.5cm}

\noindent\makebox[1.5em][l]{\textbf{14.}}\parbox[t]{\dimexpr\textwidth-1.5em}{В арифметичній прогресії $(a_n)$: $a_1 = 10$, $a_3 = 0$. Визначте різницю $d$ прогресії. \nmtyear{2026}}
\vspace{0.3cm}

\answerTable{$d = -10$}{12}{$d = -5$}{$d = -4$}{$d = 5$}

\vspace{0.5cm}

\noindent\makebox[1.5em][l]{\textbf{15.}}\parbox[t]{\dimexpr\textwidth-1.5em}{В арифметичній прогресії $(a_n)$: $a_1 = 14$, $a_3 = 18$. Визначте різницю $d$ прогресії. \nmtyear{2026}}
\vspace{0.3cm}

\answerTable{$d = 3$}{$d = 16$}{$d = 2$}{$d = -2$}{$d = 4$}

\vspace{0.5cm}

\noindent\makebox[1.5em][l]{\textbf{16.}}\parbox[t]{\dimexpr\textwidth-1.5em}{В арифметичній прогресії $(a_n)$: $a_1 = -9$, $a_3 = -21$. Визначте різницю $d$ прогресії. \nmtyear{2026}}
\vspace{0.3cm}

\answerTable{$d = -15$}{$d = 6$}{$d = -12$}{$d = -5$}{$d = -6$}

\vspace{0.5cm}

\noindent\makebox[1.5em][l]{\textbf{17.}}\parbox[t]{\dimexpr\textwidth-1.5em}{В арифметичній прогресії $(a_n)$: $a_1 = 13$, $a_3 = 1$. Визначте різницю $d$ прогресії. \nmtyear{2026}}
\vspace{0.3cm}

\answerTable{$d = -6$}{$d = 6$}{$d = 7$}{$d = -5$}{$d = -12$}

\vspace{0.5cm}

\noindent\makebox[1.5em][l]{\textbf{18.}}\parbox[t]{\dimexpr\textwidth-1.5em}{В арифметичній прогресії $(a_n)$: $a_1 = -8$, $a_3 = 6$. Визначте різницю $d$ прогресії. \nmtyear{2026}}
\vspace{0.3cm}

\answerTable{$d = -1$}{$d = 7$}{$d = 14$}{$d = -7$}{$d = 8$}

\vspace{0.5cm}

\noindent\makebox[1.5em][l]{\textbf{19.}}\parbox[t]{\dimexpr\textwidth-1.5em}{В арифметичній прогресії $(a_n)$: $a_1 = -10$, $a_3 = 10$. Визначте різницю $d$ прогресії. \nmtyear{2026}}
\vspace{0.3cm}

\answerTable{$d = 0$}{$d = 20$}{$d = -10$}{$d = 10$}{$d = 11$}

\vspace{0.5cm}

\noindent\makebox[1.5em][l]{\textbf{20.}}\parbox[t]{\dimexpr\textwidth-1.5em}{В арифметичній прогресії $(a_n)$: $a_1 = -1$, $a_3 = -3$. Визначте різницю $d$ прогресії. \nmtyear{2026}}
\vspace{0.3cm}

\answerTable{$d = 0$}{$d = -2$}{$d = 1$}{$d = -1$}{28}

\vspace{0.5cm}

% === Arithmetic Progression: Member Difference ===
\noindent\makebox[1.5em][l]{\textbf{21.}}\parbox[t]{\dimexpr\textwidth-1.5em}{В арифметичній прогресії $(a_n)$ відомо, що $a_{10} - a_{4} = 30$. Знайдіть значення виразу $a_{2} - a_{6}$. \nmtyear{2026}}
\vspace{0.3cm}

\answerTable{20}{-25}{-20}{-15}{30}

\vspace{0.5cm}

\noindent\makebox[1.5em][l]{\textbf{22.}}\parbox[t]{\dimexpr\textwidth-1.5em}{В арифметичній прогресії $(a_n)$ відомо, що $a_{4} - a_{2} = 12$. Знайдіть значення виразу $a_{5} - a_{9}$. \nmtyear{2026}}
\vspace{0.3cm}

\answerTable{-30}{-18}{-24}{24}{12}

\vspace{0.5cm}

\noindent\makebox[1.5em][l]{\textbf{23.}}\parbox[t]{\dimexpr\textwidth-1.5em}{В арифметичній прогресії $(a_n)$ відомо, що $a_{10} - a_{4} = -60$. Знайдіть значення виразу $a_{11} - a_{12}$. \nmtyear{2026}}
\vspace{0.3cm}

\answerTable{-60}{20}{10}{0}{-10}

\vspace{0.5cm}

\noindent\makebox[1.5em][l]{\textbf{24.}}\parbox[t]{\dimexpr\textwidth-1.5em}{В арифметичній прогресії $(a_n)$ відомо, що $a_{6} - a_{2} = -28$. Знайдіть значення виразу $a_{1} - a_{3}$. \nmtyear{2026}}
\vspace{0.3cm}

\answerTable{-28}{21}{-14}{7}{14}

\vspace{0.5cm}

\noindent\makebox[1.5em][l]{\textbf{25.}}\parbox[t]{\dimexpr\textwidth-1.5em}{В арифметичній прогресії $(a_n)$ відомо, що $a_{4} - a_{2} = 18$. Знайдіть значення виразу $a_{7} - a_{5}$. \nmtyear{2026}}
\vspace{0.3cm}

\answerTable{9}{23}{18}{27}{-18}

\vspace{0.5cm}

\noindent\makebox[1.5em][l]{\textbf{26.}}\parbox[t]{\dimexpr\textwidth-1.5em}{В арифметичній прогресії $(a_n)$ відомо, що $a_{7} - a_{4} = -12$. Знайдіть значення виразу $a_{11} - a_{8}$. \nmtyear{2026}}
\vspace{0.3cm}

\answerTable{-12}{-16}{12}{-10}{-8}

\vspace{0.5cm}

\noindent\makebox[1.5em][l]{\textbf{27.}}\parbox[t]{\dimexpr\textwidth-1.5em}{В арифметичній прогресії $(a_n)$ відомо, що $a_{7} - a_{5} = 4$. Знайдіть значення виразу $a_{1} - a_{5}$. \nmtyear{2026}}
\vspace{0.3cm}

\answerTable{8}{4}{-10}{-8}{-6}

\vspace{0.5cm}

\noindent\makebox[1.5em][l]{\textbf{28.}}\parbox[t]{\dimexpr\textwidth-1.5em}{В арифметичній прогресії $(a_n)$ відомо, що $a_{6} - a_{2} = -32$. Знайдіть значення виразу $a_{5} - a_{6}$. \nmtyear{2026}}
\vspace{0.3cm}

\answerTable{8}{0}{16}{-32}{-8}

\vspace{0.5cm}

\noindent\makebox[1.5em][l]{\textbf{29.}}\parbox[t]{\dimexpr\textwidth-1.5em}{В арифметичній прогресії $(a_n)$ відомо, що $a_{7} - a_{2} = -10$. Знайдіть значення виразу $a_{5} - a_{7}$. \nmtyear{2026}}
\vspace{0.3cm}

\answerTable{6}{-10}{2}{4}{-4}

\vspace{0.5cm}

\noindent\makebox[1.5em][l]{\textbf{30.}}\parbox[t]{\dimexpr\textwidth-1.5em}{В арифметичній прогресії $(a_n)$ відомо, що $a_{8} - a_{2} = 18$. Знайдіть значення виразу $a_{2} - a_{5}$. \nmtyear{2026}}
\vspace{0.3cm}

\answerTable{18}{-12}{-9}{9}{-6}

\vspace{0.5cm}

\noindent\makebox[1.5em][l]{\textbf{31.}}\parbox[t]{\dimexpr\textwidth-1.5em}{В арифметичній прогресії $(a_n)$ відомо, що $a_{11} - a_{5} = 60$. Знайдіть значення виразу $a_{13} - a_{14}$. \nmtyear{2026}}
\vspace{0.3cm}

\answerTable{-20}{10}{60}{-10}{0}

\vspace{0.5cm}

\noindent\makebox[1.5em][l]{\textbf{32.}}\parbox[t]{\dimexpr\textwidth-1.5em}{В арифметичній прогресії $(a_n)$ відомо, що $a_{7} - a_{3} = -12$. Знайдіть значення виразу $a_{8} - a_{9}$. \nmtyear{2026}}
\vspace{0.3cm}

\answerTable{-3}{3}{-12}{0}{6}

\vspace{0.5cm}

\noindent\makebox[1.5em][l]{\textbf{33.}}\parbox[t]{\dimexpr\textwidth-1.5em}{В арифметичній прогресії $(a_n)$ відомо, що $a_{4} - a_{2} = 18$. Знайдіть значення виразу $a_{6} - a_{3}$. \nmtyear{2026}}
\vspace{0.3cm}

\answerTable{32}{27}{-27}{18}{36}

\vspace{0.5cm}

\noindent\makebox[1.5em][l]{\textbf{34.}}\parbox[t]{\dimexpr\textwidth-1.5em}{В арифметичній прогресії $(a_n)$ відомо, що $a_{9} - a_{5} = 28$. Знайдіть значення виразу $a_{4} - a_{5}$. \nmtyear{2026}}
\vspace{0.3cm}

\answerTable{-14}{7}{0}{28}{-7}

\vspace{0.5cm}

\noindent\makebox[1.5em][l]{\textbf{35.}}\parbox[t]{\dimexpr\textwidth-1.5em}{В арифметичній прогресії $(a_n)$ відомо, що $a_{7} - a_{2} = 35$. Знайдіть значення виразу $a_{7} - a_{10}$. \nmtyear{2026}}
\vspace{0.3cm}

\answerTable{-28}{21}{35}{-21}{-14}

\vspace{0.5cm}

\noindent\makebox[1.5em][l]{\textbf{36.}}\parbox[t]{\dimexpr\textwidth-1.5em}{В арифметичній прогресії $(a_n)$ відомо, що $a_{11} - a_{5} = 6$. Знайдіть значення виразу $a_{2} - a_{6}$. \nmtyear{2026}}
\vspace{0.3cm}

\answerTable{4}{-4}{-3}{6}{-5}

\vspace{0.5cm}

\noindent\makebox[1.5em][l]{\textbf{37.}}\parbox[t]{\dimexpr\textwidth-1.5em}{В арифметичній прогресії $(a_n)$ відомо, що $a_{7} - a_{4} = -18$. Знайдіть значення виразу $a_{5} - a_{2}$. \nmtyear{2026}}
\vspace{0.3cm}

\answerTable{-24}{18}{-18}{-12}{-14}

\vspace{0.5cm}

\noindent\makebox[1.5em][l]{\textbf{38.}}\parbox[t]{\dimexpr\textwidth-1.5em}{В арифметичній прогресії $(a_n)$ відомо, що $a_{8} - a_{5} = 6$. Знайдіть значення виразу $a_{9} - a_{13}$. \nmtyear{2026}}
\vspace{0.3cm}

\answerTable{-8}{-6}{8}{6}{-10}

\vspace{0.5cm}

\noindent\makebox[1.5em][l]{\textbf{39.}}\parbox[t]{\dimexpr\textwidth-1.5em}{В арифметичній прогресії $(a_n)$ відомо, що $a_{9} - a_{5} = -4$. Знайдіть значення виразу $a_{11} - a_{14}$. \nmtyear{2026}}
\vspace{0.3cm}

\answerTable{3}{-3}{2}{-4}{4}

\vspace{0.5cm}

\noindent\makebox[1.5em][l]{\textbf{40.}}\parbox[t]{\dimexpr\textwidth-1.5em}{В арифметичній прогресії $(a_n)$ відомо, що $a_{7} - a_{3} = -20$. Знайдіть значення виразу $a_{5} - a_{8}$. \nmtyear{2026}}
\vspace{0.3cm}

\answerTable{10}{15}{20}{-20}{-15}

\vspace{0.5cm}

% === Arithmetic Progression: Sum ===
\noindent\makebox[1.5em][l]{\textbf{41.}}\parbox[t]{\dimexpr\textwidth-1.5em}{Обчисліть суму перших 12-ти членів арифметичної прогресії $(a_n)$, якщо $a_1 + a_{12} = -27$. \nmtyear{2026}}
\vspace{0.3cm}

\answerTable{162}{-27}{-150}{-162}{-174}

\vspace{0.5cm}

\noindent\makebox[1.5em][l]{\textbf{42.}}\parbox[t]{\dimexpr\textwidth-1.5em}{Обчисліть суму перших 10-ти членів арифметичної прогресії $(a_n)$, якщо $a_1 + a_{10} = -6$. \nmtyear{2026}}
\vspace{0.3cm}

\answerTable{30}{-40}{-6}{-30}{-20}

\vspace{0.5cm}

\noindent\makebox[1.5em][l]{\textbf{43.}}\parbox[t]{\dimexpr\textwidth-1.5em}{Обчисліть суму перших 20-ти членів арифметичної прогресії $(a_n)$, якщо $a_1 + a_{20} = 21$. \nmtyear{2026}}
\vspace{0.3cm}

\answerTable{420}{210}{21}{190}{230}

\vspace{0.5cm}

\noindent\makebox[1.5em][l]{\textbf{44.}}\parbox[t]{\dimexpr\textwidth-1.5em}{Обчисліть суму перших 100-ти членів арифметичної прогресії $(a_n)$, якщо $a_1 + a_{100} = 41$. \nmtyear{2026}}
\vspace{0.3cm}

\answerTable{41}{2150}{-2050}{2050}{4100}

\vspace{0.5cm}

\noindent\makebox[1.5em][l]{\textbf{45.}}\parbox[t]{\dimexpr\textwidth-1.5em}{Обчисліть суму перших 20-ти членів арифметичної прогресії $(a_n)$, якщо $a_1 + a_{20} = -43$. \nmtyear{2026}}
\vspace{0.3cm}

\answerTable{430}{-860}{-450}{-430}{-43}

\vspace{0.5cm}

\noindent\makebox[1.5em][l]{\textbf{46.}}\parbox[t]{\dimexpr\textwidth-1.5em}{Обчисліть суму перших 100-ти членів арифметичної прогресії $(a_n)$, якщо $a_1 + a_{100} = 46$. \nmtyear{2026}}
\vspace{0.3cm}

\answerTable{46}{2200}{4600}{-2300}{2300}

\vspace{0.5cm}

\noindent\makebox[1.5em][l]{\textbf{47.}}\parbox[t]{\dimexpr\textwidth-1.5em}{Обчисліть суму перших 12-ти членів арифметичної прогресії $(a_n)$, якщо $a_1 + a_{12} = 29$. \nmtyear{2026}}
\vspace{0.3cm}

\answerTable{348}{-174}{162}{174}{29}

\vspace{0.5cm}

\noindent\makebox[1.5em][l]{\textbf{48.}}\parbox[t]{\dimexpr\textwidth-1.5em}{Обчисліть суму перших 8-ти членів арифметичної прогресії $(a_n)$, якщо $a_1 + a_{8} = 26$. \nmtyear{2026}}
\vspace{0.3cm}

\answerTable{26}{96}{-104}{112}{104}

\vspace{0.5cm}

\noindent\makebox[1.5em][l]{\textbf{49.}}\parbox[t]{\dimexpr\textwidth-1.5em}{Обчисліть суму перших 20-ти членів арифметичної прогресії $(a_n)$, якщо $a_1 + a_{20} = 18$. \nmtyear{2026}}
\vspace{0.3cm}

\answerTable{180}{200}{-180}{160}{360}

\vspace{0.5cm}

\noindent\makebox[1.5em][l]{\textbf{50.}}\parbox[t]{\dimexpr\textwidth-1.5em}{Обчисліть суму перших 100-ти членів арифметичної прогресії $(a_n)$, якщо $a_1 + a_{100} = 21$. \nmtyear{2026}}
\vspace{0.3cm}

\answerTable{1150}{2100}{950}{1050}{-1050}

\vspace{0.5cm}

\noindent\makebox[1.5em][l]{\textbf{51.}}\parbox[t]{\dimexpr\textwidth-1.5em}{Обчисліть суму перших 100-ти членів арифметичної прогресії $(a_n)$, якщо $a_1 + a_{100} = 47$. \nmtyear{2026}}
\vspace{0.3cm}

\answerTable{4700}{2450}{2250}{2350}{-2350}

\vspace{0.5cm}

\noindent\makebox[1.5em][l]{\textbf{52.}}\parbox[t]{\dimexpr\textwidth-1.5em}{Обчисліть суму перших 100-ти членів арифметичної прогресії $(a_n)$, якщо $a_1 + a_{100} = 4$. \nmtyear{2026}}
\vspace{0.3cm}

\answerTable{4}{-200}{400}{200}{300}

\vspace{0.5cm}

\noindent\makebox[1.5em][l]{\textbf{53.}}\parbox[t]{\dimexpr\textwidth-1.5em}{Обчисліть суму перших 12-ти членів арифметичної прогресії $(a_n)$, якщо $a_1 + a_{12} = 25$. \nmtyear{2026}}
\vspace{0.3cm}

\answerTable{150}{300}{25}{162}{-150}

\vspace{0.5cm}

\noindent\makebox[1.5em][l]{\textbf{54.}}\parbox[t]{\dimexpr\textwidth-1.5em}{Обчисліть суму перших 10-ти членів арифметичної прогресії $(a_n)$, якщо $a_1 + a_{10} = 40$. \nmtyear{2026}}
\vspace{0.3cm}

\answerTable{-200}{200}{400}{40}{190}

\vspace{0.5cm}

\noindent\makebox[1.5em][l]{\textbf{55.}}\parbox[t]{\dimexpr\textwidth-1.5em}{Обчисліть суму перших 8-ти членів арифметичної прогресії $(a_n)$, якщо $a_1 + a_{8} = -31$. \nmtyear{2026}}
\vspace{0.3cm}

\answerTable{124}{-116}{-132}{-31}{-124}

\vspace{0.5cm}

\noindent\makebox[1.5em][l]{\textbf{56.}}\parbox[t]{\dimexpr\textwidth-1.5em}{Обчисліть суму перших 12-ти членів арифметичної прогресії $(a_n)$, якщо $a_1 + a_{12} = -11$. \nmtyear{2026}}
\vspace{0.3cm}

\answerTable{-11}{-54}{66}{-78}{-66}

\vspace{0.5cm}

\noindent\makebox[1.5em][l]{\textbf{57.}}\parbox[t]{\dimexpr\textwidth-1.5em}{Обчисліть суму перших 100-ти членів арифметичної прогресії $(a_n)$, якщо $a_1 + a_{100} = -27$. \nmtyear{2026}}
\vspace{0.3cm}

\answerTable{-1250}{-1450}{-1350}{-27}{1350}

\vspace{0.5cm}

\noindent\makebox[1.5em][l]{\textbf{58.}}\parbox[t]{\dimexpr\textwidth-1.5em}{Обчисліть суму перших 8-ти членів арифметичної прогресії $(a_n)$, якщо $a_1 + a_{8} = -29$. \nmtyear{2026}}
\vspace{0.3cm}

\answerTable{-232}{-29}{-116}{-124}{116}

\vspace{0.5cm}

\noindent\makebox[1.5em][l]{\textbf{59.}}\parbox[t]{\dimexpr\textwidth-1.5em}{Обчисліть суму перших 10-ти членів арифметичної прогресії $(a_n)$, якщо $a_1 + a_{10} = 31$. \nmtyear{2026}}
\vspace{0.3cm}

\answerTable{31}{-155}{165}{310}{155}

\vspace{0.5cm}

\noindent\makebox[1.5em][l]{\textbf{60.}}\parbox[t]{\dimexpr\textwidth-1.5em}{Обчисліть суму перших 10-ти членів арифметичної прогресії $(a_n)$, якщо $a_1 + a_{10} = 25$. \nmtyear{2026}}
\vspace{0.3cm}

\answerTable{115}{-125}{250}{125}{25}

\vspace{0.5cm}

% === Arithmetic Progression: Count Terms ===
\noindent\makebox[1.5em][l]{\textbf{61.}}\parbox[t]{\dimexpr\textwidth-1.5em}{В арифметичній прогресії $(a_n)$ перший член $a_1 = 5{,}4$, різниця $d = -0{,}4$. Скільки всього \textit{додатних} членів має ця прогресія? \nmtyear{2026}}
\vspace{0.3cm}

\answerTable{14}{12}{16}{13}{15}

\vspace{0.5cm}

\noindent\makebox[1.5em][l]{\textbf{62.}}\parbox[t]{\dimexpr\textwidth-1.5em}{В арифметичній прогресії $(a_n)$ перший член $a_1 = 18{,}5$, різниця $d = -3$. Скільки всього \textit{додатних} членів має ця прогресія? \nmtyear{2026}}
\vspace{0.3cm}

\answerTable{6}{8}{5}{7}{9}

\vspace{0.5cm}

\noindent\makebox[1.5em][l]{\textbf{63.}}\parbox[t]{\dimexpr\textwidth-1.5em}{В арифметичній прогресії $(a_n)$ перший член $a_1 = -4$, різниця $d = 0{,}5$. Скільки всього \textit{від'ємних} членів має ця прогресія? \nmtyear{2026}}
\vspace{0.3cm}

\answerTable{5}{4}{3}{6}{7}

\vspace{0.5cm}

\noindent\makebox[1.5em][l]{\textbf{64.}}\parbox[t]{\dimexpr\textwidth-1.5em}{В арифметичній прогресії $(a_n)$ перший член $a_1 = 5{,}2$, різниця $d = -0{,}8$. Скільки всього \textit{додатних} членів має ця прогресія? \nmtyear{2026}}
\vspace{0.3cm}

\answerTable{6}{3}{7}{4}{5}

\vspace{0.5cm}

\noindent\makebox[1.5em][l]{\textbf{65.}}\parbox[t]{\dimexpr\textwidth-1.5em}{В арифметичній прогресії $(a_n)$ перший член $a_1 = 28{,}1$, різниця $d = -2$. Скільки всього \textit{додатних} членів має ця прогресія? \nmtyear{2026}}
\vspace{0.3cm}

\answerTable{14}{17}{16}{13}{15}

\vspace{0.5cm}

\noindent\makebox[1.5em][l]{\textbf{66.}}\parbox[t]{\dimexpr\textwidth-1.5em}{В арифметичній прогресії $(a_n)$ перший член $a_1 = -37$, різниця $d = 2{,}5$. Скільки всього \textit{від'ємних} членів має ця прогресія? \nmtyear{2026}}
\vspace{0.3cm}

\answerTable{15}{13}{14}{17}{16}

\vspace{0.5cm}

\noindent\makebox[1.5em][l]{\textbf{67.}}\parbox[t]{\dimexpr\textwidth-1.5em}{В арифметичній прогресії $(a_n)$ перший член $a_1 = -5$, різниця $d = 0{,}5$. Скільки всього \textit{від'ємних} членів має ця прогресія? \nmtyear{2026}}
\vspace{0.3cm}

\answerTable{9}{8}{5}{6}{7}

\vspace{0.5cm}

\noindent\makebox[1.5em][l]{\textbf{68.}}\parbox[t]{\dimexpr\textwidth-1.5em}{В арифметичній прогресії $(a_n)$ перший член $a_1 = -35{,}2$, різниця $d = 2{,}5$. Скільки всього \textit{від'ємних} членів має ця прогресія? \nmtyear{2026}}
\vspace{0.3cm}

\answerTable{13}{17}{15}{14}{16}

\vspace{0.5cm}

\noindent\makebox[1.5em][l]{\textbf{69.}}\parbox[t]{\dimexpr\textwidth-1.5em}{В арифметичній прогресії $(a_n)$ перший член $a_1 = -13{,}1$, різниця $d = 1$. Скільки всього \textit{від'ємних} членів має ця прогресія? \nmtyear{2026}}
\vspace{0.3cm}

\answerTable{14}{15}{13}{16}{12}

\vspace{0.5cm}

\noindent\makebox[1.5em][l]{\textbf{70.}}\parbox[t]{\dimexpr\textwidth-1.5em}{В арифметичній прогресії $(a_n)$ перший член $a_1 = 10{,}2$, різниця $d = -2{,}5$. Скільки всього \textit{додатних} членів має ця прогресія? \nmtyear{2026}}
\vspace{0.3cm}

\answerTable{6}{4}{5}{7}{3}

\vspace{0.5cm}

\noindent\makebox[1.5em][l]{\textbf{71.}}\parbox[t]{\dimexpr\textwidth-1.5em}{В арифметичній прогресії $(a_n)$ перший член $a_1 = 15{,}2$, різниця $d = -1{,}5$. Скільки всього \textit{додатних} членів має ця прогресія? \nmtyear{2026}}
\vspace{0.3cm}

\answerTable{13}{10}{12}{9}{11}

\vspace{0.5cm}

\noindent\makebox[1.5em][l]{\textbf{72.}}\parbox[t]{\dimexpr\textwidth-1.5em}{В арифметичній прогресії $(a_n)$ перший член $a_1 = -22{,}6$, різниця $d = 2{,}5$. Скільки всього \textit{від'ємних} членів має ця прогресія? \nmtyear{2026}}
\vspace{0.3cm}

\answerTable{10}{9}{12}{8}{11}

\vspace{0.5cm}

\noindent\makebox[1.5em][l]{\textbf{73.}}\parbox[t]{\dimexpr\textwidth-1.5em}{В арифметичній прогресії $(a_n)$ перший член $a_1 = -20{,}3$, різниця $d = 2{,}5$. Скільки всього \textit{від'ємних} членів має ця прогресія? \nmtyear{2026}}
\vspace{0.3cm}

\answerTable{11}{8}{10}{9}{7}

\vspace{0.5cm}

\noindent\makebox[1.5em][l]{\textbf{74.}}\parbox[t]{\dimexpr\textwidth-1.5em}{В арифметичній прогресії $(a_n)$ перший член $a_1 = -2{,}6$, різниця $d = 0{,}4$. Скільки всього \textit{від'ємних} членів має ця прогресія? \nmtyear{2026}}
\vspace{0.3cm}

\answerTable{7}{6}{4}{5}{3}

\vspace{0.5cm}

\noindent\makebox[1.5em][l]{\textbf{75.}}\parbox[t]{\dimexpr\textwidth-1.5em}{В арифметичній прогресії $(a_n)$ перший член $a_1 = 42{,}3$, різниця $d = -3$. Скільки всього \textit{додатних} членів має ця прогресія? \nmtyear{2026}}
\vspace{0.3cm}

\answerTable{13}{14}{16}{17}{15}

\vspace{0.5cm}

\noindent\makebox[1.5em][l]{\textbf{76.}}\parbox[t]{\dimexpr\textwidth-1.5em}{В арифметичній прогресії $(a_n)$ перший член $a_1 = -5{,}3$, різниця $d = 1$. Скільки всього \textit{від'ємних} членів має ця прогресія? \nmtyear{2026}}
\vspace{0.3cm}

\answerTable{8}{4}{6}{5}{7}

\vspace{0.5cm}

\noindent\makebox[1.5em][l]{\textbf{77.}}\parbox[t]{\dimexpr\textwidth-1.5em}{В арифметичній прогресії $(a_n)$ перший член $a_1 = 8$, різниця $d = -1{,}5$. Скільки всього \textit{додатних} членів має ця прогресія? \nmtyear{2026}}
\vspace{0.3cm}

\answerTable{4}{8}{7}{6}{5}

\vspace{0.5cm}

\noindent\makebox[1.5em][l]{\textbf{78.}}\parbox[t]{\dimexpr\textwidth-1.5em}{В арифметичній прогресії $(a_n)$ перший член $a_1 = -5{,}1$, різниця $d = 0{,}4$. Скільки всього \textit{від'ємних} членів має ця прогресія? \nmtyear{2026}}
\vspace{0.3cm}

\answerTable{14}{15}{13}{12}{11}

\vspace{0.5cm}

\noindent\makebox[1.5em][l]{\textbf{79.}}\parbox[t]{\dimexpr\textwidth-1.5em}{В арифметичній прогресії $(a_n)$ перший член $a_1 = 12{,}1$, різниця $d = -1$. Скільки всього \textit{додатних} членів має ця прогресія? \nmtyear{2026}}
\vspace{0.3cm}

\answerTable{12}{11}{13}{14}{15}

\vspace{0.5cm}

\noindent\makebox[1.5em][l]{\textbf{80.}}\parbox[t]{\dimexpr\textwidth-1.5em}{В арифметичній прогресії $(a_n)$ перший член $a_1 = 10{,}3$, різниця $d = -2$. Скільки всього \textit{додатних} членів має ця прогресія? \nmtyear{2026}}
\vspace{0.3cm}

\answerTable{6}{5}{4}{8}{7}

\vspace{0.5cm}

% === Arithmetic Progression: Middle Term ===
\noindent\makebox[1.5em][l]{\textbf{81.}}\parbox[t]{\dimexpr\textwidth-1.5em}{Визначте 15-й член $a_{15}$ арифметичної прогресії $(a_n)$, у якої $a_{14} = 12$, $a_{16} = 16$. \nmtyear{2026}}
\vspace{0.3cm}

\answerTable{16}{14}{12}{4}{2}

\vspace{0.5cm}

\noindent\makebox[1.5em][l]{\textbf{82.}}\parbox[t]{\dimexpr\textwidth-1.5em}{Визначте 14-й член $a_{14}$ арифметичної прогресії $(a_n)$, у якої $a_{13} = 38$, $a_{15} = 39$. \nmtyear{2026}}
\vspace{0.3cm}

\answerTable{1}{39}{38}{0{,}5}{38{,}5}

\vspace{0.5cm}

\noindent\makebox[1.5em][l]{\textbf{83.}}\parbox[t]{\dimexpr\textwidth-1.5em}{Визначте 20-й член $a_{20}$ арифметичної прогресії $(a_n)$, у якої $a_{19} = 23$, $a_{21} = 17$. \nmtyear{2026}}
\vspace{0.3cm}

\answerTable{20}{40}{-3}{23}{6}

\vspace{0.5cm}

\noindent\makebox[1.5em][l]{\textbf{84.}}\parbox[t]{\dimexpr\textwidth-1.5em}{Визначте 13-й член $a_{13}$ арифметичної прогресії $(a_n)$, у якої $a_{12} = 6$, $a_{14} = 8$. \nmtyear{2026}}
\vspace{0.3cm}

\answerTable{1}{7}{2}{8}{6}

\vspace{0.5cm}

\noindent\makebox[1.5em][l]{\textbf{85.}}\parbox[t]{\dimexpr\textwidth-1.5em}{Визначте 10-й член $a_{10}$ арифметичної прогресії $(a_n)$, у якої $a_{9} = 46$, $a_{11} = 52$. \nmtyear{2026}}
\vspace{0.3cm}

\answerTable{49}{6}{98}{46}{52}

\vspace{0.5cm}

\noindent\makebox[1.5em][l]{\textbf{86.}}\parbox[t]{\dimexpr\textwidth-1.5em}{Визначте 5-й член $a_{5}$ арифметичної прогресії $(a_n)$, у якої $a_{4} = 35$, $a_{6} = 45$. \nmtyear{2026}}
\vspace{0.3cm}

\answerTable{35}{5}{40}{80}{45}

\vspace{0.5cm}

\noindent\makebox[1.5em][l]{\textbf{87.}}\parbox[t]{\dimexpr\textwidth-1.5em}{Визначте 10-й член $a_{10}$ арифметичної прогресії $(a_n)$, у якої $a_{9} = -5$, $a_{11} = -11$. \nmtyear{2026}}
\vspace{0.3cm}

\answerTable{6}{-5}{-8}{-3}{-11}

\vspace{0.5cm}

\noindent\makebox[1.5em][l]{\textbf{88.}}\parbox[t]{\dimexpr\textwidth-1.5em}{Визначте 6-й член $a_{6}$ арифметичної прогресії $(a_n)$, у якої $a_{5} = 9$, $a_{7} = 7$. \nmtyear{2026}}
\vspace{0.3cm}

\answerTable{8}{7}{9}{2}{-1}

\vspace{0.5cm}

\noindent\makebox[1.5em][l]{\textbf{89.}}\parbox[t]{\dimexpr\textwidth-1.5em}{Визначте 13-й член $a_{13}$ арифметичної прогресії $(a_n)$, у якої $a_{12} = 43$, $a_{14} = 47$. \nmtyear{2026}}
\vspace{0.3cm}

\answerTable{90}{2}{47}{43}{45}

\vspace{0.5cm}

\noindent\makebox[1.5em][l]{\textbf{90.}}\parbox[t]{\dimexpr\textwidth-1.5em}{Визначте 16-й член $a_{16}$ арифметичної прогресії $(a_n)$, у якої $a_{15} = -3$, $a_{17} = 3$. \nmtyear{2026}}
\vspace{0.3cm}

\answerTable{10}{6}{0}{-3}{3}

\vspace{0.5cm}

\noindent\makebox[1.5em][l]{\textbf{91.}}\parbox[t]{\dimexpr\textwidth-1.5em}{Визначте 8-й член $a_{8}$ арифметичної прогресії $(a_n)$, у якої $a_{7} = 7$, $a_{9} = 17$. \nmtyear{2026}}
\vspace{0.3cm}

\answerTable{7}{10}{24}{5}{12}

\vspace{0.5cm}

\noindent\makebox[1.5em][l]{\textbf{92.}}\parbox[t]{\dimexpr\textwidth-1.5em}{Визначте 15-й член $a_{15}$ арифметичної прогресії $(a_n)$, у якої $a_{14} = -16$, $a_{16} = -11$. \nmtyear{2026}}
\vspace{0.3cm}

\answerTable{-27}{-13{,}5}{-11}{-16}{5}

\vspace{0.5cm}

\noindent\makebox[1.5em][l]{\textbf{93.}}\parbox[t]{\dimexpr\textwidth-1.5em}{Визначте 20-й член $a_{20}$ арифметичної прогресії $(a_n)$, у якої $a_{19} = 28$, $a_{21} = 26$. \nmtyear{2026}}
\vspace{0.3cm}

\answerTable{-1}{28}{2}{26}{27}

\vspace{0.5cm}

\noindent\makebox[1.5em][l]{\textbf{94.}}\parbox[t]{\dimexpr\textwidth-1.5em}{Визначте 9-й член $a_{9}$ арифметичної прогресії $(a_n)$, у якої $a_{8} = 26$, $a_{10} = 24$. \nmtyear{2026}}
\vspace{0.3cm}

\answerTable{26}{25}{50}{24}{2}

\vspace{0.5cm}

\noindent\makebox[1.5em][l]{\textbf{95.}}\parbox[t]{\dimexpr\textwidth-1.5em}{Визначте 13-й член $a_{13}$ арифметичної прогресії $(a_n)$, у якої $a_{12} = 35$, $a_{14} = 40$. \nmtyear{2026}}
\vspace{0.3cm}

\answerTable{75}{5}{37{,}5}{35}{2{,}5}

\vspace{0.5cm}

\noindent\makebox[1.5em][l]{\textbf{96.}}\parbox[t]{\dimexpr\textwidth-1.5em}{Визначте 8-й член $a_{8}$ арифметичної прогресії $(a_n)$, у якої $a_{7} = 47$, $a_{9} = 50$. \nmtyear{2026}}
\vspace{0.3cm}

\answerTable{97}{48{,}5}{47}{50}{1{,}5}

\vspace{0.5cm}

\noindent\makebox[1.5em][l]{\textbf{97.}}\parbox[t]{\dimexpr\textwidth-1.5em}{Визначте 11-й член $a_{11}$ арифметичної прогресії $(a_n)$, у якої $a_{10} = 45$, $a_{12} = 47$. \nmtyear{2026}}
\vspace{0.3cm}

\answerTable{2}{46}{47}{92}{1}

\vspace{0.5cm}

\noindent\makebox[1.5em][l]{\textbf{98.}}\parbox[t]{\dimexpr\textwidth-1.5em}{Визначте 9-й член $a_{9}$ арифметичної прогресії $(a_n)$, у якої $a_{8} = 34$, $a_{10} = 28$. \nmtyear{2026}}
\vspace{0.3cm}

\answerTable{-3}{34}{28}{31}{62}

\vspace{0.5cm}

\noindent\makebox[1.5em][l]{\textbf{99.}}\parbox[t]{\dimexpr\textwidth-1.5em}{Визначте 14-й член $a_{14}$ арифметичної прогресії $(a_n)$, у якої $a_{13} = 32$, $a_{15} = 33$. \nmtyear{2026}}
\vspace{0.3cm}

\answerTable{65}{1}{0{,}5}{32{,}5}{32}

\vspace{0.5cm}

\noindent\makebox[1.5em][l]{\textbf{100.}}\parbox[t]{\dimexpr\textwidth-1.5em}{Визначте 9-й член $a_{9}$ арифметичної прогресії $(a_n)$, у якої $a_{8} = 15$, $a_{10} = 16$. \nmtyear{2026}}
\vspace{0.3cm}

\answerTable{31}{0{,}5}{15{,}5}{1}{16}

\vspace{0.5cm}

% === Arithmetic Progression: Formula Search ===
\noindent\makebox[1.5em][l]{\textbf{101.}}\parbox[t]{\dimexpr\textwidth-1.5em}{Арифметичну прогресію $(a_n)$ задано формулою $n$-го члена $a_n = 42  -4n$. Визначте номер члена, значення якого дорівнює $-94$. \nmtyear{2026}}
\vspace{0.3cm}

\answerTable{44}{23{,}5}{33}{34}{24}

\vspace{0.5cm}

\noindent\makebox[1.5em][l]{\textbf{102.}}\parbox[t]{\dimexpr\textwidth-1.5em}{Арифметичну прогресію $(a_n)$ задано формулою $n$-го члена $a_n = 17  -1{,}5n$. Визначте номер члена, значення якого дорівнює $-23{,}5$. \nmtyear{2026}}
\vspace{0.3cm}

\answerTable{26}{27}{15{,}67}{37}{28}

\vspace{0.5cm}

\noindent\makebox[1.5em][l]{\textbf{103.}}\parbox[t]{\dimexpr\textwidth-1.5em}{Арифметичну прогресію $(a_n)$ задано формулою $n$-го члена $a_n = 49  -1{,}5n$. Визначте номер члена, значення якого дорівнює $19$. \nmtyear{2026}}
\vspace{0.3cm}

\answerTable{21}{10}{20}{19}{30}

\vspace{0.5cm}

\noindent\makebox[1.5em][l]{\textbf{104.}}\parbox[t]{\dimexpr\textwidth-1.5em}{Арифметичну прогресію $(a_n)$ задано формулою $n$-го члена $a_n = 22  -2{,}5n$. Визначте номер члена, значення якого дорівнює $-35{,}5$. \nmtyear{2026}}
\vspace{0.3cm}

\answerTable{22}{14{,}2}{13}{24}{23}

\vspace{0.5cm}

\noindent\makebox[1.5em][l]{\textbf{105.}}\parbox[t]{\dimexpr\textwidth-1.5em}{Арифметичну прогресію $(a_n)$ задано формулою $n$-го члена $a_n = 11  -0{,}5n$. Визначте номер члена, значення якого дорівнює $7{,}5$. \nmtyear{2026}}
\vspace{0.3cm}

\answerTable{8}{7}{6}{1}{17}

\vspace{0.5cm}

\noindent\makebox[1.5em][l]{\textbf{106.}}\parbox[t]{\dimexpr\textwidth-1.5em}{Арифметичну прогресію $(a_n)$ задано формулою $n$-го члена $a_n = 22  -0{,}5n$. Визначте номер члена, значення якого дорівнює $8{,}5$. \nmtyear{2026}}
\vspace{0.3cm}

\answerTable{28}{27}{26}{37}{17}

\vspace{0.5cm}

\noindent\makebox[1.5em][l]{\textbf{107.}}\parbox[t]{\dimexpr\textwidth-1.5em}{Арифметичну прогресію $(a_n)$ задано формулою $n$-го члена $a_n = 39 + 1{,}5n$. Визначте номер члена, значення якого дорівнює $57$. \nmtyear{2026}}
\vspace{0.3cm}

\answerTable{11}{38}{22}{13}{12}

\vspace{0.5cm}

\noindent\makebox[1.5em][l]{\textbf{108.}}\parbox[t]{\dimexpr\textwidth-1.5em}{Арифметичну прогресію $(a_n)$ задано формулою $n$-го члена $a_n = 47  -1{,}5n$. Визначте номер члена, значення якого дорівнює $-11{,}5$. \nmtyear{2026}}
\vspace{0.3cm}

\answerTable{7{,}67}{29}{40}{49}{39}

\vspace{0.5cm}

\noindent\makebox[1.5em][l]{\textbf{109.}}\parbox[t]{\dimexpr\textwidth-1.5em}{Арифметичну прогресію $(a_n)$ задано формулою $n$-го члена $a_n = 10  -4n$. Визначте номер члена, значення якого дорівнює $-90$. \nmtyear{2026}}
\vspace{0.3cm}

\answerTable{15}{26}{24}{35}{25}

\vspace{0.5cm}

\noindent\makebox[1.5em][l]{\textbf{110.}}\parbox[t]{\dimexpr\textwidth-1.5em}{Арифметичну прогресію $(a_n)$ задано формулою $n$-го члена $a_n = 12  -4n$. Визначте номер члена, значення якого дорівнює $-80$. \nmtyear{2026}}
\vspace{0.3cm}

\answerTable{20}{24}{33}{13}{23}

\vspace{0.5cm}

\noindent\makebox[1.5em][l]{\textbf{111.}}\parbox[t]{\dimexpr\textwidth-1.5em}{Арифметичну прогресію $(a_n)$ задано формулою $n$-го члена $a_n = 34  -3n$. Визначте номер члена, значення якого дорівнює $4$. \nmtyear{2026}}
\vspace{0.3cm}

\answerTable{11}{20}{10}{1}{9}

\vspace{0.5cm}

\noindent\makebox[1.5em][l]{\textbf{112.}}\parbox[t]{\dimexpr\textwidth-1.5em}{Арифметичну прогресію $(a_n)$ задано формулою $n$-го члена $a_n = 45  -2{,}5n$. Визначте номер члена, значення якого дорівнює $0$. \nmtyear{2026}}
\vspace{0.3cm}

\answerTable{17}{18}{19}{28}{8}

\vspace{0.5cm}

\noindent\makebox[1.5em][l]{\textbf{113.}}\parbox[t]{\dimexpr\textwidth-1.5em}{Арифметичну прогресію $(a_n)$ задано формулою $n$-го члена $a_n = 45  -1{,}5n$. Визначте номер члена, значення якого дорівнює $18$. \nmtyear{2026}}
\vspace{0.3cm}

\answerTable{19}{28}{17}{8}{18}

\vspace{0.5cm}

\noindent\makebox[1.5em][l]{\textbf{114.}}\parbox[t]{\dimexpr\textwidth-1.5em}{Арифметичну прогресію $(a_n)$ задано формулою $n$-го члена $a_n = 40  -1{,}5n$. Визначте номер члена, значення якого дорівнює $13$. \nmtyear{2026}}
\vspace{0.3cm}

\answerTable{18}{8}{19}{17}{28}

\vspace{0.5cm}

\noindent\makebox[1.5em][l]{\textbf{115.}}\parbox[t]{\dimexpr\textwidth-1.5em}{Арифметичну прогресію $(a_n)$ задано формулою $n$-го члена $a_n = 44 + 4n$. Визначте номер члена, значення якого дорівнює $156$. \nmtyear{2026}}
\vspace{0.3cm}

\answerTable{29}{39}{28}{27}{18}

\vspace{0.5cm}

\noindent\makebox[1.5em][l]{\textbf{116.}}\parbox[t]{\dimexpr\textwidth-1.5em}{Арифметичну прогресію $(a_n)$ задано формулою $n$-го члена $a_n = 19  -4n$. Визначте номер члена, значення якого дорівнює $-105$. \nmtyear{2026}}
\vspace{0.3cm}

\answerTable{21}{32}{26{,}25}{41}{31}

\vspace{0.5cm}

\noindent\makebox[1.5em][l]{\textbf{117.}}\parbox[t]{\dimexpr\textwidth-1.5em}{Арифметичну прогресію $(a_n)$ задано формулою $n$-го члена $a_n = 29 + 3n$. Визначте номер члена, значення якого дорівнює $161$. \nmtyear{2026}}
\vspace{0.3cm}

\answerTable{44}{54}{53{,}67}{34}{43}

\vspace{0.5cm}

\noindent\makebox[1.5em][l]{\textbf{118.}}\parbox[t]{\dimexpr\textwidth-1.5em}{Арифметичну прогресію $(a_n)$ задано формулою $n$-го члена $a_n = 29 + 3n$. Визначте номер члена, значення якого дорівнює $53$. \nmtyear{2026}}
\vspace{0.3cm}

\answerTable{17{,}67}{8}{18}{1}{9}

\vspace{0.5cm}

\noindent\makebox[1.5em][l]{\textbf{119.}}\parbox[t]{\dimexpr\textwidth-1.5em}{Арифметичну прогресію $(a_n)$ задано формулою $n$-го члена $a_n = 14  -2{,}5n$. Визначте номер члена, значення якого дорівнює $1{,}5$. \nmtyear{2026}}
\vspace{0.3cm}

\answerTable{5}{6}{4}{15}{1}

\vspace{0.5cm}

\noindent\makebox[1.5em][l]{\textbf{120.}}\parbox[t]{\dimexpr\textwidth-1.5em}{Арифметичну прогресію $(a_n)$ задано формулою $n$-го члена $a_n = 18 + 2{,}5n$. Визначте номер члена, значення якого дорівнює $33$. \nmtyear{2026}}
\vspace{0.3cm}

\answerTable{1}{6}{5}{7}{16}

\vspace{0.5cm}

% === Arithmetic Progression: Word Problem ===
\noindent\makebox[1.5em][l]{\textbf{121.}}\parbox[t]{\dimexpr\textwidth-1.5em}{На рисунку зображено поперечний переріз стосу колод. У нижньому ряду стосу 5 колод, а у верхньому — 1. Визначте загальну кількість колод.
            \begin{center}
            \begin{tikzpicture}[scale=0.5]
                \newcommand{\woodLog}[3]{
                    \begin{scope}[shift={(#1,#2)}]
                        \draw[fill=woodinner, draw=black, thick] (0,0) circle (0.5);
                        \draw[woodouter!80, thin] (0,0) circle (0.35);
                        \draw[woodouter!80, thin] (0,0) circle (0.2);
                        \begin{scope}[rotate=#3]
                            \fill[woodouter] (0,0) -- (0.4, 0.05) -- (0.5, 0.1) -- (0.5, -0.1) -- (0.4, -0.05) -- cycle;
                        \end{scope}
                    \end{scope}
                }
                \def\rows{5} 
                \foreach \row in {1,...,\rows} {
                    \foreach \col in {1,...,\row} {
                        \pgfmathsetmacro{\x}{(\col-1) - (\row-1)*0.5}
                        \pgfmathsetmacro{\y}{-(\row-1)*0.866}
                        \pgfmathsetmacro{\angle}{mod(\col*70 + \row*50, 360)}
                        \woodLog{\x}{\y}{\angle}
                    }
                }
            \end{tikzpicture}
            \end{center}
             \nmtyear{2026}}
\vspace{0.3cm}

\answerTable{25}{5}{20}{10}{15}

\vspace{0.5cm}

\noindent\makebox[1.5em][l]{\textbf{122.}}\parbox[t]{\dimexpr\textwidth-1.5em}{На рисунку зображено поперечний переріз стосу колод. У нижньому ряду стосу 3 колод, а у верхньому — 1. Визначте загальну кількість колод.
            \begin{center}
            \begin{tikzpicture}[scale=0.5]
                \newcommand{\woodLog}[3]{
                    \begin{scope}[shift={(#1,#2)}]
                        \draw[fill=woodinner, draw=black, thick] (0,0) circle (0.5);
                        \draw[woodouter!80, thin] (0,0) circle (0.35);
                        \draw[woodouter!80, thin] (0,0) circle (0.2);
                        \begin{scope}[rotate=#3]
                            \fill[woodouter] (0,0) -- (0.4, 0.05) -- (0.5, 0.1) -- (0.5, -0.1) -- (0.4, -0.05) -- cycle;
                        \end{scope}
                    \end{scope}
                }
                \def\rows{3} 
                \foreach \row in {1,...,\rows} {
                    \foreach \col in {1,...,\row} {
                        \pgfmathsetmacro{\x}{(\col-1) - (\row-1)*0.5}
                        \pgfmathsetmacro{\y}{-(\row-1)*0.866}
                        \pgfmathsetmacro{\angle}{mod(\col*70 + \row*50, 360)}
                        \woodLog{\x}{\y}{\angle}
                    }
                }
            \end{tikzpicture}
            \end{center}
             \nmtyear{2026}}
\vspace{0.3cm}

\answerTable{9}{6}{-4}{3}{16}

\vspace{0.5cm}

\noindent\makebox[1.5em][l]{\textbf{123.}}\parbox[t]{\dimexpr\textwidth-1.5em}{За умовами договору позичальник повинен повернути кредит протягом 24 місяців. Першого місяця він має повернути 940 \textit{грн}, а кожного наступного місяця — на 10 \textit{грн} менше, ніж попереднього. Визначте загальну суму (у \textit{грн}), яку повинен позичальник повернути протягом 24 місяців. \nmtyear{2026}}
\vspace{0.3cm}

\answerTable{19800}{22560}{22800}{19300}{20800}

\vspace{0.5cm}

\noindent\makebox[1.5em][l]{\textbf{124.}}\parbox[t]{\dimexpr\textwidth-1.5em}{За умовами договору позичальник повинен повернути кредит протягом 12 місяців. Першого місяця він має повернути 420 \textit{грн}, а кожного наступного місяця — на 10 \textit{грн} менше, ніж попереднього. Визначте загальну суму (у \textit{грн}), яку повинен позичальник повернути протягом 12 місяців. \nmtyear{2026}}
\vspace{0.3cm}

\answerTable{5040}{4380}{3880}{5160}{5380}

\vspace{0.5cm}

\noindent\makebox[1.5em][l]{\textbf{125.}}\parbox[t]{\dimexpr\textwidth-1.5em}{За умовами договору позичальник повинен повернути кредит протягом 24 місяців. Першого місяця він має повернути 680 \textit{грн}, а кожного наступного місяця — на 20 \textit{грн} менше, ніж попереднього. Визначте загальну суму (у \textit{грн}), яку повинен позичальник повернути протягом 24 місяців. \nmtyear{2026}}
\vspace{0.3cm}

\answerTable{10800}{16800}{16320}{11800}{10300}

\vspace{0.5cm}

\noindent\makebox[1.5em][l]{\textbf{126.}}\parbox[t]{\dimexpr\textwidth-1.5em}{Студент вивчав мову за методикою: у перший день він запам'ятав 17 слів, а кожного наступного дня — на 2 слів більше, ніж попереднього. Скільки всього слів запам'ятав студент за 18 днів? \nmtyear{2026}}
\vspace{0.3cm}

\answerTable{918}{632}{612}{592}{306}

\vspace{0.5cm}

\noindent\makebox[1.5em][l]{\textbf{127.}}\parbox[t]{\dimexpr\textwidth-1.5em}{На рисунку зображено поперечний переріз стосу колод. У нижньому ряду стосу 4 колод, а у верхньому — 1. Визначте загальну кількість колод.
            \begin{center}
            \begin{tikzpicture}[scale=0.5]
                \newcommand{\woodLog}[3]{
                    \begin{scope}[shift={(#1,#2)}]
                        \draw[fill=woodinner, draw=black, thick] (0,0) circle (0.5);
                        \draw[woodouter!80, thin] (0,0) circle (0.35);
                        \draw[woodouter!80, thin] (0,0) circle (0.2);
                        \begin{scope}[rotate=#3]
                            \fill[woodouter] (0,0) -- (0.4, 0.05) -- (0.5, 0.1) -- (0.5, -0.1) -- (0.4, -0.05) -- cycle;
                        \end{scope}
                    \end{scope}
                }
                \def\rows{4} 
                \foreach \row in {1,...,\rows} {
                    \foreach \col in {1,...,\row} {
                        \pgfmathsetmacro{\x}{(\col-1) - (\row-1)*0.5}
                        \pgfmathsetmacro{\y}{-(\row-1)*0.866}
                        \pgfmathsetmacro{\angle}{mod(\col*70 + \row*50, 360)}
                        \woodLog{\x}{\y}{\angle}
                    }
                }
            \end{tikzpicture}
            \end{center}
             \nmtyear{2026}}
\vspace{0.3cm}

\answerTable{14}{0}{6}{10}{20}

\vspace{0.5cm}

\noindent\makebox[1.5em][l]{\textbf{128.}}\parbox[t]{\dimexpr\textwidth-1.5em}{На рисунку зображено поперечний переріз стосу колод. У нижньому ряду стосу 4 колод, а у верхньому — 1. Визначте загальну кількість колод.
            \begin{center}
            \begin{tikzpicture}[scale=0.5]
                \newcommand{\woodLog}[3]{
                    \begin{scope}[shift={(#1,#2)}]
                        \draw[fill=woodinner, draw=black, thick] (0,0) circle (0.5);
                        \draw[woodouter!80, thin] (0,0) circle (0.35);
                        \draw[woodouter!80, thin] (0,0) circle (0.2);
                        \begin{scope}[rotate=#3]
                            \fill[woodouter] (0,0) -- (0.4, 0.05) -- (0.5, 0.1) -- (0.5, -0.1) -- (0.4, -0.05) -- cycle;
                        \end{scope}
                    \end{scope}
                }
                \def\rows{4} 
                \foreach \row in {1,...,\rows} {
                    \foreach \col in {1,...,\row} {
                        \pgfmathsetmacro{\x}{(\col-1) - (\row-1)*0.5}
                        \pgfmathsetmacro{\y}{-(\row-1)*0.866}
                        \pgfmathsetmacro{\angle}{mod(\col*70 + \row*50, 360)}
                        \woodLog{\x}{\y}{\angle}
                    }
                }
            \end{tikzpicture}
            \end{center}
             \nmtyear{2026}}
\vspace{0.3cm}

\answerTable{20}{0}{14}{10}{6}

\vspace{0.5cm}

\noindent\makebox[1.5em][l]{\textbf{129.}}\parbox[t]{\dimexpr\textwidth-1.5em}{У залі для глядачів цирку встановлено 10 рядів крісел: у першому ряду 25 крісла, а в кожному наступному ряду кількість крісел на те саме число більше, ніж у попередньому. Визначте кількість крісел у \textit{3-му} ряду, якщо в останньому ряду 43 крісла. \nmtyear{2026}}
\vspace{0.3cm}

\answerTable{31}{33}{29}{25}{27}

\vspace{0.5cm}

\noindent\makebox[1.5em][l]{\textbf{130.}}\parbox[t]{\dimexpr\textwidth-1.5em}{Студент вивчав мову за методикою: у перший день він запам'ятав 13 слів, а кожного наступного дня — на 1 слів більше, ніж попереднього. Скільки всього слів запам'ятав студент за 28 днів? \nmtyear{2026}}
\vspace{0.3cm}

\answerTable{722}{742}{762}{1120}{364}

\vspace{0.5cm}

\noindent\makebox[1.5em][l]{\textbf{131.}}\parbox[t]{\dimexpr\textwidth-1.5em}{На рисунку зображено поперечний переріз стосу колод. У нижньому ряду стосу 8 колод, а у верхньому — 1. Визначте загальну кількість колод.
            \begin{center}
            \begin{tikzpicture}[scale=0.5]
                \newcommand{\woodLog}[3]{
                    \begin{scope}[shift={(#1,#2)}]
                        \draw[fill=woodinner, draw=black, thick] (0,0) circle (0.5);
                        \draw[woodouter!80, thin] (0,0) circle (0.35);
                        \draw[woodouter!80, thin] (0,0) circle (0.2);
                        \begin{scope}[rotate=#3]
                            \fill[woodouter] (0,0) -- (0.4, 0.05) -- (0.5, 0.1) -- (0.5, -0.1) -- (0.4, -0.05) -- cycle;
                        \end{scope}
                    \end{scope}
                }
                \def\rows{8} 
                \foreach \row in {1,...,\rows} {
                    \foreach \col in {1,...,\row} {
                        \pgfmathsetmacro{\x}{(\col-1) - (\row-1)*0.5}
                        \pgfmathsetmacro{\y}{-(\row-1)*0.866}
                        \pgfmathsetmacro{\angle}{mod(\col*70 + \row*50, 360)}
                        \woodLog{\x}{\y}{\angle}
                    }
                }
            \end{tikzpicture}
            \end{center}
             \nmtyear{2026}}
\vspace{0.3cm}

\answerTable{26}{44}{46}{36}{28}

\vspace{0.5cm}

\noindent\makebox[1.5em][l]{\textbf{132.}}\parbox[t]{\dimexpr\textwidth-1.5em}{За умовами договору позичальник повинен повернути кредит протягом 24 місяців. Першого місяця він має повернути 1900 \textit{грн}, а кожного наступного місяця — на 50 \textit{грн} менше, ніж попереднього. Визначте загальну суму (у \textit{грн}), яку повинен позичальник повернути протягом 24 місяців. \nmtyear{2026}}
\vspace{0.3cm}

\answerTable{31300}{45600}{31800}{46800}{32800}

\vspace{0.5cm}

\noindent\makebox[1.5em][l]{\textbf{133.}}\parbox[t]{\dimexpr\textwidth-1.5em}{На рисунку зображено поперечний переріз стосу колод. У нижньому ряду стосу 8 колод, а у верхньому — 1. Визначте загальну кількість колод.
            \begin{center}
            \begin{tikzpicture}[scale=0.5]
                \newcommand{\woodLog}[3]{
                    \begin{scope}[shift={(#1,#2)}]
                        \draw[fill=woodinner, draw=black, thick] (0,0) circle (0.5);
                        \draw[woodouter!80, thin] (0,0) circle (0.35);
                        \draw[woodouter!80, thin] (0,0) circle (0.2);
                        \begin{scope}[rotate=#3]
                            \fill[woodouter] (0,0) -- (0.4, 0.05) -- (0.5, 0.1) -- (0.5, -0.1) -- (0.4, -0.05) -- cycle;
                        \end{scope}
                    \end{scope}
                }
                \def\rows{8} 
                \foreach \row in {1,...,\rows} {
                    \foreach \col in {1,...,\row} {
                        \pgfmathsetmacro{\x}{(\col-1) - (\row-1)*0.5}
                        \pgfmathsetmacro{\y}{-(\row-1)*0.866}
                        \pgfmathsetmacro{\angle}{mod(\col*70 + \row*50, 360)}
                        \woodLog{\x}{\y}{\angle}
                    }
                }
            \end{tikzpicture}
            \end{center}
             \nmtyear{2026}}
\vspace{0.3cm}

\answerTable{26}{36}{44}{28}{46}

\vspace{0.5cm}

\noindent\makebox[1.5em][l]{\textbf{134.}}\parbox[t]{\dimexpr\textwidth-1.5em}{У залі для глядачів цирку встановлено 20 рядів крісел: у першому ряду 20 крісла, а в кожному наступному ряду кількість крісел на те саме число більше, ніж у попередньому. Визначте кількість крісел у \textit{12-му} ряду, якщо в останньому ряду 96 крісла. \nmtyear{2026}}
\vspace{0.3cm}

\answerTable{68}{60}{72}{56}{64}

\vspace{0.5cm}

\noindent\makebox[1.5em][l]{\textbf{135.}}\parbox[t]{\dimexpr\textwidth-1.5em}{На рисунку зображено поперечний переріз стосу колод. У нижньому ряду стосу 3 колод, а у верхньому — 1. Визначте загальну кількість колод.
            \begin{center}
            \begin{tikzpicture}[scale=0.5]
                \newcommand{\woodLog}[3]{
                    \begin{scope}[shift={(#1,#2)}]
                        \draw[fill=woodinner, draw=black, thick] (0,0) circle (0.5);
                        \draw[woodouter!80, thin] (0,0) circle (0.35);
                        \draw[woodouter!80, thin] (0,0) circle (0.2);
                        \begin{scope}[rotate=#3]
                            \fill[woodouter] (0,0) -- (0.4, 0.05) -- (0.5, 0.1) -- (0.5, -0.1) -- (0.4, -0.05) -- cycle;
                        \end{scope}
                    \end{scope}
                }
                \def\rows{3} 
                \foreach \row in {1,...,\rows} {
                    \foreach \col in {1,...,\row} {
                        \pgfmathsetmacro{\x}{(\col-1) - (\row-1)*0.5}
                        \pgfmathsetmacro{\y}{-(\row-1)*0.866}
                        \pgfmathsetmacro{\angle}{mod(\col*70 + \row*50, 360)}
                        \woodLog{\x}{\y}{\angle}
                    }
                }
            \end{tikzpicture}
            \end{center}
             \nmtyear{2026}}
\vspace{0.3cm}

\answerTable{16}{9}{6}{-4}{3}

\vspace{0.5cm}

\noindent\makebox[1.5em][l]{\textbf{136.}}\parbox[t]{\dimexpr\textwidth-1.5em}{Студент вивчав мову за методикою: у перший день він запам'ятав 7 слів, а кожного наступного дня — на 1 слів більше, ніж попереднього. Скільки всього слів запам'ятав студент за 14 днів? \nmtyear{2026}}
\vspace{0.3cm}

\answerTable{189}{169}{209}{280}{98}

\vspace{0.5cm}

\noindent\makebox[1.5em][l]{\textbf{137.}}\parbox[t]{\dimexpr\textwidth-1.5em}{На рисунку зображено поперечний переріз стосу колод. У нижньому ряду стосу 6 колод, а у верхньому — 1. Визначте загальну кількість колод.
            \begin{center}
            \begin{tikzpicture}[scale=0.5]
                \newcommand{\woodLog}[3]{
                    \begin{scope}[shift={(#1,#2)}]
                        \draw[fill=woodinner, draw=black, thick] (0,0) circle (0.5);
                        \draw[woodouter!80, thin] (0,0) circle (0.35);
                        \draw[woodouter!80, thin] (0,0) circle (0.2);
                        \begin{scope}[rotate=#3]
                            \fill[woodouter] (0,0) -- (0.4, 0.05) -- (0.5, 0.1) -- (0.5, -0.1) -- (0.4, -0.05) -- cycle;
                        \end{scope}
                    \end{scope}
                }
                \def\rows{6} 
                \foreach \row in {1,...,\rows} {
                    \foreach \col in {1,...,\row} {
                        \pgfmathsetmacro{\x}{(\col-1) - (\row-1)*0.5}
                        \pgfmathsetmacro{\y}{-(\row-1)*0.866}
                        \pgfmathsetmacro{\angle}{mod(\col*70 + \row*50, 360)}
                        \woodLog{\x}{\y}{\angle}
                    }
                }
            \end{tikzpicture}
            \end{center}
             \nmtyear{2026}}
\vspace{0.3cm}

\answerTable{21}{15}{31}{27}{11}

\vspace{0.5cm}

\noindent\makebox[1.5em][l]{\textbf{138.}}\parbox[t]{\dimexpr\textwidth-1.5em}{За умовами договору позичальник повинен повернути кредит протягом 24 місяців. Першого місяця він має повернути 940 \textit{грн}, а кожного наступного місяця — на 10 \textit{грн} менше, ніж попереднього. Визначте загальну суму (у \textit{грн}), яку повинен позичальник повернути протягом 24 місяців. \nmtyear{2026}}
\vspace{0.3cm}

\answerTable{19800}{20800}{22560}{19300}{22800}

\vspace{0.5cm}

\noindent\makebox[1.5em][l]{\textbf{139.}}\parbox[t]{\dimexpr\textwidth-1.5em}{У залі для глядачів цирку встановлено 24 рядів крісел: у першому ряду 37 крісла, а в кожному наступному ряду кількість крісел на те саме число більше, ніж у попередньому. Визначте кількість крісел у \textit{21-му} ряду, якщо в останньому ряду 83 крісла. \nmtyear{2026}}
\vspace{0.3cm}

\answerTable{73}{81}{75}{77}{79}

\vspace{0.5cm}

\noindent\makebox[1.5em][l]{\textbf{140.}}\parbox[t]{\dimexpr\textwidth-1.5em}{На рисунку зображено поперечний переріз стосу колод. У нижньому ряду стосу 8 колод, а у верхньому — 1. Визначте загальну кількість колод.
            \begin{center}
            \begin{tikzpicture}[scale=0.5]
                \newcommand{\woodLog}[3]{
                    \begin{scope}[shift={(#1,#2)}]
                        \draw[fill=woodinner, draw=black, thick] (0,0) circle (0.5);
                        \draw[woodouter!80, thin] (0,0) circle (0.35);
                        \draw[woodouter!80, thin] (0,0) circle (0.2);
                        \begin{scope}[rotate=#3]
                            \fill[woodouter] (0,0) -- (0.4, 0.05) -- (0.5, 0.1) -- (0.5, -0.1) -- (0.4, -0.05) -- cycle;
                        \end{scope}
                    \end{scope}
                }
                \def\rows{8} 
                \foreach \row in {1,...,\rows} {
                    \foreach \col in {1,...,\row} {
                        \pgfmathsetmacro{\x}{(\col-1) - (\row-1)*0.5}
                        \pgfmathsetmacro{\y}{-(\row-1)*0.866}
                        \pgfmathsetmacro{\angle}{mod(\col*70 + \row*50, 360)}
                        \woodLog{\x}{\y}{\angle}
                    }
                }
            \end{tikzpicture}
            \end{center}
             \nmtyear{2026}}
\vspace{0.3cm}

\answerTable{28}{26}{36}{46}{44}

\vspace{0.5cm}


\end{document}
