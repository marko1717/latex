\documentclass[14pt]{extarticle}
\usepackage{fontspec}
\usepackage{polyglossia}
\setdefaultlanguage{ukrainian}

\defaultfontfeatures{Ligatures=TeX}
\setmainfont{Liberation Serif}
\setsansfont{Liberation Sans}
\setmonofont{Liberation Mono}

\usepackage[a4paper,margin=1.5cm,bottom=2cm,top=2cm]{geometry}
\usepackage{amsmath,amssymb}
\usepackage{enumitem}
\usepackage{tikz}
\usepackage{pgfplots}
\pgfplotsset{compat=1.18}

\usetikzlibrary{calc,patterns,angles,quotes,intersections,babel}
\usetikzlibrary{3d}

\usepackage{xcolor}
\usepackage{array}
\usepackage{fancyhdr}
\usepackage{multirow}

% Кольори
\definecolor{headerblue}{RGB}{0, 102, 204}
\definecolor{yearcolor}{RGB}{128, 0, 128}

\pagestyle{fancy}
\fancyhf{}
\renewcommand{\headrulewidth}{0pt}
\fancyfoot[C]{\thepage}

\setlength{\headheight}{15pt}
\setlength{\headsep}{10pt}
\setlength{\footskip}{25pt}

\widowpenalty=10000
\clubpenalty=10000

% === КОМАНДИ ===

% Таблиця відповідей для відповідностей
\newcommand{\answerGrid}{
    \begingroup
    \renewcommand{\arraystretch}{1.3} 
    \setlength{\tabcolsep}{7pt} 
    \begin{tabular}{r|c|c|c|c|c|}
         \multicolumn{1}{c}{} & \multicolumn{1}{c}{\textbf{А}} & \multicolumn{1}{c}{\textbf{Б}} & \multicolumn{1}{c}{\textbf{В}} & \multicolumn{1}{c}{\textbf{Г}} & \multicolumn{1}{c}{\textbf{Д}} \\ \cline{2-6}
         \textbf{1} & & & & & \\ \cline{2-6}
         \textbf{2} & & & & & \\ \cline{2-6}
         \textbf{3} & & & & & \\ \cline{2-6}
    \end{tabular}
    \endgroup
}

% Макет для завдань на відповідність
\newcommand{\matchingLayout}[3]{
    \noindent
    \begin{minipage}[t]{0.40\textwidth}
        #1
    \end{minipage}%
    \hfill
    \begin{minipage}[t]{0.28\textwidth}
        #2
    \end{minipage}%
    \hfill
    \begin{minipage}[t]{0.30\textwidth}
        \vspace{0pt}
        \begin{flushright}
        #3
        \end{flushright}
    \end{minipage}
}

% Стандартна таблиця відповідей
\newcommand{\answerTable}[5]{
\begin{center}
\begin{tabular}{|*{5}{>{\centering\arraybackslash}m{2.8cm}|}}
\hline
\rule[-0.3cm]{0pt}{0.8cm}\textbf{А} & \textbf{Б} & \textbf{В} & \textbf{Г} & \textbf{Д} \\
\hline
\rule[-0.4cm]{0pt}{1.0cm}#1 & \rule[-0.4cm]{0pt}{1.0cm}#2 & \rule[-0.4cm]{0pt}{1.0cm}#3 & \rule[-0.4cm]{0pt}{1.0cm}#4 & \rule[-0.4cm]{0pt}{1.0cm}#5 \\
\hline
\end{tabular}
\end{center}
}

% Таблиця для відповідей із дробами
\newcommand{\answerTableTall}[5]{
\begin{center}
\begin{tabular}{|*{5}{>{\centering\arraybackslash}m{2.8cm}|}}
\hline
\rule[-0.3cm]{0pt}{0.8cm}\textbf{А} & \textbf{Б} & \textbf{В} & \textbf{Г} & \textbf{Д} \\
\hline
\rule[-0.9cm]{0pt}{2.0cm}#1 & 
\rule[-0.9cm]{0pt}{2.0cm}#2 & 
\rule[-0.9cm]{0pt}{2.0cm}#3 & 
\rule[-0.9cm]{0pt}{2.0cm}#4 & 
\rule[-0.9cm]{0pt}{2.0cm}#5 \\
\hline
\end{tabular}
\end{center}
}

\newcommand{\nmtyear}[1]{\hfill{\small\color{yearcolor}(AI Gen)}}

\begin{document}

\vspace{1cm}

\begin{center}
{\Large\textbf{\color{headerblue}ЗГЕНЕРОВАНІ ЗАВДАННЯ (AI)}}
\end{center}

\begin{center}
{\large Тема: \textbf{Арифметична прогресія}}
\end{center}

\vspace{0.5cm}
% === Arithmetic Progression: Find d ===
\noindent\makebox[1.5em][l]{\textbf{1.}}\parbox[t]{\dimexpr\textwidth-1.5em}{В арифметичній прогресії $(a_n)$: $a_1 = 9$, $a_3 = -11$. Визначте різницю $d$ прогресії. \nmtyear{2026}}
\vspace{0.3cm}

\answerTable{$d = -9$}{$d = -10$}{$d = -1$}{$d = -20$}{$d = 10$}

\vspace{0.5cm}

\noindent\makebox[1.5em][l]{\textbf{2.}}\parbox[t]{\dimexpr\textwidth-1.5em}{В арифметичній прогресії $(a_n)$: $a_1 = 17$, $a_3 = 15$. Визначте різницю $d$ прогресії. \nmtyear{2026}}
\vspace{0.3cm}

\answerTable{$d = 0$}{$d = 16$}{$d = -2$}{$d = -1$}{$d = 1$}

\vspace{0.5cm}

\noindent\makebox[1.5em][l]{\textbf{3.}}\parbox[t]{\dimexpr\textwidth-1.5em}{В арифметичній прогресії $(a_n)$: $a_1 = -8$, $a_3 = -24$. Визначте різницю $d$ прогресії. \nmtyear{2026}}
\vspace{0.3cm}

\answerTable{$d = -8$}{$d = -7$}{51}{$d = 8$}{$d = -16$}

\vspace{0.5cm}

\noindent\makebox[1.5em][l]{\textbf{4.}}\parbox[t]{\dimexpr\textwidth-1.5em}{В арифметичній прогресії $(a_n)$: $a_1 = 10$, $a_3 = 15$. Визначте різницю $d$ прогресії. \nmtyear{2026}}
\vspace{0.3cm}

\answerTable{$d = 12{,}5$}{$d = 5$}{$d = 2{,}5$}{$d = -2{,}5$}{$d = 3{,}5$}

\vspace{0.5cm}

\noindent\makebox[1.5em][l]{\textbf{5.}}\parbox[t]{\dimexpr\textwidth-1.5em}{В арифметичній прогресії $(a_n)$: $a_1 = 15$, $a_3 = 29$. Визначте різницю $d$ прогресії. \nmtyear{2026}}
\vspace{0.3cm}

\answerTable{$d = 14$}{$d = 22$}{$d = 8$}{$d = -7$}{$d = 7$}

\vspace{0.5cm}

\noindent\makebox[1.5em][l]{\textbf{6.}}\parbox[t]{\dimexpr\textwidth-1.5em}{В арифметичній прогресії $(a_n)$: $a_1 = 2$, $a_3 = -15$. Визначте різницю $d$ прогресії. \nmtyear{2026}}
\vspace{0.3cm}

\answerTable{$d = 8{,}5$}{$d = -7{,}5$}{$d = -17$}{$d = -6{,}5$}{$d = -8{,}5$}

\vspace{0.5cm}

\noindent\makebox[1.5em][l]{\textbf{7.}}\parbox[t]{\dimexpr\textwidth-1.5em}{В арифметичній прогресії $(a_n)$: $a_1 = -7$, $a_3 = 13$. Визначте різницю $d$ прогресії. \nmtyear{2026}}
\vspace{0.3cm}

\answerTable{$d = -10$}{$d = 11$}{$d = 3$}{$d = 10$}{$d = 20$}

\vspace{0.5cm}

\noindent\makebox[1.5em][l]{\textbf{8.}}\parbox[t]{\dimexpr\textwidth-1.5em}{В арифметичній прогресії $(a_n)$: $a_1 = 12$, $a_3 = 20$. Визначте різницю $d$ прогресії. \nmtyear{2026}}
\vspace{0.3cm}

\answerTable{$d = 4$}{$d = 5$}{$d = 8$}{$d = 16$}{$d = -4$}

\vspace{0.5cm}

\noindent\makebox[1.5em][l]{\textbf{9.}}\parbox[t]{\dimexpr\textwidth-1.5em}{В арифметичній прогресії $(a_n)$: $a_1 = 4$, $a_3 = 10$. Визначте різницю $d$ прогресії. \nmtyear{2026}}
\vspace{0.3cm}

\answerTable{$d = -3$}{$d = 4$}{$d = 7$}{$d = 3$}{$d = 6$}

\vspace{0.5cm}

\noindent\makebox[1.5em][l]{\textbf{10.}}\parbox[t]{\dimexpr\textwidth-1.5em}{В арифметичній прогресії $(a_n)$: $a_1 = -8$, $a_3 = -28$. Визначте різницю $d$ прогресії. \nmtyear{2026}}
\vspace{0.3cm}

\answerTable{$d = -10$}{$d = -9$}{$d = -20$}{$d = -18$}{$d = 10$}

\vspace{0.5cm}

\noindent\makebox[1.5em][l]{\textbf{11.}}\parbox[t]{\dimexpr\textwidth-1.5em}{В арифметичній прогресії $(a_n)$: $a_1 = -8$, $a_3 = -28$. Визначте різницю $d$ прогресії. \nmtyear{2026}}
\vspace{0.3cm}

\answerTable{$d = 10$}{$d = -18$}{$d = -9$}{$d = -10$}{$d = -20$}

\vspace{0.5cm}

\noindent\makebox[1.5em][l]{\textbf{12.}}\parbox[t]{\dimexpr\textwidth-1.5em}{В арифметичній прогресії $(a_n)$: $a_1 = -8$, $a_3 = -4$. Визначте різницю $d$ прогресії. \nmtyear{2026}}
\vspace{0.3cm}

\answerTable{$d = 2$}{$d = 4$}{$d = -6$}{$d = 3$}{$d = -2$}

\vspace{0.5cm}

\noindent\makebox[1.5em][l]{\textbf{13.}}\parbox[t]{\dimexpr\textwidth-1.5em}{В арифметичній прогресії $(a_n)$: $a_1 = -4$, $a_3 = -20$. Визначте різницю $d$ прогресії. \nmtyear{2026}}
\vspace{0.3cm}

\answerTable{$d = -8$}{$d = 8$}{$d = -12$}{$d = -16$}{$d = -7$}

\vspace{0.5cm}

\noindent\makebox[1.5em][l]{\textbf{14.}}\parbox[t]{\dimexpr\textwidth-1.5em}{В арифметичній прогресії $(a_n)$: $a_1 = 19$, $a_3 = 5$. Визначте різницю $d$ прогресії. \nmtyear{2026}}
\vspace{0.3cm}

\answerTable{$d = 12$}{$d = -14$}{$d = 7$}{$d = -7$}{$d = -6$}

\vspace{0.5cm}

\noindent\makebox[1.5em][l]{\textbf{15.}}\parbox[t]{\dimexpr\textwidth-1.5em}{В арифметичній прогресії $(a_n)$: $a_1 = 16$, $a_3 = 22$. Визначте різницю $d$ прогресії. \nmtyear{2026}}
\vspace{0.3cm}

\answerTable{$d = 3$}{$d = 6$}{$d = 19$}{$d = 4$}{$d = -3$}

\vspace{0.5cm}

\noindent\makebox[1.5em][l]{\textbf{16.}}\parbox[t]{\dimexpr\textwidth-1.5em}{В арифметичній прогресії $(a_n)$: $a_1 = -1$, $a_3 = 13$. Визначте різницю $d$ прогресії. \nmtyear{2026}}
\vspace{0.3cm}

\answerTable{$d = 14$}{$d = 7$}{$d = 6$}{$d = -7$}{$d = 8$}

\vspace{0.5cm}

\noindent\makebox[1.5em][l]{\textbf{17.}}\parbox[t]{\dimexpr\textwidth-1.5em}{В арифметичній прогресії $(a_n)$: $a_1 = 2$, $a_3 = 10$. Визначте різницю $d$ прогресії. \nmtyear{2026}}
\vspace{0.3cm}

\answerTable{$d = 4$}{$d = -4$}{$d = 5$}{$d = 6$}{$d = 8$}

\vspace{0.5cm}

\noindent\makebox[1.5em][l]{\textbf{18.}}\parbox[t]{\dimexpr\textwidth-1.5em}{В арифметичній прогресії $(a_n)$: $a_1 = 3$, $a_3 = 10$. Визначте різницю $d$ прогресії. \nmtyear{2026}}
\vspace{0.3cm}

\answerTable{$d = 7$}{$d = 3{,}5$}{$d = -3{,}5$}{$d = 6{,}5$}{$d = 4{,}5$}

\vspace{0.5cm}

\noindent\makebox[1.5em][l]{\textbf{19.}}\parbox[t]{\dimexpr\textwidth-1.5em}{В арифметичній прогресії $(a_n)$: $a_1 = 15$, $a_3 = 30$. Визначте різницю $d$ прогресії. \nmtyear{2026}}
\vspace{0.3cm}

\answerTable{$d = 22{,}5$}{$d = -7{,}5$}{$d = 7{,}5$}{$d = 15$}{$d = 8{,}5$}

\vspace{0.5cm}

\noindent\makebox[1.5em][l]{\textbf{20.}}\parbox[t]{\dimexpr\textwidth-1.5em}{В арифметичній прогресії $(a_n)$: $a_1 = 8$, $a_3 = 16$. Визначте різницю $d$ прогресії. \nmtyear{2026}}
\vspace{0.3cm}

\answerTable{$d = 8$}{$d = -4$}{$d = 12$}{$d = 4$}{$d = 5$}

\vspace{0.5cm}

% === Arithmetic Progression: Member Difference ===
\noindent\makebox[1.5em][l]{\textbf{21.}}\parbox[t]{\dimexpr\textwidth-1.5em}{В арифметичній прогресії $(a_n)$ відомо, що $a_{8} - a_{5} = 21$. Знайдіть значення виразу $a_{13} - a_{9}$. \nmtyear{2026}}
\vspace{0.3cm}

\answerTable{21}{23}{35}{28}{-28}

\vspace{0.5cm}

\noindent\makebox[1.5em][l]{\textbf{22.}}\parbox[t]{\dimexpr\textwidth-1.5em}{В арифметичній прогресії $(a_n)$ відомо, що $a_{4} - a_{1} = -27$. Знайдіть значення виразу $a_{2} - a_{3}$. \nmtyear{2026}}
\vspace{0.3cm}

\answerTable{-27}{18}{-9}{9}{0}

\vspace{0.5cm}

\noindent\makebox[1.5em][l]{\textbf{23.}}\parbox[t]{\dimexpr\textwidth-1.5em}{В арифметичній прогресії $(a_n)$ відомо, що $a_{6} - a_{2} = -28$. Знайдіть значення виразу $a_{3} - a_{5}$. \nmtyear{2026}}
\vspace{0.3cm}

\answerTable{21}{7}{-28}{14}{-14}

\vspace{0.5cm}

\noindent\makebox[1.5em][l]{\textbf{24.}}\parbox[t]{\dimexpr\textwidth-1.5em}{В арифметичній прогресії $(a_n)$ відомо, що $a_{6} - a_{3} = 9$. Знайдіть значення виразу $a_{3} - a_{5}$. \nmtyear{2026}}
\vspace{0.3cm}

\answerTable{-3}{9}{-9}{6}{-6}

\vspace{0.5cm}

\noindent\makebox[1.5em][l]{\textbf{25.}}\parbox[t]{\dimexpr\textwidth-1.5em}{В арифметичній прогресії $(a_n)$ відомо, що $a_{6} - a_{1} = 5$. Знайдіть значення виразу $a_{4} - a_{6}$. \nmtyear{2026}}
\vspace{0.3cm}

\answerTable{-3}{-2}{5}{2}{-1}

\vspace{0.5cm}

\noindent\makebox[1.5em][l]{\textbf{26.}}\parbox[t]{\dimexpr\textwidth-1.5em}{В арифметичній прогресії $(a_n)$ відомо, що $a_{8} - a_{4} = 12$. Знайдіть значення виразу $a_{11} - a_{9}$. \nmtyear{2026}}
\vspace{0.3cm}

\answerTable{6}{12}{3}{9}{-6}

\vspace{0.5cm}

\noindent\makebox[1.5em][l]{\textbf{27.}}\parbox[t]{\dimexpr\textwidth-1.5em}{В арифметичній прогресії $(a_n)$ відомо, що $a_{6} - a_{3} = -15$. Знайдіть значення виразу $a_{11} - a_{7}$. \nmtyear{2026}}
\vspace{0.3cm}

\answerTable{-20}{-25}{-19}{20}{-15}

\vspace{0.5cm}

\noindent\makebox[1.5em][l]{\textbf{28.}}\parbox[t]{\dimexpr\textwidth-1.5em}{В арифметичній прогресії $(a_n)$ відомо, що $a_{6} - a_{2} = -24$. Знайдіть значення виразу $a_{6} - a_{7}$. \nmtyear{2026}}
\vspace{0.3cm}

\answerTable{-24}{-6}{0}{6}{12}

\vspace{0.5cm}

\noindent\makebox[1.5em][l]{\textbf{29.}}\parbox[t]{\dimexpr\textwidth-1.5em}{В арифметичній прогресії $(a_n)$ відомо, що $a_{7} - a_{1} = -18$. Знайдіть значення виразу $a_{9} - a_{11}$. \nmtyear{2026}}
\vspace{0.3cm}

\answerTable{-6}{6}{3}{9}{-18}

\vspace{0.5cm}

\noindent\makebox[1.5em][l]{\textbf{30.}}\parbox[t]{\dimexpr\textwidth-1.5em}{В арифметичній прогресії $(a_n)$ відомо, що $a_{10} - a_{5} = 40$. Знайдіть значення виразу $a_{7} - a_{8}$. \nmtyear{2026}}
\vspace{0.3cm}

\answerTable{0}{8}{-16}{40}{-8}

\vspace{0.5cm}

\noindent\makebox[1.5em][l]{\textbf{31.}}\parbox[t]{\dimexpr\textwidth-1.5em}{В арифметичній прогресії $(a_n)$ відомо, що $a_{3} - a_{1} = 20$. Знайдіть значення виразу $a_{2} - a_{3}$. \nmtyear{2026}}
\vspace{0.3cm}

\answerTable{20}{10}{-20}{-10}{0}

\vspace{0.5cm}

\noindent\makebox[1.5em][l]{\textbf{32.}}\parbox[t]{\dimexpr\textwidth-1.5em}{В арифметичній прогресії $(a_n)$ відомо, що $a_{6} - a_{3} = -12$. Знайдіть значення виразу $a_{1} - a_{5}$. \nmtyear{2026}}
\vspace{0.3cm}

\answerTable{12}{16}{-12}{-16}{20}

\vspace{0.5cm}

\noindent\makebox[1.5em][l]{\textbf{33.}}\parbox[t]{\dimexpr\textwidth-1.5em}{В арифметичній прогресії $(a_n)$ відомо, що $a_{8} - a_{5} = -9$. Знайдіть значення виразу $a_{6} - a_{10}$. \nmtyear{2026}}
\vspace{0.3cm}

\answerTable{-9}{-12}{15}{12}{9}

\vspace{0.5cm}

\noindent\makebox[1.5em][l]{\textbf{34.}}\parbox[t]{\dimexpr\textwidth-1.5em}{В арифметичній прогресії $(a_n)$ відомо, що $a_{6} - a_{3} = -12$. Знайдіть значення виразу $a_{8} - a_{6}$. \nmtyear{2026}}
\vspace{0.3cm}

\answerTable{8}{-5}{-4}{-12}{-8}

\vspace{0.5cm}

\noindent\makebox[1.5em][l]{\textbf{35.}}\parbox[t]{\dimexpr\textwidth-1.5em}{В арифметичній прогресії $(a_n)$ відомо, що $a_{7} - a_{1} = 54$. Знайдіть значення виразу $a_{5} - a_{7}$. \nmtyear{2026}}
\vspace{0.3cm}

\answerTable{-27}{-9}{18}{54}{-18}

\vspace{0.5cm}

\noindent\makebox[1.5em][l]{\textbf{36.}}\parbox[t]{\dimexpr\textwidth-1.5em}{В арифметичній прогресії $(a_n)$ відомо, що $a_{7} - a_{3} = 28$. Знайдіть значення виразу $a_{9} - a_{10}$. \nmtyear{2026}}
\vspace{0.3cm}

\answerTable{28}{-14}{0}{-7}{7}

\vspace{0.5cm}

\noindent\makebox[1.5em][l]{\textbf{37.}}\parbox[t]{\dimexpr\textwidth-1.5em}{В арифметичній прогресії $(a_n)$ відомо, що $a_{7} - a_{3} = 36$. Знайдіть значення виразу $a_{7} - a_{10}$. \nmtyear{2026}}
\vspace{0.3cm}

\answerTable{27}{36}{-27}{-36}{-18}

\vspace{0.5cm}

\noindent\makebox[1.5em][l]{\textbf{38.}}\parbox[t]{\dimexpr\textwidth-1.5em}{В арифметичній прогресії $(a_n)$ відомо, що $a_{7} - a_{2} = -10$. Знайдіть значення виразу $a_{4} - a_{6}$. \nmtyear{2026}}
\vspace{0.3cm}

\answerTable{-4}{4}{-10}{2}{6}

\vspace{0.5cm}

\noindent\makebox[1.5em][l]{\textbf{39.}}\parbox[t]{\dimexpr\textwidth-1.5em}{В арифметичній прогресії $(a_n)$ відомо, що $a_{6} - a_{1} = 35$. Знайдіть значення виразу $a_{7} - a_{9}$. \nmtyear{2026}}
\vspace{0.3cm}

\answerTable{35}{-7}{-14}{-21}{14}

\vspace{0.5cm}

\noindent\makebox[1.5em][l]{\textbf{40.}}\parbox[t]{\dimexpr\textwidth-1.5em}{В арифметичній прогресії $(a_n)$ відомо, що $a_{11} - a_{5} = -60$. Знайдіть значення виразу $a_{1} - a_{3}$. \nmtyear{2026}}
\vspace{0.3cm}

\answerTable{-60}{30}{10}{-20}{20}

\vspace{0.5cm}

% === Arithmetic Progression: Sum ===
\noindent\makebox[1.5em][l]{\textbf{41.}}\parbox[t]{\dimexpr\textwidth-1.5em}{Обчисліть суму перших 100-ти членів арифметичної прогресії $(a_n)$, якщо $a_1 + a_{100} = 40$. \nmtyear{2026}}
\vspace{0.3cm}

\answerTable{2000}{4000}{2100}{40}{1900}

\vspace{0.5cm}

\noindent\makebox[1.5em][l]{\textbf{42.}}\parbox[t]{\dimexpr\textwidth-1.5em}{Обчисліть суму перших 10-ти членів арифметичної прогресії $(a_n)$, якщо $a_1 + a_{10} = -12$. \nmtyear{2026}}
\vspace{0.3cm}

\answerTable{-70}{-12}{-50}{-60}{60}

\vspace{0.5cm}

\noindent\makebox[1.5em][l]{\textbf{43.}}\parbox[t]{\dimexpr\textwidth-1.5em}{Обчисліть суму перших 16-ти членів арифметичної прогресії $(a_n)$, якщо $a_1 + a_{16} = -23$. \nmtyear{2026}}
\vspace{0.3cm}

\answerTable{184}{-200}{-168}{-184}{-23}

\vspace{0.5cm}

\noindent\makebox[1.5em][l]{\textbf{44.}}\parbox[t]{\dimexpr\textwidth-1.5em}{Обчисліть суму перших 8-ти членів арифметичної прогресії $(a_n)$, якщо $a_1 + a_{8} = -28$. \nmtyear{2026}}
\vspace{0.3cm}

\answerTable{112}{-112}{-104}{-224}{-120}

\vspace{0.5cm}

\noindent\makebox[1.5em][l]{\textbf{45.}}\parbox[t]{\dimexpr\textwidth-1.5em}{Обчисліть суму перших 8-ти членів арифметичної прогресії $(a_n)$, якщо $a_1 + a_{8} = 17$. \nmtyear{2026}}
\vspace{0.3cm}

\answerTable{76}{-68}{60}{17}{68}

\vspace{0.5cm}

\noindent\makebox[1.5em][l]{\textbf{46.}}\parbox[t]{\dimexpr\textwidth-1.5em}{Обчисліть суму перших 12-ти членів арифметичної прогресії $(a_n)$, якщо $a_1 + a_{12} = 39$. \nmtyear{2026}}
\vspace{0.3cm}

\answerTable{222}{468}{234}{-234}{246}

\vspace{0.5cm}

\noindent\makebox[1.5em][l]{\textbf{47.}}\parbox[t]{\dimexpr\textwidth-1.5em}{Обчисліть суму перших 10-ти членів арифметичної прогресії $(a_n)$, якщо $a_1 + a_{10} = -47$. \nmtyear{2026}}
\vspace{0.3cm}

\answerTable{-225}{235}{-235}{-245}{-470}

\vspace{0.5cm}

\noindent\makebox[1.5em][l]{\textbf{48.}}\parbox[t]{\dimexpr\textwidth-1.5em}{Обчисліть суму перших 8-ти членів арифметичної прогресії $(a_n)$, якщо $a_1 + a_{8} = 43$. \nmtyear{2026}}
\vspace{0.3cm}

\answerTable{180}{172}{-172}{43}{164}

\vspace{0.5cm}

\noindent\makebox[1.5em][l]{\textbf{49.}}\parbox[t]{\dimexpr\textwidth-1.5em}{Обчисліть суму перших 10-ти членів арифметичної прогресії $(a_n)$, якщо $a_1 + a_{10} = -33$. \nmtyear{2026}}
\vspace{0.3cm}

\answerTable{-175}{-165}{-330}{-33}{-155}

\vspace{0.5cm}

\noindent\makebox[1.5em][l]{\textbf{50.}}\parbox[t]{\dimexpr\textwidth-1.5em}{Обчисліть суму перших 12-ти членів арифметичної прогресії $(a_n)$, якщо $a_1 + a_{12} = 0$. \nmtyear{2026}}
\vspace{0.3cm}

\answerTable{12}{-12}{8}{6}{0}

\vspace{0.5cm}

\noindent\makebox[1.5em][l]{\textbf{51.}}\parbox[t]{\dimexpr\textwidth-1.5em}{Обчисліть суму перших 8-ти членів арифметичної прогресії $(a_n)$, якщо $a_1 + a_{8} = -5$. \nmtyear{2026}}
\vspace{0.3cm}

\answerTable{-20}{-28}{-12}{-40}{20}

\vspace{0.5cm}

\noindent\makebox[1.5em][l]{\textbf{52.}}\parbox[t]{\dimexpr\textwidth-1.5em}{Обчисліть суму перших 16-ти членів арифметичної прогресії $(a_n)$, якщо $a_1 + a_{16} = -41$. \nmtyear{2026}}
\vspace{0.3cm}

\answerTable{328}{-328}{-344}{-41}{-656}

\vspace{0.5cm}

\noindent\makebox[1.5em][l]{\textbf{53.}}\parbox[t]{\dimexpr\textwidth-1.5em}{Обчисліть суму перших 8-ти членів арифметичної прогресії $(a_n)$, якщо $a_1 + a_{8} = -49$. \nmtyear{2026}}
\vspace{0.3cm}

\answerTable{-196}{-392}{196}{-49}{-188}

\vspace{0.5cm}

\noindent\makebox[1.5em][l]{\textbf{54.}}\parbox[t]{\dimexpr\textwidth-1.5em}{Обчисліть суму перших 16-ти членів арифметичної прогресії $(a_n)$, якщо $a_1 + a_{16} = -16$. \nmtyear{2026}}
\vspace{0.3cm}

\answerTable{-128}{128}{-112}{-16}{-256}

\vspace{0.5cm}

\noindent\makebox[1.5em][l]{\textbf{55.}}\parbox[t]{\dimexpr\textwidth-1.5em}{Обчисліть суму перших 10-ти членів арифметичної прогресії $(a_n)$, якщо $a_1 + a_{10} = -37$. \nmtyear{2026}}
\vspace{0.3cm}

\answerTable{-195}{185}{-370}{-185}{-175}

\vspace{0.5cm}

\noindent\makebox[1.5em][l]{\textbf{56.}}\parbox[t]{\dimexpr\textwidth-1.5em}{Обчисліть суму перших 12-ти членів арифметичної прогресії $(a_n)$, якщо $a_1 + a_{12} = 48$. \nmtyear{2026}}
\vspace{0.3cm}

\answerTable{288}{276}{-288}{576}{300}

\vspace{0.5cm}

\noindent\makebox[1.5em][l]{\textbf{57.}}\parbox[t]{\dimexpr\textwidth-1.5em}{Обчисліть суму перших 16-ти членів арифметичної прогресії $(a_n)$, якщо $a_1 + a_{16} = -25$. \nmtyear{2026}}
\vspace{0.3cm}

\answerTable{-184}{-25}{-400}{-200}{-216}

\vspace{0.5cm}

\noindent\makebox[1.5em][l]{\textbf{58.}}\parbox[t]{\dimexpr\textwidth-1.5em}{Обчисліть суму перших 100-ти членів арифметичної прогресії $(a_n)$, якщо $a_1 + a_{100} = -47$. \nmtyear{2026}}
\vspace{0.3cm}

\answerTable{-47}{2350}{-4700}{-2350}{-2250}

\vspace{0.5cm}

\noindent\makebox[1.5em][l]{\textbf{59.}}\parbox[t]{\dimexpr\textwidth-1.5em}{Обчисліть суму перших 10-ти членів арифметичної прогресії $(a_n)$, якщо $a_1 + a_{10} = -29$. \nmtyear{2026}}
\vspace{0.3cm}

\answerTable{-145}{-155}{-135}{145}{-29}

\vspace{0.5cm}

\noindent\makebox[1.5em][l]{\textbf{60.}}\parbox[t]{\dimexpr\textwidth-1.5em}{Обчисліть суму перших 8-ти членів арифметичної прогресії $(a_n)$, якщо $a_1 + a_{8} = 15$. \nmtyear{2026}}
\vspace{0.3cm}

\answerTable{15}{60}{52}{120}{68}

\vspace{0.5cm}

% === Arithmetic Progression: Count Terms ===
\noindent\makebox[1.5em][l]{\textbf{61.}}\parbox[t]{\dimexpr\textwidth-1.5em}{В арифметичній прогресії $(a_n)$ перший член $a_1 = 24{,}1$, різниця $d = -3$. Скільки всього \textit{додатних} членів має ця прогресія? \nmtyear{2026}}
\vspace{0.3cm}

\answerTable{9}{11}{10}{8}{7}

\vspace{0.5cm}

\noindent\makebox[1.5em][l]{\textbf{62.}}\parbox[t]{\dimexpr\textwidth-1.5em}{В арифметичній прогресії $(a_n)$ перший член $a_1 = -18{,}1$, різниця $d = 1{,}5$. Скільки всього \textit{від'ємних} членів має ця прогресія? \nmtyear{2026}}
\vspace{0.3cm}

\answerTable{13}{15}{11}{12}{14}

\vspace{0.5cm}

\noindent\makebox[1.5em][l]{\textbf{63.}}\parbox[t]{\dimexpr\textwidth-1.5em}{В арифметичній прогресії $(a_n)$ перший член $a_1 = 2{,}9$, різниця $d = -0{,}4$. Скільки всього \textit{додатних} членів має ця прогресія? \nmtyear{2026}}
\vspace{0.3cm}

\answerTable{8}{6}{7}{9}{5}

\vspace{0.5cm}

\noindent\makebox[1.5em][l]{\textbf{64.}}\parbox[t]{\dimexpr\textwidth-1.5em}{В арифметичній прогресії $(a_n)$ перший член $a_1 = -13$, різниця $d = 1$. Скільки всього \textit{від'ємних} членів має ця прогресія? \nmtyear{2026}}
\vspace{0.3cm}

\answerTable{11}{12}{13}{14}{10}

\vspace{0.5cm}

\noindent\makebox[1.5em][l]{\textbf{65.}}\parbox[t]{\dimexpr\textwidth-1.5em}{В арифметичній прогресії $(a_n)$ перший член $a_1 = 6{,}9$, різниця $d = -0{,}8$. Скільки всього \textit{додатних} членів має ця прогресія? \nmtyear{2026}}
\vspace{0.3cm}

\answerTable{8}{9}{11}{7}{10}

\vspace{0.5cm}

\noindent\makebox[1.5em][l]{\textbf{66.}}\parbox[t]{\dimexpr\textwidth-1.5em}{В арифметичній прогресії $(a_n)$ перший член $a_1 = -26{,}2$, різниця $d = 2$. Скільки всього \textit{від'ємних} членів має ця прогресія? \nmtyear{2026}}
\vspace{0.3cm}

\answerTable{12}{13}{16}{14}{15}

\vspace{0.5cm}

\noindent\makebox[1.5em][l]{\textbf{67.}}\parbox[t]{\dimexpr\textwidth-1.5em}{В арифметичній прогресії $(a_n)$ перший член $a_1 = -6{,}7$, різниця $d = 0{,}8$. Скільки всього \textit{від'ємних} членів має ця прогресія? \nmtyear{2026}}
\vspace{0.3cm}

\answerTable{10}{7}{8}{9}{11}

\vspace{0.5cm}

\noindent\makebox[1.5em][l]{\textbf{68.}}\parbox[t]{\dimexpr\textwidth-1.5em}{В арифметичній прогресії $(a_n)$ перший член $a_1 = -12{,}7$, різниця $d = 2{,}5$. Скільки всього \textit{від'ємних} членів має ця прогресія? \nmtyear{2026}}
\vspace{0.3cm}

\answerTable{5}{4}{8}{6}{7}

\vspace{0.5cm}

\noindent\makebox[1.5em][l]{\textbf{69.}}\parbox[t]{\dimexpr\textwidth-1.5em}{В арифметичній прогресії $(a_n)$ перший член $a_1 = 19$, різниця $d = -2$. Скільки всього \textit{додатних} членів має ця прогресія? \nmtyear{2026}}
\vspace{0.3cm}

\answerTable{10}{8}{12}{9}{11}

\vspace{0.5cm}

\noindent\makebox[1.5em][l]{\textbf{70.}}\parbox[t]{\dimexpr\textwidth-1.5em}{В арифметичній прогресії $(a_n)$ перший член $a_1 = -4{,}6$, різниця $d = 0{,}4$. Скільки всього \textit{від'ємних} членів має ця прогресія? \nmtyear{2026}}
\vspace{0.3cm}

\answerTable{11}{10}{8}{9}{12}

\vspace{0.5cm}

\noindent\makebox[1.5em][l]{\textbf{71.}}\parbox[t]{\dimexpr\textwidth-1.5em}{В арифметичній прогресії $(a_n)$ перший член $a_1 = -3{,}6$, різниця $d = 0{,}5$. Скільки всього \textit{від'ємних} членів має ця прогресія? \nmtyear{2026}}
\vspace{0.3cm}

\answerTable{9}{10}{7}{6}{8}

\vspace{0.5cm}

\noindent\makebox[1.5em][l]{\textbf{72.}}\parbox[t]{\dimexpr\textwidth-1.5em}{В арифметичній прогресії $(a_n)$ перший член $a_1 = -5{,}8$, різниця $d = 0{,}5$. Скільки всього \textit{від'ємних} членів має ця прогресія? \nmtyear{2026}}
\vspace{0.3cm}

\answerTable{10}{13}{11}{14}{12}

\vspace{0.5cm}

\noindent\makebox[1.5em][l]{\textbf{73.}}\parbox[t]{\dimexpr\textwidth-1.5em}{В арифметичній прогресії $(a_n)$ перший член $a_1 = -6{,}1$, різниця $d = 0{,}4$. Скільки всього \textit{від'ємних} членів має ця прогресія? \nmtyear{2026}}
\vspace{0.3cm}

\answerTable{14}{17}{16}{13}{15}

\vspace{0.5cm}

\noindent\makebox[1.5em][l]{\textbf{74.}}\parbox[t]{\dimexpr\textwidth-1.5em}{В арифметичній прогресії $(a_n)$ перший член $a_1 = -25$, різниця $d = 3$. Скільки всього \textit{від'ємних} членів має ця прогресія? \nmtyear{2026}}
\vspace{0.3cm}

\answerTable{9}{11}{7}{10}{8}

\vspace{0.5cm}

\noindent\makebox[1.5em][l]{\textbf{75.}}\parbox[t]{\dimexpr\textwidth-1.5em}{В арифметичній прогресії $(a_n)$ перший член $a_1 = -5$, різниця $d = 0{,}8$. Скільки всього \textit{від'ємних} членів має ця прогресія? \nmtyear{2026}}
\vspace{0.3cm}

\answerTable{8}{4}{5}{6}{7}

\vspace{0.5cm}

\noindent\makebox[1.5em][l]{\textbf{76.}}\parbox[t]{\dimexpr\textwidth-1.5em}{В арифметичній прогресії $(a_n)$ перший член $a_1 = 7{,}6$, різниця $d = -0{,}8$. Скільки всього \textit{додатних} членів має ця прогресія? \nmtyear{2026}}
\vspace{0.3cm}

\answerTable{6}{8}{10}{7}{9}

\vspace{0.5cm}

\noindent\makebox[1.5em][l]{\textbf{77.}}\parbox[t]{\dimexpr\textwidth-1.5em}{В арифметичній прогресії $(a_n)$ перший член $a_1 = -8{,}2$, різниця $d = 0{,}8$. Скільки всього \textit{від'ємних} членів має ця прогресія? \nmtyear{2026}}
\vspace{0.3cm}

\answerTable{10}{11}{13}{12}{9}

\vspace{0.5cm}

\noindent\makebox[1.5em][l]{\textbf{78.}}\parbox[t]{\dimexpr\textwidth-1.5em}{В арифметичній прогресії $(a_n)$ перший член $a_1 = 9{,}5$, різниця $d = -1$. Скільки всього \textit{додатних} членів має ця прогресія? \nmtyear{2026}}
\vspace{0.3cm}

\answerTable{12}{8}{11}{10}{9}

\vspace{0.5cm}

\noindent\makebox[1.5em][l]{\textbf{79.}}\parbox[t]{\dimexpr\textwidth-1.5em}{В арифметичній прогресії $(a_n)$ перший член $a_1 = 32{,}7$, різниця $d = -2{,}5$. Скільки всього \textit{додатних} членів має ця прогресія? \nmtyear{2026}}
\vspace{0.3cm}

\answerTable{13}{15}{16}{12}{14}

\vspace{0.5cm}

\noindent\makebox[1.5em][l]{\textbf{80.}}\parbox[t]{\dimexpr\textwidth-1.5em}{В арифметичній прогресії $(a_n)$ перший член $a_1 = -4{,}6$, різниця $d = 0{,}4$. Скільки всього \textit{від'ємних} членів має ця прогресія? \nmtyear{2026}}
\vspace{0.3cm}

\answerTable{11}{9}{10}{8}{12}

\vspace{0.5cm}

% === Arithmetic Progression: Middle Term ===
\noindent\makebox[1.5em][l]{\textbf{81.}}\parbox[t]{\dimexpr\textwidth-1.5em}{Визначте 5-й член $a_{5}$ арифметичної прогресії $(a_n)$, у якої $a_{4} = 36$, $a_{6} = 44$. \nmtyear{2026}}
\vspace{0.3cm}

\answerTable{80}{4}{40}{36}{8}

\vspace{0.5cm}

\noindent\makebox[1.5em][l]{\textbf{82.}}\parbox[t]{\dimexpr\textwidth-1.5em}{Визначте 11-й член $a_{11}$ арифметичної прогресії $(a_n)$, у якої $a_{10} = 12$, $a_{12} = 8$. \nmtyear{2026}}
\vspace{0.3cm}

\answerTable{10}{20}{12}{8}{4}

\vspace{0.5cm}

\noindent\makebox[1.5em][l]{\textbf{83.}}\parbox[t]{\dimexpr\textwidth-1.5em}{Визначте 7-й член $a_{7}$ арифметичної прогресії $(a_n)$, у якої $a_{6} = 33$, $a_{8} = 36$. \nmtyear{2026}}
\vspace{0.3cm}

\answerTable{33}{36}{3}{34{,}5}{1{,}5}

\vspace{0.5cm}

\noindent\makebox[1.5em][l]{\textbf{84.}}\parbox[t]{\dimexpr\textwidth-1.5em}{Визначте 19-й член $a_{19}$ арифметичної прогресії $(a_n)$, у якої $a_{18} = 16$, $a_{20} = 20$. \nmtyear{2026}}
\vspace{0.3cm}

\answerTable{16}{20}{4}{18}{36}

\vspace{0.5cm}

\noindent\makebox[1.5em][l]{\textbf{85.}}\parbox[t]{\dimexpr\textwidth-1.5em}{Визначте 9-й член $a_{9}$ арифметичної прогресії $(a_n)$, у якої $a_{8} = 42$, $a_{10} = 40$. \nmtyear{2026}}
\vspace{0.3cm}

\answerTable{82}{42}{2}{41}{40}

\vspace{0.5cm}

\noindent\makebox[1.5em][l]{\textbf{86.}}\parbox[t]{\dimexpr\textwidth-1.5em}{Визначте 17-й член $a_{17}$ арифметичної прогресії $(a_n)$, у якої $a_{16} = 30$, $a_{18} = 38$. \nmtyear{2026}}
\vspace{0.3cm}

\answerTable{68}{8}{34}{30}{4}

\vspace{0.5cm}

\noindent\makebox[1.5em][l]{\textbf{87.}}\parbox[t]{\dimexpr\textwidth-1.5em}{Визначте 5-й член $a_{5}$ арифметичної прогресії $(a_n)$, у якої $a_{4} = 34$, $a_{6} = 35$. \nmtyear{2026}}
\vspace{0.3cm}

\answerTable{35}{1}{0{,}5}{34}{34{,}5}

\vspace{0.5cm}

\noindent\makebox[1.5em][l]{\textbf{88.}}\parbox[t]{\dimexpr\textwidth-1.5em}{Визначте 7-й член $a_{7}$ арифметичної прогресії $(a_n)$, у якої $a_{6} = -1$, $a_{8} = 1$. \nmtyear{2026}}
\vspace{0.3cm}

\answerTable{1}{0}{-1}{-9}{2}

\vspace{0.5cm}

\noindent\makebox[1.5em][l]{\textbf{89.}}\parbox[t]{\dimexpr\textwidth-1.5em}{Визначте 18-й член $a_{18}$ арифметичної прогресії $(a_n)$, у якої $a_{17} = 24$, $a_{19} = 25$. \nmtyear{2026}}
\vspace{0.3cm}

\answerTable{0{,}5}{1}{24{,}5}{24}{49}

\vspace{0.5cm}

\noindent\makebox[1.5em][l]{\textbf{90.}}\parbox[t]{\dimexpr\textwidth-1.5em}{Визначте 11-й член $a_{11}$ арифметичної прогресії $(a_n)$, у якої $a_{10} = -13$, $a_{12} = -3$. \nmtyear{2026}}
\vspace{0.3cm}

\answerTable{10}{5}{-16}{-8}{-13}

\vspace{0.5cm}

\noindent\makebox[1.5em][l]{\textbf{91.}}\parbox[t]{\dimexpr\textwidth-1.5em}{Визначте 19-й член $a_{19}$ арифметичної прогресії $(a_n)$, у якої $a_{18} = 43$, $a_{20} = 46$. \nmtyear{2026}}
\vspace{0.3cm}

\answerTable{89}{43}{46}{1{,}5}{44{,}5}

\vspace{0.5cm}

\noindent\makebox[1.5em][l]{\textbf{92.}}\parbox[t]{\dimexpr\textwidth-1.5em}{Визначте 10-й член $a_{10}$ арифметичної прогресії $(a_n)$, у якої $a_{9} = -7$, $a_{11} = -1$. \nmtyear{2026}}
\vspace{0.3cm}

\answerTable{3}{-7}{6}{-4}{-1}

\vspace{0.5cm}

\noindent\makebox[1.5em][l]{\textbf{93.}}\parbox[t]{\dimexpr\textwidth-1.5em}{Визначте 19-й член $a_{19}$ арифметичної прогресії $(a_n)$, у якої $a_{18} = -4$, $a_{20} = 2$. \nmtyear{2026}}
\vspace{0.3cm}

\answerTable{-2}{-1}{6}{2}{-4}

\vspace{0.5cm}

\noindent\makebox[1.5em][l]{\textbf{94.}}\parbox[t]{\dimexpr\textwidth-1.5em}{Визначте 6-й член $a_{6}$ арифметичної прогресії $(a_n)$, у якої $a_{5} = 36$, $a_{7} = 46$. \nmtyear{2026}}
\vspace{0.3cm}

\answerTable{82}{5}{36}{41}{10}

\vspace{0.5cm}

\noindent\makebox[1.5em][l]{\textbf{95.}}\parbox[t]{\dimexpr\textwidth-1.5em}{Визначте 11-й член $a_{11}$ арифметичної прогресії $(a_n)$, у якої $a_{10} = -3$, $a_{12} = -5$. \nmtyear{2026}}
\vspace{0.3cm}

\answerTable{-3}{-4}{-8}{-5}{-1}

\vspace{0.5cm}

\noindent\makebox[1.5em][l]{\textbf{96.}}\parbox[t]{\dimexpr\textwidth-1.5em}{Визначте 7-й член $a_{7}$ арифметичної прогресії $(a_n)$, у якої $a_{6} = 12$, $a_{8} = 6$. \nmtyear{2026}}
\vspace{0.3cm}

\answerTable{12}{-3}{9}{6}{18}

\vspace{0.5cm}

\noindent\makebox[1.5em][l]{\textbf{97.}}\parbox[t]{\dimexpr\textwidth-1.5em}{Визначте 13-й член $a_{13}$ арифметичної прогресії $(a_n)$, у якої $a_{12} = 32$, $a_{14} = 42$. \nmtyear{2026}}
\vspace{0.3cm}

\answerTable{37}{42}{10}{5}{32}

\vspace{0.5cm}

\noindent\makebox[1.5em][l]{\textbf{98.}}\parbox[t]{\dimexpr\textwidth-1.5em}{Визначте 10-й член $a_{10}$ арифметичної прогресії $(a_n)$, у якої $a_{9} = 13$, $a_{11} = 23$. \nmtyear{2026}}
\vspace{0.3cm}

\answerTable{5}{13}{10}{23}{18}

\vspace{0.5cm}

\noindent\makebox[1.5em][l]{\textbf{99.}}\parbox[t]{\dimexpr\textwidth-1.5em}{Визначте 20-й член $a_{20}$ арифметичної прогресії $(a_n)$, у якої $a_{19} = -2$, $a_{21} = 1$. \nmtyear{2026}}
\vspace{0.3cm}

\answerTable{1}{-0{,}5}{3}{-1}{1{,}5}

\vspace{0.5cm}

\noindent\makebox[1.5em][l]{\textbf{100.}}\parbox[t]{\dimexpr\textwidth-1.5em}{Визначте 17-й член $a_{17}$ арифметичної прогресії $(a_n)$, у якої $a_{16} = 3$, $a_{18} = -3$. \nmtyear{2026}}
\vspace{0.3cm}

\answerTable{-9}{0}{-3}{3}{6}

\vspace{0.5cm}

% === Arithmetic Progression: Formula Search ===
\noindent\makebox[1.5em][l]{\textbf{101.}}\parbox[t]{\dimexpr\textwidth-1.5em}{Арифметичну прогресію $(a_n)$ задано формулою $n$-го члена $a_n = 24  -1{,}5n$. Визначте номер члена, значення якого дорівнює $-31{,}5$. \nmtyear{2026}}
\vspace{0.3cm}

\answerTable{36}{21}{27}{37}{38}

\vspace{0.5cm}

\noindent\makebox[1.5em][l]{\textbf{102.}}\parbox[t]{\dimexpr\textwidth-1.5em}{Арифметичну прогресію $(a_n)$ задано формулою $n$-го члена $a_n = 40 + 1{,}5n$. Визначте номер члена, значення якого дорівнює $107{,}5$. \nmtyear{2026}}
\vspace{0.3cm}

\answerTable{45}{44}{35}{55}{71{,}67}

\vspace{0.5cm}

\noindent\makebox[1.5em][l]{\textbf{103.}}\parbox[t]{\dimexpr\textwidth-1.5em}{Арифметичну прогресію $(a_n)$ задано формулою $n$-го члена $a_n = 48 + 3n$. Визначте номер члена, значення якого дорівнює $69$. \nmtyear{2026}}
\vspace{0.3cm}

\answerTable{17}{23}{7}{1}{8}

\vspace{0.5cm}

\noindent\makebox[1.5em][l]{\textbf{104.}}\parbox[t]{\dimexpr\textwidth-1.5em}{Арифметичну прогресію $(a_n)$ задано формулою $n$-го члена $a_n = 20  -0{,}5n$. Визначте номер члена, значення якого дорівнює $4{,}5$. \nmtyear{2026}}
\vspace{0.3cm}

\answerTable{21}{32}{41}{30}{31}

\vspace{0.5cm}

\noindent\makebox[1.5em][l]{\textbf{105.}}\parbox[t]{\dimexpr\textwidth-1.5em}{Арифметичну прогресію $(a_n)$ задано формулою $n$-го члена $a_n = 19  -3n$. Визначте номер члена, значення якого дорівнює $4$. \nmtyear{2026}}
\vspace{0.3cm}

\answerTable{15}{1}{6}{5}{4}

\vspace{0.5cm}

\noindent\makebox[1.5em][l]{\textbf{106.}}\parbox[t]{\dimexpr\textwidth-1.5em}{Арифметичну прогресію $(a_n)$ задано формулою $n$-го члена $a_n = 32  -3n$. Визначте номер члена, значення якого дорівнює $-1$. \nmtyear{2026}}
\vspace{0.3cm}

\answerTable{21}{11}{1}{0{,}33}{12}

\vspace{0.5cm}

\noindent\makebox[1.5em][l]{\textbf{107.}}\parbox[t]{\dimexpr\textwidth-1.5em}{Арифметичну прогресію $(a_n)$ задано формулою $n$-го члена $a_n = 28 + 1{,}5n$. Визначте номер члена, значення якого дорівнює $83{,}5$. \nmtyear{2026}}
\vspace{0.3cm}

\answerTable{36}{37}{47}{38}{55{,}67}

\vspace{0.5cm}

\noindent\makebox[1.5em][l]{\textbf{108.}}\parbox[t]{\dimexpr\textwidth-1.5em}{Арифметичну прогресію $(a_n)$ задано формулою $n$-го члена $a_n = 24 + 3n$. Визначте номер члена, значення якого дорівнює $141$. \nmtyear{2026}}
\vspace{0.3cm}

\answerTable{39}{38}{49}{47}{40}

\vspace{0.5cm}

\noindent\makebox[1.5em][l]{\textbf{109.}}\parbox[t]{\dimexpr\textwidth-1.5em}{Арифметичну прогресію $(a_n)$ задано формулою $n$-го члена $a_n = 46  -4n$. Визначте номер члена, значення якого дорівнює $-146$. \nmtyear{2026}}
\vspace{0.3cm}

\answerTable{36{,}5}{38}{58}{48}{47}

\vspace{0.5cm}

\noindent\makebox[1.5em][l]{\textbf{110.}}\parbox[t]{\dimexpr\textwidth-1.5em}{Арифметичну прогресію $(a_n)$ задано формулою $n$-го члена $a_n = 21 + 3n$. Визначте номер члена, значення якого дорівнює $111$. \nmtyear{2026}}
\vspace{0.3cm}

\answerTable{29}{30}{20}{40}{37}

\vspace{0.5cm}

\noindent\makebox[1.5em][l]{\textbf{111.}}\parbox[t]{\dimexpr\textwidth-1.5em}{Арифметичну прогресію $(a_n)$ задано формулою $n$-го члена $a_n = 39  -1{,}5n$. Визначте номер члена, значення якого дорівнює $-24$. \nmtyear{2026}}
\vspace{0.3cm}

\answerTable{42}{41}{32}{52}{43}

\vspace{0.5cm}

\noindent\makebox[1.5em][l]{\textbf{112.}}\parbox[t]{\dimexpr\textwidth-1.5em}{Арифметичну прогресію $(a_n)$ задано формулою $n$-го члена $a_n = 46  -1{,}5n$. Визначте номер члена, значення якого дорівнює $-24{,}5$. \nmtyear{2026}}
\vspace{0.3cm}

\answerTable{57}{47}{37}{46}{48}

\vspace{0.5cm}

\noindent\makebox[1.5em][l]{\textbf{113.}}\parbox[t]{\dimexpr\textwidth-1.5em}{Арифметичну прогресію $(a_n)$ задано формулою $n$-го члена $a_n = 14 + 2{,}5n$. Визначте номер члена, значення якого дорівнює $106{,}5$. \nmtyear{2026}}
\vspace{0.3cm}

\answerTable{36}{37}{38}{27}{42{,}6}

\vspace{0.5cm}

\noindent\makebox[1.5em][l]{\textbf{114.}}\parbox[t]{\dimexpr\textwidth-1.5em}{Арифметичну прогресію $(a_n)$ задано формулою $n$-го члена $a_n = 17 + 3n$. Визначте номер члена, значення якого дорівнює $131$. \nmtyear{2026}}
\vspace{0.3cm}

\answerTable{43{,}67}{37}{48}{38}{39}

\vspace{0.5cm}

\noindent\makebox[1.5em][l]{\textbf{115.}}\parbox[t]{\dimexpr\textwidth-1.5em}{Арифметичну прогресію $(a_n)$ задано формулою $n$-го члена $a_n = 26  -2{,}5n$. Визначте номер члена, значення якого дорівнює $-21{,}5$. \nmtyear{2026}}
\vspace{0.3cm}

\answerTable{9}{29}{18}{20}{19}

\vspace{0.5cm}

\noindent\makebox[1.5em][l]{\textbf{116.}}\parbox[t]{\dimexpr\textwidth-1.5em}{Арифметичну прогресію $(a_n)$ задано формулою $n$-го члена $a_n = 19 + 1{,}5n$. Визначте номер члена, значення якого дорівнює $53{,}5$. \nmtyear{2026}}
\vspace{0.3cm}

\answerTable{35{,}67}{23}{22}{24}{33}

\vspace{0.5cm}

\noindent\makebox[1.5em][l]{\textbf{117.}}\parbox[t]{\dimexpr\textwidth-1.5em}{Арифметичну прогресію $(a_n)$ задано формулою $n$-го члена $a_n = 40  -3n$. Визначте номер члена, значення якого дорівнює $-20$. \nmtyear{2026}}
\vspace{0.3cm}

\answerTable{20}{30}{10}{21}{19}

\vspace{0.5cm}

\noindent\makebox[1.5em][l]{\textbf{118.}}\parbox[t]{\dimexpr\textwidth-1.5em}{Арифметичну прогресію $(a_n)$ задано формулою $n$-го члена $a_n = 16 + 2{,}5n$. Визначте номер члена, значення якого дорівнює $58{,}5$. \nmtyear{2026}}
\vspace{0.3cm}

\answerTable{7}{17}{23{,}4}{18}{27}

\vspace{0.5cm}

\noindent\makebox[1.5em][l]{\textbf{119.}}\parbox[t]{\dimexpr\textwidth-1.5em}{Арифметичну прогресію $(a_n)$ задано формулою $n$-го члена $a_n = 23 + 3n$. Визначте номер члена, значення якого дорівнює $50$. \nmtyear{2026}}
\vspace{0.3cm}

\answerTable{9}{16{,}67}{10}{19}{8}

\vspace{0.5cm}

\noindent\makebox[1.5em][l]{\textbf{120.}}\parbox[t]{\dimexpr\textwidth-1.5em}{Арифметичну прогресію $(a_n)$ задано формулою $n$-го члена $a_n = 46 + 4n$. Визначте номер члена, значення якого дорівнює $70$. \nmtyear{2026}}
\vspace{0.3cm}

\answerTable{6}{7}{16}{5}{1}

\vspace{0.5cm}

% === Arithmetic Progression: Word Problem ===
\noindent\makebox[1.5em][l]{\textbf{121.}}\parbox[t]{\dimexpr\textwidth-1.5em}{За умовами договору позичальник повинен повернути кредит протягом 12 місяців. Першого місяця він має повернути 740 \textit{грн}, а кожного наступного місяця — на 20 \textit{грн} менше, ніж попереднього. Визначте загальну суму (у \textit{грн}), яку повинен позичальник повернути протягом 12 місяців. \nmtyear{2026}}
\vspace{0.3cm}

\answerTable{9120}{7060}{8880}{7560}{8560}

\vspace{0.5cm}

\noindent\makebox[1.5em][l]{\textbf{122.}}\parbox[t]{\dimexpr\textwidth-1.5em}{У залі для глядачів цирку встановлено 15 рядів крісел: у першому ряду 41 крісла, а в кожному наступному ряду кількість крісел на те саме число більше, ніж у попередньому. Визначте кількість крісел у \textit{12-му} ряду, якщо в останньому ряду 97 крісла. \nmtyear{2026}}
\vspace{0.3cm}

\answerTable{93}{89}{81}{77}{85}

\vspace{0.5cm}

\noindent\makebox[1.5em][l]{\textbf{123.}}\parbox[t]{\dimexpr\textwidth-1.5em}{У залі для глядачів цирку встановлено 14 рядів крісел: у першому ряду 26 крісла, а в кожному наступному ряду кількість крісел на те саме число більше, ніж у попередньому. Визначте кількість крісел у \textit{13-му} ряду, якщо в останньому ряду 52 крісла. \nmtyear{2026}}
\vspace{0.3cm}

\answerTable{52}{54}{48}{46}{50}

\vspace{0.5cm}

\noindent\makebox[1.5em][l]{\textbf{124.}}\parbox[t]{\dimexpr\textwidth-1.5em}{У залі для глядачів цирку встановлено 21 рядів крісел: у першому ряду 45 крісла, а в кожному наступному ряду кількість крісел на те саме число більше, ніж у попередньому. Визначте кількість крісел у \textit{12-му} ряду, якщо в останньому ряду 105 крісла. \nmtyear{2026}}
\vspace{0.3cm}

\answerTable{72}{84}{81}{75}{78}

\vspace{0.5cm}

\noindent\makebox[1.5em][l]{\textbf{125.}}\parbox[t]{\dimexpr\textwidth-1.5em}{За умовами договору позичальник повинен повернути кредит протягом 12 місяців. Першого місяця він має повернути 1500 \textit{грн}, а кожного наступного місяця — на 50 \textit{грн} менше, ніж попереднього. Визначте загальну суму (у \textit{грн}), яку повинен позичальник повернути протягом 12 місяців. \nmtyear{2026}}
\vspace{0.3cm}

\answerTable{18000}{15700}{18600}{14200}{14700}

\vspace{0.5cm}

\noindent\makebox[1.5em][l]{\textbf{126.}}\parbox[t]{\dimexpr\textwidth-1.5em}{За умовами договору позичальник повинен повернути кредит протягом 24 місяців. Першого місяця він має повернути 1240 \textit{грн}, а кожного наступного місяця — на 10 \textit{грн} менше, ніж попереднього. Визначте загальну суму (у \textit{грн}), яку повинен позичальник повернути протягом 24 місяців. \nmtyear{2026}}
\vspace{0.3cm}

\answerTable{29760}{28000}{30000}{27000}{26500}

\vspace{0.5cm}

\noindent\makebox[1.5em][l]{\textbf{127.}}\parbox[t]{\dimexpr\textwidth-1.5em}{Студент вивчав мову за методикою: у перший день він запам'ятав 20 слів, а кожного наступного дня — на 1 слів більше, ніж попереднього. Скільки всього слів запам'ятав студент за 10 днів? \nmtyear{2026}}
\vspace{0.3cm}

\answerTable{245}{265}{200}{225}{290}

\vspace{0.5cm}

\noindent\makebox[1.5em][l]{\textbf{128.}}\parbox[t]{\dimexpr\textwidth-1.5em}{На рисунку зображено поперечний переріз стосу колод. У нижньому ряду стосу 6 колод, а у верхньому — 1. Визначте загальну кількість колод.
            \begin{center}
            \begin{tikzpicture}[scale=0.5]
                \newcommand{\woodLog}[3]{
                    \begin{scope}[shift={(#1,#2)}]
                        \draw[fill=woodinner, draw=black, thick] (0,0) circle (0.5);
                        \draw[woodouter!80, thin] (0,0) circle (0.35);
                        \draw[woodouter!80, thin] (0,0) circle (0.2);
                        \begin{scope}[rotate=#3]
                            \fill[woodouter] (0,0) -- (0.4, 0.05) -- (0.5, 0.1) -- (0.5, -0.1) -- (0.4, -0.05) -- cycle;
                        \end{scope}
                    \end{scope}
                }
                \def\rows{6} 
                \foreach \row in {1,...,\rows} {
                    \foreach \col in {1,...,\row} {
                        \pgfmathsetmacro{\x}{(\col-1) - (\row-1)*0.5}
                        \pgfmathsetmacro{\y}{-(\row-1)*0.866}
                        \pgfmathsetmacro{\angle}{mod(\col*70 + \row*50, 360)}
                        \woodLog{\x}{\y}{\angle}
                    }
                }
            \end{tikzpicture}
            \end{center}
             \nmtyear{2026}}
\vspace{0.3cm}

\answerTable{21}{31}{11}{15}{27}

\vspace{0.5cm}

\noindent\makebox[1.5em][l]{\textbf{129.}}\parbox[t]{\dimexpr\textwidth-1.5em}{У залі для глядачів цирку встановлено 13 рядів крісел: у першому ряду 58 крісла, а в кожному наступному ряду кількість крісел на те саме число більше, ніж у попередньому. Визначте кількість крісел у \textit{11-му} ряду, якщо в останньому ряду 82 крісла. \nmtyear{2026}}
\vspace{0.3cm}

\answerTable{78}{74}{80}{76}{82}

\vspace{0.5cm}

\noindent\makebox[1.5em][l]{\textbf{130.}}\parbox[t]{\dimexpr\textwidth-1.5em}{За умовами договору позичальник повинен повернути кредит протягом 10 місяців. Першого місяця він має повернути 400 \textit{грн}, а кожного наступного місяця — на 20 \textit{грн} менше, ніж попереднього. Визначте загальну суму (у \textit{грн}), яку повинен позичальник повернути протягом 10 місяців. \nmtyear{2026}}
\vspace{0.3cm}

\answerTable{4200}{4000}{2600}{3100}{4100}

\vspace{0.5cm}

\noindent\makebox[1.5em][l]{\textbf{131.}}\parbox[t]{\dimexpr\textwidth-1.5em}{Студент вивчав мову за методикою: у перший день він запам'ятав 8 слів, а кожного наступного дня — на 5 слів більше, ніж попереднього. Скільки всього слів запам'ятав студент за 11 днів? \nmtyear{2026}}
\vspace{0.3cm}

\answerTable{638}{88}{363}{383}{343}

\vspace{0.5cm}

\noindent\makebox[1.5em][l]{\textbf{132.}}\parbox[t]{\dimexpr\textwidth-1.5em}{На рисунку зображено поперечний переріз стосу колод. У нижньому ряду стосу 7 колод, а у верхньому — 1. Визначте загальну кількість колод.
            \begin{center}
            \begin{tikzpicture}[scale=0.5]
                \newcommand{\woodLog}[3]{
                    \begin{scope}[shift={(#1,#2)}]
                        \draw[fill=woodinner, draw=black, thick] (0,0) circle (0.5);
                        \draw[woodouter!80, thin] (0,0) circle (0.35);
                        \draw[woodouter!80, thin] (0,0) circle (0.2);
                        \begin{scope}[rotate=#3]
                            \fill[woodouter] (0,0) -- (0.4, 0.05) -- (0.5, 0.1) -- (0.5, -0.1) -- (0.4, -0.05) -- cycle;
                        \end{scope}
                    \end{scope}
                }
                \def\rows{7} 
                \foreach \row in {1,...,\rows} {
                    \foreach \col in {1,...,\row} {
                        \pgfmathsetmacro{\x}{(\col-1) - (\row-1)*0.5}
                        \pgfmathsetmacro{\y}{-(\row-1)*0.866}
                        \pgfmathsetmacro{\angle}{mod(\col*70 + \row*50, 360)}
                        \woodLog{\x}{\y}{\angle}
                    }
                }
            \end{tikzpicture}
            \end{center}
             \nmtyear{2026}}
\vspace{0.3cm}

\answerTable{35}{28}{38}{21}{18}

\vspace{0.5cm}

\noindent\makebox[1.5em][l]{\textbf{133.}}\parbox[t]{\dimexpr\textwidth-1.5em}{За умовами договору позичальник повинен повернути кредит протягом 12 місяців. Першого місяця він має повернути 800 \textit{грн}, а кожного наступного місяця — на 50 \textit{грн} менше, ніж попереднього. Визначте загальну суму (у \textit{грн}), яку повинен позичальник повернути протягом 12 місяців. \nmtyear{2026}}
\vspace{0.3cm}

\answerTable{5800}{7300}{10200}{9600}{6300}

\vspace{0.5cm}

\noindent\makebox[1.5em][l]{\textbf{134.}}\parbox[t]{\dimexpr\textwidth-1.5em}{На рисунку зображено поперечний переріз стосу колод. У нижньому ряду стосу 7 колод, а у верхньому — 1. Визначте загальну кількість колод.
            \begin{center}
            \begin{tikzpicture}[scale=0.5]
                \newcommand{\woodLog}[3]{
                    \begin{scope}[shift={(#1,#2)}]
                        \draw[fill=woodinner, draw=black, thick] (0,0) circle (0.5);
                        \draw[woodouter!80, thin] (0,0) circle (0.35);
                        \draw[woodouter!80, thin] (0,0) circle (0.2);
                        \begin{scope}[rotate=#3]
                            \fill[woodouter] (0,0) -- (0.4, 0.05) -- (0.5, 0.1) -- (0.5, -0.1) -- (0.4, -0.05) -- cycle;
                        \end{scope}
                    \end{scope}
                }
                \def\rows{7} 
                \foreach \row in {1,...,\rows} {
                    \foreach \col in {1,...,\row} {
                        \pgfmathsetmacro{\x}{(\col-1) - (\row-1)*0.5}
                        \pgfmathsetmacro{\y}{-(\row-1)*0.866}
                        \pgfmathsetmacro{\angle}{mod(\col*70 + \row*50, 360)}
                        \woodLog{\x}{\y}{\angle}
                    }
                }
            \end{tikzpicture}
            \end{center}
             \nmtyear{2026}}
\vspace{0.3cm}

\answerTable{35}{21}{28}{18}{38}

\vspace{0.5cm}

\noindent\makebox[1.5em][l]{\textbf{135.}}\parbox[t]{\dimexpr\textwidth-1.5em}{На рисунку зображено поперечний переріз стосу колод. У нижньому ряду стосу 6 колод, а у верхньому — 1. Визначте загальну кількість колод.
            \begin{center}
            \begin{tikzpicture}[scale=0.5]
                \newcommand{\woodLog}[3]{
                    \begin{scope}[shift={(#1,#2)}]
                        \draw[fill=woodinner, draw=black, thick] (0,0) circle (0.5);
                        \draw[woodouter!80, thin] (0,0) circle (0.35);
                        \draw[woodouter!80, thin] (0,0) circle (0.2);
                        \begin{scope}[rotate=#3]
                            \fill[woodouter] (0,0) -- (0.4, 0.05) -- (0.5, 0.1) -- (0.5, -0.1) -- (0.4, -0.05) -- cycle;
                        \end{scope}
                    \end{scope}
                }
                \def\rows{6} 
                \foreach \row in {1,...,\rows} {
                    \foreach \col in {1,...,\row} {
                        \pgfmathsetmacro{\x}{(\col-1) - (\row-1)*0.5}
                        \pgfmathsetmacro{\y}{-(\row-1)*0.866}
                        \pgfmathsetmacro{\angle}{mod(\col*70 + \row*50, 360)}
                        \woodLog{\x}{\y}{\angle}
                    }
                }
            \end{tikzpicture}
            \end{center}
             \nmtyear{2026}}
\vspace{0.3cm}

\answerTable{21}{27}{31}{15}{11}

\vspace{0.5cm}

\noindent\makebox[1.5em][l]{\textbf{136.}}\parbox[t]{\dimexpr\textwidth-1.5em}{Студент вивчав мову за методикою: у перший день він запам'ятав 15 слів, а кожного наступного дня — на 2 слів більше, ніж попереднього. Скільки всього слів запам'ятав студент за 16 днів? \nmtyear{2026}}
\vspace{0.3cm}

\answerTable{720}{500}{460}{240}{480}

\vspace{0.5cm}

\noindent\makebox[1.5em][l]{\textbf{137.}}\parbox[t]{\dimexpr\textwidth-1.5em}{Студент вивчав мову за методикою: у перший день він запам'ятав 6 слів, а кожного наступного дня — на 3 слів більше, ніж попереднього. Скільки всього слів запам'ятав студент за 30 днів? \nmtyear{2026}}
\vspace{0.3cm}

\answerTable{180}{1465}{1505}{1485}{2790}

\vspace{0.5cm}

\noindent\makebox[1.5em][l]{\textbf{138.}}\parbox[t]{\dimexpr\textwidth-1.5em}{На рисунку зображено поперечний переріз стосу колод. У нижньому ряду стосу 3 колод, а у верхньому — 1. Визначте загальну кількість колод.
            \begin{center}
            \begin{tikzpicture}[scale=0.5]
                \newcommand{\woodLog}[3]{
                    \begin{scope}[shift={(#1,#2)}]
                        \draw[fill=woodinner, draw=black, thick] (0,0) circle (0.5);
                        \draw[woodouter!80, thin] (0,0) circle (0.35);
                        \draw[woodouter!80, thin] (0,0) circle (0.2);
                        \begin{scope}[rotate=#3]
                            \fill[woodouter] (0,0) -- (0.4, 0.05) -- (0.5, 0.1) -- (0.5, -0.1) -- (0.4, -0.05) -- cycle;
                        \end{scope}
                    \end{scope}
                }
                \def\rows{3} 
                \foreach \row in {1,...,\rows} {
                    \foreach \col in {1,...,\row} {
                        \pgfmathsetmacro{\x}{(\col-1) - (\row-1)*0.5}
                        \pgfmathsetmacro{\y}{-(\row-1)*0.866}
                        \pgfmathsetmacro{\angle}{mod(\col*70 + \row*50, 360)}
                        \woodLog{\x}{\y}{\angle}
                    }
                }
            \end{tikzpicture}
            \end{center}
             \nmtyear{2026}}
\vspace{0.3cm}

\answerTable{9}{16}{3}{-4}{6}

\vspace{0.5cm}

\noindent\makebox[1.5em][l]{\textbf{139.}}\parbox[t]{\dimexpr\textwidth-1.5em}{На рисунку зображено поперечний переріз стосу колод. У нижньому ряду стосу 8 колод, а у верхньому — 1. Визначте загальну кількість колод.
            \begin{center}
            \begin{tikzpicture}[scale=0.5]
                \newcommand{\woodLog}[3]{
                    \begin{scope}[shift={(#1,#2)}]
                        \draw[fill=woodinner, draw=black, thick] (0,0) circle (0.5);
                        \draw[woodouter!80, thin] (0,0) circle (0.35);
                        \draw[woodouter!80, thin] (0,0) circle (0.2);
                        \begin{scope}[rotate=#3]
                            \fill[woodouter] (0,0) -- (0.4, 0.05) -- (0.5, 0.1) -- (0.5, -0.1) -- (0.4, -0.05) -- cycle;
                        \end{scope}
                    \end{scope}
                }
                \def\rows{8} 
                \foreach \row in {1,...,\rows} {
                    \foreach \col in {1,...,\row} {
                        \pgfmathsetmacro{\x}{(\col-1) - (\row-1)*0.5}
                        \pgfmathsetmacro{\y}{-(\row-1)*0.866}
                        \pgfmathsetmacro{\angle}{mod(\col*70 + \row*50, 360)}
                        \woodLog{\x}{\y}{\angle}
                    }
                }
            \end{tikzpicture}
            \end{center}
             \nmtyear{2026}}
\vspace{0.3cm}

\answerTable{26}{28}{46}{44}{36}

\vspace{0.5cm}

\noindent\makebox[1.5em][l]{\textbf{140.}}\parbox[t]{\dimexpr\textwidth-1.5em}{За умовами договору позичальник повинен повернути кредит протягом 10 місяців. Першого місяця він має повернути 1800 \textit{грн}, а кожного наступного місяця — на 100 \textit{грн} менше, ніж попереднього. Визначте загальну суму (у \textit{грн}), яку повинен позичальник повернути протягом 10 місяців. \nmtyear{2026}}
\vspace{0.3cm}

\answerTable{13500}{18000}{13000}{14500}{19000}

\vspace{0.5cm}


\end{document}
