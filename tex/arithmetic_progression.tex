\documentclass[14pt]{extarticle}
\usepackage{fontspec}
\usepackage{polyglossia}
\setdefaultlanguage{ukrainian}

\defaultfontfeatures{Ligatures=TeX}
\setmainfont{Liberation Serif}
\setsansfont{Liberation Sans}
\setmonofont{Liberation Mono}

\usepackage[a4paper,margin=1.5cm,bottom=2cm,top=2cm]{geometry}
\usepackage{amsmath,amssymb}
\usepackage{enumitem}
\usepackage{tikz}
\usepackage{pgfplots}
\pgfplotsset{compat=1.16}

\usetikzlibrary{calc,patterns,angles,quotes,intersections,babel}
\usetikzlibrary{3d}
\definecolor{woodinner}{RGB}{222, 184, 135}
\definecolor{woodouter}{RGB}{139, 69, 19}
\usepackage{xcolor}
\usepackage{array}
\usepackage{fancyhdr}
\usepackage{multirow}

\definecolor{headerblue}{RGB}{0, 102, 204}
\definecolor{yearcolor}{RGB}{128, 0, 128}

\pagestyle{fancy}
\fancyhf{}
\renewcommand{\headrulewidth}{0pt}
\fancyfoot[C]{\thepage}

\setlength{\headheight}{15pt}
\setlength{\headsep}{10pt}
\setlength{\footskip}{25pt}

\widowpenalty=10000
\clubpenalty=10000

\newcommand{\answerTable}[5]{
\begin{center}
\begin{tabular}{|*{5}{>{\centering\arraybackslash}m{2.8cm}|}}
\hline
\rule[-0.3cm]{0pt}{0.8cm}\textbf{А} & \textbf{Б} & \textbf{В} & \textbf{Г} & \textbf{Д} \\
\hline
\rule[-0.4cm]{0pt}{1.0cm}#1 & \rule[-0.4cm]{0pt}{1.0cm}#2 & \rule[-0.4cm]{0pt}{1.0cm}#3 & \rule[-0.4cm]{0pt}{1.0cm}#4 & \rule[-0.4cm]{0pt}{1.0cm}#5 \\
\hline
\end{tabular}
\end{center}
}

\newcommand{\shortAnswer}{
\vspace{0.3cm}
\noindent\hspace{1cm}Відповідь: \framebox(18,18){}\framebox(18,18){}\framebox(18,18){}\framebox(18,18){}{,}\framebox(18,18){}\framebox(18,18){}\framebox(18,18){}
\vspace{0.5cm}
}

\newcommand{\nmtyear}[1]{\hfill{\small\color{yearcolor}(AI Gen)}}

\begin{document}

\begin{center}
{\Large\textbf{\color{headerblue}ЗГЕНЕРОВАНІ ЗАВДАННЯ (AI)}}
\end{center}

\begin{center}
{\large Тема: \textbf{Арифметична прогресія}}
\end{center}

\vspace{0.5cm}
% === Arithmetic Progression: Find d ===
\noindent\makebox[1.5em][l]{\textbf{1.}}\parbox[t]{\dimexpr\textwidth-1.5em}{В арифметичній прогресії $(a_n)$: $a_1 = 1$, $a_3 = 3$. Визначте різницю $d$ прогресії. \nmtyear{2026}}

\answerTable{$d = 1$}{$d = 2$}{$d = -1$}{68}{6}

\vspace{0.5cm}

\noindent\makebox[1.5em][l]{\textbf{2.}}\parbox[t]{\dimexpr\textwidth-1.5em}{В арифметичній прогресії $(a_n)$: $a_1 = 20$, $a_3 = 22$. Визначте різницю $d$ прогресії. \nmtyear{2026}}

\answerTable{$d = 2$}{24}{$d = 21$}{$d = 1$}{$d = -1$}

\vspace{0.5cm}

\noindent\makebox[1.5em][l]{\textbf{3.}}\parbox[t]{\dimexpr\textwidth-1.5em}{В арифметичній прогресії $(a_n)$: $a_1 = 3$, $a_3 = -17$. Визначте різницю $d$ прогресії. \nmtyear{2026}}

\answerTable{$d = -10$}{$d = -9$}{$d = 10$}{$d = -7$}{$d = -20$}

\vspace{0.5cm}

\noindent\makebox[1.5em][l]{\textbf{4.}}\parbox[t]{\dimexpr\textwidth-1.5em}{В арифметичній прогресії $(a_n)$: $a_1 = 18$, $a_3 = 20$. Визначте різницю $d$ прогресії. \nmtyear{2026}}

\answerTable{$d = 1$}{$d = -1$}{$d = 19$}{16}{$d = 2$}

\vspace{0.5cm}

\noindent\makebox[1.5em][l]{\textbf{5.}}\parbox[t]{\dimexpr\textwidth-1.5em}{В арифметичній прогресії $(a_n)$: $a_1 = 13$, $a_3 = 33$. Визначте різницю $d$ прогресії. \nmtyear{2026}}

\answerTable{$d = 11$}{$d = -10$}{$d = 10$}{$d = 20$}{$d = 23$}

\vspace{0.5cm}

% === Arithmetic Progression: Member Difference ===
\noindent\makebox[1.5em][l]{\textbf{6.}}\parbox[t]{\dimexpr\textwidth-1.5em}{В арифметичній прогресії $(a_n)$ відомо, що $a_{7} - a_{4} = 27$. Знайдіть значення виразу $a_{6} - a_{7}$. \nmtyear{2026}}

\answerTable{27}{-9}{9}{-18}{0}

\vspace{0.5cm}

\noindent\makebox[1.5em][l]{\textbf{7.}}\parbox[t]{\dimexpr\textwidth-1.5em}{В арифметичній прогресії $(a_n)$ відомо, що $a_{6} - a_{2} = 24$. Знайдіть значення виразу $a_{6} - a_{7}$. \nmtyear{2026}}

\answerTable{6}{0}{24}{-12}{-6}

\vspace{0.5cm}

\noindent\makebox[1.5em][l]{\textbf{8.}}\parbox[t]{\dimexpr\textwidth-1.5em}{В арифметичній прогресії $(a_n)$ відомо, що $a_{9} - a_{5} = 12$. Знайдіть значення виразу $a_{10} - a_{8}$. \nmtyear{2026}}

\answerTable{12}{3}{6}{-6}{9}

\vspace{0.5cm}

\noindent\makebox[1.5em][l]{\textbf{9.}}\parbox[t]{\dimexpr\textwidth-1.5em}{В арифметичній прогресії $(a_n)$ відомо, що $a_{7} - a_{2} = -5$. Знайдіть значення виразу $a_{5} - a_{1}$. \nmtyear{2026}}

\answerTable{-3}{-1}{-4}{4}{-5}

\vspace{0.5cm}

\noindent\makebox[1.5em][l]{\textbf{10.}}\parbox[t]{\dimexpr\textwidth-1.5em}{В арифметичній прогресії $(a_n)$ відомо, що $a_{10} - a_{4} = -42$. Знайдіть значення виразу $a_{12} - a_{11}$. \nmtyear{2026}}

\answerTable{7}{-7}{-14}{0}{-42}

\vspace{0.5cm}

% === Arithmetic Progression: Sum ===
\noindent\makebox[1.5em][l]{\textbf{11.}}\parbox[t]{\dimexpr\textwidth-1.5em}{Обчисліть суму перших 100-ти членів арифметичної прогресії $(a_n)$, якщо $a_1 + a_{100} = 15$. \nmtyear{2026}}

\answerTable{650}{1500}{750}{-750}{15}

\vspace{0.5cm}

\noindent\makebox[1.5em][l]{\textbf{12.}}\parbox[t]{\dimexpr\textwidth-1.5em}{Обчисліть суму перших 8-ти членів арифметичної прогресії $(a_n)$, якщо $a_1 + a_{8} = 12$. \nmtyear{2026}}

\answerTable{12}{48}{40}{56}{96}

\vspace{0.5cm}

\noindent\makebox[1.5em][l]{\textbf{13.}}\parbox[t]{\dimexpr\textwidth-1.5em}{Обчисліть суму перших 8-ти членів арифметичної прогресії $(a_n)$, якщо $a_1 + a_{8} = 40$. \nmtyear{2026}}

\answerTable{-160}{320}{40}{160}{168}

\vspace{0.5cm}

\noindent\makebox[1.5em][l]{\textbf{14.}}\parbox[t]{\dimexpr\textwidth-1.5em}{Обчисліть суму перших 10-ти членів арифметичної прогресії $(a_n)$, якщо $a_1 + a_{10} = 1$. \nmtyear{2026}}

\answerTable{15}{5}{10}{1}{-5}

\vspace{0.5cm}

\noindent\makebox[1.5em][l]{\textbf{15.}}\parbox[t]{\dimexpr\textwidth-1.5em}{Обчисліть суму перших 10-ти членів арифметичної прогресії $(a_n)$, якщо $a_1 + a_{10} = -41$. \nmtyear{2026}}

\answerTable{-195}{-205}{-41}{-410}{205}

\vspace{0.5cm}

% === Arithmetic Progression: Count Terms ===
\noindent\makebox[1.5em][l]{\textbf{16.}}\parbox[t]{\dimexpr\textwidth-1.5em}{В арифметичній прогресії $(a_n)$ перший член $a_1 = -12$, різниця $d = 1$. Скільки всього \textit{від'ємних} членів має ця прогресія? \nmtyear{2026}}

\answerTable{10}{11}{13}{9}{12}

\vspace{0.5cm}

\noindent\makebox[1.5em][l]{\textbf{17.}}\parbox[t]{\dimexpr\textwidth-1.5em}{В арифметичній прогресії $(a_n)$ перший член $a_1 = -20{,}2$, різниця $d = 2{,}5$. Скільки всього \textit{від'ємних} членів має ця прогресія? \nmtyear{2026}}

\answerTable{11}{10}{7}{8}{9}

\vspace{0.5cm}

\noindent\makebox[1.5em][l]{\textbf{18.}}\parbox[t]{\dimexpr\textwidth-1.5em}{В арифметичній прогресії $(a_n)$ перший член $a_1 = -10{,}5$, різниця $d = 2$. Скільки всього \textit{від'ємних} членів має ця прогресія? \nmtyear{2026}}

\answerTable{8}{7}{5}{4}{6}

\vspace{0.5cm}

\noindent\makebox[1.5em][l]{\textbf{19.}}\parbox[t]{\dimexpr\textwidth-1.5em}{В арифметичній прогресії $(a_n)$ перший член $a_1 = 17{,}5$, різниця $d = -1{,}5$. Скільки всього \textit{додатних} членів має ця прогресія? \nmtyear{2026}}

\answerTable{11}{13}{10}{12}{14}

\vspace{0.5cm}

\noindent\makebox[1.5em][l]{\textbf{20.}}\parbox[t]{\dimexpr\textwidth-1.5em}{В арифметичній прогресії $(a_n)$ перший член $a_1 = 7{,}1$, різниця $d = -0{,}5$. Скільки всього \textit{додатних} членів має ця прогресія? \nmtyear{2026}}

\answerTable{15}{14}{13}{16}{17}

\vspace{0.5cm}

% === Arithmetic Progression: Middle Term ===
\noindent\makebox[1.5em][l]{\textbf{21.}}\parbox[t]{\dimexpr\textwidth-1.5em}{Визначте 6-й член $a_{6}$ арифметичної прогресії $(a_n)$, у якої $a_{5} = 49$, $a_{7} = 55$. \nmtyear{2026}}

\answerTable{6}{3}{104}{52}{55}

\vspace{0.5cm}

\noindent\makebox[1.5em][l]{\textbf{22.}}\parbox[t]{\dimexpr\textwidth-1.5em}{Визначте 14-й член $a_{14}$ арифметичної прогресії $(a_n)$, у якої $a_{13} = 1$, $a_{15} = 5$. \nmtyear{2026}}

\answerTable{3}{5}{6}{2}{4}

\vspace{0.5cm}

\noindent\makebox[1.5em][l]{\textbf{23.}}\parbox[t]{\dimexpr\textwidth-1.5em}{Визначте 16-й член $a_{16}$ арифметичної прогресії $(a_n)$, у якої $a_{15} = 11$, $a_{17} = 14$. \nmtyear{2026}}

\answerTable{1{,}5}{14}{11}{3}{12{,}5}

\vspace{0.5cm}

\noindent\makebox[1.5em][l]{\textbf{24.}}\parbox[t]{\dimexpr\textwidth-1.5em}{Визначте 4-й член $a_{4}$ арифметичної прогресії $(a_n)$, у якої $a_{3} = -6$, $a_{5} = -10$. \nmtyear{2026}}

\answerTable{-16}{-6}{-8}{-10}{4}

\vspace{0.5cm}

\noindent\makebox[1.5em][l]{\textbf{25.}}\parbox[t]{\dimexpr\textwidth-1.5em}{Визначте 5-й член $a_{5}$ арифметичної прогресії $(a_n)$, у якої $a_{4} = 12$, $a_{6} = 15$. \nmtyear{2026}}

\answerTable{27}{15}{1{,}5}{3}{13{,}5}

\vspace{0.5cm}

% === Arithmetic Progression: Formula Search ===
\noindent\makebox[1.5em][l]{\textbf{26.}}\parbox[t]{\dimexpr\textwidth-1.5em}{Арифметичну прогресію $(a_n)$ задано формулою $n$-го члена $a_n = 18  -3n$. Визначте номер члена, значення якого дорівнює $-87$. \nmtyear{2026}}

\answerTable{25}{34}{36}{35}{45}

\vspace{0.5cm}

\noindent\makebox[1.5em][l]{\textbf{27.}}\parbox[t]{\dimexpr\textwidth-1.5em}{Арифметичну прогресію $(a_n)$ задано формулою $n$-го члена $a_n = 14  -3n$. Визначте номер члена, значення якого дорівнює $-49$. \nmtyear{2026}}

\answerTable{31}{11}{21}{22}{16{,}33}

\vspace{0.5cm}

\noindent\makebox[1.5em][l]{\textbf{28.}}\parbox[t]{\dimexpr\textwidth-1.5em}{Арифметичну прогресію $(a_n)$ задано формулою $n$-го члена $a_n = 44 + 3n$. Визначте номер члена, значення якого дорівнює $179$. \nmtyear{2026}}

\answerTable{55}{59{,}67}{35}{46}{45}

\vspace{0.5cm}

\noindent\makebox[1.5em][l]{\textbf{29.}}\parbox[t]{\dimexpr\textwidth-1.5em}{Арифметичну прогресію $(a_n)$ задано формулою $n$-го члена $a_n = 17 + 3n$. Визначте номер члена, значення якого дорівнює $50$. \nmtyear{2026}}

\answerTable{16{,}67}{1}{10}{21}{11}

\vspace{0.5cm}

\noindent\makebox[1.5em][l]{\textbf{30.}}\parbox[t]{\dimexpr\textwidth-1.5em}{Арифметичну прогресію $(a_n)$ задано формулою $n$-го члена $a_n = 16 + 3n$. Визначте номер члена, значення якого дорівнює $88$. \nmtyear{2026}}

\answerTable{23}{25}{24}{14}{29{,}33}

\vspace{0.5cm}

% === Arithmetic Progression: Word Problem ===
\noindent\makebox[1.5em][l]{\textbf{31.}}\parbox[t]{\dimexpr\textwidth-1.5em}{У залі для глядачів цирку встановлено 29 рядів крісел: у першому ряду 18 крісла, а в кожному наступному ряду кількість крісел на те саме число більше, ніж у попередньому. Визначте кількість крісел у \textit{6-му} ряду, якщо в останньому ряду 46 крісла. \nmtyear{2026}}

\answerTable{24}{21}{22}{25}{23}

\vspace{0.5cm}

\noindent\makebox[1.5em][l]{\textbf{32.}}\parbox[t]{\dimexpr\textwidth-1.5em}{За умовами договору позичальник повинен повернути кредит протягом 24 місяців. Першого місяця він має повернути 1280 \textit{грн}, а кожного наступного місяця — на 20 \textit{грн} менше, ніж попереднього. Визначте загальну суму (у \textit{грн}), яку повинен позичальник повернути протягом 24 місяців. \nmtyear{2026}}

\answerTable{26200}{31200}{24700}{30720}{25200}

\vspace{0.5cm}

\noindent\makebox[1.5em][l]{\textbf{33.}}\parbox[t]{\dimexpr\textwidth-1.5em}{На рисунку зображено поперечний переріз стосу колод. У нижньому ряду стосу 3 колод, а у верхньому — 1. Визначте загальну кількість колод.
            \begin{center}
            \begin{tikzpicture}[scale=0.5]
                \newcommand{\woodLog}[3]{
                    \begin{scope}[shift={(#1,#2)}]
                        \draw[fill=woodinner, draw=black, thick] (0,0) circle (0.5);
                        \draw[woodouter!80, thin] (0,0) circle (0.35);
                        \draw[woodouter!80, thin] (0,0) circle (0.2);
                        \begin{scope}[rotate=#3]
                            \fill[woodouter] (0,0) -- (0.4, 0.05) -- (0.5, 0.1) -- (0.5, -0.1) -- (0.4, -0.05) -- cycle;
                        \end{scope}
                    \end{scope}
                }
                \def\rows{3} 
                \foreach \row in {1,...,\rows} {
                    \foreach \col in {1,...,\row} {
                        \pgfmathsetmacro{\x}{(\col-1) - (\row-1)*0.5}
                        \pgfmathsetmacro{\y}{-(\row-1)*0.866}
                        \pgfmathsetmacro{\angle}{mod(\col*70 + \row*50, 360)}
                        \woodLog{\x}{\y}{\angle}
                    }
                }
            \end{tikzpicture}
            \end{center}
             \nmtyear{2026}}

\answerTable{9}{3}{6}{-4}{16}

\vspace{0.5cm}

\noindent\makebox[1.5em][l]{\textbf{34.}}\parbox[t]{\dimexpr\textwidth-1.5em}{За умовами договору позичальник повинен повернути кредит протягом 24 місяців. Першого місяця він має повернути 440 \textit{грн}, а кожного наступного місяця — на 10 \textit{грн} менше, ніж попереднього. Визначте загальну суму (у \textit{грн}), яку повинен позичальник повернути протягом 24 місяців. \nmtyear{2026}}

\answerTable{10800}{8800}{7300}{7800}{10560}

\vspace{0.5cm}

\noindent\makebox[1.5em][l]{\textbf{35.}}\parbox[t]{\dimexpr\textwidth-1.5em}{За умовами договору позичальник повинен повернути кредит протягом 12 місяців. Першого місяця він має повернути 940 \textit{грн}, а кожного наступного місяця — на 20 \textit{грн} менше, ніж попереднього. Визначте загальну суму (у \textit{грн}), яку повинен позичальник повернути протягом 12 місяців. \nmtyear{2026}}

\answerTable{9460}{10960}{11520}{11280}{9960}

\vspace{0.5cm}


\end{document}
