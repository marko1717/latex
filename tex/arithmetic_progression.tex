\documentclass[14pt]{extarticle}
\usepackage{fontspec}
\usepackage{polyglossia}
\setdefaultlanguage{ukrainian}

\defaultfontfeatures{Ligatures=TeX}
\setmainfont{Liberation Serif}
\setsansfont{Liberation Sans}
\setmonofont{Liberation Mono}

\usepackage[a4paper,margin=1.5cm,bottom=2cm,top=2cm]{geometry}
\usepackage{amsmath,amssymb}
\usepackage{enumitem}
\usepackage{tikz}
\usepackage{pgfplots}
\pgfplotsset{compat=1.18}

\usetikzlibrary{calc,patterns,angles,quotes,intersections,babel}
\usetikzlibrary{3d}

\usepackage{xcolor}
\usepackage{array}
\usepackage{fancyhdr}
\usepackage{multirow}

% Кольори
\definecolor{headerblue}{RGB}{0, 102, 204}
\definecolor{yearcolor}{RGB}{128, 0, 128}

\pagestyle{fancy}
\fancyhf{}
\renewcommand{\headrulewidth}{0pt}
\fancyfoot[C]{\thepage}

\setlength{\headheight}{15pt}
\setlength{\headsep}{10pt}
\setlength{\footskip}{25pt}

\widowpenalty=10000
\clubpenalty=10000

% === КОМАНДИ ===

% Таблиця відповідей для відповідностей
\newcommand{\answerGrid}{
    \begingroup
    \renewcommand{\arraystretch}{1.3} 
    \setlength{\tabcolsep}{7pt} 
    \begin{tabular}{r|c|c|c|c|c|}
         \multicolumn{1}{c}{} & \multicolumn{1}{c}{\textbf{А}} & \multicolumn{1}{c}{\textbf{Б}} & \multicolumn{1}{c}{\textbf{В}} & \multicolumn{1}{c}{\textbf{Г}} & \multicolumn{1}{c}{\textbf{Д}} \\ \cline{2-6}
         \textbf{1} & & & & & \\ \cline{2-6}
         \textbf{2} & & & & & \\ \cline{2-6}
         \textbf{3} & & & & & \\ \cline{2-6}
    \end{tabular}
    \endgroup
}

% Макет для завдань на відповідність
\newcommand{\matchingLayout}[3]{
    \noindent
    \begin{minipage}[t]{0.40\textwidth}
        #1
    \end{minipage}%
    \hfill
    \begin{minipage}[t]{0.28\textwidth}
        #2
    \end{minipage}%
    \hfill
    \begin{minipage}[t]{0.30\textwidth}
        \vspace{0pt}
        \begin{flushright}
        #3
        \end{flushright}
    \end{minipage}
}

% Стандартна таблиця відповідей
\newcommand{\answerTable}[5]{
\begin{center}
\begin{tabular}{|*{5}{>{\centering\arraybackslash}m{2.8cm}|}}
\hline
\rule[-0.3cm]{0pt}{0.8cm}\textbf{А} & \textbf{Б} & \textbf{В} & \textbf{Г} & \textbf{Д} \\
\hline
\rule[-0.4cm]{0pt}{1.0cm}#1 & \rule[-0.4cm]{0pt}{1.0cm}#2 & \rule[-0.4cm]{0pt}{1.0cm}#3 & \rule[-0.4cm]{0pt}{1.0cm}#4 & \rule[-0.4cm]{0pt}{1.0cm}#5 \\
\hline
\end{tabular}
\end{center}
}

% Таблиця для відповідей із дробами
\newcommand{\answerTableTall}[5]{
\begin{center}
\begin{tabular}{|*{5}{>{\centering\arraybackslash}m{2.8cm}|}}
\hline
\rule[-0.3cm]{0pt}{0.8cm}\textbf{А} & \textbf{Б} & \textbf{В} & \textbf{Г} & \textbf{Д} \\
\hline
\rule[-0.9cm]{0pt}{2.0cm}#1 & 
\rule[-0.9cm]{0pt}{2.0cm}#2 & 
\rule[-0.9cm]{0pt}{2.0cm}#3 & 
\rule[-0.9cm]{0pt}{2.0cm}#4 & 
\rule[-0.9cm]{0pt}{2.0cm}#5 \\
\hline
\end{tabular}
\end{center}
}

\newcommand{\nmtyear}[1]{\hfill{\small\color{yearcolor}(AI Gen)}}

\begin{document}

\vspace{1cm}

\begin{center}
{\Large\textbf{\color{headerblue}ЗГЕНЕРОВАНІ ЗАВДАННЯ (AI)}}
\end{center}

\begin{center}
{\large Тема: \textbf{Арифметична прогресія}}
\end{center}

\vspace{0.5cm}
% === Arithmetic Progression: Find d ===
\noindent\makebox[1.5em][l]{\textbf{1.}}\parbox[t]{\dimexpr\textwidth-1.5em}{В арифметичній прогресії $(a_n)$: $a_1 = 2$, $a_3 = -8$. Визначте різницю $d$ прогресії. \nmtyear{2026}}
\vspace{0.3cm}

\answerTable{$d = -5$}{$d = -10$}{$d = -4$}{$d = 5$}{$d = -3$}

\vspace{0.5cm}

\noindent\makebox[1.5em][l]{\textbf{2.}}\parbox[t]{\dimexpr\textwidth-1.5em}{В арифметичній прогресії $(a_n)$: $a_1 = -8$, $a_3 = -14$. Визначте різницю $d$ прогресії. \nmtyear{2026}}
\vspace{0.3cm}

\answerTable{$d = 3$}{$d = -11$}{$d = -6$}{$d = -2$}{$d = -3$}

\vspace{0.5cm}

\noindent\makebox[1.5em][l]{\textbf{3.}}\parbox[t]{\dimexpr\textwidth-1.5em}{В арифметичній прогресії $(a_n)$: $a_1 = 7$, $a_3 = -7$. Визначте різницю $d$ прогресії. \nmtyear{2026}}
\vspace{0.3cm}

\answerTable{$d = -7$}{$d = -14$}{$d = 0$}{$d = -6$}{$d = 7$}

\vspace{0.5cm}

\noindent\makebox[1.5em][l]{\textbf{4.}}\parbox[t]{\dimexpr\textwidth-1.5em}{В арифметичній прогресії $(a_n)$: $a_1 = 11$, $a_3 = 19$. Визначте різницю $d$ прогресії. \nmtyear{2026}}
\vspace{0.3cm}

\answerTable{$d = -4$}{$d = 8$}{$d = 4$}{$d = 15$}{$d = 5$}

\vspace{0.5cm}

\noindent\makebox[1.5em][l]{\textbf{5.}}\parbox[t]{\dimexpr\textwidth-1.5em}{В арифметичній прогресії $(a_n)$: $a_1 = -9$, $a_3 = -25$. Визначте різницю $d$ прогресії. \nmtyear{2026}}
\vspace{0.3cm}

\answerTable{$d = 8$}{$d = -17$}{$d = -7$}{$d = -8$}{$d = -16$}

\vspace{0.5cm}

\noindent\makebox[1.5em][l]{\textbf{6.}}\parbox[t]{\dimexpr\textwidth-1.5em}{В арифметичній прогресії $(a_n)$: $a_1 = 18$, $a_3 = 2$. Визначте різницю $d$ прогресії. \nmtyear{2026}}
\vspace{0.3cm}

\answerTable{$d = 10$}{$d = -8$}{$d = -7$}{$d = 8$}{$d = -16$}

\vspace{0.5cm}

\noindent\makebox[1.5em][l]{\textbf{7.}}\parbox[t]{\dimexpr\textwidth-1.5em}{В арифметичній прогресії $(a_n)$: $a_1 = 8$, $a_3 = 2$. Визначте різницю $d$ прогресії. \nmtyear{2026}}
\vspace{0.3cm}

\answerTable{$d = -3$}{$d = -2$}{$d = 3$}{$d = 5$}{$d = -6$}

\vspace{0.5cm}

\noindent\makebox[1.5em][l]{\textbf{8.}}\parbox[t]{\dimexpr\textwidth-1.5em}{В арифметичній прогресії $(a_n)$: $a_1 = 6$, $a_3 = 22$. Визначте різницю $d$ прогресії. \nmtyear{2026}}
\vspace{0.3cm}

\answerTable{$d = 8$}{$d = 16$}{$d = -8$}{$d = 14$}{$d = 9$}

\vspace{0.5cm}

\noindent\makebox[1.5em][l]{\textbf{9.}}\parbox[t]{\dimexpr\textwidth-1.5em}{В арифметичній прогресії $(a_n)$: $a_1 = 17$, $a_3 = 5$. Визначте різницю $d$ прогресії. \nmtyear{2026}}
\vspace{0.3cm}

\answerTable{$d = 6$}{$d = 11$}{$d = -5$}{$d = -6$}{$d = -12$}

\vspace{0.5cm}

\noindent\makebox[1.5em][l]{\textbf{10.}}\parbox[t]{\dimexpr\textwidth-1.5em}{В арифметичній прогресії $(a_n)$: $a_1 = 20$, $a_3 = 35$. Визначте різницю $d$ прогресії. \nmtyear{2026}}
\vspace{0.3cm}

\answerTable{$d = 27{,}5$}{$d = 8{,}5$}{$d = 15$}{$d = -7{,}5$}{$d = 7{,}5$}

\vspace{0.5cm}

\noindent\makebox[1.5em][l]{\textbf{11.}}\parbox[t]{\dimexpr\textwidth-1.5em}{В арифметичній прогресії $(a_n)$: $a_1 = -9$, $a_3 = 5$. Визначте різницю $d$ прогресії. \nmtyear{2026}}
\vspace{0.3cm}

\answerTable{$d = 14$}{$d = 8$}{$d = 7$}{$d = -2$}{$d = -7$}

\vspace{0.5cm}

\noindent\makebox[1.5em][l]{\textbf{12.}}\parbox[t]{\dimexpr\textwidth-1.5em}{В арифметичній прогресії $(a_n)$: $a_1 = 10$, $a_3 = 13$. Визначте різницю $d$ прогресії. \nmtyear{2026}}
\vspace{0.3cm}

\answerTable{$d = 2{,}5$}{$d = -1{,}5$}{$d = 11{,}5$}{$d = 1{,}5$}{$d = 3$}

\vspace{0.5cm}

\noindent\makebox[1.5em][l]{\textbf{13.}}\parbox[t]{\dimexpr\textwidth-1.5em}{В арифметичній прогресії $(a_n)$: $a_1 = 9$, $a_3 = 20$. Визначте різницю $d$ прогресії. \nmtyear{2026}}
\vspace{0.3cm}

\answerTable{$d = 5{,}5$}{$d = -5{,}5$}{$d = 14{,}5$}{$d = 6{,}5$}{$d = 11$}

\vspace{0.5cm}

\noindent\makebox[1.5em][l]{\textbf{14.}}\parbox[t]{\dimexpr\textwidth-1.5em}{В арифметичній прогресії $(a_n)$: $a_1 = -8$, $a_3 = -7$. Визначте різницю $d$ прогресії. \nmtyear{2026}}
\vspace{0.3cm}

\answerTable{$d = 1$}{$d = -0{,}5$}{$d = -7{,}5$}{$d = 1{,}5$}{$d = 0{,}5$}

\vspace{0.5cm}

\noindent\makebox[1.5em][l]{\textbf{15.}}\parbox[t]{\dimexpr\textwidth-1.5em}{В арифметичній прогресії $(a_n)$: $a_1 = 5$, $a_3 = 13$. Визначте різницю $d$ прогресії. \nmtyear{2026}}
\vspace{0.3cm}

\answerTable{$d = 8$}{$d = 9$}{$d = 4$}{$d = 5$}{$d = -4$}

\vspace{0.5cm}

\noindent\makebox[1.5em][l]{\textbf{16.}}\parbox[t]{\dimexpr\textwidth-1.5em}{В арифметичній прогресії $(a_n)$: $a_1 = 17$, $a_3 = 28$. Визначте різницю $d$ прогресії. \nmtyear{2026}}
\vspace{0.3cm}

\answerTable{$d = 11$}{$d = 5{,}5$}{$d = 6{,}5$}{$d = 22{,}5$}{$d = -5{,}5$}

\vspace{0.5cm}

\noindent\makebox[1.5em][l]{\textbf{17.}}\parbox[t]{\dimexpr\textwidth-1.5em}{В арифметичній прогресії $(a_n)$: $a_1 = 18$, $a_3 = 30$. Визначте різницю $d$ прогресії. \nmtyear{2026}}
\vspace{0.3cm}

\answerTable{$d = -6$}{$d = 7$}{$d = 24$}{$d = 12$}{$d = 6$}

\vspace{0.5cm}

\noindent\makebox[1.5em][l]{\textbf{18.}}\parbox[t]{\dimexpr\textwidth-1.5em}{В арифметичній прогресії $(a_n)$: $a_1 = 8$, $a_3 = 22$. Визначте різницю $d$ прогресії. \nmtyear{2026}}
\vspace{0.3cm}

\answerTable{$d = -7$}{$d = 15$}{$d = 7$}{$d = 8$}{$d = 14$}

\vspace{0.5cm}

\noindent\makebox[1.5em][l]{\textbf{19.}}\parbox[t]{\dimexpr\textwidth-1.5em}{В арифметичній прогресії $(a_n)$: $a_1 = 16$, $a_3 = 8$. Визначте різницю $d$ прогресії. \nmtyear{2026}}
\vspace{0.3cm}

\answerTable{$d = 4$}{$d = -8$}{$d = 12$}{$d = -3$}{$d = -4$}

\vspace{0.5cm}

\noindent\makebox[1.5em][l]{\textbf{20.}}\parbox[t]{\dimexpr\textwidth-1.5em}{В арифметичній прогресії $(a_n)$: $a_1 = -1$, $a_3 = 3$. Визначте різницю $d$ прогресії. \nmtyear{2026}}
\vspace{0.3cm}

\answerTable{$d = 4$}{$d = 1$}{$d = -2$}{$d = 2$}{$d = 3$}

\vspace{0.5cm}

% === Arithmetic Progression: Member Difference ===
\noindent\makebox[1.5em][l]{\textbf{21.}}\parbox[t]{\dimexpr\textwidth-1.5em}{В арифметичній прогресії $(a_n)$ відомо, що $a_{4} - a_{2} = 18$. Знайдіть значення виразу $a_{4} - a_{8}$. \nmtyear{2026}}
\vspace{0.3cm}

\answerTable{-27}{-45}{-36}{18}{36}

\vspace{0.5cm}

\noindent\makebox[1.5em][l]{\textbf{22.}}\parbox[t]{\dimexpr\textwidth-1.5em}{В арифметичній прогресії $(a_n)$ відомо, що $a_{7} - a_{2} = -45$. Знайдіть значення виразу $a_{5} - a_{3}$. \nmtyear{2026}}
\vspace{0.3cm}

\answerTable{-27}{-18}{18}{-9}{-45}

\vspace{0.5cm}

\noindent\makebox[1.5em][l]{\textbf{23.}}\parbox[t]{\dimexpr\textwidth-1.5em}{В арифметичній прогресії $(a_n)$ відомо, що $a_{5} - a_{1} = -8$. Знайдіть значення виразу $a_{3} - a_{7}$. \nmtyear{2026}}
\vspace{0.3cm}

\answerTable{6}{-8}{10}{8}{3}

\vspace{0.5cm}

\noindent\makebox[1.5em][l]{\textbf{24.}}\parbox[t]{\dimexpr\textwidth-1.5em}{В арифметичній прогресії $(a_n)$ відомо, що $a_{5} - a_{3} = -12$. Знайдіть значення виразу $a_{5} - a_{4}$. \nmtyear{2026}}
\vspace{0.3cm}

\answerTable{6}{-3}{-12}{-6}{0}

\vspace{0.5cm}

\noindent\makebox[1.5em][l]{\textbf{25.}}\parbox[t]{\dimexpr\textwidth-1.5em}{В арифметичній прогресії $(a_n)$ відомо, що $a_{6} - a_{3} = -6$. Знайдіть значення виразу $a_{3} - a_{5}$. \nmtyear{2026}}
\vspace{0.3cm}

\answerTable{6}{-4}{-6}{4}{2}

\vspace{0.5cm}

\noindent\makebox[1.5em][l]{\textbf{26.}}\parbox[t]{\dimexpr\textwidth-1.5em}{В арифметичній прогресії $(a_n)$ відомо, що $a_{8} - a_{4} = -28$. Знайдіть значення виразу $a_{8} - a_{9}$. \nmtyear{2026}}
\vspace{0.3cm}

\answerTable{-28}{7}{-7}{0}{14}

\vspace{0.5cm}

\noindent\makebox[1.5em][l]{\textbf{27.}}\parbox[t]{\dimexpr\textwidth-1.5em}{В арифметичній прогресії $(a_n)$ відомо, що $a_{7} - a_{5} = 12$. Знайдіть значення виразу $a_{10} - a_{7}$. \nmtyear{2026}}
\vspace{0.3cm}

\answerTable{18}{-18}{12}{24}{19}

\vspace{0.5cm}

\noindent\makebox[1.5em][l]{\textbf{28.}}\parbox[t]{\dimexpr\textwidth-1.5em}{В арифметичній прогресії $(a_n)$ відомо, що $a_{7} - a_{4} = 12$. Знайдіть значення виразу $a_{7} - a_{4}$. \nmtyear{2026}}
\vspace{0.3cm}

\answerTable{16}{-12}{12}{4}{8}

\vspace{0.5cm}

\noindent\makebox[1.5em][l]{\textbf{29.}}\parbox[t]{\dimexpr\textwidth-1.5em}{В арифметичній прогресії $(a_n)$ відомо, що $a_{6} - a_{4} = -18$. Знайдіть значення виразу $a_{9} - a_{7}$. \nmtyear{2026}}
\vspace{0.3cm}

\answerTable{-19}{-27}{18}{-18}{-9}

\vspace{0.5cm}

\noindent\makebox[1.5em][l]{\textbf{30.}}\parbox[t]{\dimexpr\textwidth-1.5em}{В арифметичній прогресії $(a_n)$ відомо, що $a_{10} - a_{4} = 54$. Знайдіть значення виразу $a_{3} - a_{2}$. \nmtyear{2026}}
\vspace{0.3cm}

\answerTable{0}{9}{18}{-9}{54}

\vspace{0.5cm}

\noindent\makebox[1.5em][l]{\textbf{31.}}\parbox[t]{\dimexpr\textwidth-1.5em}{В арифметичній прогресії $(a_n)$ відомо, що $a_{7} - a_{5} = -16$. Знайдіть значення виразу $a_{4} - a_{5}$. \nmtyear{2026}}
\vspace{0.3cm}

\answerTable{-8}{0}{-16}{8}{16}

\vspace{0.5cm}

\noindent\makebox[1.5em][l]{\textbf{32.}}\parbox[t]{\dimexpr\textwidth-1.5em}{В арифметичній прогресії $(a_n)$ відомо, що $a_{4} - a_{2} = 12$. Знайдіть значення виразу $a_{5} - a_{6}$. \nmtyear{2026}}
\vspace{0.3cm}

\answerTable{-6}{6}{-12}{12}{0}

\vspace{0.5cm}

\noindent\makebox[1.5em][l]{\textbf{33.}}\parbox[t]{\dimexpr\textwidth-1.5em}{В арифметичній прогресії $(a_n)$ відомо, що $a_{6} - a_{2} = -20$. Знайдіть значення виразу $a_{5} - a_{6}$. \nmtyear{2026}}
\vspace{0.3cm}

\answerTable{-20}{0}{-5}{10}{5}

\vspace{0.5cm}

\noindent\makebox[1.5em][l]{\textbf{34.}}\parbox[t]{\dimexpr\textwidth-1.5em}{В арифметичній прогресії $(a_n)$ відомо, що $a_{3} - a_{1} = -10$. Знайдіть значення виразу $a_{3} - a_{2}$. \nmtyear{2026}}
\vspace{0.3cm}

\answerTable{-10}{0}{-5}{-8}{5}

\vspace{0.5cm}

\noindent\makebox[1.5em][l]{\textbf{35.}}\parbox[t]{\dimexpr\textwidth-1.5em}{В арифметичній прогресії $(a_n)$ відомо, що $a_{10} - a_{4} = -30$. Знайдіть значення виразу $a_{15} - a_{11}$. \nmtyear{2026}}
\vspace{0.3cm}

\answerTable{-25}{-20}{-15}{-30}{20}

\vspace{0.5cm}

\noindent\makebox[1.5em][l]{\textbf{36.}}\parbox[t]{\dimexpr\textwidth-1.5em}{В арифметичній прогресії $(a_n)$ відомо, що $a_{8} - a_{3} = -20$. Знайдіть значення виразу $a_{8} - a_{5}$. \nmtyear{2026}}
\vspace{0.3cm}

\answerTable{-8}{-12}{-16}{-20}{12}

\vspace{0.5cm}

\noindent\makebox[1.5em][l]{\textbf{37.}}\parbox[t]{\dimexpr\textwidth-1.5em}{В арифметичній прогресії $(a_n)$ відомо, що $a_{6} - a_{4} = 2$. Знайдіть значення виразу $a_{10} - a_{6}$. \nmtyear{2026}}
\vspace{0.3cm}

\answerTable{4}{3}{5}{2}{-4}

\vspace{0.5cm}

\noindent\makebox[1.5em][l]{\textbf{38.}}\parbox[t]{\dimexpr\textwidth-1.5em}{В арифметичній прогресії $(a_n)$ відомо, що $a_{6} - a_{1} = -15$. Знайдіть значення виразу $a_{6} - a_{4}$. \nmtyear{2026}}
\vspace{0.3cm}

\answerTable{-15}{-3}{-6}{6}{-9}

\vspace{0.5cm}

\noindent\makebox[1.5em][l]{\textbf{39.}}\parbox[t]{\dimexpr\textwidth-1.5em}{В арифметичній прогресії $(a_n)$ відомо, що $a_{8} - a_{3} = 10$. Знайдіть значення виразу $a_{7} - a_{4}$. \nmtyear{2026}}
\vspace{0.3cm}

\answerTable{10}{4}{8}{6}{-6}

\vspace{0.5cm}

\noindent\makebox[1.5em][l]{\textbf{40.}}\parbox[t]{\dimexpr\textwidth-1.5em}{В арифметичній прогресії $(a_n)$ відомо, що $a_{4} - a_{2} = -14$. Знайдіть значення виразу $a_{7} - a_{4}$. \nmtyear{2026}}
\vspace{0.3cm}

\answerTable{-21}{21}{-14}{-28}{-19}

\vspace{0.5cm}

% === Arithmetic Progression: Sum ===
\noindent\makebox[1.5em][l]{\textbf{41.}}\parbox[t]{\dimexpr\textwidth-1.5em}{Обчисліть суму перших 10-ти членів арифметичної прогресії $(a_n)$, якщо $a_1 + a_{10} = 21$. \nmtyear{2026}}
\vspace{0.3cm}

\answerTable{21}{-105}{105}{115}{210}

\vspace{0.5cm}

\noindent\makebox[1.5em][l]{\textbf{42.}}\parbox[t]{\dimexpr\textwidth-1.5em}{Обчисліть суму перших 10-ти членів арифметичної прогресії $(a_n)$, якщо $a_1 + a_{10} = -23$. \nmtyear{2026}}
\vspace{0.3cm}

\answerTable{115}{-115}{-230}{-105}{-125}

\vspace{0.5cm}

\noindent\makebox[1.5em][l]{\textbf{43.}}\parbox[t]{\dimexpr\textwidth-1.5em}{Обчисліть суму перших 20-ти членів арифметичної прогресії $(a_n)$, якщо $a_1 + a_{20} = -25$. \nmtyear{2026}}
\vspace{0.3cm}

\answerTable{-250}{-25}{-270}{-230}{-500}

\vspace{0.5cm}

\noindent\makebox[1.5em][l]{\textbf{44.}}\parbox[t]{\dimexpr\textwidth-1.5em}{Обчисліть суму перших 8-ти членів арифметичної прогресії $(a_n)$, якщо $a_1 + a_{8} = 18$. \nmtyear{2026}}
\vspace{0.3cm}

\answerTable{80}{-72}{72}{144}{18}

\vspace{0.5cm}

\noindent\makebox[1.5em][l]{\textbf{45.}}\parbox[t]{\dimexpr\textwidth-1.5em}{Обчисліть суму перших 20-ти членів арифметичної прогресії $(a_n)$, якщо $a_1 + a_{20} = -44$. \nmtyear{2026}}
\vspace{0.3cm}

\answerTable{440}{-880}{-460}{-440}{-420}

\vspace{0.5cm}

\noindent\makebox[1.5em][l]{\textbf{46.}}\parbox[t]{\dimexpr\textwidth-1.5em}{Обчисліть суму перших 20-ти членів арифметичної прогресії $(a_n)$, якщо $a_1 + a_{20} = -43$. \nmtyear{2026}}
\vspace{0.3cm}

\answerTable{-410}{-43}{-450}{-860}{-430}

\vspace{0.5cm}

\noindent\makebox[1.5em][l]{\textbf{47.}}\parbox[t]{\dimexpr\textwidth-1.5em}{Обчисліть суму перших 100-ти членів арифметичної прогресії $(a_n)$, якщо $a_1 + a_{100} = 9$. \nmtyear{2026}}
\vspace{0.3cm}

\answerTable{450}{9}{-450}{550}{900}

\vspace{0.5cm}

\noindent\makebox[1.5em][l]{\textbf{48.}}\parbox[t]{\dimexpr\textwidth-1.5em}{Обчисліть суму перших 100-ти членів арифметичної прогресії $(a_n)$, якщо $a_1 + a_{100} = 21$. \nmtyear{2026}}
\vspace{0.3cm}

\answerTable{2100}{1150}{950}{1050}{-1050}

\vspace{0.5cm}

\noindent\makebox[1.5em][l]{\textbf{49.}}\parbox[t]{\dimexpr\textwidth-1.5em}{Обчисліть суму перших 12-ти членів арифметичної прогресії $(a_n)$, якщо $a_1 + a_{12} = -24$. \nmtyear{2026}}
\vspace{0.3cm}

\answerTable{-24}{-288}{-132}{-144}{-156}

\vspace{0.5cm}

\noindent\makebox[1.5em][l]{\textbf{50.}}\parbox[t]{\dimexpr\textwidth-1.5em}{Обчисліть суму перших 100-ти членів арифметичної прогресії $(a_n)$, якщо $a_1 + a_{100} = -45$. \nmtyear{2026}}
\vspace{0.3cm}

\answerTable{2250}{-2350}{-2150}{-2250}{-45}

\vspace{0.5cm}

\noindent\makebox[1.5em][l]{\textbf{51.}}\parbox[t]{\dimexpr\textwidth-1.5em}{Обчисліть суму перших 20-ти членів арифметичної прогресії $(a_n)$, якщо $a_1 + a_{20} = -37$. \nmtyear{2026}}
\vspace{0.3cm}

\answerTable{-37}{-370}{370}{-390}{-350}

\vspace{0.5cm}

\noindent\makebox[1.5em][l]{\textbf{52.}}\parbox[t]{\dimexpr\textwidth-1.5em}{Обчисліть суму перших 8-ти членів арифметичної прогресії $(a_n)$, якщо $a_1 + a_{8} = 42$. \nmtyear{2026}}
\vspace{0.3cm}

\answerTable{160}{42}{168}{-168}{176}

\vspace{0.5cm}

\noindent\makebox[1.5em][l]{\textbf{53.}}\parbox[t]{\dimexpr\textwidth-1.5em}{Обчисліть суму перших 20-ти членів арифметичної прогресії $(a_n)$, якщо $a_1 + a_{20} = -26$. \nmtyear{2026}}
\vspace{0.3cm}

\answerTable{-520}{260}{-240}{-260}{-26}

\vspace{0.5cm}

\noindent\makebox[1.5em][l]{\textbf{54.}}\parbox[t]{\dimexpr\textwidth-1.5em}{Обчисліть суму перших 20-ти членів арифметичної прогресії $(a_n)$, якщо $a_1 + a_{20} = 19$. \nmtyear{2026}}
\vspace{0.3cm}

\answerTable{19}{190}{380}{-190}{170}

\vspace{0.5cm}

\noindent\makebox[1.5em][l]{\textbf{55.}}\parbox[t]{\dimexpr\textwidth-1.5em}{Обчисліть суму перших 16-ти членів арифметичної прогресії $(a_n)$, якщо $a_1 + a_{16} = 49$. \nmtyear{2026}}
\vspace{0.3cm}

\answerTable{376}{49}{392}{784}{-392}

\vspace{0.5cm}

\noindent\makebox[1.5em][l]{\textbf{56.}}\parbox[t]{\dimexpr\textwidth-1.5em}{Обчисліть суму перших 16-ти членів арифметичної прогресії $(a_n)$, якщо $a_1 + a_{16} = -21$. \nmtyear{2026}}
\vspace{0.3cm}

\answerTable{168}{-21}{-152}{-336}{-168}

\vspace{0.5cm}

\noindent\makebox[1.5em][l]{\textbf{57.}}\parbox[t]{\dimexpr\textwidth-1.5em}{Обчисліть суму перших 10-ти членів арифметичної прогресії $(a_n)$, якщо $a_1 + a_{10} = 32$. \nmtyear{2026}}
\vspace{0.3cm}

\answerTable{160}{150}{32}{-160}{320}

\vspace{0.5cm}

\noindent\makebox[1.5em][l]{\textbf{58.}}\parbox[t]{\dimexpr\textwidth-1.5em}{Обчисліть суму перших 8-ти членів арифметичної прогресії $(a_n)$, якщо $a_1 + a_{8} = -6$. \nmtyear{2026}}
\vspace{0.3cm}

\answerTable{24}{-48}{-16}{-24}{-6}

\vspace{0.5cm}

\noindent\makebox[1.5em][l]{\textbf{59.}}\parbox[t]{\dimexpr\textwidth-1.5em}{Обчисліть суму перших 10-ти членів арифметичної прогресії $(a_n)$, якщо $a_1 + a_{10} = 5$. \nmtyear{2026}}
\vspace{0.3cm}

\answerTable{50}{-25}{25}{5}{35}

\vspace{0.5cm}

\noindent\makebox[1.5em][l]{\textbf{60.}}\parbox[t]{\dimexpr\textwidth-1.5em}{Обчисліть суму перших 16-ти членів арифметичної прогресії $(a_n)$, якщо $a_1 + a_{16} = -24$. \nmtyear{2026}}
\vspace{0.3cm}

\answerTable{-24}{-192}{-384}{-176}{-208}

\vspace{0.5cm}

% === Arithmetic Progression: Count Terms ===
\noindent\makebox[1.5em][l]{\textbf{61.}}\parbox[t]{\dimexpr\textwidth-1.5em}{В арифметичній прогресії $(a_n)$ перший член $a_1 = -8{,}3$, різниця $d = 1$. Скільки всього \textit{від'ємних} членів має ця прогресія? \nmtyear{2026}}
\vspace{0.3cm}

\answerTable{8}{9}{7}{11}{10}

\vspace{0.5cm}

\noindent\makebox[1.5em][l]{\textbf{62.}}\parbox[t]{\dimexpr\textwidth-1.5em}{В арифметичній прогресії $(a_n)$ перший член $a_1 = -18{,}3$, різниця $d = 3$. Скільки всього \textit{від'ємних} членів має ця прогресія? \nmtyear{2026}}
\vspace{0.3cm}

\answerTable{5}{6}{8}{7}{9}

\vspace{0.5cm}

\noindent\makebox[1.5em][l]{\textbf{63.}}\parbox[t]{\dimexpr\textwidth-1.5em}{В арифметичній прогресії $(a_n)$ перший член $a_1 = 14{,}5$, різниця $d = -2$. Скільки всього \textit{додатних} членів має ця прогресія? \nmtyear{2026}}
\vspace{0.3cm}

\answerTable{10}{6}{8}{7}{9}

\vspace{0.5cm}

\noindent\makebox[1.5em][l]{\textbf{64.}}\parbox[t]{\dimexpr\textwidth-1.5em}{В арифметичній прогресії $(a_n)$ перший член $a_1 = -5{,}8$, різниця $d = 0{,}4$. Скільки всього \textit{від'ємних} членів має ця прогресія? \nmtyear{2026}}
\vspace{0.3cm}

\answerTable{15}{12}{14}{11}{13}

\vspace{0.5cm}

\noindent\makebox[1.5em][l]{\textbf{65.}}\parbox[t]{\dimexpr\textwidth-1.5em}{В арифметичній прогресії $(a_n)$ перший член $a_1 = -32{,}8$, різниця $d = 2{,}5$. Скільки всього \textit{від'ємних} членів має ця прогресія? \nmtyear{2026}}
\vspace{0.3cm}

\answerTable{12}{14}{16}{13}{15}

\vspace{0.5cm}

\noindent\makebox[1.5em][l]{\textbf{66.}}\parbox[t]{\dimexpr\textwidth-1.5em}{В арифметичній прогресії $(a_n)$ перший член $a_1 = 9{,}8$, різниця $d = -0{,}8$. Скільки всього \textit{додатних} членів має ця прогресія? \nmtyear{2026}}
\vspace{0.3cm}

\answerTable{13}{11}{14}{10}{12}

\vspace{0.5cm}

\noindent\makebox[1.5em][l]{\textbf{67.}}\parbox[t]{\dimexpr\textwidth-1.5em}{В арифметичній прогресії $(a_n)$ перший член $a_1 = 19$, різниця $d = -2$. Скільки всього \textit{додатних} членів має ця прогресія? \nmtyear{2026}}
\vspace{0.3cm}

\answerTable{8}{9}{12}{11}{10}

\vspace{0.5cm}

\noindent\makebox[1.5em][l]{\textbf{68.}}\parbox[t]{\dimexpr\textwidth-1.5em}{В арифметичній прогресії $(a_n)$ перший член $a_1 = 20{,}3$, різниця $d = -2{,}5$. Скільки всього \textit{додатних} членів має ця прогресія? \nmtyear{2026}}
\vspace{0.3cm}

\answerTable{7}{8}{9}{10}{11}

\vspace{0.5cm}

\noindent\makebox[1.5em][l]{\textbf{69.}}\parbox[t]{\dimexpr\textwidth-1.5em}{В арифметичній прогресії $(a_n)$ перший член $a_1 = -8{,}2$, різниця $d = 1$. Скільки всього \textit{від'ємних} членів має ця прогресія? \nmtyear{2026}}
\vspace{0.3cm}

\answerTable{10}{9}{7}{8}{11}

\vspace{0.5cm}

\noindent\makebox[1.5em][l]{\textbf{70.}}\parbox[t]{\dimexpr\textwidth-1.5em}{В арифметичній прогресії $(a_n)$ перший член $a_1 = 15{,}1$, різниця $d = -2{,}5$. Скільки всього \textit{додатних} членів має ця прогресія? \nmtyear{2026}}
\vspace{0.3cm}

\answerTable{6}{9}{8}{7}{5}

\vspace{0.5cm}

\noindent\makebox[1.5em][l]{\textbf{71.}}\parbox[t]{\dimexpr\textwidth-1.5em}{В арифметичній прогресії $(a_n)$ перший член $a_1 = -33{,}1$, різниця $d = 3$. Скільки всього \textit{від'ємних} членів має ця прогресія? \nmtyear{2026}}
\vspace{0.3cm}

\answerTable{10}{14}{13}{11}{12}

\vspace{0.5cm}

\noindent\makebox[1.5em][l]{\textbf{72.}}\parbox[t]{\dimexpr\textwidth-1.5em}{В арифметичній прогресії $(a_n)$ перший член $a_1 = 7{,}5$, різниця $d = -0{,}8$. Скільки всього \textit{додатних} членів має ця прогресія? \nmtyear{2026}}
\vspace{0.3cm}

\answerTable{8}{11}{12}{10}{9}

\vspace{0.5cm}

\noindent\makebox[1.5em][l]{\textbf{73.}}\parbox[t]{\dimexpr\textwidth-1.5em}{В арифметичній прогресії $(a_n)$ перший член $a_1 = -11{,}4$, різниця $d = 0{,}8$. Скільки всього \textit{від'ємних} членів має ця прогресія? \nmtyear{2026}}
\vspace{0.3cm}

\answerTable{15}{16}{14}{13}{12}

\vspace{0.5cm}

\noindent\makebox[1.5em][l]{\textbf{74.}}\parbox[t]{\dimexpr\textwidth-1.5em}{В арифметичній прогресії $(a_n)$ перший член $a_1 = -37$, різниця $d = 2{,}5$. Скільки всього \textit{від'ємних} членів має ця прогресія? \nmtyear{2026}}
\vspace{0.3cm}

\answerTable{14}{16}{17}{13}{15}

\vspace{0.5cm}

\noindent\makebox[1.5em][l]{\textbf{75.}}\parbox[t]{\dimexpr\textwidth-1.5em}{В арифметичній прогресії $(a_n)$ перший член $a_1 = -8$, різниця $d = 1$. Скільки всього \textit{від'ємних} членів має ця прогресія? \nmtyear{2026}}
\vspace{0.3cm}

\answerTable{6}{10}{9}{8}{7}

\vspace{0.5cm}

\noindent\makebox[1.5em][l]{\textbf{76.}}\parbox[t]{\dimexpr\textwidth-1.5em}{В арифметичній прогресії $(a_n)$ перший член $a_1 = 3{,}2$, різниця $d = -0{,}5$. Скільки всього \textit{додатних} членів має ця прогресія? \nmtyear{2026}}
\vspace{0.3cm}

\answerTable{9}{8}{6}{5}{7}

\vspace{0.5cm}

\noindent\makebox[1.5em][l]{\textbf{77.}}\parbox[t]{\dimexpr\textwidth-1.5em}{В арифметичній прогресії $(a_n)$ перший член $a_1 = 22$, різниця $d = -3$. Скільки всього \textit{додатних} членів має ця прогресія? \nmtyear{2026}}
\vspace{0.3cm}

\answerTable{8}{9}{10}{7}{6}

\vspace{0.5cm}

\noindent\makebox[1.5em][l]{\textbf{78.}}\parbox[t]{\dimexpr\textwidth-1.5em}{В арифметичній прогресії $(a_n)$ перший член $a_1 = -32{,}7$, різниця $d = 2{,}5$. Скільки всього \textit{від'ємних} членів має ця прогресія? \nmtyear{2026}}
\vspace{0.3cm}

\answerTable{15}{13}{14}{16}{12}

\vspace{0.5cm}

\noindent\makebox[1.5em][l]{\textbf{79.}}\parbox[t]{\dimexpr\textwidth-1.5em}{В арифметичній прогресії $(a_n)$ перший член $a_1 = 7{,}5$, різниця $d = -0{,}5$. Скільки всього \textit{додатних} членів має ця прогресія? \nmtyear{2026}}
\vspace{0.3cm}

\answerTable{14}{15}{13}{17}{16}

\vspace{0.5cm}

\noindent\makebox[1.5em][l]{\textbf{80.}}\parbox[t]{\dimexpr\textwidth-1.5em}{В арифметичній прогресії $(a_n)$ перший член $a_1 = 2{,}6$, різниця $d = -0{,}5$. Скільки всього \textit{додатних} членів має ця прогресія? \nmtyear{2026}}
\vspace{0.3cm}

\answerTable{6}{8}{5}{7}{4}

\vspace{0.5cm}

% === Arithmetic Progression: Middle Term ===
\noindent\makebox[1.5em][l]{\textbf{81.}}\parbox[t]{\dimexpr\textwidth-1.5em}{Визначте 19-й член $a_{19}$ арифметичної прогресії $(a_n)$, у якої $a_{18} = 8$, $a_{20} = 16$. \nmtyear{2026}}
\vspace{0.3cm}

\answerTable{4}{16}{8}{24}{12}

\vspace{0.5cm}

\noindent\makebox[1.5em][l]{\textbf{82.}}\parbox[t]{\dimexpr\textwidth-1.5em}{Визначте 18-й член $a_{18}$ арифметичної прогресії $(a_n)$, у якої $a_{17} = -16$, $a_{19} = -11$. \nmtyear{2026}}
\vspace{0.3cm}

\answerTable{-13{,}5}{2{,}5}{5}{-27}{-16}

\vspace{0.5cm}

\noindent\makebox[1.5em][l]{\textbf{83.}}\parbox[t]{\dimexpr\textwidth-1.5em}{Визначте 19-й член $a_{19}$ арифметичної прогресії $(a_n)$, у якої $a_{18} = -15$, $a_{20} = -17$. \nmtyear{2026}}
\vspace{0.3cm}

\answerTable{-1}{-16}{2}{-15}{-32}

\vspace{0.5cm}

\noindent\makebox[1.5em][l]{\textbf{84.}}\parbox[t]{\dimexpr\textwidth-1.5em}{Визначте 5-й член $a_{5}$ арифметичної прогресії $(a_n)$, у якої $a_{4} = 0$, $a_{6} = 10$. \nmtyear{2026}}
\vspace{0.3cm}

\answerTable{5}{0}{10}{15}{6}

\vspace{0.5cm}

\noindent\makebox[1.5em][l]{\textbf{85.}}\parbox[t]{\dimexpr\textwidth-1.5em}{Визначте 8-й член $a_{8}$ арифметичної прогресії $(a_n)$, у якої $a_{7} = 34$, $a_{9} = 37$. \nmtyear{2026}}
\vspace{0.3cm}

\answerTable{71}{1{,}5}{3}{35{,}5}{37}

\vspace{0.5cm}

\noindent\makebox[1.5em][l]{\textbf{86.}}\parbox[t]{\dimexpr\textwidth-1.5em}{Визначте 17-й член $a_{17}$ арифметичної прогресії $(a_n)$, у якої $a_{16} = 28$, $a_{18} = 33$. \nmtyear{2026}}
\vspace{0.3cm}

\answerTable{28}{2{,}5}{30{,}5}{33}{61}

\vspace{0.5cm}

\noindent\makebox[1.5em][l]{\textbf{87.}}\parbox[t]{\dimexpr\textwidth-1.5em}{Визначте 20-й член $a_{20}$ арифметичної прогресії $(a_n)$, у якої $a_{19} = 26$, $a_{21} = 20$. \nmtyear{2026}}
\vspace{0.3cm}

\answerTable{23}{-3}{6}{20}{26}

\vspace{0.5cm}

\noindent\makebox[1.5em][l]{\textbf{88.}}\parbox[t]{\dimexpr\textwidth-1.5em}{Визначте 10-й член $a_{10}$ арифметичної прогресії $(a_n)$, у якої $a_{9} = 41$, $a_{11} = 49$. \nmtyear{2026}}
\vspace{0.3cm}

\answerTable{4}{49}{8}{45}{41}

\vspace{0.5cm}

\noindent\makebox[1.5em][l]{\textbf{89.}}\parbox[t]{\dimexpr\textwidth-1.5em}{Визначте 20-й член $a_{20}$ арифметичної прогресії $(a_n)$, у якої $a_{19} = -6$, $a_{21} = -5$. \nmtyear{2026}}
\vspace{0.3cm}

\answerTable{-6}{0{,}5}{-5{,}5}{-5}{-11}

\vspace{0.5cm}

\noindent\makebox[1.5em][l]{\textbf{90.}}\parbox[t]{\dimexpr\textwidth-1.5em}{Визначте 10-й член $a_{10}$ арифметичної прогресії $(a_n)$, у якої $a_{9} = -18$, $a_{11} = -22$. \nmtyear{2026}}
\vspace{0.3cm}

\answerTable{-22}{-18}{-20}{4}{-40}

\vspace{0.5cm}

\noindent\makebox[1.5em][l]{\textbf{91.}}\parbox[t]{\dimexpr\textwidth-1.5em}{Визначте 9-й член $a_{9}$ арифметичної прогресії $(a_n)$, у якої $a_{8} = 30$, $a_{10} = 28$. \nmtyear{2026}}
\vspace{0.3cm}

\answerTable{2}{58}{28}{30}{29}

\vspace{0.5cm}

\noindent\makebox[1.5em][l]{\textbf{92.}}\parbox[t]{\dimexpr\textwidth-1.5em}{Визначте 5-й член $a_{5}$ арифметичної прогресії $(a_n)$, у якої $a_{4} = -16$, $a_{6} = -20$. \nmtyear{2026}}
\vspace{0.3cm}

\answerTable{-18}{-16}{-2}{-20}{-36}

\vspace{0.5cm}

\noindent\makebox[1.5em][l]{\textbf{93.}}\parbox[t]{\dimexpr\textwidth-1.5em}{Визначте 7-й член $a_{7}$ арифметичної прогресії $(a_n)$, у якої $a_{6} = 36$, $a_{8} = 40$. \nmtyear{2026}}
\vspace{0.3cm}

\answerTable{2}{38}{4}{40}{36}

\vspace{0.5cm}

\noindent\makebox[1.5em][l]{\textbf{94.}}\parbox[t]{\dimexpr\textwidth-1.5em}{Визначте 6-й член $a_{6}$ арифметичної прогресії $(a_n)$, у якої $a_{5} = 31$, $a_{7} = 27$. \nmtyear{2026}}
\vspace{0.3cm}

\answerTable{31}{29}{27}{4}{58}

\vspace{0.5cm}

\noindent\makebox[1.5em][l]{\textbf{95.}}\parbox[t]{\dimexpr\textwidth-1.5em}{Визначте 12-й член $a_{12}$ арифметичної прогресії $(a_n)$, у якої $a_{11} = 22$, $a_{13} = 28$. \nmtyear{2026}}
\vspace{0.3cm}

\answerTable{25}{50}{28}{6}{22}

\vspace{0.5cm}

\noindent\makebox[1.5em][l]{\textbf{96.}}\parbox[t]{\dimexpr\textwidth-1.5em}{Визначте 20-й член $a_{20}$ арифметичної прогресії $(a_n)$, у якої $a_{19} = 34$, $a_{21} = 39$. \nmtyear{2026}}
\vspace{0.3cm}

\answerTable{73}{36{,}5}{5}{39}{34}

\vspace{0.5cm}

\noindent\makebox[1.5em][l]{\textbf{97.}}\parbox[t]{\dimexpr\textwidth-1.5em}{Визначте 13-й член $a_{13}$ арифметичної прогресії $(a_n)$, у якої $a_{12} = 33$, $a_{14} = 39$. \nmtyear{2026}}
\vspace{0.3cm}

\answerTable{36}{72}{3}{39}{6}

\vspace{0.5cm}

\noindent\makebox[1.5em][l]{\textbf{98.}}\parbox[t]{\dimexpr\textwidth-1.5em}{Визначте 15-й член $a_{15}$ арифметичної прогресії $(a_n)$, у якої $a_{14} = -18$, $a_{16} = -13$. \nmtyear{2026}}
\vspace{0.3cm}

\answerTable{-15{,}5}{2{,}5}{-13}{5}{-31}

\vspace{0.5cm}

\noindent\makebox[1.5em][l]{\textbf{99.}}\parbox[t]{\dimexpr\textwidth-1.5em}{Визначте 14-й член $a_{14}$ арифметичної прогресії $(a_n)$, у якої $a_{13} = -10$, $a_{15} = 0$. \nmtyear{2026}}
\vspace{0.3cm}

\answerTable{10}{-10}{5}{-5}{0}

\vspace{0.5cm}

\noindent\makebox[1.5em][l]{\textbf{100.}}\parbox[t]{\dimexpr\textwidth-1.5em}{Визначте 6-й член $a_{6}$ арифметичної прогресії $(a_n)$, у якої $a_{5} = 19$, $a_{7} = 13$. \nmtyear{2026}}
\vspace{0.3cm}

\answerTable{13}{32}{16}{6}{19}

\vspace{0.5cm}

% === Arithmetic Progression: Formula Search ===
\noindent\makebox[1.5em][l]{\textbf{101.}}\parbox[t]{\dimexpr\textwidth-1.5em}{Арифметичну прогресію $(a_n)$ задано формулою $n$-го члена $a_n = 16  -4n$. Визначте номер члена, значення якого дорівнює $-108$. \nmtyear{2026}}
\vspace{0.3cm}

\answerTable{27}{41}{21}{31}{30}

\vspace{0.5cm}

\noindent\makebox[1.5em][l]{\textbf{102.}}\parbox[t]{\dimexpr\textwidth-1.5em}{Арифметичну прогресію $(a_n)$ задано формулою $n$-го члена $a_n = 15  -1{,}5n$. Визначте номер члена, значення якого дорівнює $-4{,}5$. \nmtyear{2026}}
\vspace{0.3cm}

\answerTable{3}{13}{14}{23}{12}

\vspace{0.5cm}

\noindent\makebox[1.5em][l]{\textbf{103.}}\parbox[t]{\dimexpr\textwidth-1.5em}{Арифметичну прогресію $(a_n)$ задано формулою $n$-го члена $a_n = 21 + 2{,}5n$. Визначте номер члена, значення якого дорівнює $43{,}5$. \nmtyear{2026}}
\vspace{0.3cm}

\answerTable{17{,}4}{8}{1}{9}{10}

\vspace{0.5cm}

\noindent\makebox[1.5em][l]{\textbf{104.}}\parbox[t]{\dimexpr\textwidth-1.5em}{Арифметичну прогресію $(a_n)$ задано формулою $n$-го члена $a_n = 19  -4n$. Визначте номер члена, значення якого дорівнює $-165$. \nmtyear{2026}}
\vspace{0.3cm}

\answerTable{46}{36}{47}{56}{41{,}25}

\vspace{0.5cm}

\noindent\makebox[1.5em][l]{\textbf{105.}}\parbox[t]{\dimexpr\textwidth-1.5em}{Арифметичну прогресію $(a_n)$ задано формулою $n$-го члена $a_n = 44 + 4n$. Визначте номер члена, значення якого дорівнює $244$. \nmtyear{2026}}
\vspace{0.3cm}

\answerTable{61}{50}{51}{49}{40}

\vspace{0.5cm}

\noindent\makebox[1.5em][l]{\textbf{106.}}\parbox[t]{\dimexpr\textwidth-1.5em}{Арифметичну прогресію $(a_n)$ задано формулою $n$-го члена $a_n = 48  -1{,}5n$. Визначте номер члена, значення якого дорівнює $-9$. \nmtyear{2026}}
\vspace{0.3cm}

\answerTable{28}{38}{37}{48}{39}

\vspace{0.5cm}

\noindent\makebox[1.5em][l]{\textbf{107.}}\parbox[t]{\dimexpr\textwidth-1.5em}{Арифметичну прогресію $(a_n)$ задано формулою $n$-го члена $a_n = 23 + 2{,}5n$. Визначте номер члена, значення якого дорівнює $40{,}5$. \nmtyear{2026}}
\vspace{0.3cm}

\answerTable{17}{16{,}2}{7}{1}{6}

\vspace{0.5cm}

\noindent\makebox[1.5em][l]{\textbf{108.}}\parbox[t]{\dimexpr\textwidth-1.5em}{Арифметичну прогресію $(a_n)$ задано формулою $n$-го члена $a_n = 31 + 4n$. Визначте номер члена, значення якого дорівнює $135$. \nmtyear{2026}}
\vspace{0.3cm}

\answerTable{33{,}75}{25}{36}{27}{26}

\vspace{0.5cm}

\noindent\makebox[1.5em][l]{\textbf{109.}}\parbox[t]{\dimexpr\textwidth-1.5em}{Арифметичну прогресію $(a_n)$ задано формулою $n$-го члена $a_n = 35  -2{,}5n$. Визначте номер члена, значення якого дорівнює $20$. \nmtyear{2026}}
\vspace{0.3cm}

\answerTable{7}{6}{1}{5}{16}

\vspace{0.5cm}

\noindent\makebox[1.5em][l]{\textbf{110.}}\parbox[t]{\dimexpr\textwidth-1.5em}{Арифметичну прогресію $(a_n)$ задано формулою $n$-го члена $a_n = 13  -4n$. Визначте номер члена, значення якого дорівнює $-115$. \nmtyear{2026}}
\vspace{0.3cm}

\answerTable{33}{42}{32}{22}{28{,}75}

\vspace{0.5cm}

\noindent\makebox[1.5em][l]{\textbf{111.}}\parbox[t]{\dimexpr\textwidth-1.5em}{Арифметичну прогресію $(a_n)$ задано формулою $n$-го члена $a_n = 49  -4n$. Визначте номер члена, значення якого дорівнює $21$. \nmtyear{2026}}
\vspace{0.3cm}

\answerTable{17}{7}{8}{6}{1}

\vspace{0.5cm}

\noindent\makebox[1.5em][l]{\textbf{112.}}\parbox[t]{\dimexpr\textwidth-1.5em}{Арифметичну прогресію $(a_n)$ задано формулою $n$-го члена $a_n = 32 + 2{,}5n$. Визначте номер члена, значення якого дорівнює $82$. \nmtyear{2026}}
\vspace{0.3cm}

\answerTable{19}{32{,}8}{10}{20}{30}

\vspace{0.5cm}

\noindent\makebox[1.5em][l]{\textbf{113.}}\parbox[t]{\dimexpr\textwidth-1.5em}{Арифметичну прогресію $(a_n)$ задано формулою $n$-го члена $a_n = 28 + 3n$. Визначте номер члена, значення якого дорівнює $127$. \nmtyear{2026}}
\vspace{0.3cm}

\answerTable{43}{33}{23}{42{,}33}{32}

\vspace{0.5cm}

\noindent\makebox[1.5em][l]{\textbf{114.}}\parbox[t]{\dimexpr\textwidth-1.5em}{Арифметичну прогресію $(a_n)$ задано формулою $n$-го члена $a_n = 26 + 2{,}5n$. Визначте номер члена, значення якого дорівнює $98{,}5$. \nmtyear{2026}}
\vspace{0.3cm}

\answerTable{30}{19}{39}{29}{39{,}4}

\vspace{0.5cm}

\noindent\makebox[1.5em][l]{\textbf{115.}}\parbox[t]{\dimexpr\textwidth-1.5em}{Арифметичну прогресію $(a_n)$ задано формулою $n$-го члена $a_n = 11  -3n$. Визначте номер члена, значення якого дорівнює $-79$. \nmtyear{2026}}
\vspace{0.3cm}

\answerTable{29}{31}{20}{40}{30}

\vspace{0.5cm}

\noindent\makebox[1.5em][l]{\textbf{116.}}\parbox[t]{\dimexpr\textwidth-1.5em}{Арифметичну прогресію $(a_n)$ задано формулою $n$-го члена $a_n = 37 + 1{,}5n$. Визначте номер члена, значення якого дорівнює $58$. \nmtyear{2026}}
\vspace{0.3cm}

\answerTable{15}{24}{14}{13}{38{,}67}

\vspace{0.5cm}

\noindent\makebox[1.5em][l]{\textbf{117.}}\parbox[t]{\dimexpr\textwidth-1.5em}{Арифметичну прогресію $(a_n)$ задано формулою $n$-го члена $a_n = 14  -0{,}5n$. Визначте номер члена, значення якого дорівнює $4{,}5$. \nmtyear{2026}}
\vspace{0.3cm}

\answerTable{29}{19}{20}{9}{18}

\vspace{0.5cm}

\noindent\makebox[1.5em][l]{\textbf{118.}}\parbox[t]{\dimexpr\textwidth-1.5em}{Арифметичну прогресію $(a_n)$ задано формулою $n$-го члена $a_n = 43  -0{,}5n$. Визначте номер члена, значення якого дорівнює $31$. \nmtyear{2026}}
\vspace{0.3cm}

\answerTable{23}{34}{24}{25}{14}

\vspace{0.5cm}

\noindent\makebox[1.5em][l]{\textbf{119.}}\parbox[t]{\dimexpr\textwidth-1.5em}{Арифметичну прогресію $(a_n)$ задано формулою $n$-го члена $a_n = 18  -4n$. Визначте номер члена, значення якого дорівнює $-62$. \nmtyear{2026}}
\vspace{0.3cm}

\answerTable{10}{15{,}5}{19}{20}{30}

\vspace{0.5cm}

\noindent\makebox[1.5em][l]{\textbf{120.}}\parbox[t]{\dimexpr\textwidth-1.5em}{Арифметичну прогресію $(a_n)$ задано формулою $n$-го члена $a_n = 42  -4n$. Визначте номер члена, значення якого дорівнює $-110$. \nmtyear{2026}}
\vspace{0.3cm}

\answerTable{28}{27{,}5}{37}{48}{38}

\vspace{0.5cm}

% === Arithmetic Progression: Word Problem ===
\noindent\makebox[1.5em][l]{\textbf{121.}}\parbox[t]{\dimexpr\textwidth-1.5em}{У залі для глядачів цирку встановлено 26 рядів крісел: у першому ряду 56 крісла, а в кожному наступному ряду кількість крісел на те саме число більше, ніж у попередньому. Визначте кількість крісел у \textit{17-му} ряду, якщо в останньому ряду 156 крісла. \nmtyear{2026}}
\vspace{0.3cm}

\answerTable{120}{124}{116}{112}{128}

\vspace{0.5cm}

\noindent\makebox[1.5em][l]{\textbf{122.}}\parbox[t]{\dimexpr\textwidth-1.5em}{Студент вивчав мову за методикою: у перший день він запам'ятав 13 слів, а кожного наступного дня — на 3 слів більше, ніж попереднього. Скільки всього слів запам'ятав студент за 16 днів? \nmtyear{2026}}
\vspace{0.3cm}

\answerTable{208}{928}{588}{548}{568}

\vspace{0.5cm}

\noindent\makebox[1.5em][l]{\textbf{123.}}\parbox[t]{\dimexpr\textwidth-1.5em}{На рисунку зображено поперечний переріз стосу колод. У нижньому ряду стосу 5 колод, а у верхньому — 1. Визначте загальну кількість колод.
            \begin{center}
            \begin{tikzpicture}[scale=0.5]
                \newcommand{\woodLog}[3]{
                    \begin{scope}[shift={(#1,#2)}]
                        \draw[fill=woodinner, draw=black, thick] (0,0) circle (0.5);
                        \draw[woodouter!80, thin] (0,0) circle (0.35);
                        \draw[woodouter!80, thin] (0,0) circle (0.2);
                        \begin{scope}[rotate=#3]
                            \fill[woodouter] (0,0) -- (0.4, 0.05) -- (0.5, 0.1) -- (0.5, -0.1) -- (0.4, -0.05) -- cycle;
                        \end{scope}
                    \end{scope}
                }
                \def\rows{5} 
                \foreach \row in {1,...,\rows} {
                    \foreach \col in {1,...,\row} {
                        \pgfmathsetmacro{\x}{(\col-1) - (\row-1)*0.5}
                        \pgfmathsetmacro{\y}{-(\row-1)*0.866}
                        \pgfmathsetmacro{\angle}{mod(\col*70 + \row*50, 360)}
                        \woodLog{\x}{\y}{\angle}
                    }
                }
            \end{tikzpicture}
            \end{center}
             \nmtyear{2026}}
\vspace{0.3cm}

\answerTable{5}{15}{20}{25}{10}

\vspace{0.5cm}

\noindent\makebox[1.5em][l]{\textbf{124.}}\parbox[t]{\dimexpr\textwidth-1.5em}{У залі для глядачів цирку встановлено 28 рядів крісел: у першому ряду 32 крісла, а в кожному наступному ряду кількість крісел на те саме число більше, ніж у попередньому. Визначте кількість крісел у \textit{3-му} ряду, якщо в останньому ряду 113 крісла. \nmtyear{2026}}
\vspace{0.3cm}

\answerTable{44}{35}{38}{32}{41}

\vspace{0.5cm}

\noindent\makebox[1.5em][l]{\textbf{125.}}\parbox[t]{\dimexpr\textwidth-1.5em}{У залі для глядачів цирку встановлено 23 рядів крісел: у першому ряду 48 крісла, а в кожному наступному ряду кількість крісел на те саме число більше, ніж у попередньому. Визначте кількість крісел у \textit{4-му} ряду, якщо в останньому ряду 114 крісла. \nmtyear{2026}}
\vspace{0.3cm}

\answerTable{51}{54}{57}{63}{60}

\vspace{0.5cm}

\noindent\makebox[1.5em][l]{\textbf{126.}}\parbox[t]{\dimexpr\textwidth-1.5em}{Студент вивчав мову за методикою: у перший день він запам'ятав 9 слів, а кожного наступного дня — на 3 слів більше, ніж попереднього. Скільки всього слів запам'ятав студент за 18 днів? \nmtyear{2026}}
\vspace{0.3cm}

\answerTable{1080}{641}{621}{162}{601}

\vspace{0.5cm}

\noindent\makebox[1.5em][l]{\textbf{127.}}\parbox[t]{\dimexpr\textwidth-1.5em}{За умовами договору позичальник повинен повернути кредит протягом 10 місяців. Першого місяця він має повернути 1200 \textit{грн}, а кожного наступного місяця — на 100 \textit{грн} менше, ніж попереднього. Визначте загальну суму (у \textit{грн}), яку повинен позичальник повернути протягом 10 місяців. \nmtyear{2026}}
\vspace{0.3cm}

\answerTable{7000}{13000}{8500}{12000}{7500}

\vspace{0.5cm}

\noindent\makebox[1.5em][l]{\textbf{128.}}\parbox[t]{\dimexpr\textwidth-1.5em}{Студент вивчав мову за методикою: у перший день він запам'ятав 14 слів, а кожного наступного дня — на 5 слів більше, ніж попереднього. Скільки всього слів запам'ятав студент за 26 днів? \nmtyear{2026}}
\vspace{0.3cm}

\answerTable{1969}{364}{2009}{3614}{1989}

\vspace{0.5cm}

\noindent\makebox[1.5em][l]{\textbf{129.}}\parbox[t]{\dimexpr\textwidth-1.5em}{На рисунку зображено поперечний переріз стосу колод. У нижньому ряду стосу 4 колод, а у верхньому — 1. Визначте загальну кількість колод.
            \begin{center}
            \begin{tikzpicture}[scale=0.5]
                \newcommand{\woodLog}[3]{
                    \begin{scope}[shift={(#1,#2)}]
                        \draw[fill=woodinner, draw=black, thick] (0,0) circle (0.5);
                        \draw[woodouter!80, thin] (0,0) circle (0.35);
                        \draw[woodouter!80, thin] (0,0) circle (0.2);
                        \begin{scope}[rotate=#3]
                            \fill[woodouter] (0,0) -- (0.4, 0.05) -- (0.5, 0.1) -- (0.5, -0.1) -- (0.4, -0.05) -- cycle;
                        \end{scope}
                    \end{scope}
                }
                \def\rows{4} 
                \foreach \row in {1,...,\rows} {
                    \foreach \col in {1,...,\row} {
                        \pgfmathsetmacro{\x}{(\col-1) - (\row-1)*0.5}
                        \pgfmathsetmacro{\y}{-(\row-1)*0.866}
                        \pgfmathsetmacro{\angle}{mod(\col*70 + \row*50, 360)}
                        \woodLog{\x}{\y}{\angle}
                    }
                }
            \end{tikzpicture}
            \end{center}
             \nmtyear{2026}}
\vspace{0.3cm}

\answerTable{20}{14}{6}{10}{0}

\vspace{0.5cm}

\noindent\makebox[1.5em][l]{\textbf{130.}}\parbox[t]{\dimexpr\textwidth-1.5em}{У залі для глядачів цирку встановлено 29 рядів крісел: у першому ряду 19 крісла, а в кожному наступному ряду кількість крісел на те саме число більше, ніж у попередньому. Визначте кількість крісел у \textit{6-му} ряду, якщо в останньому ряду 103 крісла. \nmtyear{2026}}
\vspace{0.3cm}

\answerTable{40}{28}{31}{37}{34}

\vspace{0.5cm}

\noindent\makebox[1.5em][l]{\textbf{131.}}\parbox[t]{\dimexpr\textwidth-1.5em}{Студент вивчав мову за методикою: у перший день він запам'ятав 12 слів, а кожного наступного дня — на 4 слів більше, ніж попереднього. Скільки всього слів запам'ятав студент за 21 днів? \nmtyear{2026}}
\vspace{0.3cm}

\answerTable{1072}{1932}{1112}{252}{1092}

\vspace{0.5cm}

\noindent\makebox[1.5em][l]{\textbf{132.}}\parbox[t]{\dimexpr\textwidth-1.5em}{У залі для глядачів цирку встановлено 22 рядів крісел: у першому ряду 60 крісла, а в кожному наступному ряду кількість крісел на те саме число більше, ніж у попередньому. Визначте кількість крісел у \textit{2-му} ряду, якщо в останньому ряду 123 крісла. \nmtyear{2026}}
\vspace{0.3cm}

\answerTable{66}{63}{57}{69}{60}

\vspace{0.5cm}

\noindent\makebox[1.5em][l]{\textbf{133.}}\parbox[t]{\dimexpr\textwidth-1.5em}{На рисунку зображено поперечний переріз стосу колод. У нижньому ряду стосу 6 колод, а у верхньому — 1. Визначте загальну кількість колод.
            \begin{center}
            \begin{tikzpicture}[scale=0.5]
                \newcommand{\woodLog}[3]{
                    \begin{scope}[shift={(#1,#2)}]
                        \draw[fill=woodinner, draw=black, thick] (0,0) circle (0.5);
                        \draw[woodouter!80, thin] (0,0) circle (0.35);
                        \draw[woodouter!80, thin] (0,0) circle (0.2);
                        \begin{scope}[rotate=#3]
                            \fill[woodouter] (0,0) -- (0.4, 0.05) -- (0.5, 0.1) -- (0.5, -0.1) -- (0.4, -0.05) -- cycle;
                        \end{scope}
                    \end{scope}
                }
                \def\rows{6} 
                \foreach \row in {1,...,\rows} {
                    \foreach \col in {1,...,\row} {
                        \pgfmathsetmacro{\x}{(\col-1) - (\row-1)*0.5}
                        \pgfmathsetmacro{\y}{-(\row-1)*0.866}
                        \pgfmathsetmacro{\angle}{mod(\col*70 + \row*50, 360)}
                        \woodLog{\x}{\y}{\angle}
                    }
                }
            \end{tikzpicture}
            \end{center}
             \nmtyear{2026}}
\vspace{0.3cm}

\answerTable{15}{21}{27}{11}{31}

\vspace{0.5cm}

\noindent\makebox[1.5em][l]{\textbf{134.}}\parbox[t]{\dimexpr\textwidth-1.5em}{На рисунку зображено поперечний переріз стосу колод. У нижньому ряду стосу 4 колод, а у верхньому — 1. Визначте загальну кількість колод.
            \begin{center}
            \begin{tikzpicture}[scale=0.5]
                \newcommand{\woodLog}[3]{
                    \begin{scope}[shift={(#1,#2)}]
                        \draw[fill=woodinner, draw=black, thick] (0,0) circle (0.5);
                        \draw[woodouter!80, thin] (0,0) circle (0.35);
                        \draw[woodouter!80, thin] (0,0) circle (0.2);
                        \begin{scope}[rotate=#3]
                            \fill[woodouter] (0,0) -- (0.4, 0.05) -- (0.5, 0.1) -- (0.5, -0.1) -- (0.4, -0.05) -- cycle;
                        \end{scope}
                    \end{scope}
                }
                \def\rows{4} 
                \foreach \row in {1,...,\rows} {
                    \foreach \col in {1,...,\row} {
                        \pgfmathsetmacro{\x}{(\col-1) - (\row-1)*0.5}
                        \pgfmathsetmacro{\y}{-(\row-1)*0.866}
                        \pgfmathsetmacro{\angle}{mod(\col*70 + \row*50, 360)}
                        \woodLog{\x}{\y}{\angle}
                    }
                }
            \end{tikzpicture}
            \end{center}
             \nmtyear{2026}}
\vspace{0.3cm}

\answerTable{0}{10}{6}{20}{14}

\vspace{0.5cm}

\noindent\makebox[1.5em][l]{\textbf{135.}}\parbox[t]{\dimexpr\textwidth-1.5em}{За умовами договору позичальник повинен повернути кредит протягом 24 місяців. Першого місяця він має повернути 880 \textit{грн}, а кожного наступного місяця — на 20 \textit{грн} менше, ніж попереднього. Визначте загальну суму (у \textit{грн}), яку повинен позичальник повернути протягом 24 місяців. \nmtyear{2026}}
\vspace{0.3cm}

\answerTable{16600}{21600}{21120}{15100}{15600}

\vspace{0.5cm}

\noindent\makebox[1.5em][l]{\textbf{136.}}\parbox[t]{\dimexpr\textwidth-1.5em}{За умовами договору позичальник повинен повернути кредит протягом 24 місяців. Першого місяця він має повернути 440 \textit{грн}, а кожного наступного місяця — на 10 \textit{грн} менше, ніж попереднього. Визначте загальну суму (у \textit{грн}), яку повинен позичальник повернути протягом 24 місяців. \nmtyear{2026}}
\vspace{0.3cm}

\answerTable{7300}{10800}{7800}{8800}{10560}

\vspace{0.5cm}

\noindent\makebox[1.5em][l]{\textbf{137.}}\parbox[t]{\dimexpr\textwidth-1.5em}{За умовами договору позичальник повинен повернути кредит протягом 6 місяців. Першого місяця він має повернути 1200 \textit{грн}, а кожного наступного місяця — на 100 \textit{грн} менше, ніж попереднього. Визначте загальну суму (у \textit{грн}), яку повинен позичальник повернути протягом 6 місяців. \nmtyear{2026}}
\vspace{0.3cm}

\answerTable{7800}{5200}{6700}{7200}{5700}

\vspace{0.5cm}

\noindent\makebox[1.5em][l]{\textbf{138.}}\parbox[t]{\dimexpr\textwidth-1.5em}{Студент вивчав мову за методикою: у перший день він запам'ятав 10 слів, а кожного наступного дня — на 4 слів більше, ніж попереднього. Скільки всього слів запам'ятав студент за 25 днів? \nmtyear{2026}}
\vspace{0.3cm}

\answerTable{1470}{250}{1430}{1450}{2650}

\vspace{0.5cm}

\noindent\makebox[1.5em][l]{\textbf{139.}}\parbox[t]{\dimexpr\textwidth-1.5em}{На рисунку зображено поперечний переріз стосу колод. У нижньому ряду стосу 8 колод, а у верхньому — 1. Визначте загальну кількість колод.
            \begin{center}
            \begin{tikzpicture}[scale=0.5]
                \newcommand{\woodLog}[3]{
                    \begin{scope}[shift={(#1,#2)}]
                        \draw[fill=woodinner, draw=black, thick] (0,0) circle (0.5);
                        \draw[woodouter!80, thin] (0,0) circle (0.35);
                        \draw[woodouter!80, thin] (0,0) circle (0.2);
                        \begin{scope}[rotate=#3]
                            \fill[woodouter] (0,0) -- (0.4, 0.05) -- (0.5, 0.1) -- (0.5, -0.1) -- (0.4, -0.05) -- cycle;
                        \end{scope}
                    \end{scope}
                }
                \def\rows{8} 
                \foreach \row in {1,...,\rows} {
                    \foreach \col in {1,...,\row} {
                        \pgfmathsetmacro{\x}{(\col-1) - (\row-1)*0.5}
                        \pgfmathsetmacro{\y}{-(\row-1)*0.866}
                        \pgfmathsetmacro{\angle}{mod(\col*70 + \row*50, 360)}
                        \woodLog{\x}{\y}{\angle}
                    }
                }
            \end{tikzpicture}
            \end{center}
             \nmtyear{2026}}
\vspace{0.3cm}

\answerTable{28}{46}{26}{44}{36}

\vspace{0.5cm}

\noindent\makebox[1.5em][l]{\textbf{140.}}\parbox[t]{\dimexpr\textwidth-1.5em}{У залі для глядачів цирку встановлено 25 рядів крісел: у першому ряду 53 крісла, а в кожному наступному ряду кількість крісел на те саме число більше, ніж у попередньому. Визначте кількість крісел у \textit{18-му} ряду, якщо в останньому ряду 149 крісла. \nmtyear{2026}}
\vspace{0.3cm}

\answerTable{129}{121}{125}{113}{117}

\vspace{0.5cm}


\end{document}
