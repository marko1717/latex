\documentclass[14pt]{extarticle}
\usepackage{fontspec}
\usepackage{polyglossia}
\setdefaultlanguage{ukrainian}

\defaultfontfeatures{Ligatures=TeX}
\setmainfont{Liberation Serif}
\setsansfont{Liberation Sans}
\setmonofont{Liberation Mono}

\usepackage[a4paper,margin=1.5cm,bottom=2cm,top=2cm]{geometry}
\usepackage{amsmath,amssymb}
\usepackage{enumitem}
\usepackage{tikz}
\usepackage{pgfplots}
\pgfplotsset{compat=1.16}

\usetikzlibrary{calc,patterns,angles,quotes,intersections,babel}
\usetikzlibrary{3d}
\definecolor{woodinner}{RGB}{222, 184, 135}
\definecolor{woodouter}{RGB}{139, 69, 19}
\usepackage{xcolor}
\usepackage{array}
\usepackage{fancyhdr}
\usepackage{multirow}

\definecolor{headerblue}{RGB}{0, 102, 204}
\definecolor{yearcolor}{RGB}{128, 0, 128}

\pagestyle{fancy}
\fancyhf{}
\renewcommand{\headrulewidth}{0pt}
\fancyfoot[C]{\thepage}

\setlength{\headheight}{15pt}
\setlength{\headsep}{10pt}
\setlength{\footskip}{25pt}

\widowpenalty=10000
\clubpenalty=10000

\newcommand{\answerTable}[5]{
\begin{center}
\begin{tabular}{|*{5}{>{\centering\arraybackslash}m{2.8cm}|}}
\hline
\rule[-0.3cm]{0pt}{0.8cm}\textbf{А} & \textbf{Б} & \textbf{В} & \textbf{Г} & \textbf{Д} \\
\hline
\rule[-0.4cm]{0pt}{1.0cm}#1 & \rule[-0.4cm]{0pt}{1.0cm}#2 & \rule[-0.4cm]{0pt}{1.0cm}#3 & \rule[-0.4cm]{0pt}{1.0cm}#4 & \rule[-0.4cm]{0pt}{1.0cm}#5 \\
\hline
\end{tabular}
\end{center}
}

\newcommand{\shortAnswer}{
\vspace{0.3cm}
\noindent\hspace{1cm}Відповідь: \framebox(18,18){}\framebox(18,18){}\framebox(18,18){}\framebox(18,18){}{,}\framebox(18,18){}\framebox(18,18){}\framebox(18,18){}
\vspace{0.5cm}
}

\newcommand{\nmtyear}[1]{\hfill{\small\color{yearcolor}(AI Gen)}}

\begin{document}

\begin{center}
{\Large\textbf{\color{headerblue}ЗГЕНЕРОВАНІ ЗАВДАННЯ (AI)}}
\end{center}

\begin{center}
{\large Тема: \textbf{Арифметична прогресія}}
\end{center}

\vspace{0.5cm}
% === Arithmetic Progression: Find d ===
\noindent\makebox[1.5em][l]{\textbf{1.}}\parbox[t]{\dimexpr\textwidth-1.5em}{В арифметичній прогресії $(a_n)$: $a_1 = 9$, $a_3 = -1$. Визначте різницю $d$ прогресії. \nmtyear{2026}}

\answerTable{$d = -5$}{$d = 5$}{$d = -10$}{$d = -4$}{$d = 4$}

\vspace{0.5cm}

\noindent\makebox[1.5em][l]{\textbf{2.}}\parbox[t]{\dimexpr\textwidth-1.5em}{В арифметичній прогресії $(a_n)$: $a_1 = 6$, $a_3 = 14$. Визначте різницю $d$ прогресії. \nmtyear{2026}}

\answerTable{$d = 5$}{$d = -4$}{$d = 8$}{$d = 4$}{$d = 10$}

\vspace{0.5cm}

\noindent\makebox[1.5em][l]{\textbf{3.}}\parbox[t]{\dimexpr\textwidth-1.5em}{В арифметичній прогресії $(a_n)$: $a_1 = 3$, $a_3 = -16$. Визначте різницю $d$ прогресії. \nmtyear{2026}}

\answerTable{$d = -6{,}5$}{$d = -9{,}5$}{$d = -8{,}5$}{$d = -19$}{$d = 9{,}5$}

\vspace{0.5cm}

\noindent\makebox[1.5em][l]{\textbf{4.}}\parbox[t]{\dimexpr\textwidth-1.5em}{В арифметичній прогресії $(a_n)$: $a_1 = 1$, $a_3 = -19$. Визначте різницю $d$ прогресії. \nmtyear{2026}}

\answerTable{$d = -9$}{50}{$d = 10$}{$d = -20$}{$d = -10$}

\vspace{0.5cm}

\noindent\makebox[1.5em][l]{\textbf{5.}}\parbox[t]{\dimexpr\textwidth-1.5em}{В арифметичній прогресії $(a_n)$: $a_1 = 1$, $a_3 = -15$. Визначте різницю $d$ прогресії. \nmtyear{2026}}

\answerTable{51}{$d = -16$}{$d = -8$}{$d = 8$}{$d = -7$}

\vspace{0.5cm}

% === Arithmetic Progression: Member Difference ===
\noindent\makebox[1.5em][l]{\textbf{6.}}\parbox[t]{\dimexpr\textwidth-1.5em}{В арифметичній прогресії $(a_n)$ відомо, що $a_{7} - a_{3} = 40$. Знайдіть значення виразу $a_{6} - a_{7}$. \nmtyear{2026}}

\answerTable{-10}{-20}{10}{0}{40}

\vspace{0.5cm}

\noindent\makebox[1.5em][l]{\textbf{7.}}\parbox[t]{\dimexpr\textwidth-1.5em}{В арифметичній прогресії $(a_n)$ відомо, що $a_{8} - a_{4} = 24$. Знайдіть значення виразу $a_{5} - a_{3}$. \nmtyear{2026}}

\answerTable{-12}{12}{18}{24}{6}

\vspace{0.5cm}

\noindent\makebox[1.5em][l]{\textbf{8.}}\parbox[t]{\dimexpr\textwidth-1.5em}{В арифметичній прогресії $(a_n)$ відомо, що $a_{8} - a_{5} = 12$. Знайдіть значення виразу $a_{10} - a_{12}$. \nmtyear{2026}}

\answerTable{-8}{-4}{-12}{8}{12}

\vspace{0.5cm}

\noindent\makebox[1.5em][l]{\textbf{9.}}\parbox[t]{\dimexpr\textwidth-1.5em}{В арифметичній прогресії $(a_n)$ відомо, що $a_{4} - a_{2} = -6$. Знайдіть значення виразу $a_{3} - a_{7}$. \nmtyear{2026}}

\answerTable{-6}{15}{12}{9}{-12}

\vspace{0.5cm}

\noindent\makebox[1.5em][l]{\textbf{10.}}\parbox[t]{\dimexpr\textwidth-1.5em}{В арифметичній прогресії $(a_n)$ відомо, що $a_{7} - a_{1} = 60$. Знайдіть значення виразу $a_{6} - a_{9}$. \nmtyear{2026}}

\answerTable{-30}{-20}{60}{30}{-40}

\vspace{0.5cm}

% === Arithmetic Progression: Sum ===
\noindent\makebox[1.5em][l]{\textbf{11.}}\parbox[t]{\dimexpr\textwidth-1.5em}{Обчисліть суму перших 16-ти членів арифметичної прогресії $(a_n)$, якщо $a_1 + a_{16} = 42$. \nmtyear{2026}}

\answerTable{352}{336}{42}{-336}{672}

\vspace{0.5cm}

\noindent\makebox[1.5em][l]{\textbf{12.}}\parbox[t]{\dimexpr\textwidth-1.5em}{Обчисліть суму перших 16-ти членів арифметичної прогресії $(a_n)$, якщо $a_1 + a_{16} = 46$. \nmtyear{2026}}

\answerTable{46}{368}{-368}{352}{384}

\vspace{0.5cm}

\noindent\makebox[1.5em][l]{\textbf{13.}}\parbox[t]{\dimexpr\textwidth-1.5em}{Обчисліть суму перших 20-ти членів арифметичної прогресії $(a_n)$, якщо $a_1 + a_{20} = -39$. \nmtyear{2026}}

\answerTable{-410}{-780}{-39}{390}{-390}

\vspace{0.5cm}

\noindent\makebox[1.5em][l]{\textbf{14.}}\parbox[t]{\dimexpr\textwidth-1.5em}{Обчисліть суму перших 20-ти членів арифметичної прогресії $(a_n)$, якщо $a_1 + a_{20} = 47$. \nmtyear{2026}}

\answerTable{47}{940}{490}{470}{450}

\vspace{0.5cm}

\noindent\makebox[1.5em][l]{\textbf{15.}}\parbox[t]{\dimexpr\textwidth-1.5em}{Обчисліть суму перших 16-ти членів арифметичної прогресії $(a_n)$, якщо $a_1 + a_{16} = -19$. \nmtyear{2026}}

\answerTable{-19}{-304}{-168}{-136}{-152}

\vspace{0.5cm}

% === Arithmetic Progression: Count Terms ===
\noindent\makebox[1.5em][l]{\textbf{16.}}\parbox[t]{\dimexpr\textwidth-1.5em}{В арифметичній прогресії $(a_n)$ перший член $a_1 = 3{,}1$, різниця $d = -0{,}4$. Скільки всього \textit{додатних} членів має ця прогресія? \nmtyear{2026}}

\answerTable{7}{6}{10}{9}{8}

\vspace{0.5cm}

\noindent\makebox[1.5em][l]{\textbf{17.}}\parbox[t]{\dimexpr\textwidth-1.5em}{В арифметичній прогресії $(a_n)$ перший член $a_1 = -5{,}5$, різниця $d = 0{,}5$. Скільки всього \textit{від'ємних} членів має ця прогресія? \nmtyear{2026}}

\answerTable{11}{9}{10}{8}{12}

\vspace{0.5cm}

\noindent\makebox[1.5em][l]{\textbf{18.}}\parbox[t]{\dimexpr\textwidth-1.5em}{В арифметичній прогресії $(a_n)$ перший член $a_1 = 12{,}8$, різниця $d = -2{,}5$. Скільки всього \textit{додатних} членів має ця прогресія? \nmtyear{2026}}

\answerTable{4}{8}{6}{7}{5}

\vspace{0.5cm}

\noindent\makebox[1.5em][l]{\textbf{19.}}\parbox[t]{\dimexpr\textwidth-1.5em}{В арифметичній прогресії $(a_n)$ перший член $a_1 = -12{,}6$, різниця $d = 2{,}5$. Скільки всього \textit{від'ємних} членів має ця прогресія? \nmtyear{2026}}

\answerTable{7}{6}{8}{5}{4}

\vspace{0.5cm}

\noindent\makebox[1.5em][l]{\textbf{20.}}\parbox[t]{\dimexpr\textwidth-1.5em}{В арифметичній прогресії $(a_n)$ перший член $a_1 = -7{,}7$, різниця $d = 0{,}8$. Скільки всього \textit{від'ємних} членів має ця прогресія? \nmtyear{2026}}

\answerTable{9}{11}{12}{8}{10}

\vspace{0.5cm}

% === Arithmetic Progression: Middle Term ===
\noindent\makebox[1.5em][l]{\textbf{21.}}\parbox[t]{\dimexpr\textwidth-1.5em}{Визначте 5-й член $a_{5}$ арифметичної прогресії $(a_n)$, у якої $a_{4} = 31$, $a_{6} = 29$. \nmtyear{2026}}

\answerTable{2}{31}{-1}{30}{29}

\vspace{0.5cm}

\noindent\makebox[1.5em][l]{\textbf{22.}}\parbox[t]{\dimexpr\textwidth-1.5em}{Визначте 19-й член $a_{19}$ арифметичної прогресії $(a_n)$, у якої $a_{18} = 25$, $a_{20} = 31$. \nmtyear{2026}}

\answerTable{3}{28}{25}{6}{31}

\vspace{0.5cm}

\noindent\makebox[1.5em][l]{\textbf{23.}}\parbox[t]{\dimexpr\textwidth-1.5em}{Визначте 20-й член $a_{20}$ арифметичної прогресії $(a_n)$, у якої $a_{19} = -1$, $a_{21} = -5$. \nmtyear{2026}}

\answerTable{4}{-5}{-3}{-1}{-2}

\vspace{0.5cm}

\noindent\makebox[1.5em][l]{\textbf{24.}}\parbox[t]{\dimexpr\textwidth-1.5em}{Визначте 19-й член $a_{19}$ арифметичної прогресії $(a_n)$, у якої $a_{18} = 39$, $a_{20} = 37$. \nmtyear{2026}}

\answerTable{37}{76}{39}{-1}{38}

\vspace{0.5cm}

\noindent\makebox[1.5em][l]{\textbf{25.}}\parbox[t]{\dimexpr\textwidth-1.5em}{Визначте 17-й член $a_{17}$ арифметичної прогресії $(a_n)$, у якої $a_{16} = 12$, $a_{18} = 16$. \nmtyear{2026}}

\answerTable{4}{12}{14}{2}{16}

\vspace{0.5cm}

% === Arithmetic Progression: Formula Search ===
\noindent\makebox[1.5em][l]{\textbf{26.}}\parbox[t]{\dimexpr\textwidth-1.5em}{Арифметичну прогресію $(a_n)$ задано формулою $n$-го члена $a_n = 39  -4n$. Визначте номер члена, значення якого дорівнює $-73$. \nmtyear{2026}}

\answerTable{18{,}25}{29}{28}{18}{27}

\vspace{0.5cm}

\noindent\makebox[1.5em][l]{\textbf{27.}}\parbox[t]{\dimexpr\textwidth-1.5em}{Арифметичну прогресію $(a_n)$ задано формулою $n$-го члена $a_n = 30  -3n$. Визначте номер члена, значення якого дорівнює $-78$. \nmtyear{2026}}

\answerTable{26}{46}{35}{36}{37}

\vspace{0.5cm}

\noindent\makebox[1.5em][l]{\textbf{28.}}\parbox[t]{\dimexpr\textwidth-1.5em}{Арифметичну прогресію $(a_n)$ задано формулою $n$-го члена $a_n = 42  -4n$. Визначте номер члена, значення якого дорівнює $-74$. \nmtyear{2026}}

\answerTable{29}{18{,}5}{28}{39}{30}

\vspace{0.5cm}

\noindent\makebox[1.5em][l]{\textbf{29.}}\parbox[t]{\dimexpr\textwidth-1.5em}{Арифметичну прогресію $(a_n)$ задано формулою $n$-го члена $a_n = 45 + 2{,}5n$. Визначте номер члена, значення якого дорівнює $132{,}5$. \nmtyear{2026}}

\answerTable{36}{25}{35}{45}{34}

\vspace{0.5cm}

\noindent\makebox[1.5em][l]{\textbf{30.}}\parbox[t]{\dimexpr\textwidth-1.5em}{Арифметичну прогресію $(a_n)$ задано формулою $n$-го члена $a_n = 18  -3n$. Визначте номер члена, значення якого дорівнює $-27$. \nmtyear{2026}}

\answerTable{25}{5}{14}{15}{9}

\vspace{0.5cm}

% === Arithmetic Progression: Word Problem ===
\noindent\makebox[1.5em][l]{\textbf{31.}}\parbox[t]{\dimexpr\textwidth-1.5em}{За умовами договору позичальник повинен повернути кредит протягом 12 місяців. Першого місяця він має повернути 1300 \textit{грн}, а кожного наступного місяця — на 50 \textit{грн} менше, ніж попереднього. Визначте загальну суму (у \textit{грн}), яку повинен позичальник повернути протягом 12 місяців. \nmtyear{2026}}

\answerTable{16200}{15600}{13300}{11800}{12300}

\vspace{0.5cm}

\noindent\makebox[1.5em][l]{\textbf{32.}}\parbox[t]{\dimexpr\textwidth-1.5em}{За умовами договору позичальник повинен повернути кредит протягом 10 місяців. Першого місяця він має повернути 700 \textit{грн}, а кожного наступного місяця — на 50 \textit{грн} менше, ніж попереднього. Визначте загальну суму (у \textit{грн}), яку повинен позичальник повернути протягом 10 місяців. \nmtyear{2026}}

\answerTable{7500}{4750}{4250}{7000}{5750}

\vspace{0.5cm}

\noindent\makebox[1.5em][l]{\textbf{33.}}\parbox[t]{\dimexpr\textwidth-1.5em}{У залі для глядачів цирку встановлено 11 рядів крісел: у першому ряду 27 крісла, а в кожному наступному ряду кількість крісел на те саме число більше, ніж у попередньому. Визначте кількість крісел у \textit{8-му} ряду, якщо в останньому ряду 67 крісла. \nmtyear{2026}}

\answerTable{59}{51}{63}{47}{55}

\vspace{0.5cm}

\noindent\makebox[1.5em][l]{\textbf{34.}}\parbox[t]{\dimexpr\textwidth-1.5em}{На рисунку зображено поперечний переріз стосу колод. У нижньому ряду стосу 4 колод, а у верхньому — 1. Визначте загальну кількість колод.
            \begin{center}
            \begin{tikzpicture}[scale=0.5]
                \newcommand{\woodLog}[3]{
                    \begin{scope}[shift={(#1,#2)}]
                        \draw[fill=woodinner, draw=black, thick] (0,0) circle (0.5);
                        \draw[woodouter!80, thin] (0,0) circle (0.35);
                        \draw[woodouter!80, thin] (0,0) circle (0.2);
                        \begin{scope}[rotate=#3]
                            \fill[woodouter] (0,0) -- (0.4, 0.05) -- (0.5, 0.1) -- (0.5, -0.1) -- (0.4, -0.05) -- cycle;
                        \end{scope}
                    \end{scope}
                }
                \def\rows{4} 
                \foreach \row in {1,...,\rows} {
                    \foreach \col in {1,...,\row} {
                        \pgfmathsetmacro{\x}{(\col-1) - (\row-1)*0.5}
                        \pgfmathsetmacro{\y}{-(\row-1)*0.866}
                        \pgfmathsetmacro{\angle}{mod(\col*70 + \row*50, 360)}
                        \woodLog{\x}{\y}{\angle}
                    }
                }
            \end{tikzpicture}
            \end{center}
             \nmtyear{2026}}

\answerTable{0}{14}{10}{6}{20}

\vspace{0.5cm}

\noindent\makebox[1.5em][l]{\textbf{35.}}\parbox[t]{\dimexpr\textwidth-1.5em}{На рисунку зображено поперечний переріз стосу колод. У нижньому ряду стосу 3 колод, а у верхньому — 1. Визначте загальну кількість колод.
            \begin{center}
            \begin{tikzpicture}[scale=0.5]
                \newcommand{\woodLog}[3]{
                    \begin{scope}[shift={(#1,#2)}]
                        \draw[fill=woodinner, draw=black, thick] (0,0) circle (0.5);
                        \draw[woodouter!80, thin] (0,0) circle (0.35);
                        \draw[woodouter!80, thin] (0,0) circle (0.2);
                        \begin{scope}[rotate=#3]
                            \fill[woodouter] (0,0) -- (0.4, 0.05) -- (0.5, 0.1) -- (0.5, -0.1) -- (0.4, -0.05) -- cycle;
                        \end{scope}
                    \end{scope}
                }
                \def\rows{3} 
                \foreach \row in {1,...,\rows} {
                    \foreach \col in {1,...,\row} {
                        \pgfmathsetmacro{\x}{(\col-1) - (\row-1)*0.5}
                        \pgfmathsetmacro{\y}{-(\row-1)*0.866}
                        \pgfmathsetmacro{\angle}{mod(\col*70 + \row*50, 360)}
                        \woodLog{\x}{\y}{\angle}
                    }
                }
            \end{tikzpicture}
            \end{center}
             \nmtyear{2026}}

\answerTable{6}{9}{-4}{3}{16}

\vspace{0.5cm}


\end{document}
