\documentclass[14pt]{extarticle}
\usepackage{fontspec}
\usepackage{polyglossia}
\setdefaultlanguage{ukrainian}

\defaultfontfeatures{Ligatures=TeX}
\setmainfont{Liberation Serif}
\setsansfont{Liberation Sans}
\setmonofont{Liberation Mono}

\usepackage[a4paper,margin=1.5cm,bottom=2cm,top=2cm]{geometry}
\usepackage{amsmath,amssymb}
\usepackage{enumitem}
\usepackage{tikz}
\usepackage{pgfplots}
\pgfplotsset{compat=1.16}

% Підключаємо бібліотеки для зручних кутів
\usetikzlibrary{calc,patterns,angles,quotes,intersections,babel}

\usepackage{xcolor}
\usepackage{array}
\usepackage{fancyhdr}
\usepackage{multirow}

% Кольори
\definecolor{headerblue}{RGB}{0, 102, 204}
\definecolor{yearcolor}{RGB}{128, 0, 128}

\pagestyle{fancy}
\fancyhf{}
\renewcommand{\headrulewidth}{0pt}
\fancyfoot[C]{\thepage}

\setlength{\headheight}{15pt}
\setlength{\headsep}{10pt}
\setlength{\footskip}{25pt}

\widowpenalty=10000
\clubpenalty=10000

% === КОМАНДИ ===

% Таблиця для відповідей із дробами (збільшена висота клітинок)
% Оновлена таблиця: підпорка додана до КОЖНОЇ клітинки
\newcommand{\answerTableTall}[5]{
\begin{center}
\begin{tabular}{|*{5}{>{\centering\arraybackslash}m{2.8cm}|}}
\hline
\rule[-0.3cm]{0pt}{0.8cm}\textbf{А} & \textbf{Б} & \textbf{В} & \textbf{Г} & \textbf{Д} \\
\hline
% Тепер rule є перед кожним аргументом (#1..#5)
\rule[-0.9cm]{0pt}{2.0cm}#1 & 
\rule[-0.9cm]{0pt}{2.0cm}#2 & 
\rule[-0.9cm]{0pt}{2.0cm}#3 & 
\rule[-0.9cm]{0pt}{2.0cm}#4 & 
\rule[-0.9cm]{0pt}{2.0cm}#5 \\
\hline
\end{tabular}
\end{center}
}

% Оновлена таблиця відповідей (заголовки зовні)
\newcommand{\answerGrid}{
    \begingroup
    % Збільшуємо висоту рядків для квадратних клітинок
    \renewcommand{\arraystretch}{1.3} 
    % Відступ всередині клітинок
    \setlength{\tabcolsep}{7pt} 
    \begin{tabular}{r|c|c|c|c|c|}
         % Перший рядок: порожня клітинка зліва + букви без рамок (multicolumn прибирає |)
         \multicolumn{1}{c}{} & \multicolumn{1}{c}{\textbf{А}} & \multicolumn{1}{c}{\textbf{Б}} & \multicolumn{1}{c}{\textbf{В}} & \multicolumn{1}{c}{\textbf{Г}} & \multicolumn{1}{c}{\textbf{Д}} \\ \cline{2-6}
         % Наступні рядки: номер зліва (r) + клітинки з рамками (|c|)
         \textbf{1} & & & & & \\ \cline{2-6}
         \textbf{2} & & & & & \\ \cline{2-6}
         \textbf{3} & & & & & \\ \cline{2-6}
    \end{tabular}
    \endgroup
}

% Макет для завдань на відповідність
% #1 - Умови (1-3)
% #2 - Варіанти (А-Д)
% #3 - Табличка
\newcommand{\matchingLayout}[3]{
    \noindent
    \begin{minipage}[t]{0.40\textwidth}
       
        #1
    \end{minipage}%
    \hfill
    \begin{minipage}[t]{0.28\textwidth}
        
        #2
    \end{minipage}%
    \hfill
    \begin{minipage}[t]{0.30\textwidth}
        \vspace{0pt} % Хаки для вирівнювання minipage по верху
        \begin{flushright}
        #3
        \end{flushright}
    \end{minipage}
}

% Стандартна таблиця відповідей (для тестів)
\newcommand{\answerTableSmall}[5]{
\begin{tabular}{|*{5}{>{\centering\arraybackslash}m{1.65cm}|}}
\hline
\rule[-0.2cm]{0pt}{0.6cm}\textbf{А} & \textbf{Б} & \textbf{В} & \textbf{Г} & \textbf{Д} \\
\hline
% Підпорка додана до кожного варіанту для ідеального вирівнювання
\rule[-0.4cm]{0pt}{0.9cm}#1 & 
\rule[-0.4cm]{0pt}{0.9cm}#2 & 
\rule[-0.4cm]{0pt}{0.9cm}#3 & 
\rule[-0.4cm]{0pt}{0.9cm}#4 & 
\rule[-0.4cm]{0pt}{0.9cm}#5 \\
\hline
\end{tabular}
}

% Таблиця для вибору одного варіанту (Task 7)
\newcommand{\answerTable}[5]{
\begin{center}
\begin{tabular}{|*{5}{>{\centering\arraybackslash}m{2.8cm}|}}
\hline
\rule[-0.3cm]{0pt}{0.8cm}\textbf{А} & \textbf{Б} & \textbf{В} & \textbf{Г} & \textbf{Д} \\
\hline
\rule[-0.4cm]{0pt}{1.0cm}#1 & \rule[-0.4cm]{0pt}{1.0cm}#2 & \rule[-0.4cm]{0pt}{1.0cm}#3 & \rule[-0.4cm]{0pt}{1.0cm}#4 & \rule[-0.4cm]{0pt}{1.0cm}#5 \\
\hline
\end{tabular}
\end{center}
}

% Команда для року
\newcommand{\nmtyear}[1]{\hfill{\small\color{yearcolor}(НМТ #1)}}

\begin{document}

\begin{center}
{\Large\textbf{\color{headerblue}БАЗА ЗАВДАНЬ НМТ 2023}}
\end{center}

\begin{center}
{\large Тема: \textbf{Коло, круг та їх елементи }}
\end{center}

\vspace{0.5cm}

\begin{center}
{\Large\textbf{\color{headerblue}БАЗА ЗАВДАНЬ НМТ 2023}}
\end{center}

\begin{center}
{\large Тема: \textbf{Коло і круг}}
\end{center}

\vspace{0.5cm}

% === ЗАВДАННЯ 1 ===
\noindent\textbf{1.} \begin{minipage}[t]{0.55\textwidth}
Рівнобедрений трикутник $ABC$ ($AB=BC$) уписано в коло (див. рисунок). Визначте градусну міру меншої дуги $AB$, якщо $\angle ABC = 20^\circ$. \nmtyear{2023}
\end{minipage}
\hfill
\begin{minipage}[t]{0.4\textwidth}
    \vspace{-0.5cm}
    \begin{flushright}
    \begin{tikzpicture}[scale=0.9]
        \coordinate (O) at (0,0);
        \draw[thick] (O) circle (2.5cm);
        
        % Triangle vertices
        % Angle B is 20 deg. Triangle is isosceles (AB=BC).
        % Arc AC = 2 * 20 = 40 deg.
        % Remaining arc = 320. Arc AB = Arc BC = 160.
        % Coordinates: B at top (90). A at 90+160 = 250. C at 90-160 = -70 (or 290).
        \coordinate (B) at (90:2.5);
        \coordinate (A) at (220:2.5);
        \coordinate (C) at (320:2.5);
        
        \draw[thick] (A) -- (B) -- (C) -- (A);
        
        % Angle 20
        \pic [draw, pic text={\scriptsize $20^\circ$}, angle radius=0.7cm, angle eccentricity=1.9] {angle = A--B--C};
        
        % Marks for AB=BC
        \draw[thick] ($(A)!0.5!(B)$) ++(160:0.1) -- ++(-20:0.2);
        \draw[thick] ($(B)!0.5!(C)$) ++(20:0.1) -- ++(-160:0.2);
        
        \node[below left] at (A) {$A$};
        \node[above] at (B) {$B$};
        \node[below right] at (C) {$C$};
        
        \fill (A) circle (1.5pt);
        \fill (B) circle (1.5pt);
        \fill (C) circle (1.5pt);
    \end{tikzpicture}
    \end{flushright}
\end{minipage}

\vspace{0.3cm}
\answerTable{$160^\circ$}{$140^\circ$}{$80^\circ$}{$70^\circ$}{$170^\circ$}

\vspace{0.7cm}

% === ЗАВДАННЯ 2 ===
\noindent\textbf{2.} \begin{minipage}[t]{0.55\textwidth}
До кола із центром у точці $O$ проведено дотичну $AB$, яка дотикається кола в точці $A$. Пряма $OB$ перетинає коло в точках $D$ і $C$, $\angle AOB = 60^\circ$. Які з наведених тверджень є правильними?
\begin{enumerate}[label=\Roman*., nosep, leftmargin=*]
    \item $OA \perp AB$.
    \item $OB = 2OA$.
    \item $OD = AC$.
\end{enumerate}
\hfill \nmtyear{2023}
\end{minipage}
\hfill
\begin{minipage}[t]{0.4\textwidth}
    \vspace{-0.5cm}
    \begin{flushright}
    \begin{tikzpicture}[scale=0.7]
        \coordinate (O) at (0,0);
        \draw[thick] (O) circle (1.5cm);
        
        % A at top
        \coordinate (A) at (0,1.5);
        % Tangent horizontal
        \draw[thick] (-2,1.5) -- (2.5,1.5);
        
        % Angle 60. B is on tangent.
        % B is at (1.5/tan(30), 1.5) = (1.5 * 1.732, 1.5) approx (2.6, 1.5).
        % Let's verify angle AOB is 60. Slope OA vertical. Slope OB: dx = 2.6, dy = 1.5.
        % Tan(theta) = x/y = tan(60). Correct.
        \coordinate (B) at ({1.5*tan(60)}, 1.5); 
        
        % Secant line OB
        \draw[thick] ($(O)!-0.4!(B)$) -- (B);
        
        \draw[thick] (O) -- (A);
        
        % Points C and D on circle along line OB
        \coordinate (C) at (intersection of O--B and O circle 1.5cm); % This finds intersection, but tricky with syntax.
        % Calculate manually. Angle is 90-60=30 from horizontal? No, A is (0,1.5).
        % Angle of OB from vertical is 60. From horizontal is 30.
        \coordinate (C) at (30:1.5);
        \coordinate (D) at (210:1.5);
        \draw[thick] (O) -- (D);
        \draw[thick] (A) -- (C);
        % Angle 60 at O (AOB)
        \pic [draw, pic text={\small $60^\circ$}, angle radius=0.4cm, angle eccentricity=1.7] {angle = C--O--A};
        
        \node[below] at (O) {$O$};
        \node[above] at (A) {$A$};
        \node[above] at (B) {$B$};
        \node[right] at (C) {$C$};
        \node[below left] at (D) {$D$};
        
        \fill (O) circle (1.5pt);
        \fill (A) circle (1.5pt);
        \fill (B) circle (1.5pt);
        \fill (C) circle (1.5pt);
        \fill (D) circle (1.5pt);
        
    \end{tikzpicture}
    \end{flushright}
\end{minipage}

\vspace{0.3cm}
\answerTable{лише II та III}{лише III}{лише I та II}{I, II та III}{лише I та III}

\vspace{0.7cm}

% === ЗАВДАННЯ 3 ===
\noindent\makebox[1.5em][l]{\textbf{3.}}\parbox[t]{\dimexpr\textwidth-1.5em}{Які з наведених тверджень є правильними?
\begin{enumerate}[label=\Roman*., nosep, leftmargin=*]
    \item Медіана трикутника з'єднує його вершину із серединою протилежної сторони.
    \item Точка перетину медіан трикутника є центром кола, вписаного в цей трикутник.
    \item У прямокутному трикутнику одна з його медіан дорівнює половині гіпотенузи.
\end{enumerate}
\hfill \nmtyear{2023}}

\vspace{0.3cm}
\answerTable{лише I}{лише I та II}{I, II та III}{лише III}{лише I та III}

\vspace{0.7cm}

% === ЗАВДАННЯ 4 ===
\noindent\textbf{4.} \begin{minipage}[t]{0.55\textwidth}
На колі вибрано точки $A$, $B$ і $C$ так, що $\angle ACB = 30^\circ$ (див. рисунок). Визначте довжину цього кола, якщо довжина меншої дуги $AB$ дорівнює $25$ \textit{см}. \nmtyear{2023}
\end{minipage}
\hfill
\begin{minipage}[t]{0.4\textwidth}
    \vspace{-0.5cm}
    \begin{flushright}
    \begin{tikzpicture}[scale=0.9]
        \coordinate (O) at (0,0);
        \draw[thick] (O) circle (1.5cm);
        
        % Arc AB corresponds to inscribed angle 30 -> Central angle 60.
        % Let C be at 180 (left).
        % A at 30, B at -30 (so arc is 60).
        % ACB angle: A(-150?), C(180), B(-30?)
        % Let's place C at 180. A at 60, B at 0. Arc AB = 60. Inscribed ACB = 30.
        \coordinate (C) at (180:1.5);
        \coordinate (A) at (60:1.5);
        \coordinate (B) at (0:1.5);
        
        \draw[thick] (A) -- (C) -- (B);
        % Bold arc AB
        \draw[very thick] (1.5,0) arc (0:60:1.5);
        
        \pic [draw, pic text={\small $30^\circ$}, angle radius=0.4cm, angle eccentricity=2.4] {angle = B--C--A};
        
        \node[above] at (A) {$A$};
        \node[right] at (B) {$B$};
        \node[left] at (C) {$C$};
        
        \fill (A) circle (1.5pt);
        \fill (B) circle (1.5pt);
        \fill (C) circle (1.5pt);
    \end{tikzpicture}
    \end{flushright}
\end{minipage}

\vspace{0.3cm}
\answerTable{$300$ \textit{см}}{$150$ \textit{см}}{$250$ \textit{см}}{$200$ \textit{см}}{$75$ \textit{см}}

\vspace{0.7cm}

% === ЗАВДАННЯ 5 ===
\noindent\makebox[1.5em][l]{\textbf{5.}}\parbox[t]{\dimexpr\textwidth-1.5em}{Які з наведених тверджень є правильними?
\begin{enumerate}[label=\Roman*., nosep, leftmargin=*]
    \item Існує ромб, навколо якого можна описати коло.
    \item Середини сторін будь-якого ромба лежать на вписаному в нього колі.
    \item Радіус кола, уписаного в ромб, удвічі менший за його висоту.
\end{enumerate}
\hfill \nmtyear{2023}}

\vspace{0.3cm}
\answerTable{лише I та III}{лише I та II}{лише I}{лише II та III}{I, II та III}

\vspace{0.7cm}

% === ЗАВДАННЯ 6 ===
\noindent\makebox[1.5em][l]{\textbf{6.}}\parbox[t]{\dimexpr\textwidth-1.5em}{Які з наведених тверджень є правильними?
\begin{enumerate}[label=\Roman*., nosep, leftmargin=*]
    \item Точкою перетину довільних хорд одного кола є його центр.
    \item Паралельні хорди одного кола мають однакову довжину.
    \item Рівні хорди одного кола рівновіддалені від його центра.
\end{enumerate}
\hfill \nmtyear{2023}}

\vspace{0.3cm}
\answerTable{лише II}{лише II та III}{лише I та III}{лише III}{лише I та II}

\vspace{0.7cm}

% === ЗАВДАННЯ 7 ===
\noindent\makebox[1.5em][l]{\textbf{7.}}\parbox[t]{\dimexpr\textwidth-1.5em}{Які з наведених тверджень є правильними?
\begin{enumerate}[label=\Roman*., nosep, leftmargin=*]
    \item Периметр прямокутника дорівнює сумі довжин його діагоналей.
    \item Сума квадратів усіх сторін прямокутника дорівнює сумі квадратів його діагоналей.
    \item Діаметр кола, описаного навколо прямокутника, дорівнює діагоналі прямокутника.
\end{enumerate}
\hfill \nmtyear{2023}}

\vspace{0.3cm}
\answerTable{I, II та III}{лише I та II}{лише I та III}{лише II та III}{лише III}

\noindent\textbf{8.} \begin{minipage}[t]{0.55\textwidth}
На рисунку зображено ромб $ABCD$ та коло, побудоване на його меншій діагоналі $BD$ так, як на діаметрі. Точка $K$ --- точка перетину цього кола з діагоналлю $AC$, $AK = 5$ \textit{см}, $KC = 35$ \textit{см}. До кожної величини (1--3) доберіть її значення (А--Д). \nmtyear{2023}
\end{minipage}
\hfill
\begin{minipage}[t]{0.4\textwidth}
    \vspace{-0.5cm}
    \begin{flushright}
    \begin{tikzpicture}[scale=0.12]
        % AC = 40. Center of Rhombus O (mid of AC) is at 20.
        % K is at 5 from A. A=(0,0), C=(40,0). K=(5,0).
        % O=(20,0).
        % Circle on BD as diameter. Center is O.
        % K is on the circle? No, K is intersection of circle and AC.
        % So distance OK is radius. OK = 20 - 5 = 15.
        % Radius = 15. BD = 30. BO = 15.
        % A=(0,0), O=(20,0), B=(20,15).
        % AB = sqrt(20^2 + 15^2) = sqrt(400+225) = sqrt(625) = 25.
        
        \coordinate (A) at (0,0);
        \coordinate (C) at (40,0);
        \coordinate (O) at (20,0);
        \coordinate (B) at (20,15);
        \coordinate (D) at (20,-15);
        \coordinate (K) at (5,0); % Actually on the left side
        
        \draw[thick] (A) -- (B) -- (C) -- (D) -- cycle;
        \draw[thick] (A) -- (C);
        \draw[thick] (B) -- (D);
        
        \draw[thick] (O) circle (15cm);
        
        \node[left] at (A) {$A$};
        \node[above] at (B) {$B$};
        \node[right] at (C) {$C$};
        \node[below] at (D) {$D$};
        \node[above right] at (K) {$K$};
        \fill (K) circle (15pt);
    \end{tikzpicture}
    \end{flushright}
\end{minipage}

\vspace{0.3cm}

\matchingLayout{
    \textit{Величина} \par \vspace{0.2cm}
    \textbf{1} \quad діаметр заданого кола \\
    \textbf{2} \quad довжина сторони ромба $ABCD$ \\
    \textbf{3} \quad висота ромба $ABCD$
}{
    \textit{Значення величини} \par \vspace{0.2cm}
    \begin{tabular}{ll}
    \textbf{А} & 15 \textit{см} \\
    \textbf{Б} & 20 \textit{см} \\
    \textbf{В} & 24 \textit{см} \\
    \textbf{Г} & 25 \textit{см} \\
    \textbf{Д} & 30 \textit{см} \\
    \end{tabular}
}{
    \answerGrid
}

\vspace{0.7cm}


% === ЗАВДАННЯ 4 (Середини сторін K і M) ===
\noindent\textbf{9.} \begin{minipage}[t]{0.55\textwidth}
На рисунку зображено прямокутник $ABCD$. Точки $K$ і $M$ --- відповідно середини сторін $AB$ і $BC$, $AB = 12$ \textit{см}, $MC = 8$ \textit{см}. До кожного відрізка (1--3) доберіть його довжину (А--Д). \nmtyear{2023}
\end{minipage}
\hfill
\begin{minipage}[t]{0.4\textwidth}
    \vspace{-0.5cm}
    \begin{flushright}
    \begin{tikzpicture}[scale=0.15]
        % AB=12, BC=16 (MC=8 => BC=16)
        % A(0,0) - ні, нехай B буде верхній лівий, як зазвичай A внизу.
        % За рисунком: A внизу зліва. B зверху зліва.
        \coordinate (A) at (0,0);
        \coordinate (B) at (0,12);
        \coordinate (C) at (16,12);
        \coordinate (D) at (16,0);
        
        \coordinate (K) at (0,6); % Середина AB
        \coordinate (M) at (8,12); % Середина BC (8, 12)
        
        \draw[thick] (A) -- (B) -- (C) -- (D) -- cycle;
        \draw[thick] (A) -- (K) -- (M) -- (C); % Лінія на рисунку йде від A до K, потім до M, потім C?
        % На рисунку просто точки. Проведемо лінію AK і MC для візуалізації, як на скріншоті.
        % На скріншоті відрізки: AK з'єднано з M? Ні.
        % Просто позначено рівність відрізків.
        
        % Позначки рівності
        \draw (-0.5,3) -- (0.5,3); % AK
        \draw (-0.5,9) -- (0.5,9); % KB
        \draw[thick] (A)  -- (C);
        \draw (4,11.5) -- (4,12.5); \draw (4.2,11.5) -- (4.2,12.5); % BM
        \draw (12,11.5) -- (12,12.5); \draw (12.2,11.5) -- (12.2,12.5); % MC
        
        \node[below left] at (A) {$A$};
        \node[above left] at (B) {$B$};
        \node[above right] at (C) {$C$};
        \node[below right] at (D) {$D$};
        \node[left] at (K) {$K$};
        \node[above] at (M) {$M$};
        
        \fill (K) circle (10pt);
        \fill (M) circle (10pt);
    \end{tikzpicture}
    \end{flushright}
\end{minipage}

\vspace{0.3cm}

\matchingLayout{
    \textit{Відрізок} \par \vspace{0.2cm}
    \textbf{1} \quad $BC$ \\
    \textbf{2} \quad діаметр кола, описаного навколо прямокутника $ABCD$ \\
    \textbf{3} \quad відстань від точки $D$ до середини $KM$
}{
    \textit{Довжина відрізка} \par \vspace{0.2cm}
    \begin{tabular}{ll}
    \textbf{А} & 10 \textit{см} \\
    \textbf{Б} & 15 \textit{см} \\
    \textbf{В} & 16 \textit{см} \\
    \textbf{Г} & 18 \textit{см} \\
    \textbf{Д} & 20 \textit{см} \\
    \end{tabular}
}{
    \answerGrid
}

\vspace{0.7cm}

% === ЗАВДАННЯ 5 (Коло і півколо) ===
\noindent\textbf{10.} \begin{minipage}[t]{0.6\textwidth}
У прямокутник $ABCD$ вписано коло із центром в точці $O$, яке дотикається до сторін $AB$, $BC$ і $AD$, та півколо з діаметром $CD$ (див. рисунок). Коло й півколо мають лише одну спільну точку. $AB = 8$ \textit{см}. До кожного початку речення (1--3) доберіть його закінчення (А--Д) так, щоб утворилося правильне твердження. \nmtyear{2023}
\end{minipage}
\hfill
\begin{minipage}[t]{0.35\textwidth}
    \vspace{-0.5cm}
    \begin{flushright}
    \begin{tikzpicture}[scale=0.3]
        % AB=8. Коло R=4. Центр O(4,4).
        % BC=12. Півколо на CD. R=4. Центр (12,4).
        % Дотикаються в точці (8,4).
        
        \coordinate (A) at (0,0);
        \coordinate (B) at (0,8);
        \coordinate (C) at (12,8);
        \coordinate (D) at (12,0);
        \coordinate (O) at (4,4);
        
        \draw[thick] (A) -- (B) -- (C) -- (D) -- cycle;
        
        % Коло
        \draw[thick] (O) circle (4cm);
        
        % Півколо (дуга від C до D зліва)
        \draw[thick] (C) arc (90:270:4cm);
        
        \node[below left] at (A) {$A$};
        \node[above left] at (B) {$B$};
        \node[above] at (C) {$C$};
        \node[below right] at (D) {$D$};
        \node at (O) {$\cdot$};
        \node[below right] at (O) {$O$};
    \end{tikzpicture}
    \end{flushright}
\end{minipage}

\vspace{0.3cm}

\matchingLayout{
    \textit{Початок речення} \par \vspace{0.2cm}
    \textbf{1} \quad Довжина сторони $BC$ \\
    \textbf{2} \quad Довжина відрізка $OC$ \\
    \textbf{3} \quad Відстань від середини відрізка $AO$ до прямої $CD$
}{
    \textit{Закінчення речення} \par \vspace{0.2cm}
    \begin{tabular}{ll}
    \textbf{А} & дорівнює 10 \textit{см}. \\
    \textbf{Б} & дорівнює 12 \textit{см}. \\
    \textbf{В} & дорівнює 16 \textit{см}. \\
    \textbf{Г} & дорівнює $4\sqrt{5}$ \textit{см}. \\
    \textbf{Д} & дорівнює $4\sqrt{3}$ \textit{см}. \\
    \end{tabular}
}{
    \answerGrid
}

\vspace{0.7cm}

% === ЗАВДАННЯ 5 ===
\noindent\textbf{11.} \begin{minipage}[t]{0.55\textwidth}
У ромб, менша діагональ якого $30$ \textit{см}, вписано коло радіуса $12$ \textit{см} (див. рисунок). До кожної величини (1--3) доберіть її значення (А--Д). \nmtyear{2023}
\end{minipage}
\hfill
\begin{minipage}[t]{0.4\textwidth}
    \vspace{-0.5cm}
    \begin{flushright}
    \begin{tikzpicture}[scale=0.15]
        \coordinate (A) at (-15,0); % Половина діагоналі
        \coordinate (C) at (15,0);
        \coordinate (B) at (0,20); % Знайдено з пропорцій (12, 15, 20)
        \coordinate (D) at (0,-20);
        \coordinate (O) at (0,0);
        
        \draw[thick] (A) -- (B) -- (C) -- (D) -- cycle;
        \draw[thick] (O) circle (12cm);
        
        % Діагоналі (тонкі, як на рисунку немає, але для розуміння центру)
        %\draw[gray, thin] (A)--(C); \draw[gray, thin] (B)--(D);
    \end{tikzpicture}
    \end{flushright}
\end{minipage}

\vspace{0.3cm}

\matchingLayout{
    \textit{Величина} \par \vspace{0.2cm}
    \textbf{1} \quad висота ромба \\
    \textbf{2} \quad проєкція меншої діагоналі на сторону ромба \\
    \textbf{3} \quad сторона ромба
}{
    \textit{Значення величини} \par \vspace{0.2cm}
    \begin{tabular}{ll}
    \textbf{А} & 18 \textit{см} \\
    \textbf{Б} & 20 \textit{см} \\
    \textbf{В} & 24 \textit{см} \\
    \textbf{Г} & 25 \textit{см} \\
    \textbf{Д} & 30 \textit{см} \\
    \end{tabular}
}{
    \answerGrid
}

\vspace{0.7cm}

% === ЗАВДАННЯ 8 ===
\noindent\textbf{12.} \begin{minipage}[t]{0.55\textwidth}
Два кола з центрами в точках $O$ і $O_1$ мають зовнішній дотик (див. рисунок). Обчисліть відстань $OO_1$, якщо радіуси кіл дорівнюють $2$ \textit{см} і $10$ \textit{см}. \nmtyear{2023}
\end{minipage}
\hfill
\begin{minipage}[t]{0.4\textwidth}
    \vspace{-0.5cm}
    \begin{flushright}
    \begin{tikzpicture}[scale=0.15]
        \coordinate (O) at (0,0);
        \coordinate (O1) at (14,0); % 2 + 10 = 12 distance
        
        \draw[thick] (O) circle (5cm);
        \draw[thick] (O1) circle (9cm);
        
        \fill (O) circle (15pt) node[left] {$O$};
        \fill (O1) circle (15pt) node[right] {$O_1$};
    \end{tikzpicture}
    \end{flushright}
\end{minipage}

\vspace{0.3cm}
\answerTable{$6$ \textit{см}}{$9$ \textit{см}}{$4$ \textit{см}}{$8$ \textit{см}}{$12$ \textit{см}}

\vspace{0.7cm}

% === ЗАВДАННЯ 9 ===
\noindent\makebox[1.5em][l]{\textbf{13.}}\parbox[t]{\dimexpr\textwidth-1.5em}{Точки $A$ і $B$ лежать на колі радіуса $16$. Укажіть найбільше можливе значення довжини відрізка $AB$. \nmtyear{2023}}

\vspace{0.3cm}
\answerTable{$64$}{$48$}{$8$}{$16$}{$32$}

\vspace{0.7cm}

% === ЗАВДАННЯ 10 ===
\noindent\textbf{14.} \begin{minipage}[t]{0.55\textwidth}
У прямокутному трикутнику $ACB$ $\angle C = 90^\circ, \angle B = 24^\circ$. На продовжені катета $AC$ вибрано точку $K$ так, що $AK = KB$ (див. рисунок). Точка $O$ --- центр кола, описаного навколо трикутника $ACB$. До кожного кута (1--3) доберіть його градусну міру (А--Д). \nmtyear{2023}
\end{minipage}
\hfill
\begin{minipage}[t]{0.4\textwidth}
    \vspace{-0.5cm}
    \begin{flushright}
    \begin{tikzpicture}[scale=0.7]
        \coordinate (C) at (0,0);
        \coordinate (B) at (4,0);
        \coordinate (A) at (0,2); % CA < CB visually
        % K is on extension of AC downwards.
        % AK = KB. Let K = (0, y_k). A = (0, 2). B = (4, 0).
        % AK = 2 - y_k (since y_k < 0).
        % KB = sqrt(4^2 + y_k^2).
        % (2 - y_k)^2 = 16 + y_k^2
        % 4 - 4y_k + y_k^2 = 16 + y_k^2
        % -4y_k = 12 => y_k = -3.
        \coordinate (K) at (0,-3);
        
        \draw[thick] (K) -- (A) -- (B) -- cycle; % Triangle ABK
        \draw[thick] (K) -- (B); 
        \draw[thick] (C) -- (B); % Leg CB
        
        % Right angle at C
        \draw (0,0) rectangle (0.3,0.3);
        
        % Mark AK = KB?
        % Visual marks
        \draw[thick] ($(A)!0.5!(K)$) ++(0.15,0) -- ++(-0.3,0);
        \draw[thick] ($(K)!0.5!(B)$) ++(120:0.15) -- ++(-60:0.3);
        
        % Angle 24
        \pic [draw, pic text={\small $24^\circ$}, angle radius=.5cm, angle eccentricity=1.8] {angle = A--B--C};
        
        \node[left] at (A) {$A$};
        \node[left] at (C) {$C$};
        \node[left] at (K) {$K$};
        \node[right] at (B) {$B$};
        
    \end{tikzpicture}
    \end{flushright}
\end{minipage}

\vspace{0.3cm}

\matchingLayout{
    \textit{Кут} \par \vspace{0.2cm}
    \textbf{1} \quad $\angle BAC$ \\
    \textbf{2} \quad $\angle KBC$ \\
    \textbf{3} \quad $\angle OKB$
}{
    \textit{Градусна міра кута} \par \vspace{0.2cm}
    \begin{tabular}{ll}
    \textbf{А} & $24^\circ$ \\
    \textbf{Б} & $34^\circ$ \\
    \textbf{В} & $42^\circ$ \\
    \textbf{Г} & $66^\circ$ \\
    \textbf{Д} & $72^\circ$ \\
    \end{tabular}
}{
    \answerGrid
}

\vspace{0.7cm}

% === ЗАВДАННЯ 11 ===
\noindent\textbf{15.} \begin{minipage}[t]{0.55\textwidth}
На катеті $BC$ прямокутного трикутника $ACB$, у якому $\angle C = 90^\circ, \angle B = 32^\circ$, вибрано точку $K$ так, що $AK = KB$ (див. рисунок). Точка $O$ --- центр кола, описаного навколо трикутника $ACB$. До кожного кута (1--3) доберіть його градусну міру (А--Д). \nmtyear{2023}
\end{minipage}
\hfill
\begin{minipage}[t]{0.4\textwidth}
    \vspace{-0.5cm}
    \begin{flushright}
    \begin{tikzpicture}[scale=0.8]
        \coordinate (C) at (0,0);
        \coordinate (A) at (0,3);
        \coordinate (B) at (5,0);
        
        % K on BC. AK = KB.
        % Let K = (x, 0).
        % AK^2 = x^2 + 3^2. KB^2 = (5-x)^2.
        % x^2 + 9 = 25 - 10x + x^2.
        % 10x = 16 => x = 1.6.
        \coordinate (K) at (1.6, 0);
        
        \draw[thick] (A) -- (C) -- (B) -- cycle;
        \draw[thick] (A) -- (K);
        
        % Marks AK = KB
        \draw[thick] ($(A)!0.5!(K)$) ++(150:0.15) -- ++(-30:0.3);
        \draw[thick] ($(K)!0.5!(B)$) ++(90:0.15) -- ++(-90:0.3);
        
        % Right angle
        \draw (0,0) rectangle (0.3,0.3);
        
        % Angle 32
        \pic [draw, pic text={\small $32^\circ$}, angle radius=1.0cm, angle eccentricity=1.4] {angle = A--B--C};
        
        \node[above] at (A) {$A$};
        \node[below] at (C) {$C$};
        \node[below] at (K) {$K$};
        \node[below] at (B) {$B$};
        
    \end{tikzpicture}
    \end{flushright}
\end{minipage}

\vspace{0.3cm}

\matchingLayout{
    \textit{Кут} \par \vspace{0.2cm}
    \textbf{1} \quad $\angle KAB$ \\
    \textbf{2} \quad $\angle KAC$ \\
    \textbf{3} \quad $\angle OKB$
}{
    \textit{Градусна міра кута} \par \vspace{0.2cm}
    \begin{tabular}{ll}
    \textbf{А} & $24^\circ$ \\
    \textbf{Б} & $26^\circ$ \\
    \textbf{В} & $32^\circ$ \\
    \textbf{Г} & $58^\circ$ \\
    \textbf{Д} & $64^\circ$ \\
    \end{tabular}
}{
    \answerGrid
}

\vspace{0.7cm}

% === ЗАВДАННЯ 12 ===
\noindent\textbf{16.} \begin{minipage}[t]{0.55\textwidth}
На рисунку зображено коло з центром у точці $O$, $AB$ --- діаметр цього кола. Пряма $a$ --- дотична до цього кола, що перетинає продовження діаметра $AB$ у точці $K$, $C$ --- точка дотику. Установіть відповідність між величиною (1--3) та її значенням (А--Д), якщо дуга $AC = 130^\circ$. \nmtyear{2023}
\end{minipage}
\hfill
\begin{minipage}[t]{0.4\textwidth}
    \vspace{-0.5cm}
    \begin{flushright}
    \begin{tikzpicture}[scale=0.9]
        \coordinate (O) at (0,0);
        \draw[thick] (O) circle (1.5cm);
        \coordinate (L) at (-1.5,2);
        \coordinate (A) at (-1.5,0);
        \coordinate (B) at (1.5,0);
        \coordinate (K) at (3.5,0);
        
        % Tangent from K.
        % OK = 3.5, R=1.5. Angle COK = arccos(1.5/3.5) approx 64.6 deg.
        % C is at (1.5*cos(64.6), 1.5*sin(64.6)).
        \coordinate (C) at (64.6:1.5);
        
        \draw[thick] (A) -- (K);
        \draw[thick] (L) -- (C);
        \draw[thick] (C) -- (K) node[midway, above right] {$a$};
        \draw[thick] (O) -- (C);
        
        % Bold arc AC
        \draw[very thick] (A) arc (180:64.6:1.5);
        
        \node[left] at (A) {$A$};
        \node[below right] at (B) {$B$}; % B is slightly covered by line
        \node[below] at (O) {$O$};
        \node[above] at (K) {$K$};
        \node[above right] at (C) {$C$};
        
        \fill (O) circle (1.5pt);
        \fill (A) circle (1.5pt);
        \fill (B) circle (1.5pt);
        \fill (K) circle (1.5pt);
        \fill (C) circle (1.5pt);
    \end{tikzpicture}
    \end{flushright}
\end{minipage}

\vspace{0.3cm}

\matchingLayout{
    \textit{Величина} \par \vspace{0.2cm}
    \textbf{1} \quad дуга $CB$ \\
    \textbf{2} \quad $\angle CKA$ \\
    \textbf{3} \quad $\angle ABC$
}{
    \textit{Значення величини} \par \vspace{0.2cm}
    \begin{tabular}{ll}
    \textbf{А} & $30^\circ$ \\
    \textbf{Б} & $40^\circ$ \\
    \textbf{В} & $50^\circ$ \\
    \textbf{Г} & $65^\circ$ \\
    \textbf{Д} & $70^\circ$ \\
    \end{tabular}
}{
    \answerGrid
}

\vspace{0.7cm}

% === ЗАВДАННЯ 13 ===
\noindent\makebox[1.5em][l]{\textbf{17.}}\parbox[t]{\dimexpr\textwidth-1.5em}{Які з наведених тверджень є правильними?
\begin{enumerate}[label=\Roman*., nosep, leftmargin=*]
    \item Пряма, що проходить через центр кола і лежить із цим колом в одній площині, має з ним дві спільні точки.
    \item Діаметр кола, перпендикулярний до його хорди, проходить через середину цієї хорди.
    \item Можна провести два діаметри кола, що не мають жодної спільної точки.
\end{enumerate}
\hfill \nmtyear{2023}}

\vspace{0.3cm}
\answerTable{лише I та II}{лише II}{лише II та III}{I, II та III}{лише I та III}

\vspace{0.7cm}

% === ЗАВДАННЯ 14 ===
\noindent\textbf{18.} \begin{minipage}[t]{0.55\textwidth}
Периметр рівнобедреного трикутника (див. рисунок) дорівнює $32$ \textit{см}. $AB = BC = 10$ \textit{см}. До кожного відрізка (1--3) доберіть його довжину (А--Д). \nmtyear{2023}
\end{minipage}
\hfill
\begin{minipage}[t]{0.4\textwidth}
    \vspace{-0.5cm}
    \begin{flushright}
    \begin{tikzpicture}[scale=0.25]
        % AB=BC=10. P=32. AC = 32 - 20 = 12.
        % Height h = sqrt(100 - 36) = 8.
        \coordinate (A) at (-6,0);
        \coordinate (C) at (6,0);
        \coordinate (B) at (0,8);
        
        \draw[thick] (A) -- (B) -- (C) -- cycle;
        
        % Marks
        \draw[thick] ($(A)!0.5!(B)$) ++(150:0.3) -- ++(-30:0.6);
        \draw[thick] ($(B)!0.5!(C)$) ++(30:0.3) -- ++(-150:0.6);
        
        \node[below left] at (A) {$A$};
        \node[above] at (B) {$B$};
        \node[below right] at (C) {$C$};
    \end{tikzpicture}
    \end{flushright}
\end{minipage}

\vspace{0.3cm}

\matchingLayout{
    \textit{Відрізок} \par \vspace{0.2cm}
    \textbf{1} \quad $AC$ \\
    \textbf{2} \quad висота, проведена з \par \quad вершини $B$ \\
    \textbf{3} \quad радіус кола, \par \quad описаного навколо \par \quad трикутника $ABC$
}{
    \textit{Довжина відрізка} \par \vspace{0.2cm}
    \begin{tabular}{ll}
    \textbf{А} & 6{,}25 \textit{см} \\
    \textbf{Б} & 7{,}5 \textit{см} \\
    \textbf{В} & 8 \textit{см} \\
    \textbf{Г} & 12 \textit{см} \\
    \textbf{Д} & 12{,}5 \textit{см} \\
    \end{tabular}
}{
    \answerGrid
}

\vspace{0.7cm}

% === ЗАВДАННЯ 15 ===
% === ЗАВДАННЯ 15 (уточнений рисунок) ===
\noindent\textbf{19.} \begin{minipage}[t]{0.55\textwidth}
У довільний трикутник $ABC$ вписано коло з центром у точці $O$, точки $K, L, M$ --- точки дотику (див. рисунок). Які з наведених тверджень є правильними?
\begin{enumerate}[label=\Roman*., nosep, leftmargin=*]
    \item Трикутник $AOK$ є прямокутним.
    \item Трикутник $BKL$ є рівнобедреним.
    \item Трикутники $MOC$ і $LOC$ є рівними.
\end{enumerate}
\hfill \nmtyear{2023}
\end{minipage}
\hfill
\begin{minipage}[t]{0.4\textwidth}
    \vspace{-0.5cm}
    \begin{flushright}
    \begin{tikzpicture}[scale=0.7]
        \coordinate (A) at (0,0);
        \coordinate (C) at (6,0);
        \coordinate (B) at (1.5,4);

        % Центр вписаного кола (розраховано за формулами)
        % a=6.02, b=6, c=4.27. Периметр ~16.3.
        % Координати O ~ (2.12, 1.47)
        \coordinate (O) at (2.12, 1.47);
        \def\r{1.47} % Радіус

        \draw[thick] (A) -- (B) -- (C) -- cycle;
        \draw[thick] (O) circle (\r);

        % Точки дотику (проєкції центра на сторони)
        \coordinate (M) at ($(A)!(O)!(C)$);
        \coordinate (K) at ($(A)!(O)!(B)$);
        \coordinate (L) at ($(B)!(O)!(C)$);

        \node[below left] at (A) {$A$};
        \node[above left] at (B) {$B$};
        \node[below right] at (C) {$C$};
        \node[below] at (M) {$M$};
        \node[left] at (K) {$K$};
        \node[above right] at (L) {$L$}; % Трохи вище для кращого вигляду
        \node[right] at (O) {$O$};

        \fill (O) circle (1.5pt);
        \fill (K) circle (1.5pt);
        \fill (L) circle (1.5pt);
        \fill (M) circle (1.5pt);
        
    \end{tikzpicture}
    \end{flushright}
\end{minipage}

\vspace{0.3cm}
\answerTable{лише II та III}{лише I та II}{I, II та III}{лише III}{лише II}

% === ЗАВДАННЯ 4 ===
\noindent\textbf{20.} \begin{minipage}[t]{0.55\textwidth}
На рисунку зображено ромб, сторона якого дорівнює $a$, а кут між стороною та меншою діагоналлю --- $\beta$. Визначте радіус кола, вписаного в цей ромб. \nmtyear{2023}
\end{minipage}
\hfill
\begin{minipage}[t]{0.4\textwidth}
    \vspace{-0.5cm}
    \begin{flushright}
    
    % Спроба 2 (як на скріншоті)
    \begin{tikzpicture}[scale=0.7]
        \coordinate (O) at (0,0);
        \coordinate (A) at (-3,0); % Лівий
        \coordinate (C) at (3,0);  % Правий
        \coordinate (B) at (0,4);  % Верхній
        \coordinate (D) at (0,-4); % Нижній
        
        % Менша діагональ AC (на скріншоті горизонтальна лінія - це менша діагональ)
        \draw[thick] (A) -- (B) -- (C) -- (D) -- cycle;
        \draw[thick] (A) -- (C);
        
        \draw[thick] (O) circle (2.4cm); % Приблизно
        
        % Кут бета
        \pic [draw, pic text={\small $\beta$}, angle radius=0.8cm, angle eccentricity=1.3] {angle = B--C--A};
        
    \end{tikzpicture}
    \end{flushright}
\end{minipage}

\vspace{0.3cm}
\answerTableTall{$a\sin^2\beta$}{$a\sin\beta$}{$\dfrac{a\sin 2\beta}{2}$}{$a\cos\beta$}{$a\cos^2\beta$}

\vspace{0.7cm}

% === ЗАВДАННЯ 7 ===
\noindent\textbf{21.} \begin{minipage}[t]{0.55\textwidth}
Ромб $ABCD$ та коло із центром у точці $O$, довжина якого $12\pi$ \textit{см}, лежать в одній площині (див. рисунок). Сторона ромба $AB$ перетинає коло в точці $K$, $AD$ --- діаметр кола, $AK = OA$. До кожного початку речення (1--3) доберіть його закінчення (А--Д) так, щоб утворилося правильне твердження. \nmtyear{2023}
\end{minipage}
\hfill
\begin{minipage}[t]{0.4\textwidth}
    \vspace{-0.5cm}
    \begin{flushright}
    \begin{tikzpicture}[scale=0.2]
        % Довжина 12pi -> R=6. AD=12. O - центр AD.
        % AK = OA = 6. O=(6,0), A=(0,0), D=(12,0).
        % Трикутник AOK: AO=6, OK=6(радіус), AK=6. Рівносторонній.
        % Кут A = 60 градусів.
        % Ромб ABCD зі стороною 12 і кутом 60.
        
        \coordinate (A) at (0,0);
        \coordinate (D) at (12,0);
        \coordinate (O) at (6,0);
        
        % Точка B (кут 60, сторона 12)
        \coordinate (B) at (60:12);
        \coordinate (C) at ($(D)+(B)-(A)$);
        
        % Точка K (перетин AB і кола). Оскільки A=60, R=6, AK=6.
        \coordinate (K) at (60:6);
        
        % Коло
        \draw[thick] (O) circle (6cm);
        
        % Ромб
        \draw[thick] (A) -- (B) -- (C) -- (D); % Нижня частина перекривається колом?
        % На рисунку коло малюється поверх сторони AD?
        % А, AD - діаметр. Значить AD лежить на осі.
        \draw[thick] (A) -- (D); 
        
        % Відрізок OK
        \draw[thick] (O) -- (K);
        
        % Позначки рівності AK = AO
        \draw (3, -0.2) -- (3, 0.2); % AO
        \draw (60:3) ++(150:0.2) -- ++(-30:0.4); % AK
        \draw ($(O)!0.5!(K)$) ++(60:0.2) -- ++(240:0.4); % OK (теж радіус, але в умові AK=OA)
        
        % Точки
        \node[left] at (A) {$A$};
        \node[above] at (B) {$B$};
        \node[right] at (C) {$C$};
        \node[right] at (D) {$D$};
        \node[above left] at (K) {$K$};
        \node[below] at (O) {$O$};
        
        \fill (O) circle (10pt);
        \fill (K) circle (10pt);
        \fill (A) circle (10pt);
        \fill (D) circle (10pt);
    \end{tikzpicture}
    \end{flushright}
\end{minipage}

\vspace{0.3cm}

\matchingLayout{
    \textit{Початок речення} \par \vspace{0.2cm}
    \textbf{1} \quad Довжина радіуса $OA$ \\
    \textbf{2} \quad Довжина діагоналі $BD$ ромба $ABCD$ \\
    \textbf{3} \quad Довжина висоти ромба
}{
    \textit{Закінчення речення} \par \vspace{0.2cm}
    \begin{tabular}{ll}
    \textbf{А} & дорівнює 6 \textit{см}. \\
    \textbf{Б} & дорівнює $6\sqrt{3}$ \textit{см}. \\
    \textbf{В} & дорівнює 12 \textit{см}. \\
    \textbf{Г} & дорівнює $6\sqrt{2}$ \textit{см}. \\
    \textbf{Д} & дорівнює $12\sqrt{3}$ \textit{см}. \\
    \end{tabular}
}{
    \answerGrid
}
\vspace{0.7cm}

% === ЗАВДАННЯ 4 ===
\noindent\textbf{22.} \begin{minipage}[t]{0.55\textwidth}
Коло радіуса $6$ вписано в квадрат (див. рисунок). Визначте периметр квадрата. \nmtyear{2023}
\end{minipage}
\hfill
\begin{minipage}[t]{0.4\textwidth}
    \vspace{-0.5cm}
    \begin{flushright}
    \begin{tikzpicture}[scale=0.25]
        \draw[thick] (0,0) rectangle (12,12);
        \draw[thick] (6,6) circle (6cm);
    \end{tikzpicture}
    \end{flushright}
\end{minipage}

\vspace{0.3cm}
\answerTable{$48$}{$36$}{$60$}{$12$}{$24$}

\vspace{0.7cm}

\begin{center}
{\Large\textbf{\color{headerblue}БАЗА ЗАВДАНЬ НМТ 2024}}
\end{center}

% === ЗАВДАННЯ 16 ===
\noindent\textbf{23.} \begin{minipage}[t]{0.55\textwidth}
До кола з центром у точці $O$ проведено дотичну $AK$, $A$ --- точка дотику (див. рисунок). На $AK$ вибрано точку $B$ так, що $\angle AOB = 55^\circ$. Знайдіть градусну міру кута $OBK$. \nmtyear{2024}
\end{minipage}
\hfill
\begin{minipage}[t]{0.4\textwidth}
    \vspace{-0.5cm}
    \begin{flushright}
    \begin{tikzpicture}[scale=0.7]
        \coordinate (O) at (0,0);
        \coordinate (A) at (0,1.5);
        % Tangent line
        \draw[thick] (-2,1.5) -- (3.5,1.5) node[above] {$K$};
        \draw[thick] (O) circle (1.5cm);
        
        % Point B. Angle AOB = 55.
        % AB / OA = tan(55). OA = 1.5. AB = 1.5 * tan(55) approx 2.14
        \coordinate (B) at (2.14, 1.5);
        
        \draw[thick] (O) -- (A);
        \draw[thick] (O) -- (B);
        
        % Angle 55
        \pic [draw, pic text={\small $55^\circ$}, angle radius=0.4cm, angle eccentricity=1.8] {angle = B--O--A};
        
        % Exterior Angle at B (OBK)
        \coordinate (K_end) at (3.5, 1.5);
        \pic [draw, pic text={\small ?}, angle radius=0.4cm, angle eccentricity=1.8] {angle = O--B--K_end};
        \pic [draw, angle radius=0.3cm, angle eccentricity=1.8] {angle = O--B--K_end};
        
        \node[left] at (O) {$O$};
        \node[above] at (A) {$A$};
        \node[above] at (B) {$B$};
        
        \fill (O) circle (1.5pt);
        \fill (A) circle (1.5pt);
        \fill (B) circle (1.5pt);
        
    \end{tikzpicture}
    \end{flushright}
\end{minipage}

\vspace{0.3cm}
\answerTable{$135^\circ$}{$125^\circ$}{$55^\circ$}{$145^\circ$}{$155^\circ$}

\vspace{0.7cm}

% === ЗАВДАННЯ 17 ===
\noindent\textbf{24.} \begin{minipage}[t]{0.55\textwidth}
На прямій $AB$ два кола з центрами в точках $A$ і $C$ мають зовнішній дотик, точка $B$ належить меншому колу (див. рисунок). Обчисліть відстань $AB$, якщо радіуси кіл дорівнюють $8$ \textit{см} і $5$ \textit{см}. \nmtyear{2024}
\end{minipage}
\hfill
\begin{minipage}[t]{0.4\textwidth}
    \vspace{-0.5cm}
    \begin{flushright}
    \begin{tikzpicture}[scale=0.15]
        \coordinate (A) at (0,0);
        % R1 = 8. Touch point at (8,0).
        % R2 = 5. Center C at (8+5, 0) = (13,0).
        % Point B at (13+5, 0) = (18,0).
        
        \coordinate (C) at (13,0);
        \coordinate (B) at (18,0);
        \coordinate (LineEnd) at (22,0);
        \coordinate (LineStart) at (-10,0);
        
        \draw[thick] (LineStart) -- (LineEnd);
        
        \draw[thick] (A) circle (8cm);
        \draw[thick] (C) circle (5cm);
        
        \node[above] at (A) {$A$};
        \node[above] at (C) {$C$};
        \node[above right] at (B) {$B$};
        
        \fill (A) circle (15pt);
        \fill (C) circle (15pt);
        \fill (B) circle (15pt);
    \end{tikzpicture}
    \end{flushright}
\end{minipage}

\vspace{0.3cm}
\answerTable{$21$ \textit{см}$}{$16$ \textit{см}}{$26$ \textit{см}}{$13$ \textit{см}}{$18$ \textit{см}}

\vspace{0.7cm}

% === ЗАВДАННЯ 18 ===
\noindent\makebox[1.5em][l]{\textbf{25.}}\parbox[t]{\dimexpr\textwidth-1.5em}{Які з наведених тверджень є правильними?
\begin{enumerate}[label=\Roman*., nosep, leftmargin=*]
    \item Будь-який ромб є паралелограмом.
    \item Центр вписаного в будь-який ромб кола лежить на перетині бісектрис його кутів.
    \item Менша діагональ будь-якого ромба ділить його на 2 правильні трикутники.
\end{enumerate}
\hfill \nmtyear{2024}}

\vspace{0.3cm}
\answerTable{I, II та III}{лише I та III}{лише I}{лише II}{лише I та II}

\vspace{0.7cm}

% === ЗАВДАННЯ 19 ===
\noindent\makebox[1.5em][l]{\textbf{26.}}\parbox[t]{\dimexpr\textwidth-1.5em}{Які з наведених тверджень є правильними?
\begin{enumerate}[label=\Roman*., nosep, leftmargin=*]
    \item Медіана трикутника з'єднує його вершину із серединою протилежної сторони.
    \item Точка перетину медіан довільного трикутника знаходиться в центрі кола, уписаного в цей трикутник.
    \item У рівносторонньому трикутнику медіана належить серединному перпендикуляру, проведеному до спільної сторони.
\end{enumerate}
\hfill \nmtyear{2024}}

\vspace{0.3cm}
\answerTable{I, II та III}{лише I та III}{лише III}{лише I та II}{лише I}

\vspace{0.7cm}

% === ЗАВДАННЯ 20 ===
\noindent\makebox[1.5em][l]{\textbf{27.}}\parbox[t]{\dimexpr\textwidth-1.5em}{Які з наведених тверджень є правильними?
\begin{enumerate}[label=\Roman*., nosep, leftmargin=*]
    \item Будь-яка хорда кола більша за радіус цього кола.
    \item Кінці діаметра ділять коло на дві рівні частини.
    \item Рівні хорди кола стягують рівні дуги.
\end{enumerate}
\hfill \nmtyear{2024}}

\vspace{0.3cm}
\answerTable{лише I та II}{лише II та III}{лише I та III}{лише III}{лише II}

\vspace{0.7cm}

% === ЗАВДАННЯ 21 ===
\noindent\textbf{28.} \begin{minipage}[t]{0.55\textwidth}
На рисунку зображено прямокутний трикутник $ABC$ ($\angle C = 90^\circ$). Точка $M$ --- середина $CB = 16$ \textit{см}. Радіус кола, описаного навколо трикутника $ABC$, дорівнює $10$ \textit{см}. До кожного відрізка (1--3) доберіть його довжину (А--Д). \nmtyear{2024}
\end{minipage}
\hfill
\begin{minipage}[t]{0.4\textwidth}
    \vspace{-0.5cm}
    \begin{flushright}
    \begin{tikzpicture}[scale=0.25]
        \coordinate (C) at (0,0);
        \coordinate (B) at (16,0);
        % R=10 -> Hypotenuse AB=20.
        % AC = sqrt(400 - 256) = 12.
        \coordinate (A) at (0,12);
        
        % M is midpoint of CB
        \coordinate (M) at (8,0);
        
        \draw[thick] (A) -- (C) -- (B) -- cycle;
        \draw[thick] (A) -- (M);
        
        % Right angle
        \draw (C) rectangle ++(1.5,1.5);
        
        % Marks for CM = MB
        \draw[thick] (4, -0.3) -- (4, 0.3);
        \draw[thick] (12, -0.3) -- (12, 0.3);
        
        \node[left] at (A) {$A$};
        \node[below left] at (C) {$C$};
        \node[below] at (M) {$M$};
        \node[below right] at (B) {$B$};
        
        \fill (M) circle (7pt);
        
    \end{tikzpicture}
    \end{flushright}
\end{minipage}

\vspace{0.3cm}

\matchingLayout{
    \textit{Відрізок} \par \vspace{0.2cm}
    \textbf{1} \quad $AC$ \\
    \textbf{2} \quad найбільша середня \par \quad лінія трикутника $ABC$ \\
    \textbf{3} \quad $AM$
}{
    \textit{Довжина відрізка} \par \vspace{0.2cm}
    \begin{tabular}{ll}
    \textbf{А} & 10 \textit{см} \\
    \textbf{Б} & 12 \textit{см} \\
    \textbf{В} & 16 \textit{см} \\
    \textbf{Г} & $4\sqrt{11}$ \textit{см} \\
    \textbf{Д} & $4\sqrt{13}$ \textit{см} \\
    \end{tabular}
}{
    \answerGrid
}

\vspace{0.7cm}

% === ЗАВДАННЯ 22 ===
\noindent\makebox[1.5em][l]{\textbf{29.}}\parbox[t]{\dimexpr\textwidth-1.5em}{Які з наведених тверджень є правильними?
\begin{enumerate}[label=\Roman*., nosep, leftmargin=*]
    \item У будь-яку рівнобічну трапецію можна вписати коло.
    \item Довжина радіуса вписаного в ромб кола дорівнює половині його висоти.
    \item Навколо будь-якої рівнобічної трапеції можна описати коло.
\end{enumerate}
\hfill \nmtyear{2024}}

\vspace{0.3cm}
\answerTable{лише II}{лише I та II}{лише II та III}{I, II та III}{лише III}


% === ЗАВДАННЯ 2 ===
\noindent\textbf{30.} \begin{minipage}[t]{0.55\textwidth}
До кола із центром у точці $O$ проведено дотичну $AB$, яка дотикається кола в точці $A$. Пряма $OB$ перетинає коло в точках $D$ і $C$, $\angle AOB = 60^\circ$. Які з наведених тверджень є правильними?
\begin{enumerate}[label=\Roman*., nosep, leftmargin=*]
    \item $OA \perp AB$.
    \item $OB = 2OA$.
    \item $OD = AC$.
\end{enumerate}
\hfill \nmtyear{2024}
\end{minipage}
\hfill
\begin{minipage}[t]{0.4\textwidth}
    \vspace{-0.5cm}
    \begin{flushright}
    \begin{tikzpicture}[scale=0.7]
        \coordinate (O) at (0,0);
        \draw[thick] (O) circle (1.5cm);
        
        % A at top
        \coordinate (A) at (0,1.5);
        % Tangent horizontal
        \draw[thick] (-2,1.5) -- (2.5,1.5);
        
        % Angle 60. B is on tangent.
        % B is at (1.5/tan(30), 1.5) = (1.5 * 1.732, 1.5) approx (2.6, 1.5).
        % Let's verify angle AOB is 60. Slope OA vertical. Slope OB: dx = 2.6, dy = 1.5.
        % Tan(theta) = x/y = tan(60). Correct.
        \coordinate (B) at ({1.5*tan(60)}, 1.5); 
        
        % Secant line OB
        \draw[thick] ($(O)!-0.4!(B)$) -- (B);
        
        \draw[thick] (O) -- (A);
        
        % Points C and D on circle along line OB
        \coordinate (C) at (intersection of O--B and O circle 1.5cm); % This finds intersection, but tricky with syntax.
        % Calculate manually. Angle is 90-60=30 from horizontal? No, A is (0,1.5).
        % Angle of OB from vertical is 60. From horizontal is 30.
        \coordinate (C) at (30:1.5);
        \coordinate (D) at (210:1.5);
        \draw[thick] (O) -- (D);
        \draw[thick] (A) -- (C);
        % Angle 60 at O (AOB)
        \pic [draw, pic text={\small $60^\circ$}, angle radius=0.4cm, angle eccentricity=1.7] {angle = C--O--A};
        
        \node[below] at (O) {$O$};
        \node[above] at (A) {$A$};
        \node[above] at (B) {$B$};
        \node[right] at (C) {$C$};
        \node[below left] at (D) {$D$};
        
        \fill (O) circle (1.5pt);
        \fill (A) circle (1.5pt);
        \fill (B) circle (1.5pt);
        \fill (C) circle (1.5pt);
        \fill (D) circle (1.5pt);
        
    \end{tikzpicture}
    \end{flushright}
\end{minipage}

\vspace{0.3cm}
\answerTable{лише I та III}{лише I та II}{лише III}{лише II та III}{I, II та III}

\vspace{0.7cm}

% === ЗАВДАННЯ 21 ===
\noindent\textbf{31.} \begin{minipage}[t]{0.55\textwidth}
Навколо кола описано рівнобічну трапецію (див. рисунок), периметр якої дорівнює $100$ \textit{см}. Різниця основ трапеції дорівнює $14$ \textit{см}. До кожного початку речення (1--3) доберіть його закінчення (А--Д) так, щоб утворилося правильне твердження. \nmtyear{2024}
\end{minipage}
\hfill
\begin{minipage}[t]{0.4\textwidth}
    \vspace{-0.5cm}
    \begin{flushright}
    \begin{tikzpicture}[scale=0.15]
        % Circumscribed isosceles trapezoid.
        % Sum of bases = Sum of legs = Perimeter / 2 = 50.
        % a + b = 50. a - b = 14.
        % 2a = 64 -> a = 32 (bottom). b = 18 (top).
        % c = 25.
        % Height h = sqrt(c^2 - ((a-b)/2)^2) = sqrt(25^2 - 7^2) = sqrt(625-49) = sqrt(576) = 24.
        % Radius = 12.
        
        \coordinate (O) at (0,12);
        \coordinate (A) at (-16,0);
        \coordinate (D) at (16,0);
        \coordinate (B) at (-9,24);
        \coordinate (C) at (9,24);
        
        \draw[thick] (A) -- (B) -- (C) -- (D) -- cycle;
        \draw[thick] (O) circle (12cm);
        
    \end{tikzpicture}
    \end{flushright}
\end{minipage}

\vspace{0.3cm}

\matchingLayout{
    \textit{Початок речення} \par \vspace{0.2cm}
    \textbf{1} \quad Довжина середньої лінії \par \quad трапеції \\
    \textbf{2} \quad Довжина більшої основи \par \quad трапеції \\
    \textbf{3} \quad Довжина висоти трапеції
}{
    \textit{Закінчення речення} \par \vspace{0.2cm}
    \begin{tabular}{ll}
    \textbf{А} & дорівнює 18 \textit{см}. \\
    \textbf{Б} & дорівнює 24 \textit{см}. \\
    \textbf{В} & дорівнює 25 \textit{см}. \\
    \textbf{Г} & дорівнює 32 \textit{см}. \\
    \textbf{Д} & дорівнює 36 \textit{см}. \\
    \end{tabular}
}{
    \answerGrid
}

\vspace{0.7cm}

% === ЗАВДАННЯ 8 ===
\noindent\textbf{32.} \begin{minipage}[t]{0.55\textwidth}
На рисунку зображено ромб $ABCD$, у який вписано коло з центром у точці $O$. З тупого кута $B$ на сторону $AD$ проведено висоту $BK$, коло дотикається до сторони $AD$ у точці $M$. $AK = 7$ \textit{см}, $KM = 9$ \textit{см}. До кожного відрізка (1--3) доберіть його довжину (А--Д). \nmtyear{2024}
\end{minipage}
\hfill
\begin{minipage}[t]{0.4\textwidth}
    \vspace{-0.5cm}
    \begin{flushright}
    \begin{tikzpicture}[scale=0.25]
        % AK = 7, KM = 9. M - точка дотику.
        % У ромбі точка дотику M ділить сторону.
        % Центр O проектується в M. OM perp AD.
        % BK - висота. BK || OM. OM = R. BK = 2R.
        % У трикутнику OAM...
        % Нехай це просто рисунок.
        % A=(0,0). K=(7,0). M=(16,0). (AK+KM=16).
        % Висота h. B = (7, h).
        % AB^2 = AK^2 + BK^2 = 49 + h^2.
        % Також AM = AB - MD? Ні. Властивість дотичної: AM = ...
        % Для малювання: приймемо h = 24 (наприклад).
        
        \coordinate (A) at (0,0);
        \coordinate (K) at (7,0);
        \coordinate (M) at (10.5,0);
        \coordinate (D) at (14,0); % AD прибл.
        \coordinate (B) at (7,12); % Висота прибл.
        \coordinate (C) at (21,12);
        
        % Центр O: середина висоти (y=9), по x: M=(16,0) -> O=(16,9).
        \coordinate (O) at (10.5,6);
        
        \draw[thick] (A) -- (B) -- (C) -- (D) -- cycle;
        \draw[thick] (B) -- (K);
        \draw[thick] (O) circle (6cm);
        
        % Прямий кут
        \pic [draw, angle radius=0.3cm] {right angle = B--K--D};
        
        \node[left] at (A) {$A$};
        \node[above] at (B) {$B$};
        \node[right] at (C) {$C$};
        \node[below] at (D) {$D$};
        \node[below] at (K) {$K$};
        \node[below] at (M) {$M$};
        \node[right] at (O) {$O$};
        
        \fill (O) circle (4pt);
        \fill (M) circle (4pt);
    \end{tikzpicture}
    \end{flushright}
\end{minipage}

\vspace{0.3cm}

\matchingLayout{
    \textit{Відрізок} \par \vspace{0.2cm}
    \textbf{1} \quad $AD$ \\
    \textbf{2} \quad $BK$ \\
    \textbf{3} \quad $OM$
}{
    \textit{Довжина відрізка} \par \vspace{0.2cm}
    \begin{tabular}{ll}
    \textbf{А} & 12 \textit{см} \\
    \textbf{Б} & 15 \textit{см} \\
    \textbf{В} & 20 \textit{см} \\
    \textbf{Г} & 24 \textit{см} \\
    \textbf{Д} & 25 \textit{см} \\
    \end{tabular}
}{
    \answerGrid
}

\vspace{0.7cm}

\noindent\textbf{33.} \begin{minipage}[t]{0.55\textwidth}
На рисунку зображено квадрат $ABCD$, площа якого $144$ \textit{см}$^2$. Точки $K$ і $M$ --- середини сторін $BC$ і $CD$ відповідно. До кожного відрізка (1--3) доберіть його довжину (А--Д). \nmtyear{2024}
\end{minipage}
\hfill
\begin{minipage}[t]{0.4\textwidth}
    \vspace{-0.5cm}
    \begin{flushright}
    \begin{tikzpicture}[scale=0.35]
        \coordinate (A) at (0,0);
        \coordinate (B) at (0,10);
        \coordinate (C) at (10,10);
        \coordinate (D) at (10,0);
        
        \coordinate (K) at (5,10);
        \coordinate (M) at (10,5);
        
        \draw[thick] (A) -- (B) -- (C) -- (D) -- cycle;
        \draw[thick] (K) -- (M);
        \draw[thick] (A) -- (K); % Для краси, хоча в умові не питають, але на рисунку є лінії
        
        % Позначки середин
        \draw (2.5, 9.8) -- (2.5, 10.2);
        \draw (7.5, 9.8) -- (7.5, 10.2);
        
        \draw (9.8, 7.5) -- (10.2, 7.5);
        \draw (9.8, 2.5) -- (10.2, 2.5);
        
        \node[below left] at (A) {$A$};
        \node[above left] at (B) {$B$};
        \node[above right] at (C) {$C$};
        \node[below right] at (D) {$D$};
        \node[above] at (K) {$K$};
        \node[right] at (M) {$M$};
        
        \fill (K) circle (5pt);
        \fill (M) circle (5pt);
    \end{tikzpicture}
    \end{flushright}
\end{minipage}

\vspace{0.3cm}

\matchingLayout{
    \textit{Відрізок} \par \vspace{0.2cm}
    \textbf{1} \quad сторона квадрата \\
    \textbf{2} \quad $KM$ \\
    \textbf{3} \quad відстань від точки $A$ до центра кола, описаного навколо трикутника $KMC$
}{
    \textit{Довжина відрізка} \par \vspace{0.2cm}
    \begin{tabular}{ll}
    \textbf{А} & 6 \textit{см} \\
    \textbf{Б} & $6\sqrt{2}$ \textit{см} \\
    \textbf{В} & 12 \textit{см} \\
    \textbf{Г} & $8\sqrt{2}$ \textit{см} \\
    \textbf{Д} & $9\sqrt{2}$ \textit{см} \\
    \end{tabular}
}{
    \answerGrid
}

\vspace{0.7cm}

\noindent\textbf{34.} \begin{minipage}[t]{0.55\textwidth}
На рисунку зображено квадрат $ABCD$ і прямокутний трикутник $KBC$ ($\angle B = 90^\circ$), що лежать в одній площині. Периметр квадрата $ABCD$ дорівнює $24$ \textit{см}, середня лінія трапеції $AKCD$ дорівнює $10$ \textit{см}. До кожного відрізка (1--3) доберіть його довжину (А--Д). \nmtyear{2024}
\end{minipage}
\hfill
\begin{minipage}[t]{0.4\textwidth}
    \vspace{-0.5cm}
    \begin{flushright}
    \begin{tikzpicture}[scale=0.3]
        % P=24 => сторона 6.
        % Трапеція AKCD. Основи AK і CD. CD=6. Середня лінія 10 -> AK+6 = 20 -> AK=14.
        % BK = 14-6 = 8.
        \coordinate (A) at (0,0);
        \coordinate (D) at (6,0);
        \coordinate (C) at (6,6);
        \coordinate (B) at (0,6);
        \coordinate (K) at (0,14);
        
        \draw[thick] (A) -- (D) -- (C) -- (K) -- cycle; % Контур трапеції
        \draw[thick] (B) -- (C); % Верхня сторона квадрата
        
        % Прямий кут
        \draw (0,6.5) -- (0.5,6.5) -- (0.5,6);
        
        \node[left] at (A) {$A$};
        \node[below right] at (D) {$D$};
        \node[right] at (C) {$C$};
        \node[left] at (B) {$B$};
        \node[left] at (K) {$K$};
    \end{tikzpicture}
    \end{flushright}
\end{minipage}

\vspace{0.3cm}

\matchingLayout{
    \textit{Відрізок} \par \vspace{0.2cm}
    \textbf{1} \quad $BK$ \\
    \textbf{2} \quad $KC$ \\
    \textbf{3} \quad відстань між центрами кіл, описаних навколо квадрата $ABCD$ та трикутника $KBC$
}{
    \textit{Довжина відрізка} \par \vspace{0.2cm}
    \begin{tabular}{ll}
    \textbf{А} & 6 \textit{см} \\
    \textbf{Б} & 7 \textit{см} \\
    \textbf{В} & 8 \textit{см} \\
    \textbf{Г} & 9 \textit{см} \\
    \textbf{Д} & 10 \textit{см} \\
    \end{tabular}
}{
    \answerGrid
}


\vspace{0.5cm}

% === ЗАВДАННЯ 30 ===
\noindent\textbf{35.} \begin{minipage}[t]{0.55\textwidth}
Навколо кола радіуса $4$ \textit{см} описано рівнобічну трапецію, середня лінія якої дорівнює $10$ \textit{см} (див. рисунок). До кожного відрізка (1--3) доберіть його довжину (А--Д).\nmtyear{2024}
\end{minipage}
\hfill
\begin{minipage}[t]{0.4\textwidth}
    \vspace{-0.5cm}
    \begin{flushright}
    \begin{tikzpicture}[scale=0.18]
        % R=4, h=8. Midline=10 => a+b=20.
        % Inscribed => a+b = 2c => 2c=20 => c=10.
        % Leg c=10. Height h=8. Projection x=6.
        % b = a - 12. 2a - 12 = 20 => 2a=32 => a=16. b=4.
        
        \coordinate (O) at (0,4);
        \coordinate (A) at (-8,0);
        \coordinate (D) at (8,0);
        \coordinate (B) at (-2,8);
        \coordinate (C) at (2,8);
        
        \draw[thick] (A) -- (B) -- (C) -- (D) -- cycle;
        \draw[thick] (O) circle (4cm);
        
        % Візуально:
        % Нижня основа 16 (від -8 до 8).
        % Верхня основа 4 (від -2 до 2).
        % Висота 8.
        
    \end{tikzpicture}
    \end{flushright}
\end{minipage}

\vspace{0.3cm}

\matchingLayout{
    \textit{Відрізок} \par \vspace{0.2cm}
    \textbf{1} \quad Висота трапеції \\
    \textbf{2} \quad Бічна сторона трапеції \\
    \textbf{3} \quad Більша основа трапеції
}{
    \textit{Довжина відрізка} \par \vspace{0.2cm}
    \begin{tabular}{ll}
    \textbf{А} & 8 \textit{см} \\
    \textbf{Б} & 10 \textit{см} \\
    \textbf{В} & 12 \textit{см} \\
    \textbf{Г} & 16 \textit{см} \\
    \textbf{Д} & 20 \textit{см} \\
    \end{tabular}
}{
    \answerGrid
}

% === ЗАВДАННЯ 23 ===
\noindent\textbf{36.} \begin{minipage}[t]{0.55\textwidth}
Точка $O$ --- центр кола, зображеного на рисунку. До цього кола проведено дотичну $KB$, $B$ --- точка дотику. На колі вибрано точки $A$ і $C$ так, що $AB = BC$, $\angle ABC = 50^\circ$. Установіть відповідність між кутом (1--3) та його градусною мірою (А--Д). \nmtyear{2024}
\end{minipage}
\hfill
\begin{minipage}[t]{0.4\textwidth}
    \vspace{-0.5cm}
    \begin{flushright}
    \begin{tikzpicture}[scale=0.7]
        \coordinate (O) at (0,0);
        \draw[thick] (O) circle (1.5cm);
        
        % B at top (90 deg)
        \coordinate (B) at (90:1.5);
        \coordinate (K) at (-2.5, 1.5);
        \draw[thick] (K) -- (2.5, 1.5); % Tangent line
        
        % Inscribed isosceles triangle ABC. AB=BC.
        % Angle ABC = 50.
        % Arc AC corresponds to 50? No, inscribed angle B=50 means Arc AC = 100.
        % Remaining arc = 260. Arc AB = Arc BC = 130.
        % B is at 90. A is at 90+130 = 220. C is at 90-130 = -40 (320).
        \coordinate (A) at (220:1.5);
        \coordinate (C) at (320:1.5);
        
        \draw[thick] (A) -- (B) -- (C);
        \draw[thick] (A) -- (O);
        \draw[thick] (C) -- (O);
        
        
        % Angle 50
        \pic [draw, pic text={\scriptsize $50^\circ$}, angle radius=0.4cm, angle eccentricity=1.8] {angle = A--B--C};
        
        % Marks for AB = BC
        \draw[thick] ($(A)!0.5!(B)$) ++(160:0.1) -- ++(-20:0.2);
        \draw[thick] ($(B)!0.5!(C)$) ++(20:0.1) -- ++(-160:0.2);
        
        \node[above] at (K) {$K$};
        \node[above] at (B) {$B$};
        \node[below left] at (A) {$A$};
        \node[below right] at (C) {$C$};
        \node[below] at (O) {$O$};
        
        \fill (O) circle (1.5pt);
        \fill (A) circle (1.5pt);
        \fill (B) circle (1.5pt);
        \fill (C) circle (1.5pt);
        
    \end{tikzpicture}
    \end{flushright}
\end{minipage}

\vspace{0.3cm}

\matchingLayout{
    \textit{Кут} \par \vspace{0.2cm}
    \textbf{1} \quad $\angle AOC$ \\
    \textbf{2} \quad $\angle BOC$ \\
    \textbf{3} \quad $\angle KBA$
}{
    \textit{Градусна міра кута} \par \vspace{0.2cm}
    \begin{tabular}{ll}
    \textbf{А} & $50^\circ$ \\
    \textbf{Б} & $65^\circ$ \\
    \textbf{В} & $100^\circ$ \\
    \textbf{Г} & $90^\circ$ \\
    \textbf{Д} & $130^\circ$ \\
    \end{tabular}
}{
    \answerGrid
}

\vspace{0.7cm}

% === ЗАВДАННЯ 24 ===
\noindent\makebox[1.5em][l]{\textbf{37.}}\parbox[t]{\dimexpr\textwidth-1.5em}{Які з наведених тверджень є правильними?
\begin{enumerate}[label=\Roman*., nosep, leftmargin=*]
    \item Бісектриса будь-якого трикутника ділить його протилежну сторону навпіл.
    \item Точка перетину бісектрис трикутника є центром вписаного кола.
    \item У рівнобедреному трикутнику одна з бісектрис утворює два рівні трикутники.
\end{enumerate}
\hfill \nmtyear{2024}}

\vspace{0.3cm}
\answerTable{лише II та III}{I, II та III}{лише II}{лише III}{лише I та III}

\noindent\textbf{38.} \begin{minipage}[t]{0.55\textwidth}
На рисунку зображено прямокутник $ABCD$ та два кола. Перше коло з центром у точці $O_1$, описане навколо цього прямокутника, друге коло з центром у точці $O_2$, довжиною $16\pi$ \textit{см}, дотикається до сторін $AB$, $BC$ та $AD$. $BC = 30$ \textit{см}. До кожного початку речення (1--3) доберіть його закінчення (А--Д) так, щоб утворилося правильне твердження. \nmtyear{2024}
\end{minipage}
\hfill
\begin{minipage}[t]{0.4\textwidth}
    \vspace{-0.5cm}
    \begin{flushright}
    \begin{tikzpicture}[scale=0.12]
        % Circle 2 length 16pi -> R=8. Diameter 16.
        % Touches AB, BC, AD -> Height AB = 16.
        % Width BC = 30.
        \def\w{30}
        \def\h{16}
        \coordinate (A) at (0,0);
        \coordinate (B) at (0,\h);
        \coordinate (C) at (\w,\h);
        \coordinate (D) at (\w,0);
        
        % Center O2: r=8 from AB, r=8 from AD. (8,8)
        \coordinate (O2) at (8,8);
        
        % Center O1: Midpoint of rect (15,8)
        \coordinate (O1) at (15,8);
        
        % Circumcircle radius: sqrt(15^2+8^2) = 17.
        
        % Draw Circumcircle (O1) - Green
        \draw[thick, green!60!black] (O1) circle (17);
        
        % Draw Inscribed-like Circle (O2) - Orange
        \draw[thick, orange!80!black] (O2) circle (8);
        
        % Draw Rectangle
        \draw[thick] (A) -- (B) -- (C) -- (D) -- cycle;
        
        % Points
        \fill (O1) circle (15pt) node[right] {$O_1$};
        \fill (O2) circle (15pt) node[left] {$O_2$};
        
        % Labels
        \node[below left] at (A) {$A$};
        \node[above left] at (B) {$B$};
        \node[above right] at (C) {$C$};
        \node[below right] at (D) {$D$};
        \node[above] at (15, \h) {30 \textit{см}};
    \end{tikzpicture}
    \end{flushright}
\end{minipage}

\vspace{0.3cm}

\matchingLayout{
    \textit{Початок речення} \par \vspace{0.2cm}
    \textbf{1} \quad Довжина сторони $AB$ дорівнює \\
    \textbf{2} \quad Довжина радіуса кола, описаного навколо прямокутника $ABCD$ дорівнює \\
    \textbf{3} \quad Довжина відрізка $O_1O_2$ дорівнює
}{
    \textit{Закінчення речення} \par \vspace{0.2cm}
    \begin{tabular}{ll}
    \textbf{А} & 7 \textit{см}. \\
    \textbf{Б} & 9 \textit{см}. \\
    \textbf{В} & 12 \textit{см}. \\
    \textbf{Г} & 16 \textit{см}. \\
    \textbf{Д} & 17 \textit{см}. \\
    \end{tabular}
}{
    \answerGrid
}

\vspace{0.7cm}

% === ЗАВДАННЯ 9 ===
\noindent\textbf{39.} \begin{minipage}[t]{0.95\textwidth}
Установіть відповідність між геометричною фігурою (1--3) та радіусом кола (А--Д), вписаного в цю фігуру. \nmtyear{2024}
\end{minipage}

\vspace{0.3cm}

\matchingLayout{
    \textit{Геометрична фігура} \par \vspace{0.2cm}
    \textbf{1} \quad ромб з висотою $4$ \textit{см} \\
    \textbf{2} \quad трикутник з площею $24$ \textit{см}$^2$ та периметром $12$ \textit{см} \\
    \textbf{3} \quad квадрат з периметром $64$ \textit{см}
}{
    \textit{Радіус кола, вписаного у фігуру} \par \vspace{0.2cm}
    \begin{tabular}{ll}
    \textbf{А} & 4 \textit{см} \\
    \textbf{Б} & $\sqrt{3}$ \textit{см} \\
    \textbf{В} & 8 \textit{см} \\
    \textbf{Г} & 6 \textit{см} \\
    \textbf{Д} & 2 \textit{см} \\
    \end{tabular}
}{
    \answerGrid
}


% === ЗАВДАННЯ 9 ===
\noindent\textbf{40.} \begin{minipage}[t]{0.95\textwidth}
Круг, площа якого $36\pi$, дотикається до паралельних прямих $m$ і $n$ (див. рисунок). Точки $L, N, P$ належать прямій $m$, а точки $K, M, Q$ --- прямій $n$. Трикутник $KLM$ рівносторонній. $MNPQ$ --- ромб, площа якого $156$. Установіть відповідність між відрізом (1--3) та його довжиною (А--Д). \nmtyear{2024}
\end{minipage}

\vspace{0.3cm}
\begin{center}

    
    \begin{tikzpicture}[scale=0.35]
        \def\h{12}
        \draw[thick] (-2,0) -- (50,0) node[above] {$n$};
        \draw[thick] (-2,\h) -- (50,\h) node[above] {$m$};
        
        % Коло
        \draw[thick] (6,6) circle (6cm);
        
        % Трикутник
        \coordinate (K) at (14,0);
        \coordinate (M) at (24,0); % Схематично
        \coordinate (L) at (19,\h);
        \draw[thick] (K) -- (L) -- (M) -- cycle;
        
        % Ромб
        % M спільна точка? Ні, точки K, M, Q на прямій.
        % M - вершина трикутника і ромба? Так, MNPQ.
        \coordinate (N) at (29,\h);
        \coordinate (Q) at (37,0); % Сторона 13
        \coordinate (P) at (42,\h);
        
        \draw[thick] (M) -- (N) -- (P) -- (Q) -- cycle;
        
        % Підписи
        \node[below] at (K) {$K$};
        \node[above] at (L) {$L$};
        \node[below] at (M) {$M$};
        \node[above] at (N) {$N$};
        \node[above] at (P) {$P$};
        \node[below] at (Q) {$Q$};
        
    \end{tikzpicture}
\end{center}

\matchingLayout{
    \textit{Відрізок} \par \vspace{0.2cm}
    \textbf{1} \quad діаметр круга \\
    \textbf{2} \quad довжина сторони трикутника $KLM$ \\
    \textbf{3} \quad довжина сторони ромба $MNPQ$
}{
    \textit{Довжина відрізка} \par \vspace{0.2cm}
    \begin{tabular}{ll}
    \textbf{А} & $8\sqrt{3}$ \\
    \textbf{Б} & 6 \\
    \textbf{В} & 12 \\
    \textbf{Г} & 13 \\
    \textbf{Д} & 15 \\
    \end{tabular}
}{
    \answerGrid
}

\vspace{0.7cm}


% === ЗАВДАННЯ 43 ===
\noindent\textbf{41.} \begin{minipage}[t]{0.55\textwidth}
На рисунку зображено прямокутник $ABCD$ і кругові сектори $BCL$ та $KAD$, що мають одну спільну точку $M$. $N$ --- проєкція точки $M$ на пряму $AB$, $BC = 12$ \textit{см}. До кожного початку речення (1--3) доберіть його закінчення (А--Д) так, щоб утворилося правильне твердження. \nmtyear{2024}
\end{minipage}
\hfill
\begin{minipage}[t]{0.4\textwidth}
    \vspace{-0.5cm}
    \begin{flushright}
    \begin{tikzpicture}[scale=0.25]
        % Centers are A and C. Radius 12.
        % AB = 12*sqrt(3) approx 20.8
        \def\r{12}
        \def\w{20.78}
        \coordinate (A) at (0,0);
        \coordinate (D) at (\w,0);
        \coordinate (B) at (0,\r); % BC is width? No, BC=12. So height is 12?
        % Text: BC=12. Drawing shows BC as top side. Usually AB is height, BC width.
        % If BC=12 is width. Then Rect is 12 wide.
        % Let's assume standard orientation: AB vertical, BC horizontal.
        % Then Center C (top right). Radius CB = 12 (height).
        % Center A (bottom left). Radius AD = 12 (width).
        % Touch at M. Diagonal AC = 24.
        % AC^2 = AB^2 + BC^2 => 576 = AB^2 + 144 => AB^2 = 432 => AB = 12sqrt(3).
        % So Height AB = 20.8, Width BC = 12.
        
        \coordinate (A) at (0,0);
        \coordinate (B) at (0,20.78);
        \coordinate (C) at (12,20.78);
        \coordinate (D) at (12,0);
        
        % Fill gray
        \fill[white!30] (A) -- (B) -- (C) -- (D) -- cycle;
        
        % Sector KAD (Center A, radius 12?) No, K is on AB.
        % Drawing shows Sector KAD. K on AB. D is corner.
        % If center is A, radius is AD = 12.
        % Arc from D to K.
        \coordinate (K) at (0, 12);
        \fill[gray!30] (A) -- (D) arc (0:90:12) -- cycle;
        \draw[thick] (D) arc (0:90:12);
        
        % Sector BCL (Center C, radius 12?) L is on CD.
        % Drawing shows Sector BCL. B is corner.
        % If center is C, radius CB = 12.
        % Arc from B to L.
        \coordinate (L) at (12, 8.78); % 20.78 - 12
        \fill[gray!30] (C) -- (B) arc (180:270:12) -- cycle;
        \draw[thick] (B) arc (180:270:12);
        
        % Point M (intersection)
        \coordinate (M) at (intersection of A--C and 0,12--12,12); % Rough
        % Exact M is midpoint of AC. (6, 10.39)
        \coordinate (M) at (6, 10.39);
        
        % Projection N on AB. N = (0, 10.39).
        \coordinate (N) at (0, 10.39);
        
        % Lines
        \draw[thick] (A) -- (B) -- (C) -- (D) -- cycle;
        \draw[thick] (A) -- (C); % Diagonal
        \draw[thick] (M) -- (N);
        
        % Right angle at N
        \draw (0, 10.39) rectangle ++(1, -1);
        
        % Labels
        \node[below left] at (A) {$A$};
        \node[above left] at (B) {$B$};
        \node[above right] at (C) {$C$};
        \node[below right] at (D) {$D$};
        \node[left] at (K) {$K$};
        \node[right] at (L) {$L$};
        \node[above] at (M) {$M$};
        \node[left] at (N) {$N$};
        
        \fill (M) circle (15pt);
        \fill (N) circle (15pt);
        \fill (K) circle (10pt);
        \fill (L) circle (10pt);

    \end{tikzpicture}
    \end{flushright}
\end{minipage}

\vspace{0.3cm}

\matchingLayout{
    \textit{Початок речення} \par \vspace{0.2cm}
    \textbf{1} \quad Довжина $AN$ \\
    \textbf{2} \quad Довжина $AB$ \\
    \textbf{3} \quad Довжина $AC$
}{
    \textit{Закінчення речення} \par \vspace{0.2cm}
    \begin{tabular}{ll}
    \textbf{А} & дорівнює 24 \textit{см}. \\
    \textbf{Б} & дорівнює 18 \textit{см}. \\
    \textbf{В} & дорівнює $8\sqrt{3}$ \textit{см}. \\
    \textbf{Г} & дорівнює $6\sqrt{3}$ \textit{см}. \\
    \textbf{Д} & є натуральним числом. \\
    \end{tabular}
}{
    \answerGrid
}

\vspace{0.7cm}

\begin{center}
{\Large\textbf{\color{headerblue}БАЗА ЗАВДАНЬ НМТ 2025}}
\end{center}

\noindent\textbf{42.} \begin{minipage}[t]{0.55\textwidth}
Діагональ $AC$ рівнобічної трапеції $ABCD$ утворює зі стороною $CD$ кут $120^\circ$ (див. рисунок). Довжини основ трапеції дорівнюють $12$ \textit{см} і $30$ \textit{см}. До кожного початку речення (1--3) доберіть його закінчення (А--Д) так, щоб утворилося правильне твердження. \nmtyear{2025}
\end{minipage}
\hfill
\begin{minipage}[t]{0.4\textwidth}
    \vspace{-0.5cm}
    \begin{flushright}
    \begin{tikzpicture}[scale=0.15]
        \coordinate (A) at (-15,0);
        \coordinate (D) at (15,0);
        % Bases 30 and 12. Top base centered.
        % Coordinates for C approx to look like the image.
        \coordinate (B) at (-6, 12);
        \coordinate (C) at (6, 12);
        
        \draw[thick] (A) -- (B) -- (C) -- (D) -- cycle;
        \draw[thick] (A) -- (C);
        
        % Obtuse angle 120 at C (ACD)
        \pic [draw, pic text={\scriptsize $120^\circ$}, angle radius=0.3cm, angle eccentricity=1.5] {angle = A--C--D};
        
        \node[below left] at (A) {$A$};
        \node[above left] at (B) {$B$};
        \node[above] at (C) {$C$};
        \node[below right] at (D) {$D$};
    \end{tikzpicture}
    \end{flushright}
\end{minipage}

\vspace{0.3cm}

\matchingLayout{
    \textit{Початок речення} \par \vspace{0.2cm}
    \textbf{1} \quad Довжина середньої лінії трапеції \par \quad дорівнює \\
    \textbf{2} \quad Довжина проєкції сторони $AB$ \par \quad на пряму $AD$ дорівнює \\
    \textbf{3} \quad Довжина радіуса кола, описаного \par \quad навколо трапеції, дорівнює
}{
    \textit{Закінчення речення} \par \vspace{0.2cm}
    \begin{tabular}{ll}
    \textbf{А} & 9 \textit{см}. \\
    \textbf{Б} & $20\sqrt{3}$ \textit{см}. \\
    \textbf{В} & 21 \textit{см}. \\
    \textbf{Г} & $10\sqrt{3}$ \textit{см}. \\
    \textbf{Д} & 18 \textit{см}. \\
    \end{tabular}
}{
    \answerGrid
}

\vspace{0.7cm}




% === ЗАВДАННЯ 25 ===
\noindent\makebox[1.5em][l]{\textbf{43.}}\parbox[t]{\dimexpr\textwidth-1.5em}{Які з наведених тверджень є правильними?
\begin{enumerate}[label=\Roman*., nosep, leftmargin=*]
    \item Існує прямокутна трапеція, навколо якої можна описати коло.
    \item Існує прямокутна трапеція, у яку можна вписати коло.
    \item Існує прямокутна трапеція, висота якої вдвічі менша за більшу бічну сторону.
\end{enumerate}
\hfill \nmtyear{2025}}

\vspace{0.3cm}
\answerTable{лише III}{лише I}{лише II}{лише II і III}{I, II та III}

\vspace{0.7cm}

% === ЗАВДАННЯ 26 ===
\noindent\textbf{44.} \begin{minipage}[t]{0.55\textwidth}
На рисунку зображено ескіз емблеми. Емблема має форму кола з центром у точці $O$. Знайдіть градусну міру кута $AOB$, якщо точки $A, B, C$ поділяють коло на три рівні частини. \nmtyear{2025}
\end{minipage}
\hfill
\begin{minipage}[t]{0.4\textwidth}
    \vspace{-0.5cm}
    \begin{flushright}
    \begin{tikzpicture}[scale=0.7]
        \coordinate (O) at (0,0);
        \draw[thick] (O) circle (1.5cm);
        
        % 3 equal parts -> 120 degrees apart
        % Let B be at 90. A at 90+120=210. C at 90-120=-30 (330).
        \coordinate (B) at (90:1.5);
        \coordinate (A) at (210:1.5);
        \coordinate (C) at (330:1.5);
        
        \draw[thick] (O) -- (A);
        \draw[thick] (O) -- (B);
        \draw[thick] (O) -- (C);
        
        \node at (O) [below =-2pt] {$O$};
        \node at (A) [below left] {$A$};
        \node at (B) [above] {$B$};
        \node at (C) [below right] {$C$};
        
        \fill (O) circle (1.5pt);
        \fill (A) circle (1.5pt);
        \fill (B) circle (1.5pt);
        \fill (C) circle (1.5pt);
        
    \end{tikzpicture}
    \end{flushright}
\end{minipage}

\vspace{0.3cm}
\answerTable{$120^\circ$}{$100^\circ$}{$60^\circ$}{$135^\circ$}{$180^\circ$}

\vspace{0.7cm}

% === ЗАВДАННЯ 27 ===
\noindent\makebox[1.5em][l]{\textbf{45.}}\parbox[t]{\dimexpr\textwidth-1.5em}{Які з наведених тверджень є правильними?
\begin{enumerate}[label=\Roman*., nosep, leftmargin=*]
    \item У будь-якому ромбі діагоналі точкою перетину діляться навпіл.
    \item Периметр ромба дорівнює сумі його діагоналей.
    \item Висота ромба вдвічі більша за радіус уписаного в нього кола.
\end{enumerate}
\hfill \nmtyear{2025}}

\vspace{0.3cm}
\answerTable{лише I та III}{лише III}{лише I}{лише I та II}{I, II та III}

\vspace{0.7cm}

% === ЗАВДАННЯ 28 ===
\noindent\makebox[1.5em][l]{\textbf{46.}}\parbox[t]{\dimexpr\textwidth-1.5em}{Які з наведених тверджень є правильними?
\begin{enumerate}[label=\Roman*., nosep, leftmargin=*]
    \item Центр кола, уписаного в трапецію, лежить на середній лінії трапеції.
    \item Центр кола, уписаного в трапецію, збігається з точкою перетину діагоналей трапеції.
    \item Центр кола, описаного навколо трапеції, обов'язково знаходиться на її більшій основі.
\end{enumerate}
\hfill \nmtyear{2025}}

\vspace{0.3cm}
\answerTable{лише I та II}{лише II}{лише I}{лише III}{лише I та III}

\vspace{0.7cm}

% === ЗАВДАННЯ 29 ===
\noindent\makebox[1.5em][l]{\textbf{47.}}\parbox[t]{\dimexpr\textwidth-1.5em}{У трикутнику $ABC$ проведено бісектрису кута $A$. Які з наведених тверджень є правильними?
\begin{enumerate}[label=\Roman*., nosep, leftmargin=*]
    \item Будь-яка точка на бісектрисі кута $A$ рівновіддалена від сторін $AC$ і $AB$.
    \item Бісектриса кута $A$ ділить сторону $BC$ на дві рівні частини.
    \item Центр уписаного в трикутник кола лежить на бісектрисі кута $A$.
\end{enumerate}
\hfill \nmtyear{2025}}

\vspace{0.3cm}
\answerTable{лише II}{лише III}{лише I та II}{лише I та III}{лише I}

% === ЗАВДАННЯ 26 ===
\noindent\textbf{48.} \begin{minipage}[t]{0.55\textwidth}
На рисунку зображено прямокутник $ABCD$ та два кола. Перше коло з центром у точці $O_1$, описане навколо цього прямокутника. Площа круга, обмеженого колом з центром у точці $O_1$, дорівнює $625\pi$ \textit{см}$^2$. Друге коло з центром у точці $O_2$ дотикається до сторін $AB$, $BC$ та $AD$. $AB = 30$ \textit{см}. Узгодьте початок речення (1--3) із його закінченням (А--Д) так, щоб утворилося правильне твердження. \nmtyear{2025}
\end{minipage}
\hfill
\begin{minipage}[t]{0.4\textwidth}
    \vspace{-0.5cm}
    \begin{flushright}
    \begin{tikzpicture}[scale=0.1]
        % R_circ = 25 (Area 625pi). Diam = 50.
        % AB = 30. BC = sqrt(50^2 - 30^2) = 40.
        % O2 touches AB, BC, AD. Diameter = AB = 30. Radius r=15.
        
        \def\w{40}
        \def\h{30}
        \coordinate (A) at (0,0);
        \coordinate (B) at (0,\h);
        \coordinate (C) at (\w,\h);
        \coordinate (D) at (\w,0);
        
        \coordinate (O1) at (20,15); % Center of rect
        \coordinate (O2) at (15,15); % Center of inscribed-like circle (r=15)
        
        % Circumcircle (Green)
        \draw[thick, green!60!black] (O1) circle (25cm);
        
        % Inscribed-like circle (Orange)
        \draw[thick, orange!80!black] (O2) circle (15cm);
        
        % Rectangle
        \draw[thick] (A) -- (B) -- (C) -- (D) -- cycle;
        
        % Points
        \fill (O1) circle (20pt) node[right] {$O_1$};
        \fill (O2) circle (20pt) node[left] {$O_2$};
        
        % Labels
        \node[below left] at (A) {$A$};
        \node[above left] at (B) {$B$};
        \node[above right] at (C) {$C$};
        \node[below right] at (D) {$D$};
    \end{tikzpicture}
    \end{flushright}
\end{minipage}

\vspace{0.3cm}

\matchingLayout{
    \textit{Початок речення} \par \vspace{0.2cm}
    \textbf{1} \quad Відстань від $O_2$ до сторони $BC$ дорівнює \\
    \textbf{2} \quad Довжина сторони $AD$ дорівнює \\
    \textbf{3} \quad Довжина відрізка $O_1O_2$ дорівнює
}{
    \textit{Закінчення речення} \par \vspace{0.2cm}
    \begin{tabular}{ll}
    \textbf{А} & 5 \textit{см}. \\
    \textbf{Б} & 10 \textit{см}. \\
    \textbf{В} & 15 \textit{см}. \\
    \textbf{Г} & 25 \textit{см}. \\
    \textbf{Д} & 40 \textit{см}. \\
    \end{tabular}
}{
    \answerGrid
}

\vspace{0.7cm}

\vspace{0.7cm}

% === ЗАВДАННЯ 26 ===
\noindent\textbf{49.} \begin{minipage}[t]{0.55\textwidth}
Точка $O$ --- середина діагоналі $AC$ ромба $ABCD$ (див. рисунок). $AK$ --- висота ромба, $AK = 12$ \textit{см}, $BK = 5$ \textit{см}. Узгодьте відрізок (1--3) із його довжиною (А--Д). \nmtyear{2025}
\end{minipage}
\hfill
\begin{minipage}[t]{0.4\textwidth}
    \vspace{-0.5cm}
    \begin{flushright}
    \begin{tikzpicture}[scale=0.25]
        % Сторона 13 (5, 12, 13).
        % A зліва, C справа. B зверху.
        % AK висота на BC. K на BC.
        % B = (x, y). C = (x+13, y)? Ні, це ромб.
        % Нехай центр (0,0). AC горизонталь? Ні, на малюнку AC горизонталь.
        % Але AK висота.
        % Побудуємо за розмірами.
        
        \coordinate (A) at (-12,0);
        \coordinate (C) at (12,0);
        \coordinate (O) at (0,0);
        
        % Для малюнка схематично:
        \coordinate (B) at (0, 8); % Приблизно
        \coordinate (D) at (0,-8);
        
        % Точка K на стороні BC. AK перпендикулярно BC.
        % Це складно вирахувати точно для схеми, намалюємо "як бачимо".
        \coordinate (K) at (4, 5.3); % Точка на BC
        
        \draw[thick] (A) -- (B) -- (C) -- (D) -- cycle;
        \draw[thick] (A) -- (C);
        
        
        % Перемальовуємо без BD, додаємо AK
        
        \draw[thick] (A) -- (B) -- (C) -- (D) -- cycle;
        \draw[thick] (A) -- (C);
        \draw[thick] (A) -- (K);
        
        % Прямий кут AKB? Ні, AK висота -> кут AKB = 90? 
        % На малюнку K лежить на BC. Кут AKC = 90? Ні, AK perp BC.
        % Значить кут AKB = 90.
        
        
        % Позначки O
        \draw (-6, -0.5) -- (-6, 0.5); % AO
        \draw (6, -0.5) -- (6, 0.5);   % OC
        
        \node[left] at (A) {$A$};
        \node[above] at (B) {$B$};
        \node[right] at (C) {$C$};
        \node[below] at (D) {$D$};
        \node[above right] at (K) {$K$};
        \node[below] at (O) {$O$};
        \pic [draw, angle radius=0.3cm] {right angle = A--K--B};
        \fill (O) circle (10pt);
        
    \end{tikzpicture}
    \end{flushright}
\end{minipage}

\vspace{0.3cm}

\matchingLayout{
    \textit{Відрізок} \par \vspace{0.2cm}
    \textbf{1} \quad Сторона ромба \\
    \textbf{2} \quad Радіус кола, \par \quad уписаного в ромб \\
    \textbf{3} \quad $OK$
}{
    \textit{Довжина відрізка} \par \vspace{0.2cm}
    \begin{tabular}{ll}
    \textbf{А} & 6 \textit{см} \\
    \textbf{Б} & 12 \textit{см} \\
    \textbf{В} & $2\sqrt{13}$ \textit{см} \\
    \textbf{Г} & 13 \textit{см} \\
    \textbf{Д} & $4\sqrt{13}$ \textit{см} \\
    \end{tabular}
}{
    \answerGrid
}

% === ЗАВДАННЯ 29 ===
\noindent\textbf{50.} \begin{minipage}[t]{0.6\textwidth}
Коло із центром у точці $O$ дотикається трьох сторін прямокутника $ABCD$ (див. рисунок). Вершина $K$ прямокутного рівнобедреного трикутника $AKB$ належить колу. $AB = 12$ \textit{см}. Доберіть до кожного початку речення (1--3) його закінчення (А--Д) так, щоб утворилося правильне твердження. \nmtyear{2025}
\end{minipage}
\hfill
\begin{minipage}[t]{0.35\textwidth}
    \vspace{-0.5cm}
    \begin{flushright}
    \begin{tikzpicture}[scale=0.25]
        % AB=12. K is vertex of isosceles right triangle on AB.
        % Height of K from AB = AB/2 = 6.
        % Circle touches AB, BC, AD? No, touches BC, AD, CD (right side).
        % Drawing: AB is left side. Circle is inside right part.
        % Center O is at distance R=6 from BC and AD. y=6.
        % K is on the circle. K is at (6,6).
        % Circle touches BC, AD. Radius 6.
        % If K(6,6) is on circle, and O is (12,6), then radius is 6.
        % So Circle touches side CD? Maybe.
        
        \coordinate (A) at (0,0);
        \coordinate (B) at (0,12);
        \coordinate (C) at (18,12); % W=18
        \coordinate (D) at (18,0);
        
        \coordinate (K) at (6,6);
        \coordinate (O) at (12,6);
        
        \draw[thick] (A) -- (B) -- (C) -- (D) -- cycle;
        \draw[thick] (O) circle (6cm);
        \draw[thick] (A) -- (K) -- (B);
        
        % Right angle at K
        \pic [draw, angle radius=0.3cm] {right angle = B--K--A};
        
        % Ticks on AK, BK
        \draw ($(A)!0.5!(K)$) ++(135:0.4) -- ++(-45:0.8);
        \draw ($(B)!0.5!(K)$) ++(45:0.4) -- ++(-135:0.8);
        
        \node[below left] at (A) {$A$};
        \node[above left] at (B) {$B$};
        \node[above right] at (C) {$C$};
        \node[below right] at (D) {$D$};
        \node[right] at (K) {$K$};
        \fill (O) circle (10pt) node[right] {$O$};
        \fill (K) circle (10pt);
    \end{tikzpicture}
    \end{flushright}
\end{minipage}

\vspace{0.3cm}

\matchingLayout{
    \textit{Початок речення} \par \vspace{0.2cm}
    \textbf{1} \quad Довжина радіуса $OK$ кола дорівнює \\
    \textbf{2} \quad Довжина відрізка $BK$ дорівнює \\
    \textbf{3} \quad Відстань від точки $O$ до вершини $A$ дорівнює
}{
    \textit{Закінчення речення} \par \vspace{0.2cm}
    \begin{tabular}{ll}
    \textbf{А} & 6 \textit{см}. \\
    \textbf{Б} & 8 \textit{см}. \\
    \textbf{В} & $6\sqrt{2}$ \textit{см}. \\
    \textbf{Г} & $6\sqrt{5}$ \textit{см}. \\
    \textbf{Д} & 18 \textit{см}. \\
    \end{tabular}
}{
    \answerGrid
}

\vspace{0.7cm}

% === ЗАВДАННЯ 31 ===
\noindent\textbf{51.} \begin{minipage}[t]{0.55\textwidth}
На рисунку зображено ромб $ABCD$ й коло із центром у точці $O$, побудоване на діагоналі ромба $BD$ як на діаметрі. Точки $K$ й $N$ є точками перетину кола з діагоналлю $AC$ й стороною $CD$ відповідно. $AK = 5$ \textit{см}, $AO = 20$ \textit{см}. Доберіть до відрізка (1--3) його довжину (А--Д). \nmtyear{2025}
\end{minipage}
\hfill
\begin{minipage}[t]{0.4\textwidth}
    \vspace{-0.5cm}
    \begin{flushright}
    \begin{tikzpicture}[scale=0.18]
        \coordinate (A) at (-20, 0); % AO = 20 за масштабом
        \coordinate (C) at (20, 0);
        \coordinate (O) at (0, 0);
        
        % Радіус OK = AO - AK = 20 - 5 = 15.
        % Коло будується на BD як на діаметрі, отже BO = OD = R = 15.
        \coordinate (B) at (0, 15);
        \coordinate (D) at (0, -15);
        
        % Точка K - перетин кола з AC (зліва)
        \coordinate (K) at (-15, 0);
        
        % Точка N - перетин кола зі стороною CD
        % Це складно вирахувати точно, поставимо точку візуально на колі і стороні CD
        % Рівняння прямої CD: y - (-15) = (15/20)*(x - 0) -> y = 0.75x - 15.
        % Коло: x^2 + y^2 = 225.
        % Перетин N.
        \coordinate (N) at (14.5,-4); % Приблизні координати

        \draw[thick] (A) -- (B) -- (C) -- (D) -- cycle;
        \draw[thick] (A) -- (C);
        
        \draw[thick] (O) circle (15cm);

        \node[left] at (A) {$A$};
        \node[above] at (B) {$B$};
        \node[right] at (C) {$C$};
        \node[below] at (D) {$D$};
        \node[above] at (O) {$O$};
        \node[above right] at (K) {$K$};
        \node[right] at (N) {$N$};

        \fill (O) circle (12pt);
        \fill (K) circle (12pt);
        \fill (N) circle (12pt);
    \end{tikzpicture}
    \end{flushright}
\end{minipage}

\vspace{0.3cm}

\matchingLayout{
    \textit{Відрізок} \par \vspace{0.2cm}
    \textbf{1} \quad $BO$ \\
    \textbf{2} \quad $AB$ \\
    \textbf{3} \quad $BN$
}{
    \textit{Довжина відрізка} \par \vspace{0.2cm}
    \begin{tabular}{ll}
    \textbf{А} & 15 \textit{см} \\
    \textbf{Б} & 20 \textit{см} \\
    \textbf{В} & 24 \textit{см} \\
    \textbf{Г} & 25 \textit{см} \\
    \textbf{Д} & 30 \textit{см} \\
    \end{tabular}
}{
    \answerGrid
}

% === ЗАВДАННЯ 38 (Кольорові фігури) ===
\noindent\textbf{52.} \begin{minipage}[t]{0.95\textwidth}
На паралельних прямих $n$ і $m$ розміщено круговий сектор $ABC$, рівнобедрений трикутник $DKL$ ($DK=KL$) й паралелограм $LMNP$ (див. рисунок). Площа сектора $ABC$ дорівнює $64\pi$ \textit{см}$^2$, площа паралелограма $LMNP$ дорівнює $288$ \textit{см}$^2$, $DK = 20$ \textit{см}. Увідповідніть відрізок (1--3) та його довжину (А--Д). \nmtyear{2025}
\end{minipage}

\vspace{0.3cm}
\begin{center}
\begin{tikzpicture}[scale=0.25]
    % Висота між прямими h.
    % Сектор ABC: 1/4 кола. Площа = 64pi => R^2 = 256 => R=16. h=16.
    \def\h{16}
    
    % Координати
    % Сектор. Центр у точці B(0,16). A(0,0). C(16,16).
    \coordinate (A) at (0,0);
    \coordinate (B) at (0,\h);
    \coordinate (C) at (\h,\h);
    
    % Трикутник DKL. DK=20. Висота 16. Проєкція катета = sqrt(400-256) = 12.
    % Основа DL = 24.
    % D починається трохи правіше C. Нехай shift=4. D(20,0).
    \coordinate (D) at (20,0);
    \coordinate (K) at (32,\h); % 20+12
    \coordinate (L) at (44,0);  % 32+12
    
    % Паралелограм LMNP. Площа 288. Висота 16. Основа LP = 288/16 = 18.
    % L(44,0). P(62,0).
    % Зсув верхньої сторони MN довільний (паралелограм). Нехай зсув 6.
    % M(50,16). N(68,16).
    \coordinate (M) at (50,\h);
    \coordinate (P) at (62,0);
    \coordinate (N) at (68,\h);
    
    % Заливка
    % Сектор (центр B, радіус 16, кут 270..360)
    \fill[yellow!20] (B) -- (A) arc [start angle=270, end angle=360, radius=\h] -- cycle;
    
    % Трикутник
    \fill[cyan!20] (D) -- (K) -- (L) -- cycle;
    
    % Паралелограм
    \fill[red!20] (L) -- (M) -- (N) -- (P) -- cycle;
    
    % Лінії m та n
    \draw[thick] (-2,0) -- (70,0) node[above] {$n$};
    \draw[thick] (-2,\h) -- (70,\h) node[above] {$m$};
    
    % Контури фігур
    \draw[thick] (B) -- (A) arc [start angle=270, end angle=360, radius=\h] -- cycle;
    \draw[thick] (D) -- (K) -- (L) -- cycle;
    \draw[thick] (L) -- (M) -- (N) -- (P) -- cycle; % LMNP
    
    % Підписи
    \node[below] at (A) {$A$};
    \node[above] at (B) {$B$};
    \node[above] at (C) {$C$};
    \node[below] at (D) {$D$};
    \node[above] at (K) {$K$};
    \node[below] at (L) {$L$};
    \node[above] at (M) {$M$};
    \node[above] at (N) {$N$};
    \node[below] at (P) {$P$};
    
\end{tikzpicture}
\end{center}

\matchingLayout{
    \textbf{1} \quad $AB$ \\
    \textbf{2} \quad $DL$ \\
    \textbf{3} \quad $LP$
}{
    \begin{tabular}{ll}
    \textbf{А} & 12 \textit{см} \\
    \textbf{Б} & 16 \textit{см} \\
    \textbf{В} & 18 \textit{см} \\
    \textbf{Г} & 20 \textit{см} \\
    \textbf{Д} & 24 \textit{см} \\
    \end{tabular}
}{\answerGrid}

\vspace{0.7cm}


\end{document}