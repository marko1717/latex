\documentclass[14pt]{extarticle}
\usepackage{fontspec}
\usepackage{polyglossia}
\setdefaultlanguage{ukrainian}

\defaultfontfeatures{Ligatures=TeX}
\setmainfont{Liberation Serif}
\setsansfont{Liberation Sans}
\setmonofont{Liberation Mono}

\usepackage[a4paper,margin=1.5cm,bottom=2cm,top=2cm]{geometry}
\usepackage{amsmath,amssymb}
\usepackage{enumitem}
\usepackage{tikz}
\usepackage{pgfplots}
\pgfplotsset{compat=1.16}

% Підключаємо бібліотеки для зручних кутів + hobby для плавних кривих
\usetikzlibrary{calc,patterns,angles,quotes,intersections,babel,hobby}
\usetikzlibrary{3d}

\usepackage{xcolor}
\usepackage{array}
\usepackage{fancyhdr}
\usepackage{multirow}

% Кольори
\definecolor{headerblue}{RGB}{0, 102, 204}
\definecolor{yearcolor}{RGB}{128, 0, 128}

\pagestyle{fancy}
\fancyhf{}
\renewcommand{\headrulewidth}{0pt}
\fancyfoot[C]{\thepage}

\setlength{\headheight}{15pt}
\setlength{\headsep}{10pt}
\setlength{\footskip}{25pt}

\widowpenalty=10000
\clubpenalty=10000

% === КОМАНДИ ===

% Таблиця для відповідей із дробами (збільшена висота клітинок)
% Оновлена таблиця: підпорка додана до КОЖНОЇ клітинки
\newcommand{\answerTableTall}[5]{
\begin{center}
\begin{tabular}{|*{5}{>{\centering\arraybackslash}m{2.8cm}|}}
\hline
\rule[-0.3cm]{0pt}{0.8cm}\textbf{А} & \textbf{Б} & \textbf{В} & \textbf{Г} & \textbf{Д} \\
\hline
% Тепер rule є перед кожним аргументом (#1..#5)
\rule[-0.9cm]{0pt}{2.0cm}#1 & 
\rule[-0.9cm]{0pt}{2.0cm}#2 & 
\rule[-0.9cm]{0pt}{2.0cm}#3 & 
\rule[-0.9cm]{0pt}{2.0cm}#4 & 
\rule[-0.9cm]{0pt}{2.0cm}#5 \\
\hline
\end{tabular}
\end{center}
}

% Оновлена таблиця відповідей (заголовки зовні)
\newcommand{\answerGrid}{
    \begingroup
    % Збільшуємо висоту рядків для квадратних клітинок
    \renewcommand{\arraystretch}{1.3} 
    % Відступ всередині клітинок
    \setlength{\tabcolsep}{7pt} 
    \begin{tabular}{r|c|c|c|c|c|}
         % Перший рядок: порожня клітинка зліва + букви без рамок (multicolumn прибирає |)
         \multicolumn{1}{c}{} & \multicolumn{1}{c}{\textbf{А}} & \multicolumn{1}{c}{\textbf{Б}} & \multicolumn{1}{c}{\textbf{В}} & \multicolumn{1}{c}{\textbf{Г}} & \multicolumn{1}{c}{\textbf{Д}} \\ \cline{2-6}
         % Наступні рядки: номер зліва (r) + клітинки з рамками (|c|)
         \textbf{1} & & & & & \\ \cline{2-6}
         \textbf{2} & & & & & \\ \cline{2-6}
         \textbf{3} & & & & & \\ \cline{2-6}
    \end{tabular}
    \endgroup
}

% Макет для завдань на відповідність
% #1 - Умови (1-3)
% #2 - Варіанти (А-Д)
% #3 - Табличка
\newcommand{\matchingLayout}[3]{
    \noindent
    \begin{minipage}[t]{0.40\textwidth}
       
        #1
    \end{minipage}%
    \hfill
    \begin{minipage}[t]{0.28\textwidth}
        
        #2
    \end{minipage}%
    \hfill
    \begin{minipage}[t]{0.30\textwidth}
        \vspace{0pt} % Хаки для вирівнювання minipage по верху
        \begin{flushright}
        #3
        \end{flushright}
    \end{minipage}
}

% Стандартна таблиця відповідей (для тестів)
\newcommand{\answerTableSmall}[5]{
\begin{tabular}{|*{5}{>{\centering\arraybackslash}m{1.65cm}|}}
\hline
\rule[-0.2cm]{0pt}{0.6cm}\textbf{А} & \textbf{Б} & \textbf{В} & \textbf{Г} & \textbf{Д} \\
\hline
% Підпорка додана до кожного варіанту для ідеального вирівнювання
\rule[-0.4cm]{0pt}{0.9cm}#1 & 
\rule[-0.4cm]{0pt}{0.9cm}#2 & 
\rule[-0.4cm]{0pt}{0.9cm}#3 & 
\rule[-0.4cm]{0pt}{0.9cm}#4 & 
\rule[-0.4cm]{0pt}{0.9cm}#5 \\
\hline
\end{tabular}
}

% Таблиця для вибору одного варіанту (Task 7)
\newcommand{\answerTable}[5]{
\begin{center}
\begin{tabular}{|*{5}{>{\centering\arraybackslash}m{2.8cm}|}}
\hline
\rule[-0.3cm]{0pt}{0.8cm}\textbf{А} & \textbf{Б} & \textbf{В} & \textbf{Г} & \textbf{Д} \\
\hline
\rule[-0.4cm]{0pt}{1.0cm}#1 & \rule[-0.4cm]{0pt}{1.0cm}#2 & \rule[-0.4cm]{0pt}{1.0cm}#3 & \rule[-0.4cm]{0pt}{1.0cm}#4 & \rule[-0.4cm]{0pt}{1.0cm}#5 \\
\hline
\end{tabular}
\end{center}
}

% Команда для року
\newcommand{\nmtyear}[1]{\hfill{\small\color{yearcolor}(НМТ #1)}}

\begin{document}
\begin{center}
{\Large\textbf{\color{headerblue}БАЗА ЗАВДАНЬ НМТ 2023}}
\end{center}

\begin{center}
{\large Тема: \textbf{Функція та її властивості}}
\end{center}

\begin{center}
{\large Тема: \textbf{Функції}}
\end{center}

\vspace{0.5cm}

\begin{center}
{\Large\textbf{\color{headerblue}БАЗА ЗАВДАНЬ НМТ 2023}}
\end{center}

\begin{center}
{\large Тема: \textbf{Функції}}
\end{center}

% === ЗАВДАННЯ 1 ===
\noindent\textbf{1.} До кожного початку речення (1--3) доберіть його закінчення (А--Д) так, щоб утворилося правильне твердження. \nmtyear{2023}

\vspace{0.3cm}

\noindent
\begin{minipage}[t]{0.40\textwidth}
    \textit{Початок речення} \par \vspace{0.2cm}
    \textbf{1} \quad Функція $y = \log_{0{,}5} x$ \\[0.2cm]
    \textbf{2} \quad Функція $y = \sin x$ \\[0.2cm]
    \textbf{3} \quad Функція $y = \dfrac{1}{2x-2}$
    
    \vspace{0.5cm}
    
    % Зменшена табличка
    \begingroup
    \setlength{\tabcolsep}{4pt} % Менші відступи
    \renewcommand{\arraystretch}{1.2} % Трохи компактніша висота
    \small % Менший шрифт
    \begin{tabular}{r|c|c|c|c|c|}
         \multicolumn{1}{c}{} & \multicolumn{1}{c}{\textbf{А}} & \multicolumn{1}{c}{\textbf{Б}} & \multicolumn{1}{c}{\textbf{В}} & \multicolumn{1}{c}{\textbf{Г}} & \multicolumn{1}{c}{\textbf{Д}} \\ \cline{2-6}
         \textbf{1} & & & & & \\ \cline{2-6}
         \textbf{2} & & & & & \\ \cline{2-6}
         \textbf{3} & & & & & \\ \cline{2-6}
    \end{tabular}
    \endgroup
\end{minipage}%
\hfill
\begin{minipage}[t]{0.55\textwidth}
    \textit{Закінчення речення} \par \vspace{0.2cm}
    \begin{tabular}{ll}
    \textbf{А} & не визначена при $x=1$. \\
    \textbf{Б} & набуває від'ємного значення при $x=2$. \\
    \textbf{В} & є непарною. \\
    \textbf{Г} & має лише одну точку локального екстремуму. \\
    \textbf{Д} & зростає на проміжку $(0; +\infty)$. \\
    \end{tabular}
\end{minipage}

\vspace{0.7cm}

% === ЗАВДАННЯ 2 ===
\noindent\textbf{2.} Установіть відповідність між функцією (1--3) та її властивістю (А--Д). \nmtyear{2023}

\vspace{0.3cm}

\noindent
\begin{minipage}[t]{0.40\textwidth}
    \textit{Функція} \par \vspace{0.2cm}
    \textbf{1} \quad $y = 7x + 4$ \\[0.2cm]
    \textbf{2} \quad $y = -\dfrac{7}{x}$ \\[0.2cm]
    \textbf{3} \quad $y = \log_{0{,}5} (x-4)$
    
    \vspace{0.5cm}
    
    % Зменшена табличка
    \begingroup
    \setlength{\tabcolsep}{4pt}
    \renewcommand{\arraystretch}{1.2}
    \small
    \begin{tabular}{r|c|c|c|c|c|}
         \multicolumn{1}{c}{} & \multicolumn{1}{c}{\textbf{А}} & \multicolumn{1}{c}{\textbf{Б}} & \multicolumn{1}{c}{\textbf{В}} & \multicolumn{1}{c}{\textbf{Г}} & \multicolumn{1}{c}{\textbf{Д}} \\ \cline{2-6}
         \textbf{1} & & & & & \\ \cline{2-6}
         \textbf{2} & & & & & \\ \cline{2-6}
         \textbf{3} & & & & & \\ \cline{2-6}
    \end{tabular}
    \endgroup
\end{minipage}%
\hfill
\begin{minipage}[t]{0.55\textwidth}
    \textit{Властивість} \par \vspace{0.2cm}
    \begin{tabular}{l p{6.5cm}}
    \textbf{А} & є спадною на всій області визначення \\
    \textbf{Б} & графік функції перетинає вісь $y$ в точці з ординатою $4$ \\
    \textbf{В} & є непарною \\
    \textbf{Г} & є парною \\
    \textbf{Д} & областю визначення є проміжок $(0; +\infty)$ \\
    \end{tabular}
\end{minipage}

\vspace{0.7cm}

% === ЗАВДАННЯ 3 ===
\noindent\textbf{3.} До кожного початку речення (1--3) доберіть його закінчення (А--Д) так, щоб утворилося правильне твердження. \nmtyear{2023}

\vspace{0.3cm}

\noindent
\begin{minipage}[t]{0.40\textwidth}
    \textit{Початок речення} \par \vspace{0.2cm}
    \textbf{1} \quad Функція $y = \log_2 x$ \\[0.2cm]
    \textbf{2} \quad Функція $y = x^2 - 4x - 4$ \\[0.2cm]
    \textbf{3} \quad Функція $y = \dfrac{1}{x}$
    
    \vspace{0.5cm}
    
    % Зменшена табличка
    \begingroup
    \setlength{\tabcolsep}{4pt}
    \renewcommand{\arraystretch}{1.2}
    \small
    \begin{tabular}{r|c|c|c|c|c|}
         \multicolumn{1}{c}{} & \multicolumn{1}{c}{\textbf{А}} & \multicolumn{1}{c}{\textbf{Б}} & \multicolumn{1}{c}{\textbf{В}} & \multicolumn{1}{c}{\textbf{Г}} & \multicolumn{1}{c}{\textbf{Д}} \\ \cline{2-6}
         \textbf{1} & & & & & \\ \cline{2-6}
         \textbf{2} & & & & & \\ \cline{2-6}
         \textbf{3} & & & & & \\ \cline{2-6}
    \end{tabular}
    \endgroup
\end{minipage}%
\hfill
\begin{minipage}[t]{0.55\textwidth}
    \textit{Закінчення речення} \par \vspace{0.2cm}
    \begin{tabular}{ll}
    \textbf{А} & не визначена при $x = -2$. \\
    \textbf{Б} & набуває від'ємного значення при $x = 2$. \\
    \textbf{В} & є непарною. \\
    \textbf{Г} & спадає на проміжку $(-\infty; 4]$. \\
    \textbf{Д} & зростає на проміжку $(-\infty; +\infty)$. \\
    \end{tabular}
\end{minipage}

\vspace{0.7cm}

% === ЗАВДАННЯ 4 (Чверті 1) ===
\noindent\textbf{4.} \begin{minipage}[t]{0.55\textwidth}
Функція $y=f(x)$ визначена на проміжку $(-\infty; +\infty)$ і набуває лише додатних значень. Укажіть \textit{усі} координатні чверті (див. рисунок), у яких розташований графік цієї функції. \nmtyear{2023}
\end{minipage}
\hfill
\begin{minipage}[t]{0.4\textwidth}
    \vspace{-0.5cm}
    \begin{flushright}
    \begin{tikzpicture}[scale=0.8]
        \draw[->, >=stealth, thick] (-2.5,0) -- (2.5,0) node[below] {$x$};
        \draw[->, >=stealth, thick] (0,-2.5) -- (0,2.5) node[left] {$y$};
        \node[below left] at (0,0) {$O$};
        
        \node at (1.5,1.5) {I чверть};
        \node at (-1.5,1.5) {II чверть};
        \node at (-1.5,-1.5) {III чверть};
        \node at (1.5,-1.5) {IV чверть};
    \end{tikzpicture}
    \end{flushright}
\end{minipage}

\vspace{0.3cm}
\begin{tabular}{|*{5}{>{\centering\arraybackslash}m{2.8cm}|}}
\hline
\textbf{А} & \textbf{Б} & \textbf{В} & \textbf{Г} & \textbf{Д} \\
\hline
лише в I та IV & лише в II & лише в III та IV & лише в I та II & лише в I \\
\hline
\end{tabular}

\vspace{0.7cm}

% === ЗАВДАННЯ 5 (Чверті 2) ===
\noindent\textbf{5.} \begin{minipage}[t]{0.55\textwidth}
Функція $y=f(x)$ визначена на проміжку $(0; +\infty)$ і набуває лише додатних значень. Укажіть \textit{усі} координатні чверті (див. рисунок), у яких розташований графік цієї функції. \nmtyear{2023}
\end{minipage}
\hfill
\begin{minipage}[t]{0.4\textwidth}
    \vspace{-0.5cm}
    \begin{flushright}
    \begin{tikzpicture}[scale=0.8]
        \draw[->, >=stealth, thick] (-2.5,0) -- (2.5,0) node[below] {$x$};
        \draw[->, >=stealth, thick] (0,-2.5) -- (0,2.5) node[left] {$y$};
        \node[below left] at (0,0) {$O$};
        
        \node at (1.5,1.5) {I чверть};
        \node at (-1.5,1.5) {II чверть};
        \node at (-1.5,-1.5) {III чверть};
        \node at (1.5,-1.5) {IV чверть};
    \end{tikzpicture}
    \end{flushright}
\end{minipage}

\vspace{0.3cm}
\begin{tabular}{|*{5}{>{\centering\arraybackslash}m{2.8cm}|}}
\hline
\textbf{А} & \textbf{Б} & \textbf{В} & \textbf{Г} & \textbf{Д} \\
\hline
лише в I & лише в I та IV & лише в III та IV & лише в I та II & лише в II \\
\hline
\end{tabular}

\vspace{0.7cm}

% === ЗАВДАННЯ 6 (Чверті 3) ===
\noindent\textbf{6.} \begin{minipage}[t]{0.55\textwidth}
Функція $y=f(x)$ визначена на проміжку $(-\infty; +\infty)$ і набуває лише від'ємних значень. Укажіть \textit{усі} координатні чверті (див. рисунок), у яких розташований графік цієї функції. \nmtyear{2023}
\end{minipage}
\hfill
\begin{minipage}[t]{0.4\textwidth}
    \vspace{-0.5cm}
    \begin{flushright}
    \begin{tikzpicture}[scale=0.8]
        \draw[->, >=stealth, thick] (-2.5,0) -- (2.5,0) node[below] {$x$};
        \draw[->, >=stealth, thick] (0,-2.5) -- (0,2.5) node[left] {$y$};
        \node[below left] at (0,0) {$O$};
        
        \node at (1.5,1.5) {I чверть};
        \node at (-1.5,1.5) {II чверть};
        \node at (-1.5,-1.5) {III чверть};
        \node at (1.5,-1.5) {IV чверть};
    \end{tikzpicture}
    \end{flushright}
\end{minipage}

\vspace{0.3cm}
\begin{tabular}{|*{5}{>{\centering\arraybackslash}m{2.8cm}|}}
\hline
\textbf{А} & \textbf{Б} & \textbf{В} & \textbf{Г} & \textbf{Д} \\
\hline
лише в III та IV & лише в IV & лише в III & лише в I та IV & лише в I \\
\hline
\end{tabular}


\vspace{0.7cm}

% === ЗАВДАННЯ 7 ===
\noindent\textbf{7.} \begin{minipage}[t]{0.55\textwidth}
На рисунку зображено графік функції $y=f(x)$, визначеної на проміжку $[-3; 3]$. На якому з наведених проміжків ця функція зростає? \nmtyear{2023}
\end{minipage}
\hfill
\begin{minipage}[t]{0.4\textwidth}
    \vspace{-0.5cm}
    \begin{flushright}
    \begin{tikzpicture}[scale=0.5]
        \draw[step=1cm,gray!50,very thin] (-3.5,-2.5) grid (3.5,3.5);
        \draw[->, >=stealth, thick] (-3.5,0) -- (3.5,0) node[below] {$x$};
        \draw[->, >=stealth, thick] (0,-2.5) -- (0,3.5) node[left] {$y$};
        
        \node[below left] at (0,0) {$0$};
        \node[below] at (1,0) {$1$};
        \node[left] at (0,1) {$1$};
        \node[below] at (3,0) {$3$};
        \node[below] at (-3,0) {$-3$};
        \draw (1,0.1) -- (1,-0.1);
        \draw (3,0.1) -- (3,-0.1);
        \draw (-3,0.1) -- (-3,-0.1);
        \draw (0.1,1) -- (-0.1,1);
        
        % Використовуємо hobby для гарантованого проходження через точки
        \draw[thick] plot[hobby] coordinates {(-3,-2) (-1.8,0) (1,3) (3,1.5)};
        
        \fill (-3,-2) circle (3pt);
        \fill (3,1.5) circle (3pt);
        \node[above right] at (2,2.5) {$y=f(x)$};
    \end{tikzpicture}
    \end{flushright}
\end{minipage}

\vspace{0.3cm}
\answerTable{$[-3; 3]$}{$[-3; 1]$}{$[-2; 3]$}{$[-2; 4]$}{$[1; 3]$}

\vspace{0.7cm}

% === ЗАВДАННЯ 8 ===
\noindent\textbf{8.} \begin{minipage}[t]{0.55\textwidth}
На рисунку зображено графік функції $y=f(x)$, визначеної на проміжку $[-3; 3]$. Укажіть нуль цієї функції. \nmtyear{2023}
\end{minipage}
\hfill
\begin{minipage}[t]{0.4\textwidth}
    \vspace{-0.5cm}
    \begin{flushright}
    \begin{tikzpicture}[scale=0.5]
        \draw[step=1cm,gray!50,very thin] (-3.5,-2.5) grid (3.5,3.5);
        \draw[->, >=stealth, thick] (-3.5,0) -- (3.5,0) node[below] {$x$};
        \draw[->, >=stealth, thick] (0,-2.5) -- (0,3.5) node[left] {$y$};
        
        \node[below left] at (0,0) {$0$};
        \node[below] at (1,0) {$1$};
        \node[left] at (0,1) {$1$};
        \node[below] at (3,0) {$3$};
        \node[below] at (-3,0) {$-3$};
        \draw (1,0.1) -- (1,-0.1);
        \draw (3,0.1) -- (3,-0.1);
        \draw (-3,0.1) -- (-3,-0.1);
        \draw (0.1,1) -- (-0.1,1);
        
        % Той самий графік з hobby
        \draw[thick] plot[hobby] coordinates {(-3,-2) (-2,0) (1,3) (3,1.5)};
        
        \fill (-3,-2) circle (3pt);
        \fill (3,1.5) circle (3pt);
        \node[above right] at (2,2.5) {$y=f(x)$};
    \end{tikzpicture}
    \end{flushright}
\end{minipage}

\vspace{0.3cm}
\answerTable{$0$}{$3$}{$4$}{$-3$}{$-2$}

\vspace{0.7cm}

% === ЗАВДАННЯ 9 ===
\noindent\textbf{9.} \begin{minipage}[t]{0.55\textwidth}
На рисунку зображено графік функції $y=f(x)$, визначеної на проміжку $[-2; 4]$. Укажіть значення $x_0$, за якого $f(x_0) > 0$. \nmtyear{2023}
\end{minipage}
\hfill
\begin{minipage}[t]{0.4\textwidth}
    \vspace{-0.5cm}
    \begin{flushright}
    \begin{tikzpicture}[scale=0.5]
        \draw[step=1cm,gray!50,very thin] (-2.5,-2.5) grid (4.5,3.5);
        \draw[->, >=stealth, thick] (-2.5,0) -- (4.5,0) node[below] {$x$};
        \draw[->, >=stealth, thick] (0,-2.5) -- (0,3.5) node[left] {$y$};
        
        \node[below left] at (0,0) {$0$};
        \node[below] at (1,0) {$1$};
        \node[left] at (0,1) {$1$};
        \node[below] at (4,0) {$4$};
        \node[below] at (-2,0) {$-2$};
        
        \draw (1,0.1) -- (1,-0.1);
        \draw (4,0.1) -- (4,-0.1);
        \draw (-2,0.1) -- (-2,-0.1);
        \draw (0.1,1) -- (-0.1,1);
        
        % З hobby
        \draw[thick] plot[hobby] coordinates {(-2, 1.2) (-1, 3) (0, 2) (1, 0) (4, -1.8)};
        
        \fill (-2,1.2) circle (3pt);
        \fill (4,-1.8) circle (3pt);
        \node[above right] at (2,1) {$y=f(x)$};
    \end{tikzpicture}
    \end{flushright}
\end{minipage}

\vspace{0.3cm}
\answerTable{$x_0 = 2$}{$x_0 = 1$}{$x_0 = -1$}{$x_0 = 4$}{$x_0 = 3$}

\vspace{0.7cm}

% === ЗАВДАННЯ 10 ===
\noindent\textbf{10.} Установіть відповідність між функцією (1--3) та її властивістю (А--Д). \nmtyear{2023}

\vspace{0.3cm}

\noindent
\begin{minipage}[t]{0.40\textwidth}
    \textit{Функція} \par \vspace{0.2cm}
    \textbf{1} \quad $y = x^2 - 1$ \\[0.2cm]
    \textbf{2} \quad $y = 3^x$ \\[0.2cm]
    \textbf{3} \quad $y = x^3$
    
    \vspace{0.5cm}
    
    % Зменшена табличка
    \begingroup
    \setlength{\tabcolsep}{4pt}
    \renewcommand{\arraystretch}{1.2}
    \small
    \begin{tabular}{r|c|c|c|c|c|}
         \multicolumn{1}{c}{} & \multicolumn{1}{c}{\textbf{А}} & \multicolumn{1}{c}{\textbf{Б}} & \multicolumn{1}{c}{\textbf{В}} & \multicolumn{1}{c}{\textbf{Г}} & \multicolumn{1}{c}{\textbf{Д}} \\ \cline{2-6}
         \textbf{1} & & & & & \\ \cline{2-6}
         \textbf{2} & & & & & \\ \cline{2-6}
         \textbf{3} & & & & & \\ \cline{2-6}
    \end{tabular}
    \endgroup
\end{minipage}%
\hfill
\begin{minipage}[t]{0.55\textwidth}
    \textit{Властивість} \par \vspace{0.2cm}
    \begin{tabular}{l p{6.5cm}}
    \textbf{А} & проходить через початок координат \\
    \textbf{Б} & не містить точок $(x_0; y_0)$ з від'ємною ординатою $y_0$ \\
    \textbf{В} & не має спільних точок з віссю $y$ \\
    \textbf{Г} & двічі перетинає графік прямої $y = 3$ \\
    \textbf{Д} & графік функції знаходиться лише в I координатній чверті \\
    \end{tabular}
\end{minipage}

\vspace{0.7cm}

% === ЗАВДАННЯ 11 ===
\noindent\textbf{11.} Установіть відповідність між твердженням (1--3) та функцією (А--Д), для якої це твердження є правильним. \nmtyear{2023}

\vspace{0.3cm}

\noindent
\begin{minipage}[t]{0.50\textwidth} % Трохи ширше для довгих тверджень
    \textit{Твердження} \par \vspace{0.2cm}
    \begin{tabular}{@{}p{0.5cm} p{6.5cm}@{}}
    \textbf{1} & областю визначення функції є проміжок $[-1; +\infty)$ \\
    \textbf{2} & графік функції проходить через точку $(0; 1)$ \\
    \textbf{3} & функція має точку локального екстремуму на проміжку $[1; 3]$ \\
    \end{tabular}
    
    \vspace{0.3cm}
    
    % Зменшена табличка
    \begingroup
    \setlength{\tabcolsep}{4pt}
    \renewcommand{\arraystretch}{1.2}
    \small
    \begin{tabular}{r|c|c|c|c|c|}
         \multicolumn{1}{c}{} & \multicolumn{1}{c}{\textbf{А}} & \multicolumn{1}{c}{\textbf{Б}} & \multicolumn{1}{c}{\textbf{В}} & \multicolumn{1}{c}{\textbf{Г}} & \multicolumn{1}{c}{\textbf{Д}} \\ \cline{2-6}
         \textbf{1} & & & & & \\ \cline{2-6}
         \textbf{2} & & & & & \\ \cline{2-6}
         \textbf{3} & & & & & \\ \cline{2-6}
    \end{tabular}
    \endgroup
\end{minipage}%
\hfill
\begin{minipage}[t]{0.45\textwidth}
    \textit{Функція} \par \vspace{0.2cm}
    \begin{tabular}{ll}
    \textbf{А} & $y = -\cos x$ \\[0.2cm]
    \textbf{Б} & $y = 4^x$ \\[0.2cm]
    \textbf{В} & $y = 2\sqrt{x+1}$ \\[0.2cm]
    \textbf{Г} & $y = x^2 - 4x + 3$ \\[0.2cm]
    \textbf{Д} & $y = x$ \\
    \end{tabular}
\end{minipage}

\vspace{0.7cm}

% === ЗАВДАННЯ 12 ===
\noindent\textbf{12.} Установіть відповідність між твердженням (1--3) та функцією (А--Д), для якої це твердження є правильним. \nmtyear{2023}

\vspace{0.3cm}

\noindent
\begin{minipage}[t]{0.50\textwidth}
    \textit{Твердження} \par \vspace{0.2cm}
    \begin{tabular}{@{}p{0.5cm} p{6.5cm}@{}}
    \textbf{1} & областю значень функції є проміжок $[0; +\infty)$ \\
    \textbf{2} & графік функції симетричний відносно осі $y$ \\
    \textbf{3} & найменше значення на відрізку $[1; 4]$ функція набуває в точці $x = 4$ \\
    \end{tabular}
    
    \vspace{0.3cm}
    
    % Зменшена табличка
    \begingroup
    \setlength{\tabcolsep}{4pt}
    \renewcommand{\arraystretch}{1.2}
    \small
    \begin{tabular}{r|c|c|c|c|c|}
         \multicolumn{1}{c}{} & \multicolumn{1}{c}{\textbf{А}} & \multicolumn{1}{c}{\textbf{Б}} & \multicolumn{1}{c}{\textbf{В}} & \multicolumn{1}{c}{\textbf{Г}} & \multicolumn{1}{c}{\textbf{Д}} \\ \cline{2-6}
         \textbf{1} & & & & & \\ \cline{2-6}
         \textbf{2} & & & & & \\ \cline{2-6}
         \textbf{3} & & & & & \\ \cline{2-6}
    \end{tabular}
    \endgroup
\end{minipage}%
\hfill
\begin{minipage}[t]{0.45\textwidth}
    \textit{Функція} \par \vspace{0.2cm}
    \begin{tabular}{ll}
    \textbf{А} & $y = x^2 + 4$ \\[0.2cm]
    \textbf{Б} & $y = x$ \\[0.2cm]
    \textbf{В} & $y = \sqrt{x}$ \\[0.2cm]
    \textbf{Г} & $y = \log_{0{,}5} x$ \\[0.2cm]
    \textbf{Д} & $y = -\dfrac{1}{x}$ \\
    \end{tabular}
\end{minipage}

\vspace{0.7cm}

% === ЗАВДАННЯ 13 ===
\noindent\textbf{13.} \begin{minipage}[t]{0.55\textwidth}
Функція $y=f(x)$ визначена на проміжку $(-\infty; 0)$ і набуває лише від'ємних значень. Укажіть \textit{усі} координатні чверті (див. рисунок), у яких розташований графік цієї функції. \nmtyear{2023}
\end{minipage}
\hfill
\begin{minipage}[t]{0.4\textwidth}
    \vspace{-0.5cm}
    \begin{flushright}
    \begin{tikzpicture}[scale=0.7]
        \draw[->, >=stealth, thick] (-2.5,0) -- (2.5,0) node[below] {$x$};
        \draw[->, >=stealth, thick] (0,-2.5) -- (0,2.5) node[left] {$y$};
        \node[below left] at (0,0) {$O$};
        
        \node at (1.9,1.5) {I чверть};
        \node at (-1.9,1.5) {II чверть};
        \node at (-1.9,-1.5) {III чверть};
        \node at (1.9,-1.5) {IV чверть};
    \end{tikzpicture}
    \end{flushright}
\end{minipage}

\vspace{0.3cm}
\answerTableTall{лише в IV}{лише в II та III}{лише в III та IV}{лише в III}{лише в II}

\vspace{0.7cm}

% === ЗАВДАННЯ 14 ===
\noindent\textbf{14.} \begin{minipage}[t]{0.55\textwidth}
На рисунку зображено графік функції $y=f(x)$, визначеної на проміжку $[-2; 4]$. Цей графік перетинає вісь $x$ в одній із зазначених точок. Укажіть цю точку. \nmtyear{2023}
\end{minipage}
\hfill
\begin{minipage}[t]{0.4\textwidth}
    \vspace{-0.5cm}
    \begin{flushright}
    \begin{tikzpicture}[scale=0.5]
        \draw[step=1cm,gray!50,very thin] (-2.5,-1.5) grid (4.5,4.5);
        \draw[->, >=stealth, thick] (-2.5,0) -- (4.5,0) node[below] {$x$};
        \draw[->, >=stealth, thick] (0,-1.5) -- (0,4.5) node[left] {$y$};
        
        \node[below left] at (0,0) {$0$};
        \node[below] at (1,0) {$1$};
        \node[left] at (0,1) {$1$};
        \node[below] at (4,0) {$4$};
        \node[below] at (-2,0) {$-2$};
        
        \draw (1,0.1) -- (1,-0.1);
        \draw (4,0.1) -- (4,-0.1);
        \draw (-2,0.1) -- (-2,-0.1);
        \draw (0.1,1) -- (-0.1,1);
        
        % З hobby
        \draw[thick] plot[hobby] coordinates {(-2, 4.5) (-1, 4.2) (0, 4) (1, 3) (2, 2) (3, 0) (4, -1)};
        
        \fill (3,0) circle (3pt);
        \node[above right] at (1,3) {$y=f(x)$};
    \end{tikzpicture}
    \end{flushright}
\end{minipage}

\vspace{0.3cm}
\answerTableTall{$(3; 0)$}{$(0; 4)$}{$(0; 3)$}{$(4; 0)$}{$(3; 4)$}

\vspace{0.7cm}

% === ЗАВДАННЯ 15 ===
\noindent\textbf{15.} \begin{minipage}[t]{0.55\textwidth}
На рисунку зображено графік функції $y=f(x)$, визначеної на проміжку $[-4; 5]$. Точка $(x_0; -2)$ належить графіку цієї функції. Визначте абсцису $x_0$ цієї точки. \nmtyear{2023}
\end{minipage}
\hfill
\begin{minipage}[t]{0.4\textwidth}
    \vspace{-0.5cm}
    \begin{flushright}
    \begin{tikzpicture}[scale=0.5]
        \draw[step=1cm,gray!50,very thin] (-4.5,-3.5) grid (5.5,3.5);
        \draw[->, >=stealth, thick] (-4.5,0) -- (5.5,0) node[below] {$x$};
        \draw[->, >=stealth, thick] (0,-3.5) -- (0,3.5) node[left] {$y$};
        
        \node[below left] at (0,0) {$0$};
        \node[below] at (1,0) {$1$};
        \node[left] at (0,1) {$1$};
        \node[below] at (5,0) {$5$};
        \node[below] at (-4,0) {$-4$};
        
        \draw (1,0.1) -- (1,-0.1);
        \draw (5,0.1) -- (5,-0.1);
        \draw (-4,0.1) -- (-4,-0.1);
        \draw (0.1,1) -- (-0.1,1);
        
        % З hobby
        \draw[thick] plot[hobby] coordinates {(-4,-3) (-3,-2) (-2,0) (-1,3) (0,2) (1,0.5) (2,-1) (3,0) (4,0.8) (5,1)};
        
        \fill (-3,-2) circle (3pt);
        \node[above right] at (-1,3) {$y=f(x)$};
    \end{tikzpicture}
    \end{flushright}
\end{minipage}

\vspace{0.3cm}
\answerTableTall{$2$}{$-2$}{$0$}{$3$}{$-3$}

\vspace{0.7cm}

% === ЗАВДАННЯ 16 ===
\noindent\textbf{16.} Установіть відповідність між функцією (1--3) та її властивістю (А--Д). \nmtyear{2023}

\vspace{0.3cm}

\noindent
\begin{minipage}[t]{0.45\textwidth}
    \textit{Функція} \par \vspace{0.2cm}
    \textbf{1} \quad $y = \sqrt{x+1}$ \\[0.2cm]
    \textbf{2} \quad $y = 4-x^2$ \\[0.2cm]
    \textbf{3} \quad $y = 3^{-x}$
    
    \vspace{0.5cm}
    
    % Зменшена табличка
    \begingroup
    \setlength{\tabcolsep}{4pt}
    \renewcommand{\arraystretch}{1.2}
    \small
    \begin{tabular}{r|c|c|c|c|c|}
         \multicolumn{1}{c}{} & \multicolumn{1}{c}{\textbf{А}} & \multicolumn{1}{c}{\textbf{Б}} & \multicolumn{1}{c}{\textbf{В}} & \multicolumn{1}{c}{\textbf{Г}} & \multicolumn{1}{c}{\textbf{Д}} \\ \cline{2-6}
         \textbf{1} & & & & & \\ \cline{2-6}
         \textbf{2} & & & & & \\ \cline{2-6}
         \textbf{3} & & & & & \\ \cline{2-6}
    \end{tabular}
    \endgroup
\end{minipage}%
\hfill
\begin{minipage}[t]{0.50\textwidth}
    \textit{Властивість} \par \vspace{0.2cm}
    \begin{tabular}{l p{8.5cm}}
    \textbf{А} & має точку локального максимуму \\
    \textbf{Б} & має точку локального мінімуму \\
    \textbf{В} & є непарною \\
    \textbf{Г} & зростає на всій області визначення \\
    \textbf{Д} & набуває лише додатних значень \\
    \end{tabular}
\end{minipage}

\vspace{0.7cm}

% === ЗАВДАННЯ 17 ===
\noindent\textbf{17.} Установіть відповідність між функцією (1--3) та її властивістю (А--Д). \nmtyear{2023}

\vspace{0.3cm}

\noindent
\begin{minipage}[t]{0.40\textwidth}
    \textit{Функція} \par \vspace{0.2cm}
    \textbf{1} \quad $y = (x+2)^2$ \\[0.2cm]
    \textbf{2} \quad $y = 2\sqrt{x}$ \\[0.2cm]
    \textbf{3} \quad $y = 2^x$
    
    \vspace{0.5cm}
    
    % Зменшена табличка
    \begingroup
    \setlength{\tabcolsep}{4pt}
    \renewcommand{\arraystretch}{1.2}
    \small
    \begin{tabular}{r|c|c|c|c|c|}
         \multicolumn{1}{c}{} & \multicolumn{1}{c}{\textbf{А}} & \multicolumn{1}{c}{\textbf{Б}} & \multicolumn{1}{c}{\textbf{В}} & \multicolumn{1}{c}{\textbf{Г}} & \multicolumn{1}{c}{\textbf{Д}} \\ \cline{2-6}
         \textbf{1} & & & & & \\ \cline{2-6}
         \textbf{2} & & & & & \\ \cline{2-6}
         \textbf{3} & & & & & \\ \cline{2-6}
    \end{tabular}
    \endgroup
\end{minipage}%
\hfill
\begin{minipage}[t]{0.5\textwidth}
    \textit{Властивість} \par \vspace{0.2cm}
    \begin{tabular}{l p{8.5cm}}
    \textbf{А} & є зростаючою на проміжку $(-\infty; +\infty)$ \\
    \textbf{Б} & графік функції має одну спільну точку із графіком функції $y = x-2$ \\
    \textbf{В} & графік функції має дві спільні точки із графіком функції $y = x-2$ \\
    \textbf{Г} & є спадною на проміжку $(-\infty; -2]$ \\
    \textbf{Д} & є спадною на проміжку $(-\infty; 2]$ \\
    \end{tabular}
\end{minipage}

\vspace{0.7cm}

\begin{center}
{\Large\textbf{\color{headerblue}БАЗА ЗАВДАНЬ НМТ 2024}}
\end{center}

% === ЗАВДАННЯ 18 ===
\noindent\textbf{18.} Установіть відповідність між функцією (1--3) та її властивістю (А--Д). \nmtyear{2024}

\vspace{0.3cm}

\noindent
\begin{minipage}[t]{0.40\textwidth}
    \textit{Функція} \par \vspace{0.2cm}
    \textbf{1} \quad $y = 4 - x^2$ \\[0.2cm]
    \textbf{2} \quad $y = -x^3$ \\[0.2cm]
    \textbf{3} \quad $y = 4^x$
    
    \vspace{0.5cm}
    
    % Табличка
    \begingroup
    \setlength{\tabcolsep}{4pt}
    \renewcommand{\arraystretch}{1.2}
    \small
    \begin{tabular}{r|c|c|c|c|c|}
         \multicolumn{1}{c}{} & \multicolumn{1}{c}{\textbf{А}} & \multicolumn{1}{c}{\textbf{Б}} & \multicolumn{1}{c}{\textbf{В}} & \multicolumn{1}{c}{\textbf{Г}} & \multicolumn{1}{c}{\textbf{Д}} \\ \cline{2-6}
         \textbf{1} & & & & & \\ \cline{2-6}
         \textbf{2} & & & & & \\ \cline{2-6}
         \textbf{3} & & & & & \\ \cline{2-6}
    \end{tabular}
    \endgroup
\end{minipage}%
\hfill
\begin{minipage}[t]{0.55\textwidth}
    \textit{Властивість} \par \vspace{0.2cm}
    \begin{tabular}{l p{6.5cm}}
    \textbf{А} & є зростаючою на всій області визначення \\
    \textbf{Б} & набуває від'ємного значення при $x = -2$ \\
    \textbf{В} & є парною \\
    \textbf{Г} & має три спільні точки з графіком функції $y = -x$ \\
    \textbf{Д} & графік функції розташований лише в I та IV координатній чверті \\
    \end{tabular}
\end{minipage}

\vspace{0.7cm}

% === ЗАВДАННЯ 19 ===
\noindent\textbf{19.} Укажіть графік непарної функції. \nmtyear{2024}

\vspace{0.3cm}
\begin{center}
\begin{tabular}{|*{5}{>{\centering\arraybackslash}m{2.8cm}|}}
\hline
\textbf{А} & \textbf{Б} & \textbf{В} & \textbf{Г} & \textbf{Д} \\
\hline
\begin{tikzpicture}[scale=0.4]
    \draw[->, >=stealth] (-2,0) -- (2,0) node[below] {$x$};
    \draw[->, >=stealth] (0,-2) -- (0,2) node[left] {$y$};
    \node[below right] at (0,0) {$0$};
    \draw[thick] (-1.5,-1.5) -- (1.5,1.5);
\end{tikzpicture} &
\begin{tikzpicture}[scale=0.4]
    \draw[->, >=stealth] (-2,0) -- (2,0) node[below] {$x$};
    \draw[->, >=stealth] (0,-2) -- (0,2) node[left] {$y$};
    \node[below left] at (0,0) {$0$};
    \draw[thick] (-3,1.5) -- (0,0) -- (1.9,1.5);
\end{tikzpicture} &
\begin{tikzpicture}[scale=0.4]
    \draw[->, >=stealth] (-2,0) -- (2,0) node[below] {$x$};
    \draw[->, >=stealth] (0,-2) -- (0,2) node[left] {$y$};
    \node[below left] at (0,0) {$0$};
    % Парабола або схоже на модуль, симетричне відносно OY
    \draw[thick] (-1.5,1.5) -- (0,0) -- (1.5,1.5); 
\end{tikzpicture} &
\begin{tikzpicture}[scale=0.4]
    \draw[->, >=stealth] (-2,0) -- (2,0) node[below] {$x$};
    \draw[->, >=stealth] (0,-2) -- (0,2) node[left] {$y$};
    \node[below right] at (0,0) {$0$};
    \draw[thick] (-2,1.5) -- (-1,0) -- (1,1);
\end{tikzpicture} &
\begin{tikzpicture}[scale=0.4]
    \draw[->, >=stealth] (-2,0) -- (2,0) node[below] {$x$};
    \draw[->, >=stealth] (0,-2) -- (0,2) node[left] {$y$};
    \node[below left] at (0,0) {$0$};
    \draw[thick] (-1,-2) -- (1.5,1);
\end{tikzpicture} \\
\hline
\end{tabular}
\end{center}

\vspace{0.7cm}

% === ЗАВДАННЯ 20 ===
\noindent\textbf{20.} \begin{minipage}[t]{0.55\textwidth}
На рисунку зображено графік функції $y=f(x)$, визначеної на проміжку $[-4; 6]$. Укажіть різницю між найбільшим і найменшим значенням функції $f(x)$ на цьому проміжку. \nmtyear{2024}
\end{minipage}
\hfill
\begin{minipage}[t]{0.4\textwidth}
    \vspace{-0.5cm}
    \begin{flushright}
    \begin{tikzpicture}[scale=0.5]
        % Розширив сітку: y від -2.5 до 6.5
        \draw[step=1cm,gray!50,very thin] (-4.5,-2.5) grid (6.5,6.5);
        \draw[->, >=stealth, thick] (-4.5,0) -- (6.5,0) node[below] {$x$};
        \draw[->, >=stealth, thick] (0,-2.5) -- (0,6.5) node[left] {$y$};
        
        \node[below left] at (0,0) {$0$};
        \node[below] at (1,0) {$1$};
        \node[left] at (0,1) {$1$};
        \node[below] at (6,0) {$6$};
        \node[below] at (-4,0) {$-4$};
        
        % З hobby
        \draw[thick] plot[hobby] coordinates {(-4, 6) (-3, 0) (-2, -2) (1, 1) (4, 3) (6, 1)};
        
        \fill (-4,6) circle (3pt);
        \fill (6,1) circle (3pt);
        \node[above right] at (4,3) {$y=f(x)$};
    \end{tikzpicture}
    \end{flushright}
\end{minipage}

\vspace{0.3cm}
\answerTableTall{$5$}{$7$}{$6$}{$3$}{$10$}

\vspace{0.7cm}

% === ЗАВДАННЯ 21 ===
\noindent\textbf{21.} На рисунку зображено графік функції $y=f(x)$, визначеної на проміжку $[-4; 4]$. Установіть відповідність між початком речення (1--3) та його закінченням (А--Д) так, щоб утворилося правильне твердження. \nmtyear{2024}

\vspace{0.3cm}

% --- ВЕРХНІЙ БЛОК: Умови + Графік ---
\noindent
\begin{minipage}[t]{0.55\textwidth}
    \textit{Початок речення} \par \vspace{0.3cm}
    \textbf{1} \quad Найменше значення функції $y=f(x)$ \\[0.4cm]
    \textbf{2} \quad Точка екстремуму функції $y=f(x)-5$ \\[0.4cm]
    \textbf{3} \quad Нуль функції $y=f(x+2)$
\end{minipage}%
\hfill
\begin{minipage}[t]{0.40\textwidth}
    \vspace{-0.5cm} % Підтягуємо графік трохи вгору
    \begin{flushright}
    \begin{tikzpicture}[scale=0.5]
        % Сітка та осі
        \draw[step=1cm,gray!50,very thin] (-4.5,-3.5) grid (4.5,3.5);
        \draw[->, >=stealth, thick] (-4.5,0) -- (4.5,0) node[below] {$x$};
        \draw[->, >=stealth, thick] (0,-3.5) -- (0,3.5) node[left] {$y$};
        
        % Підписи осей
        \node[below left] at (0,0) {$0$};
        \node[below] at (1,0) {$1$};
        \node[left] at (0,1) {$1$};
        \node[below] at (4,0) {$4$};
        \node[below] at (-4,0) {$-4$};
        
        % З hobby
        \draw[thick] plot[hobby] coordinates {(-4, -3) (-2, -1) (0, 0) (3, 3) (4, 1)};
        
        % Точки на кінцях
        \fill (-4,-3) circle (3pt);
        \fill (4,1) circle (3pt);
        
        \node[left] at (-1, 1) {$y=f(x)$};
    \end{tikzpicture}
    \end{flushright}
\end{minipage}

\vspace{0.2cm}

% --- НИЖНІЙ БЛОК: Варіанти + Таблиця ---
\noindent
\begin{minipage}[t]{0.55\textwidth}
    \textit{Закінчення речення} \par \vspace{0.2cm}
    \begin{tabular}{ll}
    \textbf{А} & дорівнює $-3$. \\
    \textbf{Б} & дорівнює $-2$. \\
    \textbf{В} & дорівнює $0$. \\
    \textbf{Г} & дорівнює $2$. \\
    \textbf{Д} & дорівнює $3$. \\
    \end{tabular}
\end{minipage}%
\hfill
\begin{minipage}[t]{0.40\textwidth}
    \vspace{0.5cm} % Вирівнювання таблички по висоті
    \begin{flushright}
    % Табличка
    \begingroup
    \setlength{\tabcolsep}{4pt}
    \renewcommand{\arraystretch}{1.2}
    \small
    \begin{tabular}{r|c|c|c|c|c|}
         \multicolumn{1}{c}{} & \multicolumn{1}{c}{\textbf{А}} & \multicolumn{1}{c}{\textbf{Б}} & \multicolumn{1}{c}{\textbf{В}} & \multicolumn{1}{c}{\textbf{Г}} & \multicolumn{1}{c}{\textbf{Д}} \\ \cline{2-6}
         \textbf{1} & & & & & \\ \cline{2-6}
         \textbf{2} & & & & & \\ \cline{2-6}
         \textbf{3} & & & & & \\ \cline{2-6}
    \end{tabular}
    \endgroup
    \end{flushright}
\end{minipage}

\vspace{0.7cm}
% === ЗАВДАННЯ 22 ===
\noindent\textbf{22.} Установіть відповідність між твердженням (1--3) та функцією (А--Д), для якої це твердження є правильним. \nmtyear{2024}

\vspace{0.3cm}

\noindent
\begin{minipage}[t]{0.50\textwidth}
    \textit{Твердження} \par \vspace{0.2cm}
    \begin{tabular}{@{}p{0.5cm} p{6.5cm}@{}}
    \textbf{1} & функція має 2 нулі \\
    \textbf{2} & на відрізку $[-1; 3]$ функція набуває від'ємних значень \\
    \textbf{3} & найменше значення функції на відрізку $[-1; 3]$ дорівнює $0{,}5$ \\
    \end{tabular}
    
    \vspace{0.3cm}
    
    % Табличка
    \begingroup
    \setlength{\tabcolsep}{4pt}
    \renewcommand{\arraystretch}{1.2}
    \small
    \begin{tabular}{r|c|c|c|c|c|}
         \multicolumn{1}{c}{} & \multicolumn{1}{c}{\textbf{А}} & \multicolumn{1}{c}{\textbf{Б}} & \multicolumn{1}{c}{\textbf{В}} & \multicolumn{1}{c}{\textbf{Г}} & \multicolumn{1}{c}{\textbf{Д}} \\ \cline{2-6}
         \textbf{1} & & & & & \\ \cline{2-6}
         \textbf{2} & & & & & \\ \cline{2-6}
         \textbf{3} & & & & & \\ \cline{2-6}
    \end{tabular}
    \endgroup
\end{minipage}%
\hfill
\begin{minipage}[t]{0.45\textwidth}
    \textit{Функція} \par \vspace{0.2cm}
    \begin{tabular}{ll}
    \textbf{А} & $y = x^2 - 4$ \\[0.2cm]
    \textbf{Б} & $y = \dfrac{1}{x-4}$ \\[0.2cm]
    \textbf{В} & $y = 2^x$ \\[0.2cm]
    \textbf{Г} & $y = 0{,}5^x$ \\[0.2cm]
    \textbf{Д} & $y = \sqrt{x+1}$ \\
    \end{tabular}
\end{minipage}

\vspace{0.7cm}

% === ЗАВДАННЯ 23 ===
\noindent\textbf{23.} На рисунку зображено графік функції $y=f(x)$, визначеної на проміжку $[-4; 5]$. Установіть відповідність між початком речення (1--3) та його закінченням (А--Д) так, щоб утворилося правильне твердження. \nmtyear{2024}

\vspace{0.3cm}

\noindent
\begin{minipage}[t]{0.55\textwidth}
    \textit{Початок речення} \par \vspace{0.3cm}
    \textbf{1} \quad Нуль функції належить проміжку \\[0.4cm]
    \textbf{2} \quad Точка максимуму функції належить проміжку \\[0.4cm]
    \textbf{3} \quad Абсциса точки перетину графіка функції з графіком функції $y = \log_{\frac{1}{3}} x$ належить проміжку
\end{minipage}%
\hfill
\begin{minipage}[t]{0.40\textwidth}
    \vspace{-0.5cm}
    \begin{flushright}
    \begin{tikzpicture}[scale=0.5]
        % Сітка розширена вниз до -3.5
        \draw[step=1cm,gray!50,very thin] (-4.5,-3.5) grid (5.5,4.5);
        \draw[->, >=stealth, thick] (-4.5,0) -- (5.5,0) node[below] {$x$};
        \draw[->, >=stealth, thick] (0,-3.5) -- (0,4.5) node[left] {$y$};
        
        \node[below left] at (0,0) {$0$};
        \node[below] at (1,0) {$1$};
        \node[left] at (0,1) {$1$};
        \node[below] at (5,0) {$5$};
        \node[below] at (-4,0) {$-4$};
        
        % З hobby
        \draw[thick] plot[hobby] coordinates {(-4, -3) (-1.2, 0) (0, 2.4) (2.5, 4) (5, 2)};
        
        \fill (-4,-3) circle (3pt);
        \fill (5,2) circle (3pt);
        \node[right] at (2.5, 4) {$y=f(x)$};
    \end{tikzpicture}
    \end{flushright}
\end{minipage}

\vspace{0.2cm}

\noindent
\begin{minipage}[t]{0.55\textwidth}
    \textit{Закінчення речення} \par \vspace{0.2cm}
    \begin{tabular}{ll}
    \textbf{А} & $(-4; -2]$. \\
    \textbf{Б} & $(-2; 0]$. \\
    \textbf{В} & $(0; 1]$. \\
    \textbf{Г} & $(1; 3]$. \\
    \textbf{Д} & $(3; 5]$. \\
    \end{tabular}
\end{minipage}%
\hfill
\begin{minipage}[t]{0.40\textwidth}
    \vspace{0.5cm}
    \begin{flushright}
    \begingroup
    \setlength{\tabcolsep}{4pt}
    \renewcommand{\arraystretch}{1.2}
    \small
    \begin{tabular}{r|c|c|c|c|c|}
         \multicolumn{1}{c}{} & \multicolumn{1}{c}{\textbf{А}} & \multicolumn{1}{c}{\textbf{Б}} & \multicolumn{1}{c}{\textbf{В}} & \multicolumn{1}{c}{\textbf{Г}} & \multicolumn{1}{c}{\textbf{Д}} \\ \cline{2-6}
         \textbf{1} & & & & & \\ \cline{2-6}
         \textbf{2} & & & & & \\ \cline{2-6}
         \textbf{3} & & & & & \\ \cline{2-6}
    \end{tabular}
    \endgroup
    \end{flushright}
\end{minipage}

\vspace{0.7cm}

% === ЗАВДАННЯ 24 ===
\noindent\textbf{24.} На рисунку зображено графік функції $y=f(x)$, визначеної на проміжку $[1; 9]$. Установіть відповідність між початком речення (1--3) та його закінченням (А--Д) так, щоб утворилося правильне твердження. \nmtyear{2024}

\vspace{0.3cm}

\noindent
\begin{minipage}[t]{0.55\textwidth}
    \textit{Початок речення} \par \vspace{0.3cm}
    \textbf{1} \quad Найбільше значення функції на проміжку $[1; 9]$ \\[0.4cm]
    \textbf{2} \quad Найменше значення функції на проміжку $[1; 3]$ \\[0.4cm]
    \textbf{3} \quad Найменше ціле значення $x$, за якого виконується нерівність $f(x) < 0$
\end{minipage}%
\hfill
\begin{minipage}[t]{0.40\textwidth}
    \vspace{-0.5cm}
    \begin{flushright}
    \begin{tikzpicture}[scale=0.5]
        % Сітка розширена вгору до 7.5
        \draw[step=1cm,gray!50,very thin] (0.5,-1.5) grid (9.5,7);
        \draw[->, >=stealth, thick] (0,0) -- (9.5,0) node[below] {$x$};
        \draw[->, >=stealth, thick] (0,-1.5) -- (0,7) node[left] {$y$};
        
        \node[below left] at (0,0) {$0$};
        \node[below] at (1,0) {$1$};
        \node[left] at (0,1) {$1$};
        \node[below] at (9,0) {$9$};
        
        % З hobby
        \draw[thick] plot[hobby] coordinates {(1, 5)  (3, 7) (6, 0) (7, -1) (8, 0) (9, 1)};
        
        \fill (1,5) circle (3pt);
        \fill (9,1) circle (3pt);
        \node[above] at (3, 7) {$y=f(x)$};
    \end{tikzpicture}
    \end{flushright}
\end{minipage}

\vspace{0.2cm}

\noindent
\begin{minipage}[t]{0.55\textwidth}
    \textit{Закінчення речення} \par \vspace{0.2cm}
    \begin{tabular}{ll}
    \textbf{А} & дорівнює $5$. \\
    \textbf{Б} & дорівнює $6$. \\
    \textbf{В} & дорівнює $7$. \\
    \textbf{Г} & дорівнює $8$. \\
    \textbf{Д} & дорівнює $9$. \\
    \end{tabular}
\end{minipage}%
\hfill
\begin{minipage}[t]{0.40\textwidth}
    \vspace{0.5cm}
    \begin{flushright}
    \begingroup
    \setlength{\tabcolsep}{4pt}
    \renewcommand{\arraystretch}{1.2}
    \small
    \begin{tabular}{r|c|c|c|c|c|}
         \multicolumn{1}{c}{} & \multicolumn{1}{c}{\textbf{А}} & \multicolumn{1}{c}{\textbf{Б}} & \multicolumn{1}{c}{\textbf{В}} & \multicolumn{1}{c}{\textbf{Г}} & \multicolumn{1}{c}{\textbf{Д}} \\ \cline{2-6}
         \textbf{1} & & & & & \\ \cline{2-6}
         \textbf{2} & & & & & \\ \cline{2-6}
         \textbf{3} & & & & & \\ \cline{2-6}
    \end{tabular}
    \endgroup
    \end{flushright}
\end{minipage}

\vspace{0.7cm}

% === ЗАВДАННЯ 25 ===
\noindent\textbf{25.} \begin{minipage}[t]{0.55\textwidth}
На рисунку зображено графік функції $y=f(x)$, визначеної на проміжку $[-5; 5]$. Укажіть поміж наведених координати точки, що належить цьому графіку. \nmtyear{2024}
\end{minipage}
\hfill
\begin{minipage}[t]{0.4\textwidth}
    \vspace{-0.5cm}
    \begin{flushright}
    \begin{tikzpicture}[scale=0.45]
        \draw[step=1cm,gray!50,very thin] (-5.5,-2.5) grid (5.5,4.5);
        \draw[->, >=stealth, thick] (-5.5,0) -- (5.5,0) node[below] {$x$};
        \draw[->, >=stealth, thick] (0,-2.5) -- (0,4.5) node[left] {$y$};
        
        \node[below left] at (0,0) {$0$};
        \node[below] at (1,0) {$1$};
        \node[left] at (0,1) {$1$};
        \node[below] at (5,0) {$5$};
        \node[below] at (-5,0) {$-5$};
        
        % З hobby
        \draw[thick] plot[hobby] coordinates {(-5, 4) (-3, 2) (0, 0) (3, -2) (5, 1)};
        
        \fill (-5,4) circle (3pt);
        \fill (5,1) circle (3pt);
        \node[right] at (-3, 2.5) {$y=f(x)$};
    \end{tikzpicture}
    \end{flushright}
\end{minipage}

\vspace{0.3cm}
\answerTableTall{$(-1; 4)$}{$(-4; 3)$}{$(2; -2)$}{$(-2; 3)$}{$(-3; 2)$}

\vspace{0.7cm}

% === ЗАВДАННЯ 26 ===
\noindent\textbf{26.} Установіть відповідність між твердженням (1--3) та функцією (А--Д), для якої це твердження є правильним. \nmtyear{2024}

\vspace{0.3cm}

\noindent
\begin{minipage}[t]{0.50\textwidth}
    \textit{Твердження} \par \vspace{0.2cm}
    \begin{tabular}{@{}p{0.5cm} p{6.5cm}@{}}
    \textbf{1} & є непарною \\
    \textbf{2} & зростає на відрізку $[1; 4]$ \\
    \textbf{3} & найбільше значення функції на відрізку $[1; 4]$ є від'ємним числом \\
    \end{tabular}
    
    \vspace{0.3cm}
    
    \begingroup
    \setlength{\tabcolsep}{4pt}
    \renewcommand{\arraystretch}{1.2}
    \small
    \begin{tabular}{r|c|c|c|c|c|}
         \multicolumn{1}{c}{} & \multicolumn{1}{c}{\textbf{А}} & \multicolumn{1}{c}{\textbf{Б}} & \multicolumn{1}{c}{\textbf{В}} & \multicolumn{1}{c}{\textbf{Г}} & \multicolumn{1}{c}{\textbf{Д}} \\ \cline{2-6}
         \textbf{1} & & & & & \\ \cline{2-6}
         \textbf{2} & & & & & \\ \cline{2-6}
         \textbf{3} & & & & & \\ \cline{2-6}
    \end{tabular}
    \endgroup
\end{minipage}%
\hfill
\begin{minipage}[t]{0.45\textwidth}
    \textit{Функція} \par \vspace{0.2cm}
    \begin{tabular}{ll}
    \textbf{А} & $y = \dfrac{1}{x+1}$ \\[0.3cm]
    \textbf{Б} & $y = \sin x$ \\[0.3cm]
    \textbf{В} & $y = x^2 - 1$ \\[0.3cm]
    \textbf{Г} & $y = 0{,}5^x$ \\[0.3cm]
    \textbf{Д} & $y = -\sqrt{x}$ \\
    \end{tabular}
\end{minipage}

\vspace{0.7cm}

% === ЗАВДАННЯ 27 ===
\noindent\textbf{27.} \begin{minipage}[t]{0.55\textwidth}
На рисунку зображено графік функції $y=f(x)$, визначеної на проміжку $[-5; 4]$. Скільки точок перетину з осями координат має ця функція на заданому проміжку? \nmtyear{2024}
\end{minipage}
\hfill
\begin{minipage}[t]{0.4\textwidth}
    \vspace{-0.5cm}
    \begin{flushright}
    \begin{tikzpicture}[scale=0.5]
        \draw[step=1cm,gray!50,very thin] (-5.5,-2.5) grid (4.5,5.5);
        \draw[->, >=stealth, thick] (-5.5,0) -- (4.5,0) node[below] {$x$};
        \draw[->, >=stealth, thick] (0,-2.5) -- (0,5.5) node[left] {$y$};
        
        \node[below left] at (0,0) {$0$};
        \node[below] at (1,0) {$1$};
        \node[left] at (0,1) {$1$};
        \node[below] at (4,0) {$4$};
        \node[below] at (-5,0) {$-5$};
        
        % З hobby
        \draw[thick] plot[hobby] coordinates {(-5, -2) (-2, 3) (1, -2) (4, 5)};
        
        \fill (-5,-2) circle (3pt);
        \fill (4,5) circle (3pt);
        \node[left] at (-2, 2.5) {$y=f(x)$};
    \end{tikzpicture}
    \end{flushright}
\end{minipage}

\vspace{0.3cm}
\answerTableTall{$3$}{$5$}{$6$}{$4$}{$2$}

\vspace{0.7cm}

% === ЗАВДАННЯ 28 ===
\noindent\textbf{28.} \begin{minipage}[t]{0.55\textwidth}
На рисунку зображено графік функції $y=f(x)$, визначеної на проміжку $[-5; 5]$. Укажіть різницю між найбільшим і найменшим значенням функції $f(x)$ на проміжку $[0; 5]$. \nmtyear{2024}
\end{minipage}
\hfill
\begin{minipage}[t]{0.4\textwidth}
    \vspace{-0.5cm}
    \begin{flushright}
    \begin{tikzpicture}
        \begin{axis}[
            width=6cm,
            height=6cm,
            xmin=-5.5, xmax=5.5,
            ymin=-2.5, ymax=7.5,
            axis lines=middle,
            xlabel={$x$},
            ylabel={$y$},
            xlabel style={anchor=west},
            ylabel style={anchor=south},
            xtick={-5,-4,-3,-2,-1,0,1,2,3,4,5},
            ytick={-2,-1,0,1,2,3,4,5,6,7},
            xticklabels={$-5$,,,,,0,$1$,,,,5},
            yticklabels={,,,0,$1$,,,,,},
            grid=both,
            grid style={line width=.1pt, draw=gray!30},
            major grid style={line width=.2pt,draw=gray!50},
            tick style={draw=none},
            enlarge x limits=false,
            enlarge y limits=false,
            clip=false
        ]
        
        % Графік через координати
        \addplot[thick, smooth] coordinates {
            (-5, 6) (-3, 4) (0, 3) (2, 1) (3, 2) (4, 4) (5, 7)
        };
        
        % Точки на кінцях
        \addplot[only marks, mark=*, mark size=2pt] coordinates {(-5,6) (5,7)};
        
        % Підпис функції
        \node at (axis cs:-1,3.5) [left] {$y=f(x)$};
        
        % Позначення нуля
        \node at (axis cs:0,0) [below left] {$0$};
        
        \end{axis}
    \end{tikzpicture}
    \end{flushright}
\end{minipage}

\vspace{0.3cm}
\answerTableTall{$10$}{$7$}{$6$}{$8$}{$3$}

\end{document}