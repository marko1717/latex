\documentclass[14pt]{extarticle}
\usepackage{fontspec}
\usepackage{polyglossia}
\setdefaultlanguage{ukrainian}

\defaultfontfeatures{Ligatures=TeX}
\setmainfont{Liberation Serif}
\setsansfont{Liberation Sans}
\setmonofont{Liberation Mono}

\usepackage[a4paper,margin=2cm,bottom=2.5cm,top=2.5cm]{geometry}
\usepackage{amsmath,amssymb}
\usepackage{enumitem}
\usepackage{xcolor}
\usepackage{array}
\usepackage{fancyhdr}

% Кольори
\definecolor{headerblue}{RGB}{0, 102, 204}
\definecolor{gencolor}{RGB}{0, 128, 0}

\pagestyle{fancy}
\fancyhf{}
\renewcommand{\headrulewidth}{0pt}
\fancyfoot[C]{\thepage}

\setlength{\headheight}{15pt}
\setlength{\headsep}{10pt}
\setlength{\footskip}{25pt}

\widowpenalty=10000
\clubpenalty=10000

% === КОМАНДИ ===

\newcommand{\answerTable}[5]{
\begin{center}
\begin{tabular}{|*{5}{>{\centering\arraybackslash}m{2.8cm}|}}
\hline
\rule[-0.3cm]{0pt}{0.8cm}\textbf{А} & \textbf{Б} & \textbf{В} & \textbf{Г} & \textbf{Д} \\
\hline
\rule[-0.4cm]{0pt}{1.0cm}#1 & \rule[-0.4cm]{0pt}{1.0cm}#2 & \rule[-0.4cm]{0pt}{1.0cm}#3 & \rule[-0.4cm]{0pt}{1.0cm}#4 & \rule[-0.4cm]{0pt}{1.0cm}#5 \\
\hline
\end{tabular}
\end{center}
}

\newcommand{\answerTableBig}[5]{
\begin{center}
\begin{tabular}{|*{5}{>{\centering\arraybackslash}m{2.8cm}|}}
\hline
\rule[-0.3cm]{0pt}{0.8cm}\textbf{А} & \textbf{Б} & \textbf{В} & \textbf{Г} & \textbf{Д} \\
\hline
\rule[-0.6cm]{0pt}{1.4cm}#1 & \rule[-0.6cm]{0pt}{1.4cm}#2 & \rule[-0.6cm]{0pt}{1.4cm}#3 & \rule[-0.6cm]{0pt}{1.4cm}#4 & \rule[-0.6cm]{0pt}{1.4cm}#5 \\
\hline
\end{tabular}
\end{center}
}

\newcommand{\task}[2]{\noindent\makebox[1.5em][l]{\textbf{#1.}}\parbox[t]{\dimexpr\textwidth-1.5em}{#2}}

\newcommand{\gentask}{\hfill{\small\color{gencolor}(Згенеровано)}}

\begin{document}

\begin{center}
{\Large\textbf{\color{headerblue}ЗГЕНЕРОВАНІ ЗАВДАННЯ}}
\end{center}

\begin{center}
{\large Тема 6: \textbf{Лінійні рівняння}}
\end{center}

\vspace{0.3cm}
\begin{center}
\textit{100 завдань з варіаціями стилів формулювання}
\end{center}

\vspace{0.5cm}

%======================================================================
% БЛОК 1: ПРОСТІ ЛІНІЙНІ РІВНЯННЯ ax = b
%======================================================================

\begin{center}
{\large\textbf{\color{headerblue}Блок 1: Прості лінійні рівняння $ax = b$}}
\end{center}

\vspace{0.3cm}

% --- Стиль: "Укажіть корінь рівняння" ---

\task{1}{Укажіть корінь рівняння $5x = 15$. \gentask}
\answerTable{$0{,}33$}{$10$}{$-3$}{$3$}{$75$}

\vspace{0.4cm}

\task{2}{Укажіть корінь рівняння $-6x = 18$. \gentask}
\answerTable{$-12$}{$12$}{$3$}{$-3$}{$-108$}

\vspace{0.4cm}

\task{3}{Укажіть корінь рівняння $4x = -12$. \gentask}
\answerTable{$-8$}{$3$}{$-3$}{$-48$}{$8$}

\vspace{0.4cm}

\task{4}{Укажіть корінь рівняння $-7x = -21$. \gentask}
\answerTable{$-14$}{$-3$}{$3$}{$147$}{$14$}

\vspace{0.4cm}

\task{5}{Укажіть корінь рівняння $9x = 27$. \gentask}
\answerTable{$-3$}{$3$}{$36$}{$18$}{$243$}

\vspace{0.4cm}

% --- Стиль: "Розв'яжіть рівняння" ---

\task{6}{Розв'яжіть рівняння $0{,}2x = 4$. \gentask}
\answerTable{$4{,}2$}{$0{,}8$}{$20$}{$-20$}{$2$}

\vspace{0.4cm}

\task{7}{Розв'яжіть рівняння $0{,}5x = -3$. \gentask}
\answerTable{$-1{,}5$}{$1{,}5$}{$6$}{$-6$}{$-2{,}5$}

\vspace{0.4cm}

\task{8}{Розв'яжіть рівняння $0{,}25x = 5$. \gentask}
\answerTable{$20$}{$-20$}{$5{,}25$}{$4$}{$1{,}25$}

\vspace{0.4cm}

\task{9}{Розв'яжіть рівняння $0{,}1x = -7$. \gentask}
\answerTable{$0{,}7$}{$70$}{$-6{,}9$}{$-0{,}7$}{$-70$}

\vspace{0.4cm}

\task{10}{Розв'яжіть рівняння $0{,}04x = 2$. \gentask}
\answerTable{$8$}{$-50$}{$2{,}04$}{$0{,}08$}{$50$}

\vspace{0.4cm}

% --- Стиль: "Знайдіть корінь рівняння" ---

\task{11}{Знайдіть корінь рівняння $-0{,}3x = 6$. \gentask}
\answerTable{$-1{,}8$}{$20$}{$5{,}7$}{$-20$}{$1{,}8$}

\vspace{0.4cm}

\task{12}{Знайдіть корінь рівняння $1{,}5x = 9$. \gentask}
\answerTable{$0{,}6$}{$6$}{$7{,}5$}{$-6$}{$13{,}5$}

\vspace{0.4cm}

\task{13}{Знайдіть корінь рівняння $-2{,}5x = -10$. \gentask}
\answerTable{$4$}{$-12{,}5$}{$-7{,}5$}{$-4$}{$25$}

\vspace{0.4cm}

\task{14}{Знайдіть корінь рівняння $0{,}8x = -4$. \gentask}
\answerTable{$-4{,}8$}{$5$}{$-3{,}2$}{$-5$}{$3{,}2$}

\vspace{0.4cm}

\task{15}{Знайдіть корінь рівняння $-1{,}2x = 3{,}6$. \gentask}
\answerTable{$3$}{$-4{,}32$}{$2{,}4$}{$-3$}{$4{,}8$}

\vspace{0.4cm}

%======================================================================
% БЛОК 2: РІВНЯННЯ З ДУЖКАМИ
%======================================================================

\newpage

\begin{center}
{\large\textbf{\color{headerblue}Блок 2: Рівняння з дужками}}
\end{center}

\vspace{0.3cm}

% --- Стиль: "Розв'яжіть рівняння" ---

\task{16}{Розв'яжіть рівняння $2(3x - 6) = 0$. \gentask}
\answerTable{$2$}{$0$}{$3$}{$6$}{$-2$}

\vspace{0.4cm}

\task{17}{Розв'яжіть рівняння $5(2x + 4) = 0$. \gentask}
\answerTable{$0$}{$4$}{$-2$}{$2$}{$-4$}

\vspace{0.4cm}

\task{18}{Розв'яжіть рівняння $4(x - 3) = 0$. \gentask}
\answerTable{$-4$}{$3$}{$-3$}{$4$}{$0$}

\vspace{0.4cm}

\task{19}{Розв'яжіть рівняння $-3(2x + 5) = 0$. \gentask}
\answerTable{$0$}{$2{,}5$}{$-2{,}5$}{$-5$}{$5$}

\vspace{0.4cm}

\task{20}{Розв'яжіть рівняння $6(4x - 1) = 0$. \gentask}
\answerTableBig{$4$}{$0$}{$0{,}25$}{$\dfrac{1}{6}$}{$-0{,}25$}

\vspace{0.4cm}

% --- Стиль: Розкриття дужок ---

\task{21}{Розв'яжіть рівняння $3(x + 2) = 15$. \gentask}
\answerTable{$5$}{$1$}{$7$}{$3$}{$-3$}

\vspace{0.4cm}

\task{22}{Розв'яжіть рівняння $4(2x - 1) = 12$. \gentask}
\answerTable{$4$}{$3$}{$-2$}{$2$}{$1$}

\vspace{0.4cm}

\task{23}{Розв'яжіть рівняння $5(x - 4) = -10$. \gentask}
\answerTable{$-2$}{$-6$}{$0$}{$6$}{$2$}

\vspace{0.4cm}

\task{24}{Розв'яжіть рівняння $2(3x + 1) = 20$. \gentask}
\answerTable{$3$}{$6$}{$9$}{$-3$}{$10$}

\vspace{0.4cm}

\task{25}{Розв'яжіть рівняння $-2(x - 5) = 6$. \gentask}
\answerTable{$-2$}{$5$}{$8$}{$-8$}{$2$}

\vspace{0.4cm}

% --- Стиль: "Укажіть корінь рівняння" ---

\task{26}{Укажіть корінь рівняння $7(x + 1) - 3x = 11$. \gentask}
\answerTable{$-2$}{$1$}{$-1$}{$0$}{$2$}

\vspace{0.4cm}

\task{27}{Укажіть корінь рівняння $5(2x - 3) + x = 7$. \gentask}
\answerTable{$-2$}{$1$}{$-1$}{$0$}{$2$}

\vspace{0.4cm}

\task{28}{Укажіть корінь рівняння $3(x - 2) - 2(x + 1) = 0$. \gentask}
\answerTable{$-8$}{$8$}{$0$}{$-4$}{$4$}

\vspace{0.4cm}

\task{29}{Укажіть корінь рівняння $4(x + 3) - 2(x - 1) = 10$. \gentask}
\answerTable{$4$}{$2$}{$0$}{$-4$}{$-2$}

\vspace{0.4cm}

\task{30}{Укажіть корінь рівняння $6(x - 1) + 3(2 - x) = 9$. \gentask}
\answerTable{$-3$}{$5$}{$3$}{$-5$}{$1$}

\vspace{0.4cm}

%======================================================================
% БЛОК 3: РІВНЯННЯ З ПРОСТИМИ ДРОБАМИ
%======================================================================

\newpage

\begin{center}
{\large\textbf{\color{headerblue}Блок 3: Рівняння з дробами}}
\end{center}

\vspace{0.3cm}

% --- Стиль: "Розв'яжіть рівняння" з пропорцією ---

\task{31}{Розв'яжіть рівняння $\dfrac{x}{3} = 4$. \gentask}
\answerTable{$1$}{$\dfrac{4}{3}$}{$-12$}{$7$}{$12$}

\vspace{0.4cm}

\task{32}{Розв'яжіть рівняння $\dfrac{x}{5} = -2$. \gentask}
\answerTable{$-10$}{$-\dfrac{2}{5}$}{$10$}{$3$}{$-7$}

\vspace{0.4cm}

\task{33}{Розв'яжіть рівняння $\dfrac{2x}{7} = 6$. \gentask}
\answerTable{$3$}{$\dfrac{6}{7}$}{$-21$}{$21$}{$\dfrac{12}{7}$}

\vspace{0.4cm}

\task{34}{Розв'яжіть рівняння $\dfrac{3x}{4} = 9$. \gentask}
\answerTable{$\dfrac{27}{4}$}{$3$}{$\dfrac{9}{4}$}{$-12$}{$12$}

\vspace{0.4cm}

\task{35}{Розв'яжіть рівняння $\dfrac{5x}{6} = -10$. \gentask}
\answerTable{$-12$}{$-\dfrac{10}{6}$}{$12$}{$-2$}{$-\dfrac{50}{6}$}

\vspace{0.4cm}

% --- Стиль: Пропорція з двох дробів ---

\task{36}{Розв'яжіть рівняння $\dfrac{x}{6} = \dfrac{2}{3}$. \gentask}
\answerTable{$4$}{$\dfrac{2}{18}$}{$-4$}{$9$}{$\dfrac{1}{9}$}

\vspace{0.4cm}

\task{37}{Розв'яжіть рівняння $\dfrac{x}{8} = \dfrac{3}{4}$. \gentask}
\answerTable{$\dfrac{24}{8}$}{$32$}{$-6$}{$\dfrac{3}{32}$}{$6$}

\vspace{0.4cm}

\task{38}{Розв'яжіть рівняння $\dfrac{2}{5} = \dfrac{x}{10}$. \gentask}
\answerTable{$1$}{$25$}{$4$}{$\dfrac{1}{25}$}{$-4$}

\vspace{0.4cm}

\task{39}{Розв'яжіть рівняння $\dfrac{5}{7} = \dfrac{x}{14}$. \gentask}
\answerTable{$\dfrac{5}{98}$}{$2$}{$10$}{$\dfrac{70}{14}$}{$-10$}

\vspace{0.4cm}

\task{40}{Розв'яжіть рівняння $\dfrac{x}{9} = \dfrac{4}{3}$. \gentask}
\answerTable{$\dfrac{36}{9}$}{$27$}{$\dfrac{4}{27}$}{$12$}{$-12$}

\vspace{0.4cm}

% --- Стиль: "Знайдіть корінь рівняння" ---

\task{41}{Знайдіть корінь рівняння $\dfrac{x + 2}{3} = 4$. \gentask}
\answerTable{$6$}{$2$}{$10$}{$-10$}{$14$}

\vspace{0.4cm}

\task{42}{Знайдіть корінь рівняння $\dfrac{x - 5}{4} = 2$. \gentask}
\answerTable{$8$}{$-3$}{$3$}{$13$}{$-13$}

\vspace{0.4cm}

\task{43}{Знайдіть корінь рівняння $\dfrac{2x + 1}{5} = 3$. \gentask}
\answerTable{$-7$}{$8$}{$2$}{$7$}{$14$}

\vspace{0.4cm}

\task{44}{Знайдіть корінь рівняння $\dfrac{3x - 4}{2} = 7$. \gentask}
\answerTable{$6$}{$5$}{$18$}{$\dfrac{11}{3}$}{$-6$}

\vspace{0.4cm}

\task{45}{Знайдіть корінь рівняння $\dfrac{4 - x}{6} = -1$. \gentask}
\answerTable{$-6$}{$2$}{$-2$}{$10$}{$-10$}

\vspace{0.4cm}

%======================================================================
% БЛОК 4: ДРОБОВО-РАЦІОНАЛЬНІ РІВНЯННЯ
%======================================================================

\newpage

\begin{center}
{\large\textbf{\color{headerblue}Блок 4: Дробово-раціональні рівняння}}
\end{center}

\vspace{0.3cm}

% --- Стиль: "Розв'яжіть рівняння" (дріб = 0) ---

\task{46}{Розв'яжіть рівняння $\dfrac{2x}{x + 3} = 0$. \gentask}
\answerTable{$2$}{$0$}{$-3$}{$3$}{$-2$}

\vspace{0.4cm}

\task{47}{Розв'яжіть рівняння $\dfrac{5x}{x - 4} = 0$. \gentask}
\answerTable{$4$}{$-5$}{$5$}{$0$}{$-4$}

\vspace{0.4cm}

\task{48}{Розв'яжіть рівняння $\dfrac{x - 7}{x + 2} = 0$. \gentask}
\answerTable{$-7$}{$7$}{$-2$}{$2$}{$0$}

\vspace{0.4cm}

\task{49}{Розв'яжіть рівняння $\dfrac{x + 6}{x - 1} = 0$. \gentask}
\answerTable{$6$}{$-6$}{$1$}{$-1$}{$0$}

\vspace{0.4cm}

\task{50}{Розв'яжіть рівняння $\dfrac{2x - 8}{x + 5} = 0$. \gentask}
\answerTable{$-4$}{$4$}{$-5$}{$0$}{$8$}

\vspace{0.4cm}

% --- Стиль: "Яке з наведених чисел є коренем" ---

\task{51}{Яке з наведених чисел є коренем рівняння $\dfrac{3x - 9}{x + 4} = 0$? \gentask}
\answerTable{$-3$}{$0$}{$-4$}{$3$}{$9$}

\vspace{0.4cm}

\task{52}{Яке з наведених чисел є коренем рівняння $\dfrac{2x + 10}{x - 3} = 0$? \gentask}
\answerTable{$5$}{$3$}{$0$}{$-10$}{$-5$}

\vspace{0.4cm}

\task{53}{Яке з наведених чисел є коренем рівняння $\dfrac{4x - 12}{x + 1} = 0$? \gentask}
\answerTable{$0$}{$-1$}{$12$}{$3$}{$-3$}

\vspace{0.4cm}

\task{54}{Яке з наведених чисел є коренем рівняння $\dfrac{x + 8}{2x - 6} = 0$? \gentask}
\answerTable{$8$}{$0$}{$-3$}{$-8$}{$3$}

\vspace{0.4cm}

\task{55}{Яке з наведених чисел є коренем рівняння $\dfrac{5x - 15}{x + 7} = 0$? \gentask}
\answerTable{$-7$}{$3$}{$0$}{$-3$}{$15$}

\vspace{0.4cm}

% --- Стиль: Рівняння типу a/x = b ---

\task{56}{Розв'яжіть рівняння $\dfrac{6}{x} = 2$. \gentask}
\answerTable{$\dfrac{1}{3}$}{$3$}{$4$}{$12$}{$-3$}

\vspace{0.4cm}

\task{57}{Розв'яжіть рівняння $\dfrac{10}{x} = -5$. \gentask}
\answerTable{$-50$}{$-2$}{$\dfrac{1}{2}$}{$2$}{$-15$}

\vspace{0.4cm}

\task{58}{Розв'яжіть рівняння $\dfrac{8}{x} = 4$. \gentask}
\answerTable{$32$}{$2$}{$\dfrac{1}{2}$}{$-2$}{$12$}

\vspace{0.4cm}

\task{59}{Розв'яжіть рівняння $\dfrac{15}{x} = -3$. \gentask}
\answerTable{$-5$}{$\dfrac{1}{5}$}{$-18$}{$-45$}{$5$}

\vspace{0.4cm}

\task{60}{Розв'яжіть рівняння $\dfrac{4}{x} = 1{,}6$. \gentask}
\answerTable{$0{,}4$}{$6{,}4$}{$2{,}5$}{$2{,}4$}{$-2{,}5$}

\vspace{0.4cm}

%======================================================================
% БЛОК 5: РІВНЯННЯ З ДВОХ ДРОБІВ
%======================================================================

\newpage

\begin{center}
{\large\textbf{\color{headerblue}Блок 5: Рівняння з двома дробами}}
\end{center}

\vspace{0.3cm}

% --- Стиль: "Розв'яжіть рівняння" ---

\task{61}{Розв'яжіть рівняння $\dfrac{x + 1}{2} = \dfrac{x - 3}{4}$. \gentask}
\answerTable{$-7$}{$-5$}{$5$}{$1$}{$7$}

\vspace{0.4cm}

\task{62}{Розв'яжіть рівняння $\dfrac{x - 2}{3} = \dfrac{x + 4}{5}$. \gentask}
\answerTable{$-1$}{$11$}{$1$}{$-11$}{$22$}

\vspace{0.4cm}

\task{63}{Розв'яжіть рівняння $\dfrac{2x + 1}{4} = \dfrac{x - 2}{3}$. \gentask}
\answerTableBig{$\dfrac{11}{2}$}{$-\dfrac{11}{2}$}{$\dfrac{2}{11}$}{$-11$}{$11$}

\vspace{0.4cm}

\task{64}{Розв'яжіть рівняння $\dfrac{3x - 1}{2} = \dfrac{2x + 5}{3}$. \gentask}
\answerTableBig{$-\dfrac{13}{5}$}{$\dfrac{13}{5}$}{$-13$}{$13$}{$\dfrac{5}{13}$}

\vspace{0.4cm}

\task{65}{Розв'яжіть рівняння $\dfrac{x + 5}{6} = \dfrac{2x - 1}{4}$. \gentask}
\answerTableBig{$\dfrac{4}{13}$}{$-\dfrac{13}{4}$}{$-13$}{$13$}{$\dfrac{13}{4}$}

\vspace{0.4cm}

% --- Стиль: "Знайдіть корінь рівняння" ---

\task{66}{Знайдіть корінь рівняння $\dfrac{x}{2} - \dfrac{x}{3} = 1$. \gentask}
\answerTable{$6$}{$-6$}{$-5$}{$1$}{$5$}

\vspace{0.4cm}

\task{67}{Знайдіть корінь рівняння $\dfrac{x}{4} + \dfrac{x}{6} = 5$. \gentask}
\answerTable{$10$}{$12$}{$-12$}{$24$}{$-10$}

\vspace{0.4cm}

\task{68}{Знайдіть корінь рівняння $\dfrac{x}{3} - \dfrac{x}{5} = 4$. \gentask}
\answerTable{$-30$}{$30$}{$-15$}{$60$}{$15$}

\vspace{0.4cm}

\task{69}{Знайдіть корінь рівняння $\dfrac{x}{2} + \dfrac{x}{4} = 9$. \gentask}
\answerTable{$6$}{$-6$}{$18$}{$12$}{$-12$}

\vspace{0.4cm}

\task{70}{Знайдіть корінь рівняння $\dfrac{x}{5} - \dfrac{x}{7} = 2$. \gentask}
\answerTable{$70$}{$35$}{$12$}{$-70$}{$-35$}

\vspace{0.4cm}

% --- Стиль: Складніші дробові рівняння ---

\task{71}{Розв'яжіть рівняння $\dfrac{2}{x - 1} = 4$. \gentask}
\answerTableBig{$\dfrac{3}{2}$}{$-1{,}5$}{$0{,}5$}{$3$}{$1{,}5$}

\vspace{0.4cm}

\task{72}{Розв'яжіть рівняння $\dfrac{6}{x + 2} = 3$. \gentask}
\answerTable{$-2$}{$4$}{$2$}{$0$}{$-4$}

\vspace{0.4cm}

\task{73}{Розв'яжіть рівняння $\dfrac{5}{2x - 3} = 1$. \gentask}
\answerTable{$-4$}{$1$}{$8$}{$-1$}{$4$}

\vspace{0.4cm}

\task{74}{Розв'яжіть рівняння $\dfrac{3}{x + 4} = -1$. \gentask}
\answerTable{$7$}{$-4$}{$-1$}{$-7$}{$1$}

\vspace{0.4cm}

\task{75}{Розв'яжіть рівняння $\dfrac{8}{3x - 1} = 2$. \gentask}
\answerTableBig{$-5$}{$-\dfrac{5}{3}$}{$\dfrac{5}{3}$}{$\dfrac{3}{5}$}{$5$}

\vspace{0.4cm}

%======================================================================
% БЛОК 6: ВИЗНАЧЕННЯ ПРОМІЖКУ ДЛЯ КОРЕНЯ
%======================================================================

\newpage

\begin{center}
{\large\textbf{\color{headerblue}Блок 6: Визначення проміжку для кореня}}
\end{center}

\vspace{0.3cm}

% --- Стиль: "Укажіть проміжок, якому належить корінь" ---

\task{76}{Укажіть проміжок, якому належить корінь рівняння $3x = 7$. \gentask}
\answerTable{$(5; 10]$}{$(0; 2]$}{$(2; 3]$}{$(3; 5]$}{$(-2; 0]$}

\vspace{0.4cm}

\task{77}{Укажіть проміжок, якому належить корінь рівняння $5x = 12$. \gentask}
\answerTable{$(5; 10]$}{$(0; 2]$}{$(3; 5]$}{$(2; 3]$}{$(-3; 0]$}

\vspace{0.4cm}

\task{78}{Укажіть проміжок, якому належить корінь рівняння $7x = -21$. \gentask}
\answerTable{$(3; 10]$}{$[-10; -5)$}{$[-3; 0)$}{$(-5; -3)$}{$(0; 3]$}

\vspace{0.4cm}

\task{79}{Укажіть проміжок, якому належить корінь рівняння $4x = 15$. \gentask}
\answerTable{$(3; 4]$}{$(5; 10]$}{$(0; 3]$}{$(4; 5]$}{$(-4; 0]$}

\vspace{0.4cm}

\task{80}{Укажіть проміжок, якому належить корінь рівняння $6x = -8$. \gentask}
\answerTable{$(-2; 0)$}{$(2; 5]$}{$(-5; -2]$}{$(0; 2]$}{$[-10; -5)$}

\vspace{0.4cm}

% --- Стиль: "Корінь рівняння ... належить проміжку" ---

\task{81}{Корінь рівняння $\dfrac{x}{7} = 3$ належить проміжку \gentask}
\answerTable{$(25; 30]$}{$(22; 25]$}{$(20; 22]$}{$(10; 15]$}{$(15; 20]$}

\vspace{0.4cm}

\task{82}{Корінь рівняння $\dfrac{2x}{9} = 4$ належить проміжку \gentask}
\answerTable{$(17; 19]$}{$(10; 15]$}{$(19; 22]$}{$(22; 25]$}{$(15; 17]$}

\vspace{0.4cm}

\task{83}{Корінь рівняння $\dfrac{5}{x - 2} = 1$ належить проміжку \gentask}
\answerTable{$(6; 8]$}{$(8; 10]$}{$(10; 15]$}{$(2; 4]$}{$(4; 6]$}

\vspace{0.4cm}

\task{84}{Корінь рівняння $\dfrac{3}{2x + 1} = \dfrac{1}{3}$ належить проміжку \gentask}
\answerTable{$(5; 7]$}{$(1; 3]$}{$(3; 5]$}{$(7; 10]$}{$(0; 1]$}

\vspace{0.4cm}

\task{85}{Корінь рівняння $\dfrac{x - 3}{4} = 2$ належить проміжку \gentask}
\answerTable{$(12; 15]$}{$(15; 20]$}{$(8; 10]$}{$(10; 12]$}{$(5; 8]$}

\vspace{0.4cm}

% --- Стиль: з числом пі ---

\task{86}{Корінь рівняння $2x = 7$ належить проміжку \gentask}
\answerTableBig{$(2\pi; 10)$}{$(0; \pi)$}{$(5; 2\pi)$}{$(4; 5)$}{$(\pi; 4)$}

\vspace{0.4cm}

\task{87}{Корінь рівняння $3x = 5$ належить проміжку \gentask}
\answerTableBig{$(4; 2\pi)$}{$(2; \pi)$}{$(1; \dfrac{\pi}{2})$}{$(\dfrac{\pi}{2}; 2)$}{$(\pi; 4)$}

\vspace{0.4cm}

\task{88}{Корінь рівняння $5x = 16$ належить проміжку \gentask}
\answerTableBig{$(\pi; 4)$}{$(5; 2\pi)$}{$(4; 5)$}{$(0; \pi)$}{$(2\pi; 10)$}

\vspace{0.4cm}

\task{89}{Корінь рівняння $\dfrac{4}{x - 1} = 2$ належить проміжку \gentask}
\answerTableBig{$(2; \pi)$}{$(\pi; 4)$}{$(4; 5)$}{$(1; 2)$}{$(0; \pi)$}

\vspace{0.4cm}

\task{90}{Корінь рівняння $\dfrac{x}{3} + 1 = 2$ належить проміжку \gentask}
\answerTableBig{$(0; \pi)$}{$(\pi; 4)$}{$(5; 2\pi)$}{$(4; 5)$}{$(2; \pi)$}

\vspace{0.4cm}

%======================================================================
% БЛОК 7: РІВНЯННЯ ЗІ ЗМІШАНИМИ ЧИСЛАМИ
%======================================================================

\newpage

\begin{center}
{\large\textbf{\color{headerblue}Блок 7: Рівняння зі змішаними числами}}
\end{center}

\vspace{0.3cm}

% --- Стиль: "Розв'яжіть рівняння" ---

\task{91}{Розв'яжіть рівняння $x + 2\dfrac{1}{3} = 5$. \gentask}
\answerTableBig{$7\dfrac{1}{3}$}{$2\dfrac{2}{3}$}{$3\dfrac{2}{3}$}{$-2\dfrac{2}{3}$}{$\dfrac{2}{3}$}

\vspace{0.4cm}

\task{92}{Розв'яжіть рівняння $x - 1\dfrac{1}{2} = 3$. \gentask}
\answerTableBig{$-4\dfrac{1}{2}$}{$4\dfrac{1}{2}$}{$2\dfrac{1}{2}$}{$1\dfrac{1}{2}$}{$\dfrac{3}{2}$}

\vspace{0.4cm}

\task{93}{Розв'яжіть рівняння $4 - x = 1\dfrac{3}{4}$. \gentask}
\answerTableBig{$2\dfrac{1}{4}$}{$5\dfrac{3}{4}$}{$-2\dfrac{1}{4}$}{$\dfrac{1}{4}$}{$3\dfrac{1}{4}$}

\vspace{0.4cm}

\task{94}{Розв'яжіть рівняння $2x = 3\dfrac{1}{5}$. \gentask}
\answerTableBig{$\dfrac{8}{5}$}{$-1\dfrac{3}{5}$}{$6\dfrac{2}{5}$}{$1\dfrac{3}{5}$}{$\dfrac{16}{10}$}

\vspace{0.4cm}

\task{95}{Розв'яжіть рівняння $\dfrac{x}{3} = 2\dfrac{1}{6}$. \gentask}
\answerTableBig{$\dfrac{2}{13}$}{$6\dfrac{1}{2}$}{$-6\dfrac{1}{2}$}{$\dfrac{13}{2}$}{$\dfrac{13}{18}$}

\vspace{0.4cm}

% --- Стиль: "Знайдіть корінь рівняння" ---

\task{96}{Знайдіть корінь рівняння $5 - x = 2\dfrac{2}{5}$. \gentask}
\answerTableBig{$-2\dfrac{3}{5}$}{$7\dfrac{2}{5}$}{$\dfrac{13}{5}$}{$3\dfrac{2}{5}$}{$2\dfrac{3}{5}$}

\vspace{0.4cm}

\task{97}{Знайдіть корінь рівняння $3x = 4\dfrac{1}{2}$. \gentask}
\answerTableBig{$\dfrac{3}{2}$}{$-1\dfrac{1}{2}$}{$\dfrac{2}{3}$}{$13\dfrac{1}{2}$}{$1\dfrac{1}{2}$}

\vspace{0.4cm}

\task{98}{Знайдіть корінь рівняння $x + 3\dfrac{3}{4} = 7$. \gentask}
\answerTableBig{$\dfrac{13}{4}$}{$4\dfrac{1}{4}$}{$-3\dfrac{1}{4}$}{$10\dfrac{3}{4}$}{$3\dfrac{1}{4}$}

\vspace{0.4cm}

\task{99}{Знайдіть корінь рівняння $\dfrac{2x}{5} = 1\dfrac{2}{5}$. \gentask}
\answerTableBig{$\dfrac{7}{25}$}{$\dfrac{7}{2}$}{$3\dfrac{1}{2}$}{$\dfrac{2}{7}$}{$-3\dfrac{1}{2}$}

\vspace{0.4cm}

\task{100}{Знайдіть корінь рівняння $6 - 2x = 1\dfrac{1}{3}$. \gentask}
\answerTableBig{$\dfrac{14}{6}$}{$-2\dfrac{1}{3}$}{$3\dfrac{2}{3}$}{$\dfrac{7}{3}$}{$2\dfrac{1}{3}$}

\vspace{0.4cm}

\end{document}
