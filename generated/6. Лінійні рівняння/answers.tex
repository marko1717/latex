\documentclass[12pt]{extarticle}
\usepackage{fontspec}
\usepackage{polyglossia}
\setdefaultlanguage{ukrainian}

\defaultfontfeatures{Ligatures=TeX}
\setmainfont{Liberation Serif}

\usepackage[a4paper,margin=2cm]{geometry}
\usepackage{amsmath,amssymb}
\usepackage{multicol}
\usepackage{xcolor}

\definecolor{headerblue}{RGB}{0, 102, 204}

\begin{document}

\begin{center}
{\Large\textbf{\color{headerblue}ВІДПОВІДІ}}
\end{center}

\begin{center}
{\large Тема 6: Лінійні рівняння}
\end{center}

\vspace{0.5cm}

\textbf{Блок 1: Прості лінійні рівняння $ax = b$}

\begin{multicols}{5}
\noindent
1. А ($x=3$) \\
2. А ($x=-3$) \\
3. А ($x=-3$) \\
4. А ($x=3$) \\
5. А ($x=3$) \\
6. А ($x=20$) \\
7. А ($x=-6$) \\
8. А ($x=20$) \\
9. А ($x=-70$) \\
10. А ($x=50$) \\
11. А ($x=-20$) \\
12. А ($x=6$) \\
13. А ($x=4$) \\
14. А ($x=-5$) \\
15. А ($x=-3$)
\end{multicols}

\textbf{Блок 2: Рівняння з дужками}

\begin{multicols}{5}
\noindent
16. А ($x=2$) \\
17. А ($x=-2$) \\
18. А ($x=3$) \\
19. А ($x=-2{,}5$) \\
20. А ($x=0{,}25$) \\
21. А ($x=3$) \\
22. А ($x=2$) \\
23. А ($x=2$) \\
24. А ($x=3$) \\
25. А ($x=2$) \\
26. А ($x=1$) \\
27. А ($x=2$) \\
28. А ($x=8$) \\
29. А ($x=-2$) \\
30. А ($x=3$)
\end{multicols}

\textbf{Блок 3: Рівняння з дробами}

\begin{multicols}{5}
\noindent
31. А ($x=12$) \\
32. А ($x=-10$) \\
33. А ($x=21$) \\
34. А ($x=12$) \\
35. А ($x=-12$) \\
36. А ($x=4$) \\
37. А ($x=6$) \\
38. А ($x=4$) \\
39. А ($x=10$) \\
40. А ($x=12$) \\
41. А ($x=10$) \\
42. А ($x=13$) \\
43. А ($x=7$) \\
44. А ($x=6$) \\
45. А ($x=10$)
\end{multicols}

\textbf{Блок 4: Дробово-раціональні рівняння}

\begin{multicols}{5}
\noindent
46. А ($x=0$) \\
47. А ($x=0$) \\
48. А ($x=7$) \\
49. А ($x=-6$) \\
50. А ($x=4$) \\
51. А ($x=3$) \\
52. А ($x=-5$) \\
53. А ($x=3$) \\
54. А ($x=-8$) \\
55. А ($x=3$) \\
56. А ($x=3$) \\
57. А ($x=-2$) \\
58. А ($x=2$) \\
59. А ($x=-5$) \\
60. А ($x=2{,}5$)
\end{multicols}

\textbf{Блок 5: Рівняння з двома дробами}

\begin{multicols}{5}
\noindent
61. А ($x=-7$) \\
62. А ($x=-11$) \\
63. А ($x=-5{,}5$) \\
64. А ($x=2{,}6$) \\
65. А ($x=3{,}25$) \\
66. А ($x=6$) \\
67. А ($x=12$) \\
68. А ($x=30$) \\
69. А ($x=12$) \\
70. А ($x=35$) \\
71. А ($x=1{,}5$) \\
72. А ($x=0$) \\
73. А ($x=4$) \\
74. А ($x=-7$) \\
75. А ($x=5/3$)
\end{multicols}

\textbf{Блок 6: Визначення проміжку для кореня}

\begin{multicols}{5}
\noindent
76. А ($x\approx2{,}33$) \\
77. А ($x=2{,}4$) \\
78. А ($x=-3$) \\
79. А ($x=3{,}75$) \\
80. А ($x\approx-1{,}33$) \\
81. А ($x=21$) \\
82. А ($x=18$) \\
83. А ($x=7$) \\
84. А ($x=4$) \\
85. А ($x=11$) \\
86. А ($x=3{,}5$) \\
87. А ($x\approx1{,}67$) \\
88. А ($x=3{,}2$) \\
89. Д ($x=3$) \\
90. Д ($x=3$)
\end{multicols}

\textbf{Блок 7: Рівняння зі змішаними числами}

\begin{multicols}{5}
\noindent
91. А ($x=2\frac{2}{3}$) \\
92. А ($x=4\frac{1}{2}$) \\
93. А ($x=2\frac{1}{4}$) \\
94. А ($x=1\frac{3}{5}$) \\
95. А ($x=6\frac{1}{2}$) \\
96. А ($x=2\frac{3}{5}$) \\
97. А ($x=1\frac{1}{2}$) \\
98. А ($x=3\frac{1}{4}$) \\
99. А ($x=3\frac{1}{2}$) \\
100. А ($x=2\frac{1}{3}$)
\end{multicols}

\vspace{1cm}

\textbf{Методи розв'язування:}

\begin{enumerate}
\item \textbf{Просте рівняння $ax = b$:} $x = \dfrac{b}{a}$

\item \textbf{Рівняння з дужками:} Розкрити дужки, звести подібні, розв'язати

\item \textbf{Дробове рівняння $\dfrac{f(x)}{g(x)} = 0$:} $f(x) = 0$ при $g(x) \neq 0$

\item \textbf{Рівняння типу $\dfrac{a}{x-b} = c$:} $a = c(x-b) \Rightarrow x = \dfrac{a}{c} + b$

\item \textbf{Проміжок для кореня:} Знайти корінь і визначити, якому проміжку він належить
\end{enumerate}

\end{document}
