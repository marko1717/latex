\documentclass[14pt]{extarticle}
\usepackage{fontspec}
\usepackage{polyglossia}
\setdefaultlanguage{ukrainian}

\defaultfontfeatures{Ligatures=TeX}
\setmainfont{Liberation Serif}
\setsansfont{Liberation Sans}
\setmonofont{Liberation Mono}

\usepackage[a4paper,margin=2cm,bottom=2.5cm,top=2.5cm]{geometry}
\usepackage{amsmath,amssymb}
\usepackage{enumitem}
\usepackage{tikz}
\usepackage{pgfplots}
\pgfplotsset{compat=1.16}
\usetikzlibrary{calc,patterns,angles,quotes}
\usepackage{xcolor}
\usepackage{array}
\usepackage{fancyhdr}
\usepackage{multicol}

% Кольори
\definecolor{headerblue}{RGB}{0, 102, 204}
\definecolor{yearcolor}{RGB}{128, 0, 128}

\pagestyle{fancy}
\fancyhf{}
\renewcommand{\headrulewidth}{0pt}
\fancyfoot[C]{\thepage}

\setlength{\headheight}{15pt}
\setlength{\headsep}{10pt}
\setlength{\footskip}{25pt}

\widowpenalty=10000
\clubpenalty=10000

% === КОМАНДИ ===

% Стандартна таблиця відповідей
\newcommand{\answerTable}[5]{
\begin{center}
\begin{tabular}{|*{5}{>{\centering\arraybackslash}m{2.8cm}|}}
\hline
\rule[-0.3cm]{0pt}{0.8cm}\textbf{А} & \textbf{Б} & \textbf{В} & \textbf{Г} & \textbf{Д} \\
\hline
\rule[-0.4cm]{0pt}{1.0cm}#1 & \rule[-0.4cm]{0pt}{1.0cm}#2 & \rule[-0.4cm]{0pt}{1.0cm}#3 & \rule[-0.4cm]{0pt}{1.0cm}#4 & \rule[-0.4cm]{0pt}{1.0cm}#5 \\
\hline
\end{tabular}
\end{center}
}

% Маленька таблиця відповідей
\newcommand{\answerTableSmall}[5]{
\begin{tabular}{|*{5}{>{\centering\arraybackslash}m{1.1cm}|}}
\hline
\rule[-0.2cm]{0pt}{0.6cm}\textbf{А} & \textbf{Б} & \textbf{В} & \textbf{Г} & \textbf{Д} \\
\hline
\rule[-0.3cm]{0pt}{0.8cm}#1 & #2 & #3 & #4 & #5 \\
\hline
\end{tabular}
}

% Команда для завдань
\newcommand{\gentask}[2]{\noindent\makebox[1.5em][l]{\textbf{#1.}}\parbox[t]{\dimexpr\textwidth-1.5em}{#2}}

\begin{document}

\begin{center}
{\Large\textbf{\color{headerblue}ЗГЕНЕРОВАНІ ЗАВДАННЯ}}
\end{center}

\begin{center}
{\large Тема 12: Чотирикутники (паралелограм, прямокутник, ромб, квадрат, трапеція)}
\end{center}

\vspace{0.5cm}

%======================================================================
% БЛОК 1: Паралелограм - основні властивості (15 завдань)
%======================================================================

\section*{Блок 1: Паралелограм - основні властивості}

\gentask{1}{У паралелограмі $ABCD$ $\angle A = 70°$. Знайдіть $\angle B$.}
\answerTable{$110°$}{$70°$}{$90°$}{$140°$}{$180°$}

\vspace{0.4cm}

\gentask{2}{У паралелограмі $ABCD$ $\angle A = 55°$. Знайдіть $\angle C$.}
\answerTable{$55°$}{$125°$}{$35°$}{$145°$}{$180°$}

\vspace{0.4cm}

\gentask{3}{У паралелограмі сума двох сусідніх кутів дорівнює:}
\answerTable{$180°$}{$360°$}{$90°$}{$270°$}{$120°$}

\vspace{0.4cm}

\gentask{4}{Сторони паралелограма дорівнюють 8 і 12 см. Знайдіть периметр.}
\answerTable{40 см}{20 см}{96 см}{80 см}{48 см}

\vspace{0.4cm}

\gentask{5}{Периметр паралелограма 56 см, одна сторона 18 см. Знайдіть другу сторону.}
\answerTable{10 см}{18 см}{28 см}{38 см}{20 см}

\vspace{0.4cm}

\gentask{6}{Діагоналі паралелограма перетинаються в точці $O$. $AC = 16$ см. Знайдіть $AO$.}
\answerTable{8 см}{16 см}{4 см}{32 см}{12 см}

\vspace{0.4cm}

\gentask{7}{Діагоналі паралелограма перетинаються в точці $O$. $BO = 7$ см. Знайдіть $BD$.}
\answerTable{14 см}{7 см}{3{,}5 см}{21 см}{28 см}

\vspace{0.4cm}

\gentask{8}{У паралелограмі один кут у 3 рази більший за інший. Знайдіть менший кут.}
\answerTable{$45°$}{$135°$}{$60°$}{$30°$}{$90°$}

\vspace{0.4cm}

\gentask{9}{У паралелограмі один кут на $40°$ більший за інший. Знайдіть більший кут.}
\answerTable{$110°$}{$70°$}{$100°$}{$120°$}{$140°$}

\vspace{0.4cm}

\gentask{10}{Діагональ паралелограма ділить його на:}
\answerTable{два рівних трикутники}{два подібних трикутники}{чотири трикутники}{нерівні частини}{залежить від форми}

\vspace{0.4cm}

\gentask{11}{Площа паралелограма 48 см$^2$, основа 8 см. Знайдіть висоту.}
\answerTable{6 см}{48 см}{384 см}{4 см}{12 см}

\vspace{0.4cm}

\gentask{12}{Сторона паралелограма 10 см, висота до неї 7 см. Знайдіть площу.}
\answerTable{70 см$^2$}{35 см$^2$}{17 см$^2$}{140 см$^2$}{100 см$^2$}

\vspace{0.4cm}

\gentask{13}{У паралелограмі протилежні сторони:}
\answerTable{рівні і паралельні}{рівні, не паралельні}{паралельні, не рівні}{перпендикулярні}{різні}

\vspace{0.4cm}

\gentask{14}{Сума кутів паралелограма дорівнює:}
\answerTable{$360°$}{$180°$}{$540°$}{$720°$}{$270°$}

\vspace{0.4cm}

\gentask{15}{Діагоналі паралелограма:}
\answerTable{діляться навпіл}{рівні}{перпендикулярні}{є бісектрисами}{усі варіанти}

\vspace{0.5cm}

%======================================================================
% БЛОК 2: Прямокутник (15 завдань)
%======================================================================

\section*{Блок 2: Прямокутник}

\gentask{16}{Сторони прямокутника 5 і 12 см. Знайдіть діагональ.}
\answerTable{13 см}{17 см}{7 см}{60 см}{$\sqrt{119}$ см}

\vspace{0.4cm}

\gentask{17}{Діагональ прямокутника 10 см, одна сторона 6 см. Знайдіть другу сторону.}
\answerTable{8 см}{4 см}{16 см}{$\sqrt{136}$ см}{14 см}

\vspace{0.4cm}

\gentask{18}{Діагональ прямокутника 15 см, одна сторона 9 см. Знайдіть периметр.}
\answerTable{42 см}{48 см}{24 см}{30 см}{54 см}

\vspace{0.4cm}

\gentask{19}{Периметр прямокутника 34 см, одна сторона 5 см. Знайдіть площу.}
\answerTable{60 см$^2$}{170 см$^2$}{85 см$^2$}{30 см$^2$}{120 см$^2$}

\vspace{0.4cm}

\gentask{20}{Площа прямокутника 84 см$^2$, одна сторона 7 см. Знайдіть периметр.}
\answerTable{38 см}{24 см}{91 см}{42 см}{48 см}

\vspace{0.4cm}

\gentask{21}{Діагоналі прямокутника:}
\answerTable{рівні}{перпендикулярні}{є бісектрисами}{не діляться навпіл}{різні}

\vspace{0.4cm}

\gentask{22}{Кути прямокутника:}
\answerTable{всі прямі}{два гострих, два тупих}{всі гострі}{всі тупі}{один прямий}

\vspace{0.4cm}

\gentask{23}{Діагональ прямокутника ділить його на:}
\answerTable{два рівних прямокутних трикутники}{два рівнобедрених}{чотири трикутники}{нерівні частини}{залежить від сторін}

\vspace{0.4cm}

\gentask{24}{Діагоналі прямокутника перетинаються в точці $O$. Трикутник $AOB$ є:}
\answerTable{рівнобедреним}{рівностороннім}{прямокутним}{тупокутним}{гострокутним}

\vspace{0.4cm}

\gentask{25}{Сторони прямокутника 3 і 4 см. Знайдіть радіус описаного кола.}
\answerTable{2{,}5 см}{5 см}{3{,}5 см}{7 см}{6 см}

\vspace{0.4cm}

\gentask{26}{Діагональ прямокутника дорівнює діаметру описаного кола. Це твердження:}
\answerTable{завжди правильне}{завжди неправильне}{іноді правильне}{залежить від сторін}{не має сенсу}

\vspace{0.4cm}

\gentask{27}{Сторони прямокутника 6 і 8 см. Знайдіть відстань від точки перетину діагоналей до вершини.}
\answerTable{5 см}{10 см}{7 см}{14 см}{4 см}

\vspace{0.4cm}

\gentask{28}{Прямокутник із рівними сторонами називається:}
\answerTable{квадратом}{ромбом}{трапецією}{паралелограмом}{шестикутником}

\vspace{0.4cm}

\gentask{29}{Площа прямокутника $S = ab$. Якщо $a = 2b$, то $S =$}
\answerTable{$2b^2$}{$b^2$}{$4b^2$}{$3b^2$}{$2b$}

\vspace{0.4cm}

\gentask{30}{Діагональ прямокутника 20 см, кут між діагоналями 60°. Знайдіть меншу сторону.}
\answerTable{10 см}{$10\sqrt{3}$ см}{20 см}{5 см}{15 см}

\vspace{0.5cm}

%======================================================================
% БЛОК 3: Ромб (15 завдань)
%======================================================================

\section*{Блок 3: Ромб}

\gentask{31}{Периметр ромба 48 см. Знайдіть його сторону.}
\answerTable{12 см}{24 см}{48 см}{6 см}{16 см}

\vspace{0.4cm}

\gentask{32}{Сторона ромба 9 см. Знайдіть периметр.}
\answerTable{36 см}{18 см}{81 см}{27 см}{45 см}

\vspace{0.4cm}

\gentask{33}{Діагоналі ромба:}
\answerTable{перпендикулярні і діляться навпіл}{рівні}{паралельні}{є сторонами}{не перетинаються}

\vspace{0.4cm}

\gentask{34}{Діагоналі ромба 6 і 8 см. Знайдіть сторону.}
\answerTable{5 см}{7 см}{10 см}{14 см}{$\sqrt{50}$ см}

\vspace{0.4cm}

\gentask{35}{Діагоналі ромба 10 і 24 см. Знайдіть площу.}
\answerTable{120 см$^2$}{240 см$^2$}{60 см$^2$}{34 см$^2$}{170 см$^2$}

\vspace{0.4cm}

\gentask{36}{Площа ромба 54 см$^2$, одна діагональ 9 см. Знайдіть другу діагональ.}
\answerTable{12 см}{6 см}{27 см}{18 см}{108 см}

\vspace{0.4cm}

\gentask{37}{Діагоналі ромба є:}
\answerTable{бісектрисами кутів}{медіанами}{висотами}{серединними перпендикулярами}{жодним з цих}

\vspace{0.4cm}

\gentask{38}{Сторона ромба 13 см, одна діагональ 10 см. Знайдіть другу діагональ.}
\answerTable{24 см}{12 см}{23 см}{26 см}{$\sqrt{69}$ см}

\vspace{0.4cm}

\gentask{39}{Один кут ромба $60°$. Знайдіть сусідній кут.}
\answerTable{$120°$}{$60°$}{$30°$}{$90°$}{$150°$}

\vspace{0.4cm}

\gentask{40}{Діагональ ромба ділить його на:}
\answerTable{два рівних трикутники}{чотири рівних трикутники}{два прямокутних}{нерівні частини}{залежить від кута}

\vspace{0.4cm}

\gentask{41}{Ромб із прямими кутами називається:}
\answerTable{квадратом}{прямокутником}{трапецією}{паралелограмом}{правильним}

\vspace{0.4cm}

\gentask{42}{Сторона ромба 10 см, висота 8 см. Знайдіть площу.}
\answerTable{80 см$^2$}{40 см$^2$}{100 см$^2$}{18 см$^2$}{160 см$^2$}

\vspace{0.4cm}

\gentask{43}{Діагоналі ромба 12 і 16 см. Знайдіть периметр.}
\answerTable{40 см}{56 см}{28 см}{80 см}{100 см}

\vspace{0.4cm}

\gentask{44}{У ромбі $ABCD$ $\angle A = 50°$. Знайдіть $\angle ABD$.}
\answerTable{$65°$}{$25°$}{$50°$}{$130°$}{$115°$}

\vspace{0.4cm}

\gentask{45}{Площа ромба $S = \dfrac{d_1 \cdot d_2}{2}$. Якщо $d_1 = d_2 = d$, то $S =$}
\answerTable{$\dfrac{d^2}{2}$}{$d^2$}{$2d^2$}{$\dfrac{d}{2}$}{$d$}

\vspace{0.5cm}

%======================================================================
% БЛОК 4: Квадрат (15 завдань)
%======================================================================

\section*{Блок 4: Квадрат}

\gentask{46}{Сторона квадрата 7 см. Знайдіть периметр.}
\answerTable{28 см}{49 см}{14 см}{21 см}{35 см}

\vspace{0.4cm}

\gentask{47}{Периметр квадрата 36 см. Знайдіть сторону.}
\answerTable{9 см}{18 см}{6 см}{12 см}{36 см}

\vspace{0.4cm}

\gentask{48}{Сторона квадрата 5 см. Знайдіть площу.}
\answerTable{25 см$^2$}{20 см$^2$}{10 см$^2$}{50 см$^2$}{125 см$^2$}

\vspace{0.4cm}

\gentask{49}{Площа квадрата 64 см$^2$. Знайдіть сторону.}
\answerTable{8 см}{32 см}{16 см}{4 см}{$\sqrt{64}$ см}

\vspace{0.4cm}

\gentask{50}{Сторона квадрата 6 см. Знайдіть діагональ.}
\answerTable{$6\sqrt{2}$ см}{12 см}{6 см}{$12\sqrt{2}$ см}{$3\sqrt{2}$ см}

\vspace{0.4cm}

\gentask{51}{Діагональ квадрата $10\sqrt{2}$ см. Знайдіть сторону.}
\answerTable{10 см}{$10\sqrt{2}$ см}{20 см}{5 см}{$5\sqrt{2}$ см}

\vspace{0.4cm}

\gentask{52}{Діагональ квадрата 14 см. Знайдіть площу.}
\answerTable{98 см$^2$}{196 см$^2$}{49 см$^2$}{$98\sqrt{2}$ см$^2$}{$49\sqrt{2}$ см$^2$}

\vspace{0.4cm}

\gentask{53}{Площа квадрата 50 см$^2$. Знайдіть діагональ.}
\answerTable{10 см}{$5\sqrt{2}$ см}{$10\sqrt{2}$ см}{25 см}{100 см}

\vspace{0.4cm}

\gentask{54}{Сторона квадрата $a$. Радіус описаного кола дорівнює:}
\answerTable{$\dfrac{a\sqrt{2}}{2}$}{$a$}{$\dfrac{a}{2}$}{$a\sqrt{2}$}{$2a$}

\vspace{0.4cm}

\gentask{55}{Сторона квадрата $a$. Радіус вписаного кола дорівнює:}
\answerTable{$\dfrac{a}{2}$}{$a$}{$\dfrac{a\sqrt{2}}{2}$}{$a\sqrt{2}$}{$2a$}

\vspace{0.4cm}

\gentask{56}{Діагоналі квадрата:}
\answerTable{рівні, перпендикулярні, діляться навпіл}{тільки рівні}{тільки перпендикулярні}{різні}{паралельні}

\vspace{0.4cm}

\gentask{57}{Діагональ квадрата ділить його кут на:}
\answerTable{два кути по $45°$}{два кути по $90°$}{нерівні частини}{три частини}{залежить від квадрата}

\vspace{0.4cm}

\gentask{58}{Площа квадрата 144 см$^2$. Знайдіть периметр.}
\answerTable{48 см}{12 см}{24 см}{36 см}{72 см}

\vspace{0.4cm}

\gentask{59}{Периметр квадрата 20 см. Знайдіть площу.}
\answerTable{25 см$^2$}{100 см$^2$}{400 см$^2$}{5 см$^2$}{50 см$^2$}

\vspace{0.4cm}

\gentask{60}{Відношення радіусів описаного і вписаного кіл квадрата дорівнює:}
\answerTable{$\sqrt{2}$}{2}{$\dfrac{1}{\sqrt{2}}$}{1}{$\sqrt{3}$}

\vspace{0.5cm}

%======================================================================
% БЛОК 5: Трапеція - основні властивості (15 завдань)
%======================================================================

\section*{Блок 5: Трапеція - основні властивості}

\gentask{61}{Основи трапеції 8 і 14 см. Знайдіть середню лінію.}
\answerTable{11 см}{22 см}{6 см}{3 см}{112 см}

\vspace{0.4cm}

\gentask{62}{Середня лінія трапеції 10 см, одна основа 6 см. Знайдіть другу основу.}
\answerTable{14 см}{4 см}{16 см}{20 см}{8 см}

\vspace{0.4cm}

\gentask{63}{Основи трапеції 10 і 18 см, висота 7 см. Знайдіть площу.}
\answerTable{98 см$^2$}{196 см$^2$}{126 см$^2$}{70 см$^2$}{180 см$^2$}

\vspace{0.4cm}

\gentask{64}{Площа трапеції 60 см$^2$, основи 8 і 12 см. Знайдіть висоту.}
\answerTable{6 см}{5 см}{10 см}{3 см}{12 см}

\vspace{0.4cm}

\gentask{65}{Площа трапеції 72 см$^2$, висота 8 см. Знайдіть середню лінію.}
\answerTable{9 см}{18 см}{576 см}{4{,}5 см}{36 см}

\vspace{0.4cm}

\gentask{66}{Середня лінія трапеції паралельна:}
\answerTable{основам}{бічним сторонам}{діагоналям}{висоті}{нічому}

\vspace{0.4cm}

\gentask{67}{Сума кутів при бічній стороні трапеції дорівнює:}
\answerTable{$180°$}{$90°$}{$360°$}{$270°$}{залежить від трапеції}

\vspace{0.4cm}

\gentask{68}{Бічна сторона трапеції 10 см, кут при основі $30°$. Знайдіть висоту.}
\answerTable{5 см}{$5\sqrt{3}$ см}{10 см}{$10\sqrt{3}$ см}{$\dfrac{10}{\sqrt{3}}$ см}

\vspace{0.4cm}

\gentask{69}{У рівнобічній трапеції бічна сторона 13 см, основи 8 і 18 см. Знайдіть висоту.}
\answerTable{12 см}{5 см}{$\sqrt{119}$ см}{10 см}{13 см}

\vspace{0.4cm}

\gentask{70}{Діагоналі рівнобічної трапеції:}
\answerTable{рівні}{перпендикулярні}{паралельні}{є бісектрисами}{різні}

\vspace{0.4cm}

\gentask{71}{Кути при основі рівнобічної трапеції:}
\answerTable{рівні}{різні}{один гострий, один тупий}{обидва прямі}{залежить від основи}

\vspace{0.4cm}

\gentask{72}{У прямокутній трапеції один кут прямий. Скільки всього прямих кутів?}
\answerTable{2}{1}{4}{3}{0}

\vspace{0.4cm}

\gentask{73}{Основи трапеції 6 і 10 см. Відрізок, що з'єднує середини діагоналей, дорівнює:}
\answerTable{2 см}{4 см}{8 см}{16 см}{3 см}

\vspace{0.4cm}

\gentask{74}{У трапеції $ABCD$ ($BC \parallel AD$) $\angle A = 70°$. Знайдіть $\angle B$.}
\answerTable{$110°$}{$70°$}{$180°$}{$90°$}{не можна визначити}

\vspace{0.4cm}

\gentask{75}{Діагональ трапеції ділить середню лінію на відрізки 3 і 5 см. Знайдіть основи.}
\answerTable{6 і 10 см}{3 і 5 см}{8 і 8 см}{4 і 12 см}{5 і 11 см}

\vspace{0.5cm}

%======================================================================
% БЛОК 6: Порівняння чотирикутників (12 завдань)
%======================================================================

\section*{Блок 6: Порівняння чотирикутників}

\gentask{76}{Який чотирикутник має рівні діагоналі і рівні сторони?}
\answerTable{квадрат}{прямокутник}{ромб}{паралелограм}{трапеція}

\vspace{0.4cm}

\gentask{77}{Який чотирикутник має перпендикулярні діагоналі, але нерівні сторони?}
\answerTable{ромб}{квадрат}{прямокутник}{трапеція}{жоден}

\vspace{0.4cm}

\gentask{78}{Прямокутник є окремим випадком:}
\answerTable{паралелограма}{ромба}{трапеції}{квадрата}{п'ятикутника}

\vspace{0.4cm}

\gentask{79}{Ромб є окремим випадком:}
\answerTable{паралелограма}{прямокутника}{трапеції}{квадрата}{трикутника}

\vspace{0.4cm}

\gentask{80}{Квадрат є одночасно:}
\answerTable{прямокутником і ромбом}{прямокутником і трапецією}{ромбом і трапецією}{тільки прямокутником}{тільки ромбом}

\vspace{0.4cm}

\gentask{81}{У якого чотирикутника діагоналі рівні, але не перпендикулярні?}
\answerTable{прямокутника}{ромба}{квадрата}{паралелограма}{усіх}

\vspace{0.4cm}

\gentask{82}{У якого чотирикутника діагоналі перпендикулярні, але не рівні?}
\answerTable{ромба}{прямокутника}{квадрата}{трапеції}{паралелограма}

\vspace{0.4cm}

\gentask{83}{Який чотирикутник може мати тільки одну пару паралельних сторін?}
\answerTable{трапеція}{паралелограм}{прямокутник}{ромб}{квадрат}

\vspace{0.4cm}

\gentask{84}{Діагоналі якого чотирикутника завжди рівні?}
\answerTable{прямокутника і квадрата}{тільки квадрата}{ромба і квадрата}{паралелограма}{трапеції}

\vspace{0.4cm}

\gentask{85}{Діагоналі якого чотирикутника є бісектрисами кутів?}
\answerTable{ромба і квадрата}{тільки квадрата}{прямокутника}{паралелограма}{трапеції}

\vspace{0.4cm}

\gentask{86}{У якого чотирикутника всі сторони рівні, а кути~--- ні?}
\answerTable{ромба}{квадрата}{прямокутника}{паралелограма}{трапеції}

\vspace{0.4cm}

\gentask{87}{У якого чотирикутника всі кути рівні, а сторони~--- ні?}
\answerTable{прямокутника}{квадрата}{ромба}{паралелограма}{трапеції}

\vspace{0.5cm}

%======================================================================
% БЛОК 7: Задачі на обчислення (13 завдань)
%======================================================================

\section*{Блок 7: Задачі на обчислення}

\gentask{88}{Діагоналі паралелограма 10 і 14 см, кут між ними $90°$. Знайдіть площу.}
\answerTable{70 см$^2$}{140 см$^2$}{35 см$^2$}{24 см$^2$}{48 см$^2$}

\vspace{0.4cm}

\gentask{89}{Сторона ромба 10 см, один кут $60°$. Знайдіть меншу діагональ.}
\answerTable{10 см}{$10\sqrt{3}$ см}{20 см}{5 см}{$5\sqrt{3}$ см}

\vspace{0.4cm}

\gentask{90}{Діагональ квадрата 8 см. Знайдіть сторону.}
\answerTable{$4\sqrt{2}$ см}{4 см}{8 см}{$8\sqrt{2}$ см}{16 см}

\vspace{0.4cm}

\gentask{91}{У рівнобічній трапеції діагональ перпендикулярна бічній стороні. Бічна сторона 8 см, менша основа 6 см. Знайдіть більшу основу.}
\answerTable{16 см}{10 см}{14 см}{12 см}{18 см}

\vspace{0.4cm}

\gentask{92}{Сторони прямокутника відносяться як 3:4, площа 48 см$^2$. Знайдіть меншу сторону.}
\answerTable{6 см}{4 см}{8 см}{12 см}{3 см}

\vspace{0.4cm}

\gentask{93}{Кут паралелограма $60°$, сторони 6 і 10 см. Знайдіть площу.}
\answerTable{$30\sqrt{3}$ см$^2$}{60 см$^2$}{30 см$^2$}{$60\sqrt{3}$ см$^2$}{120 см$^2$}

\vspace{0.4cm}

\gentask{94}{У ромбі сторона 6 см, один кут $120°$. Знайдіть меншу діагональ.}
\answerTable{6 см}{$6\sqrt{3}$ см}{12 см}{3 см}{$3\sqrt{3}$ см}

\vspace{0.4cm}

\gentask{95}{Основи трапеції 5 і 11 см, бічні сторони 4 і 6 см. Знайдіть периметр.}
\answerTable{26 см}{22 см}{16 см}{55 см}{10 см}

\vspace{0.4cm}

\gentask{96}{Площа ромба 24 см$^2$, сторона 6 см. Знайдіть висоту.}
\answerTable{4 см}{8 см}{6 см}{12 см}{2 см}

\vspace{0.4cm}

\gentask{97}{Діагоналі ромба відносяться як 3:4, площа 24 см$^2$. Знайдіть меншу діагональ.}
\answerTable{6 см}{8 см}{4 см}{3 см}{12 см}

\vspace{0.4cm}

\gentask{98}{Середня лінія трапеції 15 см. Одна основа більша за іншу на 8 см. Знайдіть більшу основу.}
\answerTable{19 см}{11 см}{15 см}{23 см}{30 см}

\vspace{0.4cm}

\gentask{99}{Діагоналі прямокутника перетинаються під кутом $60°$, менша сторона 4 см. Знайдіть діагональ.}
\answerTable{8 см}{$4\sqrt{3}$ см}{$8\sqrt{3}$ см}{4 см}{$\dfrac{8}{\sqrt{3}}$ см}

\vspace{0.4cm}

\gentask{100}{У квадрат вписано коло радіусом 5 см. Знайдіть площу квадрата.}
\answerTable{100 см$^2$}{25 см$^2$}{$25\pi$ см$^2$}{50 см$^2$}{$100\pi$ см$^2$}

\vspace{0.5cm}

\end{document}
