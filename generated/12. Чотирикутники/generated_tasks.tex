\documentclass[14pt]{extarticle}
\usepackage{fontspec}
\usepackage{polyglossia}
\setdefaultlanguage{ukrainian}

\defaultfontfeatures{Ligatures=TeX}
\setmainfont{Liberation Serif}
\setsansfont{Liberation Sans}
\setmonofont{Liberation Mono}

\usepackage[a4paper,margin=1.5cm,bottom=2cm,top=2cm]{geometry}
\usepackage{amsmath,amssymb}
\usepackage{enumitem}
\usepackage{tikz}
\usepackage{pgfplots}
\pgfplotsset{compat=1.16}
\usetikzlibrary{calc,patterns,angles,quotes,intersections,babel}
\usepackage{xcolor}
\usepackage{array}
\usepackage{fancyhdr}
\usepackage{multirow}

% Кольори
\definecolor{headerblue}{RGB}{0, 102, 204}
\definecolor{yearcolor}{RGB}{128, 0, 128}

\pagestyle{fancy}
\fancyhf{}
\renewcommand{\headrulewidth}{0pt}
\fancyfoot[C]{\thepage}

\setlength{\headheight}{15pt}
\setlength{\headsep}{10pt}
\setlength{\footskip}{25pt}

\widowpenalty=10000
\clubpenalty=10000

% === КОМАНДИ ===

% Таблиця для відповідей із дробами (збільшена висота)
\newcommand{\answerTableTall}[5]{
\begin{center}
\begin{tabular}{|*{5}{>{\centering\arraybackslash}m{2.8cm}|}}
\hline
\rule[-0.3cm]{0pt}{0.8cm}\textbf{А} & \textbf{Б} & \textbf{В} & \textbf{Г} & \textbf{Д} \\
\hline
\rule[-0.9cm]{0pt}{2.0cm}#1 &
\rule[-0.9cm]{0pt}{2.0cm}#2 &
\rule[-0.9cm]{0pt}{2.0cm}#3 &
\rule[-0.9cm]{0pt}{2.0cm}#4 &
\rule[-0.9cm]{0pt}{2.0cm}#5 \\
\hline
\end{tabular}
\end{center}
}

% Таблиця для відповідності
\newcommand{\answerGrid}{
    \begingroup
    \renewcommand{\arraystretch}{1.3}
    \setlength{\tabcolsep}{7pt}
    \begin{tabular}{r|c|c|c|c|c|}
         \multicolumn{1}{c}{} & \multicolumn{1}{c}{\textbf{А}} & \multicolumn{1}{c}{\textbf{Б}} & \multicolumn{1}{c}{\textbf{В}} & \multicolumn{1}{c}{\textbf{Г}} & \multicolumn{1}{c}{\textbf{Д}} \\ \cline{2-6}
         \textbf{1} & & & & & \\ \cline{2-6}
         \textbf{2} & & & & & \\ \cline{2-6}
         \textbf{3} & & & & & \\ \cline{2-6}
    \end{tabular}
    \endgroup
}

% Макет для завдань на відповідність
\newcommand{\matchingLayout}[3]{
    \noindent
    \begin{minipage}[t]{0.40\textwidth}
        \textit{Величина / Початок} \par \vspace{0.2cm}
        #1
    \end{minipage}%
    \hfill
    \begin{minipage}[t]{0.28\textwidth}
        \textit{Значення / Закінчення} \par \vspace{0.2cm}
        #2
    \end{minipage}%
    \hfill
    \begin{minipage}[t]{0.30\textwidth}
        \vspace{0pt}
        \begin{flushright}
        #3
        \end{flushright}
    \end{minipage}
}

% Маленька таблиця відповідей
\newcommand{\answerTableSmall}[5]{
\begin{tabular}{|*{5}{>{\centering\arraybackslash}m{1.65cm}|}}
\hline
\rule[-0.2cm]{0pt}{0.6cm}\textbf{А} & \textbf{Б} & \textbf{В} & \textbf{Г} & \textbf{Д} \\
\hline
\rule[-0.4cm]{0pt}{0.9cm}#1 &
\rule[-0.4cm]{0pt}{0.9cm}#2 &
\rule[-0.4cm]{0pt}{0.9cm}#3 &
\rule[-0.4cm]{0pt}{0.9cm}#4 &
\rule[-0.4cm]{0pt}{0.9cm}#5 \\
\hline
\end{tabular}
}

% Стандартна таблиця відповідей
\newcommand{\answerTable}[5]{
\begin{center}
\begin{tabular}{|*{5}{>{\centering\arraybackslash}m{2.8cm}|}}
\hline
\rule[-0.3cm]{0pt}{0.8cm}\textbf{А} & \textbf{Б} & \textbf{В} & \textbf{Г} & \textbf{Д} \\
\hline
\rule[-0.4cm]{0pt}{1.0cm}#1 & \rule[-0.4cm]{0pt}{1.0cm}#2 & \rule[-0.4cm]{0pt}{1.0cm}#3 & \rule[-0.4cm]{0pt}{1.0cm}#4 & \rule[-0.4cm]{0pt}{1.0cm}#5 \\
\hline
\end{tabular}
\end{center}
}

% Команда для року
\newcommand{\nmtyear}[1]{\hfill{\small\color{yearcolor}(#1)}}

\begin{document}

\begin{center}
{\Large\textbf{\color{headerblue}ЗГЕНЕРОВАНІ ЗАВДАННЯ}}
\end{center}

\begin{center}
{\large Тема 12: Чотирикутники}
\end{center}

\vspace{0.5cm}

%======================================================================
% БЛОК 1: Паралелограм - основні властивості (15 завдань)
%======================================================================

\section*{Блок 1: Паралелограм - основні властивості}

% === ЗАВДАННЯ 1 ===
\noindent\makebox[1.5em][l]{\textbf{1.}}\parbox[t]{\dimexpr\textwidth-1.5em}{Сторона $CD$ паралелограма $ABCD$ утворює з прямою $AD$ кут $48°$ (див. рисунок). Знайдіть градусну міру кута $MAB$. \nmtyear{gen}}

\vspace{0.3cm}
\begin{minipage}{0.42\textwidth}
\answerTableSmall{$132°$}{$128°$}{$138°$}{$142°$}{$48°$}
\end{minipage}
\hfill
\begin{minipage}{0.52\textwidth}
\begin{flushright}
\begin{tikzpicture}[scale=1]
    \coordinate (M) at (-0.8,0);
    \coordinate (A) at (0,0);
    \coordinate (D) at (3,0);
    \coordinate (E) at (3.8,0);
    \coordinate (B) at (0.7,1.2);
    \coordinate (C) at (3.7,1.2);

    \draw[thick] (A) -- (B) -- (C) -- (D) -- cycle;
    \draw[thick] (M) -- (A);
    \draw[thick] (D) -- (E);

    \pic [draw, pic text={$48°$}, angle radius=0.5cm, angle eccentricity=1.9] {angle = E--D--C};

    \pic [draw, angle radius=0.4cm] {angle = B--A--M};
    \pic [draw, angle radius=0.5cm, "?" anchor=south east] {angle = B--A--M};

    \node[below] at (M) {$M$};
    \node[below] at (A) {$A$};
    \node[below] at (D) {$D$};
    \node[above] at (B) {$B$};
    \node[above] at (C) {$C$};
    \fill (M) circle (1.5pt);
\end{tikzpicture}
\end{flushright}
\end{minipage}

\vspace{0.7cm}

% === ЗАВДАННЯ 2 ===
\noindent\makebox[1.5em][l]{\textbf{2.}}\parbox[t]{\dimexpr\textwidth-1.5em}{У паралелограмі $ABCD$ $\angle A = 70°$ (див. рисунок). Знайдіть градусну міру кута $B$. \nmtyear{gen}}

\vspace{0.3cm}
\begin{minipage}{0.42\textwidth}
\answerTableSmall{$140°$}{$110°$}{$70°$}{$35°$}{$90°$}
\end{minipage}
\hfill
\begin{minipage}{0.52\textwidth}
\begin{flushright}
\begin{tikzpicture}[scale=0.9]
    \coordinate (A) at (0,0);
    \coordinate (B) at (1.2,2);
    \coordinate (C) at (4.5,2);
    \coordinate (D) at (3.3,0);

    \draw[thick] (A) -- (B) -- (C) -- (D) -- cycle;

    \pic [draw, pic text={$70°$}, angle radius=0.5cm, angle eccentricity=1.7] {angle = D--A--B};
    \pic [draw, angle radius=0.4cm, "?" anchor=west] {angle = A--B--C};

    \node[below left] at (A) {$A$};
    \node[above left] at (B) {$B$};
    \node[above right] at (C) {$C$};
    \node[below right] at (D) {$D$};
\end{tikzpicture}
\end{flushright}
\end{minipage}

\vspace{0.7cm}

% === ЗАВДАННЯ 3 ===
\noindent\makebox[1.5em][l]{\textbf{3.}}\parbox[t]{\dimexpr\textwidth-1.5em}{Які з наведених тверджень є правильними? \nmtyear{gen}}

\vspace{0.2cm}
\begin{tabular}{r@{\hspace{0.5em}}p{14cm}}
I. & Діагональ паралелограма ділить його на два рівних трикутники. \\
II. & Діагоналі паралелограма є бісектрисами його кутів. \\
III. & Сума сусідніх кутів паралелограма дорівнює $180°$. \\
\end{tabular}

\answerTable{I, II та III}{лише I}{лише II}{лише I та III}{лише II та III}

\vspace{0.7cm}

% === ЗАВДАННЯ 4 ===
\noindent\textbf{4.} \begin{minipage}[t]{0.55\textwidth}
Діагональ $BD$ паралелограма $ABCD$ перпендикулярна до сторони $AB$ (див. рисунок). $\angle A = 50°$, $BD = 10$ \textit{см}. До кожного початку речення (1--3) доберіть його закінчення (А--Д). \nmtyear{gen}
\end{minipage}
\hfill
\begin{minipage}[t]{0.4\textwidth}
    \vspace{-0.5cm}
    \begin{flushright}
    \begin{tikzpicture}[scale=0.8]
        \coordinate (A) at (0,0);
        \coordinate (B) at (1.3,2.2);
        \coordinate (C) at (5.5,2.2);
        \coordinate (D) at (4.2,0);

        \draw[thick] (A) -- (B) -- (C) -- (D) -- cycle;
        \draw[thick] (B) -- (D);

        \coordinate (BA) at ($(B)!0.25cm!(A)$);
        \coordinate (BD) at ($(B)!0.25cm!(D)$);
        \draw (BA) -- ($(BA)+(BD)-(B)$) -- (BD);

        \pic [draw, pic text={$50°$}, angle radius=0.5cm, angle eccentricity=1.7] {angle = D--A--B};

        \node[below left] at (A) {$A$};
        \node[above] at (B) {$B$};
        \node[above right] at (C) {$C$};
        \node[below right] at (D) {$D$};
    \end{tikzpicture}
    \end{flushright}
\end{minipage}

\vspace{0.3cm}

\matchingLayout{
    \textbf{1} \quad Сторона $AB$ \\
    \textbf{2} \quad Сторона $AD$ \\
    \textbf{3} \quad Діагональ $AC$
}{
    \begin{tabular}{ll}
    \textbf{А} & $\approx 6{,}4$ \textit{см} \\
    \textbf{Б} & $\approx 7{,}7$ \textit{см} \\
    \textbf{В} & $\approx 13{,}1$ \textit{см} \\
    \textbf{Г} & $\approx 15{,}6$ \textit{см} \\
    \textbf{Д} & $20$ \textit{см} \\
    \end{tabular}
}{
    \answerGrid
}

\vspace{0.7cm}

% === ЗАВДАННЯ 5 ===
\noindent\makebox[1.5em][l]{\textbf{5.}}\parbox[t]{\dimexpr\textwidth-1.5em}{У паралелограмі $ABCD$ діагональ $BD$ утворює зі сторонами $AB$ і $AD$ кути $50°$ і $55°$ відповідно (див. рисунок). Знайдіть довжину сторони $BC$, якщо $AB = 3$ \textit{см}. \nmtyear{gen}}

\vspace{0.3cm}
\begin{minipage}{0.42\textwidth}
\answerTableSmall{$\approx 2{,}5$ \textit{см}}{$\approx 3{,}2$ \textit{см}}{$3\sqrt{2}$ \textit{см}}{$\approx 3{,}5$ \textit{см}}{$\approx 2{,}8$ \textit{см}}
\end{minipage}
\hfill
\begin{minipage}{0.52\textwidth}
\begin{flushright}
\begin{tikzpicture}[scale=1.1]
    \coordinate (A) at (0,0);
    \coordinate (B) at (1.2,2);
    \coordinate (C) at (4.5,2);
    \coordinate (D) at (3.3,0);

    \draw[thick] (A) -- (B) -- (C) -- (D) -- cycle;
    \draw[thick] (B) -- (D);

    \pic [draw, pic text={\small $50°$}, angle radius=0.4cm, angle eccentricity=1.7] {angle = A--B--D};

    \pic [draw, angle radius=0.4cm] {angle = B--D--A};
    \pic [draw, angle radius=0.5cm, "\small $55°$" anchor=east] {angle = B--D--A};

    \node[left] at (0.6,1) {\small 3 \textit{см}};

    \node[below left] at (A) {$A$};
    \node[above] at (B) {$B$};
    \node[above right] at (C) {$C$};
    \node[below right] at (D) {$D$};
\end{tikzpicture}
\end{flushright}
\end{minipage}

\vspace{0.7cm}

% === ЗАВДАННЯ 6 ===
\noindent\makebox[1.5em][l]{\textbf{6.}}\parbox[t]{\dimexpr\textwidth-1.5em}{У паралелограмі $ABCD$ бісектриса кута $A = 50°$ перетинає сторону $BC$ в точці $K$, $BK = 6$ \textit{см}, $KC = 4$ \textit{см} (див. рисунок). Обчисліть периметр паралелограма $ABCD$. \nmtyear{gen}}

\vspace{0.3cm}
\begin{minipage}{0.42\textwidth}
\answerTableSmall{$32$ \textit{см}}{$36$ \textit{см}}{$44$ \textit{см}}{$38$ \textit{см}}{$40$ \textit{см}}
\end{minipage}
\hfill
\begin{minipage}{0.52\textwidth}
\begin{flushright}
\begin{tikzpicture}[scale=0.75]
    \coordinate (A) at (0,0);
    \coordinate (B) at (1.5,2.6);
    \coordinate (C) at (5.5,2.6);
    \coordinate (D) at (4,0);
    \coordinate (K) at (3.7,2.6);

    \draw[thick] (A) -- (B) -- (C) -- (D) -- cycle;
    \draw[thick] (A) -- (K);

    \pic [draw, angle radius=0.5cm] {angle = D--A--K};
    \pic [draw, angle radius=0.65cm] {angle = K--A--B};

    \node[below left] at (A) {$A$};
    \node[above left] at (B) {$B$};
    \node[above right] at (C) {$C$};
    \node[below right] at (D) {$D$};
    \node[above] at (K) {$K$};
\end{tikzpicture}
\end{flushright}
\end{minipage}

\vspace{0.7cm}

% === ЗАВДАННЯ 7 ===
\noindent\makebox[1.5em][l]{\textbf{7.}}\parbox[t]{\dimexpr\textwidth-1.5em}{У паралелограмі $ABCD$ діагоналі $AC$ і $BD$ перетинаються в точці $O$. $AC = 16$ \textit{см}, $BD = 12$ \textit{см}. Знайдіть $AO$. \nmtyear{gen}}

\vspace{0.3cm}
\begin{minipage}{0.42\textwidth}
\answerTableSmall{$8$ \textit{см}}{$6$ \textit{см}}{$4$ \textit{см}}{$10$ \textit{см}}{$14$ \textit{см}}
\end{minipage}
\hfill
\begin{minipage}{0.52\textwidth}
\begin{flushright}
\begin{tikzpicture}[scale=0.7]
    \coordinate (A) at (0,0);
    \coordinate (B) at (1.2,2.2);
    \coordinate (C) at (5,2.2);
    \coordinate (D) at (3.8,0);

    \draw[thick] (A) -- (B) -- (C) -- (D) -- cycle;

    \path[name path=AC] (A) -- (C);
    \path[name path=BD] (B) -- (D);
    \path [name intersections={of=AC and BD, by=O}];

    \draw[thick] (A) -- (C);
    \draw[thick] (B) -- (D);

    \node[below left] at (A) {$A$};
    \node[above left] at (B) {$B$};
    \node[above right] at (C) {$C$};
    \node[below right] at (D) {$D$};
    \node[above right] at (O) {$O$};
    \fill (O) circle (1.5pt);
\end{tikzpicture}
\end{flushright}
\end{minipage}

\vspace{0.7cm}

% === ЗАВДАННЯ 8 ===
\noindent\makebox[1.5em][l]{\textbf{8.}}\parbox[t]{\dimexpr\textwidth-1.5em}{У паралелограмі $ABCD$ на стороні $AD$ вибрано точку $K$. Діагональ $AC$ і відрізок $BK$ перетинаються в точці $O$ (див. рисунок). Визначте довжину сторони $BC$, якщо $AK = 10$ \textit{см}, $OK = 3$ \textit{см}, $OB = 5$ \textit{см}. \nmtyear{gen}}

\vspace{0.3cm}
\begin{minipage}{0.42\textwidth}
\answerTableSmall{$18$ \textit{см}}{$\frac{50}{3}$ \textit{см}}{$20$ \textit{см}}{$12$ \textit{см}}{$15$ \textit{см}}
\end{minipage}
\hfill
\begin{minipage}{0.52\textwidth}
\begin{flushright}
\begin{tikzpicture}[scale=0.7]
    \coordinate (A) at (0,0);
    \coordinate (B) at (1.2,2.2);
    \coordinate (C) at (5,2.2);
    \coordinate (D) at (3.8,0);
    \coordinate (K) at (2.5,0);

    \draw[thick] (A) -- (B) -- (C) -- (D) -- cycle;

    \path[name path=AC] (A) -- (C);
    \path[name path=BK] (B) -- (K);
    \path [name intersections={of=AC and BK, by=O}];

    \draw[thick] (A) -- (C);
    \draw[thick] (B) -- (K);

    \node[below left] at (A) {$A$};
    \node[above left] at (B) {$B$};
    \node[above right] at (C) {$C$};
    \node[below right] at (D) {$D$};
    \node[below] at (K) {$K$};
    \node[above right] at (O) {$O$};
    \fill (O) circle (1.5pt);
\end{tikzpicture}
\end{flushright}
\end{minipage}

\vspace{0.7cm}

% === ЗАВДАННЯ 9 ===
\noindent\makebox[1.5em][l]{\textbf{9.}}\parbox[t]{\dimexpr\textwidth-1.5em}{У паралелограмі $ABCD$ $AB = 8$ \textit{см}, $AD = 14$ \textit{см}. Знайдіть периметр паралелограма. \nmtyear{gen}}

\answerTable{$36$ \textit{см}}{$22$ \textit{см}}{$40$ \textit{см}}{$44$ \textit{см}}{$56$ \textit{см}}

\vspace{0.7cm}

% === ЗАВДАННЯ 10 ===
\noindent\textbf{10.} \begin{minipage}[t]{0.55\textwidth}
У паралелограмі $ABCD$ діагональ $BD$ утворює зі сторонами $BC$ і $CD$ кути $62°$ і $70°$ (див. рисунок). Визначте градусну міру кута $ABC$. \nmtyear{gen}
\end{minipage}
\hfill
\begin{minipage}[t]{0.4\textwidth}
    \vspace{-0.5cm}
    \begin{flushright}
    \begin{tikzpicture}[scale=0.9]
        \coordinate (A) at (0,0);
        \coordinate (D) at (4,0);
        \coordinate (C) at (5.2,2.5);
        \coordinate (B) at (1.2,2.5);

        \draw[thick] (A) -- (B) -- (C) -- (D) -- cycle;
        \draw[thick] (B) -- (D);

        \pic [draw, pic text={$62°$}, angle radius=0.6cm, angle eccentricity=1.7] {angle = D--B--C};
        \pic [draw, pic text={$70°$}, angle radius=0.6cm, angle eccentricity=1.4] {angle = C--D--B};
        \pic [draw, angle radius=0.5cm] {angle = C--D--B};

        \node[below left] at (A) {$A$};
        \node[above left] at (B) {$B$};
        \node[above right] at (C) {$C$};
        \node[below right] at (D) {$D$};
    \end{tikzpicture}
    \end{flushright}
\end{minipage}

\vspace{0.2cm}
\answerTable{$132°$}{$118°$}{$48°$}{$110°$}{$62°$}

\vspace{0.7cm}

% === ЗАВДАННЯ 11 ===
\noindent\makebox[1.5em][l]{\textbf{11.}}\parbox[t]{\dimexpr\textwidth-1.5em}{Які з наведених тверджень є правильними? \nmtyear{gen}}

\vspace{0.2cm}
\begin{tabular}{r@{\hspace{0.5em}}p{14cm}}
I. & Протилежні кути паралелограма рівні. \\
II. & Діагоналі паралелограма точкою перетину діляться навпіл. \\
III. & Діагоналі паралелограма перпендикулярні. \\
\end{tabular}

\answerTable{лише I}{лише III}{лише I та II}{I, II та III}{лише II}

\vspace{0.7cm}

% === ЗАВДАННЯ 12 ===
\noindent\textbf{12.} \begin{minipage}[t]{0.55\textwidth}
Діагоналі $AC$ і $BD$ паралелограма $ABCD$ перетинаються в точці $O$ (див. рисунок). З точки $O$ на сторону $AD$ опущено перпендикуляр $OK = 10$ \textit{см}, $AK = 18$ \textit{см}, $KD = 12$ \textit{см}. До кожного відрізка (1--3) доберіть його довжину (А--Д). \nmtyear{gen}
\end{minipage}
\hfill
\begin{minipage}[t]{0.4\textwidth}
    \vspace{-0.5cm}
    \begin{flushright}
    \begin{tikzpicture}[scale=0.5]
        \coordinate (A) at (0,0);
        \coordinate (K) at (4,0);
        \coordinate (D) at (6.7,0);
        \coordinate (O) at (4, 2.2);

        \coordinate (C) at ($(O)!-1!(A)$);
        \coordinate (B) at ($(O)!-1!(D)$);

        \draw[thick] (A) -- (B) -- (C) -- (D) -- cycle;
        \draw[thick] (A) -- (C);
        \draw[thick] (B) -- (D);
        \draw[thick] (O) -- (K);

        \draw (K) ++(-0.4,0) -- ++(0,0.4) -- ++(0.4,0);

        \node[below left] at (A) {$A$};
        \node[above left] at (B) {$B$};
        \node[above right] at (C) {$C$};
        \node[below right] at (D) {$D$};
        \node[above] at (O) {$O$};
        \node[below] at (K) {$K$};
    \end{tikzpicture}
    \end{flushright}
\end{minipage}

\vspace{0.3cm}

\matchingLayout{
    \textbf{1} \quad Висота паралелограма до $AD$ \\
    \textbf{2} \quad Проєкція $AB$ на $AD$ \\
    \textbf{3} \quad $AB$
}{
    \begin{tabular}{ll}
    \textbf{А} & 6 \textit{см} \\
    \textbf{Б} & 20 \textit{см} \\
    \textbf{В} & $\sqrt{136}$ \textit{см} \\
    \textbf{Г} & 26 \textit{см} \\
    \textbf{Д} & 30 \textit{см} \\
    \end{tabular}
}{
    \answerGrid
}

\vspace{0.7cm}

% === ЗАВДАННЯ 13 ===
\noindent\makebox[1.5em][l]{\textbf{13.}}\parbox[t]{\dimexpr\textwidth-1.5em}{Площа паралелограма $ABCD$ дорівнює $72$ см$^2$. Сторона $AB = 9$ \textit{см}. Знайдіть висоту, проведену до сторони $AB$. \nmtyear{gen}}

\answerTable{$12$ \textit{см}}{$9$ \textit{см}}{$8$ \textit{см}}{$6$ \textit{см}}{$4$ \textit{см}}

\vspace{0.7cm}

% === ЗАВДАННЯ 14 ===
\noindent\makebox[1.5em][l]{\textbf{14.}}\parbox[t]{\dimexpr\textwidth-1.5em}{Сума кутів паралелограма дорівнює: \nmtyear{gen}}

\answerTable{$540°$}{$270°$}{$180°$}{$360°$}{$720°$}

\vspace{0.7cm}

% === ЗАВДАННЯ 15 ===
\noindent\makebox[1.5em][l]{\textbf{15.}}\parbox[t]{\dimexpr\textwidth-1.5em}{Діагоналі паралелограма $ABCD$ дорівнюють $10$ см і $14$ см. Точка перетину діагоналей ділить кожну з них на відрізки довжиною: \nmtyear{gen}}

\answerTable{$6$ і $8$ \textit{см}}{$10$ і $14$ \textit{см}}{$4$ і $6$ \textit{см}}{$2{,}5$ і $3{,}5$ \textit{см}}{$5$ і $7$ \textit{см}}

\vspace{1cm}

%======================================================================
% БЛОК 2: Прямокутник (15 завдань)
%======================================================================

\section*{Блок 2: Прямокутник}

% === ЗАВДАННЯ 16 ===
\noindent\makebox[1.5em][l]{\textbf{16.}}\parbox[t]{\dimexpr\textwidth-1.5em}{Сторони прямокутника дорівнюють $5$ \textit{см} і $12$ \textit{см}. Знайдіть діагональ прямокутника. \nmtyear{gen}}

\vspace{0.3cm}
\begin{minipage}{0.42\textwidth}
\answerTableSmall{$13$ \textit{см}}{$17$ \textit{см}}{$15$ \textit{см}}{$8{,}5$ \textit{см}}{$11$ \textit{см}}
\end{minipage}
\hfill
\begin{minipage}{0.52\textwidth}
\begin{flushright}
\begin{tikzpicture}[scale=0.4]
    \coordinate (A) at (0,0);
    \coordinate (B) at (0,2.5);
    \coordinate (C) at (6,2.5);
    \coordinate (D) at (6,0);

    \draw[thick] (A) -- (B) -- (C) -- (D) -- cycle;
    \draw[thick, dashed] (A) -- (C);

    \draw (A) rectangle ++(0.4,0.4);

    \node[left] at (0,1.25) {\small 5};
    \node[below] at (3,0) {\small 12};

    \node[below left] at (A) {$A$};
    \node[above left] at (B) {$B$};
    \node[above right] at (C) {$C$};
    \node[below right] at (D) {$D$};
\end{tikzpicture}
\end{flushright}
\end{minipage}

\vspace{0.7cm}

% === ЗАВДАННЯ 17 ===
\noindent\makebox[1.5em][l]{\textbf{17.}}\parbox[t]{\dimexpr\textwidth-1.5em}{Діагональ прямокутника дорівнює $10$ \textit{см}, одна зі сторін --- $6$ \textit{см}. Знайдіть другу сторону. \nmtyear{gen}}

\answerTable{$8$ \textit{см}}{$\sqrt{136}$ \textit{см}}{$12$ \textit{см}}{$7$ \textit{см}}{$4$ \textit{см}}

\vspace{0.7cm}

% === ЗАВДАННЯ 18 ===
\noindent\makebox[1.5em][l]{\textbf{18.}}\parbox[t]{\dimexpr\textwidth-1.5em}{Периметр прямокутника дорівнює $34$ \textit{см}, одна зі сторін --- $8$ \textit{см}. Знайдіть площу прямокутника. \nmtyear{gen}}

\answerTable{$64$ см$^2$}{$56$ см$^2$}{$136$ см$^2$}{$81$ см$^2$}{$72$ см$^2$}

\vspace{0.7cm}

% === ЗАВДАННЯ 19 ===
\noindent\makebox[1.5em][l]{\textbf{19.}}\parbox[t]{\dimexpr\textwidth-1.5em}{Площа прямокутника дорівнює $48$ см$^2$, одна зі сторін --- $6$ \textit{см}. Знайдіть периметр прямокутника. \nmtyear{gen}}

\answerTable{$32$ \textit{см}}{$22$ \textit{см}}{$20$ \textit{см}}{$54$ \textit{см}}{$28$ \textit{см}}

\vspace{0.7cm}

% === ЗАВДАННЯ 20 ===
\noindent\makebox[1.5em][l]{\textbf{20.}}\parbox[t]{\dimexpr\textwidth-1.5em}{Які з наведених тверджень є правильними? \nmtyear{gen}}

\vspace{0.2cm}
\begin{tabular}{r@{\hspace{0.5em}}p{14cm}}
I. & Прямокутник є паралелограмом. \\
II. & Діагоналі прямокутника рівні. \\
III. & Діагоналі прямокутника перпендикулярні. \\
\end{tabular}

\answerTable{I, II та III}{лише I та III}{лише I}{лише II}{лише I та II}

\vspace{0.7cm}

% === ЗАВДАННЯ 21 ===
\noindent\makebox[1.5em][l]{\textbf{21.}}\parbox[t]{\dimexpr\textwidth-1.5em}{У прямокутнику $ABCD$ діагоналі перетинаються в точці $O$. $\angle AOB = 60°$. Знайдіть $\angle OAB$. \nmtyear{gen}}

\vspace{0.3cm}
\begin{minipage}{0.42\textwidth}
\answerTableSmall{$45°$}{$60°$}{$90°$}{$120°$}{$30°$}
\end{minipage}
\hfill
\begin{minipage}{0.52\textwidth}
\begin{flushright}
\begin{tikzpicture}[scale=0.6]
    \coordinate (A) at (0,0);
    \coordinate (B) at (0,2);
    \coordinate (C) at (4,2);
    \coordinate (D) at (4,0);
    \coordinate (O) at (2,1);

    \draw[thick] (A) -- (B) -- (C) -- (D) -- cycle;
    \draw[thick] (A) -- (C);
    \draw[thick] (B) -- (D);

    \pic [draw, pic text={$60°$}, angle radius=0.5cm, angle eccentricity=1.7] {angle = B--O--A};
    \pic [draw, angle radius=0.4cm, "?" anchor=south] {angle = O--A--B};

    \draw (A) rectangle ++(0.3,0.3);

    \node[below left] at (A) {$A$};
    \node[above left] at (B) {$B$};
    \node[above right] at (C) {$C$};
    \node[below right] at (D) {$D$};
    \node[above] at (O) {$O$};
\end{tikzpicture}
\end{flushright}
\end{minipage}

\vspace{0.7cm}

% === ЗАВДАННЯ 22 ===
\noindent\makebox[1.5em][l]{\textbf{22.}}\parbox[t]{\dimexpr\textwidth-1.5em}{Діагональ прямокутника дорівнює $d$. Знайдіть відстань від точки перетину діагоналей до сторони прямокутника, якщо ця відстань дорівнює половині меншої сторони. \nmtyear{gen}}

\answerTable{$d$}{$\dfrac{d}{4}$}{$\dfrac{d}{2}$}{$\dfrac{d\sqrt{3}}{2}$}{$\dfrac{d\sqrt{2}}{2}$}

\vspace{0.7cm}

% === ЗАВДАННЯ 23 ===
\noindent\makebox[1.5em][l]{\textbf{23.}}\parbox[t]{\dimexpr\textwidth-1.5em}{У прямокутнику $ABCD$ $AB = 6$ \textit{см}, $BC = 8$ \textit{см}. Радіус описаного кола дорівнює: \nmtyear{gen}}

\answerTable{$10$ \textit{см}}{$5$ \textit{см}}{$4$ \textit{см}}{$7$ \textit{см}}{$3$ \textit{см}}

\vspace{0.7cm}

% === ЗАВДАННЯ 24 ===
\noindent\makebox[1.5em][l]{\textbf{24.}}\parbox[t]{\dimexpr\textwidth-1.5em}{Сторони прямокутника відносяться як $3:4$, а діагональ дорівнює $15$ \textit{см}. Знайдіть площу прямокутника. \nmtyear{gen}}

\answerTable{$108$ см$^2$}{$180$ см$^2$}{$96$ см$^2$}{$144$ см$^2$}{$81$ см$^2$}

\vspace{0.7cm}

% === ЗАВДАННЯ 25 ===
\noindent\makebox[1.5em][l]{\textbf{25.}}\parbox[t]{\dimexpr\textwidth-1.5em}{У прямокутнику $ABCD$ діагональ $AC$ утворює зі стороною $AB$ кут $30°$. $AC = 12$ \textit{см}. Знайдіть сторону $BC$. \nmtyear{gen}}

\vspace{0.3cm}
\begin{minipage}{0.42\textwidth}
\answerTableSmall{$3$ \textit{см}}{$8$ \textit{см}}{$4\sqrt{3}$ \textit{см}}{$6$ \textit{см}}{$6\sqrt{3}$ \textit{см}}
\end{minipage}
\hfill
\begin{minipage}{0.52\textwidth}
\begin{flushright}
\begin{tikzpicture}[scale=0.5]
    \coordinate (A) at (0,0);
    \coordinate (B) at (0,3);
    \coordinate (C) at (5.2,3);
    \coordinate (D) at (5.2,0);

    \draw[thick] (A) -- (B) -- (C) -- (D) -- cycle;
    \draw[thick] (A) -- (C);

    \pic [draw, pic text={$30°$}, angle radius=0.7cm, angle eccentricity=1.5] {angle = D--A--C};

    \draw (A) rectangle ++(0.4,0.4);

    \node[below left] at (A) {$A$};
    \node[above left] at (B) {$B$};
    \node[above right] at (C) {$C$};
    \node[below right] at (D) {$D$};
\end{tikzpicture}
\end{flushright}
\end{minipage}

\vspace{0.7cm}

% === ЗАВДАННЯ 26-30 ===
\noindent\makebox[1.5em][l]{\textbf{26.}}\parbox[t]{\dimexpr\textwidth-1.5em}{Діагоналі прямокутника $ABCD$ перетинаються в точці $O$. $AO = 7$ \textit{см}. Знайдіть діагональ $AC$. \nmtyear{gen}}

\answerTable{$28$ \textit{см}}{$14$ \textit{см}}{$7$ \textit{см}}{$21$ \textit{см}}{$3{,}5$ \textit{см}}

\vspace{0.7cm}

\noindent\makebox[1.5em][l]{\textbf{27.}}\parbox[t]{\dimexpr\textwidth-1.5em}{Сторони прямокутника дорівнюють $a$ і $b$. Чому дорівнює сума квадратів діагоналей? \nmtyear{gen}}

\answerTable{$a^2+b^2$}{$4(a^2+b^2)$}{$(a+b)^2$}{$2(a^2+b^2)$}{$2ab$}

\vspace{0.7cm}

\noindent\makebox[1.5em][l]{\textbf{28.}}\parbox[t]{\dimexpr\textwidth-1.5em}{Периметр прямокутника дорівнює $40$ \textit{см}, а його площа --- $96$ см$^2$. Знайдіть діагональ прямокутника. \nmtyear{gen}}

\answerTable{$\sqrt{208}$ \textit{см}}{$14$ \textit{см}}{$10$ \textit{см}}{$16$ \textit{см}}{$12$ \textit{см}}

\vspace{0.7cm}

\noindent\makebox[1.5em][l]{\textbf{29.}}\parbox[t]{\dimexpr\textwidth-1.5em}{У прямокутнику одна сторона вдвічі більша за іншу, а діагональ дорівнює $10$ \textit{см}. Знайдіть меншу сторону. \nmtyear{gen}}

\answerTable{$4\sqrt{5}$ \textit{см}}{$4$ \textit{см}}{$\sqrt{20}$ \textit{см}}{$5$ \textit{см}}{$2\sqrt{5}$ \textit{см}}

\vspace{0.7cm}

\noindent\makebox[1.5em][l]{\textbf{30.}}\parbox[t]{\dimexpr\textwidth-1.5em}{Площа прямокутника дорівнює $60$ см$^2$. Одна зі сторін більша за іншу на $7$ \textit{см}. Знайдіть меншу сторону. \nmtyear{gen}}

\answerTable{$10$ \textit{см}}{$12$ \textit{см}}{$4$ \textit{см}}{$5$ \textit{см}}{$6$ \textit{см}}

\vspace{1cm}

%======================================================================
% БЛОК 3: Ромб (15 завдань)
%======================================================================

\section*{Блок 3: Ромб}

% === ЗАВДАННЯ 31 ===
\noindent\makebox[1.5em][l]{\textbf{31.}}\parbox[t]{\dimexpr\textwidth-1.5em}{Діагоналі ромба дорівнюють $6$ \textit{см} і $8$ \textit{см}. Знайдіть сторону ромба. \nmtyear{gen}}

\vspace{0.3cm}
\begin{minipage}{0.42\textwidth}
\answerTableSmall{$\sqrt{100}$ \textit{см}}{$10$ \textit{см}}{$5$ \textit{см}}{$4$ \textit{см}}{$7$ \textit{см}}
\end{minipage}
\hfill
\begin{minipage}{0.52\textwidth}
\begin{flushright}
\begin{tikzpicture}[scale=0.5]
    \coordinate (A) at (0,0);
    \coordinate (B) at (2,3);
    \coordinate (C) at (6,0);
    \coordinate (D) at (4,-3);
    \coordinate (O) at (3,0);

    \draw[thick] (A) -- (B) -- (C) -- (D) -- cycle;
    \draw[thick] (A) -- (C);
    \draw[thick] (B) -- (D);

    \draw (O) ++(-0.3,0) -- ++(0,0.3) -- ++(0.3,0);

    \node[left] at (A) {$A$};
    \node[above] at (B) {$B$};
    \node[right] at (C) {$C$};
    \node[below] at (D) {$D$};
\end{tikzpicture}
\end{flushright}
\end{minipage}

\vspace{0.7cm}

% === ЗАВДАННЯ 32 ===
\noindent\makebox[1.5em][l]{\textbf{32.}}\parbox[t]{\dimexpr\textwidth-1.5em}{Сторона ромба дорівнює $9$ \textit{см}. Знайдіть периметр ромба. \nmtyear{gen}}

\answerTable{$36$ \textit{см}}{$18$ \textit{см}}{$45$ \textit{см}}{$27$ \textit{см}}{$81$ \textit{см}}

\vspace{0.7cm}

% === ЗАВДАННЯ 33 ===
\noindent\makebox[1.5em][l]{\textbf{33.}}\parbox[t]{\dimexpr\textwidth-1.5em}{Які з наведених тверджень є правильними? \nmtyear{gen}}

\vspace{0.2cm}
\begin{tabular}{r@{\hspace{0.5em}}p{14cm}}
I. & Діагоналі ромба перпендикулярні. \\
II. & Діагоналі ромба є бісектрисами його кутів. \\
III. & Діагоналі ромба рівні. \\
\end{tabular}

\answerTable{лише II та III}{лише I}{лише III}{лише I та II}{I, II та III}

\vspace{0.7cm}

% === ЗАВДАННЯ 34 ===
\noindent\makebox[1.5em][l]{\textbf{34.}}\parbox[t]{\dimexpr\textwidth-1.5em}{Діагоналі ромба дорівнюють $10$ \textit{см} і $24$ \textit{см}. Знайдіть площу ромба. \nmtyear{gen}}

\answerTable{$240$ см$^2$}{$60$ см$^2$}{$34$ см$^2$}{$340$ см$^2$}{$120$ см$^2$}

\vspace{0.7cm}

% === ЗАВДАННЯ 35 ===
\noindent\makebox[1.5em][l]{\textbf{35.}}\parbox[t]{\dimexpr\textwidth-1.5em}{Площа ромба дорівнює $48$ см$^2$, одна з діагоналей --- $8$ \textit{см}. Знайдіть другу діагональ. \nmtyear{gen}}

\answerTable{$12$ \textit{см}}{$6$ \textit{см}}{$40$ \textit{см}}{$24$ \textit{см}}{$16$ \textit{см}}

\vspace{0.7cm}

% === ЗАВДАННЯ 36 ===
\noindent\makebox[1.5em][l]{\textbf{36.}}\parbox[t]{\dimexpr\textwidth-1.5em}{Один із кутів ромба дорівнює $60°$. Знайдіть інший кут, що не є сусіднім до даного. \nmtyear{gen}}

\answerTable{$90°$}{$150°$}{$120°$}{$60°$}{$30°$}

\vspace{0.7cm}

% === ЗАВДАННЯ 37 ===
\noindent\makebox[1.5em][l]{\textbf{37.}}\parbox[t]{\dimexpr\textwidth-1.5em}{Сторона ромба дорівнює $10$ \textit{см}, а один із кутів --- $60°$. Знайдіть меншу діагональ. \nmtyear{gen}}

\vspace{0.3cm}
\begin{minipage}{0.42\textwidth}
\answerTableSmall{$10\sqrt{3}$ \textit{см}}{$20$ \textit{см}}{$5$ \textit{см}}{$5\sqrt{3}$ \textit{см}}{$10$ \textit{см}}
\end{minipage}
\hfill
\begin{minipage}{0.52\textwidth}
\begin{flushright}
\begin{tikzpicture}[scale=0.6]
    \coordinate (A) at (0,0);
    \coordinate (B) at (1.5,2.6);
    \coordinate (C) at (4.5,2.6);
    \coordinate (D) at (3,0);

    \draw[thick] (A) -- (B) -- (C) -- (D) -- cycle;
    \draw[thick] (A) -- (C);
    \draw[thick] (B) -- (D);

    \pic [draw, pic text={$60°$}, angle radius=0.5cm, angle eccentricity=1.7] {angle = D--A--B};

    \node[below left] at (A) {$A$};
    \node[above left] at (B) {$B$};
    \node[above right] at (C) {$C$};
    \node[below right] at (D) {$D$};
\end{tikzpicture}
\end{flushright}
\end{minipage}

\vspace{0.7cm}

% === ЗАВДАННЯ 38 ===
\noindent\makebox[1.5em][l]{\textbf{38.}}\parbox[t]{\dimexpr\textwidth-1.5em}{Периметр ромба дорівнює $40$ \textit{см}, а одна з діагоналей --- $16$ \textit{см}. Знайдіть другу діагональ. \nmtyear{gen}}

\answerTable{$24$ \textit{см}}{$12$ \textit{см}}{$8$ \textit{см}}{$20$ \textit{см}}{$6$ \textit{см}}

\vspace{0.7cm}

% === ЗАВДАННЯ 39 ===
\noindent\makebox[1.5em][l]{\textbf{39.}}\parbox[t]{\dimexpr\textwidth-1.5em}{Діагональ ромба дорівнює його стороні. Знайдіть більший кут ромба. \nmtyear{gen}}

\answerTable{$135°$}{$150°$}{$120°$}{$90°$}{$60°$}

\vspace{0.7cm}

% === ЗАВДАННЯ 40 ===
\noindent\makebox[1.5em][l]{\textbf{40.}}\parbox[t]{\dimexpr\textwidth-1.5em}{Які з наведених тверджень є правильними? \nmtyear{gen}}

\vspace{0.2cm}
\begin{tabular}{r@{\hspace{0.5em}}p{14cm}}
I. & Ромб є паралелограмом. \\
II. & Діагоналі ромба точкою перетину діляться навпіл. \\
III. & Усі сторони ромба рівні. \\
\end{tabular}

\answerTable{лише II та III}{лише I та II}{лише I та III}{I, II та III}{лише I}

\vspace{0.7cm}

% === ЗАВДАННЯ 41-45 ===
\noindent\makebox[1.5em][l]{\textbf{41.}}\parbox[t]{\dimexpr\textwidth-1.5em}{Сторона ромба дорівнює $13$ \textit{см}, а одна з діагоналей --- $24$ \textit{см}. Знайдіть другу діагональ. \nmtyear{gen}}

\answerTable{$20$ \textit{см}}{$12$ \textit{см}}{$26$ \textit{см}}{$10$ \textit{см}}{$5$ \textit{см}}

\vspace{0.7cm}

\noindent\makebox[1.5em][l]{\textbf{42.}}\parbox[t]{\dimexpr\textwidth-1.5em}{Площа ромба дорівнює $96$ см$^2$. Діагоналі відносяться як $3:4$. Знайдіть меншу діагональ. \nmtyear{gen}}

\answerTable{$8$ \textit{см}}{$6$ \textit{см}}{$12$ \textit{см}}{$24$ \textit{см}}{$16$ \textit{см}}

\vspace{0.7cm}

\noindent\makebox[1.5em][l]{\textbf{43.}}\parbox[t]{\dimexpr\textwidth-1.5em}{Радіус вписаного в ромб кола дорівнює $4$ \textit{см}, а сторона ромба --- $10$ \textit{см}. Знайдіть площу ромба. \nmtyear{gen}}

\answerTable{$160$ см$^2$}{$80$ см$^2$}{$40$ см$^2$}{$64$ см$^2$}{$100$ см$^2$}

\vspace{0.7cm}

\noindent\makebox[1.5em][l]{\textbf{44.}}\parbox[t]{\dimexpr\textwidth-1.5em}{Гострий кут ромба дорівнює $50°$. Знайдіть тупий кут ромба. \nmtyear{gen}}

\answerTable{$50°$}{$130°$}{$140°$}{$100°$}{$80°$}

\vspace{0.7cm}

\noindent\makebox[1.5em][l]{\textbf{45.}}\parbox[t]{\dimexpr\textwidth-1.5em}{Сторона ромба $a$, гострий кут $\alpha$. Знайдіть площу ромба. \nmtyear{gen}}

\answerTable{$a^2$}{$a^2\cos\alpha$}{$\dfrac{a^2\sin\alpha}{2}$}{$a^2\tg\alpha$}{$a^2\sin\alpha$}

\vspace{1cm}

%======================================================================
% БЛОК 4: Квадрат (15 завдань)
%======================================================================

\section*{Блок 4: Квадрат}

% === ЗАВДАННЯ 46 ===
\noindent\makebox[1.5em][l]{\textbf{46.}}\parbox[t]{\dimexpr\textwidth-1.5em}{Сторона квадрата дорівнює $7$ \textit{см}. Знайдіть периметр квадрата. \nmtyear{gen}}

\answerTable{$28$ \textit{см}}{$56$ \textit{см}}{$21$ \textit{см}}{$49$ \textit{см}}{$14$ \textit{см}}

\vspace{0.7cm}

% === ЗАВДАННЯ 47 ===
\noindent\makebox[1.5em][l]{\textbf{47.}}\parbox[t]{\dimexpr\textwidth-1.5em}{Периметр квадрата дорівнює $36$ \textit{см}. Знайдіть сторону квадрата. \nmtyear{gen}}

\answerTable{$4$ \textit{см}}{$6$ \textit{см}}{$9$ \textit{см}}{$12$ \textit{см}}{$18$ \textit{см}}

\vspace{0.7cm}

% === ЗАВДАННЯ 48 ===
\noindent\makebox[1.5em][l]{\textbf{48.}}\parbox[t]{\dimexpr\textwidth-1.5em}{Площа квадрата дорівнює $64$ см$^2$. Знайдіть сторону квадрата. \nmtyear{gen}}

\answerTable{$4$ \textit{см}}{$16$ \textit{см}}{$8$ \textit{см}}{$32$ \textit{см}}{$64$ \textit{см}}

\vspace{0.7cm}

% === ЗАВДАННЯ 49 ===
\noindent\makebox[1.5em][l]{\textbf{49.}}\parbox[t]{\dimexpr\textwidth-1.5em}{Сторона квадрата дорівнює $5$ \textit{см}. Знайдіть діагональ квадрата. \nmtyear{gen}}

\vspace{0.3cm}
\begin{minipage}{0.42\textwidth}
\answerTableSmall{$25$ \textit{см}}{$10$ \textit{см}}{$5\sqrt{3}$ \textit{см}}{$2{,}5\sqrt{2}$ \textit{см}}{$5\sqrt{2}$ \textit{см}}
\end{minipage}
\hfill
\begin{minipage}{0.52\textwidth}
\begin{flushright}
\begin{tikzpicture}[scale=0.5]
    \coordinate (A) at (0,0);
    \coordinate (B) at (0,3);
    \coordinate (C) at (3,3);
    \coordinate (D) at (3,0);

    \draw[thick] (A) -- (B) -- (C) -- (D) -- cycle;
    \draw[thick, dashed] (A) -- (C);

    \draw (A) rectangle ++(0.3,0.3);

    \node[left] at (0,1.5) {\small 5};

    \node[below left] at (A) {$A$};
    \node[above left] at (B) {$B$};
    \node[above right] at (C) {$C$};
    \node[below right] at (D) {$D$};
\end{tikzpicture}
\end{flushright}
\end{minipage}

\vspace{0.7cm}

% === ЗАВДАННЯ 50 ===
\noindent\makebox[1.5em][l]{\textbf{50.}}\parbox[t]{\dimexpr\textwidth-1.5em}{Діагональ квадрата дорівнює $8\sqrt{2}$ \textit{см}. Знайдіть сторону квадрата. \nmtyear{gen}}

\answerTable{$4$ \textit{см}}{$8$ \textit{см}}{$16$ \textit{см}}{$8\sqrt{2}$ \textit{см}}{$4\sqrt{2}$ \textit{см}}

\vspace{0.7cm}

% === ЗАВДАННЯ 51 ===
\noindent\makebox[1.5em][l]{\textbf{51.}}\parbox[t]{\dimexpr\textwidth-1.5em}{Діагональ квадрата дорівнює $12$ \textit{см}. Знайдіть площу квадрата. \nmtyear{gen}}

\answerTable{$72$ см$^2$}{$144$ см$^2$}{$24$ см$^2$}{$48$ см$^2$}{$36$ см$^2$}

\vspace{0.7cm}

% === ЗАВДАННЯ 52 ===
\noindent\makebox[1.5em][l]{\textbf{52.}}\parbox[t]{\dimexpr\textwidth-1.5em}{Які з наведених тверджень є правильними? \nmtyear{gen}}

\vspace{0.2cm}
\begin{tabular}{r@{\hspace{0.5em}}p{14cm}}
I. & Квадрат є прямокутником. \\
II. & Квадрат є ромбом. \\
III. & Діагоналі квадрата перпендикулярні та рівні. \\
\end{tabular}

\answerTable{лише II та III}{лише I}{лише III}{лише I та II}{I, II та III}

\vspace{0.7cm}

% === ЗАВДАННЯ 53 ===
\noindent\makebox[1.5em][l]{\textbf{53.}}\parbox[t]{\dimexpr\textwidth-1.5em}{Радіус вписаного в квадрат кола дорівнює $3$ \textit{см}. Знайдіть сторону квадрата. \nmtyear{gen}}

\answerTable{$6\sqrt{2}$ \textit{см}}{$6$ \textit{см}}{$3\sqrt{2}$ \textit{см}}{$3$ \textit{см}}{$9$ \textit{см}}

\vspace{0.7cm}

% === ЗАВДАННЯ 54 ===
\noindent\makebox[1.5em][l]{\textbf{54.}}\parbox[t]{\dimexpr\textwidth-1.5em}{Радіус описаного навколо квадрата кола дорівнює $5$ \textit{см}. Знайдіть сторону квадрата. \nmtyear{gen}}

\answerTable{$10\sqrt{2}$ \textit{см}}{$2{,}5\sqrt{2}$ \textit{см}}{$10$ \textit{см}}{$5$ \textit{см}}{$5\sqrt{2}$ \textit{см}}

\vspace{0.7cm}

% === ЗАВДАННЯ 55-60 ===
\noindent\makebox[1.5em][l]{\textbf{55.}}\parbox[t]{\dimexpr\textwidth-1.5em}{Площа квадрата дорівнює $50$ см$^2$. Знайдіть діагональ квадрата. \nmtyear{gen}}

\answerTable{$\sqrt{50}$ \textit{см}}{$5\sqrt{2}$ \textit{см}}{$25$ \textit{см}}{$10$ \textit{см}}{$100$ \textit{см}}

\vspace{0.7cm}

\noindent\makebox[1.5em][l]{\textbf{56.}}\parbox[t]{\dimexpr\textwidth-1.5em}{Периметр квадрата дорівнює $24$ \textit{см}. Знайдіть площу квадрата. \nmtyear{gen}}

\answerTable{$72$ см$^2$}{$48$ см$^2$}{$144$ см$^2$}{$36$ см$^2$}{$24$ см$^2$}

\vspace{0.7cm}

\noindent\makebox[1.5em][l]{\textbf{57.}}\parbox[t]{\dimexpr\textwidth-1.5em}{Сторона квадрата збільшилася в $3$ рази. У скільки разів збільшилася площа квадрата? \nmtyear{gen}}

\answerTable{у $6$ разів}{у $12$ разів}{у $9$ разів}{у $3$ рази}{у $27$ разів}

\vspace{0.7cm}

\noindent\makebox[1.5em][l]{\textbf{58.}}\parbox[t]{\dimexpr\textwidth-1.5em}{Діагональ квадрата збільшилася в $2$ рази. У скільки разів збільшилася площа квадрата? \nmtyear{gen}}

\answerTable{у $2$ рази}{у $\sqrt{2}$ рази}{у $8$ разів}{у $2\sqrt{2}$ рази}{у $4$ рази}

\vspace{0.7cm}

\noindent\makebox[1.5em][l]{\textbf{59.}}\parbox[t]{\dimexpr\textwidth-1.5em}{Відношення діагоналі квадрата до його сторони дорівнює: \nmtyear{gen}}

\answerTable{$1$}{$\sqrt{2}$}{$\sqrt{3}$}{$\dfrac{1}{\sqrt{2}}$}{$2$}

\vspace{0.7cm}

\noindent\makebox[1.5em][l]{\textbf{60.}}\parbox[t]{\dimexpr\textwidth-1.5em}{Сума діагоналей квадрата дорівнює $20\sqrt{2}$ \textit{см}. Знайдіть сторону квадрата. \nmtyear{gen}}

\answerTable{$5$ \textit{см}}{$5\sqrt{2}$ \textit{см}}{$10$ \textit{см}}{$10\sqrt{2}$ \textit{см}}{$20$ \textit{см}}

\vspace{1cm}

%======================================================================
% БЛОК 5: Трапеція (15 завдань)
%======================================================================

\section*{Блок 5: Трапеція}

% === ЗАВДАННЯ 61 ===
\noindent\makebox[1.5em][l]{\textbf{61.}}\parbox[t]{\dimexpr\textwidth-1.5em}{Основи трапеції дорівнюють $8$ \textit{см} і $14$ \textit{см}. Знайдіть середню лінію трапеції. \nmtyear{gen}}

\vspace{0.3cm}
\begin{minipage}{0.42\textwidth}
\answerTableSmall{$11$ \textit{см}}{$10$ \textit{см}}{$6$ \textit{см}}{$22$ \textit{см}}{$12$ \textit{см}}
\end{minipage}
\hfill
\begin{minipage}{0.52\textwidth}
\begin{flushright}
\begin{tikzpicture}[scale=0.5]
    \coordinate (A) at (0,0);
    \coordinate (B) at (1,2);
    \coordinate (C) at (5,2);
    \coordinate (D) at (7,0);

    \draw[thick] (A) -- (B) -- (C) -- (D) -- cycle;

    \coordinate (M) at (0.5,1);
    \coordinate (N) at (6,1);
    \draw[thick, dashed] (M) -- (N);

    \node[above] at (3,2) {\small 8};
    \node[below] at (3.5,0) {\small 14};

    \node[below left] at (A) {$A$};
    \node[above left] at (B) {$B$};
    \node[above right] at (C) {$C$};
    \node[below right] at (D) {$D$};
\end{tikzpicture}
\end{flushright}
\end{minipage}

\vspace{0.7cm}

% === ЗАВДАННЯ 62 ===
\noindent\makebox[1.5em][l]{\textbf{62.}}\parbox[t]{\dimexpr\textwidth-1.5em}{Середня лінія трапеції дорівнює $15$ \textit{см}, одна з основ --- $10$ \textit{см}. Знайдіть другу основу. \nmtyear{gen}}

\answerTable{$20$ \textit{см}}{$12{,}5$ \textit{см}}{$30$ \textit{см}}{$5$ \textit{см}}{$25$ \textit{см}}

\vspace{0.7cm}

% === ЗАВДАННЯ 63 ===
\noindent\makebox[1.5em][l]{\textbf{63.}}\parbox[t]{\dimexpr\textwidth-1.5em}{Основи трапеції дорівнюють $12$ \textit{см} і $18$ \textit{см}, висота --- $7$ \textit{см}. Знайдіть площу трапеції. \nmtyear{gen}}

\answerTable{$210$ см$^2$}{$126$ см$^2$}{$105$ см$^2$}{$63$ см$^2$}{$84$ см$^2$}

\vspace{0.7cm}

% === ЗАВДАННЯ 64 ===
\noindent\makebox[1.5em][l]{\textbf{64.}}\parbox[t]{\dimexpr\textwidth-1.5em}{Площа трапеції дорівнює $80$ см$^2$, середня лінія --- $10$ \textit{см}. Знайдіть висоту трапеції. \nmtyear{gen}}

\answerTable{$8$ \textit{см}}{$5$ \textit{см}}{$4$ \textit{см}}{$16$ \textit{см}}{$10$ \textit{см}}

\vspace{0.7cm}

% === ЗАВДАННЯ 65 ===
\noindent\makebox[1.5em][l]{\textbf{65.}}\parbox[t]{\dimexpr\textwidth-1.5em}{У рівнобічній трапеції основи дорівнюють $6$ \textit{см} і $12$ \textit{см}, бічна сторона --- $5$ \textit{см}. Знайдіть висоту трапеції. \nmtyear{gen}}

\vspace{0.3cm}
\begin{minipage}{0.42\textwidth}
\answerTableSmall{$\sqrt{34}$ \textit{см}}{$6$ \textit{см}}{$3$ \textit{см}}{$5$ \textit{см}}{$4$ \textit{см}}
\end{minipage}
\hfill
\begin{minipage}{0.52\textwidth}
\begin{flushright}
\begin{tikzpicture}[scale=0.45]
    \coordinate (A) at (0,0);
    \coordinate (B) at (1.5,2);
    \coordinate (C) at (4.5,2);
    \coordinate (D) at (6,0);

    \draw[thick] (A) -- (B) -- (C) -- (D) -- cycle;

    % Позначки рівності
    \draw ($(A)!0.5!(B)$) ++(-0.1,0.1) -- ++(0.2,-0.2);
    \draw ($(C)!0.5!(D)$) ++(-0.1,-0.1) -- ++(0.2,0.2);

    \node[above] at (3,2) {\small 6};
    \node[below] at (3,0) {\small 12};
    \node[left] at (0.75,1) {\small 5};

    \node[below left] at (A) {$A$};
    \node[above left] at (B) {$B$};
    \node[above right] at (C) {$C$};
    \node[below right] at (D) {$D$};
\end{tikzpicture}
\end{flushright}
\end{minipage}

\vspace{0.7cm}

% === ЗАВДАННЯ 66 ===
\noindent\makebox[1.5em][l]{\textbf{66.}}\parbox[t]{\dimexpr\textwidth-1.5em}{Які з наведених тверджень є правильними? \nmtyear{gen}}

\vspace{0.2cm}
\begin{tabular}{r@{\hspace{0.5em}}p{14cm}}
I. & Трапеція --- чотирикутник, у якого тільки дві сторони паралельні. \\
II. & Діагоналі рівнобічної трапеції рівні. \\
III. & Кути при основі рівнобічної трапеції рівні. \\
\end{tabular}

\answerTable{лише I та II}{I, II та III}{лише I та III}{лише I}{лише II та III}

\vspace{0.7cm}

% === ЗАВДАННЯ 67 ===
\noindent\makebox[1.5em][l]{\textbf{67.}}\parbox[t]{\dimexpr\textwidth-1.5em}{У трапеції $ABCD$ ($BC \parallel AD$) $\angle A = 65°$. Знайдіть $\angle B$. \nmtyear{gen}}

\answerTable{$115°$}{$130°$}{$65°$}{$90°$}{$180°$}

\vspace{0.7cm}

% === ЗАВДАННЯ 68 ===
\noindent\makebox[1.5em][l]{\textbf{68.}}\parbox[t]{\dimexpr\textwidth-1.5em}{У прямокутній трапеції одна з основ дорівнює $8$ \textit{см}, бічна сторона, перпендикулярна до основ --- $6$ \textit{см}, а діагональ --- $10$ \textit{см}. Знайдіть другу основу. \nmtyear{gen}}

\answerTable{$16$ \textit{см}}{$0$ \textit{см} (точка)}{$2$ \textit{см}}{$14$ \textit{см}}{$4$ \textit{см}}

\vspace{0.7cm}

% === ЗАВДАННЯ 69-75 ===
\noindent\makebox[1.5em][l]{\textbf{69.}}\parbox[t]{\dimexpr\textwidth-1.5em}{Основи трапеції дорівнюють $5$ \textit{см} і $13$ \textit{см}. Знайдіть відрізок, що з'єднує середини діагоналей. \nmtyear{gen}}

\answerTable{$2$ \textit{см}}{$6{,}5$ \textit{см}}{$4$ \textit{см}}{$8$ \textit{см}}{$9$ \textit{см}}

\vspace{0.7cm}

\noindent\makebox[1.5em][l]{\textbf{70.}}\parbox[t]{\dimexpr\textwidth-1.5em}{Діагоналі трапеції ділять її на 4 трикутники. Площі трикутників, прилеглих до основ, дорівнюють $4$ см$^2$ і $9$ см$^2$. Знайдіть площу одного з бічних трикутників. \nmtyear{gen}}

\answerTable{$\sqrt{36}$ см$^2$}{$6$ см$^2$}{$6{,}5$ см$^2$}{$5$ см$^2$}{$13$ см$^2$}

\vspace{0.7cm}

\noindent\makebox[1.5em][l]{\textbf{71.}}\parbox[t]{\dimexpr\textwidth-1.5em}{У рівнобічній трапеції діагональ перпендикулярна до бічної сторони і дорівнює $8$ \textit{см}. Знайдіть площу трапеції, якщо її більша основа дорівнює $10$ \textit{см}. \nmtyear{gen}}

\answerTable{$32$ см$^2$}{$40$ см$^2$}{$80$ см$^2$}{$64$ см$^2$}{$48$ см$^2$}

\vspace{0.7cm}

\noindent\makebox[1.5em][l]{\textbf{72.}}\parbox[t]{\dimexpr\textwidth-1.5em}{Основи трапеції відносяться як $2:5$, середня лінія дорівнює $14$ \textit{см}. Знайдіть меншу основу. \nmtyear{gen}}

\answerTable{$8$ \textit{см}}{$4$ \textit{см}}{$10$ \textit{см}}{$6$ \textit{см}}{$20$ \textit{см}}

\vspace{0.7cm}

\noindent\makebox[1.5em][l]{\textbf{73.}}\parbox[t]{\dimexpr\textwidth-1.5em}{Бічні сторони трапеції дорівнюють $10$ \textit{см} і $17$ \textit{см}, основи --- $3$ \textit{см} і $24$ \textit{см}. Знайдіть висоту трапеції. \nmtyear{gen}}

\answerTable{$10$ \textit{см}}{$15$ \textit{см}}{$12$ \textit{см}}{$8$ \textit{см}}{$6$ \textit{см}}

\vspace{0.7cm}

\noindent\makebox[1.5em][l]{\textbf{74.}}\parbox[t]{\dimexpr\textwidth-1.5em}{У рівнобічній трапеції $ABCD$ ($BC \parallel AD$) $\angle A = 70°$. Знайдіть $\angle D$. \nmtyear{gen}}

\answerTable{$110°$}{$35°$}{$70°$}{$140°$}{$90°$}

\vspace{0.7cm}

\noindent\makebox[1.5em][l]{\textbf{75.}}\parbox[t]{\dimexpr\textwidth-1.5em}{Сума кутів при одній з бічних сторін трапеції дорівнює: \nmtyear{gen}}

\answerTable{$90°$}{$180°$}{$270°$}{$360°$}{$120°$}

\vspace{1cm}

%======================================================================
% БЛОК 6: Порівняння чотирикутників (12 завдань)
%======================================================================

\section*{Блок 6: Порівняння чотирикутників}

\noindent\makebox[1.5em][l]{\textbf{76.}}\parbox[t]{\dimexpr\textwidth-1.5em}{Яка з фігур є водночас і ромбом, і прямокутником? \nmtyear{gen}}

\answerTable{будь-який прямокутник}{квадрат}{паралелограм}{будь-який ромб}{трапеція}

\vspace{0.7cm}

\noindent\makebox[1.5em][l]{\textbf{77.}}\parbox[t]{\dimexpr\textwidth-1.5em}{Кожен квадрат є: \nmtyear{gen}}

\answerTable{ромбом}{трапецією}{тільки прямокутником}{тільки паралелограмом}{довільним чотирикутником}

\vspace{0.7cm}

\noindent\makebox[1.5em][l]{\textbf{78.}}\parbox[t]{\dimexpr\textwidth-1.5em}{Діагоналі якого чотирикутника НЕ обов'язково рівні? \nmtyear{gen}}

\answerTable{прямокутника}{усі перераховані мають рівні діагоналі}{рівнобічної трапеції}{ромба}{квадрата}

\vspace{0.7cm}

\noindent\makebox[1.5em][l]{\textbf{79.}}\parbox[t]{\dimexpr\textwidth-1.5em}{Діагоналі якого чотирикутника завжди перпендикулярні? \nmtyear{gen}}

\answerTable{прямокутника}{паралелограма}{будь-якого чотирикутника}{ромба}{трапеції}

\vspace{0.7cm}

\noindent\makebox[1.5em][l]{\textbf{80.}}\parbox[t]{\dimexpr\textwidth-1.5em}{Який чотирикутник має всі сторони рівні, але кути не обов'язково прямі? \nmtyear{gen}}

\answerTable{трапеція}{прямокутник}{паралелограм}{ромб}{квадрат}

\vspace{0.7cm}

\noindent\makebox[1.5em][l]{\textbf{81.}}\parbox[t]{\dimexpr\textwidth-1.5em}{Який чотирикутник має всі кути прямі, але сторони не обов'язково рівні? \nmtyear{gen}}

\answerTable{ромб}{прямокутник}{трапеція}{квадрат}{паралелограм}

\vspace{0.7cm}

\noindent\makebox[1.5em][l]{\textbf{82.}}\parbox[t]{\dimexpr\textwidth-1.5em}{Чи правильно, що кожен паралелограм є трапецією? \nmtyear{gen}}

\answerTable{так}{залежить від кутів}{ні}{залежить від сторін}{тільки якщо це ромб}

\vspace{0.7cm}

\noindent\makebox[1.5em][l]{\textbf{83.}}\parbox[t]{\dimexpr\textwidth-1.5em}{У якому чотирикутнику діагоналі є бісектрисами кутів? \nmtyear{gen}}

\answerTable{трапеція}{прямокутник}{ромб}{довільний паралелограм}{будь-який чотирикутник}

\vspace{0.7cm}

\noindent\makebox[1.5em][l]{\textbf{84.}}\parbox[t]{\dimexpr\textwidth-1.5em}{Яка з фігур має тільки одну пару паралельних сторін? \nmtyear{gen}}

\answerTable{квадрат}{паралелограм}{прямокутник}{трапеція}{ромб}

\vspace{0.7cm}

\noindent\makebox[1.5em][l]{\textbf{85.}}\parbox[t]{\dimexpr\textwidth-1.5em}{Площа якого чотирикутника обчислюється як половина добутку діагоналей? \nmtyear{gen}}

\answerTable{ромба}{трапеції}{довільного чотирикутника}{паралелограма}{прямокутника}

\vspace{0.7cm}

\noindent\makebox[1.5em][l]{\textbf{86.}}\parbox[t]{\dimexpr\textwidth-1.5em}{Яка фігура НЕ є паралелограмом? \nmtyear{gen}}

\answerTable{усі перераховані є паралелограмами}{ромб}{прямокутник}{квадрат}{трапеція}

\vspace{0.7cm}

\noindent\makebox[1.5em][l]{\textbf{87.}}\parbox[t]{\dimexpr\textwidth-1.5em}{У якого чотирикутника сума протилежних кутів дорівнює $180°$? \nmtyear{gen}}

\answerTable{будь-якого паралелограма}{квадрата}{описаного навколо кола}{вписаного в коло}{будь-якої трапеції}

\vspace{1cm}

%======================================================================
% БЛОК 7: Задачі на обчислення (13 завдань)
%======================================================================

\section*{Блок 7: Задачі на обчислення}

% === ЗАВДАННЯ 88 ===
\noindent\textbf{88.} \begin{minipage}[t]{0.55\textwidth}
У паралелограмі $ABCD$ з гострим кутом $\angle A = 30°$ на стороні $AD$ вибрано точку $K$ так, що $AB = AK = KD$ (див. рисунок). Визначте периметр паралелограма $ABCD$, якщо $BK = 8$ \textit{см}. \nmtyear{gen}
\end{minipage}
\hfill
\begin{minipage}[t]{0.4\textwidth}
    \vspace{-0.5cm}
    \begin{flushright}
    \begin{tikzpicture}[scale=0.8]
        \coordinate (A) at (0,0);
        \coordinate (K) at (2,0);
        \coordinate (D) at (4,0);
        \coordinate (B) at (30:2);
        \coordinate (C) at ($(D)+(B)-(A)$);

        \draw[thick] (A) -- (B) -- (C) -- (D) -- cycle;
        \draw[thick] (B) -- (K);

        \draw ($(A)!0.5!(B)$) ++(120:0.1) -- ++(-60:0.2);
        \draw ($(A)!0.5!(K)$) ++(90:0.1) -- ++(-90:0.2);
        \draw ($(K)!0.5!(D)$) ++(90:0.1) -- ++(-90:0.2);

        \pic [draw, pic text={\small $30°$}, angle radius=0.5cm, angle eccentricity=1.7] {angle = K--A--B};

        \node[below left] at (A) {$A$};
        \node[above left] at (B) {$B$};
        \node[above right] at (C) {$C$};
        \node[below right] at (D) {$D$};
        \node[below] at (K) {$K$};
        \fill (K) circle (1.5pt);
    \end{tikzpicture}
    \end{flushright}
\end{minipage}

\vspace{0.2cm}
\answerTableTall{$48$ \textit{см}}{$32$ \textit{см}}{$24\sqrt{3}$ \textit{см}}{$\dfrac{48}{\sqrt{3}}$ \textit{см}}{$\dfrac{24}{\sin 15°}$ \textit{см}}

\vspace{0.7cm}

% === ЗАВДАННЯ 89 ===
\noindent\makebox[1.5em][l]{\textbf{89.}}\parbox[t]{\dimexpr\textwidth-1.5em}{Діагоналі ромба відносяться як $3:4$, а його площа дорівнює $96$ см$^2$. Знайдіть периметр ромба. \nmtyear{gen}}

\answerTable{$20$ \textit{см}}{$28$ \textit{см}}{$32$ \textit{см}}{$52$ \textit{см}}{$40$ \textit{см}}

\vspace{0.7cm}

% === ЗАВДАННЯ 90 ===
\noindent\makebox[1.5em][l]{\textbf{90.}}\parbox[t]{\dimexpr\textwidth-1.5em}{У прямокутник вписано коло радіуса $4$ \textit{см}. Знайдіть площу прямокутника. \nmtyear{gen}}

\answerTable{$16\pi$ см$^2$}{$128$ см$^2$}{$64$ см$^2$}{$256$ см$^2$}{$32$ см$^2$}

\vspace{0.7cm}

% === ЗАВДАННЯ 91 ===
\noindent\makebox[1.5em][l]{\textbf{91.}}\parbox[t]{\dimexpr\textwidth-1.5em}{У рівнобічну трапецію з основами $6$ \textit{см} і $10$ \textit{см} вписано коло. Знайдіть бічну сторону трапеції. \nmtyear{gen}}

\answerTable{$8$ \textit{см}}{$4$ \textit{см}}{$5$ \textit{см}}{$16$ \textit{см}}{$12$ \textit{см}}

\vspace{0.7cm}

% === ЗАВДАННЯ 92 ===
\noindent\makebox[1.5em][l]{\textbf{92.}}\parbox[t]{\dimexpr\textwidth-1.5em}{Площа квадрата, побудованого на діагоналі даного квадрата, дорівнює $50$ см$^2$. Знайдіть сторону даного квадрата. \nmtyear{gen}}

\answerTable{$\sqrt{50}$ \textit{см}}{$5\sqrt{2}$ \textit{см}}{$25$ \textit{см}}{$5$ \textit{см}}{$10$ \textit{см}}

\vspace{0.7cm}

% === ЗАВДАННЯ 93 ===
\noindent\makebox[1.5em][l]{\textbf{93.}}\parbox[t]{\dimexpr\textwidth-1.5em}{Сторона ромба дорівнює $12$ \textit{см}, а один з кутів --- $30°$. Знайдіть площу ромба. \nmtyear{gen}}

\answerTable{$72$ см$^2$}{$144$ см$^2$}{$36\sqrt{3}$ см$^2$}{$36$ см$^2$}{$72\sqrt{3}$ см$^2$}

\vspace{0.7cm}

% === ЗАВДАННЯ 94-100 ===
\noindent\makebox[1.5em][l]{\textbf{94.}}\parbox[t]{\dimexpr\textwidth-1.5em}{У паралелограмі діагоналі дорівнюють $10$ \textit{см} і $14$ \textit{см}, а кут між ними --- $90°$. Знайдіть площу паралелограма. \nmtyear{gen}}

\answerTable{$35$ см$^2$}{$70$ см$^2$}{$48$ см$^2$}{$24$ см$^2$}{$140$ см$^2$}

\vspace{0.7cm}

\noindent\makebox[1.5em][l]{\textbf{95.}}\parbox[t]{\dimexpr\textwidth-1.5em}{Периметр прямокутника дорівнює $28$ \textit{см}, а різниця сторін --- $4$ \textit{см}. Знайдіть площу прямокутника. \nmtyear{gen}}

\answerTable{$48$ см$^2$}{$40$ см$^2$}{$45$ см$^2$}{$36$ см$^2$}{$49$ см$^2$}

\vspace{0.7cm}

\noindent\makebox[1.5em][l]{\textbf{96.}}\parbox[t]{\dimexpr\textwidth-1.5em}{Основи трапеції дорівнюють $8$ \textit{см} і $14$ \textit{см}, а її площа --- $77$ см$^2$. Знайдіть висоту трапеції. \nmtyear{gen}}

\answerTable{$11$ \textit{см}}{$5{,}5$ \textit{см}}{$7$ \textit{см}}{$14$ \textit{см}}{$22$ \textit{см}}

\vspace{0.7cm}

\noindent\makebox[1.5em][l]{\textbf{97.}}\parbox[t]{\dimexpr\textwidth-1.5em}{У квадрат зі стороною $10$ \textit{см} вписано коло. Знайдіть площу частини квадрата, що знаходиться поза колом. \nmtyear{gen}}

\answerTable{$25\pi$ см$^2$}{$100-50\pi$ см$^2$}{$100-25\pi$ см$^2$}{$100-100\pi$ см$^2$}{$100\pi$ см$^2$}

\vspace{0.7cm}

\noindent\makebox[1.5em][l]{\textbf{98.}}\parbox[t]{\dimexpr\textwidth-1.5em}{Діагональ ромба дорівнює $d$, а один з кутів ромба --- $60°$. Знайдіть площу ромба. \nmtyear{gen}}

\answerTable{$d^2\sqrt{3}$}{$\dfrac{d^2}{2}$}{$\dfrac{d^2\sqrt{3}}{2}$}{$d^2$}{$\dfrac{d^2\sqrt{3}}{4}$}

\vspace{0.7cm}

\noindent\makebox[1.5em][l]{\textbf{99.}}\parbox[t]{\dimexpr\textwidth-1.5em}{У прямокутній трапеції основи дорівнюють $5$ \textit{см} і $11$ \textit{см}, а більша бічна сторона --- $10$ \textit{см}. Знайдіть меншу бічну сторону (висоту). \nmtyear{gen}}

\answerTable{$4$ \textit{см}}{$\sqrt{136}$ \textit{см}}{$6$ \textit{см}}{$8$ \textit{см}}{$12$ \textit{см}}

\vspace{0.7cm}

\noindent\makebox[1.5em][l]{\textbf{100.}}\parbox[t]{\dimexpr\textwidth-1.5em}{Периметр паралелограма дорівнює $60$ \textit{см}, а одна сторона більша за іншу в $2$ рази. Знайдіть більшу сторону паралелограма. \nmtyear{gen}}

\answerTable{$40$ \textit{см}}{$10$ \textit{см}}{$20$ \textit{см}}{$30$ \textit{см}}{$15$ \textit{см}}

\end{document}
