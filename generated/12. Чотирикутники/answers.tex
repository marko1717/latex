\documentclass[12pt]{extarticle}
\usepackage{fontspec}
\usepackage{polyglossia}
\setdefaultlanguage{ukrainian}

\defaultfontfeatures{Ligatures=TeX}
\setmainfont{Liberation Serif}

\usepackage[a4paper,margin=2cm]{geometry}
\usepackage{amsmath,amssymb}
\usepackage{multicol}
\usepackage{xcolor}

\definecolor{headerblue}{RGB}{0, 102, 204}

\begin{document}

\begin{center}
{\Large\textbf{\color{headerblue}ВІДПОВІДІ}}
\end{center}

\begin{center}
{\large Тема 12: Чотирикутники}
\end{center}

\vspace{0.5cm}

\textbf{Блок 1: Паралелограм - основні властивості}

\begin{multicols}{5}
\noindent
1. Б ($132°$) \\
2. Б ($110°$) \\
3. Б \\
4. АВГ \\
5. Б \\
6. Б ($40$) \\
7. А ($8$) \\
8. А \\
9. А ($44$) \\
10. А ($132°$) \\
11. А \\
12. БАВ \\
13. А ($8$) \\
14. А ($360°$) \\
15. А
\end{multicols}

\textbf{Блок 2: Прямокутник}

\begin{multicols}{5}
\noindent
16. А ($13$) \\
17. А ($8$) \\
18. А ($72$) \\
19. А ($28$) \\
20. А \\
21. А ($60°$) \\
22. В \\
23. А ($5$) \\
24. А ($108$) \\
25. А ($6$) \\
26. А ($14$) \\
27. А \\
28. А \\
29. А \\
30. А ($5$)
\end{multicols}

\textbf{Блок 3: Ромб}

\begin{multicols}{5}
\noindent
31. А ($5$) \\
32. А ($36$) \\
33. А \\
34. А ($120$) \\
35. А ($12$) \\
36. А ($60°$) \\
37. А ($10$) \\
38. А ($12$) \\
39. А ($120°$) \\
40. А \\
41. А ($10$) \\
42. А ($12$) \\
43. А ($80$) \\
44. А ($130°$) \\
45. А
\end{multicols}

\textbf{Блок 4: Квадрат}

\begin{multicols}{5}
\noindent
46. А ($28$) \\
47. А ($9$) \\
48. А ($8$) \\
49. А ($5\sqrt{2}$) \\
50. А ($8$) \\
51. А ($72$) \\
52. А \\
53. А ($6$) \\
54. А ($5\sqrt{2}$) \\
55. А ($10$) \\
56. А ($36$) \\
57. А \\
58. А \\
59. А ($\sqrt{2}$) \\
60. А ($10$)
\end{multicols}

\textbf{Блок 5: Трапеція}

\begin{multicols}{5}
\noindent
61. А ($11$) \\
62. А ($20$) \\
63. А ($105$) \\
64. А ($8$) \\
65. А ($4$) \\
66. А \\
67. А ($115°$) \\
68. А \\
69. А ($4$) \\
70. А ($6$) \\
71. А ($48$) \\
72. А ($8$) \\
73. А ($8$) \\
74. А ($70°$) \\
75. А ($180°$)
\end{multicols}

\textbf{Блок 6: Порівняння чотирикутників}

\begin{multicols}{5}
\noindent
76. А \\
77. А \\
78. А \\
79. А \\
80. А \\
81. А \\
82. А \\
83. А \\
84. А \\
85. А \\
86. А \\
87. А
\end{multicols}

\textbf{Блок 7: Задачі на обчислення}

\begin{multicols}{5}
\noindent
88. А \\
89. А ($40$) \\
90. А ($64$) \\
91. А ($8$) \\
92. А ($5$) \\
93. А ($72$) \\
94. А ($70$) \\
95. А ($45$) \\
96. А ($7$) \\
97. А \\
98. А \\
99. А ($8$) \\
100. А ($20$)
\end{multicols}

\vspace{1cm}

\textbf{Ключові формули:}

\begin{enumerate}
\item \textbf{Паралелограм:}
\begin{itemize}
\item Протилежні сторони і кути рівні
\item Сума сусідніх кутів $= 180°$
\item Діагоналі діляться навпіл
\item $S = a \cdot h$
\end{itemize}

\item \textbf{Прямокутник:}
\begin{itemize}
\item Усі кути прямі
\item Діагоналі рівні: $d = \sqrt{a^2 + b^2}$
\item $S = a \cdot b$
\end{itemize}

\item \textbf{Ромб:}
\begin{itemize}
\item Усі сторони рівні
\item Діагоналі перпендикулярні і є бісектрисами
\item $S = \dfrac{d_1 \cdot d_2}{2}$
\item $a^2 = \left(\dfrac{d_1}{2}\right)^2 + \left(\dfrac{d_2}{2}\right)^2$
\end{itemize}

\item \textbf{Квадрат:}
\begin{itemize}
\item Усі сторони і кути рівні
\item $d = a\sqrt{2}$
\item $S = a^2 = \dfrac{d^2}{2}$
\item $R = \dfrac{a\sqrt{2}}{2}$, $r = \dfrac{a}{2}$
\end{itemize}

\item \textbf{Трапеція:}
\begin{itemize}
\item Середня лінія $m = \dfrac{a + b}{2}$
\item $S = \dfrac{(a + b) \cdot h}{2} = m \cdot h$
\item У рівнобічній: діагоналі рівні, кути при основі рівні
\end{itemize}
\end{enumerate}

\end{document}
