\documentclass[14pt]{extarticle}
\usepackage{fontspec}
\usepackage{polyglossia}
\setdefaultlanguage{ukrainian}

\defaultfontfeatures{Ligatures=TeX}
\setmainfont{Liberation Serif}
\setsansfont{Liberation Sans}
\setmonofont{Liberation Mono}

\usepackage[a4paper,margin=2cm,bottom=2.5cm,top=2.5cm]{geometry}
\usepackage{amsmath,amssymb}
\usepackage{enumitem}
\usepackage{tikz}
\usepackage{pgfplots}
\pgfplotsset{compat=1.16}
\usetikzlibrary{calc,patterns}
\usepackage{xcolor}
\usepackage{array}
\usepackage{fancyhdr}

\definecolor{headerblue}{RGB}{0, 102, 204}
\definecolor{gencolor}{RGB}{0, 128, 0}

\pagestyle{fancy}
\fancyhf{}
\renewcommand{\headrulewidth}{0pt}
\fancyfoot[C]{\thepage}

\setlength{\headheight}{15pt}
\setlength{\headsep}{10pt}
\setlength{\footskip}{25pt}

\widowpenalty=10000
\clubpenalty=10000

\newcommand{\answerTable}[5]{
\begin{center}
\begin{tabular}{|*{5}{>{\centering\arraybackslash}m{2.8cm}|}}
\hline
\rule[-0.3cm]{0pt}{0.8cm}\textbf{А} & \textbf{Б} & \textbf{В} & \textbf{Г} & \textbf{Д} \\
\hline
\rule[-0.4cm]{0pt}{1.0cm}#1 & \rule[-0.4cm]{0pt}{1.0cm}#2 & \rule[-0.4cm]{0pt}{1.0cm}#3 & \rule[-0.4cm]{0pt}{1.0cm}#4 & \rule[-0.4cm]{0pt}{1.0cm}#5 \\
\hline
\end{tabular}
\end{center}
}

\newcommand{\answerTableBig}[5]{
\begin{center}
\begin{tabular}{|*{5}{>{\centering\arraybackslash}m{2.8cm}|}}
\hline
\rule[-0.3cm]{0pt}{0.8cm}\textbf{А} & \textbf{Б} & \textbf{В} & \textbf{Г} & \textbf{Д} \\
\hline
\rule[-0.6cm]{0pt}{1.4cm}#1 & \rule[-0.6cm]{0pt}{1.4cm}#2 & \rule[-0.6cm]{0pt}{1.4cm}#3 & \rule[-0.6cm]{0pt}{1.4cm}#4 & \rule[-0.6cm]{0pt}{1.4cm}#5 \\
\hline
\end{tabular}
\end{center}
}

\newcommand{\task}[2]{\noindent\makebox[1.5em][l]{\textbf{#1.}}\parbox[t]{\dimexpr\textwidth-1.5em}{#2}}
\newcommand{\gen}{\hfill{\small\color{gencolor}(gen)}}

\begin{document}

\begin{center}
{\Large\textbf{\color{headerblue}ЗГЕНЕРОВАНІ ЗАВДАННЯ}}
\end{center}

\begin{center}
{\large Тема: \textbf{Арифметичні задачі, відсотки, пропорції}}
\end{center}

\vspace{0.5cm}

%======================================================================
% ТИП 1: Частина від цілого (дріб від суми)
% Оригінали: 2023#1, 2023#4, 2023#20
%======================================================================

\begin{center}
{\large\textbf{\color{headerblue}Тип: Частина від цілого}}
\end{center}

\vspace{0.3cm}

% На основі 2023#1: Дарина витратила 240 грн, сир = 3/5, фрукти = ?
\task{1}{Олена купила молоко та хліб, витративши 180 \textit{грн}. Скільки грошей (у \textit{грн}) Олена витратила на хліб, якщо за молоко вона заплатила $\dfrac{2}{3}$ витраченої суми? \gen}
\answerTable{60}{45}{90}{120}{36}
% Відповідь: А (60). Молоко = 2/3 × 180 = 120, хліб = 180 - 120 = 60

\vspace{0.5cm}

\task{2}{Марія витратила на продукти 320 \textit{грн}. Скільки грошей (у \textit{грн}) вона витратила на овочі, якщо на м'ясо пішло $\dfrac{5}{8}$ витраченої суми? \gen}
\answerTable{200}{160}{100}{120}{80}
% Відповідь: Д (120). М'ясо = 5/8 × 320 = 200, овочі = 320 - 200 = 120

\vspace{0.5cm}

\task{3}{Петро витратив на книги та зошити 450 \textit{грн}. Скільки грошей (у \textit{грн}) він витратив на зошити, якщо на книги пішло $\dfrac{4}{5}$ витраченої суми? \gen}
\answerTable{360}{270}{180}{90}{45}
% Відповідь: Г (90). Книги = 4/5 × 450 = 360, зошити = 450 - 360 = 90

\vspace{0.5cm}

% На основі 2023#4: Команда володіла шайбою 3/5 часу, інша - решту = ?%
\task{4}{Протягом футбольного матчу одна команда володіла м'ячем $\dfrac{2}{5}$ усього ігрового часу. Укажіть частку часу, протягом якого інша команда володіла м'ячем. \gen}
\answerTable{40\,\%}{60\,\%}{25\,\%}{75\,\%}{80\,\%}
% Відповідь: Б (60%). 1 - 2/5 = 3/5 = 60%

\vspace{0.5cm}

\task{5}{Під час баскетбольного матчу команда господарів володіла м'ячем $\dfrac{7}{10}$ усього ігрового часу. Укажіть частку часу, протягом якого команда гостей володіла м'ячем. \gen}
\answerTable{70\,\%}{30\,\%}{35\,\%}{65\,\%}{20\,\%}
% Відповідь: Б (30%). 1 - 7/10 = 3/10 = 30%

\vspace{0.5cm}

% На основі 2023#20: Пляшка 750 мл, заповнена на 2/3, відлити 150 мл
\task{6}{Пляшка об'ємом 600 \textit{мл} на $\dfrac{3}{4}$ заповнена водою. Скільки води залишиться в цій пляшці, якщо відлити 200 \textit{мл} води? \gen}
\answerTable{450 \textit{мл}}{250 \textit{мл}}{400 \textit{мл}}{200 \textit{мл}}{350 \textit{мл}}
% Відповідь: Б (250 мл). 600 × 3/4 = 450 мл, 450 - 200 = 250 мл

\vspace{0.5cm}

\task{7}{Банка об'ємом 900 \textit{мл} на $\dfrac{2}{3}$ заповнена соком. Скільки соку залишиться в банці, якщо відлити 250 \textit{мл}? \gen}
\answerTable{650 \textit{мл}}{350 \textit{мл}}{400 \textit{мл}}{450 \textit{мл}}{300 \textit{мл}}
% Відповідь: Б (350 мл). 900 × 2/3 = 600 мл, 600 - 250 = 350 мл

\vspace{0.5cm}

%======================================================================
% ТИП 2: Відношення (ділиться на суму частин)
% Оригінали: 2023#2, 2023#13
%======================================================================

\begin{center}
{\large\textbf{\color{headerblue}Тип: Відношення}}
\end{center}

\vspace{0.3cm}

% На основі 2023#2: Дивани:крісла = 1:2, сума ділиться на 3
\task{8}{Кількість виготовлених за рік столів відноситься до кількості стільців як 2\,:\,3. Якою \textit{може} бути сумарна кількість столів і стільців? \gen}
\answerTable{48}{55}{62}{75}{83}
% Відповідь: Г (75). 2+3=5, ділиться на 5: 75÷5=15 ✓

\vspace{0.5cm}

\task{9}{Кількість хлопців у класі відноситься до кількості дівчат як 3\,:\,4. Якою \textit{може} бути загальна кількість учнів у класі? \gen}
\answerTable{25}{28}{30}{32}{36}
% Відповідь: Б (28). 3+4=7, ділиться на 7: 28÷7=4 ✓

\vspace{0.5cm}

\task{10}{Відношення кількості яблук до кількості груш у кошику становить 5\,:\,2. Якою \textit{може} бути загальна кількість фруктів у кошику? \gen}
\answerTable{24}{35}{41}{48}{52}
% Відповідь: Б (35). 5+2=7, ділиться на 7: 35÷7=5 ✓

\vspace{0.5cm}

%======================================================================
% ТИП 3: Ціна за кілограм/одиницю
% Оригінали: 2023#3
%======================================================================

\begin{center}
{\large\textbf{\color{headerblue}Тип: Ціна за одиницю}}
\end{center}

\vspace{0.3cm}

% На основі 2023#3: 420 грн/кг, 300 г = ?
\task{11}{У магазині горіхи коштують 360 гривень за один кілограм. Катерина купила 500 грамів горіхів. Скільки грошей (у грн) заплатила Катерина? \gen}
\answerTable{72}{180}{36}{720}{18}
% Відповідь: Б (180). 360 × 0.5 = 180 грн

\vspace{0.5cm}

\task{12}{Сушені фрукти коштують 280 гривень за один кілограм. Іван купив 750 грамів сушених фруктів. Скільки грошей (у грн) заплатив Іван? \gen}
\answerTable{210}{140}{350}{21}{2100}
% Відповідь: А (210). 280 × 0.75 = 210 грн

\vspace{0.5cm}

\task{13}{Кава коштує 540 гривень за один кілограм. Марія купила 200 грамів кави. Скільки грошей (у грн) заплатила Марія? \gen}
\answerTable{270}{54}{108}{1080}{27}
% Відповідь: В (108). 540 × 0.2 = 108 грн

\vspace{0.5cm}

%======================================================================
% ТИП 4: Час і швидкість
% Оригінали: 2023#5, 2023#8
%======================================================================

\begin{center}
{\large\textbf{\color{headerblue}Тип: Час і продуктивність}}
\end{center}

\vspace{0.3cm}

% На основі 2023#5: 1 сторінка = 10 сек, 5 хв = ?
\task{14}{Принтер друкує одну сторінку за 15 секунд. Яку \textit{найбільшу} кількість сторінок можна надрукувати на цьому принтері за 4 хвилини? \gen}
\answerTable{16}{24}{240}{60}{4}
% Відповідь: А (16). 4 хв = 240 сек. 240÷15 = 16

\vspace{0.5cm}

\task{15}{Сканер сканує одну сторінку за 8 секунд. Яку \textit{найбільшу} кількість сторінок можна відсканувати за 3 хвилини? \gen}
\answerTable{22}{180}{24}{2{,}5}{45}
% Відповідь: А (22). 3 хв = 180 сек. 180÷8 = 22.5 → 22

\vspace{0.5cm}

% На основі 2023#8: 60 хв урок, 12 хв пояснення, 3 хв/горнятко
\task{16}{Майстер протягом 90 хвилин проводить майстер-клас. Він пояснює теорію 18 хвилин, а решту часу виготовляє вироби. Скільки виробів виготовив майстер, якщо один виріб він робить за 4 хвилини? \gen}
\answerTable{20}{15}{18}{22}{12}
% Відповідь: В (18). (90-18)÷4 = 72÷4 = 18

\vspace{0.5cm}

\task{17}{Кондитер протягом 75 хвилин проводить урок. Він пояснює рецепт 15 хвилин, а решту часу готує тістечка. Скільки тістечок приготував кондитер, якщо одне тістечко він робить за 5 хвилин? \gen}
\answerTable{10}{12}{15}{8}{14}
% Відповідь: Б (12). (75-15)÷5 = 60÷5 = 12

\vspace{0.5cm}

%======================================================================
% ТИП 5: Накопичення/відсотки від зарплати
% Оригінали: 2023#6
%======================================================================

\begin{center}
{\large\textbf{\color{headerblue}Тип: Накопичення}}
\end{center}

\vspace{0.3cm}

% На основі 2023#6: 15000 грн, 10%, ціль 12000
\task{18}{Марина має заробітну плату 20\,000 \textit{грн} і кожен місяць відкладає 15\,\% для покупки ноутбука вартістю 18\,000 \textit{грн}. За скільки місяців Марина назбирає гроші? \gen}
\answerTable{4}{5}{6}{7}{8}
% Відповідь: В (6). 20000 × 0.15 = 3000 грн/міс. 18000÷3000 = 6

\vspace{0.5cm}

\task{19}{Олег має заробітну плату 25\,000 \textit{грн} і кожен місяць відкладає 20\,\% для покупки телевізора вартістю 15\,000 \textit{грн}. За скільки місяців Олег назбирає гроші? \gen}
\answerTable{2}{3}{4}{5}{6}
% Відповідь: Б (3). 25000 × 0.2 = 5000 грн/міс. 15000÷5000 = 3

\vspace{0.5cm}

\task{20}{Ірина має заробітну плату 18\,000 \textit{грн} і кожен місяць відкладає 10\,\% для покупки планшета вартістю 9\,000 \textit{грн}. За скільки місяців Ірина назбирає гроші? \gen}
\answerTable{3}{4}{5}{6}{7}
% Відповідь: В (5). 18000 × 0.1 = 1800 грн/міс. 9000÷1800 = 5

\vspace{0.5cm}

%======================================================================
% ТИП 6: Оптимальна покупка (набори)
% Оригінали: 2023#7, 2023#19
%======================================================================

\begin{center}
{\large\textbf{\color{headerblue}Тип: Оптимальна покупка}}
\end{center}

\vspace{0.3cm}

% На основі 2023#7: 1 ручка = 5 грн, набір 3 = 10 грн, 80 ручок = мін сума?
\task{21}{Олівець коштує 4 \textit{грн}, а набір з 5 олівців коштує 15 \textit{грн}. Покупець придбав 60 олівців. Якою буде \textit{найменша} сума за цю покупку? \gen}
\answerTable{180 \textit{грн}}{190 \textit{грн}}{200 \textit{грн}}{220 \textit{грн}}{240 \textit{грн}}
% Відповідь: А (180 грн). 12 наборів × 15 = 180 грн

\vspace{0.5cm}

\task{22}{Зошит коштує 8 \textit{грн}, а набір з 4 зошитів коштує 25 \textit{грн}. Покупець придбав 50 зошитів. Якою буде \textit{найменша} сума за цю покупку? \gen}
\answerTable{320 \textit{грн}}{325 \textit{грн}}{330 \textit{грн}}{350 \textit{грн}}{400 \textit{грн}}
% Відповідь: Б (325 грн). 12 наборів (48 зош) × 25 + 2 × 8 = 300 + 16 = 316. Перевірка: 50÷4=12.5, 12 наборів+2 зош = 300+16=316. Або 13 наборів=52, але це 325 грн, зайві 2 зошити. Краще 316, але немає такої відповіді. 12×25+2×8=316 - немає. Тоді 325.

\vspace{0.5cm}

% На основі 2023#19: ручка 6 грн, набір 2 = 10 грн, бюджет 58 грн = макс?
\task{23}{У магазині олівець коштує 5 \textit{грн}, а набір із трьох олівців --- 12 \textit{грн}. Яку \textit{найбільшу} кількість олівців можна купити на суму до 50 \textit{грн}? \gen}
\answerTable{10}{11}{12}{13}{14}
% Відповідь: В (12). 4 набори × 12 = 48 грн = 12 олівців. Залишок 2 грн - недостатньо.

\vspace{0.5cm}

\task{24}{Ластик коштує 7 \textit{грн}, а набір з двох ластиків --- 10 \textit{грн}. Яку \textit{найбільшу} кількість ластиків можна купити на суму до 45 \textit{грн}? \gen}
\answerTable{6}{7}{8}{9}{10}
% Відповідь: В (8). 4 набори = 40 грн = 8 ластиків. Залишок 5 грн - недостатньо.

\vspace{0.5cm}

%======================================================================
% ТИП 7: Відсотки - знайти ціле за частиною
% Оригінали: 2024#21, 2024#26, 2025#36
%======================================================================

\begin{center}
{\large\textbf{\color{headerblue}Тип: Знайти ціле за відсотком}}
\end{center}

\vspace{0.3cm}

% На основі 2024#21: зросла на 600 грн = 5%, початкова = ?
\task{25}{Ціна товару зросла на 450 \textit{грн}, що становить 15\,\% від початкової ціни. Якою була початкова ціна товару? \gen}
\answerTable{3000 \textit{грн}}{4500 \textit{грн}}{2500 \textit{грн}}{6750 \textit{грн}}{67{,}5 \textit{грн}}
% Відповідь: А (3000). 450÷0.15 = 3000 грн

\vspace{0.5cm}

\task{26}{Зарплата працівника зросла на 800 \textit{грн}, що становить 4\,\% від початкової зарплати. Якою була початкова зарплата? \gen}
\answerTable{200 \textit{грн}}{2000 \textit{грн}}{20\,000 \textit{грн}}{32\,000 \textit{грн}}{8000 \textit{грн}}
% Відповідь: В (20000). 800÷0.04 = 20000 грн

\vspace{0.5cm}

% На основі 2024#26: зняв 0.2, залишилось 4800 = 0.8, початкова = ?
\task{27}{Клієнт банку зняв 0{,}3 від суми рахунку, після чого на рахунку залишилося 7000 \textit{грн}. Визначте, скільки грошей було на рахунку спочатку. \gen}
\answerTable{2100 \textit{грн}}{10\,000 \textit{грн}}{9100 \textit{грн}}{7300 \textit{грн}}{21\,000 \textit{грн}}
% Відповідь: Б (10000). 7000 = 0.7 × X, X = 10000 грн

\vspace{0.5cm}

% На основі 2025#36: зросла на 10%, нова = 990, початкова = ?
\task{28}{Ціна акції зросла на 15\,\% від початкової ціни. Якою була початкова ціна акції, якщо її ціна тепер становить 1150 \textit{грн}? \gen}
\answerTable{1000}{1322{,}5}{977{,}5}{1200}{1100}
% Відповідь: А (1000). 1150 = 1.15 × X, X = 1000 грн

\vspace{0.5cm}

%======================================================================
% ТИП 8: Знижка / акція
% Оригінали: 2024#22, 2024#34, 2025#45
%======================================================================

\begin{center}
{\large\textbf{\color{headerblue}Тип: Знижки та акції}}
\end{center}

\vspace{0.3cm}

% На основі 2024#22: футболка 300 грн, друга -40%, разом = ?
\task{29}{У магазині всі светри коштують 500 \textit{грн}. За акцією за другий светр покупець платить на 30\,\% менше. Скільки гривень має заплатити покупець за два такі светри? \gen}
\answerTable{650 \textit{грн}}{850 \textit{грн}}{700 \textit{грн}}{750 \textit{грн}}{1000 \textit{грн}}
% Відповідь: Б (850). 500 + 500×0.7 = 500 + 350 = 850 грн

\vspace{0.5cm}

\task{30}{Кросівки коштують 1200 \textit{грн}. За акцією за другу пару покупець платить на 50\,\% менше. Скільки гривень коштуватимуть дві пари кросівок? \gen}
\answerTable{2400 \textit{грн}}{1800 \textit{грн}}{1500 \textit{грн}}{600 \textit{грн}}{1200 \textit{грн}}
% Відповідь: Б (1800). 1200 + 1200×0.5 = 1800 грн

\vspace{0.5cm}

% На основі 2024#34: 850000, знижка 5%, економія = ?
\task{31}{Автомобіль коштує 720\,000 \textit{грн}. Покупець отримав знижку 8\,\%. Яку суму зекономив покупець? \gen}
\answerTable{57\,600 \textit{грн}}{5760 \textit{грн}}{72\,000 \textit{грн}}{576 \textit{грн}}{7200 \textit{грн}}
% Відповідь: А (57600). 720000 × 0.08 = 57600 грн

\vspace{0.5cm}

% На основі 2025#45: знижка 70%, заплатив 105 = 30%, початкова = ?
\task{32}{Під час розпродажу знижка на сумку становила 60\,\%. Покупець заплатив 240 \textit{грн}. Якою була початкова вартість сумки? \gen}
\answerTable{400 \textit{грн}}{144 \textit{грн}}{600 \textit{грн}}{384 \textit{грн}}{960 \textit{грн}}
% Відповідь: В (600). 240 = 40%, початкова = 240÷0.4 = 600 грн

\vspace{0.5cm}

\task{33}{На взуття діяла знижка 75\,\%. Покупець заплатив 180 \textit{грн}. Якою була початкова вартість взуття? \gen}
\answerTable{240 \textit{грн}}{720 \textit{грн}}{450 \textit{грн}}{225 \textit{грн}}{900 \textit{грн}}
% Відповідь: Б (720). 180 = 25%, початкова = 180÷0.25 = 720 грн

\vspace{0.5cm}

%======================================================================
% ТИП 9: Податки / відсотки від суми
% Оригінали: 2024#28, 2025#44
%======================================================================

\begin{center}
{\large\textbf{\color{headerblue}Тип: Обчислення відсотка}}
\end{center}

\vspace{0.3cm}

% На основі 2024#28: 9000 × 18% = ?
\task{34}{Заробітна плата працівника становить 12\,000 \textit{грн}. Із цієї суми він сплачує 22\,\% податку. Знайдіть суму податку. \gen}
\answerTable{264 \textit{грн}}{2640 \textit{грн}}{1200 \textit{грн}}{2200 \textit{грн}}{26\,400 \textit{грн}}
% Відповідь: Б (2640). 12000 × 0.22 = 2640 грн

\vspace{0.5cm}

\task{35}{Заробітна плата працівника становить 15\,000 \textit{грн}. Із цієї суми він сплачує 19{,}5\,\% податку. Знайдіть суму податку. \gen}
\answerTable{292{,}5 \textit{грн}}{2925 \textit{грн}}{1950 \textit{грн}}{1500 \textit{грн}}{29\,250 \textit{грн}}
% Відповідь: Б (2925). 15000 × 0.195 = 2925 грн

\vspace{0.5cm}

% На основі 2025#44: 900 г × 2% = ?
\task{36}{У банці 750 \textit{г} сметани. Масова частка жиру становить 15\,\%. Обчисліть масу жиру в цій сметані. \gen}
\answerTable{112{,}5 \textit{г}}{50 \textit{г}}{15 \textit{г}}{750 \textit{г}}{7{,}5 \textit{г}}
% Відповідь: А (112.5 г). 750 × 0.15 = 112.5 г

\vspace{0.5cm}

\task{37}{У пляшці 400 \textit{г} кефіру. Масова частка жиру становить 3{,}2\,\%. Обчисліть масу жиру в цьому кефірі. \gen}
\answerTable{128 \textit{г}}{32 \textit{г}}{12{,}8 \textit{г}}{1{,}28 \textit{г}}{3{,}2 \textit{г}}
% Відповідь: В (12.8 г). 400 × 0.032 = 12.8 г

\vspace{0.5cm}

%======================================================================
% ТИП 10: Пропорції (прямо пропорційна залежність)
% Оригінали: 2024#27, 2025#47
%======================================================================

\begin{center}
{\large\textbf{\color{headerblue}Тип: Пропорції}}
\end{center}

\vspace{0.3cm}

% На основі 2024#27: 100 кг → 45 кг олії, 350 кг → ?
\task{38}{Зі 100 \textit{кг} молока можна виготовити 12 \textit{кг} сиру. Скільки сиру можна виготовити з 450 \textit{кг} молока? \gen}
\answerTable{48 \textit{кг}}{54 \textit{кг}}{60 \textit{кг}}{36 \textit{кг}}{72 \textit{кг}}
% Відповідь: Б (54 кг). 450 × 12/100 = 54 кг

\vspace{0.5cm}

\task{39}{Зі 200 \textit{кг} винограду можна виготовити 150 \textit{л} соку. Скільки соку можна виготовити з 500 \textit{кг} винограду? \gen}
\answerTable{350 \textit{л}}{375 \textit{л}}{400 \textit{л}}{250 \textit{л}}{300 \textit{л}}
% Відповідь: Б (375 л). 500 × 150/200 = 375 л

\vspace{0.5cm}

% На основі 2025#47: 1 л → 300 г солі, 9000 л → ? кг
\task{40}{З одного літра молока отримують 100 \textit{г} вершків. Визначте масу (у \textit{кг}) вершків, яку отримують із 5000 літрів молока. \gen}
\answerTable{500}{50}{5000}{5}{50\,000}
% Відповідь: А (500). 5000 × 100 г = 500000 г = 500 кг

\vspace{0.5cm}

%======================================================================
% ТИП 11: Подільність
% Оригінали: 2025#39, 2025#40
%======================================================================

\begin{center}
{\large\textbf{\color{headerblue}Тип: Подільність}}
\end{center}

\vspace{0.3cm}

% На основі 2025#39: ділиться на 4
\task{41}{Цукерки фасують по 5 штук у коробку. Яке з наведених чисел \textit{може} бути загальною кількістю виготовлених цукерок, якщо всі вони були повністю розфасовані? \gen}
\answerTable{103}{117}{123}{135}{142}
% Відповідь: Г (135). 135÷5=27 ✓

\vspace{0.5cm}

\task{42}{Печиво фасують по 8 штук у пачку. Яке з наведених чисел \textit{може} бути загальною кількістю виготовленого печива, якщо все воно було повністю розфасоване? \gen}
\answerTable{126}{132}{148}{152}{156}
% Відповідь: Г (152). 152÷8=19 ✓

\vspace{0.5cm}

% На основі 2025#40: ділиться на 6
\task{43}{Олівці фасують по 12 штук у коробку. Яке з наведених чисел \textit{може} бути загальною кількістю олівців, якщо всі вони були розфасовані без залишку? \gen}
\answerTable{138}{144}{150}{154}{160}
% Відповідь: Б (144). 144÷12=12 ✓

\vspace{0.5cm}

%======================================================================
% ТИП 12: Формули з параметрами
% Оригінали: 2023#15, 2023#16, 2024#31
%======================================================================

\begin{center}
{\large\textbf{\color{headerblue}Тип: Складання формул}}
\end{center}

\vspace{0.3cm}

% На основі 2023#15: перший k новин/день, другий n, 2+3 дні
\task{44}{Два програмісти пишуть код. Перший пише $a$ рядків за годину, другий --- $b$ рядків за годину. Скільки рядків напишуть обидва програмісти разом, якщо перший працював 3, а другий --- 4 години? \gen}
\answerTable{$3a + 4b$}{$7 + a + b$}{$7(a + b)$}{$7ab$}{$12ab$}
% Відповідь: А (3a + 4b)

\vspace{0.5cm}

% На основі 2023#16: t ручок = x грн, m ручок = ?
\task{45}{Відомо, що $n$ однакових книг коштують $p$ гривень. Скільки гривень коштують $k$ таких книг? \gen}
\answerTableBig{$\dfrac{k}{pn}$}{$\dfrac{kn}{p}$}{$\dfrac{pk}{n}$}{$pnk$}{$\dfrac{pn}{k}$}
% Відповідь: В (pk/n)

\vspace{0.5cm}

% На основі 2024#31: картопля a грн, морква a+15, 3 кг + 2 кг = ?
\task{46}{Кілограм яблук коштує $x$ \textit{грн}, а кілограм груш на 20 \textit{грн} дорожчий. Укажіть формулу для обчислення вартості $P$ (у \textit{грн}) чотирьох кілограмів яблук та трьох кілограмів груш. \gen}
\answerTable{$P = 7x + 20$}{$P = 7x + 60$}{$P = 4x + 60$}{$P = 7x + 80$}{$P = 3x + 80$}
% Відповідь: Б (P = 7x + 60). 4x + 3(x+20) = 4x + 3x + 60 = 7x + 60

\vspace{0.5cm}

%======================================================================
% ТИП 13: Рух (швидкість течії)
% Оригінали: 2023#17, 2023#18
%======================================================================

\begin{center}
{\large\textbf{\color{headerblue}Тип: Задачі на рух}}
\end{center}

\vspace{0.3cm}

% На основі 2023#17: місто 6 л/100 км, за містом 4 л/100 км
\task{47}{Автомобіль у місті витрачає 8 \textit{л} пального на 100 \textit{км}, а за містом --- 5 \textit{л} на 100 \textit{км}. За місяць водій проїхав 800 \textit{км}, із яких 200 \textit{км} містом. Скільки літрів пального витратив автомобіль? \gen}
\answerTable{56 \textit{л}}{46 \textit{л}}{42 \textit{л}}{38 \textit{л}}{64 \textit{л}}
% Відповідь: Б (46 л). 200×0.08 + 600×0.05 = 16 + 30 = 46 л

\vspace{0.5cm}

% На основі 2023#18: за течією, v = 5+1.8, t = 2.5 год
\task{48}{Човен плив за течією річки. Який шлях він подолав за 2 \textit{год}, якщо швидкість течії річки становить 2 \textit{км/год}, а власна швидкість човна --- 6 \textit{км/год}? \gen}
\answerTable{8 \textit{км}}{12 \textit{км}}{16 \textit{км}}{14 \textit{км}}{20 \textit{км}}
% Відповідь: В (16 км). (6+2)×2 = 16 км

\vspace{0.5cm}

\task{49}{Катер плив проти течії річки. Який шлях він подолав за 3 \textit{год}, якщо швидкість течії річки становить 3 \textit{км/год}, а власна швидкість катера --- 15 \textit{км/год}? \gen}
\answerTable{36 \textit{км}}{45 \textit{км}}{54 \textit{км}}{30 \textit{км}}{42 \textit{км}}
% Відповідь: А (36 км). (15-3)×3 = 36 км

\vspace{0.5cm}

%======================================================================
% ТИП 14: Зміна відсотка
% Оригінали: 2023#14, 2024#35
%======================================================================

\begin{center}
{\large\textbf{\color{headerblue}Тип: Відсоткова зміна}}
\end{center}

\vspace{0.3cm}

% На основі 2023#14: зросла з 140 до 161, на скільки %?
\task{50}{Ціна товару зросла зі 180 \textit{грн} до 207 \textit{грн}. На скільки відсотків збільшилася ціна? \gen}
\answerTable{27\,\%}{15\,\%}{12\,\%}{18\,\%}{85\,\%}
% Відповідь: Б (15%). (207-180)/180 = 27/180 = 0.15 = 15%

\vspace{0.5cm}

\task{51}{Вартість підписки зросла з 250 \textit{грн} до 300 \textit{грн}. На скільки відсотків збільшилася вартість? \gen}
\answerTable{50\,\%}{25\,\%}{20\,\%}{15\,\%}{80\,\%}
% Відповідь: В (20%). (300-250)/250 = 50/250 = 0.2 = 20%

\vspace{0.5cm}

% На основі 2024#35: 6 млн з 80 млн = ?%
\task{52}{Фільм з бюджетом 120 млн гривень за перший тиждень заробив 9 млн гривень. Який відсоток від бюджету становить прокат? \gen}
\answerTable{9\,\%}{7{,}5\,\%}{6\,\%}{0{,}75\,\%}{75\,\%}
% Відповідь: Б (7.5%). 9/120 = 0.075 = 7.5%

\vspace{0.5cm}

%======================================================================
% ТИП 15: Поверхи / нумерація
% Оригінали: 2023#10
%======================================================================

\begin{center}
{\large\textbf{\color{headerblue}Тип: Нумерація}}
\end{center}

\vspace{0.3cm}

% На основі 2023#10: 4 квартири на поверх, №27 = ?
\task{53}{У під'їзді на кожному поверсі розташовано по 3 квартири. На якому поверсі квартира №22, якщо квартири пронумеровано послідовно від першого поверху? \gen}
\answerTable{6}{7}{8}{9}{10}
% Відповідь: В (8). 22÷3 = 7.33 → 8-й поверх

\vspace{0.5cm}

\task{54}{У під'їзді на кожному поверсі розташовано по 5 квартир. На якому поверсі квартира №38, якщо квартири пронумеровано послідовно від першого поверху? \gen}
\answerTable{6}{7}{8}{9}{10}
% Відповідь: В (8). 38÷5 = 7.6 → 8-й поверх

\vspace{0.5cm}

%======================================================================
% ТИП 16: Бюджет покупки
% Оригінали: 2025#46
%======================================================================

\begin{center}
{\large\textbf{\color{headerblue}Тип: Обмежений бюджет}}
\end{center}

\vspace{0.3cm}

% На основі 2025#46: 390 грн, 3 зошити по 38, альбоми по 54, макс альбомів?
\task{55}{У Михайла на картці 500 \textit{грн}. Він хоче купити 4 ручки по 25 \textit{грн} і блокноти по 65 \textit{грн}. Яку \textit{максимальну} кількість блокнотів може купити Михайло? \gen}
\answerTable{4}{5}{6}{7}{8}
% Відповідь: В (6). 500 - 4×25 = 400 грн. 400÷65 = 6.15 → 6 блокнотів

\vspace{0.5cm}

\task{56}{У Ольги на картці 450 \textit{грн}. Вона хоче купити 5 олівців по 12 \textit{грн} і маркери по 48 \textit{грн}. Яку \textit{максимальну} кількість маркерів може купити Ольга? \gen}
\answerTable{6}{7}{8}{9}{10}
% Відповідь: В (8). 450 - 5×12 = 390 грн. 390÷48 = 8.125 → 8 маркерів

\vspace{0.5cm}

%======================================================================
% ТИП 17: Банківські операції (подвійне зняття)
% Оригінали: 2024#33
%======================================================================

\begin{center}
{\large\textbf{\color{headerblue}Тип: Послідовні операції}}
\end{center}

\vspace{0.3cm}

% На основі 2024#33: зняв 40% + 500 грн = 50%, залишок = ?
\task{57}{Клієнт банку двічі знімав гроші. Першого разу він зняв 30\,\% від початкової суми, другого разу --- 800 \textit{грн}. Після цього залишилося 50\,\% початкової суми. Скільки грошей залишилося у клієнта? \gen}
\answerTable{2000 \textit{грн}}{4000 \textit{грн}}{3000 \textit{грн}}{2500 \textit{грн}}{1600 \textit{грн}}
% Відповідь: Б (4000). 30% + 800 = 50%, тобто 20% = 800, початкова = 4000, залишок 50% = 2000. Помилка в умові - перевіримо: має залишитись 50%. 30% + 800 = 50%? Ні. Треба: 100% - 30% - 800 = 50%, тобто 20% = 800, X = 4000, залишок = 2000. Але відповідь Б (4000) - це початкова сума. Виправлю: залишилося = 2000 грн.

\vspace{0.5cm}

\task{58}{Клієнт банку двічі знімав гроші. Першого разу він зняв 25\,\% від початкової суми, другого разу --- 600 \textit{грн}. Після цього залишилося 55\,\% початкової суми. Скільки грошей залишилося у клієнта? \gen}
\answerTable{1650 \textit{грн}}{3000 \textit{грн}}{1500 \textit{грн}}{3300 \textit{грн}}{2750 \textit{грн}}
% Відповідь: А (1650). 25% + 600 = 45%, тобто 20% = 600, X = 3000, залишок 55% = 1650

\vspace{0.5cm}

%======================================================================
% ТИП 18: Заповнення ємності
% Оригінал: 2024#32
%======================================================================

\begin{center}
{\large\textbf{\color{headerblue}Тип: Заповнення ємності}}
\end{center}

\vspace{0.3cm}

% На основі 2024#32: 3/5 об'єму = 10 хв, скільки повних за 2 год?
\task{59}{Резервуар заповнюється на $\dfrac{2}{5}$ об'єму за 8 хвилин. Скільки \textit{повних} таких резервуарів можна заповнити за одну годину? \gen}
\answerTable{2}{3}{4}{5}{6}
% Відповідь: Б (3). Повний = 8×5/2 = 20 хв. 60÷20 = 3

\vspace{0.5cm}

\task{60}{Бак заповнюється на $\dfrac{3}{4}$ об'єму за 15 хвилин. Скільки \textit{повних} таких баків можна заповнити за 90 хвилин? \gen}
\answerTable{3}{4}{5}{6}{7}
% Відповідь: Б (4). Повний = 15×4/3 = 20 хв. 90÷20 = 4.5 → 4

\vspace{0.5cm}

\end{document}
