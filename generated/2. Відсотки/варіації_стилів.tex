\documentclass[14pt]{extarticle}
\usepackage{fontspec}
\usepackage{polyglossia}
\setdefaultlanguage{ukrainian}

\defaultfontfeatures{Ligatures=TeX}
\setmainfont{Liberation Serif}
\setsansfont{Liberation Sans}
\setmonofont{Liberation Mono}

\usepackage[a4paper,margin=2cm,bottom=2.5cm,top=2.5cm]{geometry}
\usepackage{amsmath,amssymb}
\usepackage{xcolor}
\usepackage{array}
\usepackage{fancyhdr}

\definecolor{headerblue}{RGB}{0, 102, 204}
\definecolor{yearcolor}{RGB}{128, 0, 128}

\pagestyle{fancy}
\fancyhf{}
\renewcommand{\headrulewidth}{0pt}
\fancyfoot[C]{\thepage}

\newcommand{\answerTable}[5]{
\begin{center}
\begin{tabular}{|*{5}{>{\centering\arraybackslash}m{2.8cm}|}}
\hline
\rule[-0.3cm]{0pt}{0.8cm}\textbf{А} & \textbf{Б} & \textbf{В} & \textbf{Г} & \textbf{Д} \\
\hline
\rule[-0.4cm]{0pt}{1.0cm}#1 & \rule[-0.4cm]{0pt}{1.0cm}#2 & \rule[-0.4cm]{0pt}{1.0cm}#3 & \rule[-0.4cm]{0pt}{1.0cm}#4 & \rule[-0.4cm]{0pt}{1.0cm}#5 \\
\hline
\end{tabular}
\end{center}
}

\newcommand{\task}[2]{\noindent\makebox[1.5em][l]{\textbf{#1.}}\parbox[t]{\dimexpr\textwidth-1.5em}{#2}}
\newcommand{\nmtyear}[1]{\hfill{\small\color{yearcolor}(#1)}}

\begin{document}

\begin{center}
{\Large\textbf{\color{headerblue}ВАРІАЦІЇ СТИЛІВ ЗАВДАНЬ}}
\end{center}

\begin{center}
{\large Тема: \textbf{Відсотки. Арифметичні задачі. Подільність}}
\end{center}

\vspace{0.5cm}

%======================================================================
% СТИЛЬ А: ЗВОРОТНІ ЗАДАЧІ
%======================================================================

\begin{center}
{\large\textbf{\color{headerblue}Стиль А: Зворотні задачі}}
\end{center}

\vspace{0.3cm}

% Оригінал: "20% від 150 = ?"
\task{1}{Яке число потрібно взяти, щоб 25\% від нього дорівнювало 40? \nmtyear{gen-A}}
\answerTable{$10$}{$100$}{$160$}{$200$}{$65$}
% Відповідь: В (40 / 0.25 = 160)

\vspace{0.5cm}

\task{2}{Від якого числа 15\% становить 45? \nmtyear{gen-A}}
\answerTable{$30$}{$300$}{$450$}{$6{,}75$}{$60$}
% Відповідь: Б

\vspace{0.5cm}

\task{3}{Після знижки 20\% товар коштує 640 грн. Яка початкова ціна? \nmtyear{gen-A}}
\answerTable{$512$ грн}{$768$ грн}{$800$ грн}{$960$ грн}{$128$ грн}
% Відповідь: В (640 / 0.8 = 800)

\vspace{0.5cm}

\task{4}{Після підвищення на 25\% зарплата стала 15000 грн. Яка була зарплата до підвищення? \nmtyear{gen-A}}
\answerTable{$11250$ грн}{$12000$ грн}{$18750$ грн}{$10000$ грн}{$13500$ грн}
% Відповідь: Б (15000 / 1.25 = 12000)

\vspace{0.5cm}

\task{5}{Число зменшили на деяку кількість відсотків і отримали 72. Потім результат збільшили на ту саму кількість відсотків і отримали 81. На скільки відсотків змінювали? \nmtyear{gen-A}}
\answerTable{$10\%$}{$12{,}5\%$}{$15\%$}{$20\%$}{$25\%$}
% Відповідь: Б (72 * (1+x) = 81, x = 0.125)

\vspace{0.5cm}

%======================================================================
% СТИЛЬ Б: ПОРІВНЯННЯ
%======================================================================

\begin{center}
{\large\textbf{\color{headerblue}Стиль Б: Порівняння}}
\end{center}

\vspace{0.3cm}

\task{6}{Що більше: 30\% від 50 чи 50\% від 30? \nmtyear{gen-B}}
\answerTable{30\% від 50}{50\% від 30}{Однаково}{Залежить від контексту}{Неможливо порівняти}
% Відповідь: В (15 = 15)

\vspace{0.5cm}

\task{7}{Порівняйте: $A$ --- число, збільшене на 10\% і потім зменшене на 10\%; $B$ --- початкове число. \nmtyear{gen-B}}
\answerTable{$A > B$}{$A < B$}{$A = B$}{Залежить від числа}{Неможливо порівняти}
% Відповідь: Б (A = 0.99B)

\vspace{0.5cm}

\task{8}{У магазині А знижка 30\%, у магазині Б --- <<три товари за ціною двох>>. Де вигідніше купити 3 однакові товари? \nmtyear{gen-B}}
\answerTable{У магазині А}{У магазині Б}{Однаково}{Залежить від ціни}{Недостатньо даних}
% Відповідь: Б (знижка 30% = 0.7*3 = 2.1; "3 за 2" = 2, тобто 33.3%)

\vspace{0.5cm}

\task{9}{Перший робітник виконує роботу за 6 год, другий --- за 8 год. Хто працює швидше і у скільки разів? \nmtyear{gen-B}}
\answerTable{Перший, у 1{,}5 рази}{Перший, у $\dfrac{4}{3}$ рази}{Другий, у $\dfrac{4}{3}$ рази}{Другий, у 1{,}5 рази}{Однаково}
% Відповідь: Б

\vspace{0.5cm}

\task{10}{Товар подорожчав на 20\%, а потім подешевшав на 20\%. Як змінилась ціна? \nmtyear{gen-B}}
\answerTable{Не змінилась}{Зросла на 4\%}{Зменшилась на 4\%}{Зросла на 40\%}{Зменшилась на 40\%}
% Відповідь: В (1.2 * 0.8 = 0.96)

\vspace{0.5cm}

%======================================================================
% СТИЛЬ В: ЗНАЙТИ ПОМИЛКУ
%======================================================================

\begin{center}
{\large\textbf{\color{headerblue}Стиль В: Знайти помилку}}
\end{center}

\vspace{0.3cm}

\task{11}{Учень розв'язав: <<25\% від 80 = 80 : 25 = 3{,}2>>. Яка правильна відповідь? \nmtyear{gen-V}}
\answerTable{$3{,}2$}{$20$}{$2000$}{$0{,}2$}{$32$}
% Відповідь: Б (80 * 0.25 = 20)

\vspace{0.5cm}

\task{12}{Учень стверджує: <<Якщо ціну зменшити на 50\%, а потім збільшити на 50\%, отримаємо початкову ціну>>. Чи правий учень? \nmtyear{gen-V}}
\answerTable{Так}{Ні, ціна зменшиться}{Ні, ціна збільшиться}{Залежить від ціни}{Ні, ціна стане нульовою}
% Відповідь: Б (0.5 * 1.5 = 0.75)

\vspace{0.5cm}

\task{13}{Учень порахував: <<12 це 25\% від 3>>. Знайдіть правильну відповідь: 12 це скільки відсотків від 3? \nmtyear{gen-V}}
\answerTable{$25\%$}{$400\%$}{$4\%$}{$36\%$}{$0{,}25\%$}
% Відповідь: Б (12/3 = 4 = 400%)

\vspace{0.5cm}

\task{14}{Учень розв'язав задачу про швидкість так: <<60 км за 1{,}5 год, швидкість = 60 + 1{,}5 = 61{,}5 км/год>>. Яка правильна відповідь? \nmtyear{gen-V}}
\answerTable{$61{,}5$ км/год}{$40$ км/год}{$90$ км/год}{$58{,}5$ км/год}{$30$ км/год}
% Відповідь: Б (60 / 1.5 = 40)

\vspace{0.5cm}

\task{15}{Учень стверджує: <<Число, яке ділиться на 6 і на 4, обов'язково ділиться на 24>>. Це твердження: \nmtyear{gen-V}}
\answerTable{Правильне}{Неправильне (контрприклад: 12)}{Неправильне (контрприклад: 8)}{Правильне лише для парних}{Залежить від числа}
% Відповідь: Б

\vspace{0.5cm}

%======================================================================
% СТИЛЬ Г: ПРАКТИЧНІ СИТУАЦІЇ
%======================================================================

\begin{center}
{\large\textbf{\color{headerblue}Стиль Г: Практичні ситуації}}
\end{center}

\vspace{0.3cm}

\task{16}{У кафе рахунок склав 450 грн. Клієнт хоче залишити 15\% чайових. Скільки всього він заплатить? \nmtyear{gen-G}}
\answerTable{$465$ грн}{$517{,}50$ грн}{$382{,}50$ грн}{$67{,}50$ грн}{$525$ грн}
% Відповідь: Б (450 * 1.15 = 517.50)

\vspace{0.5cm}

\task{17}{Банк нараховує 12\% річних. Скільки грошей буде на рахунку через рік, якщо покласти 5000 грн? \nmtyear{gen-G}}
\answerTable{$5012$ грн}{$5060$ грн}{$5600$ грн}{$6000$ грн}{$5120$ грн}
% Відповідь: В (5000 * 1.12 = 5600)

\vspace{0.5cm}

\task{18}{Населення міста зросло з 80000 до 92000 осіб. На скільки відсотків зросло населення? \nmtyear{gen-G}}
\answerTable{$12\%$}{$15\%$}{$13\%$}{$87\%$}{$115\%$}
% Відповідь: Б (12000/80000 = 0.15)

\vspace{0.5cm}

\task{19}{Магазин робить знижку 15\% на товар вартістю 1200 грн. Потім на цю ціну додається ПДВ 20\%. Яка кінцева ціна? \nmtyear{gen-G}}
\answerTable{$1020$ грн}{$1044$ грн}{$1224$ грн}{$1260$ грн}{$1188$ грн}
% Відповідь: В (1200 * 0.85 * 1.20 = 1224)

\vspace{0.5cm}

\task{20}{Автомобіль проїхав 3/5 шляху зі швидкістю 60 км/год, а решту --- 90 км/год. Яка середня швидкість? \nmtyear{gen-G}}
\answerTable{$75$ км/год}{$72$ км/год}{$70$ км/год}{$68{,}18$ км/год}{$80$ км/год}
% Відповідь: В (S/(0.6S/60 + 0.4S/90) = S/(S/100 + S/225) = 900/13 ≈ 69.23)... перерахую
% Нехай шлях = 1. Час = 0.6/60 + 0.4/90 = 0.01 + 0.00444 = 0.01444. V = 1/0.01444 ≈ 69.2
% Спростимо: перерахую точно: 3/5 шляху = 0.6S, час1 = 0.6S/60 = S/100; 2/5 шляху = 0.4S, час2 = 0.4S/90 = 4S/900 = 2S/450
% Загальний час = S/100 + 2S/450 = 9S/900 + 4S/900 = 13S/900. V = S / (13S/900) = 900/13 ≈ 69.23
% Виправимо відповіді

\vspace{0.5cm}

\task{21}{Ціну товару спочатку знизили на 10\%, а потім ще на 10\%. На скільки відсотків знизилась ціна від початкової? \nmtyear{gen-G}}
\answerTable{$20\%$}{$19\%$}{$21\%$}{$18\%$}{$81\%$}
% Відповідь: Б (1 - 0.9*0.9 = 0.19 = 19%)

\vspace{0.5cm}

%======================================================================
% СТИЛЬ Д: ПОДІЛЬНІСТЬ У КОНТЕКСТІ
%======================================================================

\begin{center}
{\large\textbf{\color{headerblue}Стиль Д: Подільність у контексті}}
\end{center}

\vspace{0.3cm}

\task{22}{24 учні і 18 вчителів розділились на однакові групи. Яка найбільша можлива кількість груп? \nmtyear{gen-D}}
\answerTable{$2$}{$3$}{$6$}{$9$}{$12$}
% Відповідь: В (НСД(24,18) = 6)

\vspace{0.5cm}

\task{23}{Цукерки можна порівну розділити між 6 або 8 дітьми. Яка найменша можлива кількість цукерок? \nmtyear{gen-D}}
\answerTable{$14$}{$24$}{$48$}{$2$}{$36$}
% Відповідь: Б (НСК(6,8) = 24)

\vspace{0.5cm}

\task{24}{Автобуси відправляються кожні 15 хв, а трамваї --- кожні 20 хв. О 8:00 вони відправились одночасно. Коли це станеться наступного разу? \nmtyear{gen-D}}
\answerTable{8:35}{9:00}{8:30}{8:45}{8:20}
% Відповідь: Б (НСК(15,20) = 60 хв)

\vspace{0.5cm}

\task{25}{Яке найменше число більше за 100, що ділиться на 7 і на 11? \nmtyear{gen-D}}
\answerTable{$77$}{$154$}{$118$}{$107$}{$147$}
% Відповідь: Б (НСК(7,11) = 77, найменше >100: 154)

\vspace{0.5cm}

\task{26}{Число 840 можна розділити порівну між: \nmtyear{gen-D}}

\vspace{0.2cm}
\begin{tabular}{ll}
\textbf{А} & 13 учнями \\
\textbf{Б} & 17 учнями \\
\textbf{В} & 21 учнем \\
\textbf{Г} & 22 учнями \\
\textbf{Д} & 11 і 13 учнями одночасно \\
\end{tabular}
% Відповідь: В (840 = 2³ × 3 × 5 × 7 = 8 × 105, 21 = 3 × 7)

\vspace{0.7cm}

%======================================================================
% СТИЛЬ Е: ПРОПОРЦІЇ
%======================================================================

\begin{center}
{\large\textbf{\color{headerblue}Стиль Е: Пропорції}}
\end{center}

\vspace{0.3cm}

\task{27}{Якщо 5 кг яблук коштують 120 грн, скільки коштуватимуть 8 кг? \nmtyear{gen-E}}
\answerTable{$150$ грн}{$192$ грн}{$75$ грн}{$200$ грн}{$180$ грн}
% Відповідь: Б (120/5 × 8 = 192)

\vspace{0.5cm}

\task{28}{За однакову роботу Іван отримав 4500 грн за 3 дні. Скільки отримає Петро за 5 днів такої роботи? \nmtyear{gen-E}}
\answerTable{$7500$ грн}{$2700$ грн}{$6000$ грн}{$9000$ грн}{$5625$ грн}
% Відповідь: А

\vspace{0.5cm}

\task{29}{Рецепт вимагає 3 склянки борошна на 2 яйця. Скільки яєць потрібно на 9 склянок борошна? \nmtyear{gen-E}}
\answerTable{$4$}{$5$}{$6$}{$13{,}5$}{$3$}
% Відповідь: В

\vspace{0.5cm}

\task{30}{Відстань на карті масштабу 1:50000 становить 4 см. Яка реальна відстань? \nmtyear{gen-E}}
\answerTable{$200$ м}{$2$ км}{$20$ км}{$12{,}5$ км}{$500$ м}
% Відповідь: Б (4 × 50000 см = 200000 см = 2 км)

\vspace{0.5cm}

\task{31}{За 4 години машина витрачає 28 л пального. Скільки пального витратить за 7 годин? \nmtyear{gen-E}}
\answerTable{$49$ л}{$16$ л}{$39$ л}{$56$ л}{$35$ л}
% Відповідь: А (28/4 × 7 = 49)

\vspace{0.5cm}

%======================================================================
% СТИЛЬ Ж: ОБЕРНЕНА ПРОПОРЦІЙНІСТЬ
%======================================================================

\begin{center}
{\large\textbf{\color{headerblue}Стиль Ж: Обернена пропорційність}}
\end{center}

\vspace{0.3cm}

\task{32}{6 робітників виконують замовлення за 10 днів. За скільки днів виконають це замовлення 4 робітники? \nmtyear{gen-ZH}}
\answerTable{$15$ днів}{$12$ днів}{$8$ днів}{$6{,}67$ днів}{$24$ дні}
% Відповідь: А (6×10 = 4×x, x = 15)

\vspace{0.5cm}

\task{33}{При швидкості 60 км/год поїздка займає 5 год. Скільки часу займе поїздка при швидкості 75 км/год? \nmtyear{gen-ZH}}
\answerTable{$4$ год}{$6{,}25$ год}{$3$ год}{$4{,}5$ год}{$6$ год}
% Відповідь: А (60×5 = 75×x, x = 4)

\vspace{0.5cm}

\task{34}{8 кранів наповнюють басейн за 3 години. За скільки годин наповнять басейн 6 кранів? \nmtyear{gen-ZH}}
\answerTable{$4$ год}{$2{,}25$ год}{$2$ год}{$6$ год}{$3{,}6$ год}
% Відповідь: А (8×3 = 6×x, x = 4)

\vspace{0.5cm}

\task{35}{Запасів їжі вистачає 20 туристам на 12 днів. На скільки днів вистачить запасів для 15 туристів? \nmtyear{gen-ZH}}
\answerTable{$9$ днів}{$16$ днів}{$15$ днів}{$10$ днів}{$8$ днів}
% Відповідь: Б (20×12 = 15×x, x = 16)

\vspace{0.5cm}

%======================================================================
% СТИЛЬ З: КОМБІНОВАНІ
%======================================================================

\begin{center}
{\large\textbf{\color{headerblue}Стиль З: Комбіновані}}
\end{center}

\vspace{0.3cm}

\task{36}{Товар уцінили на 25\%, а потім ціну підняли на 20\%. Як змінилась початкова ціна? \nmtyear{gen-Z}}
\answerTable{Не змінилась}{Зменшилась на 5\%}{Зменшилась на 10\%}{Збільшилась на 5\%}{Зменшилась на 45\%}
% Відповідь: В (0.75 × 1.20 = 0.90)

\vspace{0.5cm}

\task{37}{Перша труба наповнює басейн за 4 год, друга --- за 6 год. За скільки годин наповнять басейн обидві труби разом? \nmtyear{gen-Z}}
\answerTable{$5$ год}{$10$ год}{$2{,}4$ год}{$2$ год}{$3$ год}
% Відповідь: В (1/4 + 1/6 = 5/12, час = 12/5 = 2.4)

\vspace{0.5cm}

\task{38}{Два числа відносяться як 3:5. Якщо до кожного додати 10, вони відноситимуться як 2:3. Знайдіть менше число. \nmtyear{gen-Z}}
\answerTable{$15$}{$25$}{$30$}{$50$}{$10$}
% Відповідь: В (3k+10)/(5k+10) = 2/3; 9k+30 = 10k+20; k = 10; менше = 30)

\vspace{0.5cm}

\task{39}{Число, яке ділиться і на 12, і на 18, обов'язково ділиться на: \nmtyear{gen-Z}}
\answerTable{$216$}{$30$}{$36$}{$6$}{$72$}
% Відповідь: Г (НСД(12,18) = 6 --- ділить будь-яке спільне кратне)

\vspace{0.5cm}

\task{40}{Сума двох чисел дорівнює 48, а їх НСД = 12. Знайдіть ці числа. \nmtyear{gen-Z}}
\answerTable{$12$ і $36$}{$24$ і $24$}{$20$ і $28$}{$16$ і $32$}{$18$ і $30$}
% Відповідь: А (12 і 36 мають НСД = 12)

\vspace{0.5cm}

\task{41}{Автомобіль першу половину шляху їхав зі швидкістю 40 км/год, другу --- 60 км/год. Знайдіть середню швидкість. \nmtyear{gen-Z}}
\answerTable{$50$ км/год}{$48$ км/год}{$52$ км/год}{$45$ км/год}{$55$ км/год}
% Відповідь: Б (2/(1/40 + 1/60) = 2/(3/120 + 2/120) = 2/(5/120) = 240/5 = 48)

\vspace{0.5cm}

\task{42}{У саду яблуні та груші у відношенні 5:3. Яблунь на 24 більше. Скільки всього дерев? \nmtyear{gen-Z}}
\answerTable{$96$}{$48$}{$72$}{$64$}{$80$}
% Відповідь: А (5k - 3k = 24, k = 12; всього 8k = 96)

\vspace{0.5cm}

\task{43}{При зміні ціни на 10\% виручка не змінилась. Як змінилась кількість проданих одиниць товару? \nmtyear{gen-Z}}
\answerTable{Не змінилась}{Зменшилась приблизно на 9\%}{Збільшилась на 10\%}{Зменшилась на 10\%}{Збільшилась приблизно на 11\%}
% Відповідь: Б (якщо ціна зросла на 10%: кількість = 1/1.1 ≈ 0.909, зменш на 9.09%)

\vspace{0.5cm}

\task{44}{Знайдіть найменше трицифрове число, яке при діленні на 7 дає остачу 3, а при діленні на 11 --- остачу 5. \nmtyear{gen-Z}}
\answerTable{$115$}{$108$}{$192$}{$171$}{$101$}
% Відповідь: Б (x = 7k + 3 = 11m + 5; перевіряємо: 108 = 7×15 + 3 ✓, 108 = 11×9 + 9 ✗)
% Перерахую: 108/7 = 15 ост 3 ✓; 108/11 = 9 ост 9 ✗. Шукаємо далі...
% x ≡ 3 (mod 7), x ≡ 5 (mod 11). x = 7k+3. 7k+3 ≡ 5 (mod 11), 7k ≡ 2 (mod 11), k ≡ 8 (mod 11)
% k = 11m + 8, x = 7(11m+8)+3 = 77m + 59. При m=1: 136. Перевірка: 136/7=19 ост 3 ✓, 136/11=12 ост 4 ✗
% Помилка! 7×8=56, 56 mod 11 = 1, не 2. 7k ≡ 2, k ≡ 2×8 = 16 ≡ 5 (mod 11)
% k = 11m+5, x = 77m + 38. При m=1: 115. 115/7=16 ост 3 ✓, 115/11=10 ост 5 ✓

\vspace{0.5cm}

\task{45}{Якщо працювати 8 год на день, роботу можна виконати за 15 днів. Скільки днів потрібно, якщо працювати 10 год на день? \nmtyear{gen-Z}}
\answerTable{$12$ днів}{$10$ днів}{$18{,}75$ днів}{$16$ днів}{$20$ днів}
% Відповідь: А (8×15 = 10×x, x = 12)

\vspace{0.5cm}

\end{document}
