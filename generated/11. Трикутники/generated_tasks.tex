\documentclass[14pt]{extarticle}
\usepackage{fontspec}
\usepackage{polyglossia}
\setdefaultlanguage{ukrainian}

\defaultfontfeatures{Ligatures=TeX}
\setmainfont{Liberation Serif}
\setsansfont{Liberation Sans}
\setmonofont{Liberation Mono}

\usepackage[a4paper,margin=2cm,bottom=2.5cm,top=2.5cm]{geometry}
\usepackage{amsmath,amssymb}
\usepackage{enumitem}
\usepackage{tikz}
\usepackage{pgfplots}
\pgfplotsset{compat=1.16}
\usetikzlibrary{calc,patterns,angles,quotes}
\usepackage{xcolor}
\usepackage{array}
\usepackage{fancyhdr}
\usepackage{multicol}

% Кольори
\definecolor{headerblue}{RGB}{0, 102, 204}
\definecolor{yearcolor}{RGB}{128, 0, 128}

\pagestyle{fancy}
\fancyhf{}
\renewcommand{\headrulewidth}{0pt}
\fancyfoot[C]{\thepage}

\setlength{\headheight}{15pt}
\setlength{\headsep}{10pt}
\setlength{\footskip}{25pt}

\widowpenalty=10000
\clubpenalty=10000

% === КОМАНДИ ===

\newcommand{\answerTable}[5]{
\begin{center}
\begin{tabular}{|*{5}{>{\centering\arraybackslash}m{2.8cm}|}}
\hline
\rule[-0.3cm]{0pt}{0.8cm}\textbf{А} & \textbf{Б} & \textbf{В} & \textbf{Г} & \textbf{Д} \\
\hline
\rule[-0.4cm]{0pt}{1.0cm}#1 & \rule[-0.4cm]{0pt}{1.0cm}#2 & \rule[-0.4cm]{0pt}{1.0cm}#3 & \rule[-0.4cm]{0pt}{1.0cm}#4 & \rule[-0.4cm]{0pt}{1.0cm}#5 \\
\hline
\end{tabular}
\end{center}
}

\newcommand{\answerTableSmall}[5]{
\begin{tabular}{|*{5}{>{\centering\arraybackslash}m{1.1cm}|}}
\hline
\rule[-0.2cm]{0pt}{0.6cm}\textbf{А} & \textbf{Б} & \textbf{В} & \textbf{Г} & \textbf{Д} \\
\hline
\rule[-0.3cm]{0pt}{0.8cm}#1 & #2 & #3 & #4 & #5 \\
\hline
\end{tabular}
}

\newcommand{\matchTable}{
\begin{tabular}{|>{\centering\arraybackslash}p{0.3cm}|*{5}{>{\centering\arraybackslash}p{0.3cm}|}}
\hline
& \textbf{А} & \textbf{Б} & \textbf{В} & \textbf{Г} & \textbf{Д} \\
\hline
\textbf{1} & \rule{0pt}{0.3cm} & & & & \\
\hline
\textbf{2} & \rule{0pt}{0.3cm} & & & & \\
\hline
\textbf{3} & \rule{0pt}{0.3cm} & & & & \\
\hline
\end{tabular}
}

\newcommand{\task}[2]{\noindent\makebox[1.5em][l]{\textbf{#1.}}\parbox[t]{\dimexpr\textwidth-1.5em}{#2}}

\begin{document}

\begin{center}
{\Large\textbf{\color{headerblue}ЗГЕНЕРОВАНІ ЗАВДАННЯ}}
\end{center}

\begin{center}
{\large Тема 11: Трикутники}
\end{center}

\vspace{0.5cm}

%======================================================================
% БЛОК 1: Кути та сторони трикутника
%======================================================================

\section*{Блок 1: Кути та сторони трикутника}

% Завдання 1
\task{1}{Трикутник $ABC$ є рівнобедреним ($AB = BC$), $\angle ABC = 40°$ (див. рисунок). Визначте градусну міру кута при основі $AC$.}

\vspace{0.3cm}
\begin{minipage}{0.42\textwidth}
\answerTableSmall{$70°$}{$40°$}{$80°$}{$60°$}{$50°$}
\end{minipage}
\hfill
\begin{minipage}{0.52\textwidth}
\begin{flushright}
\begin{tikzpicture}[scale=0.7]
    \coordinate (A) at (0,0);
    \coordinate (B) at (2,2.5);
    \coordinate (C) at (4,0);

    \draw[thick] (A) -- (B) -- (C) -- cycle;

    % Риски рівності на AB і BC
    \draw[thick] (0.9,1.35) -- (1.1,1.15);
    \draw[thick] (3.2,1.35) -- (2.9,1.15);

    % Кут 40° при B
    \draw (B) ++(-51:0.6) arc (-51:-129:0.6);
    \node at (2,1.6) {\small $40°$};

    % Кут ? при A
    \draw (A) ++(0:0.5) arc (0:51:0.5);
    \node at (0.7,0.35) {\small $?$};

    \node[below left] at (A) {$A$};
    \node[above] at (B) {$B$};
    \node[below right] at (C) {$C$};
\end{tikzpicture}
\end{flushright}
\end{minipage}

\vspace{0.7cm}

% Завдання 2
\task{2}{Трикутник $ABC$ є рівнобедреним ($AB = BC$), $\angle ABC = 56°$ (див. рисунок). Визначте градусну міру зовнішнього кута при вершині $A$.}

\vspace{0.3cm}
\begin{minipage}{0.42\textwidth}
\answerTableSmall{$118°$}{$124°$}{$62°$}{$56°$}{$134°$}
\end{minipage}
\hfill
\begin{minipage}{0.52\textwidth}
\begin{flushright}
\begin{tikzpicture}[scale=0.7]
    \coordinate (A) at (0,0);
    \coordinate (B) at (2,2.5);
    \coordinate (C) at (4,0);
    \coordinate (M) at (-1.2,0);

    \draw[thick] (A) -- (B) -- (C) -- cycle;
    \draw[thick] (A) -- (M);

    \draw[thick] (0.9,1.35) -- (1.1,1.15);
    \draw[thick] (3.2,1.35) -- (2.9,1.15);

    \draw (B) ++(-51:0.6) arc (-51:-129:0.6);
    \node at (2,1.6) {\small $56°$};

    \draw (A) ++(180:0.5) arc (180:51:0.5);
    \draw (A) ++(180:0.65) arc (180:51:0.65);
    \node at (-0.2,0.85) {\small $?$};

    \node[below left] at (A) {$A$};
    \node[above] at (B) {$B$};
    \node[below right] at (C) {$C$};
\end{tikzpicture}
\end{flushright}
\end{minipage}

\vspace{0.7cm}

% Завдання 3
\task{3}{У трикутнику $ABC$ $\angle A = 45°$, $\angle B = 35°$ (див. рисунок). Визначте градусну міру зовнішнього кута при вершині $C$.}

\vspace{0.3cm}
\begin{minipage}{0.55\textwidth}
\answerTableSmall{$80°$}{$100°$}{$260°$}{$70°$}{$110°$}
\end{minipage}
\hfill
\begin{minipage}{0.4\textwidth}
\begin{flushright}
\begin{tikzpicture}[scale=1]
    \coordinate (A) at (0,0);
    \coordinate (C) at (3,0);
    \coordinate (B) at (2.5,2.5);
    \coordinate (D) at (4,0);

    \draw[thick] (A) -- (B) -- (C) -- cycle;
    \draw[thick] (C) -- (D);

    \pic[draw, angle radius=0.5cm] {angle = C--A--B};
    \node at (0.9,0.3) {\small $45°$};

    \pic[draw, angle radius=0.4cm] {angle = A--B--C};
    \pic[draw, angle radius=0.5cm] {angle = A--B--C};
    \node at (2.3,1.7) {\small $35°$};

    \pic[draw, angle radius=0.35cm] {angle = D--C--B};
    \pic[draw, angle radius=0.45cm] {angle = D--C--B};
    \pic[draw, angle radius=0.55cm] {angle = D--C--B};
    \node at (3.7,0.5) {\small $?$};

    \node[below left] at (A) {$A$};
    \node[below] at (C) {$C$};
    \node[above] at (B) {$B$};
\end{tikzpicture}
\end{flushright}
\end{minipage}

\vspace{0.7cm}

% Завдання 4
\task{4}{Зовнішній кут при вершині $A$ трикутника $ABC$ дорівнює $110°$, $\angle C = 25°$ (див. рисунок). Визначте градусну міру кута $B$.}

\vspace{0.3cm}
\begin{minipage}{0.55\textwidth}
\answerTableSmall{$85°$}{$70°$}{$95°$}{$65°$}{$75°$}
\end{minipage}
\hfill
\begin{minipage}{0.4\textwidth}
\begin{flushright}
\begin{tikzpicture}[scale=1]
    \coordinate (A) at (1.5,0);
    \coordinate (C) at (3.5,0);
    \coordinate (B) at (2.2,2.5);
    \coordinate (D) at (0,0);

    \draw[thick] (A) -- (B) -- (C) -- cycle;
    \draw[thick] (D) -- (A);

    \pic[draw, angle radius=0.5cm] {angle = B--A--D};
    \node at (0.75,0.4) {\small $110°$};

    \pic[draw, angle radius=0.4cm] {angle = B--C--A};
    \pic[draw, angle radius=0.5cm] {angle = B--C--A};
    \node at (2.8,0.4) {\small $25°$};

    \pic[draw, angle radius=0.35cm] {angle = A--B--C};
    \pic[draw, angle radius=0.45cm] {angle = A--B--C};
    \pic[draw, angle radius=0.55cm] {angle = A--B--C};
    \node at (2.2,1.7) {\small $?$};

    \node[below] at (A) {$A$};
    \node[below right] at (C) {$C$};
    \node[above] at (B) {$B$};
\end{tikzpicture}
\end{flushright}
\end{minipage}

\vspace{0.7cm}

% Завдання 5
\task{5}{Які з наведених тверджень є правильними?}

\vspace{0.2cm}
\begin{tabular}{r@{\hspace{0.5em}}p{14cm}}
I. & У рівнобедреному трикутнику кути при основі рівні. \\
II. & Сума кутів будь-якого трикутника дорівнює $180°$. \\
III. & У будь-якому трикутнику принаймні два кути гострі. \\
\end{tabular}

\answerTable{I, II та III}{лише I та II}{лише II та III}{лише I}{лише II}

\vspace{0.5cm}

% Завдання 6
\task{6}{Які з наведених тверджень є правильними?}

\vspace{0.2cm}
\begin{tabular}{r@{\hspace{0.5em}}p{14cm}}
I. & Зовнішній кут трикутника дорівнює сумі двох внутрішніх кутів, не суміжних з ним. \\
II. & Існує трикутник з двома тупими кутами. \\
III. & У рівносторонньому трикутнику всі кути рівні $60°$. \\
\end{tabular}

\answerTable{лише I та III}{лише I та II}{лише II та III}{I, II та III}{лише I}

\vspace{0.5cm}

% Завдання 7
\task{7}{Обчисліть довжину основи рівнобедреного трикутника, якщо його бічна сторона дорівнює 15 см, а периметр --- 46 см.}
\answerTable{16 см}{15 см}{8 см}{31 см}{14 см}

\vspace{0.5cm}

% Завдання 8
\task{8}{Обчисліть бічну сторону рівнобедреного трикутника, якщо його основа дорівнює 10 см, а периметр --- 36 см.}
\answerTable{13 см}{10 см}{26 см}{18 см}{12 см}

\vspace{0.5cm}

%======================================================================
% БЛОК 2: Середня лінія трикутника
%======================================================================

\section*{Блок 2: Середня лінія трикутника}

% Завдання 9
\task{9}{На сторонах $AB$ та $BC$ трикутника $ABC$ вибрано точки $M$ та $N$ відповідно так, що $AM = MB$, $CN = NB$ (див. рисунок). Які з наведених тверджень є правильними?}

\vspace{0.2cm}
\begin{tabular}{r@{\hspace{0.5em}}p{14cm}}
I. & $AC \parallel MN$. \\
II. & Відрізок $MN$ є середньою лінією трикутника $ABC$. \\
III. & $MN = \dfrac{1}{2}AC$. \\
\end{tabular}

\vspace{0.3cm}
\begin{minipage}{0.55\textwidth}
\answerTable{I, II та III}{лише I та II}{лише II та III}{лише I та III}{лише II}
\end{minipage}
\hfill
\begin{minipage}{0.4\textwidth}
\begin{flushright}
\begin{tikzpicture}[scale=0.8]
    \coordinate (A) at (0,0);
    \coordinate (B) at (2,3);
    \coordinate (C) at (4,0);
    \coordinate (M) at (1,1.5);
    \coordinate (N) at (3,1.5);

    \draw[thick] (A) -- (B) -- (C) -- cycle;
    \draw[thick] (M) -- (N);

    % Риски рівності
    \draw[thick] (0.4,0.7) -- (0.6,0.8);
    \draw[thick] (1.4,2.2) -- (1.6,2.3);
    \draw[thick] (2.4,2.2) -- (2.6,2.3);
    \draw[thick] (3.4,0.7) -- (3.6,0.8);

    \node[below left] at (A) {$A$};
    \node[above] at (B) {$B$};
    \node[below right] at (C) {$C$};
    \node[left] at (M) {$M$};
    \node[right] at (N) {$N$};
\end{tikzpicture}
\end{flushright}
\end{minipage}

\vspace{0.7cm}

% Завдання 10
\task{10}{У трикутнику $ABC$ точки $K$ і $L$ --- середини сторін $AB$ і $AC$ відповідно. $KL = 7$ см. Знайдіть сторону $BC$.}

\vspace{0.3cm}
\begin{minipage}{0.55\textwidth}
\answerTableSmall{14 см}{7 см}{3{,}5 см}{21 см}{28 см}
\end{minipage}
\hfill
\begin{minipage}{0.4\textwidth}
\begin{flushright}
\begin{tikzpicture}[scale=0.7]
    \coordinate (A) at (2,3);
    \coordinate (B) at (0,0);
    \coordinate (C) at (4,0);
    \coordinate (K) at (1,1.5);
    \coordinate (L) at (3,1.5);

    \draw[thick] (A) -- (B) -- (C) -- cycle;
    \draw[thick] (K) -- (L);

    \draw[thick] (0.4,0.7) -- (0.6,0.8);
    \draw[thick] (1.4,2.2) -- (1.6,2.3);
    \draw[thick] (2.4,2.2) -- (2.6,2.3);
    \draw[thick] (3.4,0.7) -- (3.6,0.8);

    \node at (2,1.2) {\small 7 см};

    \node[above] at (A) {$A$};
    \node[below left] at (B) {$B$};
    \node[below right] at (C) {$C$};
    \node[left] at (K) {$K$};
    \node[right] at (L) {$L$};
\end{tikzpicture}
\end{flushright}
\end{minipage}

\vspace{0.7cm}

% Завдання 11
\task{11}{У трикутнику $ABC$ $BC = 18$ см. $M$ і $N$ --- середини сторін $AB$ і $AC$. Знайдіть середню лінію $MN$.}

\vspace{0.3cm}
\begin{minipage}{0.55\textwidth}
\answerTableSmall{9 см}{18 см}{36 см}{6 см}{12 см}
\end{minipage}
\hfill
\begin{minipage}{0.4\textwidth}
\begin{flushright}
\begin{tikzpicture}[scale=0.7]
    \coordinate (A) at (2,3);
    \coordinate (B) at (0,0);
    \coordinate (C) at (4,0);
    \coordinate (M) at (1,1.5);
    \coordinate (N) at (3,1.5);

    \draw[thick] (A) -- (B) -- (C) -- cycle;
    \draw[thick] (M) -- (N);

    \node at (2,-0.4) {\small 18 см};

    \node[above] at (A) {$A$};
    \node[below left] at (B) {$B$};
    \node[below right] at (C) {$C$};
    \node[left] at (M) {$M$};
    \node[right] at (N) {$N$};
\end{tikzpicture}
\end{flushright}
\end{minipage}

\vspace{0.7cm}

% Завдання 12
\task{12}{Периметр трикутника $ABC$ дорівнює 42 см. Знайдіть периметр трикутника, утвореного середніми лініями.}
\answerTable{21 см}{42 см}{84 см}{14 см}{28 см}

\vspace{0.5cm}

%======================================================================
% БЛОК 3: Медіани та висоти
%======================================================================

\section*{Блок 3: Медіани та висоти}

% Завдання 13
\task{13}{Задано довільний трикутник $ABC$, у якому $AM$ --- медіана. Які з наведених тверджень є правильними?}

\vspace{0.2cm}
\begin{tabular}{r@{\hspace{0.5em}}p{14cm}}
I. & $BM = MC$. \\
II. & $\angle BAM = \angle MAC$. \\
III. & Площа трикутника $ABM$ дорівнює площі трикутника $AMC$. \\
\end{tabular}

\answerTable{лише I та III}{лише I}{лише III}{лише II та III}{I, II та III}

\vspace{0.5cm}

% Завдання 14
\task{14}{Які з наведених тверджень є правильними?}

\vspace{0.2cm}
\begin{tabular}{r@{\hspace{0.5em}}p{14cm}}
I. & Медіана трикутника з'єднує його вершину із серединою протилежної сторони. \\
II. & Точка перетину медіан трикутника є центром кола, вписаного в цей трикутник. \\
III. & У прямокутному трикутнику медіана до гіпотенузи дорівнює половині гіпотенузи. \\
\end{tabular}

\answerTable{лише I та III}{лише I}{лише III}{лише I та II}{I, II та III}

\vspace{0.5cm}

% Завдання 15
\task{15}{Медіани трикутника перетинаються в точці $O$. Медіана $AM = 18$ см. Знайдіть $AO$.}

\vspace{0.3cm}
\begin{minipage}{0.55\textwidth}
\answerTableSmall{12 см}{6 см}{9 см}{18 см}{24 см}
\end{minipage}
\hfill
\begin{minipage}{0.4\textwidth}
\begin{flushright}
\begin{tikzpicture}[scale=0.65]
    \coordinate (A) at (0,0);
    \coordinate (B) at (4,0);
    \coordinate (C) at (2.5,3);
    \coordinate (M) at (2,0);
    \coordinate (O) at (2.17,1);

    \draw[thick] (A) -- (B) -- (C) -- cycle;
    \draw[thick] (A) -- (M);
    \draw[thick] (C) -- (M);

    \fill (O) circle (2pt);

    \node at (1,0.7) {\small 18 см};

    \node[below left] at (A) {$A$};
    \node[below right] at (B) {$B$};
    \node[above] at (C) {$C$};
    \node[below] at (M) {$M$};
    \node[right] at (O) {$O$};
\end{tikzpicture}
\end{flushright}
\end{minipage}

\vspace{0.7cm}

% Завдання 16
\task{16}{Які з наведених тверджень є правильними?}

\vspace{0.2cm}
\begin{tabular}{r@{\hspace{0.5em}}p{14cm}}
I. & Одна з висот рівнобедреного трикутника ділить його на два рівних трикутники. \\
II. & Дві висоти тупокутного трикутника лежать поза його межами. \\
III. & Висота, проведена до основи рівнобедреного трикутника, є також медіаною. \\
\end{tabular}

\answerTable{лише I та III}{лише I}{лише II}{лише I та II}{I, II та III}

\vspace{0.5cm}

% Завдання 17
\task{17}{Гіпотенуза прямокутного трикутника дорівнює 26 см. Знайдіть медіану, проведену до гіпотенузи.}
\answerTable{13 см}{26 см}{52 см}{6{,}5 см}{$13\sqrt{2}$ см}

\vspace{0.5cm}

%======================================================================
% БЛОК 4: Вписане та описане кола
%======================================================================

\section*{Блок 4: Вписане та описане кола}

% Завдання 18
\task{18}{Рівнобедрений трикутник $ABC$ ($AB = BC$) вписано в коло (див. рисунок). Визначте градусну міру меншої дуги $AB$, якщо $\angle ABC = 30°$.}

\vspace{0.3cm}
\begin{minipage}{0.55\textwidth}
\answerTableSmall{$150°$}{$30°$}{$75°$}{$165°$}{$60°$}
\end{minipage}
\hfill
\begin{minipage}{0.4\textwidth}
\begin{flushright}
\begin{tikzpicture}[scale=0.9]
    \draw[thick] (0,0) circle (1.5);

    \coordinate (B) at (90:1.5);
    \coordinate (A) at (225:1.5);
    \coordinate (C) at (315:1.5);

    \draw[thick] (A) -- (B) -- (C) -- (A);

    \draw[thick] ($(A)!0.45!(B)$) -- ++(0.08,0.12) -- ++(-0.16,0);
    \draw[thick] ($(A)!0.55!(B)$) -- ++(0.08,0.12) -- ++(-0.16,0);
    \draw[thick] ($(C)!0.45!(B)$) -- ++(-0.08,0.12) -- ++(0.16,0);
    \draw[thick] ($(C)!0.55!(B)$) -- ++(-0.08,0.12) -- ++(0.16,0);

    \pic[draw, angle radius=0.5cm] {angle = A--B--C};
    \node at ($(B)+(0,-0.9)$) {\small $30°$};

    \node[above] at (B) {$B$};
    \node[below left] at (A) {$A$};
    \node[below right] at (C) {$C$};

    \fill (A) circle (1.5pt);
    \fill (B) circle (1.5pt);
    \fill (C) circle (1.5pt);
\end{tikzpicture}
\end{flushright}
\end{minipage}

\vspace{0.7cm}

% Завдання 19
\task{19}{Які з наведених тверджень є правильними?}

\vspace{0.2cm}
\begin{tabular}{r@{\hspace{0.5em}}p{14cm}}
I. & Існує лише одна точка на площині трикутника, яка рівновіддалена від його вершин. \\
II. & Медіана трикутника ділить його на два інші трикутники з однаковою площею. \\
III. & Центр описаного кола завжди лежить всередині трикутника. \\
\end{tabular}

\answerTable{лише I та II}{лише I}{лише II}{лише II та III}{I, II та III}

\vspace{0.5cm}

% Завдання 20
\task{20}{Катети прямокутного трикутника дорівнюють 6 і 8 см. Знайдіть радіус описаного кола.}
\answerTable{5 см}{10 см}{7 см}{14 см}{$5\sqrt{2}$ см}

\vspace{0.5cm}

% Завдання 21
\task{21}{У довільний трикутник $ABC$ вписано коло з центром у точці $O$, точки $K$, $L$, $M$ --- точки дотику (див. рисунок). Які з наведених тверджень є правильними?}

\vspace{0.2cm}
\begin{tabular}{r@{\hspace{0.5em}}p{10cm}}
I. & Трикутник $AOK$ є прямокутним. \\
II. & Трикутник $BKL$ є рівнобедреним. \\
III. & Трикутники $MOC$ і $LOC$ є рівними. \\
\end{tabular}

\vspace{0.3cm}
\begin{minipage}{0.55\textwidth}
\answerTable{I, II та III}{лише I та II}{лише II та III}{лише I}{лише II}
\end{minipage}
\hfill
\begin{minipage}{0.4\textwidth}
\begin{flushright}
\begin{tikzpicture}[scale=0.8]
    \coordinate (A) at (0,0);
    \coordinate (B) at (4,0);
    \coordinate (C) at (2.5,3);

    \draw[thick] (A) -- (B) -- (C) -- cycle;

    % Вписане коло (приблизно)
    \coordinate (O) at (2.1,0.9);
    \draw[thick] (O) circle (0.9);

    % Точки дотику
    \coordinate (K) at (2.1,0);
    \coordinate (L) at (3.35,1.35);
    \coordinate (M) at (1.05,1.35);

    \fill (K) circle (1.5pt);
    \fill (L) circle (1.5pt);
    \fill (M) circle (1.5pt);
    \fill (O) circle (1.5pt);

    \node[below left] at (A) {$A$};
    \node[below right] at (B) {$B$};
    \node[above] at (C) {$C$};
    \node[below] at (K) {$K$};
    \node[right] at (L) {$L$};
    \node[left] at (M) {$M$};
    \node[above right] at (O) {$O$};
\end{tikzpicture}
\end{flushright}
\end{minipage}

\vspace{0.7cm}

%======================================================================
% БЛОК 5: Задачі на відповідність
%======================================================================

\section*{Блок 5: Задачі на відповідність}

% Завдання 22
\task{22}{Периметр рівнобедреного трикутника (див. рисунок) дорівнює 40 см. $AB = BC = 12$ см. До кожного відрізка (1--3) доберіть його довжину (А--Д).}

\vspace{0.3cm}
\begin{minipage}{0.55\textwidth}
\begin{tabular}{@{}l@{\hspace{1.5cm}}l@{}}
\textit{Відрізок} & \textit{Довжина відрізка} \\[0.2cm]
\textbf{1} \quad $AC$ & \textbf{А} \quad $6$ см \\[0.15cm]
\textbf{2} \quad висота з вершини $B$ & \textbf{Б} \quad $8$ см \\[0.15cm]
\textbf{3} \quad медіана з вершини $A$ & \textbf{В} \quad $\sqrt{128}$ см \\[0.15cm]
 & \textbf{Г} \quad $16$ см \\[0.15cm]
 & \textbf{Д} \quad $4\sqrt{11}$ см \\
\end{tabular}

\vspace{0.3cm}
\hfill\matchTable
\end{minipage}
\hfill
\begin{minipage}{0.4\textwidth}
\begin{flushright}
\begin{tikzpicture}[scale=1.4]
    \coordinate (A) at (0,0);
    \coordinate (C) at (2.67,0);
    \coordinate (B) at (1.335,1.76);

    \draw[thick] (A) -- (B) -- (C) -- cycle;

    \draw (0.55,0.95) -- (0.7,0.8);
    \draw (0.6,0.9) -- (0.75,0.75);
    \draw (2.0,0.95) -- (1.85,0.8);
    \draw (1.95,0.9) -- (1.8,0.75);

    \node[below left] at (A) {$A$};
    \node[below right] at (C) {$C$};
    \node[above] at (B) {$B$};
\end{tikzpicture}
\end{flushright}
\end{minipage}

\vspace{0.7cm}

% Завдання 23
\task{23}{На рисунку зображено прямокутний трикутник $ABC$ ($\angle C = 90°$). Точка $M$ --- середина $CB = 12$ см. Радіус кола, описаного навколо трикутника $ABC$, дорівнює 10 см. До кожного відрізка (1--3) доберіть його довжину (А--Д).}

\vspace{0.3cm}
\begin{minipage}{0.3\textwidth}
\textit{Відрізок}

\vspace{0.2cm}
\textbf{1} \quad $AC$

\vspace{0.15cm}
\textbf{2} \quad найбільша середня

\quad\quad лінія трикутника $ABC$

\vspace{0.15cm}
\textbf{3} \quad $AM$
\end{minipage}
\hfill
\begin{minipage}{0.3\textwidth}
\textit{Довжина відрізка}

\vspace{0.2cm}
\textbf{А} \quad $10$ см

\vspace{0.15cm}
\textbf{Б} \quad $16$ см

\vspace{0.15cm}
\textbf{В} \quad $6\sqrt{5}$ см

\vspace{0.15cm}
\textbf{Г} \quad $4\sqrt{13}$ см

\vspace{0.15cm}
\textbf{Д} \quad $2\sqrt{52}$ см

\vspace{0.3cm}
\matchTable
\end{minipage}
\hfill
\begin{minipage}{0.3\textwidth}
\begin{flushright}
\begin{tikzpicture}[scale=0.5]
    \coordinate (A) at (0,4);
    \coordinate (B) at (4,0);
    \coordinate (C) at (0,0);
    \coordinate (M) at (2,0);

    \draw[thick] (A) -- (B) -- (C) -- cycle;
    \draw (0,0.4) -- (0.4,0.4) -- (0.4,0);

    \draw[thick] (0.9,-0.15) -- (0.9,0.15);
    \draw[thick] (A) -- (M);
    \draw[thick] (1.05,-0.15) -- (1.05,0.15);
    \draw[thick] (2.95,-0.15) -- (2.95,0.15);
    \draw[thick] (3.1,-0.15) -- (3.1,0.15);

    \node[above left] at (A) {$A$};
    \node[right] at (B) {$B$};
    \node[below left] at (C) {$C$};
    \node[below] at (M) {$M$};
\end{tikzpicture}
\end{flushright}
\end{minipage}

\vspace{0.7cm}

% Завдання 24
\task{24}{У рівносторонньому трикутнику $ABC$ $AB = 18$ см. З точки $K$, що є серединою сторони $AB$, на сторону $AC$ опущено перпендикуляр $KM$. До кожного початку речення (1--3) доберіть його закінчення (А--Д).}

\vspace{0.3cm}
\begin{minipage}{0.4\textwidth}
\textit{Початок речення}

\vspace{0.2cm}
\textbf{1} \quad Довжина $KM$

\vspace{0.15cm}
\textbf{2} \quad Довжина $AM$

\vspace{0.15cm}
\textbf{3} \quad Висота трикутника $ABC$
\end{minipage}
\hfill
\begin{minipage}{0.3\textwidth}
\textit{Закінчення речення}

\vspace{0.2cm}
\textbf{А} \quad дорівнює $4{,}5$ см.

\vspace{0.15cm}
\textbf{Б} \quad дорівнює $\dfrac{9\sqrt{3}}{2}$ см.

\vspace{0.15cm}
\textbf{В} \quad дорівнює $9\sqrt{3}$ см.

\vspace{0.15cm}
\textbf{Г} \quad дорівнює $9$ см.

\vspace{0.15cm}
\textbf{Д} \quad дорівнює $6\sqrt{3}$ см.

\vspace{0.3cm}
\matchTable
\end{minipage}
\hfill
\begin{minipage}{0.22\textwidth}
\begin{flushright}
\begin{tikzpicture}[scale=0.7]
    \coordinate (A) at (0,0);
    \coordinate (B) at (3,0);
    \coordinate (C) at (1.5,2.6);
    \coordinate (K) at (1.5,0);
    \coordinate (M) at (0.75,1.3);

    \draw[thick] (A) -- (B) -- (C) -- cycle;
    \draw[thick] (K) -- (M);

    \node[below left] at (A) {$A$};
    \node[below right] at (B) {$B$};
    \node[above] at (C) {$C$};
    \node[below] at (K) {$K$};
    \node[left] at (M) {$M$};
    \pic[draw, angle radius=0.15cm] {right angle = A--M--K};
\end{tikzpicture}
\end{flushright}
\end{minipage}

\vspace{0.7cm}

%======================================================================
% БЛОК 6: Подібність трикутників
%======================================================================

\section*{Блок 6: Подібність трикутників}

% Завдання 25
\task{25}{Які з наведених тверджень є правильними?}

\vspace{0.2cm}
\begin{tabular}{r@{\hspace{0.5em}}p{14cm}}
I. & Два трикутники подібні, якщо два кути одного трикутника відповідно рівні двом кутам іншого. \\
II. & Відношення площ подібних трикутників дорівнює квадрату коефіцієнта подібності. \\
III. & Відношення периметрів подібних трикутників дорівнює коефіцієнту подібності. \\
\end{tabular}

\answerTable{I, II та III}{лише I та II}{лише II та III}{лише I}{лише II}

\vspace{0.5cm}

% Завдання 26
\task{26}{Сторони одного трикутника дорівнюють 3, 4, 5 см, а відповідні сторони подібного трикутника --- 6, 8, 10 см. Знайдіть коефіцієнт подібності.}
\answerTable{2}{$\dfrac{1}{2}$}{4}{$\dfrac{1}{4}$}{$\sqrt{2}$}

\vspace{0.5cm}

% Завдання 27
\task{27}{Площі двох подібних трикутників відносяться як $1:9$. Знайдіть відношення їх периметрів.}
\answerTable{$1:3$}{$1:9$}{$1:81$}{$3:1$}{$1:\sqrt{3}$}

\vspace{0.5cm}

% Завдання 28
\task{28}{Периметри двох подібних трикутників дорівнюють 12 і 36 см. Знайдіть відношення їх площ.}
\answerTable{$1:9$}{$1:3$}{$1:27$}{$3:1$}{$9:1$}

\vspace{0.5cm}

%======================================================================
% БЛОК 7: Прямокутний трикутник
%======================================================================

\section*{Блок 7: Прямокутний трикутник}

% Завдання 29
\task{29}{У прямокутному трикутнику $ABC$ ($\angle C = 90°$) катети дорівнюють 5 і 12 см. Знайдіть гіпотенузу.}
\answerTable{13 см}{17 см}{7 см}{60 см}{$\sqrt{119}$ см}

\vspace{0.5cm}

% Завдання 30
\task{30}{Які з наведених тверджень є правильними?}

\vspace{0.2cm}
\begin{tabular}{r@{\hspace{0.5em}}p{14cm}}
I. & У прямокутному трикутнику найбільший кут дорівнює $90°$. \\
II. & У прямокутному трикутнику сума гострих кутів дорівнює $90°$. \\
III. & Гіпотенуза прямокутного трикутника є діаметром описаного кола. \\
\end{tabular}

\answerTable{I, II та III}{лише I та II}{лише II та III}{лише I та III}{лише II}

\vspace{0.5cm}

% Завдання 31
\task{31}{У прямокутному трикутнику катети дорівнюють 8 і 15 см. Знайдіть висоту, проведену до гіпотенузи.}
\answerTable{$\dfrac{120}{17}$ см}{$\dfrac{17}{2}$ см}{$\sqrt{289}$ см}{$\dfrac{23}{2}$ см}{12 см}

\vspace{0.5cm}

% Завдання 32
\task{32}{На рисунку зображено прямокутний трикутник $ABC$, $\angle B = 90°$, $BC = 9$ см, $AC = 15$ см. Знайдіть $AB$.}

\vspace{0.3cm}
\begin{minipage}{0.55\textwidth}
\answerTableSmall{12 см}{6 см}{$\sqrt{306}$ см}{24 см}{18 см}
\end{minipage}
\hfill
\begin{minipage}{0.4\textwidth}
\begin{flushright}
\begin{tikzpicture}[scale=0.5]
    \coordinate (A) at (0,4);
    \coordinate (B) at (0,0);
    \coordinate (C) at (3,0);

    \draw[thick] (A) -- (B) -- (C) -- cycle;
    \draw (0,0.4) -- (0.4,0.4) -- (0.4,0);

    \node at (1.5,-0.5) {\small 9 см};
    \node at (2.2,2.3) {\small 15 см};

    \node[above left] at (A) {$A$};
    \node[below left] at (B) {$B$};
    \node[below right] at (C) {$C$};
\end{tikzpicture}
\end{flushright}
\end{minipage}

\vspace{0.7cm}

%======================================================================
% БЛОК 8: Додаткові задачі
%======================================================================

\section*{Блок 8: Додаткові задачі}

% Завдання 33
\task{33}{На рисунку зображено квадрат $ABCD$ і прямокутний трикутник $KBC$ ($\angle B = 90°$), що лежать в одній площині. Сторона квадрата дорівнює 8 см, $BK = 6$ см. До кожного відрізка (1--3) доберіть його довжину (А--Д).}

\vspace{0.3cm}
\begin{minipage}{0.35\textwidth}
\textit{Відрізок}

\vspace{0.2cm}
\textbf{1} \quad $KC$

\vspace{0.15cm}
\textbf{2} \quad $AK$

\vspace{0.15cm}
\textbf{3} \quad діагональ квадрата

\vspace{0.3cm}
\matchTable
\end{minipage}
\hfill
\begin{minipage}{0.25\textwidth}
\textit{Довжина відрізка}

\vspace{0.2cm}
\textbf{А} \quad $8$ см

\vspace{0.15cm}
\textbf{Б} \quad $10$ см

\vspace{0.15cm}
\textbf{В} \quad $8\sqrt{2}$ см

\vspace{0.15cm}
\textbf{Г} \quad $14$ см

\vspace{0.15cm}
\textbf{Д} \quad $2\sqrt{65}$ см
\end{minipage}
\hfill
\begin{minipage}{0.3\textwidth}
\begin{flushright}
\begin{tikzpicture}[scale=0.4]
    \coordinate (A) at (0,0);
    \coordinate (B) at (0,4);
    \coordinate (C) at (4,4);
    \coordinate (D) at (4,0);
    \coordinate (K) at (0,7);

    \draw[thick] (A) -- (B) -- (C) -- (D) -- cycle;
    \draw[thick] (K) -- (B);
    \draw[thick] (K) -- (C);

    \draw (0,4.4) -- (0.4,4.4) -- (0.4,4);

    \node[below left] at (A) {$A$};
    \node[left] at (B) {$B$};
    \node[right] at (C) {$C$};
    \node[below right] at (D) {$D$};
    \node[above left] at (K) {$K$};
\end{tikzpicture}
\end{flushright}
\end{minipage}

\vspace{0.7cm}

% Завдання 34
\task{34}{Які з наведених тверджень є правильними?}

\vspace{0.2cm}
\begin{tabular}{r@{\hspace{0.5em}}p{14cm}}
I. & Серединний перпендикуляр до сторони трикутника проходить через центр описаного кола. \\
II. & Бісектриса кута трикутника проходить через центр вписаного кола. \\
III. & Медіана трикутника проходить через центр ваги (центроїд). \\
\end{tabular}

\answerTable{I, II та III}{лише I та II}{лише II та III}{лише I та III}{лише I}

\vspace{0.5cm}

% Завдання 35
\task{35}{Сторона рівностороннього трикутника дорівнює 10 см. Знайдіть його висоту.}
\answerTable{$5\sqrt{3}$ см}{10 см}{5 см}{$10\sqrt{3}$ см}{15 см}

\vspace{0.5cm}

\end{document}
