\documentclass[14pt]{extarticle}
\usepackage{fontspec}
\usepackage{polyglossia}
\setdefaultlanguage{ukrainian}

\defaultfontfeatures{Ligatures=TeX}
\setmainfont{Liberation Serif}
\setsansfont{Liberation Sans}
\setmonofont{Liberation Mono}

\usepackage[a4paper,margin=2cm,bottom=2.5cm,top=2.5cm]{geometry}
\usepackage{amsmath,amssymb}
\usepackage{enumitem}
\usepackage{tikz}
\usepackage{pgfplots}
\pgfplotsset{compat=1.16}
\usetikzlibrary{calc,patterns,angles,quotes}
\usepackage{xcolor}
\usepackage{array}
\usepackage{fancyhdr}
\usepackage{multicol}

% Кольори
\definecolor{headerblue}{RGB}{0, 102, 204}
\definecolor{yearcolor}{RGB}{128, 0, 128}

\pagestyle{fancy}
\fancyhf{}
\renewcommand{\headrulewidth}{0pt}
\fancyfoot[C]{\thepage}

\setlength{\headheight}{15pt}
\setlength{\headsep}{10pt}
\setlength{\footskip}{25pt}

\widowpenalty=10000
\clubpenalty=10000

% === КОМАНДИ ===

% Стандартна таблиця відповідей
\newcommand{\answerTable}[5]{
\begin{center}
\begin{tabular}{|*{5}{>{\centering\arraybackslash}m{2.8cm}|}}
\hline
\rule[-0.3cm]{0pt}{0.8cm}\textbf{А} & \textbf{Б} & \textbf{В} & \textbf{Г} & \textbf{Д} \\
\hline
\rule[-0.4cm]{0pt}{1.0cm}#1 & \rule[-0.4cm]{0pt}{1.0cm}#2 & \rule[-0.4cm]{0pt}{1.0cm}#3 & \rule[-0.4cm]{0pt}{1.0cm}#4 & \rule[-0.4cm]{0pt}{1.0cm}#5 \\
\hline
\end{tabular}
\end{center}
}

% Маленька таблиця відповідей
\newcommand{\answerTableSmall}[5]{
\begin{tabular}{|*{5}{>{\centering\arraybackslash}m{1.1cm}|}}
\hline
\rule[-0.2cm]{0pt}{0.6cm}\textbf{А} & \textbf{Б} & \textbf{В} & \textbf{Г} & \textbf{Д} \\
\hline
\rule[-0.3cm]{0pt}{0.8cm}#1 & #2 & #3 & #4 & #5 \\
\hline
\end{tabular}
}

% Команда для завдань
\newcommand{\gentask}[2]{\noindent\makebox[1.5em][l]{\textbf{#1.}}\parbox[t]{\dimexpr\textwidth-1.5em}{#2}}

\begin{document}

\begin{center}
{\Large\textbf{\color{headerblue}ЗГЕНЕРОВАНІ ЗАВДАННЯ}}
\end{center}

\begin{center}
{\large Тема 11: Трикутники}
\end{center}

\vspace{0.5cm}

%======================================================================
% БЛОК 1: Сума кутів трикутника (15 завдань)
%======================================================================

\section*{Блок 1: Сума кутів трикутника}

\gentask{1}{У трикутнику $ABC$ $\angle A = 50°$, $\angle B = 40°$. Знайдіть $\angle C$.}
\answerTable{$90°$}{$80°$}{$100°$}{$70°$}{$110°$}

\vspace{0.4cm}

\gentask{2}{У трикутнику $ABC$ $\angle A = 35°$, $\angle C = 75°$. Знайдіть $\angle B$.}
\answerTable{$70°$}{$80°$}{$60°$}{$90°$}{$110°$}

\vspace{0.4cm}

\gentask{3}{У трикутнику один кут дорівнює $45°$, другий~--- утричі більший. Знайдіть третій кут.}
\answerTable{$0°$}{$45°$}{$90°$}{$135°$}{немає}

\vspace{0.4cm}

\gentask{4}{Кути трикутника відносяться як $2:3:4$. Знайдіть найменший кут.}
\answerTable{$40°$}{$60°$}{$80°$}{$20°$}{$30°$}

\vspace{0.4cm}

\gentask{5}{Кути трикутника відносяться як $1:2:3$. Знайдіть найбільший кут.}
\answerTable{$90°$}{$60°$}{$120°$}{$30°$}{$100°$}

\vspace{0.4cm}

\gentask{6}{У рівнобедреному трикутнику кут при вершині дорівнює $40°$. Знайдіть кут при основі.}
\answerTable{$70°$}{$60°$}{$80°$}{$50°$}{$40°$}

\vspace{0.4cm}

\gentask{7}{У рівнобедреному трикутнику кут при основі дорівнює $55°$. Знайдіть кут при вершині.}
\answerTable{$70°$}{$55°$}{$110°$}{$125°$}{$80°$}

\vspace{0.4cm}

\gentask{8}{У рівностороньому трикутнику знайдіть кожен кут.}
\answerTable{$60°$}{$90°$}{$45°$}{$120°$}{$30°$}

\vspace{0.4cm}

\gentask{9}{У прямокутному трикутнику один гострий кут дорівнює $37°$. Знайдіть другий гострий кут.}
\answerTable{$53°$}{$143°$}{$63°$}{$47°$}{$127°$}

\vspace{0.4cm}

\gentask{10}{Зовнішній кут трикутника при вершині $C$ дорівнює $130°$. Знайдіть внутрішній кут $C$.}
\answerTable{$50°$}{$130°$}{$40°$}{$60°$}{$230°$}

\vspace{0.4cm}

\gentask{11}{У трикутнику $ABC$ $\angle A = 40°$, $\angle B = 60°$. Знайдіть зовнішній кут при вершині $C$.}
\answerTable{$100°$}{$80°$}{$260°$}{$120°$}{$140°$}

\vspace{0.4cm}

\gentask{12}{Зовнішній кут трикутника при вершині $A$ дорівнює $115°$, $\angle B = 45°$. Знайдіть $\angle C$.}
\answerTable{$70°$}{$50°$}{$65°$}{$80°$}{$90°$}

\vspace{0.4cm}

\gentask{13}{Один із кутів трикутника дорівнює $100°$. Яким є цей трикутник?}
\answerTable{тупокутним}{гострокутним}{прямокутним}{рівнобедреним}{рівностороннім}

\vspace{0.4cm}

\gentask{14}{У трикутнику всі кути рівні. Яким є цей трикутник?}
\answerTable{рівностороннім}{прямокутним}{тупокутним}{рівнобедреним}{різностороннім}

\vspace{0.4cm}

\gentask{15}{Сума двох кутів трикутника дорівнює $80°$. Знайдіть третій кут.}
\answerTable{$100°$}{$80°$}{$260°$}{$90°$}{$180°$}

\vspace{0.5cm}

%======================================================================
% БЛОК 2: Рівнобедрений трикутник (15 завдань)
%======================================================================

\section*{Блок 2: Рівнобедрений трикутник}

\gentask{16}{Периметр рівнобедреного трикутника дорівнює 36 см, а бічна сторона~--- 10 см. Знайдіть основу.}
\answerTable{16 см}{10 см}{12 см}{14 см}{8 см}

\vspace{0.4cm}

\gentask{17}{Периметр рівнобедреного трикутника дорівнює 50 см, а основа~--- 16 см. Знайдіть бічну сторону.}
\answerTable{17 см}{16 см}{18 см}{34 см}{25 см}

\vspace{0.4cm}

\gentask{18}{У рівнобедреному трикутнику основа дорівнює 12 см, а бічна сторона на 4 см більша за основу. Знайдіть периметр.}
\answerTable{44 см}{36 см}{40 см}{48 см}{32 см}

\vspace{0.4cm}

\gentask{19}{У рівнобедреному трикутнику бічна сторона дорівнює 15 см, а периметр~--- 48 см. Знайдіть основу.}
\answerTable{18 см}{15 см}{12 см}{21 см}{24 см}

\vspace{0.4cm}

\gentask{20}{У рівнобедреному трикутнику основа на 6 см менша за бічну сторону. Периметр дорівнює 45 см. Знайдіть бічну сторону.}
\answerTable{17 см}{11 см}{14 см}{20 см}{23 см}

\vspace{0.4cm}

\gentask{21}{У рівностороньому трикутнику периметр дорівнює 42 см. Знайдіть сторону.}
\answerTable{14 см}{21 см}{12 см}{7 см}{16 см}

\vspace{0.4cm}

\gentask{22}{Сторона рівностороннього трикутника дорівнює 9 см. Знайдіть його периметр.}
\answerTable{27 см}{18 см}{36 см}{24 см}{45 см}

\vspace{0.4cm}

\gentask{23}{У рівнобедреному трикутнику висота, проведена до основи, дорівнює 8 см, а основа~--- 12 см. Знайдіть бічну сторону.}
\answerTable{10 см}{12 см}{8 см}{14 см}{16 см}

\vspace{0.4cm}

\gentask{24}{У рівнобедреному трикутнику бічна сторона дорівнює 13 см, а висота до основи~--- 12 см. Знайдіть основу.}
\answerTable{10 см}{5 см}{24 см}{26 см}{12 см}

\vspace{0.4cm}

\gentask{25}{У рівнобедреному трикутнику бічна сторона дорівнює 17 см, а основа~--- 16 см. Знайдіть висоту до основи.}
\answerTable{15 см}{8 см}{17 см}{12 см}{20 см}

\vspace{0.4cm}

\gentask{26}{Висота рівностороннього трикутника дорівнює $6\sqrt{3}$ см. Знайдіть сторону.}
\answerTable{12 см}{6 см}{$12\sqrt{3}$ см}{$6\sqrt{3}$ см}{18 см}

\vspace{0.4cm}

\gentask{27}{Сторона рівностороннього трикутника дорівнює 10 см. Знайдіть висоту.}
\answerTable{$5\sqrt{3}$ см}{5 см}{10 см}{$10\sqrt{3}$ см}{15 см}

\vspace{0.4cm}

\gentask{28}{Медіана рівнобедреного трикутника, проведена до основи, дорівнює 9 см. Чому дорівнює висота до основи?}
\answerTable{9 см}{18 см}{$9\sqrt{2}$ см}{4{,}5 см}{не можна визначити}

\vspace{0.4cm}

\gentask{29}{У рівнобедреному трикутнику кут при вершині дорівнює $120°$. Знайдіть кут між бічною стороною та висотою до основи.}
\answerTable{$60°$}{$30°$}{$90°$}{$120°$}{$45°$}

\vspace{0.4cm}

\gentask{30}{У рівнобедреному трикутнику кут при основі дорівнює $75°$. Знайдіть кут між бічними сторонами.}
\answerTable{$30°$}{$75°$}{$105°$}{$150°$}{$15°$}

\vspace{0.5cm}

%======================================================================
% БЛОК 3: Середня лінія трикутника (12 завдань)
%======================================================================

\section*{Блок 3: Середня лінія трикутника}

\gentask{31}{Середня лінія трикутника дорівнює 7 см. Знайдіть сторону, до якої вона паралельна.}
\answerTable{14 см}{7 см}{3{,}5 см}{21 см}{28 см}

\vspace{0.4cm}

\gentask{32}{Сторона трикутника дорівнює 18 см. Знайдіть середню лінію, паралельну цій стороні.}
\answerTable{9 см}{36 см}{18 см}{6 см}{12 см}

\vspace{0.4cm}

\gentask{33}{У трикутнику $ABC$ точки $M$ і $N$~--- середини сторін $AB$ і $BC$ відповідно. $AC = 16$ см. Знайдіть $MN$.}
\answerTable{8 см}{16 см}{32 см}{4 см}{12 см}

\vspace{0.4cm}

\gentask{34}{У трикутнику $ABC$ $MN$~--- середня лінія, $MN = 5$ см. Знайдіть сторону $AC$, якщо $MN \parallel AC$.}
\answerTable{10 см}{5 см}{2{,}5 см}{15 см}{20 см}

\vspace{0.4cm}

\gentask{35}{Периметр трикутника дорівнює 30 см. Знайдіть периметр трикутника, утвореного середніми лініями.}
\answerTable{15 см}{30 см}{60 см}{7{,}5 см}{10 см}

\vspace{0.4cm}

\gentask{36}{Периметр трикутника, утвореного середніми лініями, дорівнює 12 см. Знайдіть периметр вихідного трикутника.}
\answerTable{24 см}{12 см}{6 см}{36 см}{48 см}

\vspace{0.4cm}

\gentask{37}{Площа трикутника дорівнює 48 см$^2$. Знайдіть площу трикутника, утвореного середніми лініями.}
\answerTable{12 см$^2$}{24 см$^2$}{48 см$^2$}{6 см$^2$}{96 см$^2$}

\vspace{0.4cm}

\gentask{38}{Середня лінія трикутника паралельна стороні $BC$ і перетинає медіану $BM$. Знайдіть відношення частин, на які вона ділить медіану.}
\answerTable{$1:1$}{$1:2$}{$2:1$}{$1:3$}{$3:1$}

\vspace{0.4cm}

\gentask{39}{Сторони трикутника дорівнюють 10, 14 і 18 см. Знайдіть найбільшу середню лінію.}
\answerTable{9 см}{7 см}{5 см}{14 см}{18 см}

\vspace{0.4cm}

\gentask{40}{Сторони трикутника дорівнюють 8, 12 і 16 см. Знайдіть найменшу середню лінію.}
\answerTable{4 см}{6 см}{8 см}{12 см}{2 см}

\vspace{0.4cm}

\gentask{41}{У трикутнику $ABC$ точка $M$~--- середина $AB$, точка $N$~--- середина $BC$. $\angle A = 50°$. Знайдіть $\angle BMN$.}
\answerTable{$50°$}{$130°$}{$40°$}{$90°$}{$80°$}

\vspace{0.4cm}

\gentask{42}{Три середні лінії трикутника ділять його на 4 трикутники. Скільки з них рівні між собою?}
\answerTable{4}{2}{3}{1}{0}

\vspace{0.5cm}

%======================================================================
% БЛОК 4: Медіани трикутника (12 завдань)
%======================================================================

\section*{Блок 4: Медіани трикутника}

\gentask{43}{Медіана трикутника ділить сторону на відрізки 6 см і 6 см. Чому дорівнює ця сторона?}
\answerTable{12 см}{6 см}{18 см}{3 см}{9 см}

\vspace{0.4cm}

\gentask{44}{Медіана $AM$ трикутника $ABC$ ділить сторону $BC$ на рівні частини. $BM = 7$ см. Знайдіть $BC$.}
\answerTable{14 см}{7 см}{21 см}{3{,}5 см}{28 см}

\vspace{0.4cm}

\gentask{45}{Медіани трикутника перетинаються в точці $O$. Медіана $AM = 12$ см. Знайдіть $AO$.}
\answerTable{8 см}{4 см}{6 см}{12 см}{9 см}

\vspace{0.4cm}

\gentask{46}{Медіани трикутника перетинаються в точці $O$. Медіана $BK = 15$ см. Знайдіть $KO$.}
\answerTable{5 см}{10 см}{7{,}5 см}{15 см}{3 см}

\vspace{0.4cm}

\gentask{47}{У трикутнику $ABC$ медіана $AM$ ділить його на два трикутники. Порівняйте їх площі.}
\answerTable{рівні}{$S_{ABM} > S_{AMC}$}{$S_{ABM} < S_{AMC}$}{$S_{ABM} = 2S_{AMC}$}{не можна порівняти}

\vspace{0.4cm}

\gentask{48}{Точка перетину медіан ділить кожну медіану у відношенні (від вершини):}
\answerTable{$2:1$}{$1:2$}{$1:1$}{$3:1$}{$1:3$}

\vspace{0.4cm}

\gentask{49}{Точка перетину медіан трикутника є його:}
\answerTable{центром ваги}{центром описаного кола}{центром вписаного кола}{ортоцентром}{жодним з цих}

\vspace{0.4cm}

\gentask{50}{Медіана прямокутного трикутника, проведена до гіпотенузи, дорівнює 13 см. Знайдіть гіпотенузу.}
\answerTable{26 см}{13 см}{6{,}5 см}{$13\sqrt{2}$ см}{$26\sqrt{2}$ см}

\vspace{0.4cm}

\gentask{51}{Гіпотенуза прямокутного трикутника дорівнює 20 см. Знайдіть медіану до гіпотенузи.}
\answerTable{10 см}{20 см}{40 см}{5 см}{$10\sqrt{2}$ см}

\vspace{0.4cm}

\gentask{52}{У рівностороньому трикутнику зі стороною 6 см знайдіть медіану.}
\answerTable{$3\sqrt{3}$ см}{3 см}{6 см}{$6\sqrt{3}$ см}{9 см}

\vspace{0.4cm}

\gentask{53}{Три медіани ділять трикутник на 6 менших трикутників. Що можна сказати про їх площі?}
\answerTable{усі рівні}{попарно рівні}{різні}{не можна визначити}{залежить від типу}

\vspace{0.4cm}

\gentask{54}{Медіани трикутника перетинаються в точці $O$. $AO = 10$ см. Знайдіть довжину медіани $AM$.}
\answerTable{15 см}{10 см}{20 см}{5 см}{30 см}

\vspace{0.5cm}

%======================================================================
% БЛОК 5: Висоти трикутника (10 завдань)
%======================================================================

\section*{Блок 5: Висоти трикутника}

\gentask{55}{У прямокутному трикутнику катети дорівнюють 6 і 8 см. Знайдіть висоту, проведену до гіпотенузи.}
\answerTable{4{,}8 см}{5 см}{7 см}{10 см}{3 см}

\vspace{0.4cm}

\gentask{56}{Площа трикутника 60 см$^2$, а основа 15 см. Знайдіть висоту до цієї основи.}
\answerTable{8 см}{4 см}{120 см}{30 см}{12 см}

\vspace{0.4cm}

\gentask{57}{Висота трикутника дорівнює 12 см, а основа~--- 10 см. Знайдіть площу.}
\answerTable{60 см$^2$}{120 см$^2$}{30 см$^2$}{22 см$^2$}{24 см$^2$}

\vspace{0.4cm}

\gentask{58}{Де перетинаються висоти гострокутного трикутника?}
\answerTable{всередині}{зовні}{на стороні}{у вершині}{не перетинаються}

\vspace{0.4cm}

\gentask{59}{Де перетинаються висоти тупокутного трикутника?}
\answerTable{зовні}{всередині}{на стороні}{у вершині}{не перетинаються}

\vspace{0.4cm}

\gentask{60}{Де перетинаються висоти прямокутного трикутника?}
\answerTable{у вершині прямого кута}{всередині}{зовні}{на гіпотенузі}{не перетинаються}

\vspace{0.4cm}

\gentask{61}{У рівнобедреному трикутнику висота до основи є також:}
\answerTable{медіаною і бісектрисою}{тільки медіаною}{тільки бісектрисою}{середньою лінією}{жодним з цих}

\vspace{0.4cm}

\gentask{62}{Скільки висот можна провести в трикутнику?}
\answerTable{3}{1}{2}{6}{безліч}

\vspace{0.4cm}

\gentask{63}{Точка перетину висот трикутника називається:}
\answerTable{ортоцентром}{центроїдом}{інцентром}{циркумцентром}{барицентром}

\vspace{0.4cm}

\gentask{64}{У рівностороньому трикутнику зі стороною $a$ знайдіть висоту.}
\answerTable{$\frac{a\sqrt{3}}{2}$}{$a$}{$\frac{a}{2}$}{$a\sqrt{3}$}{$\frac{a\sqrt{2}}{2}$}

\vspace{0.5cm}

%======================================================================
% БЛОК 6: Бісектриси трикутника (10 завдань)
%======================================================================

\section*{Блок 6: Бісектриси трикутника}

\gentask{65}{Бісектриса трикутника ділить протилежну сторону на відрізки, пропорційні:}
\answerTable{прилеглим сторонам}{протилежним сторонам}{рівним частинам}{висотам}{медіанам}

\vspace{0.4cm}

\gentask{66}{У трикутнику $AB = 6$ см, $AC = 9$ см. Бісектриса $AD$ ділить сторону $BC$ на відрізки $BD$ і $DC$. Знайдіть $BD : DC$.}
\answerTable{$2:3$}{$3:2$}{$1:1$}{$6:9$}{$9:6$}

\vspace{0.4cm}

\gentask{67}{Точка перетину бісектрис трикутника є центром:}
\answerTable{вписаного кола}{описаного кола}{не є центром кола}{дев'яти точок}{жодного з них}

\vspace{0.4cm}

\gentask{68}{Бісектриси трикутника перетинаються:}
\answerTable{в одній точці всередині}{в одній точці зовні}{у трьох точках}{на сторонах}{не завжди в одній точці}

\vspace{0.4cm}

\gentask{69}{У рівнобедреному трикутнику бісектриса з вершини до основи є також:}
\answerTable{висотою і медіаною}{тільки висотою}{тільки медіаною}{середньою лінією}{нічим з цього}

\vspace{0.4cm}

\gentask{70}{У трикутнику $\angle A = 80°$. Бісектриса $AD$ ділить цей кут. Знайдіть $\angle BAD$.}
\answerTable{$40°$}{$80°$}{$160°$}{$100°$}{$20°$}

\vspace{0.4cm}

\gentask{71}{У трикутнику $ABC$ бісектриси $AA_1$ і $BB_1$ перетинаються в точці $I$. $\angle C = 70°$. Знайдіть $\angle AIB$.}
\answerTable{$125°$}{$55°$}{$110°$}{$70°$}{$140°$}

\vspace{0.4cm}

\gentask{72}{У рівностороньому трикутнику бісектриса дорівнює:}
\answerTable{висоті і медіані}{тільки висоті}{тільки медіані}{половині сторони}{стороні}

\vspace{0.4cm}

\gentask{73}{У трикутнику $AB = 8$ см, $AC = 12$ см, $BC = 15$ см. Бісектриса $AD$ ділить $BC$ на $BD$ і $DC$. Знайдіть $BD$.}
\answerTable{6 см}{9 см}{7{,}5 см}{5 см}{10 см}

\vspace{0.4cm}

\gentask{74}{Відстань від точки перетину бісектрис до всіх сторін трикутника:}
\answerTable{однакова}{різна}{залежить від типу}{не визначена}{нуль}

\vspace{0.5cm}

%======================================================================
% БЛОК 7: Описане та вписане коло (12 завдань)
%======================================================================

\section*{Блок 7: Описане та вписане коло}

\gentask{75}{Центр описаного навколо трикутника кола є точкою перетину:}
\answerTable{серединних перпендикулярів}{бісектрис}{медіан}{висот}{сторін}

\vspace{0.4cm}

\gentask{76}{Центр вписаного в трикутник кола є точкою перетину:}
\answerTable{бісектрис}{серединних перпендикулярів}{медіан}{висот}{сторін}

\vspace{0.4cm}

\gentask{77}{Радіус кола, описаного навколо прямокутного трикутника з гіпотенузою 10 см, дорівнює:}
\answerTable{5 см}{10 см}{2{,}5 см}{$5\sqrt{2}$ см}{$10\sqrt{2}$ см}

\vspace{0.4cm}

\gentask{78}{Гіпотенуза прямокутного трикутника дорівнює діаметру описаного кола. Це твердження:}
\answerTable{завжди правильне}{завжди неправильне}{іноді правильне}{не має сенсу}{залежить від кутів}

\vspace{0.4cm}

\gentask{79}{Катети прямокутного трикутника 6 і 8 см. Знайдіть радіус описаного кола.}
\answerTable{5 см}{10 см}{7 см}{14 см}{$5\sqrt{2}$ см}

\vspace{0.4cm}

\gentask{80}{Сторона рівностороннього трикутника 6 см. Знайдіть радіус описаного кола.}
\answerTable{$2\sqrt{3}$ см}{$3\sqrt{3}$ см}{3 см}{6 см}{$6\sqrt{3}$ см}

\vspace{0.4cm}

\gentask{81}{Сторона рівностороннього трикутника $a$. Знайдіть радіус вписаного кола.}
\answerTable{$\frac{a\sqrt{3}}{6}$}{$\frac{a\sqrt{3}}{3}$}{$\frac{a}{2}$}{$\frac{a}{3}$}{$a\sqrt{3}$}

\vspace{0.4cm}

\gentask{82}{Відношення радіусів описаного і вписаного кіл рівностороннього трикутника дорівнює:}
\answerTable{$2:1$}{$1:2$}{$3:1$}{$1:3$}{$\sqrt{3}:1$}

\vspace{0.4cm}

\gentask{83}{Центр описаного кола гострокутного трикутника лежить:}
\answerTable{всередині}{зовні}{на стороні}{у вершині}{на висоті}

\vspace{0.4cm}

\gentask{84}{Центр описаного кола тупокутного трикутника лежить:}
\answerTable{зовні}{всередині}{на стороні}{у вершині}{на медіані}

\vspace{0.4cm}

\gentask{85}{Центр описаного кола прямокутного трикутника лежить:}
\answerTable{на гіпотенузі}{всередині}{зовні}{у вершині прямого кута}{на катеті}

\vspace{0.4cm}

\gentask{86}{Площа трикутника $S = pr$, де $p$~--- півпериметр, $r$~--- радіус вписаного кола. Знайдіть $r$, якщо $S = 24$ см$^2$, периметр 24 см.}
\answerTable{2 см}{4 см}{1 см}{12 см}{6 см}

\vspace{0.5cm}

%======================================================================
% БЛОК 8: Ознаки рівності та подібності (14 завдань)
%======================================================================

\section*{Блок 8: Ознаки рівності та подібності}

\gentask{87}{Скільки існує ознак рівності трикутників?}
\answerTable{3}{2}{4}{5}{1}

\vspace{0.4cm}

\gentask{88}{Два трикутники рівні, якщо рівні їх:}
\answerTable{три сторони}{три кути}{два кути}{дві сторони}{один кут і одна сторона}

\vspace{0.4cm}

\gentask{89}{Два трикутники рівні за першою ознакою, якщо:}
\answerTable{дві сторони і кут між ними}{дві сторони і кут}{три сторони}{сторона і два кути}{три кути}

\vspace{0.4cm}

\gentask{90}{Два трикутники рівні за другою ознакою, якщо:}
\answerTable{сторона і два прилеглих кути}{дві сторони і кут}{три сторони}{два кути}{сторона і один кут}

\vspace{0.4cm}

\gentask{91}{Два трикутники рівні за третьою ознакою, якщо рівні:}
\answerTable{три сторони}{дві сторони і кут}{сторона і два кути}{три кути}{два кути і сторона}

\vspace{0.4cm}

\gentask{92}{Два трикутники подібні, якщо:}
\answerTable{рівні два кути}{рівні три сторони}{рівні три кути та три сторони}{рівні периметри}{рівні площі}

\vspace{0.4cm}

\gentask{93}{Відношення периметрів подібних трикутників з коефіцієнтом подібності $k$ дорівнює:}
\answerTable{$k$}{$k^2$}{$\sqrt{k}$}{$1$}{$k^3$}

\vspace{0.4cm}

\gentask{94}{Відношення площ подібних трикутників з коефіцієнтом подібності $k$ дорівнює:}
\answerTable{$k^2$}{$k$}{$\sqrt{k}$}{$1$}{$k^3$}

\vspace{0.4cm}

\gentask{95}{Сторони трикутника 3, 4, 5 см. Сторони подібного трикутника 6, 8, 10 см. Знайдіть коефіцієнт подібності.}
\answerTable{2}{0{,}5}{4}{1{,}5}{3}

\vspace{0.4cm}

\gentask{96}{Площі подібних трикутників відносяться як 1:4. Знайдіть коефіцієнт подібності.}
\answerTable{$1:2$}{$1:4$}{$1:16$}{$2:1$}{$1:\sqrt{2}$}

\vspace{0.4cm}

\gentask{97}{Периметри подібних трикутників 15 і 45 см. Знайдіть відношення їх площ.}
\answerTable{$1:9$}{$1:3$}{$1:27$}{$3:1$}{$9:1$}

\vspace{0.4cm}

\gentask{98}{Два трикутники мають по два рівних кути $50°$ і $60°$. Трикутники:}
\answerTable{подібні}{рівні}{можуть бути і рівними}{не подібні}{неможливо визначити}

\vspace{0.4cm}

\gentask{99}{Медіани подібних трикутників відносяться як:}
\answerTable{відповідні сторони}{квадрати сторін}{корені зі сторін}{не пов'язані}{площі}

\vspace{0.4cm}

\gentask{100}{Висоти подібних трикутників з коефіцієнтом подібності $k$ відносяться як:}
\answerTable{$k$}{$k^2$}{$1$}{$\sqrt{k}$}{$k^3$}

\vspace{0.5cm}

\end{document}
