\documentclass[12pt]{extarticle}
\usepackage{fontspec}
\usepackage{polyglossia}
\setdefaultlanguage{ukrainian}

\defaultfontfeatures{Ligatures=TeX}
\setmainfont{Liberation Serif}

\usepackage[a4paper,margin=2cm]{geometry}
\usepackage{amsmath,amssymb}
\usepackage{multicol}
\usepackage{xcolor}

\definecolor{headerblue}{RGB}{0, 102, 204}

\begin{document}

\begin{center}
{\Large\textbf{\color{headerblue}ВІДПОВІДІ}}
\end{center}

\begin{center}
{\large Тема 11: Трикутники}
\end{center}

\vspace{0.5cm}

\textbf{Блок 1: Сума кутів трикутника}

\begin{multicols}{5}
\noindent
1. А ($90°$) \\
2. А ($70°$) \\
3. А ($0°$) \\
4. А ($40°$) \\
5. А ($90°$) \\
6. А ($70°$) \\
7. А ($70°$) \\
8. А ($60°$) \\
9. А ($53°$) \\
10. А ($50°$) \\
11. А ($100°$) \\
12. А ($70°$) \\
13. А \\
14. А \\
15. А ($100°$)
\end{multicols}

\textbf{Блок 2: Рівнобедрений трикутник}

\begin{multicols}{5}
\noindent
16. А ($16$) \\
17. А ($17$) \\
18. А ($44$) \\
19. А ($18$) \\
20. А ($17$) \\
21. А ($14$) \\
22. А ($27$) \\
23. А ($10$) \\
24. А ($10$) \\
25. А ($15$) \\
26. А ($12$) \\
27. А ($5\sqrt{3}$) \\
28. А ($9$) \\
29. А ($60°$) \\
30. А ($30°$)
\end{multicols}

\textbf{Блок 3: Середня лінія трикутника}

\begin{multicols}{5}
\noindent
31. А ($14$) \\
32. А ($9$) \\
33. А ($8$) \\
34. А ($10$) \\
35. А ($15$) \\
36. А ($24$) \\
37. А ($12$) \\
38. А ($1:1$) \\
39. А ($9$) \\
40. А ($4$) \\
41. А ($50°$) \\
42. А ($4$)
\end{multicols}

\textbf{Блок 4: Медіани трикутника}

\begin{multicols}{5}
\noindent
43. А ($12$) \\
44. А ($14$) \\
45. А ($8$) \\
46. А ($5$) \\
47. А \\
48. А ($2:1$) \\
49. А \\
50. А ($26$) \\
51. А ($10$) \\
52. А ($3\sqrt{3}$) \\
53. А \\
54. А ($15$)
\end{multicols}

\textbf{Блок 5: Висоти трикутника}

\begin{multicols}{5}
\noindent
55. А ($4{,}8$) \\
56. А ($8$) \\
57. А ($60$) \\
58. А \\
59. А \\
60. А \\
61. А \\
62. А ($3$) \\
63. А \\
64. А
\end{multicols}

\textbf{Блок 6: Бісектриси трикутника}

\begin{multicols}{5}
\noindent
65. А \\
66. А ($2:3$) \\
67. А \\
68. А \\
69. А \\
70. А ($40°$) \\
71. А ($125°$) \\
72. А \\
73. А ($6$) \\
74. А
\end{multicols}

\textbf{Блок 7: Описане та вписане коло}

\begin{multicols}{5}
\noindent
75. А \\
76. А \\
77. А ($5$) \\
78. А \\
79. А ($5$) \\
80. А ($2\sqrt{3}$) \\
81. А \\
82. А ($2:1$) \\
83. А \\
84. А \\
85. А \\
86. А ($2$)
\end{multicols}

\textbf{Блок 8: Ознаки рівності та подібності}

\begin{multicols}{5}
\noindent
87. А ($3$) \\
88. А \\
89. А \\
90. А \\
91. А \\
92. А \\
93. А ($k$) \\
94. А ($k^2$) \\
95. А ($2$) \\
96. А ($1:2$) \\
97. А ($1:9$) \\
98. А \\
99. А \\
100. А ($k$)
\end{multicols}

\vspace{1cm}

\textbf{Ключові формули:}

\begin{enumerate}
\item \textbf{Сума кутів трикутника:} $\alpha + \beta + \gamma = 180°$

\item \textbf{Зовнішній кут:} Зовнішній кут дорівнює сумі двох внутрішніх, не суміжних з ним

\item \textbf{Рівнобедрений трикутник:}
\begin{itemize}
\item Кути при основі рівні
\item Висота до основи = медіана = бісектриса
\end{itemize}

\item \textbf{Рівносторонній трикутник:}
\begin{itemize}
\item Усі кути = $60°$
\item Висота $h = \dfrac{a\sqrt{3}}{2}$
\end{itemize}

\item \textbf{Середня лінія:} $MN = \dfrac{1}{2}AC$, $MN \parallel AC$

\item \textbf{Медіани:} Точка перетину ділить кожну медіану у відношенні $2:1$ від вершини

\item \textbf{Медіана прямокутного трикутника до гіпотенузи:} $m_c = \dfrac{c}{2}$

\item \textbf{Властивість бісектриси:} $\dfrac{BD}{DC} = \dfrac{AB}{AC}$

\item \textbf{Радіус описаного кола прямокутного трикутника:} $R = \dfrac{c}{2}$ (гіпотенуза)

\item \textbf{Площа через вписане коло:} $S = pr$, де $p$ --- півпериметр

\item \textbf{Подібні трикутники:}
\begin{itemize}
\item Периметри відносяться як $k$
\item Площі відносяться як $k^2$
\end{itemize}
\end{enumerate}

\end{document}
