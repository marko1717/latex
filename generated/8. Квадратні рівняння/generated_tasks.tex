\documentclass[14pt]{extarticle}
\usepackage{fontspec}
\usepackage{polyglossia}
\setdefaultlanguage{ukrainian}

\defaultfontfeatures{Ligatures=TeX}
\setmainfont{Liberation Serif}
\setsansfont{Liberation Sans}
\setmonofont{Liberation Mono}

\usepackage[a4paper,margin=2cm,bottom=2.5cm,top=2.5cm]{geometry}
\usepackage{amsmath,amssymb}
\usepackage{enumitem}
\usepackage{xcolor}
\usepackage{array}
\usepackage{fancyhdr}

% Кольори
\definecolor{headerblue}{RGB}{0, 102, 204}
\definecolor{gencolor}{RGB}{0, 128, 0}

\pagestyle{fancy}
\fancyhf{}
\renewcommand{\headrulewidth}{0pt}
\fancyfoot[C]{\thepage}

\setlength{\headheight}{15pt}
\setlength{\headsep}{10pt}
\setlength{\footskip}{25pt}

\widowpenalty=10000
\clubpenalty=10000

% === КОМАНДИ ===

\newcommand{\answerTable}[5]{
\begin{center}
\begin{tabular}{|*{5}{>{\centering\arraybackslash}m{2.8cm}|}}
\hline
\rule[-0.3cm]{0pt}{0.8cm}\textbf{А} & \textbf{Б} & \textbf{В} & \textbf{Г} & \textbf{Д} \\
\hline
\rule[-0.4cm]{0pt}{1.0cm}#1 & \rule[-0.4cm]{0pt}{1.0cm}#2 & \rule[-0.4cm]{0pt}{1.0cm}#3 & \rule[-0.4cm]{0pt}{1.0cm}#4 & \rule[-0.4cm]{0pt}{1.0cm}#5 \\
\hline
\end{tabular}
\end{center}
}

\newcommand{\answerTableBig}[5]{
\begin{center}
\begin{tabular}{|*{5}{>{\centering\arraybackslash}m{2.8cm}|}}
\hline
\rule[-0.3cm]{0pt}{0.8cm}\textbf{А} & \textbf{Б} & \textbf{В} & \textbf{Г} & \textbf{Д} \\
\hline
\rule[-0.6cm]{0pt}{1.4cm}#1 & \rule[-0.6cm]{0pt}{1.4cm}#2 & \rule[-0.6cm]{0pt}{1.4cm}#3 & \rule[-0.6cm]{0pt}{1.4cm}#4 & \rule[-0.6cm]{0pt}{1.4cm}#5 \\
\hline
\end{tabular}
\end{center}
}

\newcommand{\task}[2]{\noindent\makebox[1.5em][l]{\textbf{#1.}}\parbox[t]{\dimexpr\textwidth-1.5em}{#2}}

\newcommand{\gentask}{\hfill{\small\color{gencolor}(Згенеровано)}}

\begin{document}

\begin{center}
{\Large\textbf{\color{headerblue}ЗГЕНЕРОВАНІ ЗАВДАННЯ}}
\end{center}

\begin{center}
{\large Тема 8: \textbf{Квадратні рівняння та рівняння, що зводяться до квадратних}}
\end{center}

\vspace{0.3cm}
\begin{center}
\textit{100 завдань з варіаціями стилів формулювання}
\end{center}

\vspace{0.5cm}

%======================================================================
% БЛОК 1: ПРОСТІ КВАДРАТНІ РІВНЯННЯ x² = a
%======================================================================

\begin{center}
{\large\textbf{\color{headerblue}Блок 1: Неповні квадратні рівняння $x^2 = a$}}
\end{center}

\vspace{0.3cm}

% --- Стиль: "Розв'яжіть рівняння" ---

\task{1}{Розв'яжіть рівняння $x^2 = 49$. \gentask}
\answerTable{$-7$; $7$}{$-7$}{$7$}{$\pm 49$}{$49$}

\vspace{0.4cm}

\task{2}{Розв'яжіть рівняння $x^2 = 81$. \gentask}
\answerTable{$81$}{$9$}{$-9$}{$-9$; $9$}{$\pm 81$}

\vspace{0.4cm}

\task{3}{Розв'яжіть рівняння $x^2 = 144$. \gentask}
\answerTable{$-12$; $12$}{$144$}{$12$}{$\pm 144$}{$-12$}

\vspace{0.4cm}

\task{4}{Розв'яжіть рівняння $x^2 - 25 = 0$. \gentask}
\answerTable{$25$}{$-5$}{$\pm 25$}{$-5$; $5$}{$5$}

\vspace{0.4cm}

\task{5}{Розв'яжіть рівняння $x^2 - 64 = 0$. \gentask}
\answerTable{$-8$}{$-8$; $8$}{$8$}{$\pm 64$}{$64$}

\vspace{0.4cm}

% --- Стиль: "x²/a = b" ---

\task{6}{Розв'яжіть рівняння $\dfrac{x^2}{3} = 12$. \gentask}
\answerTable{$4$}{$-6$}{$-6$; $6$}{$36$}{$6$}

\vspace{0.4cm}

\task{7}{Розв'яжіть рівняння $\dfrac{x^2}{5} = 20$. \gentask}
\answerTable{$100$}{$-10$; $10$}{$4$}{$10$}{$-10$}

\vspace{0.4cm}

\task{8}{Розв'яжіть рівняння $\dfrac{x^2}{4} = 9$. \gentask}
\answerTable{$2{,}25$}{$36$}{$-6$}{$-6$; $6$}{$6$}

\vspace{0.4cm}

\task{9}{Розв'яжіть рівняння $\dfrac{x^2}{2} = 18$. \gentask}
\answerTable{$36$}{$6$}{$-6$}{$-6$; $6$}{$9$}

\vspace{0.4cm}

\task{10}{Розв'яжіть рівняння $\dfrac{x^2}{7} = 7$. \gentask}
\answerTable{$7$}{$-7$; $7$}{$49$}{$1$}{$-7$}

\vspace{0.4cm}

%======================================================================
% БЛОК 2: РІВНЯННЯ ВИДУ (ax + b)² = c
%======================================================================

\newpage

\begin{center}
{\large\textbf{\color{headerblue}Блок 2: Рівняння виду $(ax + b)^2 = c$}}
\end{center}

\vspace{0.3cm}

% --- Стиль: "Розв'яжіть рівняння" ---

\task{11}{Розв'яжіть рівняння $(x - 3)^2 = 16$. \gentask}
\answerTable{$19$}{$-7$; $1$}{$7$}{$-1$}{$-1$; $7$}

\vspace{0.4cm}

\task{12}{Розв'яжіть рівняння $(x + 2)^2 = 25$. \gentask}
\answerTable{$23$}{$-7$}{$7$; $-3$}{$-7$; $3$}{$3$}

\vspace{0.4cm}

\task{13}{Розв'яжіть рівняння $(x - 5)^2 = 9$. \gentask}
\answerTable{$2$; $8$}{$2$}{$14$}{$8$}{$-2$; $-8$}

\vspace{0.4cm}

\task{14}{Розв'яжіть рівняння $(x + 4)^2 = 49$. \gentask}
\answerTable{$11$; $-3$}{$-11$; $3$}{$3$}{$45$}{$-11$}

\vspace{0.4cm}

\task{15}{Розв'яжіть рівняння $(x - 1)^2 = 36$. \gentask}
\answerTable{$5$; $-7$}{$35$}{$7$}{$-5$; $7$}{$-5$}

\vspace{0.4cm}

% --- Стиль: "(ax + b)² = 0" ---

\task{16}{Розв'яжіть рівняння $(3x - 6)^2 = 0$. \gentask}
\answerTable{$0$}{$6$}{$2$}{$-2$; $2$}{$-2$}

\vspace{0.4cm}

\task{17}{Розв'яжіть рівняння $(2x + 8)^2 = 0$. \gentask}
\answerTable{$0$}{$8$}{$-4$}{$4$}{$-4$; $4$}

\vspace{0.4cm}

\task{18}{Розв'яжіть рівняння $(5x - 10)^2 = 0$. \gentask}
\answerTable{$-2$; $2$}{$10$}{$0$}{$-2$}{$2$}

\vspace{0.4cm}

\task{19}{Розв'яжіть рівняння $(4x + 12)^2 = 0$. \gentask}
\answerTable{$-3$; $3$}{$3$}{$-3$}{$0$}{$12$}

\vspace{0.4cm}

\task{20}{Розв'яжіть рівняння $(7x - 21)^2 = 0$. \gentask}
\answerTable{$21$}{$0$}{$-3$; $3$}{$-3$}{$3$}

\vspace{0.4cm}

%======================================================================
% БЛОК 3: СУМА/ДОБУТОК КОРЕНІВ (ТЕОРЕМА ВІЄТА)
%======================================================================

\newpage

\begin{center}
{\large\textbf{\color{headerblue}Блок 3: Сума та добуток коренів (теорема Вієта)}}
\end{center}

\vspace{0.3cm}

% --- Стиль: "Знайдіть суму коренів" ---

\task{21}{Знайдіть суму коренів рівняння $x^2 - 7x + 12 = 0$. \gentask}
\answerTable{$-12$}{$7$}{$3$}{$12$}{$-7$}

\vspace{0.4cm}

\task{22}{Знайдіть суму коренів рівняння $x^2 + 5x - 14 = 0$. \gentask}
\answerTable{$14$}{$-5$}{$2$}{$5$}{$-14$}

\vspace{0.4cm}

\task{23}{Знайдіть суму коренів рівняння $x^2 - 10x + 21 = 0$. \gentask}
\answerTable{$7$}{$10$}{$21$}{$-10$}{$-21$}

\vspace{0.4cm}

\task{24}{Знайдіть суму коренів рівняння $2x^2 - 8x + 6 = 0$. \gentask}
\answerTable{$3$}{$8$}{$-8$}{$-4$}{$4$}

\vspace{0.4cm}

\task{25}{Знайдіть суму коренів рівняння $3x^2 + 12x - 15 = 0$. \gentask}
\answerTable{$-4$}{$-5$}{$12$}{$4$}{$-12$}

\vspace{0.4cm}

% --- Стиль: "Знайдіть добуток коренів" ---

\task{26}{Знайдіть добуток коренів рівняння $x^2 - 6x + 8 = 0$. \gentask}
\answerTable{$-8$}{$-6$}{$6$}{$2$}{$8$}

\vspace{0.4cm}

\task{27}{Знайдіть добуток коренів рівняння $x^2 + 3x - 18 = 0$. \gentask}
\answerTable{$-6$}{$-18$}{$18$}{$3$}{$-3$}

\vspace{0.4cm}

\task{28}{Знайдіть добуток коренів рівняння $x^2 - 9x + 20 = 0$. \gentask}
\answerTable{$20$}{$9$}{$4$}{$-20$}{$-9$}

\vspace{0.4cm}

\task{29}{Знайдіть добуток коренів рівняння $2x^2 - 10x + 12 = 0$. \gentask}
\answerTable{$-12$}{$6$}{$5$}{$-6$}{$12$}

\vspace{0.4cm}

\task{30}{Знайдіть добуток коренів рівняння $5x^2 + 20x - 25 = 0$. \gentask}
\answerTable{$5$}{$25$}{$-5$}{$-4$}{$-25$}

\vspace{0.4cm}

%======================================================================
% БЛОК 4: ОБЧИСЛЕННЯ ДИСКРИМІНАНТА
%======================================================================

\newpage

\begin{center}
{\large\textbf{\color{headerblue}Блок 4: Обчислення дискримінанта}}
\end{center}

\vspace{0.3cm}

% --- Стиль: "Обчисліть дискримінант" ---

\task{31}{Обчисліть дискримінант рівняння $x^2 - 6x + 5 = 0$. \gentask}
\answerTable{$4$}{$16$}{$-16$}{$36$}{$56$}

\vspace{0.4cm}

\task{32}{Обчисліть дискримінант рівняння $x^2 + 4x - 12 = 0$. \gentask}
\answerTable{$32$}{$16$}{$64$}{$-64$}{$8$}

\vspace{0.4cm}

\task{33}{Обчисліть дискримінант рівняння $2x^2 - 5x + 2 = 0$. \gentask}
\answerTable{$25$}{$3$}{$-9$}{$41$}{$9$}

\vspace{0.4cm}

\task{34}{Обчисліть дискримінант рівняння $3x^2 + 2x - 5 = 0$. \gentask}
\answerTable{$8$}{$64$}{$56$}{$-64$}{$4$}

\vspace{0.4cm}

\task{35}{Обчисліть дискримінант рівняння $x^2 - 8x + 16 = 0$. \gentask}
\answerTable{$64$}{$128$}{$-128$}{$0$}{$16$}

\vspace{0.4cm}

% --- Стиль: "Скільки коренів має рівняння" ---

\task{36}{Скільки коренів має рівняння $x^2 - 4x + 5 = 0$? \gentask}
\answerTable{$\infty$}{$-1$}{$1$}{$2$}{$0$}

\vspace{0.4cm}

\task{37}{Скільки коренів має рівняння $x^2 - 6x + 9 = 0$? \gentask}
\answerTable{$\infty$}{$2$}{$1$}{$0$}{$3$}

\vspace{0.4cm}

\task{38}{Скільки коренів має рівняння $x^2 + 2x - 8 = 0$? \gentask}
\answerTable{$1$}{$2$}{$0$}{$\infty$}{$-2$}

\vspace{0.4cm}

\task{39}{Скільки коренів має рівняння $4x^2 + 4x + 1 = 0$? \gentask}
\answerTable{$0$}{$1$}{$4$}{$2$}{$\infty$}

\vspace{0.4cm}

\task{40}{Скільки коренів має рівняння $x^2 + x + 1 = 0$? \gentask}
\answerTable{$\infty$}{$2$}{$0$}{$-1$}{$1$}

\vspace{0.4cm}

%======================================================================
% БЛОК 5: ПОВНІ КВАДРАТНІ РІВНЯННЯ
%======================================================================

\newpage

\begin{center}
{\large\textbf{\color{headerblue}Блок 5: Повні квадратні рівняння}}
\end{center}

\vspace{0.3cm}

% --- Стиль: "Розв'яжіть рівняння" ---

\task{41}{Розв'яжіть рівняння $x^2 - 5x + 6 = 0$. \gentask}
\answerTable{$-2$; $3$}{$-2$; $-3$}{$2$; $3$}{$2$; $-3$}{$1$; $6$}

\vspace{0.4cm}

\task{42}{Розв'яжіть рівняння $x^2 + 7x + 10 = 0$. \gentask}
\answerTable{$-1$; $-10$}{$-2$; $-5$}{$2$; $5$}{$2$; $-5$}{$-2$; $5$}

\vspace{0.4cm}

\task{43}{Розв'яжіть рівняння $x^2 - x - 12 = 0$. \gentask}
\answerTable{$3$; $-4$}{$3$; $4$}{$-3$; $4$}{$-3$; $-4$}{$-2$; $6$}

\vspace{0.4cm}

\task{44}{Розв'яжіть рівняння $x^2 + 2x - 15 = 0$. \gentask}
\answerTable{$-5$; $3$}{$-5$; $-3$}{$-1$; $15$}{$5$; $-3$}{$5$; $3$}

\vspace{0.4cm}

\task{45}{Розв'яжіть рівняння $x^2 - 9x + 18 = 0$. \gentask}
\answerTable{$3$; $-6$}{$3$; $6$}{$2$; $9$}{$-3$; $6$}{$-3$; $-6$}

\vspace{0.4cm}

% --- Стиль: "Укажіть більший корінь" ---

\task{46}{Укажіть більший корінь рівняння $x^2 - 4x - 21 = 0$. \gentask}
\answerTable{$-7$}{$21$}{$-3$}{$7$}{$3$}

\vspace{0.4cm}

\task{47}{Укажіть більший корінь рівняння $x^2 + 3x - 28 = 0$. \gentask}
\answerTable{$4$}{$-4$}{$28$}{$7$}{$-7$}

\vspace{0.4cm}

\task{48}{Укажіть більший корінь рівняння $x^2 - 2x - 35 = 0$. \gentask}
\answerTable{$-7$}{$5$}{$-5$}{$35$}{$7$}

\vspace{0.4cm}

\task{49}{Укажіть менший корінь рівняння $x^2 + 5x - 24 = 0$. \gentask}
\answerTable{$-3$}{$-24$}{$8$}{$-8$}{$3$}

\vspace{0.4cm}

\task{50}{Укажіть менший корінь рівняння $x^2 - 6x - 16 = 0$. \gentask}
\answerTable{$-2$}{$2$}{$8$}{$-8$}{$-16$}

\vspace{0.4cm}

%======================================================================
% БЛОК 6: РІВНЯННЯ З ПАРАМЕТРОМ (ax + b)² = c
%======================================================================

\newpage

\begin{center}
{\large\textbf{\color{headerblue}Блок 6: Рівняння виду $a(x + b)^2 = c$}}
\end{center}

\vspace{0.3cm}

% --- Стиль: "Знайдіть суму коренів" ---

\task{51}{Знайдіть суму коренів рівняння $2(x + 1)^2 = 32$. \gentask}
\answerTable{$0$}{$-2$}{$2$}{$6$}{$-6$}

\vspace{0.4cm}

\task{52}{Знайдіть суму коренів рівняння $3(x - 2)^2 = 48$. \gentask}
\answerTable{$0$}{$4$}{$-4$}{$-8$}{$8$}

\vspace{0.4cm}

\task{53}{Знайдіть суму коренів рівняння $5(x + 3)^2 = 125$. \gentask}
\answerTable{$-10$}{$10$}{$6$}{$0$}{$-6$}

\vspace{0.4cm}

\task{54}{Знайдіть суму коренів рівняння $4(x - 1)^2 = 64$. \gentask}
\answerTable{$2$}{$0$}{$6$}{$-6$}{$-2$}

\vspace{0.4cm}

\task{55}{Знайдіть суму коренів рівняння $2(x + 4)^2 = 72$. \gentask}
\answerTable{$-8$}{$-14$}{$0$}{$8$}{$14$}

\vspace{0.4cm}

% --- Стиль: "Обчисліть суму коренів" ---

\task{56}{Обчисліть суму коренів рівняння $10(x + 2)^2 - 90 = 0$. \gentask}
\answerTable{$-4$}{$7$}{$4$}{$-7$}{$0$}

\vspace{0.4cm}

\task{57}{Обчисліть суму коренів рівняння $6(x - 3)^2 - 54 = 0$. \gentask}
\answerTable{$9$}{$6$}{$-6$}{$0$}{$-9$}

\vspace{0.4cm}

\task{58}{Обчисліть суму коренів рівняння $8(x + 5)^2 - 200 = 0$. \gentask}
\answerTable{$-10$}{$15$}{$0$}{$-15$}{$10$}

\vspace{0.4cm}

\task{59}{Обчисліть суму коренів рівняння $12(x - 4)^2 - 108 = 0$. \gentask}
\answerTable{$11$}{$8$}{$-8$}{$0$}{$-11$}

\vspace{0.4cm}

\task{60}{Обчисліть суму коренів рівняння $15(x + 1)^2 - 375 = 0$. \gentask}
\answerTable{$6$}{$-6$}{$0$}{$2$}{$-2$}

\vspace{0.4cm}

%======================================================================
% БЛОК 7: БІКВАДРАТНІ РІВНЯННЯ
%======================================================================

\newpage

\begin{center}
{\large\textbf{\color{headerblue}Блок 7: Біквадратні рівняння}}
\end{center}

\vspace{0.3cm}

% --- Стиль: "Розв'яжіть рівняння" ---

\task{61}{Розв'яжіть рівняння $x^4 - 13x^2 + 36 = 0$. \gentask}
\answerTable{$-3$; $-2$; $2$; $3$}{$\varnothing$}{$2$; $3$}{$-3$; $3$}{$-2$; $2$}

\vspace{0.4cm}

\task{62}{Розв'яжіть рівняння $x^4 - 10x^2 + 9 = 0$. \gentask}
\answerTable{$-3$; $-1$; $1$; $3$}{$\varnothing$}{$-1$; $1$}{$-3$; $3$}{$1$; $3$}

\vspace{0.4cm}

\task{63}{Розв'яжіть рівняння $x^4 - 5x^2 + 4 = 0$. \gentask}
\answerTable{$-2$; $2$}{$\varnothing$}{$-1$; $1$}{$-2$; $-1$; $1$; $2$}{$1$; $2$}

\vspace{0.4cm}

\task{64}{Розв'яжіть рівняння $x^4 - 17x^2 + 16 = 0$. \gentask}
\answerTable{$-4$; $-1$; $1$; $4$}{$\varnothing$}{$-1$; $1$}{$-4$; $4$}{$1$; $4$}

\vspace{0.4cm}

\task{65}{Розв'яжіть рівняння $x^4 - 20x^2 + 64 = 0$. \gentask}
\answerTable{$\varnothing$}{$-4$; $4$}{$-4$; $-2$; $2$; $4$}{$2$; $4$}{$-2$; $2$}

\vspace{0.4cm}

% --- Стиль: "Укажіть різницю найбільшого і найменшого коренів" ---

\task{66}{Укажіть різницю найбільшого і найменшого коренів рівняння $x^4 - 10x^2 + 9 = 0$. \gentask}
\answerTable{$0$}{$8$}{$2$}{$4$}{$6$}

\vspace{0.4cm}

\task{67}{Укажіть різницю найбільшого і найменшого коренів рівняння $x^4 - 13x^2 + 36 = 0$. \gentask}
\answerTable{$5$}{$0$}{$1$}{$4$}{$6$}

\vspace{0.4cm}

\task{68}{Укажіть різницю найбільшого і найменшого коренів рівняння $x^4 - 5x^2 + 4 = 0$. \gentask}
\answerTable{$4$}{$2$}{$3$}{$0$}{$1$}

\vspace{0.4cm}

\task{69}{Укажіть суму всіх коренів рівняння $x^4 - 25x^2 + 144 = 0$. \gentask}
\answerTable{$7$}{$0$}{$-10$}{$-7$}{$10$}

\vspace{0.4cm}

\task{70}{Укажіть добуток всіх коренів рівняння $x^4 - 10x^2 + 9 = 0$. \gentask}
\answerTable{$0$}{$9$}{$-9$}{$10$}{$-10$}

\vspace{0.4cm}

%======================================================================
% БЛОК 8: ДРОБОВО-РАЦІОНАЛЬНІ РІВНЯННЯ
%======================================================================

\newpage

\begin{center}
{\large\textbf{\color{headerblue}Блок 8: Дробово-раціональні рівняння}}
\end{center}

\vspace{0.3cm}

% --- Стиль: "Розв'яжіть рівняння" (дріб = 0) ---

\task{71}{Розв'яжіть рівняння $\dfrac{x^2 - 9}{x + 3} = 0$. \gentask}
\answerTable{$-3$; $3$}{$-3$}{$0$}{$9$}{$3$}

\vspace{0.4cm}

\task{72}{Розв'яжіть рівняння $\dfrac{x^2 - 16}{x - 4} = 0$. \gentask}
\answerTable{$16$}{$-4$; $4$}{$0$}{$-4$}{$4$}

\vspace{0.4cm}

\task{73}{Розв'яжіть рівняння $\dfrac{x^2 - 25}{x + 5} = 0$. \gentask}
\answerTable{$-5$}{$0$}{$5$}{$25$}{$-5$; $5$}

\vspace{0.4cm}

\task{74}{Розв'яжіть рівняння $\dfrac{x^2 - 4}{x + 2} = 0$. \gentask}
\answerTable{$2$}{$-2$}{$-2$; $2$}{$0$}{$4$}

\vspace{0.4cm}

\task{75}{Розв'яжіть рівняння $\dfrac{x^2 - 36}{x - 6} = 0$. \gentask}
\answerTable{$-6$; $6$}{$-6$}{$36$}{$0$}{$6$}

\vspace{0.4cm}

% --- Стиль: Рівняння виду a/(x-b) + c = x ---

\task{76}{Розв'яжіть рівняння $\dfrac{4}{x - 1} + 1 = x$. \gentask}
\answerTable{$-1$}{$3$}{$1$; $3$}{$-1$; $3$}{$1$; $-3$}

\vspace{0.4cm}

\task{77}{Розв'яжіть рівняння $\dfrac{6}{x + 2} - 1 = x$. \gentask}
\answerTable{$-4$}{$-2$; $1$}{$-4$; $1$}{$4$; $-1$}{$1$}

\vspace{0.4cm}

\task{78}{Розв'яжіть рівняння $\dfrac{8}{x - 2} + 2 = x$. \gentask}
\answerTable{$6$}{$-2$; $6$}{$-2$}{$2$; $6$}{$2$; $-6$}

\vspace{0.4cm}

\task{79}{Розв'яжіть рівняння $\dfrac{12}{x + 3} - 1 = x$. \gentask}
\answerTable{$-6$}{$2$}{$-6$; $2$}{$-3$; $2$}{$6$; $-2$}

\vspace{0.4cm}

\task{80}{Розв'яжіть рівняння $\dfrac{10}{x - 1} + 3 = x$. \gentask}
\answerTable{$2$; $-6$}{$-2$; $6$}{$6$}{$-2$}{$1$; $6$}

\vspace{0.4cm}

%======================================================================
% БЛОК 9: ПРОМІЖОК ДЛЯ КОРЕНЯ
%======================================================================

\newpage

\begin{center}
{\large\textbf{\color{headerblue}Блок 9: Визначення проміжку для кореня}}
\end{center}

\vspace{0.3cm}

% --- Стиль: "Укажіть проміжок, якому належить більший корінь" ---

\task{81}{Укажіть проміжок, якому належить більший корінь рівняння $x^2 - 5x + 4 = 0$. \gentask}
\answerTable{$(3; 5]$}{$(5; 7]$}{$(7; 10]$}{$(1; 3]$}{$(-1; 1]$}

\vspace{0.4cm}

\task{82}{Укажіть проміжок, якому належить більший корінь рівняння $x^2 - 7x + 10 = 0$. \gentask}
\answerTable{$(8; 10]$}{$(0; 2]$}{$(2; 4]$}{$(4; 6]$}{$(6; 8]$}

\vspace{0.4cm}

\task{83}{Укажіть проміжок, якому належить більший корінь рівняння $x^2 + 2x - 15 = 0$. \gentask}
\answerTable{$(4; 6]$}{$(6; 10]$}{$(-2; 0]$}{$(2; 4]$}{$(0; 2]$}

\vspace{0.4cm}

\task{84}{Укажіть проміжок, якому належить менший корінь рівняння $x^2 - 3x - 10 = 0$. \gentask}
\answerTable{$(3; 6]$}{$(0; 3]$}{$[-3; 0)$}{$[-6; -3)$}{$[-10; -6)$}

\vspace{0.4cm}

\task{85}{Укажіть проміжок, якому належить менший корінь рівняння $x^2 + 4x - 12 = 0$. \gentask}
\answerTable{$[-10; -7)$}{$(0; 3]$}{$[-7; -5)$}{$(-3; 0]$}{$[-5; -3)$}

\vspace{0.4cm}

% --- Стиль: з числом пі ---

\task{86}{Більший корінь рівняння $x^2 - 6x + 5 = 0$ належить проміжку \gentask}
\answerTableBig{$(4; 2\pi)$}{$(2\pi; 8)$}{$(0; \pi)$}{$(8; 10)$}{$(\pi; 4)$}

\vspace{0.4cm}

\task{87}{Менший корінь рівняння $x^2 - 4x + 3 = 0$ належить проміжку \gentask}
\answerTableBig{$(0; \dfrac{\pi}{2})$}{$(\dfrac{\pi}{2}; 2)$}{$(4; 2\pi)$}{$(\pi; 4)$}{$(2; \pi)$}

\vspace{0.4cm}

\task{88}{Більший корінь рівняння $x^2 - 8x + 12 = 0$ належить проміжку \gentask}
\answerTableBig{$(0; \pi)$}{$(8; 10)$}{$(5; 2\pi)$}{$(2\pi; 8)$}{$(\pi; 5)$}

\vspace{0.4cm}

\task{89}{Менший корінь рівняння $x^2 - 7x + 6 = 0$ належить проміжку \gentask}
\answerTableBig{$(4; 2\pi)$}{$(0; \dfrac{\pi}{2})$}{$(\dfrac{\pi}{2}; 2)$}{$(\pi; 4)$}{$(2; \pi)$}

\vspace{0.4cm}

\task{90}{Більший корінь рівняння $(x-2)^2 = 9$ належить проміжку \gentask}
\answerTableBig{$(2\pi; 8)$}{$(4; 2\pi)$}{$(0; \pi)$}{$(\pi; 4)$}{$(8; 10)$}

\vspace{0.4cm}

%======================================================================
% БЛОК 10: РІВНЯННЯ, ЩО ЗВОДЯТЬСЯ ДО КВАДРАТНИХ
%======================================================================

\newpage

\begin{center}
{\large\textbf{\color{headerblue}Блок 10: Рівняння, що зводяться до квадратних}}
\end{center}

\vspace{0.3cm}

% --- Стиль: "Розв'яжіть рівняння" x(ax + b) = ax + b ---

\task{91}{Розв'яжіть рівняння $x(3x + 6) = 3x + 6$. \gentask}
\answerTable{$-2$; $1$}{$2$; $-1$}{$2$; $1$}{$-2$}{$1$}

\vspace{0.4cm}

\task{92}{Розв'яжіть рівняння $x(2x - 4) = 2x - 4$. \gentask}
\answerTable{$-1$; $-2$}{$1$; $2$}{$1$}{$-2$; $1$}{$2$}

\vspace{0.4cm}

\task{93}{Розв'яжіть рівняння $x(4x + 8) = 4x + 8$. \gentask}
\answerTable{$-2$; $1$}{$2$; $-1$}{$1$}{$2$; $1$}{$-2$}

\vspace{0.4cm}

\task{94}{Розв'яжіть рівняння $x(5x - 10) = 5x - 10$. \gentask}
\answerTable{$1$; $2$}{$-1$; $-2$}{$-2$; $1$}{$2$}{$1$}

\vspace{0.4cm}

\task{95}{Розв'яжіть рівняння $x(2x + 10) = 2x + 10$. \gentask}
\answerTable{$1$}{$-5$; $1$}{$5$; $-1$}{$5$; $1$}{$-5$}

\vspace{0.4cm}

% --- Стиль: Рівняння виду a(x² - x) = x² + b ---

\task{96}{Розв'яжіть рівняння $2(x^2 - x) = x^2 + 2$. \gentask}
\answerTable{$-1$; $2$}{$-1$}{$2$}{$\varnothing$}{$1$; $-2$}

\vspace{0.4cm}

\task{97}{Розв'яжіть рівняння $3(x^2 + x) = 2x^2 + 6$. \gentask}
\answerTable{$-3$; $2$}{$-3$}{$3$; $-2$}{$2$}{$\varnothing$}

\vspace{0.4cm}

\task{98}{Розв'яжіть рівняння $2(x^2 + 2x) = x^2 + 8$. \gentask}
\answerTable{$2$}{$-4$; $2$}{$4$; $-2$}{$\varnothing$}{$-4$}

\vspace{0.4cm}

\task{99}{Розв'яжіть рівняння $4(x^2 - x) = 3x^2 + 5$. \gentask}
\answerTable{$-1$; $5$}{$5$}{$\varnothing$}{$1$; $-5$}{$-1$}

\vspace{0.4cm}

\task{100}{Розв'яжіть рівняння $3(x^2 - 2x) = 2x^2 + 3$. \gentask}
\answerTable{$-1$}{$1$; $-3$}{$-1$; $3$}{$\varnothing$}{$3$}

\vspace{0.4cm}

\end{document}
