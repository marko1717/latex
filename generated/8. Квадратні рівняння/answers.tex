\documentclass[12pt]{extarticle}
\usepackage{fontspec}
\usepackage{polyglossia}
\setdefaultlanguage{ukrainian}

\defaultfontfeatures{Ligatures=TeX}
\setmainfont{Liberation Serif}

\usepackage[a4paper,margin=2cm]{geometry}
\usepackage{amsmath,amssymb}
\usepackage{multicol}
\usepackage{xcolor}

\definecolor{headerblue}{RGB}{0, 102, 204}

\begin{document}

\begin{center}
{\Large\textbf{\color{headerblue}ВІДПОВІДІ}}
\end{center}

\begin{center}
{\large Тема 8: Квадратні рівняння та рівняння, що зводяться до квадратних}
\end{center}

\vspace{0.5cm}

\textbf{Блок 1: Неповні квадратні рівняння $x^2 = a$}

\begin{multicols}{5}
\noindent
1. А \\
2. А \\
3. А \\
4. А \\
5. А \\
6. А \\
7. А \\
8. А \\
9. А \\
10. А
\end{multicols}

\textbf{Блок 2: Рівняння виду $(ax + b)^2 = c$}

\begin{multicols}{5}
\noindent
11. А \\
12. А \\
13. А \\
14. А \\
15. А \\
16. А \\
17. А \\
18. А \\
19. А \\
20. А
\end{multicols}

\textbf{Блок 3: Сума та добуток коренів (теорема Вієта)}

\begin{multicols}{5}
\noindent
21. А ($7$) \\
22. А ($-5$) \\
23. А ($10$) \\
24. А ($4$) \\
25. А ($-4$) \\
26. А ($8$) \\
27. А ($-18$) \\
28. А ($20$) \\
29. А ($6$) \\
30. А ($-5$)
\end{multicols}

\textbf{Блок 4: Обчислення дискримінанта}

\begin{multicols}{5}
\noindent
31. А ($16$) \\
32. А ($64$) \\
33. А ($9$) \\
34. А ($64$) \\
35. А ($0$) \\
36. А ($0$) \\
37. А ($1$) \\
38. А ($2$) \\
39. А ($1$) \\
40. А ($0$)
\end{multicols}

\textbf{Блок 5: Повні квадратні рівняння}

\begin{multicols}{5}
\noindent
41. А \\
42. А \\
43. А \\
44. А \\
45. А \\
46. А ($7$) \\
47. А ($4$) \\
48. А ($7$) \\
49. А ($-8$) \\
50. А ($-2$)
\end{multicols}

\textbf{Блок 6: Рівняння виду $a(x + b)^2 = c$}

\begin{multicols}{5}
\noindent
51. А ($-2$) \\
52. А ($4$) \\
53. А ($-6$) \\
54. А ($2$) \\
55. А ($-8$) \\
56. А ($-4$) \\
57. А ($6$) \\
58. А ($-10$) \\
59. А ($8$) \\
60. А ($-2$)
\end{multicols}

\textbf{Блок 7: Біквадратні рівняння}

\begin{multicols}{5}
\noindent
61. А \\
62. А \\
63. А \\
64. А \\
65. А \\
66. А ($6$) \\
67. А ($6$) \\
68. А ($4$) \\
69. А ($0$) \\
70. А ($9$)
\end{multicols}

\textbf{Блок 8: Дробово-раціональні рівняння}

\begin{multicols}{5}
\noindent
71. А ($3$) \\
72. А ($-4$) \\
73. А ($5$) \\
74. А ($2$) \\
75. А ($-6$) \\
76. А \\
77. А \\
78. А \\
79. А \\
80. А
\end{multicols}

\textbf{Блок 9: Визначення проміжку для кореня}

\begin{multicols}{5}
\noindent
81. А \\
82. А \\
83. А \\
84. А \\
85. А \\
86. А \\
87. А \\
88. А \\
89. А \\
90. А
\end{multicols}

\textbf{Блок 10: Рівняння, що зводяться до квадратних}

\begin{multicols}{5}
\noindent
91. А \\
92. А \\
93. А \\
94. А \\
95. А \\
96. А \\
97. А \\
98. А \\
99. А \\
100. А
\end{multicols}

\vspace{1cm}

\textbf{Ключові формули:}

\begin{enumerate}
\item \textbf{Теорема Вієта:} $x_1 + x_2 = -\dfrac{b}{a}$, \quad $x_1 \cdot x_2 = \dfrac{c}{a}$

\item \textbf{Дискримінант:} $D = b^2 - 4ac$
\begin{itemize}
\item $D > 0$ --- два різних корені
\item $D = 0$ --- один корінь (кратності 2)
\item $D < 0$ --- немає дійсних коренів
\end{itemize}

\item \textbf{Формула коренів:} $x = \dfrac{-b \pm \sqrt{D}}{2a}$

\item \textbf{Біквадратне рівняння:} Заміна $t = x^2$, $t \geq 0$

\item \textbf{Дробове рівняння $\dfrac{f(x)}{g(x)} = 0$:} $f(x) = 0$ при $g(x) \neq 0$
\end{enumerate}

\end{document}
