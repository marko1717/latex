\documentclass[14pt]{extarticle}
\usepackage{fontspec}
\usepackage{polyglossia}
\setdefaultlanguage{ukrainian}

\defaultfontfeatures{Ligatures=TeX}
\setmainfont{Liberation Serif}
\setsansfont{Liberation Sans}
\setmonofont{Liberation Mono}

\usepackage[a4paper,margin=2cm,bottom=2.5cm,top=2.5cm]{geometry}
\usepackage{amsmath,amssymb}
\usepackage{xcolor}
\usepackage{array}
\usepackage{fancyhdr}

\definecolor{headerblue}{RGB}{0, 102, 204}
\definecolor{yearcolor}{RGB}{128, 0, 128}

\pagestyle{fancy}
\fancyhf{}
\renewcommand{\headrulewidth}{0pt}
\fancyfoot[C]{\thepage}

\newcommand{\answerTable}[5]{
\begin{center}
\begin{tabular}{|*{5}{>{\centering\arraybackslash}m{2.8cm}|}}
\hline
\rule[-0.3cm]{0pt}{0.8cm}\textbf{А} & \textbf{Б} & \textbf{В} & \textbf{Г} & \textbf{Д} \\
\hline
\rule[-0.4cm]{0pt}{1.0cm}#1 & \rule[-0.4cm]{0pt}{1.0cm}#2 & \rule[-0.4cm]{0pt}{1.0cm}#3 & \rule[-0.4cm]{0pt}{1.0cm}#4 & \rule[-0.4cm]{0pt}{1.0cm}#5 \\
\hline
\end{tabular}
\end{center}
}

\newcommand{\task}[2]{\noindent\makebox[1.5em][l]{\textbf{#1.}}\parbox[t]{\dimexpr\textwidth-1.5em}{#2}}
\newcommand{\nmtyear}[1]{\hfill{\small\color{yearcolor}(#1)}}

\begin{document}

\begin{center}
{\Large\textbf{\color{headerblue}ВАРІАЦІЇ СТИЛІВ ЗАВДАНЬ}}
\end{center}

\begin{center}
{\large Тема: \textbf{Числа. Дроби. Модуль числа}}
\end{center}

\vspace{0.5cm}

%======================================================================
% СТИЛЬ А: ЗВОРОТНІ ЗАВДАННЯ
%======================================================================

\begin{center}
{\large\textbf{\color{headerblue}Стиль А: Зворотні завдання}}
\end{center}

\vspace{0.3cm}

% Оригінал: "Скільки цілих чисел на проміжку [-3; 5)?"
\task{1}{На проміжку міститься рівно 7 цілих чисел, найменше з яких $-2$. Який це проміжок? \nmtyear{gen-A}}
\answerTable{$[-2; 5)$}{$[-2; 4]$}{$[-2; 5]$}{$(-2; 5)$}{$[-2; 6)$}
% Відповідь: А (числа: -2,-1,0,1,2,3,4)

\vspace{0.5cm}

\task{2}{На проміжку міститься рівно 5 цілих чисел. Яка найменша довжина цього проміжку? \nmtyear{gen-A}}
\answerTable{$4$}{$5$}{$6$}{$3$}{$4{,}5$}
% Відповідь: А (напр. [0;4] має 5 чисел, довжина 4)

\vspace{0.5cm}

\task{3}{Модуль якого цілого числа найближчий до $3{,}7$? \nmtyear{gen-A}}
\answerTable{$3$}{$4$}{$-4$}{$4$ або $-4$}{$3$ або $-3$}
% Відповідь: Г

\vspace{0.5cm}

% Оригінал: "|x - 3| = 5, знайти x"
\task{4}{Рівняння $|x - a| = 4$ має розв'язки $x = 1$ і $x = 9$. Знайдіть $a$. \nmtyear{gen-A}}
\answerTable{$4$}{$5$}{$8$}{$-3$}{$3$}
% Відповідь: Б (середина між 1 і 9 = 5)

\vspace{0.5cm}

\task{5}{Рівняння $|x + b| = 3$ має розв'язки $x = -5$ і $x = 1$. Знайдіть $b$. \nmtyear{gen-A}}
\answerTable{$-2$}{$2$}{$3$}{$-3$}{$4$}
% Відповідь: Б (середина -2, отже x + 2 = ±3)

\vspace{0.5cm}

%======================================================================
% СТИЛЬ Б: ПРАКТИЧНИЙ КОНТЕКСТ
%======================================================================

\begin{center}
{\large\textbf{\color{headerblue}Стиль Б: Практичний контекст}}
\end{center}

\vspace{0.3cm}

\task{6}{Температура вранці була $-5°$C, а ввечері $+3°$C. На скільки градусів змінилась температура? \nmtyear{gen-B}}
\answerTable{$2$}{$-2$}{$8$}{$-8$}{$15$}
% Відповідь: В

\vspace{0.5cm}

\task{7}{Підводний човен занурився на глибину 120 м (позначимо як $-120$), а потім піднявся на 45 м. На якій глибині він опинився? \nmtyear{gen-B}}
\answerTable{$-165$ м}{$-75$ м}{$75$ м}{$-45$ м}{$165$ м}
% Відповідь: Б

\vspace{0.5cm}

\task{8}{Борг Петра становить 500 грн ($-500$), а в Марії на рахунку 300 грн ($+300$). Яка різниця між їхніми станами рахунків? \nmtyear{gen-B}}
\answerTable{$200$ грн}{$-200$ грн}{$800$ грн}{$-800$ грн}{$500$ грн}
% Відповідь: В (300 - (-500) = 800)

\vspace{0.5cm}

\task{9}{Ліфт спустився з 5-го поверху на 3 поверхи вниз, потім піднявся на 7 поверхів. На якому поверсі він зупинився? \nmtyear{gen-B}}
\answerTable{$5$}{$9$}{$7$}{$15$}{$2$}
% Відповідь: Б (5 - 3 + 7 = 9)

\vspace{0.5cm}

\task{10}{Відстань від точки $A(-3)$ до точки $B(5)$ на числовій прямій дорівнює: \nmtyear{gen-B}}
\answerTable{$2$}{$-8$}{$8$}{$-2$}{$15$}
% Відповідь: В (|5 - (-3)| = 8)

\vspace{0.5cm}

\task{11}{Термометр показував $+12°$C, а потім температура впала на $|{-}15|$ градусів. Яка температура зараз? \nmtyear{gen-B}}
\answerTable{$-3°$C}{$27°$C}{$3°$C}{$-27°$C}{$-15°$C}
% Відповідь: А (12 - 15 = -3)

\vspace{0.5cm}

%======================================================================
% СТИЛЬ В: ЗНАЙТИ ПОМИЛКУ
%======================================================================

\begin{center}
{\large\textbf{\color{headerblue}Стиль В: Знайти помилку}}
\end{center}

\vspace{0.3cm}

\task{12}{Учень стверджує, що $|-7| = -7$. Яка правильна відповідь? \nmtyear{gen-V}}
\answerTable{$-7$}{$7$}{$0$}{$\pm 7$}{$49$}
% Відповідь: Б

\vspace{0.5cm}

\task{13}{Учень порахував: $-3 - (-5) = -8$. Яка правильна відповідь? \nmtyear{gen-V}}
\answerTable{$-8$}{$8$}{$2$}{$-2$}{$15$}
% Відповідь: В (-3 + 5 = 2)

\vspace{0.5cm}

\task{14}{Учень стверджує, що на проміжку $(-2; 3)$ є 6 цілих чисел. Скільки насправді? \nmtyear{gen-V}}
\answerTable{$6$}{$5$}{$4$}{$3$}{$7$}
% Відповідь: В (числа: -1, 0, 1, 2)

\vspace{0.5cm}

\task{15}{Учень записав: $|a - b| = |a| - |b|$. Це твердження: \nmtyear{gen-V}}

\vspace{0.2cm}
\begin{tabular}{ll}
\textbf{А} & завжди правильне \\
\textbf{Б} & завжди неправильне \\
\textbf{В} & правильне лише при $a \geq 0$ і $b \geq 0$ \\
\textbf{Г} & правильне лише при $ab \geq 0$ і $|a| \geq |b|$ \\
\textbf{Д} & правильне лише при $a = b$ \\
\end{tabular}
% Відповідь: Г

\vspace{0.7cm}

\task{16}{Учень стверджує: <<Число $-0{,}5$ більше за $-0{,}3$>>. Це твердження: \nmtyear{gen-V}}
\answerTable{правильне}{неправильне}{залежить від контексту}{не має сенсу}{правильне лише для цілих}
% Відповідь: Б (-0.5 < -0.3)

\vspace{0.5cm}

%======================================================================
% СТИЛЬ Г: ПОРІВНЯННЯ ТА ТВЕРДЖЕННЯ
%======================================================================

\begin{center}
{\large\textbf{\color{headerblue}Стиль Г: Порівняння та твердження}}
\end{center}

\vspace{0.3cm}

\task{17}{Яке твердження правильне? \nmtyear{gen-G}}

\vspace{0.2cm}
\begin{tabular}{ll}
\textbf{А} & $|a| \geq 0$ для будь-якого $a$ \\
\textbf{Б} & $|a| > 0$ для будь-якого $a$ \\
\textbf{В} & $|a| = a$ для будь-якого $a$ \\
\textbf{Г} & $|a| = -a$ для будь-якого $a$ \\
\textbf{Д} & $|-a| = -|a|$ для будь-якого $a$ \\
\end{tabular}
% Відповідь: А

\vspace{0.7cm}

\task{18}{Порівняйте: $A = |{-}5 + 3|$ і $B = |{-}5| + |3|$. \nmtyear{gen-G}}
\answerTable{$A > B$}{$A < B$}{$A = B$}{Залежить від знаку}{Неможливо порівняти}
% Відповідь: Б (A = 2, B = 8)

\vspace{0.5cm}

\task{19}{Для яких значень $x$ виконується $|x| = x$? \nmtyear{gen-G}}
\answerTable{для всіх $x$}{$x > 0$}{$x \geq 0$}{$x < 0$}{$x = 0$}
% Відповідь: В

\vspace{0.5cm}

\task{20}{Скільки цілих значень $x$ задовольняють нерівність $|x| < 5$? \nmtyear{gen-G}}
\answerTable{$5$}{$9$}{$10$}{$4$}{$8$}
% Відповідь: Б (від -4 до 4, тобто 9 чисел)

\vspace{0.5cm}

\task{21}{Яке з чисел найбільше? \nmtyear{gen-G}}
\answerTable{$-0{,}1$}{$-0{,}01$}{$-1$}{$-0{,}001$}{$-10$}
% Відповідь: Г (-0.001 найближче до 0)

\vspace{0.5cm}

\task{22}{Яке твердження НЕПРАВИЛЬНЕ? \nmtyear{gen-G}}

\vspace{0.2cm}
\begin{tabular}{ll}
\textbf{А} & Сума двох від'ємних чисел --- від'ємне число \\
\textbf{Б} & Добуток двох від'ємних чисел --- додатне число \\
\textbf{В} & Різниця двох від'ємних чисел --- від'ємне число \\
\textbf{Г} & Модуль від'ємного числа --- додатне число \\
\textbf{Д} & Протилежне до від'ємного числа --- додатне число \\
\end{tabular}
% Відповідь: В (може бути і додатною)

\vspace{0.7cm}

%======================================================================
% СТИЛЬ Д: ОБЧИСЛЕННЯ З ПІДСТАНОВКОЮ
%======================================================================

\begin{center}
{\large\textbf{\color{headerblue}Стиль Д: Підстановка}}
\end{center}

\vspace{0.3cm}

\task{23}{При якому значенні $x$ вираз $|x - 4| + |x + 2|$ набуває найменшого значення? \nmtyear{gen-D}}
\answerTable{$x = 4$}{$x = -2$}{$x = 1$}{Будь-яке $x \in [-2; 4]$}{$x = 0$}
% Відповідь: Г (мінімум = 6 при будь-якому x від -2 до 4)

\vspace{0.5cm}

\task{24}{Знайдіть $|a - b| + |b - a|$, якщо $a = 7$ і $b = 3$. \nmtyear{gen-D}}
\answerTable{$0$}{$4$}{$8$}{$10$}{$-8$}
% Відповідь: В (|4| + |-4| = 8)

\vspace{0.5cm}

\task{25}{При якому цілому $x$ значення виразу $|x - 3|$ найменше? \nmtyear{gen-D}}
\answerTable{$0$}{$1$}{$2$}{$3$}{$4$}
% Відповідь: Г

\vspace{0.5cm}

\task{26}{Знайдіть найменше ціле $x$, для якого $|x| > 5$. \nmtyear{gen-D}}
\answerTable{$6$}{$-6$}{$5$}{$-5$}{$0$}
% Відповідь: Б

\vspace{0.5cm}

\task{27}{Скільки цілих розв'язків має рівняння $|x| = |x - 6|$? \nmtyear{gen-D}}
\answerTable{$0$}{$1$}{$2$}{$3$}{безліч}
% Відповідь: Б (x = 3)

\vspace{0.5cm}

%======================================================================
% СТИЛЬ Е: ЛОГІЧНИЙ АНАЛІЗ
%======================================================================

\begin{center}
{\large\textbf{\color{headerblue}Стиль Е: Логічний аналіз}}
\end{center}

\vspace{0.3cm}

\task{28}{Якщо $|a| = |b|$, то обов'язково: \nmtyear{gen-E}}
\answerTable{$a = b$}{$a = -b$}{$a = b$ або $a = -b$}{$a > b$}{$ab > 0$}
% Відповідь: В

\vspace{0.5cm}

\task{29}{Якщо $a < 0$ і $b > 0$, то $|a \cdot b|$ дорівнює: \nmtyear{gen-E}}
\answerTable{$ab$}{$-ab$}{$a \cdot b$}{$|a| \cdot |b|$}{$-|a| \cdot |b|$}
% Відповідь: Б (або Г, бо -ab = |a|·|b|)

\vspace{0.5cm}

\task{30}{Якщо $|x - 2| = 0$, то $x$ дорівнює: \nmtyear{gen-E}}
\answerTable{$0$}{$2$}{$-2$}{$\pm 2$}{не існує}
% Відповідь: Б

\vspace{0.5cm}

\task{31}{Яка умова необхідна і достатня для того, щоб $|a + b| = |a| + |b|$? \nmtyear{gen-E}}
\answerTable{$a = b$}{$a = -b$}{$ab \geq 0$}{$ab > 0$}{$a > 0$ і $b > 0$}
% Відповідь: В

\vspace{0.5cm}

\task{32}{Якщо $a$ --- ціле і $-3 < a < 4$, то сума всіх можливих $a$ дорівнює: \nmtyear{gen-E}}
\answerTable{$0$}{$3$}{$7$}{$-3$}{$1$}
% Відповідь: Б (-2 + -1 + 0 + 1 + 2 + 3 = 3)

\vspace{0.5cm}

%======================================================================
% СТИЛЬ Ж: ЧИСЛОВА ПРЯМА
%======================================================================

\begin{center}
{\large\textbf{\color{headerblue}Стиль Ж: Числова пряма}}
\end{center}

\vspace{0.3cm}

\task{33}{Точка $A$ має координату $-2$, точка $B$ --- координату $6$. Знайдіть координату середини відрізка $AB$. \nmtyear{gen-ZH}}
\answerTable{$2$}{$4$}{$-4$}{$8$}{$3$}
% Відповідь: А ((-2 + 6)/2 = 2)

\vspace{0.5cm}

\task{34}{Точка $M$ рівновіддалена від точок $A(-5)$ і $B(3)$. Знайдіть координату точки $M$. \nmtyear{gen-ZH}}
\answerTable{$-1$}{$1$}{$4$}{$-4$}{$0$}
% Відповідь: А ((-5 + 3)/2 = -1)

\vspace{0.5cm}

\task{35}{Яка з точок $A(-4)$, $B(2)$, $C(-1)$, $D(5)$ найближча до точки $P(1)$? \nmtyear{gen-ZH}}
\answerTable{$A$}{$B$}{$C$}{$D$}{$B$ і $C$}
% Відповідь: Б (відстань до B = 1)

\vspace{0.5cm}

\task{36}{На числовій прямій позначено точки $-3$, $0$, $4$. Яка точка рівновіддалена від двох інших? \nmtyear{gen-ZH}}
\answerTable{$-3$}{$0$}{$4$}{Жодна}{$0{,}5$}
% Відповідь: Г (відстані різні для всіх)

\vspace{0.5cm}

%======================================================================
% СТИЛЬ З: ДРОБИ У КОНТЕКСТІ
%======================================================================

\begin{center}
{\large\textbf{\color{headerblue}Стиль З: Дроби у контексті}}
\end{center}

\vspace{0.3cm}

\task{37}{Марія прочитала $\dfrac{2}{5}$ книги, а Петро --- $\dfrac{3}{8}$ тієї самої книги. Хто прочитав більше? \nmtyear{gen-Z}}
\answerTable{Марія}{Петро}{Однаково}{Недостатньо даних}{Залежить від книги}
% Відповідь: А (2/5 = 16/40 > 15/40 = 3/8)

\vspace{0.5cm}

\task{38}{Яка частина години становить 45 хвилин? \nmtyear{gen-Z}}
\answerTable{$\dfrac{3}{4}$}{$\dfrac{4}{5}$}{$\dfrac{2}{3}$}{$\dfrac{45}{100}$}{$\dfrac{9}{10}$}
% Відповідь: А

\vspace{0.5cm}

\task{39}{Бак автомобіля вміщує 60 л. Скільки літрів залишилось, якщо витрачено $\dfrac{5}{6}$ бака? \nmtyear{gen-Z}}
\answerTable{$50$ л}{$10$ л}{$12$ л}{$48$ л}{$6$ л}
% Відповідь: Б (60 · 1/6 = 10)

\vspace{0.5cm}

\task{40}{У класі 30 учнів. $\dfrac{2}{5}$ з них --- хлопці. Скільки дівчат у класі? \nmtyear{gen-Z}}
\answerTable{$12$}{$18$}{$15$}{$20$}{$10$}
% Відповідь: Б (хлопців 12, дівчат 18)

\vspace{0.5cm}

\task{41}{Турист пройшов $\dfrac{3}{7}$ шляху, що становить 12 км. Яка довжина всього шляху? \nmtyear{gen-Z}}
\answerTable{$28$ км}{$21$ км}{$36$ км}{$24$ км}{$35$ км}
% Відповідь: А (12 · 7/3 = 28)

\vspace{0.5cm}

%======================================================================
% СТИЛЬ И: КОМБІНОВАНІ
%======================================================================

\begin{center}
{\large\textbf{\color{headerblue}Стиль И: Комбіновані завдання}}
\end{center}

\vspace{0.3cm}

\task{42}{Знайдіть $\left| |{-}3| - |{-}7| \right|$. \nmtyear{gen-I}}
\answerTable{$4$}{$-4$}{$10$}{$-10$}{$21$}
% Відповідь: А (|3 - 7| = 4)

\vspace{0.5cm}

\task{43}{Обчисліть $\dfrac{|{-}12|}{|{-}3|} - |{-}2|$. \nmtyear{gen-I}}
\answerTable{$2$}{$6$}{$-2$}{$4$}{$0$}
% Відповідь: А (12/3 - 2 = 2)

\vspace{0.5cm}

\task{44}{На скільки $|{-}8| + |3|$ більше за $|{-}8 + 3|$? \nmtyear{gen-I}}
\answerTable{$0$}{$6$}{$16$}{$5$}{$11$}
% Відповідь: Б (11 - 5 = 6)

\vspace{0.5cm}

\task{45}{Яке ціле число $x$ задовольняє систему $\begin{cases} |x| < 4 \\ x > 2 \end{cases}$? \nmtyear{gen-I}}
\answerTable{$3$}{$-3$}{$4$}{$2$}{Не існує}
% Відповідь: А

\vspace{0.5cm}

\task{46}{Скільки цілих чисел $x$ задовольняють $|x - 1| + |x + 1| = 2$? \nmtyear{gen-I}}
\answerTable{$0$}{$1$}{$2$}{$3$}{безліч}
% Відповідь: Д (всі x від -1 до 1, включно)

\vspace{0.5cm}

\task{47}{Розташуйте в порядку зростання: $A = -\dfrac{1}{3}$, $B = -0{,}5$, $C = -\dfrac{2}{5}$. \nmtyear{gen-I}}
\answerTable{$A < B < C$}{$B < C < A$}{$C < B < A$}{$B < A < C$}{$A < C < B$}
% Відповідь: Б (-0.5 < -0.4 < -0.333...)

\vspace{0.5cm}

\task{48}{Знайдіть найменше ціле $n$, для якого дріб $\dfrac{n}{12}$ є правильним і нескоротним. \nmtyear{gen-I}}
\answerTable{$1$}{$5$}{$7$}{$11$}{$-1$}
% Відповідь: А

\vspace{0.5cm}

\end{document}
