\documentclass[14pt]{extarticle}
\usepackage{fontspec}
\usepackage{polyglossia}
\setdefaultlanguage{ukrainian}

\defaultfontfeatures{Ligatures=TeX}
\setmainfont{Liberation Serif}
\setsansfont{Liberation Sans}
\setmonofont{Liberation Mono}

\usepackage[a4paper,margin=2cm,bottom=2.5cm,top=2.5cm]{geometry}
\usepackage{amsmath,amssymb}
\usepackage{enumitem}
\usepackage{tikz}
\usepackage{xcolor}
\usepackage{array}
\usepackage{fancyhdr}

% Кольори
\definecolor{headerblue}{RGB}{0, 102, 204}
\definecolor{yearcolor}{RGB}{128, 0, 128}
\definecolor{gencolor}{RGB}{0, 128, 0}

\pagestyle{fancy}
\fancyhf{}
\renewcommand{\headrulewidth}{0pt}
\fancyfoot[C]{\thepage}

\setlength{\headheight}{15pt}
\setlength{\headsep}{10pt}
\setlength{\footskip}{25pt}

\widowpenalty=10000
\clubpenalty=10000

% === КОМАНДИ ===

% Стандартна таблиця відповідей
\newcommand{\answerTable}[5]{
\begin{center}
\begin{tabular}{|*{5}{>{\centering\arraybackslash}m{2.8cm}|}}
\hline
\rule[-0.3cm]{0pt}{0.8cm}\textbf{А} & \textbf{Б} & \textbf{В} & \textbf{Г} & \textbf{Д} \\
\hline
\rule[-0.4cm]{0pt}{1.0cm}#1 & \rule[-0.4cm]{0pt}{1.0cm}#2 & \rule[-0.4cm]{0pt}{1.0cm}#3 & \rule[-0.4cm]{0pt}{1.0cm}#4 & \rule[-0.4cm]{0pt}{1.0cm}#5 \\
\hline
\end{tabular}
\end{center}
}

% Таблиця відповідей для дробів
\newcommand{\answerTableBig}[5]{
\begin{center}
\begin{tabular}{|*{5}{>{\centering\arraybackslash}m{2.8cm}|}}
\hline
\rule[-0.3cm]{0pt}{0.8cm}\textbf{А} & \textbf{Б} & \textbf{В} & \textbf{Г} & \textbf{Д} \\
\hline
\rule[-0.6cm]{0pt}{1.4cm}#1 & \rule[-0.6cm]{0pt}{1.4cm}#2 & \rule[-0.6cm]{0pt}{1.4cm}#3 & \rule[-0.6cm]{0pt}{1.4cm}#4 & \rule[-0.6cm]{0pt}{1.4cm}#5 \\
\hline
\end{tabular}
\end{center}
}

% Таблиця для відповідності
\newcommand{\matchTable}{
\begin{tabular}{|>{\centering\arraybackslash}p{0.25cm}|*{5}{>{\centering\arraybackslash}p{0.25cm}|}}
\hline
& \scriptsize\textbf{А} & \scriptsize\textbf{Б} & \scriptsize\textbf{В} & \scriptsize\textbf{Г} & \scriptsize\textbf{Д} \\
\hline
\scriptsize\textbf{1} & \rule{0pt}{0.25cm} & & & & \\
\hline
\scriptsize\textbf{2} & \rule{0pt}{0.25cm} & & & & \\
\hline
\scriptsize\textbf{3} & \rule{0pt}{0.25cm} & & & & \\
\hline
\end{tabular}
}

\newcommand{\task}[2]{\noindent\makebox[1.5em][l]{\textbf{#1.}}\parbox[t]{\dimexpr\textwidth-1.5em}{#2}}
\newcommand{\gen}{\hfill{\small\color{gencolor}(gen)}}

\begin{document}

\begin{center}
{\Large\textbf{\color{headerblue}ЗГЕНЕРОВАНІ ЗАВДАННЯ}}
\end{center}

\begin{center}
{\large Тема: \textbf{Числа, звичайні та десяткові дроби. Модуль числа}}
\end{center}

\vspace{0.5cm}

%======================================================================
% ТИП 1: Кількість цілих чисел в інтервалі/проміжку
%======================================================================

\begin{center}
{\large\textbf{\color{headerblue}Тип: Кількість цілих чисел у проміжку}}
\end{center}

\vspace{0.3cm}

% На основі 2023#1: інтервал (-2.07; 15.9) → 18
\task{1}{Скільки всього цілих чисел містить інтервал $(-4{,}3; 11{,}2)$? \gen}
\answerTable{$17$}{$14$}{$13$}{$16$}{$15$}
% Відповідь: Г (16). Цілі: -4,-3,...,11. Кількість: 11-(-4)+1=16

\vspace{0.5cm}

\task{2}{Скільки всього цілих чисел містить інтервал $(-1{,}5; 18{,}7)$? \gen}
\answerTable{$21$}{$18$}{$20$}{$17$}{$19$}
% Відповідь: В (20). Цілі: -1,0,1,...,18. Кількість: 18-(-1)+1=20

\vspace{0.5cm}

\task{3}{Скільки всього цілих чисел містить інтервал $(-5{,}8; 7{,}1)$? \gen}
\answerTable{$14$}{$11$}{$13$}{$12$}{$10$}
% Відповідь: В (13). Цілі: -5,-4,...,7. Кількість: 7-(-5)+1=13

\vspace{0.5cm}

% На основі 2024#4: проміжок [-4; √11] → 8
\task{4}{Скільки всього цілих чисел містить проміжок $[-3; \sqrt{17}]$? \gen}
\answerTable{$6$}{$8$}{$7$}{$9$}{$5$}
% Відповідь: Б (8). √17≈4.12. Цілі: -3,-2,-1,0,1,2,3,4. Кількість: 8

\vspace{0.5cm}

\task{5}{Скільки всього цілих чисел містить проміжок $[-5; \sqrt{20}]$? \gen}
\answerTable{$8$}{$9$}{$10$}{$11$}{$7$}
% Відповідь: В (10). √20≈4.47. Цілі: -5,-4,-3,-2,-1,0,1,2,3,4. Кількість: 10

\vspace{0.5cm}

% На основі 2025#7
\task{6}{Укажіть кількість цілих чисел, що є розв'язками нерівності $-3 < x \leqslant 4{,}7$. \gen}
\answerTable{$6$}{$7$}{$8$}{$9$}{$5$}
% Відповідь: Б (7). Цілі: -2,-1,0,1,2,3,4. Кількість: 7

\vspace{0.5cm}

\task{7}{Укажіть кількість цілих чисел, що є розв'язками нерівності $-6 < x \leqslant 1{,}3$. \gen}
\answerTable{$5$}{$6$}{$7$}{$8$}{$9$}
% Відповідь: В (7). Цілі: -5,-4,-3,-2,-1,0,1. Кількість: 7

\vspace{0.5cm}

\task{8}{Скільки всього цілих чисел містить інтервал $(0{,}9; 12{,}01)$? \gen}
\answerTable{$13$}{$10$}{$12$}{$11$}{$9$}
% Відповідь: Г (11). Цілі: 1,2,...,12. Кількість: 12-1+1=12. Уточнення: 12.01>12, тому 12 входить. Відповідь: В (12)

\vspace{0.5cm}

%======================================================================
% ТИП 2: Модуль з ірраціональним числом
%======================================================================

\begin{center}
{\large\textbf{\color{headerblue}Тип: Модуль ірраціонального виразу}}
\end{center}

\vspace{0.3cm}

% На основі 2023#2: |2 - √7| = √7 - 2
\task{9}{$|3 - \sqrt{11}| =$ \gen}
\answerTable{$3 + \sqrt{11}$}{$\sqrt{8}$}{$3 - \sqrt{11}$}{$\sqrt{11} - 3$}{$\sqrt{2}$}
% Відповідь: Г. √11≈3.32>3, тому |3-√11|=√11-3

\vspace{0.5cm}

\task{10}{$|4 - \sqrt{13}| =$ \gen}
\answerTable{$4 + \sqrt{13}$}{$4 - \sqrt{13}$}{$\sqrt{3}$}{$\sqrt{13} - 4$}{$\sqrt{9}$}
% Відповідь: Б. √13≈3.61<4, тому |4-√13|=4-√13

\vspace{0.5cm}

\task{11}{$|5 - \sqrt{30}| =$ \gen}
\answerTable{$\sqrt{30} - 5$}{$5 + \sqrt{30}$}{$5 - \sqrt{30}$}{$\sqrt{5}$}{$\sqrt{25}$}
% Відповідь: А. √30≈5.48>5, тому |5-√30|=√30-5

\vspace{0.5cm}

% На основі 2023#6: |√8 - 5| = 5 - √8
\task{12}{$|\sqrt{12} - 4| =$ \gen}
\answerTable{$\sqrt{12} + 4$}{$4 - \sqrt{12}$}{$\sqrt{4}$}{$\sqrt{12} - 4$}{$-\sqrt{12} - 4$}
% Відповідь: Б. √12≈3.46<4, тому |√12-4|=4-√12

\vspace{0.5cm}

\task{13}{$|\sqrt{18} - 4| =$ \gen}
\answerTable{$4 - \sqrt{18}$}{$\sqrt{18} + 4$}{$\sqrt{2}$}{$\sqrt{18} - 4$}{$-4 - \sqrt{18}$}
% Відповідь: Г. √18≈4.24>4, тому |√18-4|=√18-4

\vspace{0.5cm}

\task{14}{$|\sqrt{5} - 3| =$ \gen}
\answerTable{$\sqrt{5} + 3$}{$3 - \sqrt{5}$}{$\sqrt{5} - 3$}{$\sqrt{4}$}{$-3 - \sqrt{5}$}
% Відповідь: Б. √5≈2.24<3, тому |√5-3|=3-√5

\vspace{0.5cm}

% На основі 2025#3: |2√2 - 3|
\task{15}{$|3\sqrt{2} - 5| =$ \gen}
\answerTable{$5 - 3\sqrt{2}$}{$3\sqrt{2} - 5$}{$3\sqrt{2} + 5$}{$-3\sqrt{2} - 5$}{$\sqrt{2}$}
% Відповідь: А. 3√2≈4.24<5, тому |3√2-5|=5-3√2

\vspace{0.5cm}

\task{16}{$|2\sqrt{3} - 4| =$ \gen}
\answerTable{$4 - 2\sqrt{3}$}{$2\sqrt{3} - 4$}{$2\sqrt{3} + 4$}{$-2\sqrt{3} - 4$}{$\sqrt{3}$}
% Відповідь: А. 2√3≈3.46<4, тому |2√3-4|=4-2√3

\vspace{0.5cm}

\task{17}{$|3\sqrt{3} - 6| =$ \gen}
\answerTable{$6 - 3\sqrt{3}$}{$3\sqrt{3} - 6$}{$3\sqrt{3} + 6$}{$-3\sqrt{3} - 6$}{$\sqrt{3}$}
% Відповідь: А. 3√3≈5.20<6, тому |3√3-6|=6-3√3

\vspace{0.5cm}

%======================================================================
% ТИП 3: Модуль арифметичного виразу |a - b·c|
%======================================================================

\begin{center}
{\large\textbf{\color{headerblue}Тип: Модуль арифметичного виразу}}
\end{center}

\vspace{0.3cm}

% На основі 2023#5: |2 - 5·3| = 13
\task{18}{$|3 - 4 \cdot 5| =$ \gen}
\answerTable{$23$}{$-17$}{$-5$}{$5$}{$17$}
% Відповідь: Д. |3-20|=17

\vspace{0.5cm}

\task{19}{$|5 - 3 \cdot 7| =$ \gen}
\answerTable{$26$}{$-16$}{$-11$}{$11$}{$16$}
% Відповідь: Д. |5-21|=16

\vspace{0.5cm}

\task{20}{$|4 - 6 \cdot 3| =$ \gen}
\answerTable{$22$}{$-14$}{$-6$}{$6$}{$14$}
% Відповідь: Д. |4-18|=14

\vspace{0.5cm}

% На основі 2024#6: |1 - 4·2.5| = 9
\task{21}{$|2 - 3 \cdot 4{,}5| =$ \gen}
\answerTable{$-11{,}5$}{$15{,}5$}{$9{,}5$}{$11{,}5$}{$-15{,}5$}
% Відповідь: Г. |2-13.5|=11.5

\vspace{0.5cm}

\task{22}{$|3 - 5 \cdot 2{,}4| =$ \gen}
\answerTable{$-9$}{$15$}{$6$}{$9$}{$-15$}
% Відповідь: Г. |3-12|=9

\vspace{0.5cm}

% На основі 2025#1: |2 - 80·0.1| = 6
\task{23}{$|3 - 70 \cdot 0{,}1| =$ \gen}
\answerTable{$1$}{$-4$}{$4$}{$10$}{$-10$}
% Відповідь: В. |3-7|=4

\vspace{0.5cm}

\task{24}{$|5 - 90 \cdot 0{,}1| =$ \gen}
\answerTable{$1$}{$-4$}{$4$}{$14$}{$-14$}
% Відповідь: В. |5-9|=4

\vspace{0.5cm}

\task{25}{$|1 - 60 \cdot 0{,}1| =$ \gen}
\answerTable{$0{,}4$}{$-5$}{$5$}{$7$}{$-7$}
% Відповідь: В. |1-6|=5

\vspace{0.5cm}

%======================================================================
% ТИП 4: Дії з дробами
%======================================================================

\begin{center}
{\large\textbf{\color{headerblue}Тип: Дії з дробами}}
\end{center}

\vspace{0.3cm}

% На основі 2023#4: |2·(-5/6)| = 5/3
\task{26}{Обчисліть $\left|3 \cdot \left(-\dfrac{4}{9}\right)\right|$. \gen}
\answerTableBig{$-\dfrac{4}{3}$}{$\dfrac{9}{4}$}{$\dfrac{1}{3}$}{$\dfrac{4}{3}$}{$-\dfrac{1}{3}$}
% Відповідь: Г. |3·(-4/9)|=|-4/3|=4/3

\vspace{0.5cm}

\task{27}{Обчисліть $\left|4 \cdot \left(-\dfrac{3}{8}\right)\right|$. \gen}
\answerTableBig{$-\dfrac{3}{2}$}{$\dfrac{8}{3}$}{$\dfrac{1}{2}$}{$\dfrac{3}{2}$}{$-\dfrac{1}{2}$}
% Відповідь: Г. |4·(-3/8)|=|-3/2|=3/2

\vspace{0.5cm}

\task{28}{Обчисліть $\left|5 \cdot \left(-\dfrac{2}{15}\right)\right|$. \gen}
\answerTableBig{$-\dfrac{2}{3}$}{$\dfrac{15}{2}$}{$\dfrac{1}{3}$}{$\dfrac{2}{3}$}{$-\dfrac{1}{3}$}
% Відповідь: Г. |5·(-2/15)|=|-2/3|=2/3

\vspace{0.5cm}

% На основі 2023#7: 3·(-1/15) = -1/5
\task{29}{$4 \cdot \left(-\dfrac{1}{12}\right) =$ \gen}
\answerTableBig{$\dfrac{3}{12}$}{$\dfrac{1}{3}$}{$-3$}{$\dfrac{47}{12}$}{$-\dfrac{1}{3}$}
% Відповідь: Д. 4·(-1/12)=-4/12=-1/3

\vspace{0.5cm}

\task{30}{$5 \cdot \left(-\dfrac{1}{20}\right) =$ \gen}
\answerTableBig{$\dfrac{4}{20}$}{$\dfrac{1}{4}$}{$-4$}{$\dfrac{99}{20}$}{$-\dfrac{1}{4}$}
% Відповідь: Д. 5·(-1/20)=-5/20=-1/4

\vspace{0.5cm}

\task{31}{$6 \cdot \left(-\dfrac{1}{18}\right) =$ \gen}
\answerTableBig{$\dfrac{5}{18}$}{$\dfrac{1}{3}$}{$-3$}{$\dfrac{107}{18}$}{$-\dfrac{1}{3}$}
% Відповідь: Д. 6·(-1/18)=-6/18=-1/3

\vspace{0.5cm}

% На основі 2023#9: 2·(-1/3) = -2/3
\task{32}{$3 \cdot \left(-\dfrac{1}{4}\right) =$ \gen}
\answerTableBig{$-\dfrac{1}{12}$}{$\dfrac{3}{4}$}{$2\dfrac{3}{4}$}{$-12$}{$-\dfrac{3}{4}$}
% Відповідь: Д. 3·(-1/4)=-3/4

\vspace{0.5cm}

\task{33}{$4 \cdot \left(-\dfrac{1}{5}\right) =$ \gen}
\answerTableBig{$-\dfrac{1}{20}$}{$\dfrac{4}{5}$}{$3\dfrac{4}{5}$}{$-20$}{$-\dfrac{4}{5}$}
% Відповідь: Д. 4·(-1/5)=-4/5

\vspace{0.5cm}

%======================================================================
% ТИП 5: Модуль десяткових/звичайних дробів
%======================================================================

\begin{center}
{\large\textbf{\color{headerblue}Тип: Модуль десяткових/звичайних дробів}}
\end{center}

\vspace{0.3cm}

% На основі 2024#5: |4.2 - 68/10| = 2.6
\task{34}{$\left|3{,}5 - \dfrac{72}{10}\right| =$ \gen}
\answerTable{$-3{,}7$}{$-3{,}5$}{$3{,}5$}{$3{,}7$}{$10{,}7$}
% Відповідь: Г. |3.5-7.2|=3.7

\vspace{0.5cm}

\task{35}{$\left|5{,}3 - \dfrac{91}{10}\right| =$ \gen}
\answerTable{$-3{,}8$}{$-3{,}6$}{$3{,}6$}{$3{,}8$}{$14{,}4$}
% Відповідь: Г. |5.3-9.1|=3.8

\vspace{0.5cm}

\task{36}{$\left|2{,}8 - \dfrac{54}{10}\right| =$ \gen}
\answerTable{$-2{,}6$}{$-2{,}4$}{$2{,}4$}{$2{,}6$}{$8{,}2$}
% Відповідь: Г. |2.8-5.4|=2.6

\vspace{0.5cm}

% На основі 2024#10: |2·1(3/8) - 3| = 1/4
\task{37}{$\left|2 \cdot 1\dfrac{1}{4} - 3\right| =$ \gen}
\answerTableBig{$\dfrac{1}{2}$}{$2\dfrac{1}{4}$}{$\dfrac{3}{4}$}{$-\dfrac{1}{2}$}{$1\dfrac{1}{2}$}
% Відповідь: А. |2·(5/4)-3|=|5/2-3|=|-1/2|=1/2

\vspace{0.5cm}

\task{38}{$\left|3 \cdot 1\dfrac{1}{6} - 4\right| =$ \gen}
\answerTableBig{$\dfrac{1}{2}$}{$2\dfrac{1}{6}$}{$\dfrac{5}{6}$}{$-\dfrac{1}{2}$}{$1\dfrac{1}{6}$}
% Відповідь: А. |3·(7/6)-4|=|7/2-4|=|-1/2|=1/2

\vspace{0.5cm}

% На основі 2025#10: |2.3 - 24/5| = 2.5
\task{39}{$\left|3{,}1 - \dfrac{32}{5}\right| =$ \gen}
\answerTable{$3{,}15$}{$2{,}5$}{$0{,}3$}{$-3{,}3$}{$3{,}3$}
% Відповідь: Д. |3.1-6.4|=3.3

\vspace{0.5cm}

\task{40}{$\left|1{,}7 - \dfrac{28}{5}\right| =$ \gen}
\answerTable{$3{,}85$}{$2{,}1$}{$0{,}4$}{$-3{,}9$}{$3{,}9$}
% Відповідь: Д. |1.7-5.6|=3.9

\vspace{0.5cm}

%======================================================================
% ТИП 6: Модуль степеневого виразу
%======================================================================

\begin{center}
{\large\textbf{\color{headerblue}Тип: Модуль степеневого виразу}}
\end{center}

\vspace{0.3cm}

% На основі 2024#2: |1 - 0.5^{-2}| = 3
\task{41}{$|1 - 0{,}5^{-3}| =$ \gen}
\answerTable{$7$}{$0{,}125$}{$0{,}875$}{$1{,}125$}{$9$}
% Відповідь: А. |1-8|=7

\vspace{0.5cm}

\task{42}{$|1 - 0{,}25^{-1}| =$ \gen}
\answerTable{$3$}{$0{,}25$}{$0{,}75$}{$1{,}25$}{$5$}
% Відповідь: А. |1-4|=3

\vspace{0.5cm}

\task{43}{$|1 - 0{,}2^{-2}| =$ \gen}
\answerTable{$24$}{$0{,}04$}{$0{,}96$}{$1{,}04$}{$26$}
% Відповідь: А. |1-25|=24

\vspace{0.5cm}

\task{44}{$|2 - 0{,}5^{-2}| =$ \gen}
\answerTable{$2$}{$0{,}25$}{$1{,}75$}{$2{,}25$}{$6$}
% Відповідь: А. |2-4|=2

\vspace{0.5cm}

% На основі 2025#8: |2.1·10^3 - 50| = 2050
\task{45}{$|3{,}2 \cdot 10^3 - 150| =$ \gen}
\answerTable{$3050$}{$2995$}{$170$}{$3350$}{$31\,850$}
% Відповідь: А. |3200-150|=3050

\vspace{0.5cm}

\task{46}{$|1{,}5 \cdot 10^3 - 80| =$ \gen}
\answerTable{$1420$}{$1495$}{$70$}{$1580$}{$14\,920$}
% Відповідь: А. |1500-80|=1420

\vspace{0.5cm}

%======================================================================
% ТИП 7: Ділення десяткових дробів
%======================================================================

\begin{center}
{\large\textbf{\color{headerblue}Тип: Ділення десяткових дробів}}
\end{center}

\vspace{0.3cm}

% На основі 2025#4: 3 : 0.3 = 10
\task{47}{$4 : 0{,}4 =$ \gen}
\answerTableBig{$1{,}6$}{$0{,}1$}{$\dfrac{1}{16}$}{$10$}{$1{,}0$}
% Відповідь: Г. 4/0.4=10

\vspace{0.5cm}

\task{48}{$5 : 0{,}5 =$ \gen}
\answerTableBig{$2{,}5$}{$0{,}1$}{$\dfrac{1}{25}$}{$10$}{$1{,}0$}
% Відповідь: Г. 5/0.5=10

\vspace{0.5cm}

\task{49}{$6 : 0{,}2 =$ \gen}
\answerTableBig{$1{,}2$}{$0{,}03$}{$\dfrac{1}{12}$}{$30$}{$3{,}0$}
% Відповідь: Г. 6/0.2=30

\vspace{0.5cm}

\task{50}{$8 : 0{,}4 =$ \gen}
\answerTableBig{$3{,}2$}{$0{,}05$}{$\dfrac{1}{20}$}{$20$}{$2{,}0$}
% Відповідь: Г. 8/0.4=20

\vspace{0.5cm}

%======================================================================
% ТИП 8: Округлення
%======================================================================

\begin{center}
{\large\textbf{\color{headerblue}Тип: Округлення чисел}}
\end{center}

\vspace{0.3cm}

% На основі 2025#13: 1.31499 → 1.31
\task{51}{Округліть до сотих число $2{,}46501$. \gen}
\answerTable{$2{,}465$}{$2{,}47$}{$2{,}464$}{$2{,}5$}{$2{,}46$}
% Відповідь: Б. 2.46501 → 2.47 (5≥5)

\vspace{0.5cm}

\task{52}{Округліть до сотих число $3{,}72489$. \gen}
\answerTable{$3{,}725$}{$3{,}73$}{$3{,}724$}{$3{,}7$}{$3{,}72$}
% Відповідь: Д. 3.72489 → 3.72 (4<5)

\vspace{0.5cm}

\task{53}{Округліть до сотих число $5{,}18512$. \gen}
\answerTable{$5{,}185$}{$5{,}19$}{$5{,}184$}{$5{,}2$}{$5{,}18$}
% Відповідь: Б. 5.18512 → 5.19 (5≥5)

\vspace{0.5cm}

\task{54}{Округліть до десятих число $7{,}3498$. \gen}
\answerTable{$7{,}35$}{$7{,}4$}{$7{,}34$}{$7$}{$7{,}3$}
% Відповідь: Д. 7.3498 → 7.3 (4<5)

\vspace{0.5cm}

%======================================================================
% ТИП 9: Модуль з параметром
%======================================================================

\begin{center}
{\large\textbf{\color{headerblue}Тип: Модуль з параметром}}
\end{center}

\vspace{0.3cm}

% На основі 2025#11: Якщо a < -2, то 1 - |a - 2| = a - 1
\task{55}{Якщо $a < -3$, то $2 - |a - 3| =$ \gen}
\answerTable{$-a - 4$}{$a - 1$}{$a + 5$}{$-a + 5$}{$-a - 1$}
% Відповідь: Б. a<-3 ⟹ a-3<0 ⟹ |a-3|=3-a ⟹ 2-(3-a)=a-1

\vspace{0.5cm}

\task{56}{Якщо $a < -1$, то $3 - |a - 1| =$ \gen}
\answerTable{$-a - 2$}{$a + 2$}{$a + 4$}{$-a + 4$}{$-a - 4$}
% Відповідь: Б. a<-1 ⟹ a-1<0 ⟹ |a-1|=1-a ⟹ 3-(1-a)=a+2

\vspace{0.5cm}

\task{57}{Якщо $a < -4$, то $1 - |a - 4| =$ \gen}
\answerTable{$-a - 5$}{$a - 3$}{$a + 5$}{$-a + 5$}{$-a - 3$}
% Відповідь: Б. a<-4 ⟹ a-4<0 ⟹ |a-4|=4-a ⟹ 1-(4-a)=a-3

\vspace{0.5cm}

%======================================================================
% ТИП 10: Значення виразу при заданих змінних
%======================================================================

\begin{center}
{\large\textbf{\color{headerblue}Тип: Значення виразу при заданих змінних}}
\end{center}

\vspace{0.3cm}

% На основі 2024#1: (1/3)m + (1/5)n при m=-18, n=55 → 5
\task{58}{Знайдіть значення виразу $\dfrac{1}{4}m + \dfrac{1}{3}n$, якщо $m = -12$, $n = 27$. \gen}
\answerTable{$-6$}{$-12$}{$3$}{$6$}{$12$}
% Відповідь: Г. -12/4+27/3=-3+9=6

\vspace{0.5cm}

\task{59}{Знайдіть значення виразу $\dfrac{1}{2}m + \dfrac{1}{7}n$, якщо $m = -8$, $n = 49$. \gen}
\answerTable{$-3$}{$-11$}{$1$}{$3$}{$11$}
% Відповідь: Г. -8/2+49/7=-4+7=3

\vspace{0.5cm}

\task{60}{Знайдіть значення виразу $\dfrac{1}{5}m + \dfrac{1}{4}n$, якщо $m = -20$, $n = 36$. \gen}
\answerTable{$-5$}{$-13$}{$4$}{$5$}{$13$}
% Відповідь: Г. -20/5+36/4=-4+9=5

\vspace{0.5cm}

%======================================================================
% ТИП 11: Формула різниці квадратів
%======================================================================

\begin{center}
{\large\textbf{\color{headerblue}Тип: Формула різниці квадратів}}
\end{center}

\vspace{0.3cm}

% На основі 2025#6: |(2.5² - 7.5²)/(3.5 + 6.5)| = 5
\task{61}{$\left|\dfrac{3{,}5^2 - 8{,}5^2}{4{,}5 + 7{,}5}\right| =$ \gen}
\answerTable{$-5$}{$-1$}{$0{,}5$}{$1$}{$5$}
% Відповідь: Д. |(12.25-72.25)/12|=|-60/12|=5

\vspace{0.5cm}

\task{62}{$\left|\dfrac{1{,}5^2 - 6{,}5^2}{2{,}5 + 5{,}5}\right| =$ \gen}
\answerTable{$-5$}{$-1$}{$0{,}5$}{$1$}{$5$}
% Відповідь: Д. |(2.25-42.25)/8|=|-40/8|=5

\vspace{0.5cm}

\task{63}{$\left|\dfrac{2^2 - 8^2}{3 + 7}\right| =$ \gen}
\answerTable{$-6$}{$-1$}{$0{,}6$}{$1$}{$6$}
% Відповідь: Д. |(4-64)/10|=|-60/10|=6

\vspace{0.5cm}

%======================================================================
% ТИП 12: Відповідність вираз-проміжок
%======================================================================

\begin{center}
{\large\textbf{\color{headerblue}Тип: Відповідність вираз--проміжок}}
\end{center}

\vspace{0.3cm}

\task{64}{Установіть відповідність між виразом (1--3) і проміжком (А--Д), якому належить значення цього виразу. \gen}

\vspace{0.3cm}
\noindent
\begin{minipage}[t]{0.25\textwidth}
\textit{Вираз}

\vspace{0.2cm}
\textbf{1} \quad $\sin \dfrac{\pi}{2}$

\vspace{0.3cm}
\textbf{2} \quad $|\pi - 4|$

\vspace{0.2cm}
\textbf{3} \quad $3^\pi$
\end{minipage}
\hfill
\begin{minipage}[t]{0.25\textwidth}
\textit{Проміжок}

\vspace{0.2cm}
\textbf{А} \quad $(-\infty; 0]$

\vspace{0.2cm}
\textbf{Б} \quad $(0; 1]$

\vspace{0.2cm}
\textbf{В} \quad $(1; 4)$

\vspace{0.2cm}
\textbf{Г} \quad $[4; 30)$

\vspace{0.2cm}
\textbf{Д} \quad $[30; +\infty)$
\end{minipage}
\hfill
\begin{minipage}[t]{0.2\textwidth}
\vspace{0pt}
\matchTable
\end{minipage}
% Відповіді: 1-Б (sin(π/2)=1), 2-Б (|π-4|≈0.86), 3-Д (3^π≈31.5)

\vspace{0.7cm}

\task{65}{Установіть відповідність між виразом (1--3) і проміжком (А--Д), якому належить значення цього виразу. \gen}

\vspace{0.3cm}
\noindent
\begin{minipage}[t]{0.25\textwidth}
\textit{Вираз}

\vspace{0.2cm}
\textbf{1} \quad $\cos \pi$

\vspace{0.3cm}
\textbf{2} \quad $|2\pi - 7|$

\vspace{0.2cm}
\textbf{3} \quad $2^{2\pi}$
\end{minipage}
\hfill
\begin{minipage}[t]{0.25\textwidth}
\textit{Проміжок}

\vspace{0.2cm}
\textbf{А} \quad $(-\infty; -1]$

\vspace{0.2cm}
\textbf{Б} \quad $(-1; 0)$

\vspace{0.2cm}
\textbf{В} \quad $[0; 1)$

\vspace{0.2cm}
\textbf{Г} \quad $[1; 50)$

\vspace{0.2cm}
\textbf{Д} \quad $[50; +\infty)$
\end{minipage}
\hfill
\begin{minipage}[t]{0.2\textwidth}
\vspace{0pt}
\matchTable
\end{minipage}
% Відповіді: 1-А (cos(π)=-1), 2-В (|2π-7|≈0.72), 3-Д (2^{2π}≈84.4)

\vspace{0.7cm}

\task{66}{Установіть відповідність між виразом (1--3) і проміжком (А--Д), якому належить значення цього виразу. \gen}

\vspace{0.3cm}
\noindent
\begin{minipage}[t]{0.28\textwidth}
\textit{Вираз}

\vspace{0.2cm}
\textbf{1} \quad $\mathrm{tg}\,\dfrac{\pi}{4}$

\vspace{0.3cm}
\textbf{2} \quad $\pi - 3$

\vspace{0.2cm}
\textbf{3} \quad $\log_2 \pi + \log_2 (2\pi)$
\end{minipage}
\hfill
\begin{minipage}[t]{0.25\textwidth}
\textit{Проміжок}

\vspace{0.2cm}
\textbf{А} \quad $[-4; 0)$

\vspace{0.2cm}
\textbf{Б} \quad $[0; 1)$

\vspace{0.2cm}
\textbf{В} \quad $[1; 2)$

\vspace{0.2cm}
\textbf{Г} \quad $[2; 4)$

\vspace{0.2cm}
\textbf{Д} \quad $[4; 8)$
\end{minipage}
\hfill
\begin{minipage}[t]{0.2\textwidth}
\vspace{0pt}
\matchTable
\end{minipage}
% Відповіді: 1-В (tg(π/4)=1), 2-Б (π-3≈0.14), 3-Д (log₂(2π²)≈4.3)

\vspace{0.7cm}

%======================================================================
% ТИП 13: Відповідність з числовою прямою
%======================================================================

\begin{center}
{\large\textbf{\color{headerblue}Тип: Відповідність з числовою прямою}}
\end{center}

\vspace{0.3cm}

\task{67}{Узгодьте вираз (1--3) з точкою (А--Д) на координатній прямій, координатою якої є значення виразу, якщо $a = -3$. \gen}

\vspace{0.3cm}
\begin{center}
\begin{tikzpicture}[scale=1.3]
    \draw[->] (-3,0) -- (3,0);
    \foreach \x/\name in {-2/K, -1/L, 0/M, 1/N, 2/P} {
        \draw (\x,0.1) -- (\x,-0.1);
        \node[above] at (\x,0.15) {$\name$};
        \node[below] at (\x,-0.15) {$\x$};
    }
\end{tikzpicture}
\end{center}

\vspace{0.3cm}
\noindent
\begin{minipage}[t]{0.22\textwidth}
\textit{Вираз}

\vspace{0.2cm}
\textbf{1} \quad $|a| - 1$

\vspace{0.2cm}
\textbf{2} \quad $a^0$

\vspace{0.2cm}
\textbf{3} \quad $\sin(\pi a)$
\end{minipage}
\hfill
\begin{minipage}[t]{0.25\textwidth}
\textit{Точка}

\vspace{0.2cm}
\textbf{А} \quad $K$

\vspace{0.2cm}
\textbf{Б} \quad $L$

\vspace{0.2cm}
\textbf{В} \quad $M$

\vspace{0.2cm}
\textbf{Г} \quad $N$

\vspace{0.2cm}
\textbf{Д} \quad $P$
\end{minipage}
\hfill
\begin{minipage}[t]{0.2\textwidth}
\vspace{0pt}
\matchTable
\end{minipage}
% Відповіді: 1-Д (|-3|-1=2), 2-Г ((-3)⁰=1), 3-В (sin(-3π)=0)

\vspace{0.7cm}

\task{68}{Узгодьте вираз (1--3) з точкою (А--Д) на координатній прямій, координатою якої є значення виразу, якщо $b = 4$. \gen}

\vspace{0.3cm}
\begin{center}
\begin{tikzpicture}[scale=1.3]
    \draw[->] (-3,0) -- (3,0);
    \foreach \x/\name in {-2/K, -1/L, 0/M, 1/N, 2/P} {
        \draw (\x,0.1) -- (\x,-0.1);
        \node[above] at (\x,0.15) {$\name$};
        \node[below] at (\x,-0.15) {$\x$};
    }
\end{tikzpicture}
\end{center}

\vspace{0.3cm}
\noindent
\begin{minipage}[t]{0.22\textwidth}
\textit{Вираз}

\vspace{0.2cm}
\textbf{1} \quad $\log_2 b$

\vspace{0.2cm}
\textbf{2} \quad $\cos(\pi b)$

\vspace{0.2cm}
\textbf{3} \quad $\sqrt{b} - 4$
\end{minipage}
\hfill
\begin{minipage}[t]{0.25\textwidth}
\textit{Точка}

\vspace{0.2cm}
\textbf{А} \quad $K$

\vspace{0.2cm}
\textbf{Б} \quad $L$

\vspace{0.2cm}
\textbf{В} \quad $M$

\vspace{0.2cm}
\textbf{Г} \quad $N$

\vspace{0.2cm}
\textbf{Д} \quad $P$
\end{minipage}
\hfill
\begin{minipage}[t]{0.2\textwidth}
\vspace{0pt}
\matchTable
\end{minipage}
% Відповіді: 1-Д (log₂4=2), 2-Г (cos(4π)=1), 3-А (√4-4=-2)

\vspace{0.7cm}

%======================================================================
% ТИП 14: Відповідність вираз-твердження
%======================================================================

\begin{center}
{\large\textbf{\color{headerblue}Тип: Відповідність вираз--твердження}}
\end{center}

\vspace{0.3cm}

\task{69}{Установіть відповідність між виразом (1--3) та твердженням про його значення (А--Д), яке є правильним. \gen}

\vspace{0.3cm}
\noindent
\begin{minipage}[t]{0.25\textwidth}
\textit{Вираз}

\vspace{0.2cm}
\textbf{1} \quad $(\sqrt{3} + 4)(\sqrt{3} - 4)$

\vspace{0.2cm}
\textbf{2} \quad $3\log_3 \sqrt{27}$

\vspace{0.2cm}
\textbf{3} \quad $|1 - \sqrt{3}|$
\end{minipage}
\hfill
\begin{minipage}[t]{0.4\textwidth}
\textit{Твердження про значення виразу}

\vspace{0.2cm}
\textbf{А} \quad є цілим додатним числом

\vspace{0.2cm}
\textbf{Б} \quad є цілим від'ємним числом

\vspace{0.2cm}
\textbf{В} \quad дорівнює $0$

\vspace{0.2cm}
\textbf{Г} \quad є нецілим додатним числом

\vspace{0.2cm}
\textbf{Д} \quad є нецілим від'ємним числом
\end{minipage}
\hfill
\begin{minipage}[t]{0.18\textwidth}
\vspace{0pt}
\matchTable
\end{minipage}
% Відповіді: 1-Б (3-16=-13), 2-Г (3·(3/2)=4.5), 3-Г (√3-1≈0.73)

\vspace{0.7cm}

\task{70}{Установіть відповідність між виразом (1--3) та твердженням про його значення (А--Д), яке є правильним. \gen}

\vspace{0.3cm}
\noindent
\begin{minipage}[t]{0.25\textwidth}
\textit{Вираз}

\vspace{0.2cm}
\textbf{1} \quad $\cos \dfrac{3\pi}{2}$

\vspace{0.3cm}
\textbf{2} \quad $2^{\sin 0°}$

\vspace{0.3cm}
\textbf{3} \quad $\dfrac{\pi}{4}$
\end{minipage}
\hfill
\begin{minipage}[t]{0.42\textwidth}
\textit{Твердження про значення виразу}

\vspace{0.2cm}
\textbf{А} \quad є раціональним нецілим числом

\vspace{0.2cm}
\textbf{Б} \quad є ірраціональним числом

\vspace{0.2cm}
\textbf{В} \quad дорівнює $0$

\vspace{0.2cm}
\textbf{Г} \quad є натуральним числом

\vspace{0.2cm}
\textbf{Д} \quad є цілим від'ємним числом
\end{minipage}
\hfill
\begin{minipage}[t]{0.18\textwidth}
\vspace{0pt}
\matchTable
\end{minipage}
% Відповіді: 1-В (cos(3π/2)=0), 2-Г (2⁰=1), 3-Б (π/4 ірраціональне)

\vspace{0.7cm}

%======================================================================
% ТИП 15: Тотожно рівні вирази
%======================================================================

\begin{center}
{\large\textbf{\color{headerblue}Тип: Тотожно рівні вирази}}
\end{center}

\vspace{0.3cm}

\task{71}{До кожного виразу (1--3) доберіть тотожно рівний йому вираз (А--Д). \gen}

\vspace{0.3cm}
\noindent
\begin{minipage}[t]{0.3\textwidth}
\textit{Вираз}

\vspace{0.2cm}
\textbf{1} \quad $|1 - \sqrt{3}| - \sqrt{3} + 1$

\vspace{0.3cm}
\textbf{2} \quad $\dfrac{3\sqrt{3} - 9}{\sqrt{3}}$

\vspace{0.3cm}
\textbf{3} \quad $\log_{\sqrt{3}} 9$
\end{minipage}
\hfill
\begin{minipage}[t]{0.28\textwidth}
\textit{Тотожно рівний вираз}

\vspace{0.2cm}
\textbf{А} \quad $\sqrt{3}$

\vspace{0.2cm}
\textbf{Б} \quad $0$

\vspace{0.2cm}
\textbf{В} \quad $3 - 3\sqrt{3}$

\vspace{0.2cm}
\textbf{Г} \quad $4$

\vspace{0.2cm}
\textbf{Д} \quad $-6$
\end{minipage}
\hfill
\begin{minipage}[t]{0.2\textwidth}
\vspace{0pt}
\matchTable
\end{minipage}
% Відповіді: 1-Б (√3-1-√3+1=0), 2-В (3-9/√3=3-3√3), 3-Г (log_{√3}9=4)

\vspace{0.7cm}

%======================================================================
% ТИП 16: Відповідність при заданому значенні
%======================================================================

\begin{center}
{\large\textbf{\color{headerblue}Тип: Відповідність при заданому значенні}}
\end{center}

\vspace{0.3cm}

\task{72}{Установіть відповідність між виразом (1--3) та проміжком (А--Д), якому належить значення цього виразу, якщо $a = -0{,}25$. \gen}

\vspace{0.3cm}
\noindent
\begin{minipage}[t]{0.22\textwidth}
\textit{Вираз}

\vspace{0.2cm}
\textbf{1} \quad $|a|$

\vspace{0.2cm}
\textbf{2} \quad $a^2$

\vspace{0.2cm}
\textbf{3} \quad $\dfrac{1}{a}$
\end{minipage}
\hfill
\begin{minipage}[t]{0.28\textwidth}
\textit{Проміжок}

\vspace{0.2cm}
\textbf{А} \quad $(-\infty; -4)$

\vspace{0.2cm}
\textbf{Б} \quad $[-4; -1)$

\vspace{0.2cm}
\textbf{В} \quad $[-1; 0)$

\vspace{0.2cm}
\textbf{Г} \quad $[0; 1)$

\vspace{0.2cm}
\textbf{Д} \quad $[1; +\infty)$
\end{minipage}
\hfill
\begin{minipage}[t]{0.2\textwidth}
\vspace{0pt}
\matchTable
\end{minipage}
% Відповіді: 1-Г (|–0.25|=0.25), 2-Г ((–0.25)²=0.0625), 3-Б (1/(–0.25)=–4)

\vspace{0.7cm}

\task{73}{Установіть відповідність між виразом (1--3) та проміжком (А--Д), якому належить значення цього виразу, якщо $b = -2$. \gen}

\vspace{0.3cm}
\noindent
\begin{minipage}[t]{0.22\textwidth}
\textit{Вираз}

\vspace{0.2cm}
\textbf{1} \quad $|b| + 1$

\vspace{0.2cm}
\textbf{2} \quad $b^3$

\vspace{0.2cm}
\textbf{3} \quad $\dfrac{2}{b}$
\end{minipage}
\hfill
\begin{minipage}[t]{0.28\textwidth}
\textit{Проміжок}

\vspace{0.2cm}
\textbf{А} \quad $(-\infty; -8]$

\vspace{0.2cm}
\textbf{Б} \quad $(-8; -1)$

\vspace{0.2cm}
\textbf{В} \quad $[-1; 0)$

\vspace{0.2cm}
\textbf{Г} \quad $[0; 3)$

\vspace{0.2cm}
\textbf{Д} \quad $[3; +\infty)$
\end{minipage}
\hfill
\begin{minipage}[t]{0.2\textwidth}
\vspace{0pt}
\matchTable
\end{minipage}
% Відповіді: 1-Д (|-2|+1=3), 2-А ((-2)³=-8), 3-В (2/(-2)=-1)

\vspace{0.5cm}

\end{document}
