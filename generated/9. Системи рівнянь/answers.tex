\documentclass[12pt]{extarticle}
\usepackage{fontspec}
\usepackage{polyglossia}
\setdefaultlanguage{ukrainian}

\defaultfontfeatures{Ligatures=TeX}
\setmainfont{Liberation Serif}

\usepackage[a4paper,margin=2cm]{geometry}
\usepackage{amsmath,amssymb}
\usepackage{multicol}
\usepackage{xcolor}

\definecolor{headerblue}{RGB}{0, 102, 204}

\begin{document}

\begin{center}
{\Large\textbf{\color{headerblue}ВІДПОВІДІ}}
\end{center}

\begin{center}
{\large Тема 9: Системи рівнянь}
\end{center}

\vspace{0.5cm}

\textbf{Блок 1: Прості системи лінійних рівнянь}

\begin{multicols}{5}
\noindent
1. А ($7$) \\
2. А ($21$) \\
3. А ($5$) \\
4. А ($4{,}5$) \\
5. А ($5$) \\
6. Г \\
7. А \\
8. А \\
9. А \\
10. А
\end{multicols}

\textbf{Блок 2: Системи з дробами}

\begin{multicols}{5}
\noindent
11. А ($5$) \\
12. А ($6$) \\
13. А \\
14. А ($5$) \\
15. А ($-4$) \\
16. А ($12$) \\
17. А ($24$) \\
18. А ($7$) \\
19. А ($2$) \\
20. А ($6$)
\end{multicols}

\textbf{Блок 3: Системи з добутком змінних}

\begin{multicols}{5}
\noindent
21. А ($1$) \\
22. А ($9$) \\
23. А ($5$) \\
24. А ($6$) \\
25. А ($15$) \\
26. А ($4$) \\
27. А ($4$) \\
28. А ($0$) \\
29. А ($2$) \\
30. А ($7$)
\end{multicols}

\textbf{Блок 4: Системи з $\frac{1}{x}$, $\frac{1}{y}$}

\begin{multicols}{5}
\noindent
31. А ($4$) \\
32. А ($5$) \\
33. А \\
34. А \\
35. А ($-14$) \\
36. А ($3$) \\
37. А ($3$) \\
38. А ($1$) \\
39. А ($3$) \\
40. А ($1$)
\end{multicols}

\textbf{Блок 5: Системи методом підстановки}

\begin{multicols}{5}
\noindent
41. А ($2$) \\
42. А ($11$) \\
43. А ($4$) \\
44. А ($1$) \\
45. А ($3$) \\
46. А ($4$) \\
47. А ($0$) \\
48. А ($5$) \\
49. А \\
50. А
\end{multicols}

\textbf{Блок 6: Системи методом додавання}

\begin{multicols}{5}
\noindent
51. А ($3$) \\
52. А ($2$) \\
53. А ($3$) \\
54. А ($6$) \\
55. А ($3$) \\
56. А ($3$) \\
57. А ($3$) \\
58. А ($5$) \\
59. А ($12$) \\
60. А ($4$)
\end{multicols}

\textbf{Блок 7: Системи зі спеціальними функціями}

\begin{multicols}{5}
\noindent
61. А ($2$) \\
62. А ($1$) \\
63. А ($5$) \\
64. А \\
65. А ($5$) \\
66. А ($2$) \\
67. А ($4$) \\
68. А ($4$) \\
69. А ($25$) \\
70. А ($2$)
\end{multicols}

\textbf{Блок 8: Системи з коренем}

\begin{multicols}{5}
\noindent
71. А ($4$) \\
72. А ($14$) \\
73. А ($4$) \\
74. А ($6$) \\
75. А ($8$) \\
76. А ($16$) \\
77. А ($9$) \\
78. А ($13$) \\
79. А ($50$) \\
80. А ($4$)
\end{multicols}

\textbf{Блок 9: Системи з нестандартними умовами}

\begin{multicols}{5}
\noindent
81. А ($6{,}5$) \\
82. А ($0$) \\
83. А ($8$) \\
84. А ($6$) \\
85. А ($6$) \\
86. А ($-3$) \\
87. А ($5$) \\
88. А ($2$) \\
89. А ($3$) \\
90. А ($8$)
\end{multicols}

\vspace{1cm}

\textbf{Методи розв'язування систем:}

\begin{enumerate}
\item \textbf{Метод підстановки:} Виразити одну змінну через іншу і підставити

\item \textbf{Метод додавання:} Скласти або відняти рівняння, щоб виключити одну змінну

\item \textbf{Для систем з $\frac{1}{x}$, $\frac{1}{y}$:} Заміна $u = \frac{1}{x}$, $v = \frac{1}{y}$

\item \textbf{Для показникових:} $a^{f(x,y)} = a^c \Rightarrow f(x,y) = c$

\item \textbf{Для систем з коренем:} Заміна $t = \sqrt{x}$ або $t = \sqrt{y}$
\end{enumerate}

\end{document}
