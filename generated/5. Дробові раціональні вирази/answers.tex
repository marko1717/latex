\documentclass[12pt]{extarticle}
\usepackage{fontspec}
\usepackage{polyglossia}
\setdefaultlanguage{ukrainian}

\defaultfontfeatures{Ligatures=TeX}
\setmainfont{Liberation Serif}

\usepackage[a4paper,margin=2cm]{geometry}
\usepackage{amsmath,amssymb}
\usepackage{multicol}
\usepackage{xcolor}

\definecolor{headerblue}{RGB}{0, 102, 204}

\begin{document}

\begin{center}
{\Large\textbf{\color{headerblue}ВІДПОВІДІ}}
\end{center}

\begin{center}
{\large Тема 5: Дробові раціональні вирази та їх перетворення}
\end{center}

\vspace{0.5cm}

\textbf{Блок 1: Скорочення дробів (ФСМ)}

\begin{multicols}{5}
\noindent
1. Б \\
2. Б \\
3. Б \\
4. Б \\
5. Б \\
6. Б \\
7. Б \\
8. Б \\
9. Б \\
10. Б \\
11. Б \\
12. Б \\
13. Б \\
14. Б \\
15. Б
\end{multicols}

\textbf{Блок 2: Повний квадрат у знаменнику}

\begin{multicols}{5}
\noindent
16. А \\
17. А \\
18. А \\
19. В \\
20. А \\
21. А \\
22. А \\
23. А \\
24. А \\
25. А \\
26. А \\
27. А \\
28. А \\
29. А \\
30. А
\end{multicols}

\textbf{Блок 3: Скорочення з числовими коефіцієнтами}

\begin{multicols}{5}
\noindent
31. А \\
32. А \\
33. А \\
34. Б \\
35. Б \\
36. Б \\
37. А \\
38. А \\
39. Б \\
40. А \\
41. А \\
42. Б \\
43. А \\
44. Б \\
45. А
\end{multicols}

\textbf{Блок 4: Складніші спрощення}

\begin{multicols}{5}
\noindent
46. А \\
47. Б \\
48. А \\
49. А \\
50. А \\
51. А \\
52. А \\
53. А \\
54. А \\
55. А \\
56. А \\
57. А \\
58. А \\
59. А \\
60. А
\end{multicols}

\textbf{Блок 5: Ділення дробів}

\begin{multicols}{5}
\noindent
61. А \\
62. А \\
63. А \\
64. А \\
65. А \\
66. А \\
67. А \\
68. А \\
69. А \\
70. А \\
71. А \\
72. А \\
73. А \\
74. А \\
75. А
\end{multicols}

\textbf{Блок 6: Від'ємні степені}

\begin{multicols}{5}
\noindent
76. А \\
77. А \\
78. А \\
79. А \\
80. А \\
81. А \\
82. А \\
83. А \\
84. А \\
85. А \\
86. А \\
87. А \\
88. А \\
89. А \\
90. А
\end{multicols}

\textbf{Блок 7: Виразити змінну з формули}

\begin{multicols}{5}
\noindent
91. А \\
92. А \\
93. А \\
94. А \\
95. А \\
96. А \\
97. А \\
98. А \\
99. А \\
100. А \\
101. А \\
102. А \\
103. А \\
104. А \\
105. А
\end{multicols}

\textbf{Блок 8: Обчислення числових значень}

\begin{multicols}{5}
\noindent
106. А \\
107. А \\
108. А \\
109. А \\
110. А \\
111. А \\
112. А \\
113. А \\
114. А \\
115. А \\
116. А \\
117. А \\
118. А \\
119. А \\
120. А
\end{multicols}

\vspace{1cm}

\textbf{Пояснення типових розв'язків:}

\begin{enumerate}
\item \textbf{Різниця квадратів:} $\dfrac{a^2 - b^2}{a + b} = \dfrac{(a-b)(a+b)}{a+b} = a - b$

\item \textbf{Повний квадрат:} $\dfrac{a^2 - b^2}{(a-b)^2} = \dfrac{(a-b)(a+b)}{(a-b)^2} = \dfrac{a+b}{a-b}$

\item \textbf{Від'ємний степінь:} $(a^2 - b^2)(a-b)^{-1} = \dfrac{(a-b)(a+b)}{a-b} = a + b$

\item \textbf{Квадрат різниці у виразі:} $\dfrac{(ax + b)^2 - b^2}{x} = \dfrac{(ax+b-b)(ax+b+b)}{x} = \dfrac{ax(ax+2b)}{x} = ax + 2b \cdot a$

\item \textbf{З коренями:} $\dfrac{x - a^2}{c\sqrt{x} - ca} = \dfrac{(\sqrt{x}-a)(\sqrt{x}+a)}{c(\sqrt{x}-a)} = \dfrac{\sqrt{x}+a}{c}$
\end{enumerate}

\end{document}
