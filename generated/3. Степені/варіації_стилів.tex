\documentclass[14pt]{extarticle}
\usepackage{fontspec}
\usepackage{polyglossia}
\setdefaultlanguage{ukrainian}

\defaultfontfeatures{Ligatures=TeX}
\setmainfont{Liberation Serif}
\setsansfont{Liberation Sans}
\setmonofont{Liberation Mono}

\usepackage[a4paper,margin=2cm,bottom=2.5cm,top=2.5cm]{geometry}
\usepackage{amsmath,amssymb}
\usepackage{xcolor}
\usepackage{array}
\usepackage{fancyhdr}

\definecolor{headerblue}{RGB}{0, 102, 204}
\definecolor{yearcolor}{RGB}{128, 0, 128}

\pagestyle{fancy}
\fancyhf{}
\renewcommand{\headrulewidth}{0pt}
\fancyfoot[C]{\thepage}

\newcommand{\answerTable}[5]{
\begin{center}
\begin{tabular}{|*{5}{>{\centering\arraybackslash}m{2.8cm}|}}
\hline
\rule[-0.3cm]{0pt}{0.8cm}\textbf{А} & \textbf{Б} & \textbf{В} & \textbf{Г} & \textbf{Д} \\
\hline
\rule[-0.4cm]{0pt}{1.0cm}#1 & \rule[-0.4cm]{0pt}{1.0cm}#2 & \rule[-0.4cm]{0pt}{1.0cm}#3 & \rule[-0.4cm]{0pt}{1.0cm}#4 & \rule[-0.4cm]{0pt}{1.0cm}#5 \\
\hline
\end{tabular}
\end{center}
}

\newcommand{\answerTableBig}[5]{
\begin{center}
\begin{tabular}{|*{5}{>{\centering\arraybackslash}m{2.8cm}|}}
\hline
\rule[-0.3cm]{0pt}{0.8cm}\textbf{А} & \textbf{Б} & \textbf{В} & \textbf{Г} & \textbf{Д} \\
\hline
\rule[-0.6cm]{0pt}{1.4cm}#1 & \rule[-0.6cm]{0pt}{1.4cm}#2 & \rule[-0.6cm]{0pt}{1.4cm}#3 & \rule[-0.6cm]{0pt}{1.4cm}#4 & \rule[-0.6cm]{0pt}{1.4cm}#5 \\
\hline
\end{tabular}
\end{center}
}

\newcommand{\task}[2]{\noindent\makebox[1.5em][l]{\textbf{#1.}}\parbox[t]{\dimexpr\textwidth-1.5em}{#2}}
\newcommand{\nmtyear}[1]{\hfill{\small\color{yearcolor}(#1)}}

\begin{document}

\begin{center}
{\Large\textbf{\color{headerblue}ВАРІАЦІЇ СТИЛІВ ЗАВДАНЬ}}
\end{center}

\begin{center}
{\large Тема: \textbf{Степені. Одночлени. Стандартний вигляд числа}}
\end{center}

\vspace{0.5cm}

%======================================================================
% СТИЛЬ А: ЗВОРОТНІ ЗАДАЧІ
%======================================================================

\begin{center}
{\large\textbf{\color{headerblue}Стиль А: Зворотні задачі}}
\end{center}

\vspace{0.3cm}

% Оригінал: "Обчисліть $2^5$" → Зворотне: "Який показник степеня?"
\task{1}{Яке число потрібно піднести до квадрата, щоб отримати $81$? \nmtyear{gen-A}}
\answerTable{$3$}{$9$}{$27$}{$41$}{$\pm 9$}
% Відповідь: Б (або Д якщо враховувати від'ємне)

\vspace{0.5cm}

\task{2}{До якого степеня потрібно піднести $3$, щоб отримати $243$? \nmtyear{gen-A}}
\answerTable{$3$}{$4$}{$5$}{$6$}{$81$}
% Відповідь: В (3^5 = 243)

\vspace{0.5cm}

\task{3}{Яка основа степеня, якщо $a^4 = 16$, де $a > 0$? \nmtyear{gen-A}}
\answerTable{$2$}{$4$}{$8$}{$12$}{$\sqrt{2}$}
% Відповідь: А

\vspace{0.5cm}

\task{4}{При якому $n$ виконується $5^n = 125$? \nmtyear{gen-A}}
\answerTable{$2$}{$3$}{$4$}{$5$}{$25$}
% Відповідь: Б

\vspace{0.5cm}

\task{5}{Число $10^6$ --- це результат піднесення числа $100$ до степеня: \nmtyear{gen-A}}
\answerTable{$2$}{$3$}{$4$}{$6$}{$600$}
% Відповідь: Б (100^3 = 10^6)

\vspace{0.5cm}

%======================================================================
% СТИЛЬ Б: ПРАКТИЧНИЙ КОНТЕКСТ
%======================================================================

\begin{center}
{\large\textbf{\color{headerblue}Стиль Б: Практичний контекст}}
\end{center}

\vspace{0.3cm}

\task{6}{Бактерія ділиться кожну годину. Скільки бактерій буде через 8 годин, якщо спочатку була одна? \nmtyear{gen-B}}
\answerTable{$8$}{$64$}{$128$}{$256$}{$512$}
% Відповідь: Г (2^8 = 256)

\vspace{0.5cm}

\task{7}{Папір складають навпіл 6 разів. У скільки разів збільшилась товщина? \nmtyear{gen-B}}
\answerTable{$6$}{$12$}{$32$}{$64$}{$128$}
% Відповідь: Г (2^6 = 64)

\vspace{0.5cm}

\task{8}{Радіоактивна речовина розпадається вдвічі кожні 10 років. Яка частина залишиться через 40 років? \nmtyear{gen-B}}
\answerTableBig{$\dfrac{1}{4}$}{$\dfrac{1}{8}$}{$\dfrac{1}{16}$}{$\dfrac{1}{32}$}{$\dfrac{1}{2}$}
% Відповідь: В (1/2^4 = 1/16)

\vspace{0.5cm}

\task{9}{Комп'ютер виконує $10^9$ операцій за секунду. Скільки операцій виконає за 1 мілісекунду? \nmtyear{gen-B}}
\answerTable{$10^3$}{$10^6$}{$10^{12}$}{$10^{-3}$}{$10^9$}
% Відповідь: Б (10^9 / 10^3 = 10^6)

\vspace{0.5cm}

\task{10}{Відстань до зірки --- $4 \cdot 10^{13}$ км. Світло долає $3 \cdot 10^5$ км/с. За скільки секунд світло долає цю відстань? \nmtyear{gen-B}}
\answerTableBig{$\dfrac{4}{3} \cdot 10^{8}$}{$1{,}2 \cdot 10^{19}$}{$12 \cdot 10^{8}$}{$\dfrac{4}{3} \cdot 10^{18}$}{$1{,}33 \cdot 10^{8}$}
% Відповідь: А (4·10^13 / 3·10^5 = 4/3 · 10^8)

\vspace{0.5cm}

\task{11}{Вірус має розмір $10^{-7}$ м. У скільки разів він менший за клітину розміром $10^{-5}$ м? \nmtyear{gen-B}}
\answerTable{$2$}{$10$}{$100$}{$1000$}{$10^{-2}$}
% Відповідь: В (10^-5 / 10^-7 = 10^2 = 100)

\vspace{0.5cm}

%======================================================================
% СТИЛЬ В: ЗНАЙТИ ПОМИЛКУ
%======================================================================

\begin{center}
{\large\textbf{\color{headerblue}Стиль В: Знайти помилку}}
\end{center}

\vspace{0.3cm}

\task{12}{Учень записав: $2^3 \cdot 2^4 = 2^{12}$. Яка правильна відповідь? \nmtyear{gen-V}}
\answerTable{$2^{12}$}{$2^7$}{$4^7$}{$2^{1}$}{$4^{12}$}
% Відповідь: Б

\vspace{0.5cm}

\task{13}{Учень спростив: $(3^2)^4 = 3^6$. Яка правильна відповідь? \nmtyear{gen-V}}
\answerTable{$3^6$}{$3^8$}{$9^4$}{$3^{16}$}{$9^6$}
% Відповідь: Б

\vspace{0.5cm}

\task{14}{Учень записав: $5^0 = 0$. Яка правильна відповідь? \nmtyear{gen-V}}
\answerTable{$0$}{$1$}{$5$}{$-1$}{Не існує}
% Відповідь: Б

\vspace{0.5cm}

\task{15}{Учень обчислив: $10^{-2} = -100$. Яка правильна відповідь? \nmtyear{gen-V}}
\answerTable{$-100$}{$-0{,}01$}{$0{,}01$}{$100$}{$-20$}
% Відповідь: В

\vspace{0.5cm}

\task{16}{Учень записав: $\dfrac{a^5}{a^2} = a^{2{,}5}$. Яка правильна відповідь? \nmtyear{gen-V}}
\answerTable{$a^{2{,}5}$}{$a^3$}{$a^{10}$}{$a^7$}{$a^{-3}$}
% Відповідь: Б

\vspace{0.5cm}

\task{17}{Учень стверджує: $(-2)^4 = -16$. Яка правильна відповідь? \nmtyear{gen-V}}
\answerTable{$-16$}{$16$}{$-8$}{$8$}{$-32$}
% Відповідь: Б

\vspace{0.5cm}

%======================================================================
% СТИЛЬ Г: ПОРІВНЯННЯ ТА ТВЕРДЖЕННЯ
%======================================================================

\begin{center}
{\large\textbf{\color{headerblue}Стиль Г: Порівняння}}
\end{center}

\vspace{0.3cm}

\task{18}{Порівняйте: $A = 2^{10}$ і $B = 10^3$. \nmtyear{gen-G}}
\answerTable{$A > B$}{$A < B$}{$A = B$}{Залежить від контексту}{Неможливо порівняти}
% Відповідь: А (1024 > 1000)

\vspace{0.5cm}

\task{19}{Яке число більше: $3^{20}$ чи $9^{10}$? \nmtyear{gen-G}}
\answerTable{$3^{20}$}{$9^{10}$}{Рівні}{Залежить від контексту}{$3^{20}$ вдвічі більше}
% Відповідь: В (9^10 = (3^2)^10 = 3^20)

\vspace{0.5cm}

\task{20}{Порівняйте $2^{30}$ і $4^{15}$. \nmtyear{gen-G}}
\answerTable{$2^{30} > 4^{15}$}{$2^{30} < 4^{15}$}{$2^{30} = 4^{15}$}{Залежить від контексту}{Неможливо порівняти}
% Відповідь: В (4^15 = 2^30)

\vspace{0.5cm}

\task{21}{Яке твердження правильне для $a > 1$? \nmtyear{gen-G}}

\vspace{0.2cm}
\begin{tabular}{ll}
\textbf{А} & $a^{-1} > a$ \\
\textbf{Б} & $a^{-1} < 1$ \\
\textbf{В} & $a^{-1} = 1$ \\
\textbf{Г} & $a^{-1} > 1$ \\
\textbf{Д} & $a^{-1} < 0$ \\
\end{tabular}
% Відповідь: Б

\vspace{0.7cm}

\task{22}{Порівняйте $0{,}5^{-2}$ і $2^2$. \nmtyear{gen-G}}
\answerTable{$0{,}5^{-2} > 2^2$}{$0{,}5^{-2} < 2^2$}{$0{,}5^{-2} = 2^2$}{Залежить від контексту}{Неможливо порівняти}
% Відповідь: В (0.5^-2 = 2^2 = 4)

\vspace{0.5cm}

\task{23}{Яке з чисел найбільше? \nmtyear{gen-G}}
\answerTable{$2^{100}$}{$3^{70}$}{$5^{50}$}{$10^{30}$}{$100^{15}$}
% Відповідь: А (log: 100log2 ≈ 30.1, 70log3 ≈ 33.4, 50log5 ≈ 34.9, 30, 30)
% Перевірка: 2^100 = 10^30.1, 3^70 = 10^33.4, 5^50 = 10^34.9, 10^30, 100^15 = 10^30
% Найбільше: В (5^50)

\vspace{0.5cm}

%======================================================================
% СТИЛЬ Д: ВЛАСТИВОСТІ СТЕПЕНІВ
%======================================================================

\begin{center}
{\large\textbf{\color{headerblue}Стиль Д: Властивості степенів}}
\end{center}

\vspace{0.3cm}

\task{24}{Яке твердження є правильним для будь-яких $a \neq 0$ і натуральних $m$, $n$? \nmtyear{gen-D}}

\vspace{0.2cm}
\begin{tabular}{ll}
\textbf{А} & $a^m \cdot a^n = a^{m \cdot n}$ \\
\textbf{Б} & $(a^m)^n = a^{m + n}$ \\
\textbf{В} & $a^m \cdot a^n = a^{m + n}$ \\
\textbf{Г} & $\dfrac{a^m}{a^n} = a^{m + n}$ \\
\textbf{Д} & $(a \cdot b)^n = a^n + b^n$ \\
\end{tabular}
% Відповідь: В

\vspace{0.7cm}

\task{25}{При якому значенні $k$ вірна рівність $2^{12} = 8^k$? \nmtyear{gen-D}}
\answerTable{$3$}{$4$}{$6$}{$9$}{$36$}
% Відповідь: Б (8^4 = (2^3)^4 = 2^12)

\vspace{0.5cm}

\task{26}{Якщо $3^x = 5$, то $3^{2x}$ дорівнює: \nmtyear{gen-D}}
\answerTable{$10$}{$15$}{$25$}{$125$}{$9$}
% Відповідь: В (3^2x = (3^x)^2 = 25)

\vspace{0.5cm}

\task{27}{Якщо $2^a = 3$ і $2^b = 5$, то $2^{a+b}$ дорівнює: \nmtyear{gen-D}}
\answerTable{$8$}{$15$}{$125$}{$6$}{$243$}
% Відповідь: Б (2^a · 2^b = 3 · 5 = 15)

\vspace{0.5cm}

\task{28}{Чому дорівнює $\dfrac{4^{10}}{2^{18}}$? \nmtyear{gen-D}}
\answerTable{$2$}{$4$}{$8$}{$16$}{$2^{-2}$}
% Відповідь: Б (2^20 / 2^18 = 2^2 = 4)

\vspace{0.5cm}

%======================================================================
% СТИЛЬ Е: СТАНДАРТНИЙ ВИГЛЯД
%======================================================================

\begin{center}
{\large\textbf{\color{headerblue}Стиль Е: Стандартний вигляд}}
\end{center}

\vspace{0.3cm}

\task{29}{Яке з чисел записано у стандартному вигляді? \nmtyear{gen-E}}
\answerTable{$0{,}5 \cdot 10^3$}{$12{,}3 \cdot 10^5$}{$5{,}67 \cdot 10^{-4}$}{$50 \cdot 10^2$}{$0{,}05 \cdot 10^6$}
% Відповідь: В (1 ≤ |a| < 10)

\vspace{0.5cm}

\task{30}{Число $0{,}00045$ у стандартному вигляді: \nmtyear{gen-E}}
\answerTable{$45 \cdot 10^{-5}$}{$4{,}5 \cdot 10^{-4}$}{$0{,}45 \cdot 10^{-3}$}{$4{,}5 \cdot 10^{4}$}{$45 \cdot 10^{-4}$}
% Відповідь: Б

\vspace{0.5cm}

\task{31}{Скільки нулів після коми у числі $3{,}5 \cdot 10^{-6}$? \nmtyear{gen-E}}
\answerTable{$4$}{$5$}{$6$}{$7$}{$3$}
% Відповідь: Б (0.0000035 - 5 нулів)

\vspace{0.5cm}

\task{32}{Число $2{,}5 \cdot 10^8$ у звичайному записі має: \nmtyear{gen-E}}
\answerTable{$8$ цифр}{$9$ цифр}{$10$ цифр}{$7$ цифр}{$6$ цифр}
% Відповідь: Б (250000000 - 9 цифр)

\vspace{0.5cm}

\task{33}{Яке число найменше? \nmtyear{gen-E}}
\answerTable{$10^{-3}$}{$10^{-5}$}{$10^{-2}$}{$10^{-4}$}{$10^{-1}$}
% Відповідь: Б

\vspace{0.5cm}

%======================================================================
% СТИЛЬ Ж: КОМБІНОВАНІ
%======================================================================

\begin{center}
{\large\textbf{\color{headerblue}Стиль Ж: Комбіновані}}
\end{center}

\vspace{0.3cm}

\task{34}{Знайдіть $x$, якщо $2^x \cdot 4^x = 64$. \nmtyear{gen-ZH}}
\answerTable{$1$}{$2$}{$3$}{$6$}{$\dfrac{3}{2}$}
% Відповідь: Б (2^x · 2^2x = 2^3x = 2^6, x = 2)

\vspace{0.5cm}

\task{35}{Обчисліть $\dfrac{3^{100} + 3^{99}}{3^{99}}$. \nmtyear{gen-ZH}}
\answerTable{$2$}{$3$}{$4$}{$3^{100}$}{$1$}
% Відповідь: В (3^99(3+1)/3^99 = 4)

\vspace{0.5cm}

\task{36}{Якщо $5^{2x} = 3$, то $5^{6x}$ дорівнює: \nmtyear{gen-ZH}}
\answerTable{$9$}{$27$}{$81$}{$15$}{$243$}
% Відповідь: Б (5^6x = (5^2x)^3 = 27)

\vspace{0.5cm}

\task{37}{Знайдіть останню цифру числа $7^{100}$. \nmtyear{gen-ZH}}
\answerTable{$1$}{$3$}{$7$}{$9$}{$49$}
% Відповідь: А (7^1=7, 7^2=49, 7^3=343, 7^4=2401, період 4; 100/4=25 ост 0, отже як 7^4 → 1)

\vspace{0.5cm}

\task{38}{Спростіть $\dfrac{6^n \cdot 12^n}{2^n \cdot 18^n}$. \nmtyear{gen-ZH}}
\answerTable{$1$}{$2$}{$2^n$}{$4$}{$3$}
% Відповідь: Б ((6·12)/(2·18))^n = (72/36)^n = 2^n... ні, результат = 2^n)
% Перерахую: 6^n · 12^n / 2^n · 18^n = (6·12/2·18)^n = (72/36)^n = 2^n. Відповідь В

\vspace{0.5cm}

\task{39}{При якому $n$ виконується $8^n = 4^{12}$? \nmtyear{gen-ZH}}
\answerTable{$6$}{$8$}{$12$}{$18$}{$4$}
% Відповідь: Б (2^3n = 2^24, 3n = 24, n = 8)

\vspace{0.5cm}

\task{40}{Обчисліть $25^{13} \cdot 4^{13} : 10^{25}$. \nmtyear{gen-ZH}}
\answerTable{$1$}{$10$}{$100$}{$0{,}1$}{$1000$}
% Відповідь: Б ((25·4)^13 / 10^25 = 100^13 / 10^25 = 10^26 / 10^25 = 10)

\vspace{0.5cm}

\task{41}{Скільки цифр у числі $2^{10} \cdot 5^{13}$? \nmtyear{gen-ZH}}
\answerTable{$10$}{$11$}{$13$}{$14$}{$23$}
% Відповідь: Г (2^10 · 5^13 = 2^10 · 5^10 · 5^3 = 10^10 · 125 = 125·10^10 = 1.25·10^12 → 13 цифр? 
% Ні: = 125 000 000 000 → 12 цифр. Точніше: 1250000000000 → 13 цифр)

\vspace{0.5cm}

\task{42}{Якщо $a = 2^{10}$ і $b = 5^{10}$, то $ab$ дорівнює: \nmtyear{gen-ZH}}
\answerTable{$10^{10}$}{$10^{20}$}{$7^{10}$}{$10^{100}$}{$7^{20}$}
% Відповідь: А

\vspace{0.5cm}

\end{document}
