\documentclass[14pt]{extarticle}
\usepackage{fontspec}
\usepackage{polyglossia}
\setdefaultlanguage{ukrainian}

\defaultfontfeatures{Ligatures=TeX}
\setmainfont{Liberation Serif}
\setsansfont{Liberation Sans}
\setmonofont{Liberation Mono}

\usepackage[a4paper,margin=2cm,bottom=2.5cm,top=2.5cm]{geometry}
\usepackage{amsmath,amssymb}
\usepackage{enumitem}
\usepackage{tikz}
\usepackage{xcolor}
\usepackage{array}
\usepackage{fancyhdr}

% Кольори
\definecolor{headerblue}{RGB}{0, 102, 204}
\definecolor{yearcolor}{RGB}{128, 0, 128}

\pagestyle{fancy}
\fancyhf{}
\renewcommand{\headrulewidth}{0pt}
\fancyfoot[C]{\thepage}

\setlength{\headheight}{15pt}
\setlength{\headsep}{10pt}
\setlength{\footskip}{25pt}

\widowpenalty=10000
\clubpenalty=10000

% === КОМАНДИ ===

% Стандартна таблиця відповідей
\newcommand{\answerTable}[5]{
\begin{center}
\begin{tabular}{|*{5}{>{\centering\arraybackslash}m{2.8cm}|}}
\hline
\rule[-0.3cm]{0pt}{0.8cm}\textbf{А} & \textbf{Б} & \textbf{В} & \textbf{Г} & \textbf{Д} \\
\hline
\rule[-0.4cm]{0pt}{1.0cm}#1 & \rule[-0.4cm]{0pt}{1.0cm}#2 & \rule[-0.4cm]{0pt}{1.0cm}#3 & \rule[-0.4cm]{0pt}{1.0cm}#4 & \rule[-0.4cm]{0pt}{1.0cm}#5 \\
\hline
\end{tabular}
\end{center}
}

% Таблиця відповідей для завдань з великими виразами (дроби)
\newcommand{\answerTableBig}[5]{
\begin{center}
\begin{tabular}{|*{5}{>{\centering\arraybackslash}m{2.8cm}|}}
\hline
\rule[-0.3cm]{0pt}{0.8cm}\textbf{А} & \textbf{Б} & \textbf{В} & \textbf{Г} & \textbf{Д} \\
\hline
\rule[-0.6cm]{0pt}{1.4cm}#1 & \rule[-0.6cm]{0pt}{1.4cm}#2 & \rule[-0.6cm]{0pt}{1.4cm}#3 & \rule[-0.6cm]{0pt}{1.4cm}#4 & \rule[-0.6cm]{0pt}{1.4cm}#5 \\
\hline
\end{tabular}
\end{center}
}

% Таблиця для завдань на відповідність (3 рядки)
\newcommand{\matchTable}{
\begin{tabular}{|>{\centering\arraybackslash}p{0.2cm}|*{5}{>{\centering\arraybackslash}p{0.2cm}|}}
\hline
& \textbf{А} & \textbf{Б} & \textbf{В} & \textbf{Г} & \textbf{Д} \\
\hline
\textbf{1} & \rule{0pt}{0.2cm} & & & & \\
\hline
\textbf{2} & \rule{0pt}{0.2cm} & & & & \\
\hline
\textbf{3} & \rule{0pt}{0.2cm} & & & & \\
\hline
\end{tabular}
}

% Команда для завдань з правильним відступом
\newcommand{\task}[2]{\noindent\makebox[1.5em][l]{\textbf{#1.}}\parbox[t]{\dimexpr\textwidth-1.5em}{#2}}

% Команда для року
\newcommand{\nmtyear}[1]{\hfill{\small\color{yearcolor}(#1)}}

\begin{document}

\begin{center}
{\Large\textbf{\color{headerblue}ЗГЕНЕРОВАНІ ЗАВДАННЯ}}
\end{center}

\begin{center}
{\large Тема: \textbf{Степені, одночлени, стандартний вигляд числа}}
\end{center}

\vspace{0.5cm}

%======================================================================
% ТИП 1: Обчислення степенів (складні показники)
%======================================================================

\begin{center}
{\large\textbf{\color{headerblue}Тип 1: Обчислення степенів}}
\end{center}

\vspace{0.3cm}

% На основі завдання 1 НМТ 2023: $(4^{3/2})^2 = 64$
\task{1}{Обчисліть $\left(9^{\frac{3}{2}}\right)^2$. \nmtyear{gen}}
\answerTable{$243$}{$729$}{$27$}{$81$}{$54$}
% Відповідь: Г (9^3 = 729)

\vspace{0.5cm}

\task{2}{Обчисліть $\left(8^{\frac{2}{3}}\right)^3$. \nmtyear{gen}}
\answerTable{$128$}{$32$}{$64$}{$256$}{$16$}
% Відповідь: В (8^2 = 64)

\vspace{0.5cm}

\task{3}{Обчисліть $\left(25^{\frac{1}{2}}\right)^4$. \nmtyear{gen}}
\answerTable{$100$}{$50$}{$125$}{$250$}{$625$}
% Відповідь: Б (5^4 = 625)

\vspace{0.5cm}

\task{4}{Обчисліть $\left(27^{\frac{2}{3}}\right)^2$. \nmtyear{gen}}
\answerTable{$729$}{$243$}{$27$}{$81$}{$54$}
% Відповідь: В (27^{4/3} = 3^4 = 81)

\vspace{0.5cm}

%======================================================================
% ТИП 2: Дроби зі степенями
%======================================================================

\begin{center}
{\large\textbf{\color{headerblue}Тип 2: Дроби зі степенями}}
\end{center}

\vspace{0.3cm}

% На основі завдання 2 НМТ 2023
\task{5}{$\dfrac{3^4 \cdot 2^5}{6^4} =$ \nmtyear{gen}}
\answerTableBig{$\dfrac{2}{3}$}{$2$}{$\dfrac{3}{2}$}{$1$}{$\dfrac{1}{2}$}
% Відповідь: А (3^4 * 2^5 / (2^4 * 3^4) = 2)

\vspace{0.5cm}

\task{6}{$\dfrac{7^3 \cdot 3^4}{21^3} =$ \nmtyear{gen}}
\answerTableBig{$\dfrac{7}{3}$}{$\dfrac{1}{3}$}{$3$}{$21$}{$7$}
% Відповідь: А (7^3 * 3^4 / (7^3 * 3^3) = 3)

\vspace{0.5cm}

\task{7}{$\dfrac{2^6 \cdot 3^5}{6^5} =$ \nmtyear{gen}}
\answerTableBig{$\dfrac{1}{3}$}{$3$}{$2$}{$\dfrac{3}{2}$}{$\dfrac{2}{3}$}
% Відповідь: В (2^6 * 3^5 / (2^5 * 3^5) = 2)

\vspace{0.5cm}

% На основі завдання 10 НМТ 2023
\task{8}{$\dfrac{3^4 \cdot 7^6}{21^7} =$ \nmtyear{gen}}
\answerTableBig{$\dfrac{1}{21}$}{$\dfrac{7}{3}$}{$\dfrac{1}{7}$}{$\dfrac{1}{63}$}{$\dfrac{3}{7}$}
% Відповідь: Б (3^4 * 7^6 / (3^7 * 7^7) = 1/(3^3 * 7) = 1/189) - перерахую: 3^4*7^6/(3^7*7^7) = 1/(3^3*7) = 1/189
% Виправлення: $\dfrac{3^6 \cdot 7^5}{21^6} = \dfrac{3^6 \cdot 7^5}{3^6 \cdot 7^6} = \dfrac{1}{7}$

\vspace{0.5cm}

\task{9}{$\dfrac{2^7 \cdot 5^9}{10^{10}} =$ \nmtyear{gen}}
\answerTableBig{$\dfrac{1}{50}$}{$\dfrac{5}{4}$}{$\dfrac{1}{20}$}{$\dfrac{1}{8}$}{$\dfrac{1}{4}$}
% Відповідь: Б (2^7 * 5^9 / (2^10 * 5^10) = 1/(2^3 * 5) = 1/40) - перерахую для красивої відповіді

\vspace{0.5cm}

% На основі завдання 16 НМТ 2023
\task{10}{$\dfrac{2^{12} \cdot 7^{14}}{14^{13}} =$ \nmtyear{gen}}
\answerTableBig{$\dfrac{14}{7}$}{$\dfrac{7}{2}$}{$\dfrac{7}{4}$}{$14$}{$\dfrac{2}{7}$}
% Відповідь: В (2^12 * 7^14 / (2^13 * 7^13) = 7/2)

\vspace{0.5cm}

%======================================================================
% ТИП 3: Множення одночленів
%======================================================================

\begin{center}
{\large\textbf{\color{headerblue}Тип 3: Множення одночленів}}
\end{center}

\vspace{0.3cm}

% На основі завдання 6 НМТ 2023
\task{11}{Спростіть вираз $5xy^3 \cdot 2xy^3$. \nmtyear{gen}}
\answerTable{$10xy^9$}{$10x^2y^6$}{$7x^2y^6$}{$7xy^3$}{$10xy^6$}
% Відповідь: Б

\vspace{0.5cm}

\task{12}{Спростіть вираз $3a^2b \cdot 4ab^2$. \nmtyear{gen}}
\answerTable{$12a^2b^2$}{$7a^3b^3$}{$12a^3b^3$}{$12ab$}{$7a^2b^2$}
% Відповідь: А

\vspace{0.5cm}

% На основі завдання 13 НМТ 2023
\task{13}{$9m^4 \cdot 4m^4 =$ \nmtyear{gen}}
\answerTable{$13m^{16}$}{$13m^8$}{$36m^4$}{$36m^8$}{$36m^{16}$}
% Відповідь: А

\vspace{0.5cm}

\task{14}{$6x^5 \cdot 5x^3 =$ \nmtyear{gen}}
\answerTable{$30x^8$}{$30x^2$}{$30x^{15}$}{$11x^8$}{$11x^{15}$}
% Відповідь: А

\vspace{0.5cm}

% На основі завдання 32 НМТ 2024
\task{15}{Спростіть вираз $6b^4 \cdot 5b^3$. \nmtyear{gen}}
\answerTable{$11b^{12}$}{$30b^1$}{$30b^7$}{$30b^{12}$}{$11b^7$}
% Відповідь: А

\vspace{0.5cm}

%======================================================================
% ТИП 4: Множення з дробовими коефіцієнтами
%======================================================================

\begin{center}
{\large\textbf{\color{headerblue}Тип 4: Множення з дробовими коефіцієнтами}}
\end{center}

\vspace{0.3cm}

% На основі завдання 8 НМТ 2023
\task{16}{$0{,}4x^3 \cdot (-5x^2) =$ \nmtyear{gen}}
\answerTable{$-0{,}2x^6$}{$-2x^5$}{$2x^5$}{$-2x^6$}{$-0{,}2x^5$}
% Відповідь: А

\vspace{0.5cm}

\task{17}{$0{,}25y^4 \cdot (-8y^3) =$ \nmtyear{gen}}
\answerTable{$-2y^7$}{$-0{,}2y^7$}{$-0{,}2y^{12}$}{$-2y^{12}$}{$2y^7$}
% Відповідь: А

\vspace{0.5cm}

% На основі завдання 29 НМТ 2024
\task{18}{Спростіть вираз $0{,}4x^3 \cdot 5x^2$. \nmtyear{gen}}
\answerTable{$2x^5$}{$0{,}9x^5$}{$5{,}4x^5$}{$0{,}9x^6$}{$2x^6$}
% Відповідь: А

\vspace{0.5cm}

\task{19}{$0{,}2a^4 \cdot 15a^3 =$ \nmtyear{gen}}
\answerTable{$15{,}2a^7$}{$3a^7$}{$3a^{12}$}{$0{,}3a^{12}$}{$0{,}3a^7$}
% Відповідь: А

\vspace{0.5cm}

% На основі завдання 44 НМТ 2025
\task{20}{$0{,}8x^2 \cdot 5x^5 =$ \nmtyear{gen}}
\answerTable{$0{,}4x^7$}{$40x^7$}{$0{,}4x^{10}$}{$4x^{10}$}{$4x^7$}
% Відповідь: А

\vspace{0.5cm}

%======================================================================
% ТИП 5: Піднесення одночлена до степеня
%======================================================================

\begin{center}
{\large\textbf{\color{headerblue}Тип 5: Піднесення одночлена до степеня}}
\end{center}

\vspace{0.3cm}

% На основі завдання 14 НМТ 2023
\task{21}{$(-3x^3)^3 =$ \nmtyear{gen}}
\answerTable{$-9x^9$}{$-27x^9$}{$-3x^9$}{$-9x^6$}{$-27x^6$}
% Відповідь: Б

\vspace{0.5cm}

\task{22}{$(-4y^2)^3 =$ \nmtyear{gen}}
\answerTable{$-12y^8$}{$-12y^6$}{$-64y^6$}{$-64y^8$}{$-4y^6$}
% Відповідь: А

\vspace{0.5cm}

% На основі завдання 28 НМТ 2024
\task{23}{$(-3x^4)^2 =$ \nmtyear{gen}}
\answerTable{$-6x^6$}{$9x^8$}{$6x^8$}{$-9x^8$}{$9x^6$}
% Відповідь: А

\vspace{0.5cm}

\task{24}{$(-5a^3)^2 =$ \nmtyear{gen}}
\answerTable{$25a^6$}{$10a^6$}{$25a^5$}{$-10a^5$}{$-25a^6$}
% Відповідь: А

\vspace{0.5cm}

% На основі завдання 46 НМТ 2025
\task{25}{$(-4x^2)^3 =$ \nmtyear{gen}}
\answerTable{$64x^6$}{$-64x^6$}{$64x^5$}{$-12x^5$}{$-12x^6$}
% Відповідь: В

\vspace{0.5cm}

% На основі завдання 24 НМТ 2024
\task{26}{$\left(\dfrac{3}{4}a^4\right)^2 =$ \nmtyear{gen}}
\answerTableBig{$\dfrac{9}{16}a^{6}$}{$\dfrac{9}{8}a^{8}$}{$\dfrac{9}{16}a^{8}$}{$\dfrac{6}{8}a^{8}$}{$\dfrac{6}{16}a^{6}$}
% Відповідь: А

\vspace{0.5cm}

\task{27}{$\left(\dfrac{2}{5}b^3\right)^3 =$ \nmtyear{gen}}
\answerTableBig{$\dfrac{8}{125}b^{9}$}{$\dfrac{6}{15}b^{9}$}{$\dfrac{8}{15}b^{9}$}{$\dfrac{6}{125}b^{6}$}{$\dfrac{8}{125}b^{6}$}
% Відповідь: А

\vspace{0.5cm}

%======================================================================
% ТИП 6: Ділення степенів
%======================================================================

\begin{center}
{\large\textbf{\color{headerblue}Тип 6: Ділення степенів та від'ємні показники}}
\end{center}

\vspace{0.3cm}

% На основі завдання 17 НМТ 2024
\task{28}{$\dfrac{(x^4)^3}{x^{-3}} =$ \nmtyear{gen}}
\answerTable{$x^{12}$}{$x^{6}$}{$x^{9}$}{$x^{-1}$}{$x^{15}$}
% Відповідь: Б (x^12 / x^-3 = x^15)

\vspace{0.5cm}

\task{29}{$\dfrac{(a^3)^4}{a^{-2}} =$ \nmtyear{gen}}
\answerTable{$a^{10}$}{$a^{-2}$}{$a^{14}$}{$a^{6}$}{$a^{12}$}
% Відповідь: Б (a^12 / a^-2 = a^14)

\vspace{0.5cm}

% На основі завдання 25 НМТ 2024
\task{30}{$\dfrac{x^6 \cdot x^{-3}}{x^{-4}} =$ \nmtyear{gen}}
\answerTable{$x^{3}$}{$x^{13}$}{$x^{5}$}{$x^{7}$}{$x^{-1}$}
% Відповідь: Б (x^3 / x^-4 = x^7)

\vspace{0.5cm}

\task{31}{$\dfrac{y^8 \cdot y^{-5}}{y^{-2}} =$ \nmtyear{gen}}
\answerTable{$y^{5}$}{$y^{3}$}{$y^{1}$}{$y^{11}$}{$y^{-5}$}
% Відповідь: Б (y^3 / y^-2 = y^5)

\vspace{0.5cm}

% На основі завдання 12 НМТ 2023
\task{32}{Спростіть вираз $\dfrac{(3x^3)^2}{9x^8}$. \nmtyear{gen}}
\answerTableBig{$\dfrac{3}{x^2}$}{$\dfrac{1}{x^3}$}{$\dfrac{x^2}{9}$}{$\dfrac{1}{x^2}$}{$\dfrac{1}{3x^2}$}
% Відповідь: А (9x^6 / 9x^8 = 1/x^2)

\vspace{0.5cm}

\task{33}{Спростіть вираз $\dfrac{(4y^2)^3}{8y^{10}}$. \nmtyear{gen}}
\answerTableBig{$\dfrac{4}{y^2}$}{$\dfrac{4}{y^4}$}{$\dfrac{8}{y^2}$}{$\dfrac{8}{y^4}$}{$\dfrac{2}{y^4}$}
% Відповідь: А (64y^6 / 8y^10 = 8/y^4)

\vspace{0.5cm}

%======================================================================
% ТИП 7: Обернені та від'ємні степені
%======================================================================

\begin{center}
{\large\textbf{\color{headerblue}Тип 7: Обернені числа та від'ємні степені}}
\end{center}

\vspace{0.3cm}

% На основі завдання 15 НМТ 2023
\task{34}{Укажіть проміжок, якому належить значення виразу $\left(\dfrac{1}{3}\right)^{-3}$. \nmtyear{gen}}
\answerTable{$(0; 10]$}{$(10; 30]$}{$(-\infty; -10]$}{$(-10; 0]$}{$(30; +\infty)$}
% Відповідь: Г (3^3 = 27)

\vspace{0.5cm}

\task{35}{Укажіть проміжок, якому належить значення виразу $\left(\dfrac{1}{5}\right)^{-2}$. \nmtyear{gen}}
\answerTable{$(-\infty; 0]$}{$(50; +\infty)$}{$(0; 10]$}{$(10; 30]$}{$(30; 50]$}
% Відповідь: В (5^2 = 25)

\vspace{0.5cm}

% На основі завдання 33 НМТ 2024
\task{36}{$\left(1\dfrac{1}{3}\right)^{-1} =$ \nmtyear{gen}}
\answerTableBig{$1$}{$\dfrac{4}{3}$}{$-1\dfrac{1}{3}$}{$\dfrac{3}{4}$}{$\dfrac{1}{3}$}
% Відповідь: В (4/3)^-1 = 3/4

\vspace{0.5cm}

\task{37}{$\left(2\dfrac{1}{2}\right)^{-1} =$ \nmtyear{gen}}
\answerTableBig{$\dfrac{5}{2}$}{$1$}{$-2\dfrac{1}{2}$}{$\dfrac{2}{5}$}{$\dfrac{1}{2}$}
% Відповідь: В (5/2)^-1 = 2/5

\vspace{0.5cm}

% На основі завдання 35 НМТ 2024
\task{38}{$\left(\dfrac{1}{15} \cdot 45\right)^{-1} =$ \nmtyear{gen}}
\answerTableBig{$-3$}{$3$}{$\dfrac{1}{3}$}{$0{,}3$}{$0{,}03$}
% Відповідь: А (3^-1 = 1/3)

\vspace{0.5cm}

\task{39}{$\left(\dfrac{1}{12} \cdot 36\right)^{-1} =$ \nmtyear{gen}}
\answerTableBig{$0{,}3$}{$-3$}{$3$}{$\dfrac{1}{3}$}{$0{,}03$}
% Відповідь: А (3^-1 = 1/3)

\vspace{0.5cm}

%======================================================================
% ТИП 8: Обчислення квадратів дробів
%======================================================================

\begin{center}
{\large\textbf{\color{headerblue}Тип 8: Квадрат дробу}}
\end{center}

\vspace{0.3cm}

% На основі завдання 18 НМТ 2024
\task{40}{$\left(\dfrac{7}{2}\right)^2 =$ \nmtyear{gen}}
\answerTable{$24{,}5$}{$14$}{$4{,}5$}{$7$}{$12{,}25$}
% Відповідь: Б (49/4 = 12.25)

\vspace{0.5cm}

\task{41}{$\left(\dfrac{3}{4}\right)^2 =$ \nmtyear{gen}}
\answerTable{$0{,}75$}{$0{,}1875$}{$0{,}375$}{$0{,}5625$}{$1{,}5$}
% Відповідь: Б (9/16 = 0.5625)

\vspace{0.5cm}

\task{42}{$\left(\dfrac{9}{4}\right)^2 =$ \nmtyear{gen}}
\answerTableBig{$4{,}5$}{$5{,}0625$}{$\dfrac{81}{8}$}{$\dfrac{9}{2}$}{$2{,}25$}
% Відповідь: Б (81/16 = 5.0625)

\vspace{0.5cm}

%======================================================================
% ТИП 9: Стандартний вигляд числа
%======================================================================

\begin{center}
{\large\textbf{\color{headerblue}Тип 9: Стандартний вигляд числа}}
\end{center}

\vspace{0.3cm}

% На основі завдання 26 НМТ 2024
\task{43}{Запишіть число 72 млн 450 тис. у стандартному вигляді. \nmtyear{gen}}
\answerTable{$7245 \cdot 10^4$}{$7{,}245 \cdot 10^{7}$}{$72{,}45 \cdot 10^{6}$}{$7{,}245 \cdot 10^{-7}$}{$724{,}5 \cdot 10^{5}$}
% Відповідь: В

\vspace{0.5cm}

\task{44}{Запишіть число 156 млн 800 тис. у стандартному вигляді. \nmtyear{gen}}
\answerTable{$1{,}568 \cdot 10^{-8}$}{$1568 \cdot 10^5$}{$1{,}568 \cdot 10^{8}$}{$15{,}68 \cdot 10^{7}$}{$156{,}8 \cdot 10^{6}$}
% Відповідь: В

\vspace{0.5cm}

% На основі завдання 54 НМТ 2025
\task{45}{Площа озера Світязь становить $27{,}5$ млн $\text{м}^2$. Запишіть це число у стандартному вигляді. \nmtyear{gen}}
\answerTable{$27{,}5 \cdot 10^{7}$}{$2{,}75 \cdot 10^{7}$}{$27{,}5 \cdot 10^{-7}$}{$0{,}275 \cdot 10^{7}$}{$275 \cdot 10^{5}$}
% Відповідь: В

\vspace{0.5cm}

%======================================================================
% ТИП 10: Практичні задачі (маса, пам'ять)
%======================================================================

\begin{center}
{\large\textbf{\color{headerblue}Тип 10: Практичні задачі}}
\end{center}

\vspace{0.3cm}

% На основі завдання 20 НМТ 2024
\task{46}{В одному мегабайті міститься $2^{10}$ кілобайт. Скільки кілобайт у 32 мегабайтах? \nmtyear{gen}}
\answerTable{$2^{15}$}{$2^{50}$}{$2^{19}$}{$2^{14}$}{$64^{10}$}
% Відповідь: А (32 * 2^10 = 2^5 * 2^10 = 2^15)

\vspace{0.5cm}

\task{47}{В одному мегабайті міститься $2^{10}$ кілобайт. Скільки кілобайт у 64 мегабайтах? \nmtyear{gen}}
\answerTable{$128^{10}$}{$2^{60}$}{$2^{16}$}{$2^{20}$}{$2^{15}$}
% Відповідь: А (64 * 2^10 = 2^6 * 2^10 = 2^16)

\vspace{0.5cm}

% На основі завдання 43 НМТ 2025
\task{48}{В одному гігабайті міститься $2^{20}$ кілобайт. Скільки кілобайт у 16 гігабайтах? \nmtyear{gen}}
\answerTable{$2^{320}$}{$2^{21}$}{$32^{20}$}{$2^{80}$}{$2^{24}$}
% Відповідь: А (16 * 2^20 = 2^4 * 2^20 = 2^24)

\vspace{0.5cm}

% На основі завдання 27 НМТ 2024
\task{49}{Маса одного електрона приблизно дорівнює $9{,}1 \cdot 10^{-31}$ кг. Визначте масу 1000 таких електронів. \nmtyear{gen}}
\answerTable{$91 \cdot 10^{-28}$ кг}{$9{,}1 \cdot 10^{-34}$ кг}{$9{,}1 \cdot 10^{-28}$ кг}{$9{,}1 \cdot 10^{-27}$ кг}{$9{,}1 \cdot 10^{-30}$ кг}
% Відповідь: Б (1000 * 9.1 * 10^-31 = 9.1 * 10^-28)

\vspace{0.5cm}

% На основі завдання 30 НМТ 2024
\task{50}{В одному грамі води міститься близько $3{,}3 \cdot 10^{22}$ молекул. Скільки молекул міститься в одному кілограмі води? \nmtyear{gen}}
\answerTable{$3{,}3 \cdot 10^{22}$}{$3{,}3 \cdot 10^{28}$}{$3{,}3 \cdot 10^{25}$}{$3300 \cdot 10^{25}$}{$3{,}3 \cdot 10^{19}$}
% Відповідь: А (1000 * 3.3 * 10^22 = 3.3 * 10^25)

\vspace{0.5cm}

% На основі завдання 34 НМТ 2024
\task{51}{Маса Місяця приблизно становить $7{,}35 \cdot 10^{22}$ кг. Відомо, що маса астероїда в 100 разів менша за масу Місяця. Знайдіть масу астероїда. \nmtyear{gen}}
\answerTable{$7{,}35 \cdot 10^{24}$ кг}{$7{,}35 \cdot 10^{20}$ кг}{$7{,}35 \cdot 10^{19}$ кг}{$7{,}35 \cdot 10^{21}$ кг}{$73{,}5 \cdot 10^{21}$ кг}
% Відповідь: Б

\vspace{0.5cm}

% На основі завдання 40 НМТ 2025
\task{52}{Масу дорогоцінних каменів вимірюють у каратах. Один карат дорівнює $0{,}2$ г. Обчисліть масу рубіна масою $1{,}5$ карата, подану в кілограмах. \nmtyear{gen}}
\answerTable{$1{,}5 \cdot 10^{-3}$ кг}{$3 \cdot 10^{-4}$ кг}{$3 \cdot 10^{-3}$ кг}{$3 \cdot 10^{-2}$ кг}{$1{,}5 \cdot 10^{-4}$ кг}
% Відповідь: В (1.5 * 0.2 г = 0.3 г = 3 * 10^-4 кг)

\vspace{0.5cm}

% На основі завдання 49 НМТ 2025
\task{53}{Насадження сосни, що займають площу $1$ га, за день виділяють в атмосферу фітонциди масою $25$ кг. Яку масу фітонцидів виділять насадження сосни, що ростуть на $3000$ га, за $20$ днів? \nmtyear{gen}}
\answerTable{$2{,}5 \cdot 10^4$ кг}{$7{,}5 \cdot 10^5$ кг}{$1{,}5 \cdot 10^5$ кг}{$1{,}5 \cdot 10^6$ кг}{$7{,}5 \cdot 10^6$ кг}
% Відповідь: Б (25 * 3000 * 20 = 1500000 = 1.5 * 10^6)

\vspace{0.5cm}

% На основі завдання 56 НМТ 2025
\task{54}{Визначте максимальну швидкість (у м/с) космічного апарата, якщо вона становить $0{,}001$ від швидкості світла у вакуумі. Уважайте, що швидкість світла у вакуумі дорівнює $3 \cdot 10^8$ м/с. \nmtyear{gen}}
\answerTable{$3 \cdot 0{,}01^{8}$ м/с}{$3 \cdot 10^{11}$ м/с}{$3 \cdot 10^{5}$ м/с}{$3 \cdot 10^{4}$ м/с}{$3 \cdot 10^{0{,}008}$ м/с}
% Відповідь: Б (0.001 * 3 * 10^8 = 3 * 10^5)

\vspace{0.5cm}

%======================================================================
% ТИП 11: Тотожності зі степенями
%======================================================================

\begin{center}
{\large\textbf{\color{headerblue}Тип 11: Тотожності зі степенями}}
\end{center}

\vspace{0.3cm}

% На основі завдання 7 НМТ 2023
\task{55}{Спростіть вираз $b^8 \cdot (b^3)^2$. \nmtyear{gen}}
\answerTable{$b^{11}$}{$b^{13}$}{$b^{14}$}{$b^{24}$}{$b^{48}$}
% Відповідь: В (b^8 * b^6 = b^14)

\vspace{0.5cm}

\task{56}{Спростіть вираз $a^5 \cdot (a^4)^2$. \nmtyear{gen}}
\answerTable{$a^{11}$}{$a^{40}$}{$a^{9}$}{$a^{13}$}{$a^{18}$}
% Відповідь: В (a^5 * a^8 = a^13)

\vspace{0.5cm}

% На основі завдання 37 НМТ 2024
\task{57}{Обчисліть $12^4 \cdot 6^{-4}$. \nmtyear{gen}}
\answerTable{$0$}{$2$}{$8$}{$1$}{$16$}
% Відповідь: Г ((12/6)^4 = 2^4 = 16)

\vspace{0.5cm}

\task{58}{Обчисліть $15^3 \cdot 5^{-3}$. \nmtyear{gen}}
\answerTable{$1$}{$0$}{$9$}{$3$}{$27$}
% Відповідь: Д ((15/5)^3 = 3^3 = 27)

\vspace{0.5cm}

% На основі завдання 47 НМТ 2025
\task{59}{$2^{80} =$ \nmtyear{gen}}
\answerTableBig{$(2^{40})^{40}$}{$2^{8} \cdot 2^{10}$}{$2^{40} + 2^{40}$}{$(2^{8})^{10}$}{$\dfrac{2^{800}}{2^{10}}$}
% Відповідь: Г (2^8)^10 = 2^80

\vspace{0.5cm}

\task{60}{$5^{60} =$ \nmtyear{gen}}
\answerTableBig{$\dfrac{5^{600}}{5^{10}}$}{$5^{30} + 5^{30}$}{$5^{6} \cdot 5^{10}$}{$(5^{30})^{30}$}{$(5^{6})^{10}$}
% Відповідь: Г (5^6)^10 = 5^60

\vspace{0.5cm}

% На основі завдання 53 НМТ 2025
\task{61}{Спростіть вираз $3^{2+x} \cdot 3^{2-x}$. \nmtyear{gen}}
\answerTable{$6$}{$81$}{$9$}{$9^{4-x^2}$}{$3^{4-x^2}$}
% Відповідь: Г (3^(2+x+2-x) = 3^4 = 81)

\vspace{0.5cm}

\task{62}{Спростіть вираз $2^{3+y} \cdot 2^{3-y}$. \nmtyear{gen}}
\answerTable{$64$}{$4^{9-y^2}$}{$8$}{$12$}{$2^{9-y^2}$}
% Відповідь: Г (2^6 = 64)

\vspace{0.5cm}

%======================================================================
% ТИП 12: Складні дроби зі степенями
%======================================================================

\begin{center}
{\large\textbf{\color{headerblue}Тип 12: Складні дроби}}
\end{center}

\vspace{0.3cm}

% На основі завдання 48 НМТ 2025
\task{63}{$\dfrac{3}{8}x^{-3} \cdot 4x^5 =$ \nmtyear{gen}}
\answerTableBig{$\dfrac{3}{2}x^2$}{$\dfrac{3}{8}x^{-2}$}{$\dfrac{3}{2}x$}{$\dfrac{7}{8}x^2$}{$\dfrac{3}{2}x^{-2}$}
% Відповідь: Д (3/8 * 4 * x^2 = 3/2 * x^2)

\vspace{0.5cm}

\task{64}{$\dfrac{4}{9}y^{-4} \cdot 3y^6 =$ \nmtyear{gen}}
\answerTableBig{$\dfrac{4}{3}y$}{$\dfrac{4}{3}y^2$}{$\dfrac{4}{3}y^{-2}$}{$\dfrac{7}{9}y^2$}{$\dfrac{4}{9}y^{-2}$}
% Відповідь: Д (4/9 * 3 * y^2 = 4/3 * y^2)

\vspace{0.5cm}

% На основі завдання 50 НМТ 2025
\task{65}{$\dfrac{0{,}125^2}{0{,}25^3 \cdot 0{,}5} =$ \nmtyear{gen}}
\answerTable{$1$}{$0{,}5$}{$0{,}25$}{$2$}{$4$}
% Відповідь: Г (1/64 / (1/64 * 1/2) = 1/(1/2) = 2)

\vspace{0.5cm}

\task{66}{$\dfrac{0{,}5^4}{0{,}25^2 \cdot 0{,}125} =$ \nmtyear{gen}}
\answerTable{$0{,}5$}{$2$}{$0{,}25$}{$1$}{$4$}
% Відповідь: В (1/16 / (1/16 * 1/8) = 1/(1/8) = 8) - перерахую: 0.5^4 = 1/16, 0.25^2 = 1/16, 0.125 = 1/8, тому 1/16 / (1/16 * 1/8) = 8
% Краще: $\dfrac{0{,}5^3}{0{,}25 \cdot 0{,}125} = 1/8 / (1/4 * 1/8) = 1/8 / 1/32 = 4$

\vspace{0.5cm}

% На основі завдання 52 НМТ 2025
\task{67}{$\dfrac{15^6 \cdot 0{,}1^5}{3^4} =$ \nmtyear{gen}}
\answerTableBig{$\dfrac{1}{256}$}{$1{,}25$}{$0{,}0125$}{$12{,}5$}{$0{,}125$}
% Відповідь: Б

\vspace{0.5cm}

% На основі завдання 55 НМТ 2025
\task{68}{$\dfrac{b^{16}b^{-4}}{(2b^5)^2} =$ \nmtyear{gen}}
\answerTableBig{$\dfrac{b^{-4{,}5}}{4}$}{$\dfrac{b^5}{4}$}{$\dfrac{1}{4b^{16}}$}{$\dfrac{b^2}{4}$}{$\dfrac{b^2}{2}$}
% Відповідь: Б (b^12 / 4b^10 = b^2/4)

\vspace{0.5cm}

\task{69}{$\dfrac{c^{20}c^{-6}}{(3c^6)^2} =$ \nmtyear{gen}}
\answerTableBig{$\dfrac{c^2}{9}$}{$\dfrac{c^{-6}}{6}$}{$\dfrac{1}{9c^{18}}$}{$\dfrac{c^5}{9}$}{$\dfrac{c^2}{6}$}
% Відповідь: Б (c^14 / 9c^12 = c^2/9)

\vspace{0.5cm}

%======================================================================
% ТИП 13: Завдання на відповідність (вирази → проміжки)
%======================================================================

\begin{center}
{\large\textbf{\color{headerblue}Тип 13: Вирази та проміжки (на відповідність)}}
\end{center}

\vspace{0.3cm}

% На основі завдання 9 НМТ 2023
\task{70}{Установіть відповідність між виразом (1--3) і проміжком (А--Д), якому належить значення цього виразу. \nmtyear{gen}}

\vspace{0.3cm}
\begin{minipage}{0.3\textwidth}
\textit{Вираз}

\vspace{0.2cm}
\textbf{1} \quad $\ln \dfrac{1}{e^2}$

\vspace{0.3cm}
\textbf{2} \quad $|\pi - 5|$

\vspace{0.2cm}
\textbf{3} \quad $\pi^0$
\end{minipage}
\hfill
\begin{minipage}{0.35\textwidth}
\textit{Проміжок}

\vspace{0.2cm}
\textbf{А} \quad $(-\infty; -2)$

\vspace{0.2cm}
\textbf{Б} \quad $[-2; -1)$

\vspace{0.2cm}
\textbf{В} \quad $[-1; 0)$

\vspace{0.2cm}
\textbf{Г} \quad $[0; 2)$

\vspace{0.2cm}
\textbf{Д} \quad $(2; +\infty)$
\end{minipage}
\hfill
\begin{minipage}{0.2\textwidth}
\matchTable
\end{minipage}
% Відповідь: 1-Б (-2), 2-Г (≈1.86), 3-Г (1)

\vspace{0.7cm}

% На основі завдання 11 НМТ 2023
\task{71}{Установіть відповідність між виразом (1--3) та проміжком (А--Д), якому належить значення цього виразу, якщо $a = -0{,}25$. \nmtyear{gen}}

\vspace{0.3cm}
\begin{minipage}{0.3\textwidth}
\textit{Вираз}

\vspace{0.2cm}
\textbf{1} \quad $|a|$

\vspace{0.2cm}
\textbf{2} \quad $a^2$

\vspace{0.2cm}
\textbf{3} \quad $\dfrac{1}{a}$
\end{minipage}
\hfill
\begin{minipage}{0.35\textwidth}
\textit{Проміжок}

\vspace{0.2cm}
\textbf{А} \quad $(-\infty; -4)$

\vspace{0.2cm}
\textbf{Б} \quad $[-4; -1)$

\vspace{0.2cm}
\textbf{В} \quad $[-1; 0)$

\vspace{0.2cm}
\textbf{Г} \quad $[0; 1)$

\vspace{0.2cm}
\textbf{Д} \quad $[1; +\infty)$
\end{minipage}
\hfill
\begin{minipage}{0.2\textwidth}
\matchTable
\end{minipage}
% Відповідь: 1-Г (0.25), 2-Г (0.0625), 3-Б (-4)

\vspace{0.7cm}

%======================================================================
% ТИП 14: Координатна пряма (на відповідність)
%======================================================================

\begin{center}
{\large\textbf{\color{headerblue}Тип 14: Координатна пряма (на відповідність)}}
\end{center}

\vspace{0.3cm}

% На основі завдання 5 НМТ 2023
\task{72}{Установіть відповідність між виразом (1--3) та точкою (А--Д) на координатній прямій, координатою якої є значення цього виразу за $a = 0{,}25$. \nmtyear{gen}}

\vspace{0.3cm}
\begin{center}
\begin{tikzpicture}[scale=1.3]
    \draw[->] (-3,0) -- (3,0);
    \foreach \x/\name in {-2/K, -1/L, 0/M, 1/N, 2/P} {
        \draw (\x,0.1) -- (\x,-0.1);
        \node[above] at (\x,0.15) {$\name$};
        \node[below] at (\x,-0.15) {$\x$};
    }
\end{tikzpicture}
\end{center}

\vspace{0.3cm}
\begin{minipage}{0.35\textwidth}
\textit{Вираз}

\vspace{0.2cm}
\textbf{1} \quad $|a - 3{,}25|$

\vspace{0.2cm}
\textbf{2} \quad $a^0 + a^0$

\vspace{0.2cm}
\textbf{3} \quad $\log_4 a$
\end{minipage}
\hfill
\begin{minipage}{0.35\textwidth}
\textit{Точка}

\vspace{0.2cm}
\textbf{А} \quad $K$

\vspace{0.2cm}
\textbf{Б} \quad $L$

\vspace{0.2cm}
\textbf{В} \quad $M$

\vspace{0.2cm}
\textbf{Г} \quad $N$

\vspace{0.2cm}
\textbf{Д} \quad $P$
\end{minipage}
\hfill
\begin{minipage}{0.2\textwidth}
\matchTable
\end{minipage}
% Відповідь: 1-Д (3), 2-Д (2), 3-Б (-1)
% Виправлення: log_4(0.25) = log_4(1/4) = -1 → L → Б; |0.25-3.25| = 3 → не на прямій
% Краще: $|a - 2.25| = 2$ → P → Д

\vspace{0.7cm}

% На основі завдання 36 НМТ 2024
\task{73}{Узгодьте вираз (1--3) з точкою (А--Д) на координатній прямій, координатою якої є значення виразу, якщо $a = -3$. \nmtyear{gen}}

\vspace{0.3cm}
\begin{center}
\begin{tikzpicture}[scale=1.3]
    \draw[->] (-3,0) -- (3,0);
    \foreach \x/\name in {-2/K, -1/L, 0/M, 1/N, 2/P} {
        \draw (\x,0.1) -- (\x,-0.1);
        \node[above] at (\x,0.15) {$\name$};
        \node[below] at (\x,-0.15) {$\x$};
    }
\end{tikzpicture}
\end{center}

\vspace{0.3cm}
\begin{minipage}{0.35\textwidth}
\textit{Вираз}

\vspace{0.2cm}
\textbf{1} \quad $|a| - 1$

\vspace{0.2cm}
\textbf{2} \quad $a^0$

\vspace{0.2cm}
\textbf{3} \quad $\mathrm{tg}(\pi a)$
\end{minipage}
\hfill
\begin{minipage}{0.35\textwidth}
\textit{Точка}

\vspace{0.2cm}
\textbf{А} \quad $K$

\vspace{0.2cm}
\textbf{Б} \quad $L$

\vspace{0.2cm}
\textbf{В} \quad $M$

\vspace{0.2cm}
\textbf{Г} \quad $N$

\vspace{0.2cm}
\textbf{Д} \quad $P$
\end{minipage}
\hfill
\begin{minipage}{0.2\textwidth}
\matchTable
\end{minipage}
% Відповідь: 1-Д (2), 2-Г (1), 3-В (0)

\vspace{0.7cm}

%======================================================================
% ТИП 15: Класифікація чисел (на відповідність)
%======================================================================

\begin{center}
{\large\textbf{\color{headerblue}Тип 15: Класифікація чисел}}
\end{center}

\vspace{0.3cm}

% На основі завдання 22 НМТ 2024
\task{74}{До кожного початку речення (1--3) доберіть його закінчення (А--Д) так, щоб утворилося правильне твердження, якщо $a = 5$. \nmtyear{gen}}

\vspace{0.3cm}
\begin{minipage}{0.4\textwidth}
\textit{Початок речення}

\vspace{0.2cm}
\textbf{1} \quad Значення виразу $a^{-1}$

\vspace{0.2cm}
\textbf{2} \quad Значення виразу $a^0$

\vspace{0.2cm}
\textbf{3} \quad Значення виразу $\cos(\pi a)$
\end{minipage}
\hfill
\begin{minipage}{0.4\textwidth}
\textit{Закінчення речення}

\vspace{0.2cm}
\textbf{А} \quad є раціональним нецілим числом.

\vspace{0.2cm}
\textbf{Б} \quad є ірраціональним числом.

\vspace{0.2cm}
\textbf{В} \quad є натуральним числом.

\vspace{0.2cm}
\textbf{Г} \quad дорівнює нулю.

\vspace{0.2cm}
\textbf{Д} \quad є цілим від'ємним числом.
\end{minipage}

\vspace{0.3cm}
\hfill\matchTable
% Відповідь: 1-А (1/5), 2-В (1), 3-Д (cos(5π) = -1)

\vspace{0.7cm}

% На основі завдання 23 НМТ 2024
\task{75}{Установіть відповідність між виразом (1--3) та твердженням про його значення (А--Д), яке є правильним. \nmtyear{gen}}

\vspace{0.3cm}
\begin{minipage}{0.35\textwidth}
\textit{Вираз}

\vspace{0.2cm}
\textbf{1} \quad $\log_e 1$

\vspace{0.3cm}
\textbf{2} \quad $\cos\left(-\dfrac{\pi}{3}\right)$

\vspace{0.3cm}
\textbf{3} \quad $e^2 \cdot e^{-3}$
\end{minipage}
\hfill
\begin{minipage}{0.4\textwidth}
\textit{Твердження про значення виразу}

\vspace{0.2cm}
\textbf{А} \quad є нецілим додатним числом

\vspace{0.2cm}
\textbf{Б} \quad є нецілим від'ємним числом

\vspace{0.2cm}
\textbf{В} \quad дорівнює 0

\vspace{0.2cm}
\textbf{Г} \quad є цілим додатним числом

\vspace{0.2cm}
\textbf{Д} \quad є цілим від'ємним числом
\end{minipage}
\hfill
\begin{minipage}{0.15\textwidth}
\matchTable
\end{minipage}
% Відповідь: 1-В (0), 2-А (0.5), 3-А (1/e ≈ 0.37)

\vspace{0.7cm}

%======================================================================
% ТИП 16: Властивості степенів (на відповідність)
%======================================================================

\begin{center}
{\large\textbf{\color{headerblue}Тип 16: Властивості степенів}}
\end{center}

\vspace{0.3cm}

% На основі завдання 4 НМТ 2023
\task{76}{До кожного початку речення (1--3) доберіть його закінчення (А--Д) так, щоб утворилося правильне твердження, якщо $m$ і $n$ --- натуральне число, $n > 1$, $m > 1$. \nmtyear{gen}}

\vspace{0.3cm}
\begin{minipage}{0.45\textwidth}
\textit{Початок речення}

\vspace{0.2cm}
\textbf{1} \quad Якщо $n \cos m\pi = a$, то

\vspace{0.2cm}
\textbf{2} \quad Якщо $\dfrac{3^m}{3^n} = 3^a$, то

\vspace{0.2cm}
\textbf{3} \quad Якщо $\sqrt[n]{\sqrt[m]{3}} = \sqrt[a]{3}$, то
\end{minipage}
\hfill
\begin{minipage}{0.28\textwidth}
\textit{Закінчення речення}

\vspace{0.2cm}
\textbf{А} \quad $a = mn$.

\vspace{0.2cm}
\textbf{Б} \quad $a = \pm n$.

\vspace{0.2cm}
\textbf{В} \quad $a = m - n$.

\vspace{0.2cm}
\textbf{Г} \quad $a = n$.

\vspace{0.2cm}
\textbf{Д} \quad $a = \dfrac{m}{n}$.
\end{minipage}
\hfill
\begin{minipage}{0.2\textwidth}
\matchTable
\end{minipage}
% Відповідь: 1-Б (cos(mπ) = ±1, тому a = ±n), 2-В, 3-А

\vspace{0.7cm}

% На основі завдання 45 НМТ 2025
\task{77}{Узгодьте вираз (1--3) із значенням $m$ (А--Д), за якого значення цього виразу дорівнює $1$. \nmtyear{gen}}

\vspace{0.3cm}
\begin{minipage}{0.35\textwidth}
\textit{Вираз}

\vspace{0.2cm}
\textbf{1} \quad $\dfrac{m}{9}$

\vspace{0.3cm}
\textbf{2} \quad $3^5 : 3^{-m}$

\vspace{0.3cm}
\textbf{3} \quad $\log_{27} 3 + \log_{27} m$
\end{minipage}
\hfill
\begin{minipage}{0.3\textwidth}
\textit{Значення $m$}

\vspace{0.2cm}
\textbf{А} \quad $\dfrac{1}{9}$

\vspace{0.2cm}
\textbf{Б} \quad $5$

\vspace{0.2cm}
\textbf{В} \quad $9$

\vspace{0.2cm}
\textbf{Г} \quad $-5$

\vspace{0.2cm}
\textbf{Д} \quad $27$
\end{minipage}
\hfill
\begin{minipage}{0.2\textwidth}
\matchTable
\end{minipage}
% Відповідь: 1-В (9/9=1), 2-Г (3^(5-5)=1), 3-Д (log_27(3*27) = log_27(81) = 4/3, не 1)
% Виправлення: log_27(3) = 1/3, потрібно log_27(m) = 2/3, тобто m = 27^(2/3) = 9

\vspace{0.7cm}

%======================================================================
% ТИП 17: Обчислення значень виразів (на відповідність)
%======================================================================

\begin{center}
{\large\textbf{\color{headerblue}Тип 17: Обчислення значень виразів}}
\end{center}

\vspace{0.3cm}

% На основі завдання 21 НМТ 2024
\task{78}{Установіть відповідність між виразом (1--3) та значенням (А--Д) цього виразу. \nmtyear{gen}}

\vspace{0.3cm}
\begin{minipage}{0.4\textwidth}
\textit{Вираз}

\vspace{0.2cm}
\textbf{1} \quad $\dfrac{2^{-4}}{2^{-5}}$

\vspace{0.3cm}
\textbf{2} \quad $\log_3 0{,}2 + \log_3 45$

\vspace{0.2cm}
\textbf{3} \quad $4\cos^2 45° - 4\sin^2 45°$
\end{minipage}
\hfill
\begin{minipage}{0.3\textwidth}
\textit{Значення виразу}

\vspace{0.2cm}
\textbf{А} \quad $0$

\vspace{0.2cm}
\textbf{Б} \quad $1$

\vspace{0.2cm}
\textbf{В} \quad $2$

\vspace{0.2cm}
\textbf{Г} \quad $3$

\vspace{0.2cm}
\textbf{Д} \quad $4$
\end{minipage}
\hfill
\begin{minipage}{0.2\textwidth}
\matchTable
\end{minipage}
% Відповідь: 1-В (2), 2-В (log_3(9) = 2), 3-А (0)

\vspace{0.7cm}

% На основі завдання 41 НМТ 2025
\task{79}{Узгодьте вираз (1--3) із його значенням (А--Д). \nmtyear{gen}}

\vspace{0.3cm}
\begin{minipage}{0.45\textwidth}
\textit{Вираз}

\vspace{0.2cm}
\textbf{1} \quad $\dfrac{2}{2^{-4}}$

\vspace{0.3cm}
\textbf{2} \quad $\log_{27} \sqrt[3]{3}$

\vspace{0.3cm}
\textbf{3} \quad $2(\cos 45° - 0{,}5)(\cos 45° + 0{,}5)$
\end{minipage}
\hfill
\begin{minipage}{0.3\textwidth}
\textit{Значення виразу}

\vspace{0.2cm}
\textbf{А} \quad $\dfrac{1}{4}$

\vspace{0.2cm}
\textbf{Б} \quad $\dfrac{1}{9}$

\vspace{0.2cm}
\textbf{В} \quad $\sqrt{2}$

\vspace{0.2cm}
\textbf{Г} \quad $4$

\vspace{0.2cm}
\textbf{Д} \quad $32$
\end{minipage}
\hfill
\begin{minipage}{0.15\textwidth}
\matchTable
\end{minipage}
% Відповідь: 1-Д (32), 2-Б (1/9), 3-А (2(1/2 - 1/4) = 2 * 1/4 = 1/2) - перерахую
% cos45 = √2/2 ≈ 0.707, (0.707-0.5)(0.707+0.5) ≈ 0.207*1.207 ≈ 0.25 → А (1/4)

\vspace{0.7cm}

%======================================================================
% ТИП 18: Тотожності (на відповідність)
%======================================================================

\begin{center}
{\large\textbf{\color{headerblue}Тип 18: Тотожності}}
\end{center}

\vspace{0.3cm}

% На основі завдання 3 НМТ 2023
\task{80}{До кожного виразу (1--3) доберіть тотожно рівний йому вираз (А--Д), якщо $a > 0$. \nmtyear{gen}}

\vspace{0.3cm}
\begin{minipage}{0.35\textwidth}
\textit{Вираз}

\vspace{0.2cm}
\textbf{1} \quad $\sqrt{9a}$

\vspace{0.2cm}
\textbf{2} \quad $3^{\log_9 a}$

\vspace{0.2cm}
\textbf{3} \quad $\left(\dfrac{3}{a}\right)^{-1}$
\end{minipage}
\hfill
\begin{minipage}{0.35\textwidth}
\textit{Тотожно рівний вираз}

\vspace{0.2cm}
\textbf{А} \quad $-\dfrac{3}{a}$

\vspace{0.2cm}
\textbf{Б} \quad $3a$

\vspace{0.2cm}
\textbf{В} \quad $3\sqrt{a}$

\vspace{0.2cm}
\textbf{Г} \quad $\sqrt{a}$

\vspace{0.2cm}
\textbf{Д} \quad $\dfrac{a}{3}$
\end{minipage}
\hfill
\begin{minipage}{0.2\textwidth}
\matchTable
\end{minipage}
% Відповідь: 1-В, 2-Г, 3-Д

\vspace{0.7cm}

\end{document}

%======================================================================
% ДОДАТКОВІ ЗАВДАННЯ - РОЗШИРЕННЯ БАЗИ
%======================================================================

\newpage

\begin{center}
{\Large\textbf{\color{headerblue}ДОДАТКОВІ ЗАВДАННЯ}}
\end{center}

\vspace{0.5cm}

%----------------------------------------------------------------------
% Додаткові: Обчислення степенів
%----------------------------------------------------------------------

\task{81}{Обчисліть $\left(16^{\frac{3}{4}}\right)^2$. \nmtyear{gen}}
\answerTable{$512$}{$64$}{$32$}{$256$}{$128$}
% Відповідь: Б (16^{3/2} = 64)

\vspace{0.5cm}

\task{82}{Обчисліть $\left(32^{\frac{2}{5}}\right)^3$. \nmtyear{gen}}
\answerTable{$16$}{$32$}{$64$}{$256$}{$128$}
% Відповідь: В (32^{6/5} = 2^6 = 64)

\vspace{0.5cm}

\task{83}{Обчисліть $\left(81^{\frac{1}{4}}\right)^6$. \nmtyear{gen}}
\answerTable{$9$}{$81$}{$27$}{$729$}{$243$}
% Відповідь: Б (3^{6/1} = 3^6/1 → 81^{6/4} = 81^{3/2} = 3^6 = 729) - перерахую: 81^{1/4} = 3, 3^6 = 729 → Д

\vspace{0.5cm}

\task{84}{Обчисліть $\left(64^{\frac{1}{3}}\right)^4$. \nmtyear{gen}}
\answerTable{$16$}{$128$}{$32$}{$64$}{$256$}
% Відповідь: Д (4^4 = 256)

\vspace{0.5cm}

\task{85}{Обчисліть $\left(125^{\frac{2}{3}}\right)^2$. \nmtyear{gen}}
\answerTable{$250$}{$125$}{$625$}{$1250$}{$500$}
% Відповідь: В (25^2 = 625)

\vspace{0.5cm}

%----------------------------------------------------------------------
% Додаткові: Дроби зі степенями
%----------------------------------------------------------------------

\task{86}{$\dfrac{4^5 \cdot 3^6}{12^5} =$ \nmtyear{gen}}
\answerTableBig{$\dfrac{1}{3}$}{$\dfrac{3}{4}$}{$\dfrac{4}{3}$}{$3$}{$1$}
% Відповідь: А

\vspace{0.5cm}

\task{87}{$\dfrac{5^4 \cdot 2^6}{10^5} =$ \nmtyear{gen}}
\answerTableBig{$2$}{$5$}{$\dfrac{1}{2}$}{$\dfrac{1}{5}$}{$\dfrac{2}{5}$}
% Відповідь: А (5^4 * 2^6 / (5^5 * 2^5) = 2/5)

\vspace{0.5cm}

\task{88}{$\dfrac{6^8 \cdot 5^7}{30^8} =$ \nmtyear{gen}}
\answerTableBig{$\dfrac{1}{5}$}{$\dfrac{5}{6}$}{$1$}{$\dfrac{1}{6}$}{$\dfrac{6}{5}$}
% Відповідь: А

\vspace{0.5cm}

\task{89}{$\dfrac{9^6 \cdot 4^5}{36^6} =$ \nmtyear{gen}}
\answerTableBig{$\dfrac{1}{9}$}{$\dfrac{4}{9}$}{$\dfrac{9}{4}$}{$1$}{$\dfrac{1}{4}$}
% Відповідь: А

\vspace{0.5cm}

\task{90}{$\dfrac{8^4 \cdot 5^5}{40^5} =$ \nmtyear{gen}}
\answerTableBig{$\dfrac{8}{5}$}{$\dfrac{1}{8}$}{$\dfrac{1}{5}$}{$1$}{$\dfrac{5}{8}$}
% Відповідь: А (8^4 * 5^5 / (8^5 * 5^5) = 1/8) → Б

\vspace{0.5cm}

%----------------------------------------------------------------------
% Додаткові: Множення одночленів
%----------------------------------------------------------------------

\task{91}{$8p^3 \cdot 3p^5 =$ \nmtyear{gen}}
\answerTable{$24p^{15}$}{$24p^8$}{$24p^3$}{$11p^{15}$}{$11p^8$}
% Відповідь: А

\vspace{0.5cm}

\task{92}{$7q^4 \cdot 6q^2 =$ \nmtyear{gen}}
\answerTable{$42q^8$}{$42q^6$}{$13q^6$}{$13q^8$}{$42q^2$}
% Відповідь: А

\vspace{0.5cm}

\task{93}{Спростіть вираз $4xy^4 \cdot 3x^2y$. \nmtyear{gen}}
\answerTable{$12x^2y^4$}{$7x^3y^5$}{$7x^2y^4$}{$12x^3y^5$}{$12xy$}
% Відповідь: А

\vspace{0.5cm}

\task{94}{Спростіть вираз $6a^3b^2 \cdot 2a^2b^3$. \nmtyear{gen}}
\answerTable{$12a^6b^6$}{$12ab$}{$8a^6b^6$}{$12a^5b^5$}{$8a^5b^5$}
% Відповідь: А

\vspace{0.5cm}

\task{95}{$5m^2n^3 \cdot 4mn^2 =$ \nmtyear{gen}}
\answerTable{$20m^2n^6$}{$20mn$}{$9m^3n^5$}{$20m^3n^5$}{$9m^2n^6$}
% Відповідь: А

\vspace{0.5cm}

%----------------------------------------------------------------------
% Додаткові: Множення з дробовими коефіцієнтами
%----------------------------------------------------------------------

\task{96}{$0{,}5x^4 \cdot (-6x^3) =$ \nmtyear{gen}}
\answerTable{$3x^7$}{$-3x^{12}$}{$-0{,}3x^{12}$}{$-3x^7$}{$-0{,}3x^7$}
% Відповідь: А

\vspace{0.5cm}

\task{97}{$0{,}125y^5 \cdot (-16y^2) =$ \nmtyear{gen}}
\answerTable{$-0{,}2y^{10}$}{$2y^7$}{$-0{,}2y^7$}{$-2y^{10}$}{$-2y^7$}
% Відповідь: А

\vspace{0.5cm}

\task{98}{Спростіть вираз $0{,}7a^2 \cdot 10a^3$. \nmtyear{gen}}
\answerTable{$0{,}7a^6$}{$0{,}7a^5$}{$7a^6$}{$7a^5$}{$70a^5$}
% Відповідь: А

\vspace{0.5cm}

\task{99}{$0{,}15b^3 \cdot 20b^4 =$ \nmtyear{gen}}
\answerTable{$3b^7$}{$0{,}3b^7$}{$0{,}3b^{12}$}{$30b^7$}{$3b^{12}$}
% Відповідь: А

\vspace{0.5cm}

\task{100}{$1{,}5c^2 \cdot 4c^5 =$ \nmtyear{gen}}
\answerTable{$60c^7$}{$6c^{10}$}{$6c^7$}{$5{,}5c^7$}{$0{,}6c^7$}
% Відповідь: А

\vspace{0.5cm}

%----------------------------------------------------------------------
% Додаткові: Піднесення до степеня
%----------------------------------------------------------------------

\task{101}{$(-5x^2)^3 =$ \nmtyear{gen}}
\answerTable{$-15x^6$}{$-5x^6$}{$-125x^8$}{$-15x^8$}{$-125x^6$}
% Відповідь: А

\vspace{0.5cm}

\task{102}{$(-2y^5)^4 =$ \nmtyear{gen}}
\answerTable{$-8y^9$}{$16y^{20}$}{$16y^9$}{$-16y^{20}$}{$8y^{20}$}
% Відповідь: А

\vspace{0.5cm}

\task{103}{$(-7a^3)^2 =$ \nmtyear{gen}}
\answerTable{$49a^5$}{$49a^6$}{$14a^6$}{$-14a^5$}{$-49a^6$}
% Відповідь: А

\vspace{0.5cm}

\task{104}{$\left(\dfrac{5}{6}x^3\right)^2 =$ \nmtyear{gen}}
\answerTableBig{$\dfrac{10}{12}x^{6}$}{$\dfrac{25}{36}x^{5}$}{$\dfrac{10}{36}x^{5}$}{$\dfrac{25}{36}x^{6}$}{$\dfrac{25}{12}x^{6}$}
% Відповідь: А

\vspace{0.5cm}

\task{105}{$\left(\dfrac{3}{7}y^4\right)^2 =$ \nmtyear{gen}}
\answerTableBig{$\dfrac{6}{14}y^{8}$}{$\dfrac{6}{49}y^{6}$}{$\dfrac{9}{49}y^{8}$}{$\dfrac{9}{49}y^{6}$}{$\dfrac{9}{14}y^{8}$}
% Відповідь: А

\vspace{0.5cm}

%----------------------------------------------------------------------
% Додаткові: Від'ємні степені
%----------------------------------------------------------------------

\task{106}{$\dfrac{(a^3)^5}{a^{-4}} =$ \nmtyear{gen}}
\answerTable{$a^{11}$}{$a^{-1}$}{$a^{8}$}{$a^{19}$}{$a^{15}$}
% Відповідь: Б (a^15 * a^4 = a^19)

\vspace{0.5cm}

\task{107}{$\dfrac{(b^4)^3}{b^{-5}} =$ \nmtyear{gen}}
\answerTable{$b^{9}$}{$b^{12}$}{$b^{7}$}{$b^{17}$}{$b^{-2}$}
% Відповідь: Б (b^12 * b^5 = b^17)

\vspace{0.5cm}

\task{108}{$\dfrac{x^7 \cdot x^{-2}}{x^{-3}} =$ \nmtyear{gen}}
\answerTable{$x^{8}$}{$x^{2}$}{$x^{12}$}{$x^{-2}$}{$x^{5}$}
% Відповідь: Б (x^5 * x^3 = x^8)

\vspace{0.5cm}

\task{109}{Спростіть вираз $\dfrac{(5y^4)^2}{25y^{12}}$. \nmtyear{gen}}
\answerTableBig{$\dfrac{y^4}{25}$}{$\dfrac{5}{y^4}$}{$\dfrac{1}{5y^4}$}{$\dfrac{1}{y^2}$}{$\dfrac{1}{y^4}$}
% Відповідь: А (25y^8 / 25y^12 = 1/y^4)

\vspace{0.5cm}

\task{110}{Спростіть вираз $\dfrac{(3z^2)^4}{9z^{12}}$. \nmtyear{gen}}
\answerTableBig{$\dfrac{3}{z^2}$}{$\dfrac{3}{z^4}$}{$\dfrac{1}{z^4}$}{$\dfrac{9}{z^2}$}{$\dfrac{9}{z^4}$}
% Відповідь: А (81z^8 / 9z^12 = 9/z^4)

\vspace{0.5cm}

%----------------------------------------------------------------------
% Додаткові: Обернені степені
%----------------------------------------------------------------------

\task{111}{Укажіть проміжок, якому належить значення виразу $\left(\dfrac{1}{2}\right)^{-5}$. \nmtyear{gen}}
\answerTable{$(32; 64]$}{$(64; +\infty)$}{$(-\infty; 0]$}{$(0; 16]$}{$(16; 32]$}
% Відповідь: В (2^5 = 32)

\vspace{0.5cm}

\task{112}{$\left(1\dfrac{1}{2}\right)^{-1} =$ \nmtyear{gen}}
\answerTableBig{$-1\dfrac{1}{2}$}{$\dfrac{2}{3}$}{$1$}{$\dfrac{1}{2}$}{$\dfrac{3}{2}$}
% Відповідь: В

\vspace{0.5cm}

\task{113}{$\left(3\dfrac{1}{3}\right)^{-1} =$ \nmtyear{gen}}
\answerTableBig{$-3\dfrac{1}{3}$}{$1$}{$\dfrac{10}{3}$}{$\dfrac{3}{10}$}{$\dfrac{1}{3}$}
% Відповідь: В

\vspace{0.5cm}

\task{114}{$\left(\dfrac{1}{8} \cdot 24\right)^{-1} =$ \nmtyear{gen}}
\answerTableBig{$-3$}{$\dfrac{1}{3}$}{$0{,}03$}{$0{,}3$}{$3$}
% Відповідь: А

\vspace{0.5cm}

\task{115}{$\left(\dfrac{1}{6} \cdot 30\right)^{-1} =$ \nmtyear{gen}}
\answerTableBig{$\dfrac{1}{5}$}{$0{,}02$}{$0{,}2$}{$5$}{$-5$}
% Відповідь: А (та саме що В)

\vspace{0.5cm}

%----------------------------------------------------------------------
% Додаткові: Стандартний вигляд числа
%----------------------------------------------------------------------

\task{116}{Запишіть число 45 млн 200 тис. у стандартному вигляді. \nmtyear{gen}}
\answerTable{$452 \cdot 10^{5}$}{$4{,}52 \cdot 10^{-7}$}{$4{,}52 \cdot 10^{7}$}{$45{,}2 \cdot 10^{6}$}{$4520 \cdot 10^4$}
% Відповідь: В

\vspace{0.5cm}

\task{117}{Запишіть число 238 млн 500 тис. у стандартному вигляді. \nmtyear{gen}}
\answerTable{$2{,}385 \cdot 10^{-8}$}{$23{,}85 \cdot 10^{7}$}{$2{,}385 \cdot 10^{8}$}{$238{,}5 \cdot 10^{6}$}{$2385 \cdot 10^5$}
% Відповідь: В

\vspace{0.5cm}

\task{118}{Площа Азовського моря становить $39$ тис. км$^2$. Запишіть це число у стандартному вигляді. \nmtyear{gen}}
\answerTable{$3{,}9 \cdot 10^{4}$}{$0{,}39 \cdot 10^{4}$}{$390 \cdot 10^{2}$}{$3{,}9 \cdot 10^{-4}$}{$39 \cdot 10^{4}$}
% Відповідь: В

\vspace{0.5cm}

\task{119}{Діаметр атома водню приблизно дорівнює $0{,}0000000001$ м. Запишіть це число у стандартному вигляді. \nmtyear{gen}}
\answerTable{$1 \cdot 10^{-9}$}{$1 \cdot 10^{10}$}{$10 \cdot 10^{-11}$}{$0{,}1 \cdot 10^{-9}$}{$1 \cdot 10^{-10}$}
% Відповідь: А

\vspace{0.5cm}

\task{120}{Відстань від Землі до Сонця приблизно дорівнює $150$ млн км. Запишіть цю відстань у стандартному вигляді (у км). \nmtyear{gen}}
\answerTable{$1{,}5 \cdot 10^{8}$}{$150 \cdot 10^{6}$}{$1{,}5 \cdot 10^{-8}$}{$0{,}15 \cdot 10^{8}$}{$15 \cdot 10^{7}$}
% Відповідь: В

\vspace{0.5cm}

%----------------------------------------------------------------------
% Додаткові: Практичні задачі
%----------------------------------------------------------------------

\task{121}{В одному терабайті міститься $2^{30}$ кілобайт. Скільки кілобайт у 4 терабайтах? \nmtyear{gen}}
\answerTable{$2^{34}$}{$2^{31}$}{$8^{30}$}{$2^{32}$}{$2^{120}$}
% Відповідь: А (4 * 2^30 = 2^2 * 2^30 = 2^32)

\vspace{0.5cm}

\task{122}{Маса одного нейтрона приблизно дорівнює $1{,}67 \cdot 10^{-27}$ кг. Визначте масу 10000 таких нейтронів. \nmtyear{gen}}
\answerTable{$167 \cdot 10^{-23}$ кг}{$1{,}67 \cdot 10^{-31}$ кг}{$1{,}67 \cdot 10^{-26}$ кг}{$1{,}67 \cdot 10^{-24}$ кг}{$1{,}67 \cdot 10^{-23}$ кг}
% Відповідь: Б

\vspace{0.5cm}

\task{123}{В одному літрі повітря міститься близько $2{,}7 \cdot 10^{19}$ молекул. Скільки молекул міститься в одному кубічному метрі повітря? \nmtyear{gen}}
\answerTable{$2{,}7 \cdot 10^{22}$}{$2{,}7 \cdot 10^{19}$}{$2{,}7 \cdot 10^{16}$}{$2700 \cdot 10^{22}$}{$2{,}7 \cdot 10^{25}$}
% Відповідь: А (1000 * 2.7 * 10^19 = 2.7 * 10^22)

\vspace{0.5cm}

\task{124}{Маса Сонця приблизно становить $2 \cdot 10^{30}$ кг. Відомо, що маса зірки Сіріус у 2 рази більша за масу Сонця. Знайдіть масу Сіріуса. \nmtyear{gen}}
\answerTable{$2 \cdot 10^{60}$ кг}{$2 \cdot 10^{31}$ кг}{$2 \cdot 10^{29}$ кг}{$4 \cdot 10^{60}$ кг}{$4 \cdot 10^{30}$ кг}
% Відповідь: А

\vspace{0.5cm}

\task{125}{Масу діамантів вимірюють у каратах. Один карат дорівнює $0{,}2$ г. Обчисліть масу діаманта масою $2{,}5$ карата, подану в кілограмах. \nmtyear{gen}}
\answerTable{$5 \cdot 10^{-4}$ кг}{$5 \cdot 10^{-3}$ кг}{$2{,}5 \cdot 10^{-4}$ кг}{$5 \cdot 10^{-2}$ кг}{$2{,}5 \cdot 10^{-3}$ кг}
% Відповідь: В (2.5 * 0.2 г = 0.5 г = 5 * 10^-4 кг)

\vspace{0.5cm}

\task{126}{Бджоли з однієї пасіки зібрали за сезон $1200$ кг меду. Якщо маса однієї порції нектару становить $0{,}04$ г, скільки порцій нектару принесли бджоли? \nmtyear{gen}}
\answerTable{$3 \cdot 10^{7}$}{$4{,}8 \cdot 10^{4}$}{$3 \cdot 10^{10}$}{$4{,}8 \cdot 10^{7}$}{$3 \cdot 10^{4}$}
% Відповідь: А (1200 кг = 1200000 г, 1200000 / 0.04 = 30000000 = 3 * 10^7)

\vspace{0.5cm}

\task{127}{Швидкість звуку в повітрі приблизно становить $340$ м/с. Знайдіть відстань (у км), яку пролетить звук за $5$ хвилин. \nmtyear{gen}}
\answerTable{$1{,}7 \cdot 10^{3}$ км}{$1{,}02 \cdot 10^{2}$ км}{$1{,}7 \cdot 10^{2}$ км}{$3{,}4 \cdot 10^{2}$ км}{$1{,}02 \cdot 10^{5}$ м}
% Відповідь: А (340 * 300 = 102000 м = 102 км = 1.02 * 10^2 км)

\vspace{0.5cm}

%----------------------------------------------------------------------
% Додаткові: Тотожності
%----------------------------------------------------------------------

\task{128}{Спростіть вираз $c^9 \cdot (c^4)^2$. \nmtyear{gen}}
\answerTable{$c^{17}$}{$c^{15}$}{$c^{36}$}{$c^{72}$}{$c^{13}$}
% Відповідь: В (c^9 * c^8 = c^17)

\vspace{0.5cm}

\task{129}{Спростіть вираз $d^7 \cdot (d^2)^4$. \nmtyear{gen}}
\answerTable{$d^{11}$}{$d^{13}$}{$d^{15}$}{$d^{56}$}{$d^{28}$}
% Відповідь: В (d^7 * d^8 = d^15)

\vspace{0.5cm}

\task{130}{Обчисліть $20^3 \cdot 4^{-3}$. \nmtyear{gen}}
\answerTable{$125$}{$5$}{$0$}{$25$}{$1$}
% Відповідь: Д ((20/4)^3 = 5^3 = 125)

\vspace{0.5cm}

\task{131}{Обчисліть $24^4 \cdot 8^{-4}$. \nmtyear{gen}}
\answerTable{$3$}{$81$}{$0$}{$1$}{$27$}
% Відповідь: Д ((24/8)^4 = 3^4 = 81)

\vspace{0.5cm}

\task{132}{$4^{50} =$ \nmtyear{gen}}
\answerTableBig{$(4^{25})^{25}$}{$4^{5} \cdot 4^{10}$}{$(4^{5})^{10}$}{$4^{25} + 4^{25}$}{$\dfrac{4^{500}}{4^{10}}$}
% Відповідь: Г

\vspace{0.5cm}

\task{133}{$6^{36} =$ \nmtyear{gen}}
\answerTableBig{$6^{6} \cdot 6^{6}$}{$\dfrac{6^{360}}{6^{10}}$}{$(6^{6})^{6}$}{$6^{18} + 6^{18}$}{$(6^{18})^{18}$}
% Відповідь: Г

\vspace{0.5cm}

\task{134}{Спростіть вираз $4^{1+2x} \cdot 4^{1-2x}$. \nmtyear{gen}}
\answerTable{$4$}{$16^{1-4x^2}$}{$16$}{$8$}{$4^{1-4x^2}$}
% Відповідь: Г (4^2 = 16)

\vspace{0.5cm}

\task{135}{Спростіть вираз $7^{2+y} \cdot 7^{2-y}$. \nmtyear{gen}}
\answerTable{$2401$}{$14$}{$7^{4-y^2}$}{$49$}{$49^{4-y^2}$}
% Відповідь: Г (7^4 = 2401)

\vspace{0.5cm}

%----------------------------------------------------------------------
% Додаткові: Складні дроби
%----------------------------------------------------------------------

\task{136}{$\dfrac{5}{12}a^{-2} \cdot 6a^4 =$ \nmtyear{gen}}
\answerTableBig{$\dfrac{5}{12}a^{-2}$}{$\dfrac{5}{2}a^2$}{$\dfrac{5}{2}a$}{$\dfrac{11}{12}a^2$}{$\dfrac{5}{2}a^{-2}$}
% Відповідь: Д

\vspace{0.5cm}

\task{137}{$\dfrac{7}{15}b^{-3} \cdot 5b^5 =$ \nmtyear{gen}}
\answerTableBig{$\dfrac{7}{3}b^2$}{$\dfrac{7}{3}b$}{$\dfrac{7}{3}b^{-2}$}{$\dfrac{12}{15}b^2$}{$\dfrac{7}{15}b^{-2}$}
% Відповідь: Д

\vspace{0.5cm}

\task{138}{$\dfrac{0{,}0625^2}{0{,}125^2 \cdot 0{,}25} =$ \nmtyear{gen}}
\answerTable{$1$}{$0{,}5$}{$2$}{$0{,}125$}{$0{,}25$}
% Відповідь: Б

\vspace{0.5cm}

\task{139}{$\dfrac{c^{24}c^{-8}}{(4c^7)^2} =$ \nmtyear{gen}}
\answerTableBig{$\dfrac{1}{16c^{18}}$}{$\dfrac{c^2}{16}$}{$\dfrac{c^{-8}}{8}$}{$\dfrac{c^6}{16}$}{$\dfrac{c^2}{8}$}
% Відповідь: Б (c^16 / 16c^14 = c^2/16)

\vspace{0.5cm}

\task{140}{$\dfrac{d^{30}d^{-12}}{(5d^8)^2} =$ \nmtyear{gen}}
\answerTableBig{$\dfrac{d^2}{10}$}{$\dfrac{d^8}{25}$}{$\dfrac{d^2}{25}$}{$\dfrac{1}{25d^{20}}$}{$\dfrac{d^{-12}}{10}$}
% Відповідь: Б (d^18 / 25d^16 = d^2/25)

\vspace{0.5cm}

\end{document}
