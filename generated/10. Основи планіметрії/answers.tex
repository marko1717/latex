\documentclass[12pt]{extarticle}
\usepackage{fontspec}
\usepackage{polyglossia}
\setdefaultlanguage{ukrainian}

\defaultfontfeatures{Ligatures=TeX}
\setmainfont{Liberation Serif}

\usepackage[a4paper,margin=2cm]{geometry}
\usepackage{amsmath,amssymb}
\usepackage{multicol}
\usepackage{xcolor}

\definecolor{headerblue}{RGB}{0, 102, 204}

\begin{document}

\begin{center}
{\Large\textbf{\color{headerblue}ВІДПОВІДІ}}
\end{center}

\begin{center}
{\large Тема 10: Основи планіметрії}
\end{center}

\vspace{0.5cm}

\textbf{Блок 1: Відстань між точками на відрізку}

\begin{multicols}{5}
\noindent
1. Б ($15$) \\
2. А ($16$) \\
3. А ($15$) \\
4. Б ($17$) \\
5. А ($25$) \\
6. Б ($20$) \\
7. А ($12$) \\
8. А ($12$) \\
9. А ($20$) \\
10. А ($15$) \\
11. А ($28$) \\
12. А ($16$) \\
13. А ($13{,}5$) \\
14. А ($7$) \\
15. А ($26$)
\end{multicols}

\textbf{Блок 2: Кути при перетині прямих}

\begin{multicols}{5}
\noindent
16. А ($145°$) \\
17. А ($138°$) \\
18. А ($127°$) \\
19. А ($72°$) \\
20. А ($295°$) \\
21. Г ($125°$) \\
22. А ($48°$) \\
23. А ($144°$) \\
24. А \\
25. А ($113°$) \\
26. А ($117°$) \\
27. А ($30°$) \\
28. А ($360°$) \\
29. А ($4$) \\
30. А ($135°$)
\end{multicols}

\textbf{Блок 3: Кути з вершиною на прямій}

\begin{multicols}{5}
\noindent
31. А ($115°$) \\
32. А ($80°$) \\
33. А ($60°$) \\
34. А ($65°$) \\
35. А ($144°$) \\
36. А ($120°$) \\
37. А ($135°$) \\
38. А ($105°$) \\
39. А ($60°$) \\
40. А ($150°$) \\
41. А ($30°$) \\
42. А ($20°$) \\
43. А ($15°$) \\
44. А ($85°$) \\
45. А ($45°$)
\end{multicols}

\textbf{Блок 4: Паралельні прямі та січна}

\begin{multicols}{5}
\noindent
46. А ($140°$) \\
47. А ($125°$) \\
48. А ($43°$) \\
49. А ($56°$) \\
50. А ($107°$) \\
51. А ($62°$) \\
52. А ($180°$) \\
53. А ($4$) \\
54. А ($8$) \\
55. А ($720°$) \\
56. А ($107°$) \\
57. А ($66°$) \\
58. А ($\alpha$) \\
59. А ($36°$) \\
60. А ($160°$)
\end{multicols}

\textbf{Блок 5: Координатна площина та кути}

\begin{multicols}{5}
\noindent
61. А ($70°$) \\
62. А ($78°$) \\
63. А ($55°$) \\
64. А ($155°$) \\
65. А ($140°$) \\
66. А ($20°$) \\
67. А ($8°$) \\
68. А \\
69. А \\
70. А ($143°$) \\
71. А ($45°$) \\
72. А ($135°$)
\end{multicols}

\textbf{Блок 6: Кути в плані (доріжки, промені)}

\begin{multicols}{5}
\noindent
73. А ($135°$) \\
74. А ($120°$) \\
75. А ($90°$) \\
76. А ($60°$) \\
77. А ($72°$) \\
78. А ($30°$) \\
79. А ($180°$) \\
80. А ($135°$) \\
81. А \\
82. А ($45°$)
\end{multicols}

\textbf{Блок 7: Компас і напрямки}

\begin{multicols}{5}
\noindent
83. А ($110°$) \\
84. А ($60°$) \\
85. А ($135°$) \\
86. А ($135°$) \\
87. А ($110°$) \\
88. А ($90°$) \\
89. А ($180°$) \\
90. А \\
91. А ($45°$) \\
92. А
\end{multicols}

\textbf{Блок 8: Практичні задачі (кут нахилу)}

\begin{multicols}{5}
\noindent
93. А ($72°$) \\
94. А ($37°$) \\
95. А ($30°$) \\
96. А ($45°$) \\
97. А ($25°$) \\
98. А ($55°$) \\
99. А \\
100. А ($82°$)
\end{multicols}

\vspace{1cm}

\textbf{Ключові формули:}

\begin{enumerate}
\item \textbf{Суміжні кути:} $\alpha + \beta = 180°$

\item \textbf{Вертикальні кути:} $\alpha = \gamma$ (рівні)

\item \textbf{Сума кутів при перетині:} $\alpha + \beta + \gamma + \delta = 360°$

\item \textbf{Паралельні прямі і січна:}
\begin{itemize}
\item Відповідні кути рівні
\item Внутрішні різносторонні кути рівні
\item Внутрішні односторонні кути в сумі $180°$
\end{itemize}

\item \textbf{Відстань між серединами:} Якщо $M$ на $AB$, то відстань між серединами $AM$ і $MB$ дорівнює $\dfrac{AM + MB}{2} = \dfrac{AB}{2}$

\item \textbf{Пропорційний поділ:} Якщо $AC : CB = m : n$, то $AC = \dfrac{m}{n} \cdot CB$

\item \textbf{Кути з осями координат:} Якщо кут з віссю $y$ дорівнює $\alpha$, то кут з віссю $x$ дорівнює $90° - \alpha$

\item \textbf{Поділ кута на рівні частини:} Якщо $n$ променів ділять кут $\theta$ на $(n+1)$ частину, кожна частина $= \dfrac{\theta}{n+1}$
\end{enumerate}

\end{document}
