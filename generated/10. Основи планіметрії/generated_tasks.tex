\documentclass[14pt]{extarticle}
\usepackage{fontspec}
\usepackage{polyglossia}
\setdefaultlanguage{ukrainian}

\defaultfontfeatures{Ligatures=TeX}
\setmainfont{Liberation Serif}
\setsansfont{Liberation Sans}
\setmonofont{Liberation Mono}

\usepackage[a4paper,margin=2cm,bottom=2.5cm,top=2.5cm]{geometry}
\usepackage{amsmath,amssymb}
\usepackage{enumitem}
\usepackage{tikz}
\usepackage{pgfplots}
\pgfplotsset{compat=1.16}
\usetikzlibrary{calc,patterns,angles,quotes}
\usepackage{xcolor}
\usepackage{array}
\usepackage{fancyhdr}
\usepackage{multicol}

% Кольори
\definecolor{headerblue}{RGB}{0, 102, 204}
\definecolor{yearcolor}{RGB}{128, 0, 128}

\pagestyle{fancy}
\fancyhf{}
\renewcommand{\headrulewidth}{0pt}
\fancyfoot[C]{\thepage}

\setlength{\headheight}{15pt}
\setlength{\headsep}{10pt}
\setlength{\footskip}{25pt}

\widowpenalty=10000
\clubpenalty=10000

% === КОМАНДИ ===

% Стандартна таблиця відповідей
\newcommand{\answerTable}[5]{
\begin{center}
\begin{tabular}{|*{5}{>{\centering\arraybackslash}m{2.8cm}|}}
\hline
\rule[-0.3cm]{0pt}{0.8cm}\textbf{А} & \textbf{Б} & \textbf{В} & \textbf{Г} & \textbf{Д} \\
\hline
\rule[-0.4cm]{0pt}{1.0cm}#1 & \rule[-0.4cm]{0pt}{1.0cm}#2 & \rule[-0.4cm]{0pt}{1.0cm}#3 & \rule[-0.4cm]{0pt}{1.0cm}#4 & \rule[-0.4cm]{0pt}{1.0cm}#5 \\
\hline
\end{tabular}
\end{center}
}

% Маленька таблиця відповідей
\newcommand{\answerTableSmall}[5]{
\begin{tabular}{|*{5}{>{\centering\arraybackslash}m{1.1cm}|}}
\hline
\rule[-0.2cm]{0pt}{0.6cm}\textbf{А} & \textbf{Б} & \textbf{В} & \textbf{Г} & \textbf{Д} \\
\hline
\rule[-0.3cm]{0pt}{0.8cm}#1 & #2 & #3 & #4 & #5 \\
\hline
\end{tabular}
}

% Команда для завдань
\newcommand{\task}[2]{\noindent\makebox[1.5em][l]{\textbf{#1.}}\parbox[t]{\dimexpr\textwidth-1.5em}{#2}}

% Команда для згенерованих завдань
\newcommand{\gentask}[2]{\noindent\makebox[1.5em][l]{\textbf{#1.}}\parbox[t]{\dimexpr\textwidth-1.5em}{#2}}

\begin{document}

\begin{center}
{\Large\textbf{\color{headerblue}ЗГЕНЕРОВАНІ ЗАВДАННЯ}}
\end{center}

\begin{center}
{\large Тема 10: Основи планіметрії}
\end{center}

\vspace{0.5cm}

%======================================================================
% БЛОК 1: Відстань між точками на відрізку (15 завдань)
%======================================================================

\section*{Блок 1: Відстань між точками на відрізку}

\gentask{1}{На відрізку $AB$ вибрано точку $M$. Визначте відстань між серединами відрізків $AM$ і $MB$, якщо $AM = 10$ см, $MB = 2AM$.}
\answerTable{10 см}{15 см}{20 см}{25 см}{18 см}

\vspace{0.4cm}

\gentask{2}{На відрізку $PQ$ вибрано точку $K$. Визначте відстань між серединами відрізків $PK$ і $KQ$, якщо $PK = 8$ см, $KQ = 3PK$.}
\answerTable{16 см}{12 см}{20 см}{14 см}{24 см}

\vspace{0.4cm}

\gentask{3}{На відрізку $CD$ вибрано точку $N$. Визначте відстань між серединами відрізків $CN$ і $ND$, якщо $CN = 6$ см, $ND = 4CN$.}
\answerTable{15 см}{12 см}{18 см}{20 см}{10 см}

\vspace{0.4cm}

\gentask{4}{На відрізку $AB$ вибрано точку $M$. Визначте відстань від середини відрізка $AM$ до точки $B$, якщо $AB = 20$ см, $AM = 6$ см.}
\answerTable{14 см}{17 см}{12 см}{15 см}{18 см}

\vspace{0.4cm}

\gentask{5}{На відрізку $EF$ вибрано точку $G$. Визначте відстань від середини відрізка $EG$ до точки $F$, якщо $EF = 30$ см, $EG = 10$ см.}
\answerTable{25 см}{20 см}{22 см}{28 см}{24 см}

\vspace{0.4cm}

\gentask{6}{На відрізку $KL$ вибрано точку $M$. Визначте відстань від середини відрізка $KM$ до точки $L$, якщо $KL = 24$ см, $KM = 8$ см.}
\answerTable{18 см}{20 см}{16 см}{22 см}{14 см}

\vspace{0.4cm}

\gentask{7}{На відрізку $AB$ вибрано точку $C$ так, що $AC : CB = 3 : 2$. Визначте довжину відрізка $AC$, якщо $CB = 8$ см.}
\answerTable{12 см}{10 см}{14 см}{16 см}{6 см}

\vspace{0.4cm}

\gentask{8}{На відрізку $PQ$ вибрано точку $R$ так, що $PR : RQ = 4 : 3$. Визначте довжину відрізка $PR$, якщо $RQ = 9$ см.}
\answerTable{12 см}{15 см}{8 см}{10 см}{14 см}

\vspace{0.4cm}

\gentask{9}{На відрізку $MN$ вибрано точку $K$ так, що $MK : KN = 5 : 3$. Визначте довжину відрізка $MK$, якщо $KN = 12$ см.}
\answerTable{20 см}{18 см}{15 см}{24 см}{16 см}

\vspace{0.4cm}

\gentask{10}{На відрізку $AB$ вибрано точку $C$ так, що $AC : CB = 2 : 5$. Визначте довжину відрізка $CB$, якщо $AC = 6$ см.}
\answerTable{15 см}{12 см}{18 см}{10 см}{20 см}

\vspace{0.4cm}

\gentask{11}{На відрізку $XY$ вибрано точку $Z$. Відстань від середини $XZ$ до середини $ZY$ дорівнює 14 см. Знайдіть $XY$.}
\answerTable{28 см}{21 см}{14 см}{24 см}{32 см}

\vspace{0.4cm}

\gentask{12}{На відрізку $AB$ вибрано точку $M$ так, що $AM = 3MB$. Знайдіть відстань між серединами $AM$ і $MB$, якщо $AB = 32$ см.}
\answerTable{16 см}{12 см}{20 см}{24 см}{18 см}

\vspace{0.4cm}

\gentask{13}{На відрізку $CD$ вибрано точку $E$ так, що $CE = 2DE$. Знайдіть відстань між серединами $CE$ і $DE$, якщо $CD = 27$ см.}
\answerTable{13{,}5 см}{15 см}{12 см}{18 см}{9 см}

\vspace{0.4cm}

\gentask{14}{На відрізку $PQ$ вибрано точки $R$ і $S$ так, що $PR = RS = SQ$. Знайдіть $RS$, якщо $PQ = 21$ см.}
\answerTable{7 см}{6 см}{8 см}{9 см}{10 см}

\vspace{0.4cm}

\gentask{15}{На відрізку $AB$ вибрано точку $M$. Відстань від $A$ до середини $MB$ дорівнює 18 см. Знайдіть $AB$, якщо $AM = 10$ см.}
\answerTable{26 см}{28 см}{24 см}{22 см}{30 см}

\vspace{0.5cm}

%======================================================================
% БЛОК 2: Кути при перетині прямих (15 завдань)
%======================================================================

\section*{Блок 2: Кути при перетині прямих}

\gentask{16}{На рисунку зображено прямі, що перетинаються. Визначте градусну міру кута $\beta$, якщо $\alpha = 35°$.}

\vspace{0.3cm}
\begin{minipage}{0.42\textwidth}
\answerTableSmall{$145°$}{$155°$}{$125°$}{$55°$}{$135°$}
\end{minipage}
\hfill
\begin{minipage}{0.52\textwidth}
\begin{flushright}
\begin{tikzpicture}[scale=0.85]
    \coordinate (O) at (0,0);
    \coordinate (P1) at (-2,0);
    \coordinate (P2) at (2.5,0);
    \coordinate (Q1) at (215:1.8);
    \coordinate (Q2) at (35:1.8);
    \coordinate (ARight) at (1.5,0);
    \coordinate (AUp) at (35:1.2);
    \coordinate (BLeft) at (-1.5,0);
    \draw[thick] (P1) -- (P2);
    \draw[thick] (Q1) -- (Q2);
    \pic[draw, angle radius=0.8cm] {angle = ARight--O--AUp};
    \node at (1.25,0.35) {$\alpha$};
    \pic[draw, angle radius=0.4cm] {angle = AUp--O--BLeft};
    \pic[draw, angle radius=0.55cm] {angle = AUp--O--BLeft};
    \node at (-0.55,0.75) {$\beta$};
\end{tikzpicture}
\end{flushright}
\end{minipage}

\vspace{0.7cm}

\gentask{17}{На рисунку зображено прямі, що перетинаються. Визначте градусну міру кута $\beta$, якщо $\alpha = 42°$.}

\vspace{0.3cm}
\begin{minipage}{0.42\textwidth}
\answerTableSmall{$138°$}{$148°$}{$128°$}{$48°$}{$132°$}
\end{minipage}
\hfill
\begin{minipage}{0.52\textwidth}
\begin{flushright}
\begin{tikzpicture}[scale=0.85]
    \coordinate (O) at (0,0);
    \coordinate (P1) at (-2,0);
    \coordinate (P2) at (2.5,0);
    \coordinate (Q1) at (222:1.8);
    \coordinate (Q2) at (42:1.8);
    \coordinate (ARight) at (1.5,0);
    \coordinate (AUp) at (42:1.2);
    \coordinate (BLeft) at (-1.5,0);
    \draw[thick] (P1) -- (P2);
    \draw[thick] (Q1) -- (Q2);
    \pic[draw, angle radius=0.8cm] {angle = ARight--O--AUp};
    \node at (1.25,0.35) {$\alpha$};
    \pic[draw, angle radius=0.4cm] {angle = AUp--O--BLeft};
    \pic[draw, angle radius=0.55cm] {angle = AUp--O--BLeft};
    \node at (-0.55,0.75) {$\beta$};
\end{tikzpicture}
\end{flushright}
\end{minipage}

\vspace{0.7cm}

\gentask{18}{Дві прямі перетинаються. Один із кутів дорівнює $53°$. Знайдіть суміжний кут.}
\answerTable{$127°$}{$137°$}{$117°$}{$53°$}{$143°$}

\vspace{0.4cm}

\gentask{19}{Дві прямі перетинаються. Сума двох суміжних кутів дорівнює $180°$. Один із кутів на $36°$ більший за інший. Знайдіть менший кут.}
\answerTable{$72°$}{$78°$}{$68°$}{$82°$}{$90°$}

\vspace{0.4cm}

\gentask{20}{При перетині двох прямих утворилися чотири кути. Знайдіть суму трьох кутів, якщо четвертий дорівнює $65°$.}
\answerTable{$295°$}{$315°$}{$285°$}{$305°$}{$275°$}

\vspace{0.4cm}

\gentask{21}{На рисунку зображено прямі $m$ і $n$, що перетинаються. Визначте градусну міру кута $\beta$, якщо $\alpha + \beta + \gamma = 250°$.}

\vspace{0.3cm}
\begin{minipage}{0.42\textwidth}
\answerTableSmall{$140°$}{$110°$}{$55°$}{$125°$}{$130°$}
\end{minipage}
\hfill
\begin{minipage}{0.52\textwidth}
\begin{flushright}
\begin{tikzpicture}[scale=0.85]
    \coordinate (O) at (0,0);
    \draw[thick] (-2,0) -- (2.5,0);
    \node[below] at (-1.8,0) {$m$};
    \draw[thick] (-0.8,0.8) -- (1.5,-1.5);
    \node[above left] at (-0.6,0.6) {$n$};
    \coordinate (R) at (2.5,0);
    \coordinate (L) at (-2,0);
    \coordinate (U) at (-0.8,0.8);
    \coordinate (D) at (1.5,-1.5);
    \pic[draw, angle radius=0.45cm] {angle = U--O--L};
    \node at (-0.65,0.35) {\small $\alpha$};
    \pic[draw, angle radius=0.4cm] {angle = R--O--U};
    \pic[draw, angle radius=0.50cm] {angle = R--O--U};
    \node at (0.75,0.45) {\small $\beta$};
    \pic[draw, angle radius=0.35cm] {angle = D--O--R};
    \pic[draw, angle radius=0.45cm] {angle = D--O--R};
    \pic[draw, angle radius=0.55cm] {angle = D--O--R};
    \node at (0.75,-0.45) {\small $\gamma$};
\end{tikzpicture}
\end{flushright}
\end{minipage}

\vspace{0.7cm}

\gentask{22}{Два кути є вертикальними. Сума цих кутів дорівнює $96°$. Знайдіть кожен із цих кутів.}
\answerTable{$48°$}{$42°$}{$52°$}{$54°$}{$46°$}

\vspace{0.4cm}

\gentask{23}{Дві прямі перетинаються. Різниця вертикальних кутів дорівнює $0°$. Сума суміжних кутів дорівнює $180°$. Знайдіть один із кутів, якщо він у 4 рази більший за суміжний.}
\answerTable{$144°$}{$36°$}{$120°$}{$140°$}{$150°$}

\vspace{0.4cm}

\gentask{24}{При перетині двох прямих один із кутів дорівнює $\alpha$. Виразіть суму трьох інших кутів через $\alpha$.}
\answerTable{$360° - \alpha$}{$180° + \alpha$}{$270° + \alpha$}{$180° - \alpha$}{$360° - 2\alpha$}

\vspace{0.4cm}

\gentask{25}{Дві прямі перетинаються під кутом $67°$. Знайдіть тупий кут між цими прямими.}
\answerTable{$113°$}{$123°$}{$103°$}{$117°$}{$127°$}

\vspace{0.4cm}

\gentask{26}{При перетині двох прямих різниця двох суміжних кутів дорівнює $54°$. Знайдіть більший кут.}
\answerTable{$117°$}{$127°$}{$107°$}{$63°$}{$123°$}

\vspace{0.4cm}

\gentask{27}{При перетині двох прямих сума двох суміжних кутів дорівнює $180°$. Більший кут у 5 разів більший за менший. Знайдіть менший кут.}
\answerTable{$30°$}{$36°$}{$24°$}{$40°$}{$45°$}

\vspace{0.4cm}

\gentask{28}{Дві прямі перетинаються. Знайдіть суму всіх чотирьох кутів.}
\answerTable{$360°$}{$180°$}{$270°$}{$540°$}{$720°$}

\vspace{0.4cm}

\gentask{29}{При перетині двох прямих один кут дорівнює $90°$. Скільки прямих кутів утворилося?}
\answerTable{4}{2}{3}{1}{0}

\vspace{0.4cm}

\gentask{30}{Дві прямі перетинаються. Один із кутів втричі більший за інший суміжний кут. Знайдіть більший кут.}
\answerTable{$135°$}{$120°$}{$140°$}{$150°$}{$125°$}

\vspace{0.5cm}

%======================================================================
% БЛОК 3: Кути з вершиною на прямій (15 завдань)
%======================================================================

\section*{Блок 3: Кути з вершиною на прямій}

\gentask{31}{З вершини розгорнутого кута $AOB$ проведено промінь $OC$ так, що $\angle AOC = 65°$. Знайдіть $\angle BOC$.}
\answerTable{$115°$}{$125°$}{$105°$}{$65°$}{$135°$}

\vspace{0.4cm}

\gentask{32}{З вершини розгорнутого кута $AOB$ проведено два промені $OK$ і $OM$ так, що $\angle KOB = 120°$, $\angle MOB = 40°$. Обчисліть $\angle KOM$.}

\vspace{0.3cm}
\begin{minipage}{0.42\textwidth}
\answerTableSmall{$80°$}{$160°$}{$60°$}{$100°$}{$70°$}
\end{minipage}
\hfill
\begin{minipage}{0.52\textwidth}
\begin{flushright}
\begin{tikzpicture}[scale=0.85]
    \coordinate (A) at (-2.5,0);
    \coordinate (O) at (0,0);
    \coordinate (B) at (2.5,0);
    \coordinate (K) at (120:2);
    \coordinate (M) at (40:2);
    \draw[thick] (A) -- (B);
    \draw[thick] (O) -- (K);
    \draw[thick] (O) -- (M);
    \node[below] at (A) {$A$};
    \node[below] at (O) {$O$};
    \node[below] at (B) {$B$};
    \node[above] at (K) {$K$};
    \node[right] at (M) {$M$};
\end{tikzpicture}
\end{flushright}
\end{minipage}

\vspace{0.7cm}

\gentask{33}{З вершини розгорнутого кута проведено два промені, що ділять його на три рівні частини. Знайдіть кожну частину.}
\answerTable{$60°$}{$45°$}{$90°$}{$72°$}{$54°$}

\vspace{0.4cm}

\gentask{34}{Із точки $O$, яка лежить на прямій $AB$, проведено промені $OM$ і $OK$. Відомо, що $\angle BOM = 45°$, $\angle MOK = 70°$. Визначте $\angle AOK$.}

\vspace{0.3cm}
\begin{minipage}{0.42\textwidth}
\answerTableSmall{$65°$}{$115°$}{$55°$}{$75°$}{$125°$}
\end{minipage}
\hfill
\begin{minipage}{0.52\textwidth}
\begin{flushright}
\begin{tikzpicture}[scale=0.85]
    \coordinate (A) at (-2.5,0);
    \coordinate (O) at (0,0);
    \coordinate (B) at (2.5,0);
    \coordinate (K) at (115:2);
    \coordinate (M) at (45:2);
    \draw[thick] (A) -- (B);
    \draw[thick] (O) -- (K);
    \draw[thick] (O) -- (M);
    \node[below] at (A) {$A$};
    \node[below] at (O) {$O$};
    \node[below] at (B) {$B$};
    \node[above] at (K) {$K$};
    \node[right] at (M) {$M$};
\end{tikzpicture}
\end{flushright}
\end{minipage}

\vspace{0.7cm}

\gentask{35}{Кут $\alpha$ дорівнює п'ятій частині розгорнутого кута. Знайдіть кут $\beta$, що суміжний із кутом $\alpha$.}
\answerTable{$144°$}{$36°$}{$150°$}{$140°$}{$135°$}

\vspace{0.4cm}

\gentask{36}{Кут $\alpha$ дорівнює третій частині розгорнутого кута. Знайдіть кут $\beta$, що суміжний із кутом $\alpha$.}
\answerTable{$120°$}{$60°$}{$90°$}{$150°$}{$135°$}

\vspace{0.4cm}

\gentask{37}{Кут $\alpha$ дорівнює четвертій частині розгорнутого кута. Знайдіть кут $\beta$, що суміжний із кутом $\alpha$.}
\answerTable{$135°$}{$45°$}{$120°$}{$150°$}{$90°$}

\vspace{0.4cm}

\gentask{38}{Із точки $O$ на прямій $AB$ проведено промінь $OC$. Кут $AOC$ на $30°$ більший за кут $BOC$. Знайдіть $\angle AOC$.}
\answerTable{$105°$}{$75°$}{$90°$}{$120°$}{$115°$}

\vspace{0.4cm}

\gentask{39}{Із точки $O$ на прямій $AB$ проведено промінь $OC$. Кут $AOC$ у 2 рази більший за кут $BOC$. Знайдіть $\angle BOC$.}
\answerTable{$60°$}{$90°$}{$45°$}{$120°$}{$72°$}

\vspace{0.4cm}

\gentask{40}{Із точки $O$ на прямій $AB$ проведено промінь $OC$. Кут $AOC$ у 5 разів більший за кут $BOC$. Знайдіть $\angle AOC$.}
\answerTable{$150°$}{$30°$}{$120°$}{$144°$}{$135°$}

\vspace{0.4cm}

\gentask{41}{З вершини розгорнутого кута проведено 5 променів, що ділять його на 6 рівних частин. Знайдіть кут між двома сусідніми променями.}
\answerTable{$30°$}{$36°$}{$20°$}{$45°$}{$25°$}

\vspace{0.4cm}

\gentask{42}{З вершини розгорнутого кута проведено 8 променів, що ділять його на 9 рівних частин. Знайдіть кут між двома сусідніми променями.}
\answerTable{$20°$}{$18°$}{$22{,}5°$}{$15°$}{$24°$}

\vspace{0.4cm}

\gentask{43}{З вершини розгорнутого кута проведено 11 променів, що ділять його на 12 рівних частин. Знайдіть кут між двома сусідніми променями.}
\answerTable{$15°$}{$12°$}{$18°$}{$20°$}{$10°$}

\vspace{0.4cm}

\gentask{44}{Із точки $O$ на прямій $AB$ проведено промені $OC$ і $OD$ по різні сторони від прямої. Кут $AOC = 55°$, кут $BOD = 40°$. Знайдіть кут $COD$.}
\answerTable{$85°$}{$95°$}{$75°$}{$105°$}{$115°$}

\vspace{0.4cm}

\gentask{45}{Із точки $O$ на прямій $AB$ проведено промені $OC$ і $OD$ в одну півплощину. Кут $AOC = 35°$, кут $AOD = 80°$. Знайдіть кут $COD$.}
\answerTable{$45°$}{$55°$}{$35°$}{$115°$}{$65°$}

\vspace{0.5cm}

%======================================================================
% БЛОК 4: Паралельні прямі та січна (15 завдань)
%======================================================================

\section*{Блок 4: Паралельні прямі та січна}

\gentask{46}{Пряма $l$ перетинає паралельні прямі $m$ і $n$. Визначте градусну міру кута $\beta$, якщо $\alpha = 40°$.}

\vspace{0.3cm}
\begin{minipage}{0.42\textwidth}
\answerTableSmall{$140°$}{$130°$}{$150°$}{$50°$}{$120°$}
\end{minipage}
\hfill
\begin{minipage}{0.52\textwidth}
\begin{flushright}
\begin{tikzpicture}[scale=0.75]
    \coordinate (M1) at (0,-1.5);
    \coordinate (M2) at (0,1.5);
    \coordinate (N1) at (1.8,-1.5);
    \coordinate (N2) at (1.8,1.5);
    \coordinate (L1) at (-0.5,1.3);
    \coordinate (L2) at (2.3,-1.3);
    \coordinate (P) at (0,0.836);
    \coordinate (Q) at (1.8,-0.836);
    \coordinate (MDown) at (0,0);
    \coordinate (LRightP) at (0.9,0);
    \coordinate (NUp) at (1.8,0);
    \coordinate (LBelowQ) at (2.3,-1.3);
    \draw[thick] (M1) -- (M2);
    \draw[thick] (N1) -- (N2);
    \draw[thick] (L1) -- (L2);
    \node[left] at (-0.3,1.4) {$l$};
    \node[above] at (0,1.5) {$m$};
    \node[above] at (1.8,1.5) {$n$};
    \pic[draw, angle radius=0.4cm] {angle = MDown--P--LRightP};
    \node at (0.32,0.2) {\small $\alpha$};
    \pic[draw, angle radius=0.4cm] {angle = LBelowQ--Q--NUp};
    \pic[draw, angle radius=0.55cm] {angle = LBelowQ--Q--NUp};
    \node at (2.85,-0.45) {\small $\beta$};
\end{tikzpicture}
\end{flushright}
\end{minipage}

\vspace{0.7cm}

\gentask{47}{Пряма $l$ перетинає паралельні прямі $m$ і $n$. Визначте градусну міру кута $\beta$, якщо $\alpha = 55°$.}

\vspace{0.3cm}
\begin{minipage}{0.42\textwidth}
\answerTableSmall{$125°$}{$135°$}{$115°$}{$55°$}{$145°$}
\end{minipage}
\hfill
\begin{minipage}{0.52\textwidth}
\begin{flushright}
\begin{tikzpicture}[scale=0.75]
    \coordinate (M1) at (0,-1.5);
    \coordinate (M2) at (0,1.5);
    \coordinate (N1) at (1.8,-1.5);
    \coordinate (N2) at (1.8,1.5);
    \coordinate (L1) at (-0.5,1.3);
    \coordinate (L2) at (2.3,-1.3);
    \coordinate (P) at (0,0.836);
    \coordinate (Q) at (1.8,-0.836);
    \coordinate (MDown) at (0,0);
    \coordinate (LRightP) at (0.9,0);
    \coordinate (NUp) at (1.8,0);
    \coordinate (LBelowQ) at (2.3,-1.3);
    \draw[thick] (M1) -- (M2);
    \draw[thick] (N1) -- (N2);
    \draw[thick] (L1) -- (L2);
    \node[left] at (-0.3,1.4) {$l$};
    \node[above] at (0,1.5) {$m$};
    \node[above] at (1.8,1.5) {$n$};
    \pic[draw, angle radius=0.4cm] {angle = MDown--P--LRightP};
    \node at (0.32,0.2) {\small $\alpha$};
    \pic[draw, angle radius=0.4cm] {angle = LBelowQ--Q--NUp};
    \pic[draw, angle radius=0.55cm] {angle = LBelowQ--Q--NUp};
    \node at (2.85,-0.45) {\small $\beta$};
\end{tikzpicture}
\end{flushright}
\end{minipage}

\vspace{0.7cm}

\gentask{48}{Пряма $l$ перетинає паралельні прямі $m$ і $n$. Кут $\alpha + \beta = 86°$. Знайдіть кут $\alpha$.}
\answerTable{$43°$}{$47°$}{$37°$}{$53°$}{$39°$}

\vspace{0.4cm}

\gentask{49}{Пряма $l$ перетинає паралельні прямі $m$ і $n$. Кут $\alpha + \beta = 112°$. Знайдіть кут $\beta$.}
\answerTable{$56°$}{$68°$}{$52°$}{$62°$}{$58°$}

\vspace{0.4cm}

\gentask{50}{Пряма перетинає дві паралельні прямі. Один із внутрішніх односторонніх кутів дорівнює $73°$. Знайдіть другий.}
\answerTable{$107°$}{$73°$}{$117°$}{$97°$}{$127°$}

\vspace{0.4cm}

\gentask{51}{Пряма перетинає дві паралельні прямі. Один із внутрішніх різносторонніх кутів дорівнює $62°$. Знайдіть другий.}
\answerTable{$62°$}{$118°$}{$128°$}{$72°$}{$108°$}

\vspace{0.4cm}

\gentask{52}{Пряма перетинає дві паралельні прямі. Сума двох внутрішніх односторонніх кутів дорівнює:}
\answerTable{$180°$}{$90°$}{$360°$}{$270°$}{$120°$}

\vspace{0.4cm}

\gentask{53}{При перетині двох паралельних прямих січною утворюється 8 кутів. Скільки з них рівні, якщо один із кутів дорівнює $50°$?}
\answerTable{4}{2}{6}{8}{3}

\vspace{0.4cm}

\gentask{54}{Пряма перетинає дві паралельні прямі. Один із кутів дорівнює $90°$. Скільки всього прямих кутів утворилося?}
\answerTable{8}{4}{2}{6}{1}

\vspace{0.4cm}

\gentask{55}{Пряма перетинає дві паралельні прямі під кутом $\alpha$. Знайдіть суму всіх восьми утворених кутів.}
\answerTable{$720°$}{$360°$}{$540°$}{$180°$}{$1080°$}

\vspace{0.4cm}

\gentask{56}{Пряма перетинає дві паралельні прямі. Різниця двох внутрішніх односторонніх кутів дорівнює $34°$. Знайдіть більший кут.}
\answerTable{$107°$}{$97°$}{$117°$}{$73°$}{$127°$}

\vspace{0.4cm}

\gentask{57}{Пряма перетинає дві паралельні прямі. Різниця двох внутрішніх односторонніх кутів дорівнює $48°$. Знайдіть менший кут.}
\answerTable{$66°$}{$114°$}{$56°$}{$76°$}{$84°$}

\vspace{0.4cm}

\gentask{58}{Пряма перетинає паралельні прямі $a$ і $b$. Кут при прямій $a$ дорівнює $\alpha$. Який кут при прямій $b$ є йому відповідним?}
\answerTable{$\alpha$}{$180° - \alpha$}{$90° - \alpha$}{$90° + \alpha$}{$360° - \alpha$}

\vspace{0.4cm}

\gentask{59}{Січна перетинає дві паралельні прямі. Один внутрішній односторонній кут у 4 рази більший за інший. Знайдіть менший.}
\answerTable{$36°$}{$144°$}{$45°$}{$30°$}{$40°$}

\vspace{0.4cm}

\gentask{60}{Січна перетинає дві паралельні прямі. Один внутрішній односторонній кут у 8 разів більший за інший. Знайдіть більший.}
\answerTable{$160°$}{$20°$}{$150°$}{$170°$}{$140°$}

\vspace{0.5cm}

%======================================================================
% БЛОК 5: Координатна площина та кути (12 завдань)
%======================================================================

\section*{Блок 5: Координатна площина та кути}

\gentask{61}{У прямокутній системі координат відрізок $OA$ утворює з віссю $y$ кут $20°$. Знайдіть кут між $OA$ і віссю $x$.}

\vspace{0.3cm}
\begin{minipage}{0.42\textwidth}
\answerTableSmall{$70°$}{$80°$}{$110°$}{$160°$}{$90°$}
\end{minipage}
\hfill
\begin{minipage}{0.52\textwidth}
\begin{flushright}
\begin{tikzpicture}[scale=0.85]
    \draw[->] (-0.5,0) -- (2.5,0) node[right] {$x$};
    \draw[->] (0,-0.5) -- (0,2.5) node[above] {$y$};
    \node[below left] at (0,0) {$O$};
    \coordinate (O) at (0,0);
    \coordinate (A) at (70:2.2);
    \draw[thick] (O) -- (A);
    \node[above right] at (A) {$A$};
    \coordinate (Y) at (0,2.5);
    \pic[draw, angle radius=0.8cm] {angle = A--O--Y};
    \node at (0.2,1.5) {\small $20°$};
    \coordinate (X) at (2.5,0);
    \pic[draw, angle radius=0.5cm] {angle = X--O--A};
    \pic[draw, angle radius=0.65cm] {angle = X--O--A};
    \node at (0.85,0.5) {\small $?$};
\end{tikzpicture}
\end{flushright}
\end{minipage}

\vspace{0.7cm}

\gentask{62}{У прямокутній системі координат відрізок $OA$ утворює з віссю $y$ кут $12°$. Знайдіть кут між $OA$ і віссю $x$.}
\answerTable{$78°$}{$102°$}{$88°$}{$68°$}{$82°$}

\vspace{0.4cm}

\gentask{63}{У прямокутній системі координат відрізок $OA$ утворює з віссю $x$ кут $35°$. Знайдіть кут між $OA$ і віссю $y$.}
\answerTable{$55°$}{$145°$}{$125°$}{$65°$}{$35°$}

\vspace{0.4cm}

\gentask{64}{Відрізок $OA$ утворює з віссю $y$ кут $25°$. Точка $B$ лежить на від'ємній півосі $y$. Знайдіть $\angle AOB$.}

\vspace{0.3cm}
\begin{minipage}{0.42\textwidth}
\answerTableSmall{$155°$}{$165°$}{$145°$}{$115°$}{$175°$}
\end{minipage}
\hfill
\begin{minipage}{0.52\textwidth}
\begin{flushright}
\begin{tikzpicture}[scale=0.85]
    \draw[->] (-0.5,0) -- (2.5,0) node[right] {$x$};
    \draw[->] (0,-2) -- (0,2.5) node[above] {$y$};
    \node[above left] at (0,0) {$O$};
    \coordinate (O) at (0,0);
    \coordinate (A) at (65:2.5);
    \draw[thick] (O) -- (A);
    \node[above right] at (A) {$A$};
    \coordinate (B) at (0,-1.5);
    \filldraw (B) circle (1.5pt);
    \node[left] at (B) {$B$};
    \coordinate (Yup) at (0,2.5);
    \pic[draw, angle radius=0.7cm] {angle = A--O--Yup};
    \node at (0.3,1.3) {\footnotesize $25°$};
    \pic[draw, angle radius=0.4cm] {angle = B--O--A};
    \pic[draw, angle radius=0.55cm] {angle = B--O--A};
    \node at (0.8,0.25) {\small $?$};
\end{tikzpicture}
\end{flushright}
\end{minipage}

\vspace{0.7cm}

\gentask{65}{Відрізок $OA$ утворює з додатним напрямком осі $x$ кут $40°$. Точка $B$ лежить на від'ємній півосі $x$. Знайдіть $\angle AOB$.}
\answerTable{$140°$}{$130°$}{$150°$}{$120°$}{$160°$}

\vspace{0.4cm}

\gentask{66}{Кут між орбітою та віссю обертання планети дорівнює $70°$. Знайдіть кут нахилу осі до перпендикуляра орбіти.}
\answerTable{$20°$}{$70°$}{$110°$}{$160°$}{$25°$}

\vspace{0.4cm}

\gentask{67}{Кут між орбітою та віссю обертання планети дорівнює $82°$. Знайдіть кут нахилу осі до перпендикуляра орбіти.}
\answerTable{$8°$}{$82°$}{$98°$}{$172°$}{$12°$}

\vspace{0.4cm}

\gentask{68}{Вісь обертання утворює з площиною орбіти кут $\alpha$. Перпендикуляр до орбіти утворює з віссю кут:}
\answerTable{$90° - \alpha$}{$\alpha$}{$180° - \alpha$}{$90° + \alpha$}{$\alpha - 90°$}

\vspace{0.4cm}

\gentask{69}{Відрізок у першій чверті утворює з віссю $x$ кут $\alpha$. Кут з віссю $y$ дорівнює:}
\answerTable{$90° - \alpha$}{$\alpha$}{$180° - \alpha$}{$90° + \alpha$}{$\alpha - 90°$}

\vspace{0.4cm}

\gentask{70}{Відрізок $OA$ у першій чверті утворює з додатним напрямком осі $x$ кут $53°$. Знайдіть кут $AOB$, де $B$ --- точка на від'ємній півосі $y$.}
\answerTable{$143°$}{$127°$}{$137°$}{$153°$}{$133°$}

\vspace{0.4cm}

\gentask{71}{Точка $A$ лежить на бісектрисі першої координатної чверті. Знайдіть кут $AOx$.}
\answerTable{$45°$}{$90°$}{$60°$}{$30°$}{$135°$}

\vspace{0.4cm}

\gentask{72}{Точка $A$ лежить на бісектрисі другої координатної чверті. Знайдіть кут між $OA$ і додатним напрямком осі $x$.}
\answerTable{$135°$}{$45°$}{$90°$}{$120°$}{$150°$}

\vspace{0.5cm}

%======================================================================
% БЛОК 6: Кути в парку/плані (10 завдань)
%======================================================================

\section*{Блок 6: Кути в плані (доріжки, промені)}

\gentask{73}{На плані парку від точки $O$ проведено доріжки $OA$, $OB$ і $OC$ так, що $OA \perp OB$, $\angle COA = \angle COB = \alpha$. Знайдіть $\alpha$.}
\answerTable{$135°$}{$45°$}{$90°$}{$120°$}{$150°$}

\vspace{0.4cm}

\gentask{74}{Від фонтана $O$ проведено три доріжки $OA$, $OB$, $OC$. Кут $AOB = 50°$, кут $BOC = 70°$, доріжка $OB$ лежить між $OA$ і $OC$. Знайдіть $\angle AOC$.}
\answerTable{$120°$}{$20°$}{$130°$}{$110°$}{$140°$}

\vspace{0.4cm}

\gentask{75}{Від точки $O$ проведено 4 промені, що ділять площину на 4 рівні частини. Знайдіть кут між сусідніми променями.}
\answerTable{$90°$}{$45°$}{$60°$}{$120°$}{$72°$}

\vspace{0.4cm}

\gentask{76}{Від точки $O$ проведено 6 променів, що ділять площину на 6 рівних частин. Знайдіть кут між сусідніми променями.}
\answerTable{$60°$}{$30°$}{$45°$}{$72°$}{$90°$}

\vspace{0.4cm}

\gentask{77}{Від точки $O$ проведено 5 променів, що ділять площину на 5 рівних частин. Знайдіть кут між сусідніми променями.}
\answerTable{$72°$}{$60°$}{$90°$}{$45°$}{$36°$}

\vspace{0.4cm}

\gentask{78}{Від точки $O$ проведено доріжки $OA$ і $OB$ під кутом $60°$. Бісектриса $OC$ ділить цей кут. Знайдіть $\angle AOC$.}
\answerTable{$30°$}{$60°$}{$45°$}{$120°$}{$90°$}

\vspace{0.4cm}

\gentask{79}{На плані від точки $O$ проведено доріжки $OA$, $OB$, $OC$, $OD$ так, що вони ділять повний кут на 4 рівні частини. Знайдіть $\angle AOC$.}
\answerTable{$180°$}{$90°$}{$270°$}{$120°$}{$135°$}

\vspace{0.4cm}

\gentask{80}{Від точки $O$ проведено 3 промені в одну півплощину так, що вони ділять розгорнутий кут на 4 рівні частини. Знайдіть кут між крайніми променями.}
\answerTable{$135°$}{$180°$}{$90°$}{$120°$}{$150°$}

\vspace{0.4cm}

\gentask{81}{На плані від точки $O$ проведено доріжки. Кут між $OA$ і $OB$ дорівнює $\alpha$, кут між $OB$ і $OC$ дорівнює $\beta$, причому $OB$ між $OA$ і $OC$. Знайдіть кут $AOC$.}
\answerTable{$\alpha + \beta$}{$\alpha - \beta$}{$|\alpha - \beta|$}{$180° - \alpha - \beta$}{$\frac{\alpha + \beta}{2}$}

\vspace{0.4cm}

\gentask{82}{Від точки $O$ проведено 8 променів, що ділять площину на 8 рівних частин. Знайдіть кут між двома сусідніми променями.}
\answerTable{$45°$}{$30°$}{$60°$}{$40°$}{$36°$}

\vspace{0.5cm}

%======================================================================
% БЛОК 7: Компас і напрямки (10 завдань)
%======================================================================

\section*{Блок 7: Компас і напрямки}

\gentask{83}{Група туристів рухається під кутом $20°$ від напрямку «північ». На який кут потрібно повернути, щоб рухатися на «захід»?}
\answerTable{$110°$}{$70°$}{$100°$}{$90°$}{$120°$}

\vspace{0.4cm}

\gentask{84}{Група туристів рухається під кутом $30°$ від напрямку «північ». На який кут потрібно повернути, щоб рухатися на «схід»?}
\answerTable{$60°$}{$120°$}{$90°$}{$30°$}{$150°$}

\vspace{0.4cm}

\gentask{85}{Турист рухається на «північний схід» (45° від «півночі»). На який кут треба повернути для руху на «південь»?}
\answerTable{$135°$}{$45°$}{$90°$}{$180°$}{$225°$}

\vspace{0.4cm}

\gentask{86}{Корабель рухається на «південний захід» (225° від «півночі» за годинниковою стрілкою). Який кут між його напрямком і напрямком «північ»?}
\answerTable{$135°$}{$225°$}{$45°$}{$180°$}{$90°$}

\vspace{0.4cm}

\gentask{87}{Літак летить під кутом $25°$ від напрямку «північ». На який кут треба повернути для польоту на «південний схід» (135° від «півночі»)?}
\answerTable{$110°$}{$160°$}{$70°$}{$120°$}{$100°$}

\vspace{0.4cm}

\gentask{88}{Кут між напрямками «північ» і «схід» дорівнює:}
\answerTable{$90°$}{$45°$}{$180°$}{$270°$}{$135°$}

\vspace{0.4cm}

\gentask{89}{Кут між напрямками «північний схід» і «південний захід» дорівнює:}
\answerTable{$180°$}{$90°$}{$135°$}{$270°$}{$45°$}

\vspace{0.4cm}

\gentask{90}{Турист рухається під кутом $\alpha$ від напрямку «північ» (за годинниковою стрілкою). Кут повороту для руху на «південь» дорівнює:}
\answerTable{$180° - \alpha$}{$\alpha$}{$90° - \alpha$}{$180° + \alpha$}{$90° + \alpha$}

\vspace{0.4cm}

\gentask{91}{Яхта рухається на «північний захід». Кут від напрямку «північ» (проти годинникової стрілки) дорівнює:}
\answerTable{$45°$}{$315°$}{$135°$}{$225°$}{$90°$}

\vspace{0.4cm}

\gentask{92}{Автомобіль рухається на «схід». Він повернув на $45°$ за годинниковою стрілкою. У якому напрямку він тепер рухається?}
\answerTable{Південний схід}{Північний схід}{Південь}{Південний захід}{Схід}

\vspace{0.5cm}

%======================================================================
% БЛОК 8: Практичні задачі (кут нахилу) (8 завдань)
%======================================================================

\section*{Блок 8: Практичні задачі (кут нахилу)}

\gentask{93}{Драбина приставлена до стіни. Кут між драбиною і стіною дорівнює $18°$. Знайдіть кут між драбиною і підлогою.}
\answerTable{$72°$}{$108°$}{$18°$}{$162°$}{$82°$}

\vspace{0.4cm}

\gentask{94}{Смартфон приставлено до підставки. Кут між смартфоном і підставкою дорівнює $53°$. Підставка вертикальна. Знайдіть кут нахилу смартфона до горизонту.}
\answerTable{$37°$}{$53°$}{$127°$}{$143°$}{$47°$}

\vspace{0.4cm}

\gentask{95}{Сонячна панель перпендикулярна до напрямку сонячних променів. Кут падіння променів до горизонту дорівнює $60°$. Знайдіть кут між панеллю і горизонтом.}
\answerTable{$30°$}{$60°$}{$120°$}{$150°$}{$90°$}

\vspace{0.4cm}

\gentask{96}{Промені Сонця падають під кутом $45°$ до горизонту. Під яким кутом до горизонту треба встановити панель, щоб вона була перпендикулярна до променів?}
\answerTable{$45°$}{$90°$}{$135°$}{$0°$}{$60°$}

\vspace{0.4cm}

\gentask{97}{Драбина довжиною $AB$ приставлена до вертикальної стіни. Кут між драбиною та підлогою дорівнює $65°$. Знайдіть кут між драбиною та стіною.}
\answerTable{$25°$}{$115°$}{$65°$}{$155°$}{$35°$}

\vspace{0.4cm}

\gentask{98}{Дах будинку нахилений під кутом $35°$ до горизонту. Знайдіть кут між дахом і вертикаллю.}
\answerTable{$55°$}{$35°$}{$125°$}{$145°$}{$65°$}

\vspace{0.4cm}

\gentask{99}{Промені світла падають під кутом $\alpha$ до горизонтальної поверхні. Щоб дзеркало відбило їх вертикально вгору, під яким кутом до горизонту його треба встановити?}
\answerTable{$45° + \frac{\alpha}{2}$}{$\alpha$}{$90° - \alpha$}{$\frac{\alpha}{2}$}{$45°$}

\vspace{0.4cm}

\gentask{100}{Пандус для інвалідів має кут нахилу не більше $8°$. Якщо пандус нахилено під кутом $8°$ до горизонту, який кут він утворює з вертикаллю?}
\answerTable{$82°$}{$8°$}{$98°$}{$172°$}{$88°$}

\vspace{0.5cm}

\end{document}
