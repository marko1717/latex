\documentclass[14pt]{extarticle}
\usepackage{fontspec}
\usepackage{polyglossia}
\setdefaultlanguage{ukrainian}

\defaultfontfeatures{Ligatures=TeX}
\setmainfont{Liberation Serif}
\setsansfont{Liberation Sans}
\setmonofont{Liberation Mono}

\usepackage[a4paper,margin=1.5cm,bottom=2cm,top=2cm]{geometry}
\usepackage{amsmath,amssymb}
\usepackage{enumitem}
\usepackage{tikz}
\usepackage{pgfplots}
\pgfplotsset{compat=1.18}

\usetikzlibrary{calc,patterns,angles,quotes,intersections,babel}
\usetikzlibrary{3d}

\usepackage{xcolor}
\usepackage{array}
\usepackage{fancyhdr}
\usepackage{multirow}

% Кольори
\definecolor{headerblue}{RGB}{0, 102, 204}
\definecolor{yearcolor}{RGB}{128, 0, 128}
\definecolor{solutioncolor}{RGB}{200, 220, 255}

\pagestyle{fancy}
\fancyhf{}
\renewcommand{\headrulewidth}{0pt}
\fancyfoot[C]{\thepage}

\setlength{\headheight}{15pt}
\setlength{\headsep}{10pt}
\setlength{\footskip}{25pt}

\widowpenalty=10000
\clubpenalty=10000

% === КОМАНДИ ===

% Стандартна таблиця відповідей
\newcommand{\answerTable}[5]{
\begin{center}
\begin{tabular}{|*{5}{>{\centering\arraybackslash}m{2.8cm}|}}
\hline
\rule[-0.3cm]{0pt}{0.8cm}\textbf{А} & \textbf{Б} & \textbf{В} & \textbf{Г} & \textbf{Д} \\
\hline
\rule[-0.4cm]{0pt}{1.0cm}#1 & \rule[-0.4cm]{0pt}{1.0cm}#2 & \rule[-0.4cm]{0pt}{1.0cm}#3 & \rule[-0.4cm]{0pt}{1.0cm}#4 & \rule[-0.4cm]{0pt}{1.0cm}#5 \\
\hline
\end{tabular}
\end{center}
}

% Таблиця для відповідей із дробами
\newcommand{\answerTableTall}[5]{
\begin{center}
\begin{tabular}{|*{5}{>{\centering\arraybackslash}m{2.8cm}|}}
\hline
\rule[-0.3cm]{0pt}{0.8cm}\textbf{А} & \textbf{Б} & \textbf{В} & \textbf{Г} & \textbf{Д} \\
\hline
\rule[-0.9cm]{0pt}{2.0cm}#1 &
\rule[-0.9cm]{0pt}{2.0cm}#2 &
\rule[-0.9cm]{0pt}{2.0cm}#3 &
\rule[-0.9cm]{0pt}{2.0cm}#4 &
\rule[-0.9cm]{0pt}{2.0cm}#5 \\
\hline
\end{tabular}
\end{center}
}

% Таблиця відповідей для відповідностей
\newcommand{\answerGrid}{
    \begingroup
    \renewcommand{\arraystretch}{1.3}
    \setlength{\tabcolsep}{7pt}
    \begin{tabular}{r|c|c|c|c|c|}
         \multicolumn{1}{c}{} & \multicolumn{1}{c}{\textbf{А}} & \multicolumn{1}{c}{\textbf{Б}} & \multicolumn{1}{c}{\textbf{В}} & \multicolumn{1}{c}{\textbf{Г}} & \multicolumn{1}{c}{\textbf{Д}} \\ \cline{2-6}
         \textbf{1} & & & & & \\ \cline{2-6}
         \textbf{2} & & & & & \\ \cline{2-6}
         \textbf{3} & & & & & \\ \cline{2-6}
    \end{tabular}
    \endgroup
}

% Макет для завдань на відповідність
\newcommand{\matchingLayout}[3]{
    \noindent
    \begin{minipage}[t]{0.40\textwidth}
        #1
    \end{minipage}%
    \hfill
    \begin{minipage}[t]{0.28\textwidth}
        #2
    \end{minipage}%
    \hfill
    \begin{minipage}[t]{0.30\textwidth}
        \vspace{0pt}
        \begin{flushright}
        #3
        \end{flushright}
    \end{minipage}
}

% Поле для розв'язку
\newcommand{\solutionBox}[1][3cm]{
\vspace{0.3cm}
\noindent\textbf{Розв'язок:}
\vspace{0.2cm}

\noindent\fbox{\begin{minipage}[t][#1]{\dimexpr\textwidth-2\fboxsep-2\fboxrule}
\hfill\vfill
\end{minipage}}
\vspace{0.5cm}
}

% Поле для відповіді (числове завдання)
\newcommand{\answerBox}{
\vspace{0.3cm}
\noindent\textbf{Відповідь:} \framebox[4cm]{\rule{0pt}{0.8cm}}
\vspace{0.5cm}
}

% Команда для завдання
\newcommand{\task}[2]{\noindent\makebox[1.5em][l]{\textbf{#1.}}\parbox[t]{\dimexpr\textwidth-1.5em}{#2}}

\begin{document}

\begin{center}
{\LARGE\textbf{\color{headerblue}19 червня 2025 року}}
\end{center}

\begin{center}
{\large Робочий аркуш для розв'язання}
\end{center}

\vspace{0.3cm}

\hrule
\vspace{0.3cm}
\begin{center}
\textit{Завдання 1–15 мають по п'ять варіантів відповіді, з яких лише один правильний.\\
Виберіть правильний варіант відповіді й позначте його.}
\end{center}
\vspace{0.3cm}
\hrule

\vspace{0.5cm}

%======================================================================
% ЗАВДАННЯ 1
%======================================================================
\task{1}{Округліть до сотих число 1,31499.}

\answerTable{1,315}{1,32}{1,314}{1,3}{1,31}

\solutionBox[2cm]

%======================================================================
% ЗАВДАННЯ 2
%======================================================================
\begin{minipage}[t]{0.55\textwidth}
\task{2}{На рисунку зображено автомобільний номер, що складається з 4 букв і 4 цифр. Одна з цифр номеру замазана брудом і не розпізнається. Комп'ютерна програма для зчитування номерів автоматично підбирає відсутню цифру з десяти можливих значень. Яка ймовірність того, що програма правильно визначить замасковану цифру?}
\end{minipage}
\hfill
\begin{minipage}[t]{0.40\textwidth}
\vspace{-0.5cm}
\begin{center}
\begin{tikzpicture}[scale=0.8]
    % Рамка номера
    \draw[fill=white, rounded corners=3pt, line width=1pt] (0,0) rectangle (6, 1.3);

    % Синьо-жовта смуга зліва
    \fill[headerblue] (0,0) [rounded corners=3pt] -- (0,1.3) -- (0.8,1.3) -- (0.8,0) [rounded corners=3pt] -- cycle;
    \fill[yellow] (0.1, 0.7) rectangle (0.7, 0.45);
    \fill[blue!80!cyan] (0.1, 0.95) rectangle (0.7, 0.7);
    \node[white, font=\bfseries\tiny] at (0.4, 0.25) {UA};

    % Текст номера
    \node[font=\bfseries\sffamily\Large, anchor=west] at (0.9, 0.65) {AA 37};
    \node[font=\bfseries\sffamily\Large, anchor=east] at (5.8, 0.65) {0 CI};

    % Пляма (Бруд)
    \fill[gray!60] (3.6, 0.3)
        to[out=100, in=200] (3.7, 1.0)
        to[out=20, in=120] (4.2, 0.9)
        to[out=300, in=40] (4.1, 0.4)
        to[out=180, in=0] (3.6, 0.3);
\end{tikzpicture}
\end{center}
\end{minipage}

\vspace{0.3cm}
\answerTableTall{$\dfrac{1}{10}$}{$\dfrac{1}{4}$}{$\dfrac{1}{2}$}{$\dfrac{1}{8}$}{$\dfrac{1}{9}$}

\solutionBox[2cm]

%======================================================================
% ЗАВДАННЯ 3
%======================================================================
\noindent
\begin{minipage}[t]{0.55\textwidth}
\task{3}{Пряма $l$ перетинає паралельні прямі $m$ і $n$ (див. рисунок). Визначте градусну міру кута $\alpha$, якщо $\alpha + \beta = 58^\circ$.}

\vspace{0.3cm}
\answerTable{$116^\circ$}{$32^\circ$}{$24^\circ$}{$29^\circ$}{$19^\circ$}
\end{minipage}
\hfill
\begin{minipage}[t]{0.40\textwidth}
\begin{flushright}
\begin{tikzpicture}[scale=1]
    % Паралельні прямі m і n (вертикальні)
    \coordinate (M1) at (0,-1.5);
    \coordinate (M2) at (0,1.5);
    \coordinate (N1) at (1.8,-1.5);
    \coordinate (N2) at (1.8,1.5);

    % Січна l
    \coordinate (L1) at (-0.5,1.3);
    \coordinate (L2) at (2.3,-1.3);

    % Точки перетину
    \coordinate (P) at (0,0.836);
    \coordinate (Q) at (1.8,-0.836);

    % Допоміжні точки для кутів
    \coordinate (MDown) at (0,0);
    \coordinate (LRightP) at (0.9,0);
    \coordinate (NDown) at (1.8,1.4);
    \coordinate (LLeftQ) at (0.9,0);

    % Малюємо прямі
    \draw[thick] (M1) -- (M2);
    \draw[thick] (N1) -- (N2);
    \draw[thick] (L1) -- (L2);

    % Підписи прямих
    \node[left] at (-0.3,1.4) {$l$};
    \node[above] at (0,1.5) {$m$};
    \node[above] at (1.8,1.5) {$n$};

    % Дуга для кута alpha
    \pic[draw, angle radius=0.4cm] {angle = MDown--P--LRightP};
    \node at (0.25,0.1) {$\alpha$};

    % Дуга для кута beta
    \pic[draw, angle radius=0.4cm] {angle = NDown--Q--LLeftQ};
    \node at (1.45,0.08) {$\beta$};
\end{tikzpicture}
\end{flushright}
\end{minipage}

\solutionBox[3cm]

%======================================================================
% ЗАВДАННЯ 4
%======================================================================
\task{4}{Яке з наведених чисел є коренем рівняння $|x^2 - 35| = 10$?}

\answerTable{$-15$}{$45$}{$-5$}{$25$}{$7$}

\solutionBox[3cm]

%======================================================================
% ЗАВДАННЯ 5
%======================================================================
\begin{minipage}[t]{0.60\textwidth}
\task{5}{На рисунку зображено куб $ABCDA_1B_1C_1D_1$. Укажіть пряму перетину площин $(BB_1C_1)$ і $(CDD_1)$.}

\vspace{0.3cm}
\textbf{А} \quad $DD_1$

\textbf{Б} \quad $CC_1$

\textbf{В} \quad $BB_1$

\textbf{Г} \quad $BC$

\textbf{Д} \quad $AC_1$
\end{minipage}
\hfill
\begin{minipage}[t]{0.35\textwidth}
\vspace{-0.5cm}
\begin{center}
\begin{tikzpicture}[scale=0.8]
    \coordinate (A) at (0,0);
    \coordinate (D) at (2.5,0);
    \coordinate (B) at (1,1.2);
    \coordinate (C) at (3.5,1.2);
    \coordinate (A1) at (0,2.5);
    \coordinate (D1) at (2.5,2.5);
    \coordinate (B1) at (1,3.7);
    \coordinate (C1) at (3.5,3.7);

    \draw[dashed] (A) -- (B) -- (C);
    \draw[dashed] (B) -- (B1);
    \draw[thick] (A) -- (D) -- (C) -- (C1) -- (D1) -- (A1) -- cycle;
    \draw[thick] (A1) -- (B1) -- (C1);
    \draw[thick] (D) -- (D1);
    \draw[thick] (A) -- (A1);

    \node[below left] at (A) {$A$};
    \node[below right] at (D) {$D$};
    \node[above left] at (B) {$B$};
    \node[right] at (C) {$C$};
    \node[left] at (A1) {$A_1$};
    \node[right] at (D1) {$D_1$};
    \node[above left] at (B1) {$B_1$};
    \node[right] at (C1) {$C_1$};
\end{tikzpicture}
\end{center}
\end{minipage}

\solutionBox[3cm]

\newpage

%======================================================================
% ЗАВДАННЯ 6
%======================================================================
\begin{minipage}[t]{0.55\textwidth}
\task{6}{Графік якої функції зображено на рисунку?}

\vspace{0.3cm}
\begin{tabular}{ll}
    \textbf{А} & $y = -2^x$ \\[0.2cm]
    \textbf{Б} & $y = \sqrt{-x}$ \\[0.2cm]
    \textbf{В} & $y = -\sqrt{x}$ \\[0.2cm]
    \textbf{Г} & $y = \dfrac{1}{x}$ \\[0.2cm]
    \textbf{Д} & $y = \log_{\frac{1}{3}} x$ \\
\end{tabular}
\end{minipage}
\hfill
\begin{minipage}[t]{0.4\textwidth}
    \vspace{-0.5cm}
    \begin{flushright}
    \begin{tikzpicture}[scale=0.6]
        \draw[step=1cm,gray!50,very thin] (-0.5,-2.5) grid (5.5,3.5);
        \draw[->, >=stealth, thick] (-0.5,0) -- (5.5,0) node[below] {$x$};
        \draw[->, >=stealth, thick] (0,-2.5) -- (0,3.5) node[left] {$y$};

        \node[below left] at (0,0) {$0$};
        \node[below] at (1,0) {$1$};
        \node[left] at (0,1) {$1$};

        % Графік y = -sqrt(x)
        \draw[thick] plot [smooth, domain=0:5.2] (\x, {-sqrt(\x)});

        \fill (1,-1) circle (2pt);
    \end{tikzpicture}
    \end{flushright}
\end{minipage}

\solutionBox[3cm]

%======================================================================
% ЗАВДАННЯ 7
%======================================================================
\task{7}{Які з наведених тверджень є правильними?

\begin{enumerate}[label=\Roman*., itemsep=0pt]
\item Існує трапеція, точка перетину діагоналей якої рівновіддалена від її вершин.
\item Існує трапеція, сума довжин бічних сторін якої дорівнює сумі довжин її основ.
\item Існує трапеція, середня лінія якої проходить через точку перетину її діагоналей.
\end{enumerate}}

\answerTable{лише I}{лише II}{лише III}{лише I та II}{лише II та III}

\solutionBox[4cm]

%======================================================================
% ЗАВДАННЯ 8
%======================================================================
\task{8}{$\dfrac{9 - y^2}{3 - y} - 3 =$}

\answerTableTall{$y$}{$\dfrac{-y^2 - y}{3 - y}$}{$1 - y$}{$-y$}{$\dfrac{-y^2 - 3y}{3 - y}$}

\solutionBox[4cm]

%======================================================================
% ЗАВДАННЯ 9
%======================================================================
\task{9}{Визначте максимальну швидкість (у м/с) фантастичного зорельота, якщо вона становить 0,01 від швидкості світла у вакуумі. Уважайте, що швидкість світла у вакуумі дорівнює $3 \cdot 10^8$ м/с.}

\answerTable{$3 \cdot 10^{10}$}{$3 \cdot 10^4$}{$3 \cdot 0{,}1^8$}{$3 \cdot 10^{0{,}08}$}{$3 \cdot 10^6$}

\solutionBox[3cm]

\newpage

%======================================================================
% ЗАВДАННЯ 10
%======================================================================
\task{10}{У прямокутній системі координат у просторі задано точки $A(-6; -3; 6)$ і $B$, що симетричні одна одній відносно початку координат. Визначте довжину (модуль) вектора $\overrightarrow{AB}$.}

\answerTable{$6\sqrt{3}$}{$3$}{$4{,}5$}{$18$}{$9$}

\solutionBox[4cm]

%======================================================================
% ЗАВДАННЯ 11
%======================================================================
\task{11}{Якому проміжку належить корінь рівняння $\log_6 x - \log_6 2 = 1$?}

\answerTable{$(-\infty; 0]$}{$(0; 1]$}{$(1; 4]$}{$(4; 10]$}{$(10; +\infty)$}

\solutionBox[4cm]

%======================================================================
% ЗАВДАННЯ 12
%======================================================================
\task{12}{На стороні $BC$ прямокутника $ABCD$ вибрано точку $K$ так, що $\angle KAB = 30^\circ$. Визначте довжину відрізка $AK$, якщо периметр прямокутника дорівнює 96 см, $AB : BC = 3 : 5$.}

\answerTable{$36\sqrt{3}$ см}{$12$ см}{$18$ см}{$12\sqrt{3}$ см}{$36$ см}

\solutionBox[5cm]

\newpage

%======================================================================
% ЗАВДАННЯ 13
%======================================================================
\task{13}{Знайдіть суму чотирьох перших членів геометричної прогресії $(b_n)$, у якої $b_2 = 6$, а знаменник $q = -2$.}

\answerTable{$15$}{$-7{,}5$}{$-9$}{$-45$}{$-15$}

\solutionBox[4cm]

%======================================================================
% ЗАВДАННЯ 14
%======================================================================
\task{14}{Обчисліть $2\sin^2 x - 2$, якщо $\cos^2 x = 0{,}4$.}

\answerTable{$-0{,}8$}{$-1{,}2$}{$0{,}8$}{$-3{,}2$}{$1{,}2$}

\solutionBox[4cm]

%======================================================================
% ЗАВДАННЯ 15
%======================================================================
\task{15}{Обчисліть суму всіх цілих розв'язків системи нерівностей $\begin{cases} 3x - 5 < 2x, \\ 12 - 9x \leqslant 3x. \end{cases}$}

\answerTable{$14$}{$10$}{$9$}{$15$}{$7$}

\solutionBox[5cm]

\newpage

%======================================================================
\hrule
\vspace{0.3cm}
\begin{center}
\textit{У завданнях 16–18 до кожного з трьох рядків інформації, позначених цифрами,\\
доберіть один правильний, на Вашу думку, варіант, позначений буквою.}
\end{center}
\vspace{0.3cm}
\hrule
\vspace{0.5cm}

%======================================================================
% ЗАВДАННЯ 16
%======================================================================
\begin{minipage}[t]{0.55\textwidth}
\noindent\textbf{16.} На рисунку зображено ромб $ABCD$. Точки $K$ і $M$ --- середини сторін $AD$ і $DC$ відповідно, діагональ $BD$ перетинає відрізок $KM$ у точці $O$. $AC = 32$ см, $DO = 6$ см. Узгодьте відрізок (1--3) та його довжину (А--Д).
\end{minipage}
\hfill
\begin{minipage}[t]{0.4\textwidth}
    \vspace{-0.5cm}
    \begin{flushright}
    \begin{tikzpicture}[scale=0.25]
        \coordinate (A) at (-10, 0);
        \coordinate (C) at (10, 0);
        \coordinate (B) at (0, 12);
        \coordinate (D) at (0, -12);

        \coordinate (K) at ($(A)!0.5!(D)$);
        \coordinate (M) at ($(D)!0.5!(C)$);
        \coordinate (O) at (0, -6);

        \draw[thick] (A) -- (B) -- (C) -- (D) -- cycle;
        \draw[thick] (K) -- (M);
        \draw[thick] (B) -- (D);

        % Позначки середин
        \draw[thick] ($(A)!0.5!(K)$) ++(60:1) -- ++(-120:2);
        \draw[thick] ($(K)!0.5!(D)$) ++(60:1) -- ++(-120:2);
        \draw[thick] ($(D)!0.5!(M)$) ++(-60:1) -- ++(120:2);
        \draw[thick] ($(M)!0.5!(C)$) ++(-60:1) -- ++(120:2);

        \node[left] at (A) {$A$};
        \node[above] at (B) {$B$};
        \node[right] at (C) {$C$};
        \node[below] at (D) {$D$};
        \node[below left] at (K) {$K$};
        \node[below right] at (M) {$M$};
        \node[above right] at (O) {$O$};

        \fill (K) circle (8pt);
        \fill (M) circle (8pt);
        \fill (O) circle (8pt);
    \end{tikzpicture}
    \end{flushright}
\end{minipage}

\vspace{0.3cm}

\matchingLayout{
    \textit{Відрізок} \par \vspace{0.2cm}
    \textbf{1} \quad $KM$ \\
    \textbf{2} \quad $BO$ \\
    \textbf{3} \quad $AB$
}{
    \textit{Довжина відрізка} \par \vspace{0.2cm}
    \begin{tabular}{ll}
    \textbf{А} & 12 см \\
    \textbf{Б} & 16 см \\
    \textbf{В} & 18 см \\
    \textbf{Г} & 20 см \\
    \textbf{Д} & 24 см \\
    \end{tabular}
}{
    \answerGrid
}

\solutionBox[6cm]

\newpage

%======================================================================
% ЗАВДАННЯ 17
%======================================================================
\noindent\textbf{17.} На рисунку зображено графік функції $y=f(x)$, визначеної на проміжку $[-4; 5]$. Узгодьте проміжок (1–3) з властивістю (А–Д), яку має функція на цьому проміжку.

\vspace{0.2cm}

\noindent
\begin{minipage}[t]{0.55\textwidth}
    \vspace{0pt}
    \textit{Проміжок} \par \vspace{0.2cm}
    \begin{tabular}{@{}p{0.5cm} p{3cm}@{}}
    \textbf{1} & $[-4; -2]$ \\[0.3cm]
    \textbf{2} & $[-2; 2]$ \\[0.3cm]
    \textbf{3} & $[2; 5]$ \\
    \end{tabular}

    \vspace{0.3cm}
    \textit{Властивість функції} \par \vspace{0.2cm}
    \begin{tabular}{@{}p{0.5cm} p{7cm}@{}}
    \textbf{А} & функція спадає \\[0.2cm]
    \textbf{Б} & функція має точку максимуму \\[0.2cm]
    \textbf{В} & графік функції перетинає пряму $y = -4{,}5$ \\[0.2cm]
    \textbf{Г} & для кожної точки $(x_0; y_0)$, що належить графіку функції, добуток $x_0 \cdot y_0 < 0$ \\[0.2cm]
    \textbf{Д} & функція набуває лише додатних значень \\
    \end{tabular}

    \vspace{0.3cm}

    \begingroup
    \setlength{\tabcolsep}{4pt}
    \renewcommand{\arraystretch}{1.2}
    \small
    \begin{tabular}{r|c|c|c|c|c|}
         \multicolumn{1}{c}{} & \multicolumn{1}{c}{\textbf{А}} & \multicolumn{1}{c}{\textbf{Б}} & \multicolumn{1}{c}{\textbf{В}} & \multicolumn{1}{c}{\textbf{Г}} & \multicolumn{1}{c}{\textbf{Д}} \\ \cline{2-6}
         \textbf{1} & & & & & \\ \cline{2-6}
         \textbf{2} & & & & & \\ \cline{2-6}
         \textbf{3} & & & & & \\ \cline{2-6}
    \end{tabular}
    \endgroup
\end{minipage}%
\hfill
\begin{minipage}[t]{0.40\textwidth}
    \vspace{0pt}
    \begin{flushright}
    \begin{tikzpicture}[scale=0.5]
        \draw[step=1cm,gray!50,very thin] (-6.5,-7.5) grid (5.5,4.5);
        \draw[->, >=stealth, thick] (-6.5,0) -- (5.5,0) node[below] {$x$};
        \draw[->, >=stealth, thick] (0,-7.5) -- (0,4.5) node[left] {$y$};

        \node[below left] at (0,0) {$0$};
        \node[above] at (1,0) {$1$};
        \node[above left] at (0,1) {$1$};
        \node[below] at (5,0) {$5$};
        \node[below] at (-4,0) {$-4$};

        % Графік (хвиля)
        \draw[thick] plot [smooth, tension=0.44] coordinates {(-4, -6) (-3, -5) (-2, -3)(-1, 0) (0, 1)(1, 0) (2, -3) (3, -4) (5, -1)};
        \fill (-4,-6) circle (3pt);
        \fill (5, -1) circle (3pt);
        \node[below] at (3, -3.5) {\small $y=f(x)$};
    \end{tikzpicture}
    \end{flushright}
\end{minipage}

\solutionBox[6cm]

\newpage

%======================================================================
% ЗАВДАННЯ 18
%======================================================================
\noindent\textbf{18.} Доберіть до кожного початку речення (1–3) його закінчення (А–Д) так, щоб утворилося правильне твердження, якщо $n \neq 0$.

\vspace{0.5cm}

\matchingLayout{
\textit{Початок речення}

\textbf{1} \quad Якщо $(m + 1)^2 - 1 = n$, то

\vspace{0.4cm}

\textbf{2} \quad Якщо $\sqrt{2^m} = 2^n$, то

\vspace{0.4cm}

\textbf{3} \quad Якщо $\log_{2^n} 4^m = 1$, то

\vspace{0.4cm}

}{
\textit{Закінчення речення}

\textbf{А} \quad $n = 2m$.

\vspace{0.4cm}

\textbf{Б} \quad $n = m^2$.

\vspace{0.4cm}

\textbf{В} \quad $n = m^2 + 2m$.

\vspace{0.4cm}

\textbf{Г} \quad $n = \sqrt{m}$.

\vspace{0.4cm}

\textbf{Д} \quad $n = \dfrac{m}{2}$.

\vspace{0.4cm}

}{\answerGrid}

\solutionBox[6cm]

\newpage

%======================================================================
\hrule
\vspace{0.3cm}
\begin{center}
\textit{Розв'яжіть завдання 19–22. Одержані числові відповіді запишіть у спеціально\\
відведеному місці. Відповідь записуйте лише десятковим дробом, урахувавши\\
положення коми. Знак «мінус» записуйте перед першою цифрою числа.}
\end{center}
\vspace{0.3cm}
\hrule
\vspace{0.5cm}

%======================================================================
% ЗАВДАННЯ 19
%======================================================================
\task{19}{Визначте суму всіх цілих значень $a$ з проміжку $[-8; 5]$, за кожного з яких рівняння $\dfrac{5^{4x+2a-6} - 1}{\sqrt{x - 2}} = 0$ має корінь.}

\solutionBox[7cm]

\answerBox

%======================================================================
% ЗАВДАННЯ 20
%======================================================================
\task{20}{Задано функцію $f(x) = x(5 - x)$, $f'(x)$ — її похідна. Обчисліть значення виразу $f(-2) \cdot f'(4)$.}

\solutionBox[6cm]

\answerBox

\newpage

%======================================================================
% ЗАВДАННЯ 21
%======================================================================
\noindent\textbf{21.} Студентів і студенток опитали щодо улюбленого жанру кінофільмів. Результати опитування відображено на круговій діаграмі. Жанр «Фантастика» вибрало в 1,5 раза більше осіб, ніж жанр «Хоррор». Скільки всього було опитаних, якщо жанр «Фантастика» вибрало на 30 осіб більше, ніж жанр «Хоррор»?

\vspace{0.3cm}
\begin{center}
\begin{tikzpicture}[scale=1.2]
\def\radius{2.5}
% Комедія 40%
\filldraw[fill=orange!70, draw=black] (0,0) -- (0:\radius) arc (0:144:\radius) -- cycle;
\node at (72:1.7) {\small Комедія};
\node at (72:1.2) {\small 40\,\%};
% Мелодрама 35%
\filldraw[fill=pink!70, draw=black] (0,0) -- (144:\radius) arc (144:270:\radius) -- cycle;
\node at (207:1.7) {\small Мелодрама};
\node at (207:1.2) {\small 35\,\%};
% Фантастика 15%
\filldraw[fill=green!70, draw=black] (0,0) -- (270:\radius) arc (270:324:\radius) -- cycle;
\node at (297:1.8) {\small Фантастика};
% Хоррор 10%
\filldraw[fill=yellow!70, draw=black] (0,0) -- (324:\radius) arc (324:360:\radius) -- cycle;
\node at (342:1.9) {\small Хоррор};
\end{tikzpicture}
\end{center}

\solutionBox[7cm]

\answerBox

%======================================================================
% ЗАВДАННЯ 22
%======================================================================
\task{22}{Конус і правильна трикутна піраміда мають рівні висоти. Радіус кола, описаного навколо основи піраміди, дорівнює радіусу основи конуса. Обчисліть об'єм (у см$^3$) піраміди, якщо медіана основи піраміди дорівнює 9 см, а твірна конуса — 12 см.}

\solutionBox[8cm]

\answerBox

\end{document}
