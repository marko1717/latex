\documentclass[14pt]{extarticle}
\usepackage{fontspec}
\usepackage{polyglossia}
\setdefaultlanguage{ukrainian}

\defaultfontfeatures{Ligatures=TeX}
\setmainfont{Liberation Serif}
\setsansfont{Liberation Sans}
\setmonofont{Liberation Mono}

\usepackage[a4paper,margin=1.5cm,bottom=2cm,top=2.5cm]{geometry}
\usepackage{amsmath,amssymb}
\usepackage{enumitem}
\usepackage{tikz}
\usepackage{pgfplots}
\pgfplotsset{compat=1.18}
\usepackage{hyperref}

\usetikzlibrary{calc,patterns,angles,quotes,intersections,babel,3d}

\usepackage{xcolor}
\usepackage{array}
\usepackage{fancyhdr}
\usepackage{multirow}

% Кольори
\definecolor{headerblue}{RGB}{0, 102, 204}
\definecolor{yearcolor}{RGB}{128, 0, 128}
\definecolor{gridcolor}{RGB}{210, 210, 210}

% === КОЛОНТИТУЛИ ===
\pagestyle{fancy}
\fancyhf{}
\fancyhead[L]{\textcolor{headerblue}{\textbf{@pvtr2525}} | Підготовка до НМТ}
\fancyhead[R]{Математика 2026}
\fancyfoot[C]{\thepage}
\renewcommand{\headrulewidth}{0.5pt}

\setlength{\headheight}{18pt}
\setlength{\footskip}{25pt}

% === КОМАНДИ ===
\newcommand{\nmtyear}[1]{\hfill \textcolor{yearcolor}{\small #1}}

\newcommand{\solutionGrid}[1][2cm]{

\vspace{0.2cm}
\noindent
\begin{tikzpicture}
    \draw[step=0.5cm, gridcolor, thin] (0,0) grid (\textwidth-0.1cm, #1);
    \draw[black, thick] (0,0) rectangle (\textwidth-0.1cm, #1);
\end{tikzpicture}
\vspace{0.4cm}
}

\newcommand{\answerTable}[5]{
\begin{center}
\begin{tabular}{|*{5}{>{\centering\arraybackslash}m{2.8cm}|}}
\hline
\rule[-0.3cm]{0pt}{0.8cm}\textbf{А} & \textbf{Б} & \textbf{В} & \textbf{Г} & \textbf{Д} \\
\hline
\rule[-0.4cm]{0pt}{1.0cm}#1 & \rule[-0.4cm]{0pt}{1.0cm}#2 & \rule[-0.4cm]{0pt}{1.0cm}#3 & \rule[-0.4cm]{0pt}{1.0cm}#4 & \rule[-0.4cm]{0pt}{1.0cm}#5 \\
\hline
\end{tabular}
\end{center}
}

\newcommand{\answerTableTall}[5]{
\begin{center}
\begin{tabular}{|*{5}{>{\centering\arraybackslash}m{2.8cm}|}}
\hline
\rule[-0.3cm]{0pt}{0.8cm}\textbf{А} & \textbf{Б} & \textbf{В} & \textbf{Г} & \textbf{Д} \\
\hline
\rule[-0.9cm]{0pt}{2.0cm}#1 & \rule[-0.9cm]{0pt}{2.0cm}#2 & \rule[-0.9cm]{0pt}{2.0cm}#3 & \rule[-0.9cm]{0pt}{2.0cm}#4 & \rule[-0.9cm]{0pt}{2.0cm}#5 \\
\hline
\end{tabular}
\end{center}
}

\newcommand{\answerGrid}{
    \begingroup
    \renewcommand{\arraystretch}{1.3}
    \setlength{\tabcolsep}{7pt}
    \begin{tabular}{r|c|c|c|c|c|}
         \multicolumn{1}{c}{} & \multicolumn{1}{c}{\textbf{А}} & \multicolumn{1}{c}{\textbf{Б}} & \multicolumn{1}{c}{\textbf{В}} & \multicolumn{1}{c}{\textbf{Г}} & \multicolumn{1}{c}{\textbf{Д}} \\ \cline{2-6}
         \textbf{1} & & & & & \\ \cline{2-6}
         \textbf{2} & & & & & \\ \cline{2-6}
         \textbf{3} & & & & & \\ \cline{2-6}
    \end{tabular}
    \endgroup
}

\newcommand{\matchingLayout}[3]{
    \noindent
    \begin{minipage}[t]{0.40\textwidth} #1 \end{minipage}%
    \hfill
    \begin{minipage}[t]{0.28\textwidth} #2 \end{minipage}%
    \hfill
    \begin{minipage}[t]{0.30\textwidth} \vspace{0pt} \begin{flushright} #3 \end{flushright} \end{minipage}
}

\newcommand{\answerBox}{
\vspace{0.3cm}
\noindent\textbf{Відповідь:} \framebox[4cm]{\rule{0pt}{0.8cm}}
\vspace{0.5cm}
}

\newcommand{\task}[2]{\noindent\makebox[1.5em][l]{\textbf{#1.}}\parbox[t]{\dimexpr\textwidth-1.5em}{#2}}

\begin{document}

\begin{center}
{\LARGE\textbf{\color{headerblue}19 червня 2025 року}} \\
{\large Робочий аркуш для розв'язання}
\end{center}

\hrule \vspace{0.3cm}
\begin{center} \textit{Завдання 1–15 мають лише один правильний варіант відповіді.} \end{center}
\hrule \vspace{0.5cm}

\task{1}{Округліть до сотих число 1,31499.}
\answerTable{1,315}{1,32}{1,314}{1,3}{1,31}
\solutionGrid[2cm]

\begin{minipage}[t]{0.55\textwidth}
\task{2}{На рисунку зображено автомобільний номер... Яка ймовірність того, що програма правильно визначить замасковану цифру?}
\end{minipage}
\hfill
\begin{minipage}[t]{0.40\textwidth}
\vspace{-0.5cm}
\begin{center}
\begin{tikzpicture}[scale=0.8]
    \draw[fill=white, rounded corners=3pt, line width=1pt] (0,0) rectangle (6, 1.3);
    \fill[headerblue] (0,0) [rounded corners=3pt] -- (0,1.3) -- (0.8,1.3) -- (0.8,0) [rounded corners=3pt] -- cycle;
    \fill[yellow] (0.1, 0.7) rectangle (0.7, 0.45);
    \fill[blue!80!cyan] (0.1, 0.95) rectangle (0.7, 0.7);
    \node[white, font=\bfseries\tiny] at (0.4, 0.25) {UA};
    \node[font=\bfseries\sffamily\Large, anchor=west] at (0.9, 0.65) {AA 37};
    \node[font=\bfseries\sffamily\Large, anchor=east] at (5.8, 0.65) {0 CI};
    \fill[gray!60] (3.6, 0.3) to[out=100, in=200] (3.7, 1.0) to[out=20, in=120] (4, 0.9) to[out=300, in=40] (4.1, 0.4) to[out=180, in=0] (3.6, 0.3);
\end{tikzpicture}
\end{center}
\end{minipage}
\answerTableTall{$\dfrac{1}{10}$}{$\dfrac{1}{4}$}{$\dfrac{1}{2}$}{$\dfrac{1}{8}$}{$\dfrac{1}{9}$}
\solutionGrid[2cm]

\task{3}{Пряма $l$ перетинає паралельні прямі $m$ і $n$ (див. рисунок). Визначте градусну міру кута $\alpha$, якщо $\alpha + \beta = 58^\circ$.}
\begin{minipage}[t]{0.55\textwidth}
\vspace{0.3cm}
\answerTable{$116^\circ$}{$32^\circ$}{$24^\circ$}{$29^\circ$}{$19^\circ$}
\end{minipage}
\hfill
\begin{minipage}[t]{0.40\textwidth}
\begin{flushright}
\begin{tikzpicture}[scale=1]
    \coordinate (M1) at (0,-1.5); \coordinate (M2) at (0,1.5);
    \coordinate (N1) at (1.8,-1.5); \coordinate (N2) at (1.8,1.5);
    \coordinate (L1) at (-0.5,1.3); \coordinate (L2) at (2.3,-1.3);
    \coordinate (P) at (0,0.836); \coordinate (Q) at (1.8,-0.836);
    \coordinate (MDown) at (0,0); \coordinate (LRightP) at (0.9,0);
    \coordinate (NDown) at (1.8,1.4); \coordinate (LLeftQ) at (0.9,0);
    \draw[thick] (M1) -- (M2); \draw[thick] (N1) -- (N2); \draw[thick] (L1) -- (L2);
    \node[left] at (-0.3,1.4) {$l$}; \node[above] at (0,1.5) {$m$}; \node[above] at (1.8,1.5) {$n$};
    \pic[draw, angle radius=0.4cm] {angle = MDown--P--LRightP}; \node at (0.25,0.1) {$\alpha$};
    \pic[draw, angle radius=0.4cm] {angle = NDown--Q--LLeftQ}; \node at (1.45,0.08) {$\beta$};
    \pic[draw, angle radius=0.6cm] {angle = NDown--Q--LLeftQ}; \node at (1.45,0.08) {$\beta$};
\end{tikzpicture}
\end{flushright}
\end{minipage}
\solutionGrid[3cm]

\task{4}{Яке з наведених чисел є коренем рівняння $|x^2 - 35| = 10$?}
\answerTable{$-15$}{$45$}{$-5$}{$25$}{$7$}
\solutionGrid[3cm]

%======================================================================
% ЗАВДАННЯ 5
%======================================================================
\noindent
\begin{minipage}[t]{0.60\textwidth}
\task{5}{На рисунку зображено куб $ABCDA_1B_1C_1D_1$. Укажіть пряму перетину площин $(BB_1C_1)$ і $(CDD_1)$.}

\vspace{0.3cm}
\textbf{А} \quad $DD_1$ 

\vspace{0.2cm}
\textbf{Б} \quad $CC_1$ 

\vspace{0.2cm}
\textbf{В} \quad $BB_1$ 

\vspace{0.2cm}
\textbf{Г} \quad $BC$ 

\vspace{0.2cm}
\textbf{Д} \quad $AC_1$
\end{minipage}
\hfill
\begin{minipage}[t]{0.35\textwidth}
\vspace{-0.5cm}
\begin{center}
\begin{tikzpicture}[scale=0.8]
    \coordinate (A) at (0,0);
    \coordinate (D) at (2.5,0);
    \coordinate (B) at (1,1.2);
    \coordinate (C) at (3.5,1.2);
    \coordinate (A1) at (0,2.5);
    \coordinate (D1) at (2.5,2.5);
    \coordinate (B1) at (1,3.7);
    \coordinate (C1) at (3.5,3.7);

    \draw[dashed] (A) -- (B) -- (C);
    \draw[dashed] (B) -- (B1);
    \draw[thick] (A) -- (D) -- (C) -- (C1) -- (D1) -- (A1) -- cycle;
    \draw[thick] (A1) -- (B1) -- (C1);
    \draw[thick] (D) -- (D1);
    \draw[thick] (A) -- (A1);

    \node[below left] at (A) {$A$};
    \node[below right] at (D) {$D$};
    \node[above left] at (B) {$B$};
    \node[right] at (C) {$C$};
    \node[left] at (A1) {$A_1$};
    \node[right] at (D1) {$D_1$};
    \node[above left] at (B1) {$B_1$};
    \node[right] at (C1) {$C_1$};
\end{tikzpicture}
\end{center}
\end{minipage}

\solutionGrid[3cm]

\newpage

\begin{minipage}[t]{0.55\textwidth}
\task{6}{Графік якої функції зображено на рисунку?}
\vspace{0.3cm}
\begin{tabular}{ll}
    \textbf{А} & $y = -2^x$ \\[0.2cm]
    \textbf{Б} & $y = \sqrt{-x}$ \\[0.2cm]
    \textbf{В} & $y = -\sqrt{x}$ \\[0.2cm]
    \textbf{Г} & $y = \dfrac{1}{x}$ \\[0.2cm]
    \textbf{Д} & $y = \log_{\frac{1}{3}} x$ \\
\end{tabular}
\end{minipage}
\hfill
\begin{minipage}[t]{0.4\textwidth}
    \vspace{-0.5cm}
    \begin{flushright}
    \begin{tikzpicture}[scale=0.6]
        \draw[step=1cm,gray!50,very thin] (-0.5,-2.5) grid (5.5,3.5);
        \draw[->, >=stealth, thick] (-0.5,0) -- (5.5,0) node[below] {$x$};
        \draw[->, >=stealth, thick] (0,-2.5) -- (0,3.5) node[left] {$y$};
        \node[below left] at (0,0) {$0$}; \node[below] at (1,0) {$1$}; \node[left] at (0,1) {$1$};
        \draw[thick] plot [smooth, domain=0:5.2] (\x, {-sqrt(\x)});
        \fill (1,-1) circle (2pt);
    \end{tikzpicture}
    \end{flushright}
\end{minipage}
\solutionGrid[3cm]

\task{7}{Які з наведених тверджень є правильними?
\begin{enumerate}[label=\Roman*., itemsep=0pt]
\item Існує трапеція, точка перетину діагоналей якої рівновіддалена від її вершин.
\item Існує трапеція, сума довжин бічних сторін якої дорівнює сумі довжин її основ.
\item Існує трапеція, середня лінія якої проходить через точку перетину її діагоналей.
\end{enumerate}}
\answerTable{лише I}{лише II}{лише III}{лише I та II}{лише II та III}
\solutionGrid[4cm]

\task{8}{$\dfrac{9 - y^2}{3 - y} - 3 =$}
\answerTableTall{$y$}{$\dfrac{-y^2 - y}{3 - y}$}{$1 - y$}{$-y$}{$\dfrac{-y^2 - 3y}{3 - y}$}
\solutionGrid[4cm]

\task{9}{Визначте швидкість зорельота, якщо вона становить 0,01 від швидкості світла ($3 \cdot 10^8$ м/с).}
\answerTable{$3 \cdot 10^{10}$}{$3 \cdot 10^4$}{$3 \cdot 0,1^8$}{$3 \cdot 10^{0,08}$}{$3 \cdot 10^6$}
\solutionGrid[3cm]

\newpage

\task{10}{Визначте довжину (модуль) вектора $\overrightarrow{AB}$, де $A(-6; -3; 6)$, а $B$ симетрична їй відносно початку координат.}
\answerTable{$6\sqrt{3}$}{$3$}{$4,5$}{$18$}{$9$}
\solutionGrid[4cm]

\task{11}{Якому проміжку належить корінь рівняння $\log_6 x - \log_6 2 = 1$?}
\answerTable{$(-\infty; 0]$}{$(0; 1]$}{$(1; 4]$}{$(4; 10]$}{$(10; +\infty)$}
\solutionGrid[4cm]

\vspace{0.7cm}
%======================================================================
\task{12}{
\begin{minipage}[t]{0.55\textwidth}
На стороні $BC$ прямокутника $ABCD$ вибрано точку $K$ так, що $\angle KAB = 30^\circ$ (див. рисунок). Визначте довжину відрізка $AK$, якщо периметр прямокутника дорівнює $96$ см, $AB : BC = 3 : 5$. \nmtyear{2025}
\end{minipage}
\hfill
\begin{minipage}[t]{0.4\textwidth}
\begin{flushright}
\begin{tikzpicture}[scale=0.25]
    \coordinate (A) at (0,0); \coordinate (B) at (0,10.8);
    \coordinate (C) at (18,10.8); \coordinate (D) at (18,0);
    \coordinate (K) at (6.23, 10.8);
    \draw[thick] (A) -- (B) -- (C) -- (D) -- cycle; \draw[thick] (A) -- (K);
    \pic [draw, pic text={\small $30^\circ$}, angle radius=0.9cm, angle eccentricity=1.5] {angle = K--A--B};
    \node[below left] at (A) {$A$}; \node[above left] at (B) {$B$}; \node[above right] at (C) {$C$};
    \node[below right] at (D) {$D$}; \node[above] at (K) {$K$}; \fill (K) circle (8pt);
\end{tikzpicture}
\end{flushright}
\end{minipage}
}
\answerTable{$36\sqrt{3}$ см}{$12$ см}{$18$ см}{$12\sqrt{3}$ см}{$36$ см}
\solutionGrid[5cm]

\newpage

\task{13}{Знайдіть суму чотирьох перших членів геометричної прогресії $(b_n)$, де $b_2 = 6, q = -2$.}
\answerTable{$15$}{$-7,5$}{$-9$}{$-45$}{$-15$}
\solutionGrid[4cm]

\task{14}{Обчисліть $2\sin^2 x - 2$, якщо $\cos^2 x = 0,4$.}
\answerTable{$-0,8$}{$-1,2$}{$0,8$}{$-3,2$}{$1,2$}
\solutionGrid[4cm]

\task{15}{Обчисліть суму всіх цілих розв'язків системи $\begin{cases} 3x - 5 < 2x \\ 12 - 9x \leq 3x \end{cases}$}
\answerTable{$14$}{$10$}{$9$}{$15$}{$7$}
\solutionGrid[5cm]

\newpage
\hrule \vspace{0.3cm}
\begin{center} \textit{Завдання 16–18: встановлення відповідності.} \end{center}
\hrule \vspace{0.5cm}

\begin{minipage}[t]{0.55\textwidth}
\noindent\textbf{16.} На рисунку ромб $ABCD$... $AC = 32, DO = 6$. Узгодьте відрізок (1–3) та його довжину (А–Д).
\end{minipage}
\hfill
\begin{minipage}[t]{0.4\textwidth}
    \begin{flushright}
    \begin{tikzpicture}[scale=0.25]
        \coordinate (A) at (-10, 0); \coordinate (C) at (10, 0); \coordinate (B) at (0, 12); \coordinate (D) at (0, -12);
        \coordinate (K) at ($(A)!0.5!(D)$); \coordinate (M) at ($(D)!0.5!(C)$); \coordinate (O) at (0, -6);
        \draw[thick] (A) -- (B) -- (C) -- (D) -- cycle; \draw[thick] (K) -- (M); \draw[thick] (B) -- (D);
        \node[left] at (A) {$A$}; \node[above] at (B) {$B$}; \node[right] at (C) {$C$}; \node[below] at (D) {$D$};
        \node[below left] at (K) {$K$}; \node[below right] at (M) {$M$}; \node[above right] at (O) {$O$};
        \fill (K) circle (8pt); \fill (M) circle (8pt); \fill (O) circle (8pt);
    \end{tikzpicture}
    \end{flushright}
\end{minipage}

\vspace{0.3cm}
\matchingLayout{
    1. $KM$ \\ 2. $BO$ \\ 3. $AB$
}{
    А 12 см; Б 16 см; В 18 см; Г 20 см; Д 24 см
}{
    \answerGrid
}
\solutionGrid[5cm]

\newpage

\noindent\textbf{17.} Узгодьте проміжок (1–3) з властивістю функції (А–Д) (згідно з графіком).
\begin{minipage}[t]{0.55\textwidth}
    1. $[-4; -2]$ \\ 2. $[-2; 2]$ \\ 3. $[2; 5]$
    \vspace{0.3cm}
    А спадає; Б точка макс.; В перетин $y = -4,5$
\end{minipage}
\hfill
\begin{minipage}[t]{0.40\textwidth}
    \begin{flushright}
    \begin{tikzpicture}[scale=0.5]
        \draw[step=1cm,gray!50,very thin] (-5.5,-7.5) grid (5.5,4.5);
        \draw[->, thick] (-5.5,0) -- (5.5,0) node[below] {$x$};
        \draw[->, thick] (0,-7.5) -- (0,4.5) node[left] {$y$};
        \draw[thick] plot [smooth, tension=0.44] coordinates {(-4, -6) (-2, -3) (0, 1) (2, -3) (5, -1)};
    \end{tikzpicture}
    \end{flushright}
\end{minipage}
\solutionGrid[5cm]

\task{18}{Узгодьте початок речення з його закінченням ($n \neq 0$).}
\matchingLayout{
    1. $(m+1)^2-1=n$ \\ 2. $\sqrt{2^m}=2^n$ \\ 3. $\log_{2^n} 4^m = 1$
}{
    А $n=2m$; Б $n=m^2$; В $n=m^2+2m$
}{
    \answerGrid
}
\solutionGrid[5cm]

\newpage
\hrule \vspace{0.3cm}
\begin{center} \textit{Завдання 19–22: запишіть числову відповідь.} \end{center}
\hrule \vspace{0.5cm}

\task{19}{Визначте суму всіх цілих $a \in [-8; 5]$, при яких рівняння $\dfrac{5^{4x+2a-6}-1}{\sqrt{x-2}}=0$ має корінь.}
\solutionGrid[6cm]
\answerBox

\task{20}{Обчисліть $f(-2) \cdot f'(4)$ для $f(x)=x(5-x)$.}
\solutionGrid[5cm]
\answerBox

\task{21}{Скільки було опитаних, якщо «Фантастику» вибрало на 30 осіб більше за «Хоррор»?}
\begin{center}
\begin{tikzpicture}[scale=1]
\filldraw[fill=orange!20] (0,0) -- (0:2) arc (0:144:2) -- cycle;
\filldraw[fill=green!20] (0,0) -- (270:2) arc (270:324:2) -- cycle;
\node at (72:1.3) {40\%}; \node at (297:1.3) {Ф};
\end{tikzpicture}
\end{center}
\solutionGrid[6cm]
\answerBox

\task{22}{Обчисліть об'єм піраміди (у см$^3$), якщо її висота як у конуса, медіана основи — 9 см, а твірна конуса — 12 см.}
\solutionGrid[7cm]
\answerBox

\end{document}