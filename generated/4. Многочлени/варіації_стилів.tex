\documentclass[14pt]{extarticle}
\usepackage{fontspec}
\usepackage{polyglossia}
\setdefaultlanguage{ukrainian}

\defaultfontfeatures{Ligatures=TeX}
\setmainfont{Liberation Serif}
\setsansfont{Liberation Sans}
\setmonofont{Liberation Mono}

\usepackage[a4paper,margin=2cm,bottom=2.5cm,top=2.5cm]{geometry}
\usepackage{amsmath,amssymb}
\usepackage{enumitem}
\usepackage{tikz}
\usepackage{xcolor}
\usepackage{array}
\usepackage{fancyhdr}

\definecolor{headerblue}{RGB}{0, 102, 204}
\definecolor{yearcolor}{RGB}{128, 0, 128}

\pagestyle{fancy}
\fancyhf{}
\renewcommand{\headrulewidth}{0pt}
\fancyfoot[C]{\thepage}

\newcommand{\answerTable}[5]{
\begin{center}
\begin{tabular}{|*{5}{>{\centering\arraybackslash}m{2.8cm}|}}
\hline
\rule[-0.3cm]{0pt}{0.8cm}\textbf{А} & \textbf{Б} & \textbf{В} & \textbf{Г} & \textbf{Д} \\
\hline
\rule[-0.4cm]{0pt}{1.0cm}#1 & \rule[-0.4cm]{0pt}{1.0cm}#2 & \rule[-0.4cm]{0pt}{1.0cm}#3 & \rule[-0.4cm]{0pt}{1.0cm}#4 & \rule[-0.4cm]{0pt}{1.0cm}#5 \\
\hline
\end{tabular}
\end{center}
}

\newcommand{\answerTableBig}[5]{
\begin{center}
\begin{tabular}{|*{5}{>{\centering\arraybackslash}m{2.8cm}|}}
\hline
\rule[-0.3cm]{0pt}{0.8cm}\textbf{А} & \textbf{Б} & \textbf{В} & \textbf{Г} & \textbf{Д} \\
\hline
\rule[-0.6cm]{0pt}{1.4cm}#1 & \rule[-0.6cm]{0pt}{1.4cm}#2 & \rule[-0.6cm]{0pt}{1.4cm}#3 & \rule[-0.6cm]{0pt}{1.4cm}#4 & \rule[-0.6cm]{0pt}{1.4cm}#5 \\
\hline
\end{tabular}
\end{center}
}

\newcommand{\task}[2]{\noindent\makebox[1.5em][l]{\textbf{#1.}}\parbox[t]{\dimexpr\textwidth-1.5em}{#2}}
\newcommand{\nmtyear}[1]{\hfill{\small\color{yearcolor}(#1)}}

\begin{document}

\begin{center}
{\Large\textbf{\color{headerblue}ВАРІАЦІЇ СТИЛІВ ЗАВДАНЬ}}
\end{center}

\begin{center}
{\large Тема: \textbf{Многочлени. ФСМ --- різні формулювання}}
\end{center}

\vspace{0.5cm}

%======================================================================
% СТИЛЬ А: ЗВОРОТНІ ЗАВДАННЯ
% Замість "спростіть" - "який вираз дає такий результат"
%======================================================================

\begin{center}
{\large\textbf{\color{headerblue}Стиль А: Зворотні завдання}}
\end{center}

\vspace{0.3cm}

% Оригінал: $(4x-5)^2 = 16x^2 - 40x + 25$
\task{1}{Який вираз при піднесенні до квадрата дає $9x^2 - 24x + 16$? \nmtyear{gen-A}}
\answerTable{$(9x - 4)$}{$(3x - 4)$}{$(3x + 4)$}{$(9x + 16)$}{$(3x - 8)$}
% Відповідь: Б (3x-4)^2

\vspace{0.5cm}

\task{2}{Який вираз при піднесенні до квадрата дає $25y^2 + 30y + 9$? \nmtyear{gen-A}}
\answerTable{$(25y + 3)$}{$(5y - 3)$}{$(5y + 3)$}{$(5y + 9)$}{$(25y + 9)$}
% Відповідь: В (5y+3)^2

\vspace{0.5cm}

\task{3}{Добуток яких двох однакових двочленів дорівнює $4a^2 - 12a + 9$? \nmtyear{gen-A}}
\answerTable{$(4a - 3)(4a - 3)$}{$(2a - 3)(2a - 3)$}{$(2a + 3)(2a + 3)$}{$(a - 3)(4a - 3)$}{$(2a - 9)(2a - 1)$}
% Відповідь: Б

\vspace{0.5cm}

% Оригінал: $(a-b)(a+b) = a^2 - b^2$
\task{4}{Добуток яких двочленів дорівнює $16m^2 - 49$? \nmtyear{gen-A}}
\answerTable{$(4m - 7)^2$}{$(4m + 7)^2$}{$(4m - 7)(4m + 7)$}{$(16m - 49)(16m + 49)$}{$(8m - 7)(8m + 7)$}
% Відповідь: В

\vspace{0.5cm}

\task{5}{Вираз $36x^2 - 25y^2$ можна записати як добуток: \nmtyear{gen-A}}
\answerTable{$(6x - 5y)^2$}{$(6x + 5y)^2$}{$(6x - 5y)(6x + 5y)$}{$(36x - 25y)(36x + 25y)$}{$(18x - 5y)(18x + 5y)$}
% Відповідь: В

\vspace{0.5cm}

%======================================================================
% СТИЛЬ Б: ГЕОМЕТРИЧНИЙ КОНТЕКСТ
% Площі, периметри, сторони фігур
%======================================================================

\begin{center}
{\large\textbf{\color{headerblue}Стиль Б: Геометричний контекст}}
\end{center}

\vspace{0.3cm}

\task{6}{Сторона квадрата дорівнює $(3x + 2)$ см. Чому дорівнює площа цього квадрата? \nmtyear{gen-B}}
\answerTable{$(9x^2 + 4)$ см$^2$}{$(9x^2 + 12x + 4)$ см$^2$}{$(6x + 4)$ см$^2$}{$(3x^2 + 4)$ см$^2$}{$(9x + 4)$ см$^2$}
% Відповідь: Б

\vspace{0.5cm}

\task{7}{Сторони прямокутника дорівнюють $(5a - 3)$ і $(5a + 3)$ см. Знайдіть площу прямокутника. \nmtyear{gen-B}}
\answerTable{$(25a^2 - 9)$ см$^2$}{$(25a^2 + 9)$ см$^2$}{$(10a)$ см$^2$}{$(25a - 9)$ см$^2$}{$(5a^2 - 9)$ см$^2$}
% Відповідь: А

\vspace{0.5cm}

\task{8}{Площа квадрата дорівнює $(x^2 - 6x + 9)$ см$^2$. Чому дорівнює сторона квадрата? \nmtyear{gen-B}}
\answerTable{$(x - 9)$ см}{$(x + 3)$ см}{$(x - 3)$ см}{$(x^2 - 3)$ см}{$(x - 6)$ см}
% Відповідь: В

\vspace{0.5cm}

\task{9}{Площа квадрата дорівнює $(4y^2 + 20y + 25)$ м$^2$. Знайдіть периметр квадрата. \nmtyear{gen-B}}
\answerTable{$(2y + 5)$ м}{$(4y + 10)$ м}{$(8y + 20)$ м}{$(16y + 40)$ м}{$(2y + 25)$ м}
% Відповідь: В (сторона = 2y+5, периметр = 4(2y+5) = 8y+20)

\vspace{0.5cm}

\task{10}{Різниця площ двох квадратів зі сторонами $7x$ і $3$ дорівнює: \nmtyear{gen-B}}
\answerTable{$(7x - 3)(7x + 3)$}{$(49x^2 - 9)$}{$(7x - 3)^2$}{$(4x)^2$}{$49x - 9$}
% Відповідь: Б (і А теж правильно - тотожні)

\vspace{0.5cm}

%======================================================================
% СТИЛЬ В: ЗНАЙТИ ПОМИЛКУ
% "Учень зробив так... Де помилка / Яка правильна відповідь"
%======================================================================

\begin{center}
{\large\textbf{\color{headerblue}Стиль В: Знайти помилку}}
\end{center}

\vspace{0.3cm}

\task{11}{Учень спростив вираз $(2x - 5)^2$ і отримав $4x^2 - 25$. Яка правильна відповідь? \nmtyear{gen-V}}
\answerTable{$4x^2 + 25$}{$4x^2 - 10x + 25$}{$4x^2 - 20x + 25$}{$2x^2 - 20x + 25$}{$4x^2 - 5x + 25$}
% Відповідь: В (учень забув подвоєний добуток)

\vspace{0.5cm}

\task{12}{Учень записав: $(a + b)^2 = a^2 + b^2$. Яка правильна формула? \nmtyear{gen-V}}
\answerTable{$a^2 - b^2$}{$a^2 + 2ab + b^2$}{$a^2 - 2ab + b^2$}{$2a^2 + 2b^2$}{$(a + b)(a - b)$}
% Відповідь: Б

\vspace{0.5cm}

\task{13}{Учень розклав $x^2 - 9$ як $(x - 3)^2$. Яке правильне розкладання? \nmtyear{gen-V}}
\answerTable{$(x + 3)^2$}{$(x - 9)(x + 1)$}{$(x - 3)(x + 3)$}{$(x - 3)(x - 3)$}{$x(x - 9)$}
% Відповідь: В

\vspace{0.5cm}

\task{14}{Учень спростив $3(x - 2) - 2(x + 1)$ і отримав $x - 8$. Знайдіть правильну відповідь. \nmtyear{gen-V}}
\answerTable{$x - 4$}{$x + 4$}{$x - 8$}{$5x - 8$}{$x$}
% Відповідь: А (3x - 6 - 2x - 2 = x - 8... так учень правий! Треба інший приклад)
% Переробимо: 3(x-2) - 2(x+1) = 3x - 6 - 2x - 2 = x - 8 ✓
% Новий: Учень спростив 4(x-3) - 2(x-1) і отримав 2x - 14

\vspace{0.5cm}

\task{15}{Учень спростив $4(x - 3) - 2(x - 1)$ і отримав $2x - 14$. Яка правильна відповідь? \nmtyear{gen-V}}
\answerTable{$2x - 10$}{$2x - 14$}{$2x - 8$}{$6x - 14$}{$2x - 4$}
% Відповідь: А (4x - 12 - 2x + 2 = 2x - 10)

\vspace{0.5cm}

\task{16}{Учень розклав $4x^2 - 12x + 9$ як $(4x - 3)(x - 3)$. Яке правильне розкладання? \nmtyear{gen-V}}
\answerTable{$(2x - 3)^2$}{$(2x + 3)^2$}{$(4x - 9)(x - 1)$}{$(2x - 3)(2x + 3)$}{$(4x - 3)^2$}
% Відповідь: А

\vspace{0.5cm}

%======================================================================
% СТИЛЬ Г: ПОРІВНЯННЯ / ТВЕРДЖЕННЯ
% "Яке твердження правильне" або "Порівняйте"
%======================================================================

\begin{center}
{\large\textbf{\color{headerblue}Стиль Г: Порівняння та твердження}}
\end{center}

\vspace{0.3cm}

\task{17}{Яке з тверджень є правильним? \nmtyear{gen-G}}

\vspace{0.2cm}
\begin{tabular}{ll}
\textbf{А} & $(a + b)^2 = a^2 + b^2$ \\
\textbf{Б} & $(a - b)^2 = a^2 - b^2$ \\
\textbf{В} & $(a + b)(a - b) = a^2 - b^2$ \\
\textbf{Г} & $(a + b)^2 = a^2 - 2ab + b^2$ \\
\textbf{Д} & $(a - b)(a + b) = a^2 + b^2$ \\
\end{tabular}
% Відповідь: В

\vspace{0.7cm}

\task{18}{Для виразу $(3x + 4)^2$ правильним є твердження: \nmtyear{gen-G}}

\vspace{0.2cm}
\begin{tabular}{ll}
\textbf{А} & коефіцієнт при $x^2$ дорівнює 3 \\
\textbf{Б} & коефіцієнт при $x$ дорівнює 12 \\
\textbf{В} & вільний член дорівнює 8 \\
\textbf{Г} & коефіцієнт при $x$ дорівнює 24 \\
\textbf{Д} & коефіцієнт при $x^2$ дорівнює 6 \\
\end{tabular}
% Відповідь: Г (9x^2 + 24x + 16)

\vspace{0.7cm}

\task{19}{Порівняйте значення виразів $A = (x + 3)^2$ і $B = x^2 + 9$ при $x > 0$. \nmtyear{gen-G}}
\answerTable{$A > B$}{$A < B$}{$A = B$}{$A \geq B$}{Залежить від $x$}
% Відповідь: А (A = x^2 + 6x + 9 > x^2 + 9 при x > 0)

\vspace{0.5cm}

\task{20}{Яке з тверджень є правильним для будь-яких $a$ і $b$? \nmtyear{gen-G}}

\vspace{0.2cm}
\begin{tabular}{ll}
\textbf{А} & $(a + b)^2 > a^2 + b^2$ \\
\textbf{Б} & $(a + b)^2 \geq a^2 + b^2$ \\
\textbf{В} & $(a + b)^2 = a^2 + b^2$ \\
\textbf{Г} & $(a + b)^2 < a^2 + b^2$ \\
\textbf{Д} & $(a + b)^2 \leq a^2 + b^2$ \\
\end{tabular}
% Відповідь: Б (рівність при ab = 0)

\vspace{0.7cm}

\task{21}{Скільки членів містить многочлен $(2x - 3)^2$ після спрощення? \nmtyear{gen-G}}
\answerTable{$1$}{$2$}{$3$}{$4$}{$5$}
% Відповідь: В (4x^2 - 12x + 9)

\vspace{0.5cm}

%======================================================================
% СТИЛЬ Д: ПІДСТАНОВКА / ЗНАЙТИ ЗНАЧЕННЯ
% "При якому x вираз дорівнює..."
%======================================================================

\begin{center}
{\large\textbf{\color{headerblue}Стиль Д: Підстановка та обчислення}}
\end{center}

\vspace{0.3cm}

\task{22}{При якому значенні $x$ вираз $(x + 3)^2 - (x - 3)^2$ дорівнює 48? \nmtyear{gen-D}}
\answerTable{$2$}{$3$}{$4$}{$6$}{$12$}
% Відповідь: В ((x+3)^2 - (x-3)^2 = 12x, 12x = 48, x = 4)

\vspace{0.5cm}

\task{23}{При якому значенні $a$ вираз $(a - 2)(a + 2)$ дорівнює 21? \nmtyear{gen-D}}
\answerTable{$3$}{$4$}{$5$}{$6$}{$7$}
% Відповідь: В (a^2 - 4 = 21, a^2 = 25, a = 5)

\vspace{0.5cm}

\task{24}{Знайдіть значення виразу $a^2 - 2ab + b^2$, якщо $a - b = 7$. \nmtyear{gen-D}}
\answerTable{$7$}{$14$}{$49$}{$-49$}{Недостатньо даних}
% Відповідь: В ((a-b)^2 = 49)

\vspace{0.5cm}

\task{25}{Якщо $x + y = 5$ і $xy = 6$, то $x^2 + y^2$ дорівнює: \nmtyear{gen-D}}
\answerTable{$11$}{$13$}{$19$}{$25$}{$31$}
% Відповідь: Б (x^2 + y^2 = (x+y)^2 - 2xy = 25 - 12 = 13)

\vspace{0.5cm}

\task{26}{Обчисліть $97^2 - 3^2$ без калькулятора. \nmtyear{gen-D}}
\answerTable{$9400$}{$9409$}{$9000$}{$9394$}{$8800$}
% Відповідь: А ((97-3)(97+3) = 94 * 100 = 9400)

\vspace{0.5cm}

\task{27}{Обчисліть $53 \cdot 47$ без калькулятора. \nmtyear{gen-D}}
\answerTable{$2491$}{$2500$}{$2401$}{$2509$}{$2409$}
% Відповідь: А ((50+3)(50-3) = 2500 - 9 = 2491)

\vspace{0.5cm}

\task{28}{Знайдіть значення виразу $(a + b)^2 + (a - b)^2$, якщо $a = 3$ і $b = 2$. \nmtyear{gen-D}}
\answerTable{$24$}{$26$}{$13$}{$50$}{$10$}
% Відповідь: Б (25 + 1 = 26, або 2(a^2 + b^2) = 2(9+4) = 26)

\vspace{0.5cm}

%======================================================================
% СТИЛЬ Е: РІВНОСИЛЬНІСТЬ / ТОТОЖНІСТЬ
% "Який вираз тотожно рівний даному"
%======================================================================

\begin{center}
{\large\textbf{\color{headerblue}Стиль Е: Тотожності}}
\end{center}

\vspace{0.3cm}

\task{29}{Який вираз тотожно рівний $(x + 2)^2 - 4x$? \nmtyear{gen-E}}
\answerTable{$x^2 + 4$}{$(x - 2)^2$}{$(x + 2)(x - 2)$}{$x^2 - 4$}{$x^2$}
% Відповідь: А (x^2 + 4x + 4 - 4x = x^2 + 4)

\vspace{0.5cm}

\task{30}{Який вираз тотожно рівний $(a + b)^2 - (a - b)^2$? \nmtyear{gen-E}}
\answerTable{$2ab$}{$4ab$}{$2a^2 + 2b^2$}{$4a^2$}{$4b^2$}
% Відповідь: Б

\vspace{0.5cm}

\task{31}{Який вираз тотожно рівний $(a + b)^2 + (a - b)^2$? \nmtyear{gen-E}}
\answerTable{$2ab$}{$4ab$}{$2a^2 + 2b^2$}{$2a^2 - 2b^2$}{$4a^2 + 4b^2$}
% Відповідь: В

\vspace{0.5cm}

\task{32}{Яка з рівностей є тотожністю? \nmtyear{gen-E}}

\vspace{0.2cm}
\begin{tabular}{ll}
\textbf{А} & $x^2 + 4x + 4 = (x + 4)^2$ \\
\textbf{Б} & $x^2 - 4x + 4 = (x - 4)^2$ \\
\textbf{В} & $x^2 + 4x + 4 = (x + 2)^2$ \\
\textbf{Г} & $x^2 - 4x + 4 = (x + 2)^2$ \\
\textbf{Д} & $x^2 + 4x + 4 = (x - 2)^2$ \\
\end{tabular}
% Відповідь: В

\vspace{0.7cm}

\task{33}{Вираз $x^2 - y^2$ можна замінити на: \nmtyear{gen-E}}
\answerTable{$(x - y)^2$}{$(x + y)^2$}{$(x - y)(x + y)$}{$x^2 + y^2 - 2xy$}{$(x - y)(x - y)$}
% Відповідь: В

\vspace{0.5cm}

%======================================================================
% СТИЛЬ Ж: ПРАКТИЧНИЙ КОНТЕКСТ
% Реальні ситуації з використанням ФСМ
%======================================================================

\begin{center}
{\large\textbf{\color{headerblue}Стиль Ж: Практичний контекст}}
\end{center}

\vspace{0.3cm}

\task{34}{Вартість товару спочатку зросла на $x\%$, а потім знизилась на $x\%$. Як змінилась початкова ціна 100 грн? \nmtyear{gen-ZH}}
\answerTable{Не змінилась}{Зросла}{Зменшилась на $\dfrac{x^2}{100}$ грн}{Зменшилась на $x$ грн}{Зменшилась на $2x$ грн}
% Відповідь: В (100 * (1 + x/100)(1 - x/100) = 100(1 - x²/10000) = 100 - x²/100)

\vspace{0.5cm}

\task{35}{Квадратну ділянку зі стороною $a$ м розширили на 3 м з кожного боку. На скільки збільшилась площа? \nmtyear{gen-ZH}}
\answerTable{$9$ м$^2$}{$6a$ м$^2$}{$6a + 9$ м$^2$}{$12a$ м$^2$}{$(a + 3)^2$ м$^2$}
% Відповідь: В ((a+3)^2 - a^2 = a^2 + 6a + 9 - a^2 = 6a + 9)

\vspace{0.5cm}

\task{36}{Добуток двох послідовних парних чисел дорівнює 48. Знайдіть їх суму. \nmtyear{gen-ZH}}
\answerTable{$10$}{$12$}{$14$}{$16$}{$18$}
% Відповідь: В (6 * 8 = 48, сума = 14)

\vspace{0.5cm}

\task{37}{Різниця квадратів двох послідовних натуральних чисел дорівнює 15. Знайдіть менше число. \nmtyear{gen-ZH}}
\answerTable{$5$}{$6$}{$7$}{$8$}{$9$}
% Відповідь: В ((n+1)^2 - n^2 = 2n + 1 = 15, n = 7)

\vspace{0.5cm}

\task{38}{Сума двох чисел дорівнює 10, а їх добуток --- 21. Чому дорівнює сума їх квадратів? \nmtyear{gen-ZH}}
\answerTable{$52$}{$58$}{$64$}{$79$}{$100$}
% Відповідь: Б (x^2 + y^2 = (x+y)^2 - 2xy = 100 - 42 = 58)

\vspace{0.5cm}

%======================================================================
% СТИЛЬ З: НЕПОВНА ІНФОРМАЦІЯ
% "Чого не вистачає" або "Знайдіть невідомий коефіцієнт"
%======================================================================

\begin{center}
{\large\textbf{\color{headerblue}Стиль З: Знайти невідоме}}
\end{center}

\vspace{0.3cm}

\task{39}{При якому значенні $k$ тричлен $x^2 + kx + 16$ є повним квадратом? \nmtyear{gen-Z}}
\answerTable{$4$}{$8$}{$\pm 8$}{$16$}{$\pm 4$}
% Відповідь: В (x^2 ± 8x + 16 = (x ± 4)^2)

\vspace{0.5cm}

\task{40}{При якому значенні $m$ вираз $9x^2 - mx + 25$ є повним квадратом двочлена? \nmtyear{gen-Z}}
\answerTable{$15$}{$30$}{$\pm 30$}{$45$}{$\pm 15$}
% Відповідь: В ((3x ± 5)^2 = 9x^2 ± 30x + 25)

\vspace{0.5cm}

\task{41}{Знайдіть $a$, якщо $x^2 + 10x + a = (x + 5)^2$. \nmtyear{gen-Z}}
\answerTable{$5$}{$10$}{$20$}{$25$}{$50$}
% Відповідь: Г

\vspace{0.5cm}

\task{42}{Яке число потрібно додати до $x^2 + 6x$, щоб отримати повний квадрат? \nmtyear{gen-Z}}
\answerTable{$3$}{$6$}{$9$}{$12$}{$36$}
% Відповідь: В (x^2 + 6x + 9 = (x+3)^2)

\vspace{0.5cm}

\task{43}{При якому $n$ вираз $(2x + n)(2x - n)$ дорівнює $4x^2 - 9$? \nmtyear{gen-Z}}
\answerTable{$1$}{$2$}{$3$}{$4$}{$9$}
% Відповідь: В

\vspace{0.5cm}

%======================================================================
% СТИЛЬ И: ПОСЛІДОВНІСТЬ ДІЙ
% Покроковий аналіз
%======================================================================

\begin{center}
{\large\textbf{\color{headerblue}Стиль И: Аналіз розв'язання}}
\end{center}

\vspace{0.3cm}

\task{44}{Для спрощення виразу $(x + 2)^2 - (x - 2)^2$ найзручніше: \nmtyear{gen-I}}

\vspace{0.2cm}
\begin{tabular}{ll}
\textbf{А} & піднести кожну дужку до квадрата і відняти \\
\textbf{Б} & скористатися формулою $a^2 - b^2 = (a-b)(a+b)$ \\
\textbf{В} & підставити $x = 1$ і обчислити \\
\textbf{Г} & розкрити дужки методом FOIL \\
\textbf{Д} & обидва способи А і Б однаково зручні \\
\end{tabular}
% Відповідь: Б (результат = 4 * 2x = 8x миттєво)

\vspace{0.7cm}

\task{45}{Для обчислення $99^2$ найзручніше використати формулу: \nmtyear{gen-I}}

\vspace{0.2cm}
\begin{tabular}{ll}
\textbf{А} & $(a + b)^2 = a^2 + 2ab + b^2$ \\
\textbf{Б} & $(a - b)^2 = a^2 - 2ab + b^2$ \\
\textbf{В} & $(a - b)(a + b) = a^2 - b^2$ \\
\textbf{Г} & Жодну з формул \\
\textbf{Д} & $a^2 + b^2$ \\
\end{tabular}
% Відповідь: Б (99^2 = (100-1)^2 = 10000 - 200 + 1 = 9801)

\vspace{0.7cm}

\task{46}{Для розкладання $x^2 - 10x + 25$ на множники потрібно: \nmtyear{gen-I}}

\vspace{0.2cm}
\begin{tabular}{ll}
\textbf{А} & скористатися формулою різниці квадратів \\
\textbf{Б} & винести спільний множник \\
\textbf{В} & скористатися формулою квадрата різниці \\
\textbf{Г} & скористатися формулою квадрата суми \\
\textbf{Д} & вираз не розкладається на множники \\
\end{tabular}
% Відповідь: В ((x-5)^2)

\vspace{0.7cm}

%======================================================================
% СТИЛЬ К: КОМБІНОВАНІ
% Кілька кроків або ФСМ
%======================================================================

\begin{center}
{\large\textbf{\color{headerblue}Стиль К: Комбіновані завдання}}
\end{center}

\vspace{0.3cm}

\task{47}{Спростіть вираз $(x + 3)^2 - 6x - 9$. \nmtyear{gen-K}}
\answerTable{$x^2$}{$x^2 + 9$}{$(x - 3)^2$}{$x^2 - 9$}{$x^2 + 6x$}
% Відповідь: А (x^2 + 6x + 9 - 6x - 9 = x^2)

\vspace{0.5cm}

\task{48}{Спростіть вираз $(a + b)^2 - 2ab$. \nmtyear{gen-K}}
\answerTable{$a^2 + b^2$}{$(a - b)^2$}{$a^2 - b^2$}{$2ab$}{$(a + b)(a - b)$}
% Відповідь: А

\vspace{0.5cm}

\task{49}{Спростіть вираз $(x - 1)(x + 1)(x^2 + 1)$. \nmtyear{gen-K}}
\answerTable{$x^4 - 1$}{$x^4 + 1$}{$(x^2 - 1)^2$}{$x^4 - 2x^2 + 1$}{$x^2 - 1$}
% Відповідь: А ((x^2 - 1)(x^2 + 1) = x^4 - 1)

\vspace{0.5cm}

\task{50}{Розкладіть на множники $x^4 - 16$. \nmtyear{gen-K}}

\vspace{0.2cm}
\begin{tabular}{ll}
\textbf{А} & $(x^2 - 4)(x^2 + 4)$ \\
\textbf{Б} & $(x - 2)(x + 2)(x^2 + 4)$ \\
\textbf{В} & $(x - 2)^2(x + 2)^2$ \\
\textbf{Г} & $(x^2 - 4)^2$ \\
\textbf{Д} & $(x - 4)(x + 4)$ \\
\end{tabular}
% Відповідь: Б (найповніший розклад)

\vspace{0.7cm}

\task{51}{Якщо $a + b = 7$ і $a - b = 3$, то $a^2 - b^2$ дорівнює: \nmtyear{gen-K}}
\answerTable{$4$}{$10$}{$21$}{$40$}{$49$}
% Відповідь: В ((a+b)(a-b) = 7 * 3 = 21)

\vspace{0.5cm}

\task{52}{Спростіть вираз $\dfrac{x^2 - 4}{x - 2}$ при $x \neq 2$. \nmtyear{gen-K}}
\answerTable{$x - 2$}{$x + 2$}{$x^2 - 2$}{$\dfrac{x - 2}{1}$}{$x$}
% Відповідь: Б

\vspace{0.5cm}

\task{53}{Спростіть вираз $\dfrac{a^2 - 6a + 9}{a - 3}$ при $a \neq 3$. \nmtyear{gen-K}}
\answerTable{$a - 3$}{$a + 3$}{$a^2 - 3$}{$\dfrac{a - 3}{1}$}{$a - 9$}
% Відповідь: А ((a-3)^2 / (a-3) = a - 3)

\vspace{0.5cm}

\task{54}{Обчисліть $(2 + \sqrt{3})^2 + (2 - \sqrt{3})^2$. \nmtyear{gen-K}}
\answerTable{$8$}{$14$}{$12$}{$10$}{$16$}
% Відповідь: Б (2(4 + 3) = 14)

\vspace{0.5cm}

\task{55}{Обчисліть $(5 + \sqrt{2})(5 - \sqrt{2}) + \sqrt{2}$. \nmtyear{gen-K}}
\answerTable{$23$}{$25$}{$23 + \sqrt{2}$}{$25 + \sqrt{2}$}{$27$}
% Відповідь: В (25 - 2 + √2 = 23 + √2)

\vspace{0.5cm}

\end{document}
