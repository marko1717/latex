\documentclass[12pt]{extarticle}
\usepackage{fontspec}
\usepackage{polyglossia}
\setdefaultlanguage{ukrainian}

\defaultfontfeatures{Ligatures=TeX}
\setmainfont{Liberation Serif}

\usepackage[a4paper,margin=2cm]{geometry}
\usepackage{amsmath,amssymb}
\usepackage{multicol}
\usepackage{xcolor}

\definecolor{headerblue}{RGB}{0, 102, 204}

\begin{document}

\begin{center}
{\Large\textbf{\color{headerblue}ВІДПОВІДІ}}
\end{center}

\begin{center}
{\large Тема 17: Коло, круг і їх елементи}
\end{center}

\vspace{0.5cm}

\textbf{Блок 1: Радіус, діаметр, довжина кола}

\begin{multicols}{5}
\noindent
1. А ($10$) \\
2. А ($9$) \\
3. А ($14\pi$) \\
4. А ($10\pi$) \\
5. А ($6$) \\
6. А ($20$) \\
7. А \\
8. А \\
9. А ($5$) \\
10. А ($70\pi$) \\
11. А ($\pi$) \\
12. А ($\pi$) \\
13. А ($6\pi$) \\
14. А ($12{,}5$) \\
15. А ($3:5$)
\end{multicols}

\textbf{Блок 2: Площа круга}

\begin{multicols}{5}
\noindent
16. А ($16\pi$) \\
17. А ($36\pi$) \\
18. А ($5$) \\
19. А ($14$) \\
20. А \\
21. А \\
22. А ($10$) \\
23. А ($18\pi$) \\
24. А ($16\pi$) \\
25. А ($1:9$) \\
26. А ($25\pi$) \\
27. А ($16\pi$)
\end{multicols}

\textbf{Блок 3: Дуги та центральні кути}

\begin{multicols}{5}
\noindent
28. А ($60°$) \\
29. А ($90°$) \\
30. А ($2\pi$) \\
31. А ($6\pi$) \\
32. А ($60°$) \\
33. А \\
34. А ($10\pi$) \\
35. А ($120°$) \\
36. А ($4\pi$) \\
37. А ($360°$) \\
38. А ($\frac{1}{8}$) \\
39. А \\
40. А ($6$) \\
41. А ($180°$) \\
42. А ($5:4$)
\end{multicols}

\textbf{Блок 4: Вписані кути}

\begin{multicols}{5}
\noindent
43. А ($40°$) \\
44. А ($70°$) \\
45. А ($60°$) \\
46. А ($90°$) \\
47. А \\
48. А \\
49. А ($80°$) \\
50. А ($110°$) \\
51. А ($180°$) \\
52. А \\
53. А \\
54. А ($50°$) \\
55. А ($70°$) \\
56. А \\
57. А ($110°$)
\end{multicols}

\textbf{Блок 5: Хорди та дотичні}

\begin{multicols}{5}
\noindent
58. А \\
59. А ($24$) \\
60. А ($6$) \\
61. А \\
62. А ($4$) \\
63. А \\
64. А ($12$) \\
65. А ($12$) \\
66. А ($120°$) \\
67. А ($30°$) \\
68. А \\
69. А \\
70. А ($8$) \\
71. А ($90°$) \\
72. А ($5$)
\end{multicols}

\textbf{Блок 6: Площа сектора та сегмента}

\begin{multicols}{5}
\noindent
73. А ($6\pi$) \\
74. А ($27\pi$) \\
75. А \\
76. А ($45°$) \\
77. А ($50\pi$) \\
78. А ($4\pi$) \\
79. А \\
80. А ($12\pi$) \\
81. А ($6\pi$) \\
82. А \\
83. А ($9\sqrt{3}$) \\
84. А ($20\pi$) \\
85. А ($6\sqrt{2}$)
\end{multicols}

\textbf{Блок 7: Взаємне розташування кіл}

\begin{multicols}{5}
\noindent
86. А ($0$) \\
87. А \\
88. А ($2$) \\
89. А \\
90. А \\
91. А ($R+r$) \\
92. А ($|R-r|$) \\
93. А ($11$) \\
94. А ($6$) \\
95. А \\
96. А \\
97. А \\
98. А \\
99. А \\
100. А ($34$)
\end{multicols}

\vspace{1cm}

\textbf{Ключові формули:}

\begin{enumerate}
\item \textbf{Довжина кола:} $C = 2\pi r = \pi d$

\item \textbf{Площа круга:} $S = \pi r^2$

\item \textbf{Довжина дуги:} $l = \dfrac{\pi r \alpha}{180°}$ (де $\alpha$ в градусах)

\item \textbf{Площа сектора:} $S = \dfrac{\pi r^2 \alpha}{360°}$

\item \textbf{Вписаний кут:} $\beta = \dfrac{\alpha}{2}$ (де $\alpha$ --- градусна міра дуги)

\item \textbf{Центральний кут:} рівний градусній мірі дуги

\item \textbf{Відстань від центра до хорди:} $d = \sqrt{r^2 - \left(\dfrac{a}{2}\right)^2}$

\item \textbf{Довжина дотичної:} $t = \sqrt{d^2 - r^2}$ (де $d$ --- відстань від точки до центра)

\item \textbf{Взаємне розташування кіл:}
\begin{itemize}
\item Не перетинаються зовні: $d > R + r$
\item Зовнішній дотик: $d = R + r$
\item Перетинаються: $|R - r| < d < R + r$
\item Внутрішній дотик: $d = |R - r|$
\item Одне всередині іншого: $d < |R - r|$
\end{itemize}

\item \textbf{Вписаний чотирикутник:} сума протилежних кутів $= 180°$
\end{enumerate}

\end{document}
