\documentclass[14pt]{extarticle}
\usepackage{fontspec}
\usepackage{polyglossia}
\setdefaultlanguage{ukrainian}

\defaultfontfeatures{Ligatures=TeX}
\setmainfont{Liberation Serif}
\setsansfont{Liberation Sans}
\setmonofont{Liberation Mono}

\usepackage[a4paper,margin=2cm,bottom=2.5cm,top=2.5cm]{geometry}
\usepackage{amsmath,amssymb}
\usepackage{enumitem}
\usepackage{tikz}
\usepackage{pgfplots}
\pgfplotsset{compat=1.16}
\usetikzlibrary{calc,patterns,angles,quotes}
\usepackage{xcolor}
\usepackage{array}
\usepackage{fancyhdr}
\usepackage{multicol}

% Кольори
\definecolor{headerblue}{RGB}{0, 102, 204}
\definecolor{yearcolor}{RGB}{128, 0, 128}

\pagestyle{fancy}
\fancyhf{}
\renewcommand{\headrulewidth}{0pt}
\fancyfoot[C]{\thepage}

\setlength{\headheight}{15pt}
\setlength{\headsep}{10pt}
\setlength{\footskip}{25pt}

\widowpenalty=10000
\clubpenalty=10000

% === КОМАНДИ ===

% Стандартна таблиця відповідей
\newcommand{\answerTable}[5]{
\begin{center}
\begin{tabular}{|*{5}{>{\centering\arraybackslash}m{2.8cm}|}}
\hline
\rule[-0.3cm]{0pt}{0.8cm}\textbf{А} & \textbf{Б} & \textbf{В} & \textbf{Г} & \textbf{Д} \\
\hline
\rule[-0.4cm]{0pt}{1.0cm}#1 & \rule[-0.4cm]{0pt}{1.0cm}#2 & \rule[-0.4cm]{0pt}{1.0cm}#3 & \rule[-0.4cm]{0pt}{1.0cm}#4 & \rule[-0.4cm]{0pt}{1.0cm}#5 \\
\hline
\end{tabular}
\end{center}
}

% Маленька таблиця відповідей
\newcommand{\answerTableSmall}[5]{
\begin{tabular}{|*{5}{>{\centering\arraybackslash}m{1.1cm}|}}
\hline
\rule[-0.2cm]{0pt}{0.6cm}\textbf{А} & \textbf{Б} & \textbf{В} & \textbf{Г} & \textbf{Д} \\
\hline
\rule[-0.3cm]{0pt}{0.8cm}#1 & #2 & #3 & #4 & #5 \\
\hline
\end{tabular}
}

% Команда для завдань
\newcommand{\gentask}[2]{\noindent\makebox[1.5em][l]{\textbf{#1.}}\parbox[t]{\dimexpr\textwidth-1.5em}{#2}}

\begin{document}

\begin{center}
{\Large\textbf{\color{headerblue}ЗГЕНЕРОВАНІ ЗАВДАННЯ}}
\end{center}

\begin{center}
{\large Тема 17: Коло, круг і їх елементи}
\end{center}

\vspace{0.5cm}

%======================================================================
% БЛОК 1: Радіус, діаметр, довжина кола (15 завдань)
%======================================================================

\section*{Блок 1: Радіус, діаметр, довжина кола}

\gentask{1}{Радіус кола дорівнює 5 см. Знайдіть діаметр.}
\answerTable{10 см}{5 см}{25 см}{2{,}5 см}{$5\pi$ см}

\vspace{0.4cm}

\gentask{2}{Діаметр кола дорівнює 18 см. Знайдіть радіус.}
\answerTable{9 см}{18 см}{36 см}{$9\pi$ см}{4{,}5 см}

\vspace{0.4cm}

\gentask{3}{Радіус кола дорівнює 7 см. Знайдіть довжину кола.}
\answerTable{$14\pi$ см}{$7\pi$ см}{$49\pi$ см}{14 см}{$7\pi^2$ см}

\vspace{0.4cm}

\gentask{4}{Діаметр кола дорівнює 10 см. Знайдіть довжину кола.}
\answerTable{$10\pi$ см}{$5\pi$ см}{$25\pi$ см}{$100\pi$ см}{20 см}

\vspace{0.4cm}

\gentask{5}{Довжина кола дорівнює $12\pi$ см. Знайдіть радіус.}
\answerTable{6 см}{12 см}{$12\pi$ см}{3 см}{24 см}

\vspace{0.4cm}

\gentask{6}{Довжина кола дорівнює $20\pi$ см. Знайдіть діаметр.}
\answerTable{20 см}{10 см}{40 см}{$20\pi$ см}{$10\pi$ см}

\vspace{0.4cm}

\gentask{7}{Формула довжини кола:}
\answerTable{$C = 2\pi r$}{$C = \pi r$}{$C = \pi r^2$}{$C = 2r$}{$C = \pi d^2$}

\vspace{0.4cm}

\gentask{8}{Радіус кола збільшили вдвічі. У скільки разів збільшилася довжина кола?}
\answerTable{у 2 рази}{у 4 рази}{у $\pi$ разів}{не змінилася}{у $2\pi$ разів}

\vspace{0.4cm}

\gentask{9}{Довжина кола 31{,}4 см. Знайдіть радіус (приймаємо $\pi \approx 3{,}14$).}
\answerTable{5 см}{10 см}{15 см}{2{,}5 см}{$\dfrac{31{,}4}{\pi}$ см}

\vspace{0.4cm}

\gentask{10}{Колесо велосипеда має радіус 35 см. Яку відстань проїде велосипед за один оберт колеса?}
\answerTable{$70\pi$ см}{$35\pi$ см}{35 см}{70 см}{$35\pi^2$ см}

\vspace{0.4cm}

\gentask{11}{Довжина кола дорівнює його діаметру, помноженому на:}
\answerTable{$\pi$}{2}{$2\pi$}{$\dfrac{\pi}{2}$}{$\pi^2$}

\vspace{0.4cm}

\gentask{12}{Відношення довжини кола до його діаметра дорівнює:}
\answerTable{$\pi$}{$2\pi$}{2}{$\dfrac{\pi}{2}$}{1}

\vspace{0.4cm}

\gentask{13}{Радіус кола 6 см. Знайдіть довжину півкола.}
\answerTable{$6\pi$ см}{$12\pi$ см}{$3\pi$ см}{6 см}{$36\pi$ см}

\vspace{0.4cm}

\gentask{14}{Довжина кола 50 см. Знайдіть довжину чверті кола.}
\answerTable{12{,}5 см}{25 см}{100 см}{6{,}25 см}{50 см}

\vspace{0.4cm}

\gentask{15}{Два кола мають радіуси 3 і 5 см. Знайдіть відношення їх довжин.}
\answerTable{$3:5$}{$9:25$}{$6:10$}{$5:3$}{$1:2$}

\vspace{0.5cm}

%======================================================================
% БЛОК 2: Площа круга (12 завдань)
%======================================================================

\section*{Блок 2: Площа круга}

\gentask{16}{Радіус круга дорівнює 4 см. Знайдіть площу.}
\answerTable{$16\pi$ см$^2$}{$8\pi$ см$^2$}{$4\pi$ см$^2$}{16 см$^2$}{$64\pi$ см$^2$}

\vspace{0.4cm}

\gentask{17}{Діаметр круга дорівнює 12 см. Знайдіть площу.}
\answerTable{$36\pi$ см$^2$}{$144\pi$ см$^2$}{$12\pi$ см$^2$}{$6\pi$ см$^2$}{$24\pi$ см$^2$}

\vspace{0.4cm}

\gentask{18}{Площа круга дорівнює $25\pi$ см$^2$. Знайдіть радіус.}
\answerTable{5 см}{25 см}{$5\pi$ см}{$\sqrt{25\pi}$ см}{10 см}

\vspace{0.4cm}

\gentask{19}{Площа круга дорівнює $49\pi$ см$^2$. Знайдіть діаметр.}
\answerTable{14 см}{7 см}{49 см}{$14\pi$ см}{28 см}

\vspace{0.4cm}

\gentask{20}{Формула площі круга:}
\answerTable{$S = \pi r^2$}{$S = 2\pi r$}{$S = \pi d$}{$S = \pi r$}{$S = 2\pi r^2$}

\vspace{0.4cm}

\gentask{21}{Радіус круга збільшили вдвічі. У скільки разів збільшилася площа?}
\answerTable{у 4 рази}{у 2 рази}{у $\pi$ разів}{у 8 разів}{у $2\pi$ разів}

\vspace{0.4cm}

\gentask{22}{Площа круга 314 см$^2$ ($\pi \approx 3{,}14$). Знайдіть радіус.}
\answerTable{10 см}{100 см}{5 см}{$\sqrt{100}$ см}{20 см}

\vspace{0.4cm}

\gentask{23}{Знайдіть площу півкруга радіусом 6 см.}
\answerTable{$18\pi$ см$^2$}{$36\pi$ см$^2$}{$9\pi$ см$^2$}{$6\pi$ см$^2$}{$12\pi$ см$^2$}

\vspace{0.4cm}

\gentask{24}{Знайдіть площу чверті круга радіусом 8 см.}
\answerTable{$16\pi$ см$^2$}{$64\pi$ см$^2$}{$32\pi$ см$^2$}{$8\pi$ см$^2$}{$4\pi$ см$^2$}

\vspace{0.4cm}

\gentask{25}{Два круги мають радіуси 2 і 6 см. Знайдіть відношення їх площ.}
\answerTable{$1:9$}{$1:3$}{$2:6$}{$4:36$}{$1:4$}

\vspace{0.4cm}

\gentask{26}{Довжина кола $10\pi$ см. Знайдіть площу круга.}
\answerTable{$25\pi$ см$^2$}{$100\pi$ см$^2$}{$10\pi$ см$^2$}{$5\pi$ см$^2$}{$50\pi$ см$^2$}

\vspace{0.4cm}

\gentask{27}{Площа круга $64\pi$ см$^2$. Знайдіть довжину кола.}
\answerTable{$16\pi$ см}{$64\pi$ см}{$8\pi$ см}{$32\pi$ см}{$128\pi$ см}

\vspace{0.5cm}

%======================================================================
% БЛОК 3: Дуги та центральні кути (15 завдань)
%======================================================================

\section*{Блок 3: Дуги та центральні кути}

\gentask{28}{Центральний кут дорівнює $60°$. Знайдіть градусну міру дуги.}
\answerTable{$60°$}{$30°$}{$120°$}{$180°$}{$300°$}

\vspace{0.4cm}

\gentask{29}{Градусна міра дуги $90°$. Знайдіть центральний кут.}
\answerTable{$90°$}{$45°$}{$180°$}{$270°$}{$135°$}

\vspace{0.4cm}

\gentask{30}{Радіус кола 6 см, центральний кут $60°$. Знайдіть довжину дуги.}
\answerTable{$2\pi$ см}{$\pi$ см}{$6\pi$ см}{$12\pi$ см}{$3\pi$ см}

\vspace{0.4cm}

\gentask{31}{Радіус кола 9 см, центральний кут $120°$. Знайдіть довжину дуги.}
\answerTable{$6\pi$ см}{$3\pi$ см}{$18\pi$ см}{$9\pi$ см}{$12\pi$ см}

\vspace{0.4cm}

\gentask{32}{Довжина дуги $4\pi$ см, радіус 12 см. Знайдіть центральний кут.}
\answerTable{$60°$}{$30°$}{$90°$}{$120°$}{$45°$}

\vspace{0.4cm}

\gentask{33}{Формула довжини дуги:}
\answerTable{$l = \dfrac{\pi r \alpha}{180°}$}{$l = 2\pi r$}{$l = \pi r^2 \cdot \dfrac{\alpha}{360°}$}{$l = \pi r$}{$l = r \alpha$}

\vspace{0.4cm}

\gentask{34}{Радіус кола 10 см. Знайдіть довжину дуги в $180°$.}
\answerTable{$10\pi$ см}{$20\pi$ см}{$5\pi$ см}{10 см}{$100\pi$ см}

\vspace{0.4cm}

\gentask{35}{Дуга становить третину кола. Чому дорівнює її градусна міра?}
\answerTable{$120°$}{$60°$}{$180°$}{$90°$}{$240°$}

\vspace{0.4cm}

\gentask{36}{Дуга становить чверть кола. Знайдіть довжину дуги, якщо радіус 8 см.}
\answerTable{$4\pi$ см}{$2\pi$ см}{$8\pi$ см}{$16\pi$ см}{$\pi$ см}

\vspace{0.4cm}

\gentask{37}{Сума двох дуг, на які точка ділить коло, дорівнює:}
\answerTable{$360°$}{$180°$}{$90°$}{$270°$}{залежить від точки}

\vspace{0.4cm}

\gentask{38}{Центральний кут $45°$. Яку частину кола становить відповідна дуга?}
\answerTable{$\dfrac{1}{8}$}{$\dfrac{1}{4}$}{$\dfrac{1}{2}$}{$\dfrac{1}{6}$}{$\dfrac{1}{12}$}

\vspace{0.4cm}

\gentask{39}{Радіус кола $R$, центральний кут $\alpha$ (в градусах). Довжина дуги:}
\answerTable{$\dfrac{\pi R \alpha}{180}$}{$\pi R \alpha$}{$2\pi R \alpha$}{$R \alpha$}{$\dfrac{R \alpha}{180}$}

\vspace{0.4cm}

\gentask{40}{Довжина дуги $3\pi$ см, центральний кут $90°$. Знайдіть радіус.}
\answerTable{6 см}{3 см}{12 см}{9 см}{$\dfrac{3\pi}{90}$ см}

\vspace{0.4cm}

\gentask{41}{Радіус 5 см, довжина дуги $5\pi$ см. Знайдіть центральний кут.}
\answerTable{$180°$}{$90°$}{$360°$}{$60°$}{$120°$}

\vspace{0.4cm}

\gentask{42}{Дві дуги одного кола мають градусні міри $100°$ і $80°$. Відношення їх довжин:}
\answerTable{$5:4$}{$10:8$}{$100:80$}{$4:5$}{$1:1$}

\vspace{0.5cm}

%======================================================================
% БЛОК 4: Вписані кути (15 завдань)
%======================================================================

\section*{Блок 4: Вписані кути}

\gentask{43}{Вписаний кут опирається на дугу $80°$. Знайдіть вписаний кут.}
\answerTable{$40°$}{$80°$}{$160°$}{$20°$}{$100°$}

\vspace{0.4cm}

\gentask{44}{Вписаний кут дорівнює $35°$. Знайдіть градусну міру дуги, на яку він опирається.}
\answerTable{$70°$}{$35°$}{$17{,}5°$}{$140°$}{$105°$}

\vspace{0.4cm}

\gentask{45}{Центральний кут $120°$. Знайдіть вписаний кут, що опирається на ту саму дугу.}
\answerTable{$60°$}{$120°$}{$240°$}{$30°$}{$180°$}

\vspace{0.4cm}

\gentask{46}{Вписаний кут опирається на діаметр. Чому він дорівнює?}
\answerTable{$90°$}{$180°$}{$45°$}{$60°$}{$0°$}

\vspace{0.4cm}

\gentask{47}{Вписаний кут дорівнює $90°$. На яку дугу він опирається?}
\answerTable{$180°$ (півколо)}{$90°$}{$270°$}{$360°$}{$45°$}

\vspace{0.4cm}

\gentask{48}{Вписані кути, що опираються на одну дугу:}
\answerTable{рівні}{різні}{один більший}{залежить від кола}{не існують}

\vspace{0.4cm}

\gentask{49}{Дуга $ABC$ має градусну міру $200°$. Знайдіть вписаний кут $ABC$.}
\answerTable{$80°$}{$100°$}{$200°$}{$160°$}{$40°$}

\vspace{0.4cm}

\gentask{50}{Чотирикутник $ABCD$ вписаний у коло. $\angle A = 70°$. Знайдіть $\angle C$.}
\answerTable{$110°$}{$70°$}{$180°$}{$90°$}{$140°$}

\vspace{0.4cm}

\gentask{51}{Сума протилежних кутів вписаного чотирикутника дорівнює:}
\answerTable{$180°$}{$360°$}{$90°$}{$270°$}{залежить від чотирикутника}

\vspace{0.4cm}

\gentask{52}{Вписаний кут у 2 рази менший за:}
\answerTable{центральний кут на ту саму дугу}{дугу}{діаметр}{радіус}{хорду}

\vspace{0.4cm}

\gentask{53}{Трикутник вписаний у коло, одна його сторона~--- діаметр. Який це трикутник?}
\answerTable{прямокутний}{гострокутний}{тупокутний}{рівнобедрений}{рівносторонній}

\vspace{0.4cm}

\gentask{54}{Вписаний кут $ABC = 25°$. Центральний кут $AOC$ (та сама дуга) дорівнює:}
\answerTable{$50°$}{$25°$}{$12{,}5°$}{$75°$}{$100°$}

\vspace{0.4cm}

\gentask{55}{Дуга $AC$ дорівнює $140°$. Вписаний кут $ABC$ дорівнює:}
\answerTable{$70°$}{$140°$}{$280°$}{$35°$}{$110°$}

\vspace{0.4cm}

\gentask{56}{Вписаний кут, що опирається на дугу більшу за півколо:}
\answerTable{тупий}{гострий}{прямий}{не існує}{будь-який}

\vspace{0.4cm}

\gentask{57}{Вписаний кут $55°$ опирається на дугу $AB$. Центральний кут $AOB$ дорівнює:}
\answerTable{$110°$}{$55°$}{$27{,}5°$}{$165°$}{$220°$}

\vspace{0.5cm}

%======================================================================
% БЛОК 5: Хорди та дотичні (15 завдань)
%======================================================================

\section*{Блок 5: Хорди та дотичні}

\gentask{58}{Найбільша хорда кола~--- це:}
\answerTable{діаметр}{радіус}{дуга}{дотична}{січна}

\vspace{0.4cm}

\gentask{59}{Радіус кола 13 см, відстань від центра до хорди 5 см. Знайдіть хорду.}
\answerTable{24 см}{12 см}{18 см}{26 см}{10 см}

\vspace{0.4cm}

\gentask{60}{Хорда кола 16 см, радіус 10 см. Знайдіть відстань від центра до хорди.}
\answerTable{6 см}{8 см}{4 см}{12 см}{2 см}

\vspace{0.4cm}

\gentask{61}{Дотична до кола перпендикулярна до:}
\answerTable{радіуса в точці дотику}{діаметра}{хорди}{будь-якого радіуса}{дуги}

\vspace{0.4cm}

\gentask{62}{З точки $A$ проведено дотичну $AB$ і січну $AC$. $AB = 8$ см, $AC = 16$ см. Знайдіть зовнішню частину січної.}
\answerTable{4 см}{8 см}{12 см}{2 см}{6 см}

\vspace{0.4cm}

\gentask{63}{Дві дотичні, проведені з однієї точки до кола:}
\answerTable{рівні}{перпендикулярні}{паралельні}{різні}{не існують}

\vspace{0.4cm}

\gentask{64}{З точки $P$ проведено дві дотичні до кола, точки дотику $A$ і $B$. $PA = 12$ см. Знайдіть $PB$.}
\answerTable{12 см}{6 см}{24 см}{не можна визначити}{$12\sqrt{2}$ см}

\vspace{0.4cm}

\gentask{65}{Радіус кола 5 см, відстань від точки до центра 13 см. Знайдіть довжину дотичної з цієї точки.}
\answerTable{12 см}{8 см}{18 см}{$\sqrt{194}$ см}{14 см}

\vspace{0.4cm}

\gentask{66}{Кут між двома дотичними, проведеними з однієї точки, $60°$. Знайдіть кут $AOB$ (де $A$, $B$~--- точки дотику).}
\answerTable{$120°$}{$60°$}{$30°$}{$180°$}{$90°$}

\vspace{0.4cm}

\gentask{67}{Хорда стягує дугу $120°$. Знайдіть кут між хордою і радіусом до кінця хорди.}
\answerTable{$30°$}{$60°$}{$90°$}{$120°$}{$45°$}

\vspace{0.4cm}

\gentask{68}{Рівні хорди віддалені від центра кола на:}
\answerTable{рівні відстані}{різні відстані}{залежить від кола}{не можна визначити}{нуль}

\vspace{0.4cm}

\gentask{69}{Діаметр, перпендикулярний до хорди:}
\answerTable{ділить її навпіл}{рівний їй}{паралельний їй}{не перетинає її}{є дотичною}

\vspace{0.4cm}

\gentask{70}{Радіус кола 10 см, хорда 12 см. Знайдіть відстань від центра до хорди.}
\answerTable{8 см}{6 см}{4 см}{2 см}{$\sqrt{44}$ см}

\vspace{0.4cm}

\gentask{71}{Дотична і радіус до точки дотику утворюють кут:}
\answerTable{$90°$}{$180°$}{$0°$}{$45°$}{залежить від кола}

\vspace{0.4cm}

\gentask{72}{Відстань від центра кола радіусом 5 см до дотичної дорівнює:}
\answerTable{5 см}{10 см}{0 см}{2{,}5 см}{$5\sqrt{2}$ см}

\vspace{0.5cm}

%======================================================================
% БЛОК 6: Площа сектора та сегмента (13 завдань)
%======================================================================

\section*{Блок 6: Площа сектора та сегмента}

\gentask{73}{Радіус кола 6 см, центральний кут сектора $60°$. Знайдіть площу сектора.}
\answerTable{$6\pi$ см$^2$}{$36\pi$ см$^2$}{$12\pi$ см$^2$}{$3\pi$ см$^2$}{$18\pi$ см$^2$}

\vspace{0.4cm}

\gentask{74}{Радіус кола 9 см, центральний кут сектора $120°$. Знайдіть площу сектора.}
\answerTable{$27\pi$ см$^2$}{$81\pi$ см$^2$}{$9\pi$ см$^2$}{$54\pi$ см$^2$}{$13{,}5\pi$ см$^2$}

\vspace{0.4cm}

\gentask{75}{Формула площі сектора:}
\answerTable{$S = \dfrac{\pi r^2 \alpha}{360°}$}{$S = \pi r^2$}{$S = 2\pi r$}{$S = \dfrac{\pi r \alpha}{180°}$}{$S = r^2 \alpha$}

\vspace{0.4cm}

\gentask{76}{Площа сектора $8\pi$ см$^2$, радіус 8 см. Знайдіть центральний кут.}
\answerTable{$45°$}{$90°$}{$60°$}{$30°$}{$120°$}

\vspace{0.4cm}

\gentask{77}{Площа півкруга радіусом 10 см дорівнює:}
\answerTable{$50\pi$ см$^2$}{$100\pi$ см$^2$}{$25\pi$ см$^2$}{$10\pi$ см$^2$}{$200\pi$ см$^2$}

\vspace{0.4cm}

\gentask{78}{Площа чверті круга радіусом 4 см дорівнює:}
\answerTable{$4\pi$ см$^2$}{$16\pi$ см$^2$}{$8\pi$ см$^2$}{$2\pi$ см$^2$}{$\pi$ см$^2$}

\vspace{0.4cm}

\gentask{79}{Сектор має кут $90°$. Яку частину круга він становить?}
\answerTable{$\dfrac{1}{4}$}{$\dfrac{1}{2}$}{$\dfrac{1}{3}$}{$\dfrac{1}{6}$}{$\dfrac{1}{8}$}

\vspace{0.4cm}

\gentask{80}{Радіус кола 12 см. Знайдіть площу сектора з кутом $30°$.}
\answerTable{$12\pi$ см$^2$}{$6\pi$ см$^2$}{$144\pi$ см$^2$}{$24\pi$ см$^2$}{$3\pi$ см$^2$}

\vspace{0.4cm}

\gentask{81}{Площа круга $36\pi$ см$^2$. Знайдіть площу сектора з кутом $60°$.}
\answerTable{$6\pi$ см$^2$}{$36\pi$ см$^2$}{$12\pi$ см$^2$}{$3\pi$ см$^2$}{$18\pi$ см$^2$}

\vspace{0.4cm}

\gentask{82}{Площа сегмента дорівнює:}
\answerTable{$S_{\text{сектора}} - S_{\text{трик.}}$}{$S_{\text{круга}}$}{$S_{\text{сектора}}$}{$S_{\text{трик.}}$}{$S_{\text{сектора}} + S_{\text{трик.}}$}

\vspace{0.4cm}

\gentask{83}{Радіус кола 6 см, центральний кут $60°$. Площа трикутника, утвореного радіусами і хордою:}
\answerTable{$9\sqrt{3}$ см$^2$}{$18$ см$^2$}{$9$ см$^2$}{$36$ см$^2$}{$18\sqrt{3}$ см$^2$}

\vspace{0.4cm}

\gentask{84}{Радіус кола 10 см. Знайдіть площу сектора, якщо дуга сектора $= \dfrac{1}{5}$ кола.}
\answerTable{$20\pi$ см$^2$}{$100\pi$ см$^2$}{$50\pi$ см$^2$}{$10\pi$ см$^2$}{$25\pi$ см$^2$}

\vspace{0.4cm}

\gentask{85}{Площа сектора $18\pi$ см$^2$, центральний кут $90°$. Знайдіть радіус.}
\answerTable{$6\sqrt{2}$ см}{6 см}{9 см}{$3\sqrt{2}$ см}{12 см}

\vspace{0.5cm}

%======================================================================
% БЛОК 7: Взаємне розташування кіл (15 завдань)
%======================================================================

\section*{Блок 7: Взаємне розташування кіл}

\gentask{86}{Два кола мають радіуси 5 і 3 см, відстань між центрами 10 см. Скільки спільних точок?}
\answerTable{0}{1}{2}{3}{4}

\vspace{0.4cm}

\gentask{87}{Два кола мають радіуси 4 і 6 см, відстань між центрами 10 см. Скільки спільних точок?}
\answerTable{1 (зовнішній дотик)}{0}{2}{1 (внутрішній)}{3}

\vspace{0.4cm}

\gentask{88}{Два кола мають радіуси 3 і 5 см, відстань між центрами 6 см. Скільки спільних точок?}
\answerTable{2}{0}{1}{3}{4}

\vspace{0.4cm}

\gentask{89}{Радіуси двох кіл 7 і 3 см, відстань між центрами 4 см. Яке взаємне розташування?}
\answerTable{внутрішній дотик}{зовнішній дотик}{перетинаються}{не перетинаються}{одне всередині іншого}

\vspace{0.4cm}

\gentask{90}{Два кола перетинаються в двох точках. Яке співвідношення між $d$, $R$ і $r$?}
\answerTable{$|R-r| < d < R+r$}{$d = R+r$}{$d > R+r$}{$d < |R-r|$}{$d = |R-r|$}

\vspace{0.4cm}

\gentask{91}{Кола дотикаються зовні. Відстань між центрами дорівнює:}
\answerTable{$R + r$}{$R - r$}{$|R - r|$}{$\sqrt{R^2 + r^2}$}{$Rr$}

\vspace{0.4cm}

\gentask{92}{Кола дотикаються внутрішньо. Відстань між центрами дорівнює:}
\answerTable{$|R - r|$}{$R + r$}{$R - r$}{$\sqrt{R^2 - r^2}$}{0}

\vspace{0.4cm}

\gentask{93}{Радіуси кіл 8 і 3 см. При якій відстані між центрами вони дотикаються зовні?}
\answerTable{11 см}{5 см}{24 см}{$\sqrt{73}$ см}{8 см}

\vspace{0.4cm}

\gentask{94}{Радіуси кіл 10 і 4 см. При якій відстані між центрами вони дотикаються внутрішньо?}
\answerTable{6 см}{14 см}{40 см}{$\sqrt{84}$ см}{2 см}

\vspace{0.4cm}

\gentask{95}{Два кола не мають спільних точок. Можливо:}
\answerTable{$d > R+r$ або $d < |R-r|$}{$d = R+r$}{$d = |R-r|$}{$|R-r| < d < R+r$}{тільки $d > R+r$}

\vspace{0.4cm}

\gentask{96}{Два кола з радіусами 6 і 2 см мають спільний центр. Яке взаємне розташування?}
\answerTable{концентричні}{дотикаються}{перетинаються}{зовнішні}{немає такого}

\vspace{0.4cm}

\gentask{97}{Пряма може мати з колом:}
\answerTable{0, 1 або 2 точки}{тільки 2 точки}{тільки 1 точку}{0 або 1 точку}{3 точки}

\vspace{0.4cm}

\gentask{98}{Січна~--- це пряма, яка має з колом:}
\answerTable{2 спільні точки}{1 спільну точку}{0 спільних точок}{3 спільні точки}{безліч точок}

\vspace{0.4cm}

\gentask{99}{Дотична~--- це пряма, яка має з колом:}
\answerTable{1 спільну точку}{2 спільні точки}{0 спільних точок}{3 спільні точки}{безліч точок}

\vspace{0.4cm}

\gentask{100}{Кола з радіусами 5 і 12 см дотикаються зовні. Знайдіть відстань між найвіддаленішими точками кіл.}
\answerTable{34 см}{17 см}{7 см}{24 см}{29 см}

\vspace{0.5cm}

\end{document}
