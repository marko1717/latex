\documentclass[14pt]{extarticle}
\usepackage{fontspec}
\usepackage{polyglossia}
\setdefaultlanguage{ukrainian}

\defaultfontfeatures{Ligatures=TeX}
\setmainfont{Liberation Serif}
\setsansfont{Liberation Sans}
\setmonofont{Liberation Mono}

\usepackage[a4paper,margin=1.5cm,bottom=2cm,top=2cm]{geometry}
\usepackage{amsmath,amssymb}
\usepackage{enumitem}
\usepackage{tikz}
\usepackage{pgfplots}
\pgfplotsset{compat=1.18}

% Підключаємо бібліотеки для зручних кутів
\usetikzlibrary{calc,patterns,angles,quotes,intersections,babel}
\usetikzlibrary{3d}

\usepackage{xcolor}
\usepackage{array}
\usepackage{fancyhdr}
\usepackage{multirow}

% Кольори
\definecolor{headerblue}{RGB}{0, 102, 204}
\definecolor{yearcolor}{RGB}{128, 0, 128}

\pagestyle{fancy}
\fancyhf{}
\renewcommand{\headrulewidth}{0pt}
\fancyfoot[C]{\thepage}

\setlength{\headheight}{15pt}
\setlength{\headsep}{10pt}
\setlength{\footskip}{25pt}

\widowpenalty=10000
\clubpenalty=10000

% === КОМАНДИ ===

% Таблиця для відповідей із дробами (збільшена висота клітинок)
% Оновлена таблиця: підпорка додана до КОЖНОЇ клітинки
\newcommand{\answerTableTall}[5]{
\begin{center}
\begin{tabular}{|*{5}{>{\centering\arraybackslash}m{2.8cm}|}}
\hline
\rule[-0.3cm]{0pt}{0.8cm}\textbf{А} & \textbf{Б} & \textbf{В} & \textbf{Г} & \textbf{Д} \\
\hline
% Тепер rule є перед кожним аргументом (#1..#5)
\rule[-0.9cm]{0pt}{2.0cm}#1 & 
\rule[-0.9cm]{0pt}{2.0cm}#2 & 
\rule[-0.9cm]{0pt}{2.0cm}#3 & 
\rule[-0.9cm]{0pt}{2.0cm}#4 & 
\rule[-0.9cm]{0pt}{2.0cm}#5 \\
\hline
\end{tabular}
\end{center}
}

% Оновлена таблиця відповідей (заголовки зовні)
\newcommand{\answerGrid}{
    \begingroup
    % Збільшуємо висоту рядків для квадратних клітинок
    \renewcommand{\arraystretch}{1.3} 
    % Відступ всередині клітинок
    \setlength{\tabcolsep}{7pt} 
    \begin{tabular}{r|c|c|c|c|c|}
         % Перший рядок: порожня клітинка зліва + букви без рамок (multicolumn прибирає |)
         \multicolumn{1}{c}{} & \multicolumn{1}{c}{\textbf{А}} & \multicolumn{1}{c}{\textbf{Б}} & \multicolumn{1}{c}{\textbf{В}} & \multicolumn{1}{c}{\textbf{Г}} & \multicolumn{1}{c}{\textbf{Д}} \\ \cline{2-6}
         % Наступні рядки: номер зліва (r) + клітинки з рамками (|c|)
         \textbf{1} & & & & & \\ \cline{2-6}
         \textbf{2} & & & & & \\ \cline{2-6}
         \textbf{3} & & & & & \\ \cline{2-6}
    \end{tabular}
    \endgroup
}

% Макет для завдань на відповідність
% #1 - Умови (1-3)
% #2 - Варіанти (А-Д)
% #3 - Табличка
\newcommand{\matchingLayout}[3]{
    \noindent
    \begin{minipage}[t]{0.40\textwidth}
       
        #1
    \end{minipage}%
    \hfill
    \begin{minipage}[t]{0.28\textwidth}
        
        #2
    \end{minipage}%
    \hfill
    \begin{minipage}[t]{0.30\textwidth}
        \vspace{0pt} % Хаки для вирівнювання minipage по верху
        \begin{flushright}
        #3
        \end{flushright}
    \end{minipage}
}

% Стандартна таблиця відповідей (для тестів)
\newcommand{\answerTableSmall}[5]{
\begin{tabular}{|*{5}{>{\centering\arraybackslash}m{1.65cm}|}}
\hline
\rule[-0.2cm]{0pt}{0.6cm}\textbf{А} & \textbf{Б} & \textbf{В} & \textbf{Г} & \textbf{Д} \\
\hline
% Підпорка додана до кожного варіанту для ідеального вирівнювання
\rule[-0.4cm]{0pt}{0.9cm}#1 & 
\rule[-0.4cm]{0pt}{0.9cm}#2 & 
\rule[-0.4cm]{0pt}{0.9cm}#3 & 
\rule[-0.4cm]{0pt}{0.9cm}#4 & 
\rule[-0.4cm]{0pt}{0.9cm}#5 \\
\hline
\end{tabular}
}

% Таблиця для вибору одного варіанту (Task 7)
\newcommand{\answerTable}[5]{
\begin{center}
\begin{tabular}{|*{5}{>{\centering\arraybackslash}m{2.8cm}|}}
\hline
\rule[-0.3cm]{0pt}{0.8cm}\textbf{А} & \textbf{Б} & \textbf{В} & \textbf{Г} & \textbf{Д} \\
\hline
\rule[-0.4cm]{0pt}{1.0cm}#1 & \rule[-0.4cm]{0pt}{1.0cm}#2 & \rule[-0.4cm]{0pt}{1.0cm}#3 & \rule[-0.4cm]{0pt}{1.0cm}#4 & \rule[-0.4cm]{0pt}{1.0cm}#5 \\
\hline
\end{tabular}
\end{center}
}

% Команда для року
\newcommand{\nmtyear}[1]{\hfill{\small\color{yearcolor}(НМТ #1)}}

\begin{document}

\vspace{1cm}

\begin{center}
{\Large\textbf{\color{headerblue}БАЗА ЗАВДАНЬ НМТ 2023}}
\end{center}

\begin{center}
{\large Тема: \textbf{Показникові нерівності}}
\end{center}

% === НМТ 2023 ===

% === ЗАВДАННЯ 1 ===
\noindent\textbf{1.} Яке з наведених чисел є розв'язком нерівності $5^{-x} > 25$? \nmtyear{2023}

\answerTable{$0$}{$3$}{$-1$}{$-3$}{$1$}

% === ЗАВДАННЯ 2 ===
\noindent\textbf{2.} Розв'яжіть нерівність $\left(\displaystyle\frac{1}{5}\right)^x > \displaystyle\frac{1}{25}$. \nmtyear{2023}

\answerTable{$(5; +\infty)$}{$(-\infty; 5)$}{$(2; +\infty)$}{$(0; 2)$}{$(-\infty; 2)$}

% === ЗАВДАННЯ 3 ===
\noindent\textbf{3.} Розв'яжіть нерівність $8^x > \displaystyle\frac{1}{64}$. \nmtyear{2023}

\answerTable{$(-\infty; -2)$}{$(0; 2)$}{$(2; +\infty)$}{$(-2; +\infty)$}{$(-\infty; 2)$}

% === НМТ 2024 ===

% === ЗАВДАННЯ 4 ===
\noindent\textbf{4.} Розв'яжіть нерівність $\left(\displaystyle\frac{1}{3}\right)^{2x-1} \leqslant 27^{-1}$. \nmtyear{2024}

\answerTable{$[2; +\infty)$}{$[-1; +\infty)$}{$(-\infty; 4]$}{$(-\infty; -1]$}{$(-\infty; 2]$}

% === ЗАВДАННЯ 5 ===
\noindent\textbf{5.} Розв'яжіть систему нерівностей $\begin{cases} 5^x < 25, \\ 2 - x < 8. \end{cases}$ \nmtyear{2024}

\answerTable{$(-6; 5)$}{$(2; 6)$}{$(-\infty; -6)$}{$(-6; 2)$}{$(2; +\infty)$}

% === ЗАВДАННЯ 6 ===
\noindent\textbf{6.} Розв'яжіть систему нерівностей $\begin{cases} x^2 + 9 \geqslant 0, \\ 2^x > \displaystyle\frac{1}{16}. \end{cases}$ \nmtyear{2024}

\answerTable{$[-3; +\infty)$}{$(-4; +\infty)$}{$(-4; -3]$}{$(-4; -3] \cup [3; +\infty)$}{$\varnothing$}

% === ЗАВДАННЯ 7 ===
\noindent\textbf{7.} Розв'яжіть нерівність $(0{,}1)^{x+2} < 0{,}1$. \nmtyear{2024}

\answerTable{$(-1; +\infty)$}{$(-\infty; 3)$}{$(-\infty; -1)$}{$(-2; +\infty)$}{$(3; +\infty)$}

% === ЗАВДАННЯ 8 ===
\noindent\textbf{8.} Розв'яжіть нерівність $5^x \leqslant 1$. \nmtyear{2024}

\answerTableTall{$[1; +\infty)$}{$(-\infty; 1]$}{$\left(-\infty; \displaystyle\frac{1}{5}\right)$}{$(-\infty; 0]$}{$[0; +\infty)$}

% === НМТ 2025 ===

% === ЗАВДАННЯ 9 ===
\noindent\textbf{9.} Розв'яжіть нерівність $7^x \leqslant \displaystyle\frac{1}{49}$. \nmtyear{2025}

\answerTableTall{$\left(-\infty; \displaystyle\frac{1}{243}\right]$}{$[-2; +\infty)$}{$\left[\displaystyle\frac{1}{7}; +\infty\right)$}{$(-\infty; -2]$}{$\left(-\infty; \displaystyle\frac{1}{2}\right]$}

% === ЗАВДАННЯ 10 ===
\noindent\textbf{10.} Розв'яжіть нерівність $\left(\displaystyle\frac{1}{4}\right)^{2x-6} > \displaystyle\frac{1}{16}$. \nmtyear{2025}

\answerTable{$(-\infty; -2)$}{$(-\infty; 2)$}{$(-2; +\infty)$}{$(4; +\infty)$}{$(-\infty; 4)$}

% === ЗАВДАННЯ 11 ===
\noindent\textbf{11.} Розв'яжіть нерівність $3^x \cdot 5^x \leqslant \displaystyle\frac{1}{15}$. \nmtyear{2025}

\answerTableTall{$(-\infty; 1]$}{$[1; +\infty)$}{$\left(-\infty; \displaystyle\frac{1}{15}\right]$}{$[-1; +\infty)$}{$(-\infty; -1]$}

% === ЗАВДАННЯ 12 ===
\noindent\textbf{12.} Розв'яжіть нерівність $2^{3-x} > \displaystyle\frac{1}{4}$. \nmtyear{2025}

\answerTable{$(-\infty; 5)$}{$(-5; +\infty)$}{$(-1; +\infty)$}{$(5; +\infty)$}{$(-\infty; -1)$}

% === ЗАВДАННЯ 13 ===
\noindent\textbf{13.} Розв'яжіть нерівність $10^{x-2} > 1$. \nmtyear{2025}

\answerTable{$(3; +\infty)$}{$(-2; +\infty)$}{$(-\infty; 2)$}{$(2{,}1; +\infty)$}{$(2; +\infty)$}

\end{document}