\documentclass[14pt]{extarticle}
\usepackage{fontspec}
\usepackage{polyglossia}
\setdefaultlanguage{ukrainian}

\defaultfontfeatures{Ligatures=TeX}
\setmainfont{Liberation Serif}
\setsansfont{Liberation Sans}
\setmonofont{Liberation Mono}

\usepackage[a4paper,margin=1.5cm,bottom=2cm,top=2cm]{geometry}
\usepackage{amsmath,amssymb}
\usepackage{enumitem}
\usepackage{tikz}
\usepackage{pgfplots}
\pgfplotsset{compat=1.16}

% Підключаємо бібліотеки для зручних кутів
\usetikzlibrary{calc,patterns,angles,quotes,intersections,babel}

\usepackage{xcolor}
\usepackage{array}
\usepackage{fancyhdr}
\usepackage{multirow}

% Кольори
\definecolor{headerblue}{RGB}{0, 102, 204}
\definecolor{yearcolor}{RGB}{128, 0, 128}

\pagestyle{fancy}
\fancyhf{}
\renewcommand{\headrulewidth}{0pt}
\fancyfoot[C]{\thepage}

\setlength{\headheight}{15pt}
\setlength{\headsep}{10pt}
\setlength{\footskip}{25pt}

\widowpenalty=10000
\clubpenalty=10000

% === КОМАНДИ ===

% Таблиця для відповідей із дробами (збільшена висота клітинок)
\newcommand{\answerTableTall}[5]{
\begin{center}
\begin{tabular}{|*{5}{>{\centering\arraybackslash}m{2.8cm}|}}
\hline
\rule[-0.3cm]{0pt}{0.8cm}\textbf{А} & \textbf{Б} & \textbf{В} & \textbf{Г} & \textbf{Д} \\
\hline
\rule[-0.9cm]{0pt}{2.0cm}#1 &
\rule[-0.9cm]{0pt}{2.0cm}#2 &
\rule[-0.9cm]{0pt}{2.0cm}#3 &
\rule[-0.9cm]{0pt}{2.0cm}#4 &
\rule[-0.9cm]{0pt}{2.0cm}#5 \\
\hline
\end{tabular}
\end{center}
}

% Оновлена таблиця відповідей (заголовки зовні)
\newcommand{\answerGrid}{
    \begingroup
    \renewcommand{\arraystretch}{1.3}
    \setlength{\tabcolsep}{7pt}
    \begin{tabular}{r|c|c|c|c|c|}
         \multicolumn{1}{c}{} & \multicolumn{1}{c}{\textbf{А}} & \multicolumn{1}{c}{\textbf{Б}} & \multicolumn{1}{c}{\textbf{В}} & \multicolumn{1}{c}{\textbf{Г}} & \multicolumn{1}{c}{\textbf{Д}} \\ \cline{2-6}
         \textbf{1} & & & & & \\ \cline{2-6}
         \textbf{2} & & & & & \\ \cline{2-6}
         \textbf{3} & & & & & \\ \cline{2-6}
    \end{tabular}
    \endgroup
}

% Макет для завдань на відповідність
\newcommand{\matchingLayout}[3]{
    \noindent
    \begin{minipage}[t]{0.40\textwidth}
        \textit{Величина / Початок} \par \vspace{0.2cm}
        #1
    \end{minipage}%
    \hfill
    \begin{minipage}[t]{0.28\textwidth}
        \textit{Значення / Закінчення} \par \vspace{0.2cm}
        #2
    \end{minipage}%
    \hfill
    \begin{minipage}[t]{0.30\textwidth}
        \vspace{0pt}
        \begin{flushright}
        #3
        \end{flushright}
    \end{minipage}
}

% Стандартна таблиця відповідей (для тестів)
\newcommand{\answerTableSmall}[5]{
\begin{tabular}{|*{5}{>{\centering\arraybackslash}m{1.65cm}|}}
\hline
\rule[-0.2cm]{0pt}{0.6cm}\textbf{А} & \textbf{Б} & \textbf{В} & \textbf{Г} & \textbf{Д} \\
\hline
\rule[-0.4cm]{0pt}{0.9cm}#1 &
\rule[-0.4cm]{0pt}{0.9cm}#2 &
\rule[-0.4cm]{0pt}{0.9cm}#3 &
\rule[-0.4cm]{0pt}{0.9cm}#4 &
\rule[-0.4cm]{0pt}{0.9cm}#5 \\
\hline
\end{tabular}
}

% Таблиця для вибору одного варіанту
\newcommand{\answerTable}[5]{
\begin{center}
\begin{tabular}{|*{5}{>{\centering\arraybackslash}m{2.8cm}|}}
\hline
\rule[-0.3cm]{0pt}{0.8cm}\textbf{А} & \textbf{Б} & \textbf{В} & \textbf{Г} & \textbf{Д} \\
\hline
\rule[-0.4cm]{0pt}{1.0cm}#1 & \rule[-0.4cm]{0pt}{1.0cm}#2 & \rule[-0.4cm]{0pt}{1.0cm}#3 & \rule[-0.4cm]{0pt}{1.0cm}#4 & \rule[-0.4cm]{0pt}{1.0cm}#5 \\
\hline
\end{tabular}
\end{center}
}

% Команда для року
\newcommand{\nmtyear}[1]{\hfill{\small\color{yearcolor}(НМТ #1)}}

\begin{document}

\begin{center}
{\Large\textbf{\color{headerblue}БАЗА ЗАВДАНЬ НМТ}}
\end{center}

\begin{center}
{\large Тема: \textbf{Паралелограм}}
\end{center}

\vspace{0.5cm}

% === ЗАВДАННЯ 1 ===
\noindent\textbf{1.} \begin{minipage}[t]{0.55\textwidth}
У паралелограмі $ABCD$ діагональ $BD$ утворює зі сторонами $BC$ і $CD$ кути $48^\circ$ і $72^\circ$ (див. рисунок). Визначте градусну міру кута $ABC$. \nmtyear{2024}
\end{minipage}
\hfill
\begin{minipage}[t]{0.4\textwidth}
    \vspace{-0.5cm}
    \begin{flushright}
    \begin{tikzpicture}[scale=0.9]
        \coordinate (A) at (0,0);
        \coordinate (D) at (4,0);
        \coordinate (C) at (5.2,2.5);
        \coordinate (B) at (1.2,2.5);

        \draw[thick] (A) -- (B) -- (C) -- (D) -- cycle;
        \draw[thick] (B) -- (D);

        \pic [draw, pic text={$48^\circ$}, angle radius=0.6cm, angle eccentricity=1.7] {angle = D--B--C};
        \pic [draw, pic text={$72^\circ$}, angle radius=0.6cm, angle eccentricity=1.4] {angle = C--D--B};
        \pic [draw, angle radius=0.5cm] {angle = C--D--B};

        \node[below left] at (A) {$A$};
        \node[above left] at (B) {$B$};
        \node[above right] at (C) {$C$};
        \node[below right] at (D) {$D$};
    \end{tikzpicture}
    \end{flushright}
\end{minipage}

\vspace{0.2cm}
\answerTable{$132^\circ$}{$60^\circ$}{$108^\circ$}{$72^\circ$}{$120^\circ$}

\vspace{0.7cm}

% === ЗАВДАННЯ 2 ===
\noindent\makebox[1.5em][l]{\textbf{2.}}\parbox[t]{\dimexpr\textwidth-1.5em}{У прямокутній системі координат на площині задано паралелограм $ABCD$ (див. рисунок). Обчисліть площу цього паралелограма. \nmtyear{2023}}

\vspace{0.3cm}
\begin{minipage}{0.42\textwidth}
\answerTableSmall{$10{,}5$}{$21$}{$18$}{$3\sqrt{85}$}{$1{,}5\sqrt{85}$}
\end{minipage}
\hfill
\begin{minipage}{0.52\textwidth}
\begin{flushright}
\begin{tikzpicture}[scale=0.45]
    \draw[gray!40, very thin] (-5,-5) grid (5,5);
    \draw[->] (-5,0) -- (5,0) node[right] {$x$};
    \draw[->] (0,-5) -- (0,5) node[above] {$y$};
    \node[below left] at (0,0) {$0$};
    \node[below] at (1,0) {$1$};
    \node[left] at (0,1) {$1$};

    \coordinate (A) at (-3,1);
    \coordinate (B) at (-3,4);
    \coordinate (C) at (4,-2);
    \coordinate (D) at (4,-5);

    \fill[gray!40, opacity=0.5] (A) -- (B) -- (C) -- (D) -- cycle;
    \draw[thick] (A) -- (B) -- (C) -- (D) -- cycle;

    \node[left] at (A) {$A$};
    \node[above] at (B) {$B$};
    \node[right] at (C) {$C$};
    \node[below] at (D) {$D$};
\end{tikzpicture}
\end{flushright}
\end{minipage}

\vspace{0.7cm}

% === ЗАВДАННЯ 3 ===
\noindent\makebox[1.5em][l]{\textbf{3.}}\parbox[t]{\dimexpr\textwidth-1.5em}{Сторона $CD$ паралелограма $ABCD$ утворює з прямою $AD$ кут $65°$ (див. рисунок). Знайдіть градусну міру кута $MAB$. \nmtyear{2023}}

\vspace{0.3cm}
\begin{minipage}{0.42\textwidth}
\answerTableSmall{$145°$}{$65°$}{$125°$}{$115°$}{$135°$}
\end{minipage}
\hfill
\begin{minipage}{0.52\textwidth}
\begin{flushright}
\begin{tikzpicture}[scale=1]
    \coordinate (M) at (-0.8,0);
    \coordinate (A) at (0,0);
    \coordinate (D) at (3,0);
    \coordinate (E) at (3.8,0);
    \coordinate (B) at (0.7,1.3);
    \coordinate (C) at (3.7,1.3);

    \draw[thick] (A) -- (B) -- (C) -- (D) -- cycle;
    \draw[thick] (M) -- (A);
    \draw[thick] (D) -- (E);

    \pic [draw, pic text={$65^\circ$}, angle radius=0.5cm, angle eccentricity=1.9] {angle = E--D--C};

    \pic [draw, angle radius=0.4cm] {angle = B--A--M};
    \pic [draw, angle radius=0.5cm, "?" anchor=south east] {angle = B--A--M};

    \node[below] at (M) {$M$};
    \node[below] at (A) {$A$};
    \node[below] at (D) {$D$};
    \node[above] at (B) {$B$};
    \node[above] at (C) {$C$};
    \fill (M) circle (1.5pt);
\end{tikzpicture}
\end{flushright}
\end{minipage}

\vspace{0.7cm}

% === ЗАВДАННЯ 4 ===
\noindent\textbf{4.} \begin{minipage}[t]{0.55\textwidth}
Діагональ $BD$ паралелограма $ABCD$ перпендикулярна до сторони $AB$ (див. рисунок). $\angle A = 45°$, $BD = 10$ \textit{см}. До кожного початку речення (1--3) доберіть його закінчення (А--Д). \nmtyear{2023}
\end{minipage}
\hfill
\begin{minipage}[t]{0.4\textwidth}
    \vspace{-0.5cm}
    \begin{flushright}
    \begin{tikzpicture}[scale=0.8]
        \coordinate (A) at (0,0);
        \coordinate (B) at (1.5,2.1);
        \coordinate (C) at (5.5,2.1);
        \coordinate (D) at (4,0);

        \draw[thick] (A) -- (B) -- (C) -- (D) -- cycle;
        \draw[thick] (B) -- (D);

        \coordinate (BA) at ($(B)!0.25cm!(A)$);
        \coordinate (BD) at ($(B)!0.25cm!(D)$);
        \draw (BA) -- ($(BA)+(BD)-(B)$) -- (BD);

        \pic [draw, angle radius=0.6cm, " $45°$" anchor=south west] {angle = D--A--B};

        \node[below left] at (A) {$A$};
        \node[above] at (B) {$B$};
        \node[above right] at (C) {$C$};
        \node[below right] at (D) {$D$};
    \end{tikzpicture}
    \end{flushright}
\end{minipage}

\vspace{0.3cm}

\matchingLayout{
    \textbf{1} \quad Сторона $AB$ \\
    \textbf{2} \quad Сторона $AD$ \\
    \textbf{3} \quad Діагональ $AC$
}{
    \begin{tabular}{ll}
    \textbf{А} & $5\sqrt{2}$ \textit{см} \\
    \textbf{Б} & $10\sqrt{2}$ \textit{см} \\
    \textbf{В} & $5\sqrt{6}$ \textit{см} \\
    \textbf{Г} & $10$ \textit{см} \\
    \textbf{Д} & $20$ \textit{см} \\
    \end{tabular}
}{
    \answerGrid
}

\vspace{0.7cm}

% === ЗАВДАННЯ 5 ===
\noindent\makebox[1.5em][l]{\textbf{5.}}\parbox[t]{\dimexpr\textwidth-1.5em}{У паралелограмі $ABCD$ діагональ $BD$ утворює зі сторонами $AB$ і $AD$ кути $50°$ і $55°$ відповідно (див. рисунок). Знайдіть довжину сторони $BC$, якщо $AB = 4$ \textit{см}. \nmtyear{2023}}

\vspace{0.3cm}
\begin{minipage}{0.42\textwidth}
\answerTableSmall{$2\sqrt{5}$ \textit{см}}{$\dfrac{4\sin 55°}{\sin 50°}$ \textit{см}}{$4\sqrt{2}$ \textit{см}}{$\sqrt{10}$ \textit{см}}{$\dfrac{4\sin 50°}{\sin 55°}$ \textit{см}}
\end{minipage}
\hfill
\begin{minipage}{0.52\textwidth}
\begin{flushright}
\begin{tikzpicture}[scale=1.2]
    \coordinate (A) at (0,0);
    \coordinate (B) at (1.2,2);
    \coordinate (C) at (4.5,2);
    \coordinate (D) at (3.3,0);

    \draw[thick] (A) -- (B) -- (C) -- (D) -- cycle;
    \draw[thick] (B) -- (D);

    \pic [draw, angle radius=0.4cm, "\small $50°$" anchor=north] {angle = A--B--D};

    \pic [draw, angle radius=0.4cm] {angle = B--D--A};
    \pic [draw, angle radius=0.5cm, "\small $55°$" anchor=east] {angle = B--D--A};

    \node[left] at (0.6,1) {\small 4 \textit{см}};

    \node[below left] at (A) {$A$};
    \node[above] at (B) {$B$};
    \node[above right] at (C) {$C$};
    \node[below right] at (D) {$D$};
\end{tikzpicture}
\end{flushright}
\end{minipage}

\vspace{0.7cm}

% === ЗАВДАННЯ 6 ===
\noindent\makebox[1.5em][l]{\textbf{6.}}\parbox[t]{\dimexpr\textwidth-1.5em}{У паралелограмі $ABCD$ бісектриса кута $A = 45°$ перетинає сторону $BC$ в точці $K$, $BK = 6$ \textit{см}, $KC = 4$ \textit{см} (див. рисунок). Обчисліть площу паралелограма $ABCD$. \nmtyear{2023}}

\vspace{0.3cm}
\begin{minipage}{0.42\textwidth}
\answerTableSmall{$30\sqrt{2}$ \textit{см}$^2$}{$60$ \textit{см}$^2$}{$36$ \textit{см}$^2$}{$30$ \textit{см}$^2$}{$60\sqrt{2}$ \textit{см}$^2$}
\end{minipage}
\hfill
\begin{minipage}{0.52\textwidth}
\begin{flushright}
\begin{tikzpicture}[scale=0.75]
    \coordinate (A) at (0,0);
    \coordinate (B) at (1.2,2.1);
    \coordinate (C) at (5.2,2.1);
    \coordinate (D) at (4,0);
    \coordinate (K) at (3.3,2.1);

    \draw[thick] (A) -- (B) -- (C) -- (D) -- cycle;
    \draw[thick] (A) -- (K);

    \pic [draw, angle radius=0.5cm] {angle = D--A--K};
    \pic [draw, angle radius=0.65cm] {angle = K--A--B};

    \node[below left] at (A) {$A$};
    \node[above left] at (B) {$B$};
    \node[above right] at (C) {$C$};
    \node[below right] at (D) {$D$};
    \node[above] at (K) {$K$};
\end{tikzpicture}
\end{flushright}
\end{minipage}

\vspace{0.7cm}

% === ЗАВДАННЯ 7 ===
\noindent\makebox[1.5em][l]{\textbf{7.}}\parbox[t]{\dimexpr\textwidth-1.5em}{Які з наведених тверджень є правильними? \nmtyear{2023}}

\vspace{0.2cm}
\begin{tabular}{r@{\hspace{0.5em}}p{14cm}}
I. & Діагональ паралелограма ділить його на два рівних трикутники. \\
II. & Діагоналі паралелограма є бісектрисами його кутів. \\
III. & Більша діагональ паралелограма лежить проти більшого кута. \\
\end{tabular}

\answerTable{I, II та III}{лише I та III}{лише I та II}{лише II}{лише I}

\vspace{0.5cm}

% === ЗАВДАННЯ 8 ===
\noindent\makebox[1.5em][l]{\textbf{8.}}\parbox[t]{\dimexpr\textwidth-1.5em}{У паралелограмі $ABCD$ на стороні $AD$ вибрано точку $K$. Діагональ $AC$ і відрізок $BK$ перетинаються в точці $O$ (див. рисунок). Визначте довжину сторони $BC$, якщо $AK = 15$ \textit{см}, $OK = 3$ \textit{см}, $OB = 5$ \textit{см}. \nmtyear{2023}}

\vspace{0.3cm}
\begin{minipage}{0.42\textwidth}
\answerTableSmall{$12$ \textit{см}}{$18$ \textit{см}}{$9$ \textit{см}}{$20$ \textit{см}}{$25$ \textit{см}}
\end{minipage}
\hfill
\begin{minipage}{0.52\textwidth}
\begin{flushright}
\begin{tikzpicture}[scale=0.7]
    \coordinate (A) at (0,0);
    \coordinate (B) at (1.2,2.2);
    \coordinate (C) at (5,2.2);
    \coordinate (D) at (3.8,0);
    \coordinate (K) at (2.5,0);

    \draw[thick] (A) -- (B) -- (C) -- (D) -- cycle;

    \path[name path=AC] (A) -- (C);
    \path[name path=BK] (B) -- (K);
    \path [name intersections={of=AC and BK, by=O}];

    \draw[thick] (A) -- (C);
    \draw[thick] (B) -- (K);

    \node[below left] at (A) {$A$};
    \node[above left] at (B) {$B$};
    \node[above right] at (C) {$C$};
    \node[below right] at (D) {$D$};
    \node[below] at (K) {$K$};
    \node[above right] at (O) {$O$};
    \fill (O) circle (1.5pt);
\end{tikzpicture}
\end{flushright}
\end{minipage}

\vspace{0.7cm}

% === ЗАВДАННЯ 9 ===
\noindent\textbf{9.} \begin{minipage}[t]{0.55\textwidth}
У паралелограмі $ABCD$ з гострим кутом $\angle A = \alpha$ проведено висоту $BK = 10$ (див. рисунок). Визначте площу паралелограма $ABCD$, якщо $BD = 26$. \nmtyear{2024}
\end{minipage}
\hfill
\begin{minipage}[t]{0.4\textwidth}
    \vspace{-0.5cm}
    \begin{flushright}
    \begin{tikzpicture}[scale=0.2]
        \coordinate (A) at (0,0);
        \coordinate (K) at (5,0);
        \coordinate (B) at (5,10);

        \coordinate (D) at (29,0);
        \coordinate (C) at ($(D)+(B)-(A)$);

        \draw[thick] (A) -- (B) -- (C) -- (D) -- cycle;
        \draw[thick] (B) -- (K);
        \draw[thick] (B) -- (D);

        \draw (K) ++(-0.8,0) -- ++(0,0.8) -- ++(0.8,0);

        \pic [draw, pic text={\small $\alpha$}, angle radius=0.5cm, angle eccentricity=1.5] {angle = K--A--B};

        \node[below left] at (A) {$A$};
        \node[above left] at (B) {$B$};
        \node[above right] at (C) {$C$};
        \node[below right] at (D) {$D$};
        \node[below] at (K) {$K$};
    \end{tikzpicture}
    \end{flushright}
\end{minipage}

\vspace{0.2cm}
\answerTableTall{$100\tg\alpha+240$}{$\dfrac{50}{\tg\alpha}+120$}{$\dfrac{100}{\tg\alpha}+240$}{$200\cos\alpha+240$}{$50\tg\alpha+120$}

\vspace{0.7cm}

% === ЗАВДАННЯ 10 ===
\noindent\makebox[1.5em][l]{\textbf{10.}}\parbox[t]{\dimexpr\textwidth-1.5em}{Які з наведених тверджень є правильними? \nmtyear{2024}}

\vspace{0.2cm}
\begin{tabular}{r@{\hspace{0.5em}}p{14cm}}
I. & Будь-який ромб є паралелограмом. \\
II. & Центр вписаного в будь-який ромб кола лежить на перетині бісектрис його кутів. \\
III. & Менша діагональ будь-якого ромба ділить його на 2 правильні трикутники. \\
\end{tabular}

\vspace{0.3cm}
\answerTable{лише II}{лише I}{лише III}{I, II та III}{лише I та II}

\vspace{0.7cm}

% === ЗАВДАННЯ 11 ===
\noindent\textbf{11.} \begin{minipage}[t]{0.55\textwidth}
У паралелограмі $ABCD$ бісектриса кута $A$ перетинає сторону $BC$ в точці $K$ так, що $BK : KC = 5 : 2$ (див. рисунок). $AK = c$, $\angle KAD = \beta$. Знайдіть периметр цього паралелограма. \nmtyear{2024}
\end{minipage}
\hfill
\begin{minipage}[t]{0.4\textwidth}
    \vspace{-0.5cm}
    \begin{flushright}
    \begin{tikzpicture}[scale=0.8]
        \coordinate (A) at (0,0);
        \coordinate (B) at (1.5,2.5);
        \coordinate (K) at (4,2.5);
        \coordinate (C) at (5.5,2.5);
        \coordinate (D) at (4,0);

        \draw[thick] (A) -- (B) -- (C) -- (D) -- cycle;
        \draw[thick] (A) -- (K);

        \pic [draw, angle radius=0.5cm] {angle = D--A--K};
        \pic [draw, angle radius=0.6cm] {angle = K--A--B};

        \node[below left] at (A) {$A$};
        \node[above left] at (B) {$B$};
        \node[above] at (K) {$K$};
        \node[above right] at (C) {$C$};
        \node[below right] at (D) {$D$};

        \fill (K) circle (1.5pt);
    \end{tikzpicture}
    \end{flushright}
\end{minipage}

\vspace{0.2cm}
\answerTableTall{$\dfrac{5c}{\sin\beta}$}{$\dfrac{7c}{5\cos\beta}$}{$\dfrac{12c}{7\cos\beta}$}{$\dfrac{7c}{12\tg\beta}$}{$\dfrac{14c}{5\cos\beta}$}

\vspace{0.7cm}

% === ЗАВДАННЯ 12 ===
\noindent\textbf{12.} \begin{minipage}[t]{0.55\textwidth}
У паралелограмі $ABCD$ з гострим кутом $\angle A = 2\gamma$ на стороні $AD$ вибрано точку $K$ так, що $AB = AK = KD$ (див. рисунок). Визначте периметр паралелограма $ABCD$, якщо $BK = m$. \nmtyear{2024}
\end{minipage}
\hfill
\begin{minipage}[t]{0.4\textwidth}
    \vspace{-0.5cm}
    \begin{flushright}
    \begin{tikzpicture}[scale=0.9]
        \coordinate (A) at (0,0);
        \coordinate (K) at (2,0);
        \coordinate (D) at (4,0);

        \coordinate (B) at (60:2);
        \coordinate (C) at ($(D)+(B)-(A)$);

        \draw[thick] (A) -- (B) -- (C) -- (D) -- cycle;
        \draw[thick] (B) -- (K) node[midway, right] {$m$};

        \draw ($(A)!0.5!(B)$) ++(150:0.1) -- ++(-30:0.2);
        \draw ($(A)!0.5!(K)$) ++(90:0.1) -- ++(-90:0.2);
        \draw ($(K)!0.5!(D)$) ++(90:0.1) -- ++(-90:0.2);

        \pic [draw, pic text={\small $2\gamma$}, angle radius=0.5cm, angle eccentricity=1.7] {angle = K--A--B};

        \node[below left] at (A) {$A$};
        \node[above left] at (B) {$B$};
        \node[above right] at (C) {$C$};
        \node[below right] at (D) {$D$};
        \node[below] at (K) {$K$};
        \fill (K) circle (1.5pt);
    \end{tikzpicture}
    \end{flushright}
\end{minipage}

\vspace{0.2cm}
\answerTableTall{$\dfrac{3m}{\sin\gamma}$}{$\dfrac{6m}{\sin\gamma}$}{$12m\cos\gamma$}{$6m\sin\gamma$}{$\dfrac{3m}{\cos\gamma}$}

\vspace{0.7cm}

% === ЗАВДАННЯ 13 ===
\noindent\makebox[1.5em][l]{\textbf{13.}}\parbox[t]{\dimexpr\textwidth-1.5em}{Які з наведених тверджень є правильними? \nmtyear{2024}}

\vspace{0.2cm}
\begin{tabular}{r@{\hspace{0.5em}}p{14cm}}
I. & Існує паралелограм, діагональ якого дорівнює сумі його сусідніх сторін. \\
II. & Існує паралелограм, один із кутів якого втричі більший за інший кут. \\
III. & Існує паралелограм, діагоналі якого перпендикулярні. \\
\end{tabular}

\vspace{0.3cm}
\answerTable{лише II та III}{лише I та II}{I, II та III}{лише III}{лише I та III}

\vspace{0.7cm}

% === ЗАВДАННЯ 14 ===
\noindent\textbf{14.} \begin{minipage}[t]{0.55\textwidth}
Діагоналі $AC$ і $BD$ паралелограма $ABCD$ перетинаються в точці $O$ (див. рисунок). З точки $O$ на сторону $AD$ опущено перпендикуляр $OK = 9$ \textit{см}, $AK = 16$ \textit{см}, $KD = 12$ \textit{см}. До кожного відрізка (1--3) доберіть його довжину (А--Д). \nmtyear{2024}
\end{minipage}
\hfill
\begin{minipage}[t]{0.4\textwidth}
    \vspace{-0.5cm}
    \begin{flushright}
    \begin{tikzpicture}[scale=0.4]
        \coordinate (A) at (0,0);
        \coordinate (K) at (4,0);
        \coordinate (D) at (7,0);
        \coordinate (O) at (4, 2.25);

        \coordinate (C) at ($(O)!-1!(A)$);
        \coordinate (B) at ($(O)!-1!(D)$);

        \draw[thick] (A) -- (B) -- (C) -- (D) -- cycle;
        \draw[thick] (A) -- (C);
        \draw[thick] (B) -- (D);
        \draw[thick] (O) -- (K);

        \draw (K) ++(-0.4,0) -- ++(0,0.4) -- ++(0.4,0);

        \node[below left] at (A) {$A$};
        \node[above left] at (B) {$B$};
        \node[above right] at (C) {$C$};
        \node[below right] at (D) {$D$};
        \node[above] at (O) {$O$};
        \node[below] at (K) {$K$};
    \end{tikzpicture}
    \end{flushright}
\end{minipage}

\vspace{0.3cm}

\matchingLayout{
    \textbf{1} \quad Висота, проведена до $AD$ \\
    \textbf{2} \quad Проєкція $AB$ на $AD$ \\
    \textbf{3} \quad $AB$
}{
    \begin{tabular}{ll}
    \textbf{А} & 4 \textit{см} \\
    \textbf{Б} & 15 \textit{см} \\
    \textbf{В} & 18 \textit{см} \\
    \textbf{Г} & 20 \textit{см} \\
    \textbf{Д} & 25 \textit{см} \\
    \end{tabular}
}{
    \answerGrid
}

\vspace{0.7cm}

% === ЗАВДАННЯ 15 ===
\noindent\textbf{15.} \begin{minipage}[t]{0.55\textwidth}
У паралелограмі $ABCD$ на середині сторони $BC$ вибрано точку $K$ так, що $AK \perp KD$, $KP$ --- висота паралелограма (див. рисунок). $AD = 26$ \textit{см}, $AK = 13$ \textit{см}. До кожного відрізка (1--3) доберіть його довжину (А--Д). \nmtyear{2023}
\end{minipage}
\hfill
\begin{minipage}[t]{0.4\textwidth}
    \vspace{-0.5cm}
    \begin{flushright}
    \begin{tikzpicture}[scale=0.5]
        \coordinate (A) at (0,0);
        \coordinate (B) at (1.5,3);
        \coordinate (C) at (7,3);
        \coordinate (D) at (5.5,0);
        \coordinate (K) at (4.25,3);
        \coordinate (P) at (4.25,0);

        \draw[thick] (A) -- (B) -- (C) -- (D) -- cycle;
        \draw[thick] (A) -- (K);
        \draw[thick] (K) -- (D);
        \draw[thick] (K) -- (P);

        \pic [draw, angle radius=0.3cm] {right angle = A--K--D};
        \pic [draw, angle radius=0.2cm] {right angle = K--P--D};

        \draw (2.7,3.15) -- (2.7,2.85);
        \draw (2.9,3.15) -- (2.9,2.85);
        \draw (5.5,3.15) -- (5.5,2.85);
        \draw (5.7,3.15) -- (5.7,2.85);

        \node[below left] at (A) {$A$};
        \node[above left] at (B) {$B$};
        \node[above right] at (C) {$C$};
        \node[below right] at (D) {$D$};
        \node[above] at (K) {$K$};
        \node[below] at (P) {$P$};
    \end{tikzpicture}
    \end{flushright}
\end{minipage}

\vspace{0.3cm}

\matchingLayout{
    \textbf{1} \quad $KC$ \\
    \textbf{2} \quad $KD$ \\
    \textbf{3} \quad $KP$
}{
    \begin{tabular}{ll}
    \textbf{А} & $5$ \textit{см} \\
    \textbf{Б} & $12$ \textit{см} \\
    \textbf{В} & $13$ \textit{см} \\
    \textbf{Г} & $\sqrt{145}$ \textit{см} \\
    \textbf{Д} & $\sqrt{194}$ \textit{см} \\
    \end{tabular}
}{
    \answerGrid
}

\vspace{0.7cm}

% === ЗАВДАННЯ 16 ===
\noindent\textbf{16.} \begin{minipage}[t]{0.55\textwidth}
На рисунку зображено паралелограм $ABCD$. Які з наведених тверджень є правильними? \nmtyear{2024}

\vspace{0.2cm}
I. \quad $\angle A = \angle C$. \\
II. \quad $AB + BC = CD + AD$. \\
III. \quad $AC = BD$.
\end{minipage}
\hfill
\begin{minipage}[t]{0.4\textwidth}
    \vspace{-0.5cm}
    \begin{flushright}
    \begin{tikzpicture}[scale=0.7]
        \draw[thick] (0,0) node[below left]{$A$} -- (1,2) node[above left]{$B$} -- (5,2) node[above right]{$C$} -- (4,0) node[below right]{$D$} -- cycle;
    \end{tikzpicture}
    \end{flushright}
\end{minipage}

\vspace{0.2cm}
\answerTable{лише III}{лише I та II}{лише I}{лише II}{I, II та III}

\vspace{0.7cm}

% === ЗАВДАННЯ 17 ===
\noindent\textbf{17.} \begin{minipage}[t]{0.55\textwidth}
У паралелограмі $ABCD$ на стороні $AD$ вибрано точку $K$ так, що $BK = CD$, $AK : KD = 2 : 3$ (див. рисунок). $BC = 15$, $\angle AKB = \varphi$. Знайдіть площу цього паралелограма. \nmtyear{2024}
\end{minipage}
\hfill
\begin{minipage}[t]{0.4\textwidth}
    \vspace{-0.5cm}
    \begin{flushright}
    \begin{tikzpicture}[scale=0.8]
        \coordinate (A) at (0,0);
        \coordinate (B) at (1.5,2.5);
        \coordinate (C) at (6.5,2.5);
        \coordinate (D) at (5,0);
        \coordinate (K) at (2,0);

        \draw[thick] (A) -- (B) -- (C) -- (D) -- cycle;
        \draw[thick] (B) -- (K);

        \pic [draw, pic text={$\varphi$}, angle radius=0.4cm, angle eccentricity=1.5] {angle = B--K--A};

        \draw ($(B)!0.5!(K)$) ++(-180:0.1) -- ++(0:0.2);
        \draw ($(C)!0.5!(D)$) ++(180:0.1) -- ++(0:0.2);

        \node[below left] at (A) {$A$};
        \node[above left] at (B) {$B$};
        \node[above right] at (C) {$C$};
        \node[below right] at (D) {$D$};
        \node[below] at (K) {$K$};
        \fill (K) circle (1.5pt);
    \end{tikzpicture}
    \end{flushright}
\end{minipage}

\vspace{0.2cm}
\answerTableTall{$45\tg\varphi$}{$90\cos\varphi$}{$\dfrac{45}{\tg\varphi}$}{$90\tg\varphi$}{$\dfrac{90}{\tg\varphi}$}

\vspace{0.7cm}

% === ЗАВДАННЯ 18 ===
\noindent\makebox[1.5em][l]{\textbf{18.}}\parbox[t]{\dimexpr\textwidth-1.5em}{Які з наведених тверджень є правильними? \nmtyear{2025}}

\vspace{0.2cm}
\begin{tabular}{r@{\hspace{0.5em}}p{14cm}}
I. & Якщо паралелограми мають рівні сторони, то вони мають рівні периметри. \\
II. & Якщо паралелограми мають рівні сторони, то вони мають рівну площу. \\
III. & Якщо прямокутники мають рівні діагоналі, то вони мають рівні сторони. \\
\end{tabular}

\vspace{0.3cm}
\answerTable{лише I та III}{лише I та II}{лише I}{лише III}{лише II}

\vspace{0.7cm}

% === ЗАВДАННЯ 19 ===
\noindent\textbf{19.} \begin{minipage}[t]{0.55\textwidth}
У паралелограмі $ABCD$ діагональ $AC$ утворює зі сторонами $AD$ і $CD$ кути $25^\circ$ і $35^\circ$ (див. рисунок). Визначте градусну міру гострого кута паралелограма. \nmtyear{2025}
\end{minipage}
\hfill
\begin{minipage}[t]{0.4\textwidth}
    \vspace{-0.5cm}
    \begin{flushright}
    \begin{tikzpicture}[scale=1]
        \coordinate (A) at (0,0);
        \coordinate (D) at (3,0);

        \coordinate (C) at ($(D) + (120:1.8)$);
        \coordinate (B) at ($(A) + (C) - (D)$);

        \draw[thick] (A) -- (B) -- (C) -- (D) -- cycle;
        \draw[thick] (A) -- (C);

        \pic [draw, pic text={\small $25^\circ$}, angle radius=0.5cm, angle eccentricity=1.8] {angle = D--A--C};

        \pic [draw, pic text={\small $35^\circ$}, angle radius=0.5cm, angle eccentricity=1.6] {angle = A--C--D};
        \pic [draw, angle radius=0.3cm] {angle = A--C--D};

        \node[below left] at (A) {$A$};
        \node[above left] at (B) {$B$};
        \node[above right] at (C) {$C$};
        \node[below right] at (D) {$D$};
    \end{tikzpicture}
    \end{flushright}
\end{minipage}

\vspace{0.2cm}
\answerTableTall{$60^\circ$}{$55^\circ$}{$70^\circ$}{$65^\circ$}{$120^\circ$}

\vspace{0.7cm}

% === ЗАВДАННЯ 20 ===
\noindent\textbf{20.} \begin{minipage}[t]{0.55\textwidth}
У паралелограмі $ABCD$ діагоналі перетинаються в точці $O$. Знайдіть довжину сторони $AD$ паралелограма, якщо $AC = 10$ \textit{см}, $BD = 8$ \textit{см}, $\cos \angle AOD = -\frac{1}{5}$. \nmtyear{2025}
\end{minipage}
\hfill
\begin{minipage}[t]{0.4\textwidth}
    \vspace{-0.5cm}
    \begin{flushright}
    \begin{tikzpicture}[scale=0.6]
        \coordinate (A) at (0,0);
        \coordinate (D) at (5,0);
        \coordinate (O) at (2.5, 1.2);

        \coordinate (C) at ($(O)!-1!(A)$);
        \coordinate (B) at ($(O)!-1!(D)$);

        \draw[thick] (A) -- (B) -- (C) -- (D) -- cycle;
        \draw[thick] (A) -- (C);
        \draw[thick] (B) -- (D);

        \node[below left] at (A) {$A$};
        \node[above left] at (B) {$B$};
        \node[above right] at (C) {$C$};
        \node[below right] at (D) {$D$};
        \node[above] at (O) {$O$};
    \end{tikzpicture}
    \end{flushright}
\end{minipage}

\vspace{0.2cm}
\answerTableSmall{$2\sqrt{10}$ \textit{см}}{$6$ \textit{см}}{$\sqrt{37}$ \textit{см}}{$7$ \textit{см}}{$\sqrt{29}$ \textit{см}}

\vspace{0.7cm}

% === ЗАВДАННЯ 21 ===
\noindent\makebox[1.5em][l]{\textbf{21.}}\parbox[t]{\dimexpr\textwidth-1.5em}{Обчисліть \textit{більшу} сторону паралелограма, якщо його периметр дорівнює $24$ \textit{дм}, а сума трьох сторін паралелограма дорівнює $19$ \textit{дм}. \nmtyear{2025}}

\vspace{0.3cm}
\answerTable{$6$ \textit{дм}}{$8$ \textit{дм}}{$5$ \textit{дм}}{$9$ \textit{дм}}{$7$ \textit{дм}}

\vspace{0.7cm}

% === ЗАВДАННЯ 22 ===
\noindent\makebox[1.5em][l]{\textbf{22.}}\parbox[t]{\dimexpr\textwidth-1.5em}{Які з наведених тверджень є правильними? \nmtyear{2025}}

\vspace{0.2cm}
\begin{tabular}{r@{\hspace{0.5em}}p{14cm}}
I. & Існує паралелограм, у якого всі кути прямі. \\
II. & Існує паралелограм, сума двох протилежних кутів якого дорівнює $240^\circ$. \\
III. & Існує паралелограм, у якого сума кутів, прилеглих до однієї сторони, дорівнює $200^\circ$. \\
\end{tabular}

\vspace{0.3cm}
\answerTable{лише I та II}{лише I та III}{лише III}{лише II}{лише I}

\vspace{0.7cm}

% === ЗАВДАННЯ 23 ===
\noindent\textbf{23.} \begin{minipage}[t]{0.55\textwidth}
У паралелограмі $ABCD$ з точки $B$ на сторону $AD$ опущено висоту $BK = 8$ \textit{см}, $AK = 6$ \textit{см}, $KD = 9$ \textit{см}. До кожного відрізка (1--3) доберіть його довжину (А--Д). \nmtyear{2024}
\end{minipage}
\hfill
\begin{minipage}[t]{0.4\textwidth}
    \vspace{-0.5cm}
    \begin{flushright}
    \begin{tikzpicture}[scale=0.35]
        \coordinate (A) at (0,0);
        \coordinate (K) at (6,0);
        \coordinate (D) at (15,0);
        \coordinate (B) at (6,8);
        \coordinate (C) at (21,8);

        \draw[thick] (A) -- (B) -- (C) -- (D) -- cycle;
        \draw[thick] (B) -- (K);

        \draw (K) ++(-0.6,0) -- ++(0,0.6) -- ++(0.6,0);

        \node[below left] at (A) {$A$};
        \node[above left] at (B) {$B$};
        \node[above right] at (C) {$C$};
        \node[below right] at (D) {$D$};
        \node[below] at (K) {$K$};
    \end{tikzpicture}
    \end{flushright}
\end{minipage}

\vspace{0.3cm}

\matchingLayout{
    \textbf{1} \quad Середня лінія трапеції $KBCD$ \\
    \textbf{2} \quad $AB$ \\
    \textbf{3} \quad Відстань від точки $B$ до сторони $CD$
}{
    \begin{tabular}{ll}
    \textbf{А} & $\dfrac{120}{17}$ \textit{см} \\
    \textbf{Б} & 10 \textit{см} \\
    \textbf{В} & 12 \textit{см} \\
    \textbf{Г} & $\sqrt{145}$ \textit{см} \\
    \textbf{Д} & 15 \textit{см} \\
    \end{tabular}
}{
    \answerGrid
}

\vspace{0.7cm}

% === ЗАВДАННЯ 24 ===
\noindent\textbf{24.} \begin{minipage}[t]{0.55\textwidth}
На стороні $BC$ паралелограма $ABCD$ вибрано точку $M$ так, що $BM = MC$, $\angle CDM = 90^\circ$ (див. рисунок). Знайдіть площу паралелограма $ABCD$, якщо $MD = a$, $\angle A = \theta$. \nmtyear{2024}
\end{minipage}
\hfill
\begin{minipage}[t]{0.4\textwidth}
    \vspace{-0.5cm}
    \begin{flushright}
    \begin{tikzpicture}[scale=0.9]
        \coordinate (A) at (0,0);
        \coordinate (D) at (4.5,0);
        \coordinate (B) at (1.5, 2);
        \coordinate (C) at (6, 2);
        \coordinate (M) at (3.2, 2);

        \draw[thick] (A) -- (B) -- (C) -- (D) -- cycle;
        \draw[thick] (D) -- (M) node[midway, below left] {$a$};

        \draw ($(B)!0.5!(M)$) ++(90:0.1) -- ++(-90:0.2);
        \draw ($(M)!0.5!(C)$) ++(90:0.1) -- ++(-90:0.2);

        \pic [draw, pic text={\small $\theta$}, angle radius=0.4cm, angle eccentricity=1.5] {angle = D--A--B};

        \pic [draw, angle radius=0.25cm] {right angle = M--D--C};

        \node[below left] at (A) {$A$};
        \node[above left] at (B) {$B$};
        \node[above right] at (C) {$C$};
        \node[below right] at (D) {$D$};
        \node[above] at (M) {$M$};
        \fill (M) circle (1.5pt);
    \end{tikzpicture}
    \end{flushright}
\end{minipage}

\vspace{0.2cm}
\answerTableTall{$2a^2\tg\theta$}{$2a^2\sin\theta$}{$\dfrac{4a^2}{\sin 2\theta}$}{$\dfrac{2a^2}{\sin\theta}$}{$\dfrac{2a^2}{\tg\theta}$}

\vspace{0.7cm}

% === ЗАВДАННЯ 25 ===
\noindent\textbf{25.} \begin{minipage}[t]{0.95\textwidth}
На паралельних прямих $m$ та $n$ розміщено основи трапеції $ABCD$, сторони квадрата $DKLM$ та сторони паралелограма $MNPQ$ (див. рисунок). Периметр квадрата дорівнює $20$, $BC=KL$, $BC:AD = 2:3$, $AD=MQ$. Узгодьте фігуру (1--3) з її площею (А--Д). \nmtyear{2024}
\end{minipage}

\vspace{0.3cm}
\begin{center}
\begin{tikzpicture}[scale=0.5]
    \coordinate (A) at (0,0);
    \coordinate (B) at (0,2.5);
    \coordinate (C) at (2.5,2.5);
    \coordinate (D) at (3.75,0);

    \coordinate (K_sq) at (3.75,2.5);
    \coordinate (L_sq) at (6.25,2.5);
    \coordinate (M_sq) at (6.25,0);

    \coordinate (N_par) at (8,2.5);
    \coordinate (P) at (11.75,2.5);
    \coordinate (Q) at (10,0);

    \fill[cyan!20] (A) -- (B) -- (C) -- (D) -- cycle;
    \fill[violet!20] (D) -- (K_sq) -- (L_sq) -- (M_sq) -- cycle;
    \fill[yellow!20] (M_sq) -- (N_par) -- (P) -- (Q) -- cycle;

    \draw[thick] (A) -- (B) -- (C) -- (D) -- cycle;
    \draw[thick] (D) -- (K_sq) -- (L_sq) -- (M_sq) -- cycle;
    \draw[thick] (M_sq) -- (N_par) -- (P) -- (Q) -- cycle;

    \draw[thick] (-1,0) -- (13,0) node[above] {$n$};
    \draw[thick] (-1,2.5) -- (13,2.5) node[above] {$m$};

    \draw (A) rectangle ++(0.25,0.25);

    \node[below] at (A) {$A$};
    \node[above] at (B) {$B$};
    \node[above] at (C) {$C$};
    \node[below] at (D) {$D$};
    \node[above] at (K_sq) {$K$};
    \node[above] at (L_sq) {$L$};
    \node[below] at (M_sq) {$M$};
    \node[above] at (N_par) {$N$};
    \node[above] at (P) {$P$};
    \node[below] at (Q) {$Q$};

\end{tikzpicture}
\end{center}

\matchingLayout{
    \textbf{1} \quad квадрат $DKLM$ \\
    \textbf{2} \quad паралелограм $MNPQ$ \\
    \textbf{3} \quad трапеція $ABCD$
}{
    \begin{tabular}{ll}
    \textbf{А} & 25 \\
    \textbf{Б} & 37,5 \\
    \textbf{В} & 31,25 \\
    \textbf{Г} & 50 \\
    \textbf{Д} & 62,5 \\
    \end{tabular}
}{
    \answerGrid
}

\end{document}
